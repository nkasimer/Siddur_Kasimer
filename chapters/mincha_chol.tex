\chapter[מנחה לחול]{\adforn{47} מנחה לחול \adforn{19}}
\vspace{0.25in}
\ifboolexpr{togl {minchainshacharit}}{\longenginst{
Min\d{h}a, the afternoon prayer, is contained in selections from sha\d{h}ari\thav\space, as follows:\begin{itemize}
	\item Some say a selection of korbano\thav\space on page \pageref{korbanos}.
	\item Ashrei is on page \pageref{PDZashrei}.
	\item The reader recites Half Kaddish on page \pageref{shacharitchatzikaddish}.
	\item On fast days the Torah and Haftara are chanted.  The Torah service is on page \pageref{weekday torah}. The readings are on page \pageref{torah taanis tzibbur}.
	\item The amida is recited silently, and repeated by the reader, on page \pageref{amidahshacharitchol}.
	\item On fast days, Avinu Malkeinu is read on page \pageref{avinu malkeinu}.
	\item Ta\d{h}anun is recited on applicable days on page \pageref{nefilas_apayim}.
	\item The reader recites the full Kaddish on page \pageref{end of shacharis}. This is followed by Aleinu and Mourner's Kaddish.
\end{itemize}

}}{\ashrei

\halfkaddish

\instruction{בתענית קוראים התורה עמ׳ \pageref{weekday torah}},

\section[תפילת העמידה]{\adforn{53} תפילת העמידה \adforn{25}}


\amidaopening{\ayt}{\englishinst{During the repetition of the Amidah, Kedusha is said here}}

\weekdaysakedusha \vspace{0.5\baselineskip}

\sepline

\weekdaysabinah

\weekdaysateshuva

\weekdaysaselichah

\weekdaysageulah

\weekdaysaanneinu

\weekdaysarefuah

\weekdaysaberacha

\weekdaysashofar

\weekdaysamishpat

\weekdaysaminim

\weekdaysatzadikim

\yerushwithnachem

\weekdaysamalchus

\firstword{שְׁמַע קוֹלֵֽנוּ}
יְיָ אֱלֹהֵֽינוּ חוּס וְרַחֵם עָלֵֽינוּ וְקַבֵּל בְּרַחֲמִים וּבְרָצוֹן אֶת־תְּפִלָּתֵֽנוּ כִּי אֵל שׁוֹמֵעַ תְּפִלּוֹת וְתַחֲנוּנִים אַֽתָּה וּמִלְּפָנֶֽיךָ מַלְכֵּֽנוּ רֵיקָם אַל תְּשִׁיבֵֽנוּ
\footnote{
\instruction{בתענית ציבור היחיד אומר כאן עננו, אלא מסיימים בכִּי אַתָּה שׁוֹמֵֽעַ...\\
}
עֲנֵֽנוּ יְיָ עֲנֵֽנוּ בְּיוֹם צוֹם תַּעֲנִיתֵֽנוּ כִּי בְצָרָה גְדוֹלָה אֲנָֽחְנוּ אַל תֵּֽפֶן אֶל רִשְׁעֵֽנוּ וְאַל תַּסְתֵּר פָּנֶֽיךָ מִמֶּֽנּוּ וְאַל תִּתְעַלַּם מִתְּחִנָּתֵֽנוּ׃ הֱיֵה־נָא קָרוֹב לְשַׁוְעָתֵֽנוּ יְהִי־נָא חַסְדְּךָ לְנַחֲמֵֽנוּ טֶֽרֶם נִקְרָא אֵלֶֽיךָ עֲנֵֽנוּ כַּדָּבָר שֶׁנֶּאֱמַר׃
\mdsource{ישעיה סה}%
וְהָיָ֥ה טֶֽרֶם־יִקְרָ֖אוּ וַאֲנִ֣י אֶעֱנֶ֑ה ע֛וֹד הֵ֥ם מְדַבְּרִ֖ים וַאֲנִ֥י אֶשְׁמָֽע׃ כִּי אַתָּה יְיָ הָעוֹנֶה בְּעֵת צָרָה פּוֹדֶה וּמַצִּיל בְּכׇל־עֵת צָרָה וְצוּקָה׃
}
כִּי אַתָּה שׁוֹמֵֽעַ תְּפִלַּת עַמְּךָ יִשְׂרָאֵל בְּרַחֲמִים׃ בָּרוּךְ אַתָּה יְיָ שׁוֹמֵֽעַ תְּפִלָּה׃

\retzeh

\yaalehveyavo

\zion

\modim

\alhanisim

\weekdaysahodos

\shatzbirkaskohanim{בחזרת הש״ץ בתענית ציבור:}

\rule[-0.5ex]{3in}{1pt}

\columnratio{0.73}
\begin{paracol}{2}
\instruction{בתענית ציבור:}\\
\firstword{שִׂים שָׁלוֹם}
טוֹבָה וּבְרָכָה חֵן וָחֶֽסֶד וְרַחֲמִים עָלֵֽינוּ וְעַל כׇּל־יִשְׂרָאֵל עַמֶּֽךָ׃ בָּרְכֵֽנוּ אָבִֽינוּ כֻּלָּֽנוּ כְּאֶחָד בְּאוֹר פָּנֶֽיךָ כִּי בְאוֹר פָּנֶֽיךָ נָתַֽתָּ לָֽנוּ יְיָ אֱלֹהֵֽינוּ תּוֹרַת חַיִּים וְאַהֲבַת חֶֽסֶד וּצְדָקָה וּבְרָכָה וְרַחֲמִים וְחַיִּים וְשָׁלוֹם׃
\switchcolumn
\firstword{שָׁלוֹם}
רָב עַל יִשְׂרָאֵל עַמְּךָ תָּשִׂים לְעוֹלָם כִּי אַתָּה הוּא מֶֽלֶךְ אָדוֹן לְכׇל־הַשָּׁלוֹם׃
\end{paracol}
וְטוֹב בְּעֵינֶֽיךָ לְבָרֵךְ אֶת־עַמְּךָ יִשְׂרָאֵל בְּכׇל־עֵת וּבְכׇל־שָׁעָה בִּשְׁלוֹמֶֽךָ׃


\columnratio{0.7}
\begin{paracol}{2}
\begin{small}
\instruction{בעשי״ת:}
בְּסֵֽפֶר חַיִּים בְּרָכָה וְשָׁלוֹם וּפַרְנָסָה טוֹבָה נִזָּכֵר וְנִכָּתֵב לְפָנֶֽיךָ אָֽנוּ וְכׇל־עַמְּךָ בֵּית יִשְׂרָאֵל לְחַיִּים וּלְשָׁלוֹם׃ בָּרוּךְ אַתָּה יְיָ עוֹשֵׂה הַשָּׁלוֹם׃

\end{small}
\switchcolumn
בָּרוּךְ אַתָּה יְיָ הַמְבָרֵךְ אֶת־עַמּוֹ יִשְׂרָאֵל בַּשָּׁלוֹם׃

\end{paracol}

\firstword{אֱלֹהַי}
נְצֹר לְשׁוֹנִי מֵרָע וּשְׂפָתַי מִדַּבֵּר מִרְמָה וְלִמְקַלְלַי נַפְשִׁי תִדּוֹם וְנַפְשִׁי כֶּעָפָר לַכֹּל תִּהְיֶה׃ פְּתַח לִבִּי בְּתוֹרָתֶֽךָ וּבְמִצְוֹתֶֽיךָ תִּרְדּוֹף נַפְשִׁי׃ וְכֹל הַחוֹשְׁבִים עָלַי רָעָה מְהֵרָה הָפֵר עֲצָתָם וְקַלְקֵל מַחֲשַׁבְתָם׃ עֲשֵׂה לְמַֽעַן שְׁמֶֽךָ עֲשֵׂה לְמַֽעַן יְמִינֶֽךָ עֲשֵׂה לְמַֽעַן קְדֻשָּׁתֶֽךָ עֲשֵׂה לְמַֽעַן תּוֹרָתֶֽךָ׃ לְ֭מַעַן \source{תהלים ס}יֵחָלְצ֣וּן יְדִידֶ֑יךָ
הוֹשִׁ֖יעָה יְמִינְךָ֣ וַעֲנֵֽנִי׃

\personalfast

יִֽהְי֥וּ לְרָצ֨וֹן אִמְרֵי־פִ֡י \source{תהלים יט}וְהֶגְי֣וֹן לִבִּ֣י לְפָנֶ֑יךָ יְ֜יָ֗ צוּרִ֥י וְגֹֽאֲלִֽי׃ עֹשֶׂה שָׁלוֹם בִּמְרוֹמָיו הוּא יַעֲשֶׂה שָׁלוֹם עָלֵֽינוּ וְעַל כׇּל־יִשְׂרָאֵל וְאִמְרוּ אָמֵן׃


\begin{small}

יְהִי רָצוֹן מִלְּפָנֶֽיךָ יְיָ אֱלֹהֵֽינוּ וִֵאלֹהֵי אֲבוֹתֵֽינוּ שֶׁיִבָּנֶה בֵּית הַמִּקְדָּשׁ בִּמְהֵרָה בְיָמֵֽינוּ וְתֵן חֶלְקֵֽנוּ בְּתוֹרָתֶֽךָ׃ וְשָׁם נַעֲבׇדְךָ בְּיִרְאָה כִּימֵי עוֹלָם וּכְשָׁנִים קַדְמֹנִיּוֹת׃
וְעָֽרְבָה֙ \source{מלאכי ג}לַֽיְיָ֔ מִנְחַ֥ת יְהוּדָ֖ה וִירוּשָׁלָ֑םִ כִּימֵ֣י עוֹלָ֔ם וּכְשָׁנִ֖ים קַדְמֹֽנִיּֽוֹת׃


\end{small}



\instruction{בימים שאין בהם תחנון ממשיכים עם עלינו עמ׳ \pageref{mincha aleinu}}\\
\instruction{בעשי״ת (לא בערב שבת וערב יוה״כ) ובת״צ אומרים אבינו מלכנו}

\section[אבינו מלכנו]{\adforn{53} אבינו מלכנו \adforn{25}}

\instruction{פותחים הארון}

\avinumalkeinu

\instruction{סגורים הארון}\\

\section[תחנון]{\adforn{53} תחנון \adforn{25}}

\instruction{אין אומרים תחנון בימים ובערב ימים אלו׃ שבת, יו״ט, ר״ח, פסח שני, יום העצמאות, ל״ג בעומר, יום ירושלים, ט׳ באב, חנכה, פורים, שושן פורים, פורים קטן, ושושן פורים קטן. גם א״א תחנון כשיש חתן או כלה בבהכ״נ, בבית אבל, בחודש ניסן, ר״ח סיון עד י״ב סיון, ערב יום כפור עד אחרי ר״ח מרחשון, ושאר ימי שמחה}

\nefilasapayim

\shomeryisroel

\fullkaddish

\label{mincha aleinu}

\aleinu
\mournerskaddish}