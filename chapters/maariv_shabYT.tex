\chapter[ערובין והדלקת נרות]{\adforn{47} ערובין והדלקת נרות \adforn{19}}
\vspace{0.25in}
\ifboolexpr{togl {includefestival}}{
\instruction{עושים עירוב תבשילין כשיו״ט חל ביום ששי}\\
\instruction{אוחז העירוב ומבורך:}\\
\firstword{בָּרוּךְ}
אַתָּה יְיָ אֱלֹהֵינוּ מֶלֶךְ הָעוֹלָם, אֲשֶׁר קִדְּשָׁנוּ בְּמִצְוֹתָיו, וְצִוָּנוּ עַל מִצְוַת עֵרוּב׃

\instruction{ערוב תבשילין׃ }\firstword{בְּדֵן עֵרוּבָא}
 יְהֵא שְׁרֵא לַֽנָא לְמֵיפֵא וּלְבַשּּׁלָא וּלְאַטְמָנָא וּלְאַדְלָקָא שְׁרָגָא וּלְמֶעְבַּד כׇּל־צָרְכָּנָא מִיּוֹמָא טָבָא לְשַׁבְּתָא (לָֽנוּ וּלְכׇל־הַדָּרִים בָּעִיר הַזּׂאת)׃‏}{}

\instruction{ערוב תחומין׃ }\firstword{בַּהֲדֵין עֵירוּבָא}
  יְהֵא שְׁרֵא לָֽנָא לֵילֵךְ מִמָקוֹם זֶה אַלְפַּֽיִם אַמָּה לְכׇל־רֽוּחַ׃

\instruction{ערוב חצרות׃ }\firstword{בַּהֲדֵין עֵירוּבָא}
בַּהֲדֵין עֵירוּבָא יְהֵא שְׁרֵא לָֽנָא לְאַפּוּקֵי וּלְעַיוּלֵי מִבָּתִּים לֶחָצֵר, וּמֵחָצֵר לְבָתִּים, וּמִבַּֽיִת לְבַֽיִת, לָֽנוּ וּלְכׇל־יִשְׂרָאֵל הַדָּרִים בֶּחָצֵר הַזֶּה׃

\ifboolexpr{togl {includeshabbat}}{\shabbos\firstword{בָּרוּךְ}
אַתָּה, יְיָ אֱלֹהֵֽינוּ, מֶֽלֶךְ הָעוֹלָם, אֲשֶׁר קִדְשָֽׁנוּ בְּמִצְוֹתָיו, וְצִוְּֽנוּ לְהַדְלִיק נֵר שֶׁל שַׁבָּת׃\\}{}

\ifboolexpr{togl {includefestival}}{\instruction{בערב יום טוב׃}\\
\firstword{בָּרוּךְ}
אַתָּה, יְיָ אֱלֹהֵֽינוּ, מֶֽלֶךְ הָעוֹלָם, אֲשֶׁר קִדְשָֽׁנוּ בְּמִצְוֹתָיו, וְצִוְּֽנוּ לְהַדְלִיק נֵר שֶׁל
\shabaddition{\\שַׁבָּת וְ}יוֹם טוֹב׃

\instruction{בערב יום כפור׃}\\
\firstword{בָּרוּךְ}
אַתָּה, יְיָ אֱלֹהֵֽינוּ, מֶֽלֶךְ הָעוֹלָם, אֲשֶׁר קִדְשָֽׁנוּ בְּמִצְוֹתָיו, וְצִוְּֽנוּ לְהַדְלִיק נֵר שֶׁל
\shabaddition{\\שַׁבָּת וְ}יוֹם הַכִּפּוּרִים׃

\instruction{בערב יום טוב (חוץ משביעי של פסח), ראש השנה, וים כפור׃}\\
\firstword{בָּרוּךְ}
אַתָּה יְיָ אֱלֹהֵינוּ מֶלֶךְ הָעוֹלָם, שֶׁהֶחֱיָנוּ וְקִיְּמָנוּ וְהִגִּיעָנוּ לַזְמַן הַזֶּה׃\\}{}


\chapter[קבלת שבת]{\adforn{47} קבלת שבת \adforn{19}}
\label{kabalas_shabbos}

\ifboolexpr{togl {includeshabbat}}{
	\englishinst{Some read the Song of Songs on Friday afternoon. If time does not permit that, the following verses from Song of Songs may be recited:}
	%\instruction{יש נוהגים לקרא שיר השירים בע״ש. ואם אין זמן לקרא כל הספר, יש לקרא פסוקים אלו׃}
יִשָּׁקֵ֙נִי֙ \source{שיר השירים ב}מִנְּשִׁיק֣וֹת פִּ֔יהוּ כִּֽי־טוֹבִ֥ים דֹּדֶ֖יךָ מִיָּֽיִן׃
ע֤וּרִי\source{שיר השירים ד} צָפוֹן֙ וּב֣וֹאִי תֵימָ֔ן הָפִ֥יחִי גַנִּ֖י יִזְּל֣וּ בְשָׂמָ֑יו יָבֹ֤א דוֹדִי֙ לְגַנּ֔וֹ וְיֹאכַ֖ל פְּרִ֥י מְגָדָֽיו׃ ק֣וֹל \source{שיר השירים ב}דּוֹדִ֔י הִנֵּה־זֶ֖ה בָּ֑א מְדַלֵּג֙ עַל־הֶ֣הָרִ֔ים מְקַפֵּ֖ץ עַל־הַגְּבָעֽוֹת׃ בָּ֣אתִי \source{שיר השירים ה}לְגַנִּי֮ אֲחֹתִ֣י כַלָּה֒ אָרִ֤יתִי מוֹרִי֙ עִם־בְּשָׂמִ֔י אָכַ֤לְתִּי יַעְרִי֙ עִם־דִּבְשִׁ֔י שָׁתִ֥יתִי יֵינִ֖י עִם־חֲלָבִ֑י אִכְל֣וּ רֵעִ֔ים שְׁת֥וּ וְשִׁכְר֖וּ דּוֹדִֽים׃

\englishinst{Some sing the following before Kabbalat Shabbat:}
\firstword{יְדִיד נֶפֶשׁ}
אָב הָרַחְמָן מְשֹׁךְ עַבְדָּךְ אֶל רְצוֹנָךְ.\\
יָרוּץ עַבְדָּךְ כְּמוֹ אַיָּל יִשְׁתַּחֲוֶה מוּל הֲדָרָךְ.\\
כִּי יֶעְרַב לוֹ יְדִידוּתָךְ מִנֹּפֶת צוּף וְכׇל־טָעַם׃\\

הָדוּר נָאֶה זִיו הָעוֹלָם נַפְשִׁי חוֹלַת אַהֲבָתָךְ.\\
אָנָּא אֵל נָא, רְפָא־נָא לָהּ בְּהַרְאוֹת לָהּ נֹעַם זִיוָךְ.\\
אָז תִּתְחַזֵּק וְתִתְרַפֵּא וְהָיְתָה לָּךְ שִׁפְחַת עוֹלָם׃\\

וָתִיק יֶהְמוּ־נָא רַחְמֶיךָ וְחוּס־נָא עַל בֵּן אוֹהֲבָךְ.\\
כִּי זֶה כַּמָּה נִכְסֹף נִכְסַף לִרְאוֹת בְּתִפְאֶרֶת עֻזָּךְ.\\
אָנָּא אֵלִי מַחְמַד לִבִּי חוּשָׁה נָּא וְאַל תִּתְעַלָּם׃\\

הִגָּלֵה־נָא וּפְרֹס חָבִיב עָלַי אֶת־סֻכַּת שְׁלוֹמָךְ.\\
תָּאִיר אֶרֶץ מִכְּבוֹדָךְ נָגִילָה וְנִשְׂמְחָה בָךְ.\\
מַהֵר אָהוּב כִּי בָא מוֹעֵד וְחׇנֵּנִי כִּימֵי עוֹלָם.\\

\firstword{לְ֭כוּ נְרַנְּנָ֣ה}\source{תהלים צה}
לַייָ֑ נָ֝רִ֗יעָה לְצ֣וּר יִשְׁעֵֽנוּ׃
נְקַדְּמָ֣ה פָנָ֣יו בְּתוֹדָ֑ה בִּ֝זְמִר֗וֹת נָרִ֥יעַֽ לֽוֹ׃
כִּ֤י אֵ֣ל גָּד֣וֹל יְיָ֑ וּמֶ֥לֶךְ גָּ֝ד֗וֹל עַל־כׇּל־אֱלֹהִֽים׃
אֲשֶׁ֣ר בְּ֭יָדוֹ מֶחְקְרֵי־אָ֑רֶץ וְתוֹעֲפֹ֖ת הָרִ֣ים לֽוֹ׃
אֲשֶׁר־ל֣וֹ הַ֭יָּם וְה֣וּא עָשָׂ֑הוּ וְ֝יַבֶּ֗שֶׁת יָדָ֥יו יָצָֽרוּ׃
בֹּ֭אוּ נִשְׁתַּחֲוֶ֣ה וְנִכְרָ֑עָה נִ֝בְרְכָ֗ה לִֽפְנֵי־יְיָ֥ עֹשֵֽׂנוּ׃
כִּ֘י ה֤וּא אֱלֹהֵ֗ינוּ וַאֲנַ֤חְנוּ עַ֣ם מַ֭רְעִיתוֹ וְצֹ֣אן יָד֑וֹ הַ֝יּ֗וֹם אִֽם־בְּקֹל֥וֹ תִשְׁמָֽעוּ׃
אַל־תַּקְשׁ֣וּ לְ֭בַבְכֶם כִּמְרִיבָ֑ה כְּי֥וֹם מַ֝סָּ֗ה בַּמִּדְבָּֽר׃
אֲשֶׁ֣ר נִ֭סּוּנִי אֲבֽוֹתֵיכֶ֑ם בְּ֝חָנ֗וּנִי גַּם־רָא֥וּ פׇעֳלִֽי׃
אַרְבָּ֘עִ֤ים שָׁנָ֨ה ׀ אָ֘ק֤וּט בְּד֗וֹר וָאֹמַ֗ר עַ֤ם תֹּעֵ֣י לֵבָ֣ב הֵ֑ם וְ֝הֵ֗ם לֹא־יָדְע֥וּ דְרָכָֽי׃
אֲשֶׁר־נִשְׁבַּ֥עְתִּי בְאַפִּ֑י אִם־יְ֝בֹא֗וּן אֶל־מְנוּחָתִֽי׃


\firstword{שִׁ֣ירוּ לַ֭יְיָ}\source{תהלים צו}
שִׁ֣יר חָדָ֑שׁ שִׁ֥ירוּ לַ֝ייָ֗ כׇּל־הָאָֽרֶץ׃
שִׁ֣ירוּ לַ֭ייָ בָּרְכ֣וּ שְׁמ֑וֹ בַּשְּׂר֥וּ מִיּֽוֹם־לְ֝י֗וֹם יְשׁוּעָתֽוֹ׃
סַפְּר֣וּ בַגּוֹיִ֣ם כְּבוֹד֑וֹ בְּכׇל־הָ֝עַמִּ֗ים נִפְלְאוֹתָֽיו׃
כִּ֥י גָ֘ד֤וֹל יְיָ֣ וּמְהֻלָּ֣ל מְאֹ֑ד נוֹרָ֥א ה֗֝וּא עַל־כׇּל־אֱלֹהִֽים׃
כִּ֤י ׀ כׇּל־אֱלֹהֵ֣י הָעַמִּ֣ים אֱלִילִ֑ים וַ֝ייָ֗ שָׁמַ֥יִם עָשָֽׂה׃
הוֹד־וְהָדָ֥ר לְפָנָ֑יו עֹ֥ז וְ֝תִפְאֶ֗רֶת בְּמִקְדָּשֽׁוֹ׃
הָב֣וּ לַ֭ייָ מִשְׁפְּח֣וֹת עַמִּ֑ים הָב֥וּ לַ֝ייָ֗ כָּב֥וֹד וָעֹֽז׃
הָב֣וּ לַ֭ייָ כְּב֣וֹד שְׁמ֑וֹ שְׂאֽוּ־מִ֝נְחָ֗ה וּבֹ֥אוּ לְחַצְרוֹתָֽיו׃
הִשְׁתַּחֲו֣וּ לַ֭ייָ בְּהַדְרַת־קֹ֑דֶשׁ חִ֥ילוּ מִ֝פָּנָ֗יו כׇּל־הָאָֽרֶץ׃
אִמְר֤וּ בַגּוֹיִ֨ם ׀ יְ֘יָ֤ מָלָ֗ךְ אַף־תִּכּ֣וֹן תֵּ֭בֵל בַּל־תִּמּ֑וֹט יָדִ֥ין עַ֝מִּ֗ים בְּמֵישָׁרִֽים׃
יִשְׂמְח֣וּ הַ֭שָּׁמַיִם וְתָגֵ֣ל הָאָ֑רֶץ יִֽרְעַ֥ם הַ֝יָּ֗ם וּמְלֹאֽוֹ׃
יַעֲלֹ֣ז שָׂ֭דַי וְכׇל־אֲשֶׁר־בּ֑וֹ אָ֥ז יְ֝רַנְּנ֗וּ כׇּל־עֲצֵי־יָֽעַר׃
לִפְנֵ֤י יְיָ֨ ׀ כִּ֬י בָ֗א כִּ֥י בָא֮ לִשְׁפֹּ֢ט הָ֫אָ֥רֶץ יִשְׁפֹּֽט־תֵּבֵ֥ל בְּצֶ֑דֶק וְ֝עַמִּ֗ים בֶּאֱמוּנָתֽוֹ׃

\firstword{יְיָ֣ מָלָךְ}\source{תהלים צז}
תָּגֵ֣ל הָאָ֑רֶץ יִ֝שְׂמְח֗וּ אִיִּ֥ים רַבִּֽים׃
עָנָ֣ן וַעֲרָפֶ֣ל סְבִיבָ֑יו צֶ֥דֶק וּ֝מִשְׁפָּ֗ט מְכ֣וֹן כִּסְאֽוֹ׃
אֵ֭שׁ לְפָנָ֣יו תֵּלֵ֑ךְ וּתְלַהֵ֖ט סָבִ֣יב צָרָֽיו׃
הֵאִ֣ירוּ בְרָקָ֣יו תֵּבֵ֑ל רָאֲתָ֖ה וַתָּחֵ֣ל הָאָֽרֶץ׃
הָרִ֗ים כַּדּוֹנַ֗ג נָ֭מַסּוּ מִלִּפְנֵ֣י יְיָ֑ מִ֝לִּפְנֵ֗י אֲד֣וֹן כׇּל־הָאָֽרֶץ׃
הִגִּ֣ידוּ הַשָּׁמַ֣יִם צִדְק֑וֹ וְרָא֖וּ כׇל־הָעַמִּ֣ים כְּבוֹדֽוֹ׃
יֵבֹ֤שׁוּ ׀ כׇּל־עֹ֬בְדֵי פֶ֗סֶל הַמִּֽתְהַלְלִ֥ים בָּאֱלִילִ֑ים הִשְׁתַּחֲווּ־ל֗֝וֹ כׇּל־אֱלֹהִֽים׃
שָׁמְעָ֬ה וַתִּשְׂמַ֨ח ׀ צִיּ֗וֹן וַ֭תָּגֵלְנָה בְּנ֣וֹת יְהוּדָ֑ה לְמַ֖עַן מִשְׁפָּטֶ֣יךָ יְיָ׃
כִּֽי־אַתָּ֤ה יְיָ֗ עֶלְי֥וֹן עַל־כׇּל־הָאָ֑רֶץ מְאֹ֥ד נַ֝עֲלֵ֗יתָ עַל־כׇּל־אֱלֹהִֽים׃
אֹֽהֲבֵ֥י יְיָ֗ שִׂנְא֫וּ רָ֥ע שֹׁ֭מֵר נַפְשׁ֣וֹת חֲסִידָ֑יו מִיַּ֥ד רְ֝שָׁעִ֗ים יַצִּילֵֽם׃
א֭וֹר זָרֻ֣עַ לַצַּדִּ֑יק וּֽלְיִשְׁרֵי־לֵ֥ב שִׂמְחָֽה׃
שִׂמְח֣וּ צַ֭דִּיקִים בַּייָ֑ וְ֝הוֹד֗וּ לְזֵ֣כֶר קׇדְשֽׁוֹ׃

\firstword{מִזְמ֡וֹר שִׁ֤ירוּ}\source{תהלים צח}
לַֽיְיָ֙ שִׁ֥ייָ֨ ׀ שִׁ֣יר חָ֭דָשׁ כִּֽי־נִפְלָא֣וֹת עָשָׂ֑ה הוֹשִׁיעָה־לּ֥וֹ יְ֝מִינ֗וֹ וּזְר֥וֹעַ קׇדְשֽׁוֹ׃
הוֹדִ֣יעַ יְיָ֭ יְשׁוּעָת֑וֹ לְעֵינֵ֥י הַ֝גּוֹיִ֗ם גִּלָּ֥ה צִדְקָתֽוֹ׃
זָ֘כַ֤ר חַסְדּ֨וֹ ׀ וֶ֥אֱֽמוּנָתוֹ֮ לְבֵ֢ית יִשְׂרָ֫אֵ֥ל רָא֥וּ כׇל־אַפְסֵי־אָ֑רֶץ אֵ֗֝ת יְשׁוּעַ֥ת אֱלֹהֵֽינוּ׃
הָרִ֣יעוּ לַ֭ייָ כׇּל־הָאָ֑רֶץ פִּצְח֖וּ וְרַנְּנ֣וּ וְזַמֵּֽרוּ׃
זַמְּר֣וּ לַייָ֣ בְּכִנּ֑וֹר בְּ֝כִנּ֗וֹר וְק֣וֹל זִמְרָֽה׃
בַּ֭חֲצֹ֣צְרוֹת וְק֣וֹל שׁוֹפָ֑ר הָ֝רִ֗יעוּ לִפְנֵ֤י ׀ הַמֶּ֬לֶךְ יְיָ׃
יִרְעַ֣ם הַ֭יָּם וּמְלֹא֑וֹ תֵּ֝בֵ֗ל וְיֹ֣שְׁבֵי בָֽהּ׃
נְהָר֥וֹת יִמְחֲאוּ־כָ֑ף יַ֗֝חַד הָרִ֥ים יְרַנֵּֽנוּ׃
לִ֥פְֽנֵי יְיָ֗ כִּ֥י בָא֮ לִשְׁפֹּ֢ט הָ֫אָ֥רֶץ יִשְׁפֹּֽט־תֵּבֵ֥ל בְּצֶ֑דֶק וְ֝עַמִּ֗ים בְּמֵישָׁרִֽים׃

\firstword{יְיָ֣ מָ֭לָךְ}\source{תהלים צט}
יִרְגְּז֣וּ עַמִּ֑ים יֹשֵׁ֥ב כְּ֝רוּבִ֗ים תָּנ֥וּט הָאָֽרֶץ׃
יְיָ֭ בְּצִיּ֣וֹן גָּד֑וֹל וְרָ֥ם ה֗֝וּא עַל־כׇּל־הָעַמִּֽים׃
יוֹד֣וּ שִׁ֭מְךָ גָּד֥וֹל וְנוֹרָ֗א קָד֥וֹשׁ הֽוּא׃
וְעֹ֥ז מֶלֶךְ֮ מִשְׁפָּ֢ט אָ֫הֵ֥ב אַ֭תָּה כּוֹנַ֣נְתָּ מֵישָׁרִ֑ים מִשְׁפָּ֥ט וּ֝צְדָקָ֗ה בְּיַעֲקֹ֤ב ׀ אַתָּ֬ה עָשִֽׂיתָ׃
רוֹמְמ֡וּ יְ֘יָ֤ אֱלֹהֵ֗ינוּ וְֽ֭הִשְׁתַּחֲווּ לַהֲדֹ֥ם רַגְלָ֗יו קָד֥וֹשׁ הֽוּא׃
מֹ֘שֶׁ֤ה וְאַֽהֲרֹ֨ן ׀ בְּֽכֹהֲנָ֗יו וּ֭שְׁמוּאֵל בְּקֹרְאֵ֣י שְׁמ֑וֹ קֹרִ֥אים אֶל־יְ֝יָ֗ וְה֣וּא יַעֲנֵֽם׃
בְּעַמּ֣וּד עָ֭נָן יְדַבֵּ֣ר אֲלֵיהֶ֑ם שָׁמְר֥וּ עֵ֝דֹתָ֗יו וְחֹ֣ק נָֽתַן־לָֽמוֹ׃
יְיָ֣ אֱלֹהֵינוּ֮ אַתָּ֢ה עֲנִ֫יתָ֥ם אֵ֣ל נֹ֭שֵׂא הָיִ֣יתָ לָהֶ֑ם וְ֝נֹקֵ֗ם עַל־עֲלִילוֹתָֽם׃
רוֹמְמ֡וּ יְ֘יָ֤ אֱלֹהֵ֗ינוּ וְֽ֭הִשְׁתַּחֲווּ לְהַ֣ר קׇדְשׁ֑וֹ כִּי־קָ֝ד֗וֹשׁ יְיָ֥ אֱלֹהֵֽינוּ׃

\englishinst{The following Psalm is said standing.}
\firstword{מִזְמ֗וֹר לְדָ֫וִ֥ד}\source{תהלים כט}
הָב֣וּ לַ֭ייָ בְּנֵ֣י אֵלִ֑ים הָב֥וּ לַ֝ייָ֗ כָּב֥וֹד וָעֹֽז׃
הָב֣וּ לַ֭ייָ כְּב֣וֹד שְׁמ֑וֹ הִשְׁתַּחֲו֥וּ לַ֝ייָ֗ בְּהַדְרַת־קֹֽדֶשׁ׃
ק֥וֹל יְיָ֗ עַל־הַ֫מָּ֥יִם אֵֽל־הַכָּב֥וֹד הִרְעִ֑ים יְ֝יָ֗ עַל־מַ֥יִם רַבִּֽים׃
קוֹל־יְיָ֥ בַּכֹּ֑חַ ק֥וֹל יְ֝יָ֗ בֶּהָדָֽר׃
ק֣וֹל יְיָ֭ שֹׁבֵ֣ר אֲרָזִ֑ים וַיְשַׁבֵּ֥ר יְ֝יָ֗ אֶת־אַרְזֵ֥י הַלְּבָנֽוֹן׃
וַיַּרְקִידֵ֥ם כְּמוֹ־עֵ֑גֶל לְבָנ֥וֹן וְ֝שִׂרְיֹ֗ן כְּמ֣וֹ בֶן־רְאֵמִֽים׃
קוֹל־יְיָ֥ חֹצֵ֗ב לַהֲב֥וֹת אֵֽשׁ׃
ק֣וֹל יְיָ֭ יָחִ֣יל מִדְבָּ֑ר יָחִ֥יל יְ֝יָ֗ מִדְבַּ֥ר קָדֵֽשׁ׃
ק֤וֹל יְיָ֨ ׀ יְחוֹלֵ֣ל אַיָּלוֹת֮ וַֽיֶּחֱשֹׂ֢ף יְעָ֫ר֥וֹת וּבְהֵיכָל֑וֹ כֻּ֝לּ֗וֹ אֹמֵ֥ר כָּבֽוֹד׃
יְיָ֭ לַמַּבּ֣וּל יָשָׁ֑ב וַיֵּ֥שֶׁב יְ֝יָ֗ מֶ֣לֶךְ לְעוֹלָֽם׃
יְיָ֗ עֹ֭ז לְעַמּ֣וֹ יִתֵּ֑ן יְיָ֓ ׀ יְבָרֵ֖ךְ אֶת־עַמּ֣וֹ בַשָּׁלֽוֹם׃


\section[לכה דודי]{\adforn{53} לכה דודי \adforn{25}}

\newcommand{\lechadodi}{\textbf{לְכָה דוֹדִי לִקְרַאת כַּלָּה פְּנֵי שַׁבָּת נְקַבְּלָה׃}\\}

%\leftskip=0pt plus-.2fil
%\rightskip=0pt plus.2fil
%\parfillskip=0pt plus1fil
%\vspace{-0.5\baselineskip}
\newcommand{\lechadodiverse}[4]{
#1 \hfill #2\\
#3 \hfill #4\\
}

\lechadodi
\lechadodiverse{שָׁ֗מוֹר וְזָכוֹר בְּדִבּוּר אֶחָד}{הִשְׁמִיעָֽנוּ אֵל הַמְיֻחָד}{יְיָ אֶחָד וּשְׁמוֹ אֶחָד}{לְשֵׁם וּלְתִפְאֶֽרֶת וְלִתְהִלָּה׃}
\lechadodi
\lechadodiverse{ שָׁ֗מוֹר וְזָכוֹר בְּדִבּוּר אֶחָד}{הִשְׁמִיעָֽנוּ אֵל הַמְיֻחָד}{יְיָ אֶחָד וּשְׁמוֹ אֶחָד}{לְשֵׁם וּלְתִפְאֶֽרֶת וְלִתְהִלָּה׃}
\lechadodi
\lechadodiverse{לִ֗קְרַאת שַׁבָּת לְכוּ וְנֵלְכָה}{כִּי הִיא מְקוֹר הַבְּרָכָה}{מֵרֹאשׁ מִקֶּֽדֶם נְסוּכָה}{סוֹף מַעֲשֶׂה בְּמַחֲשָׁבָה תְּחִלָּה׃}
\lechadodi
\lechadodiverse{מִ֗קְדַּשׁ מֶֽלֶךְ עִיר מְלוּכָה}{קֽוּמִי צְאִי מִתּוֹךְ הַהֲפֵכָה}{רַב לָךְ שֶֽׁבֶת בְּעֵֽמֶק הַבָּכָא}{וְהוּא יַחֲמוֹל עָלַֽיִךְ חֶמְלָה׃}
\lechadodi
\lechadodiverse{הִ֗תְנַעֲרִי מֵעָפָר קֽוּמִי}{לִבְשִׁי בִּגְדֵי תִפְאַרְתֵּךְ עַמִּי}{עַל יַד בֶּן יִשַׁי בֵּית הַלַּחְמִי}{קׇרְבָה אֶל נַפְשִׁי גְאָלָהּ׃}
\lechadodi
\lechadodiverse{הִ֗תְעוֹרְרִי הִתְעוֹרְרִי}{כִּי בָא אוֹרֵךְ קֽוּמִי אֽוֹרִי}{עֽוּרִי עֽוּרִי שִׁיר דַבֵּֽרִי}{כְּבוֹד יְיָ עָלַֽיִךְ נִגְלָה׃}
\lechadodi
\lechadodiverse{לֹ֗א תֵבֽוֹשִׁי וְלֹא תִכָּלְמִי}{מַה תִּשְׁתּוֹחֲחִי וּמַה תֶּהֱמִי}{בָּךְ יֶחֱסוּ עֲנִיֵּי עַמִּי}{וְנִבְנְתָה עִיר עַל תִּלָּהּ׃}
\lechadodi
\lechadodiverse{וְ֗הָיוּ לִמְשִׁסָּה שֹׁאסָֽיִךְ}{וְרָחֲקוּ כׇּל־מְבַלְּעָֽיִךְ}{יָשִׂישׂ עָלַֽיִךְ אֱלֹהָֽיִךְ}{כִּמְשׂוֹשׂ חָתָן עַל כַּלָּה׃}
\lechadodi
\lechadodiverse{יָ֗מִין וּשְׂמֹאל תִּפְרֽוֹצִי}{וְאֶת יְיָ תַּעֲרִֽיצִי}{עַל יַד אִישׁ בֶּן פַּרְצִי}{וְנִשְׂמְחָה וְנָגִֽילָה׃}
\lechadodi
\englishinst{Stand, and face the synagogue entrance for this verse:}
\lechadodiverse{בּֽוֹאִי בְשָׁלוֹם עֲטֶרֶת בַּעְלָהּ}{גַּם בְּשִׂמְחָה וּבְצׇהֳלָה}{תּוֹךְ אֱמוּנֵי עַם סְגֻּלָּה}{בּֽוֹאִי כַלָּה בּֽוֹאִי כַלָּה׃}
\lechadodi \vspace{-0.5\baselineskip}

\begin{sometimes}

\englishinst{Mourners do not attend Kabbalat Shabbat until this point.  As they enter, the congregation says to the mourners:}
%\instruction{הציבור לאבלים:}\\
הַמָּקוֹם יְנַחֵם אֶתְכֶם בְּתוֹךְ שְׁאָר אֲבֵלֵי צִיּוֹן וִירוּשָׁלָֽיִם׃

\end{sometimes}

\ifboolexpr{togl {includefestival}}{\englishinst{On Festivals, Kabbalat Shabbat begins here.}}{}
}{\englishinst{When a Festival falls on Friday evening, the following Psalms are said:}}
\ifboolexpr{togl {includefestival}}{\englishinst{}}{}

\mizmorshabbat

\firstword{יְיָ֣ מָלָךְ֘ גֵּא֢וּת לָ֫בֵ֥שׁ}\source{תהלים צג}
לָבֵ֣שׁ יְיָ֭ עֹ֣ז הִתְאַזָּ֑ר אַף־תִּכּ֥וֹן תֵּ֝בֵ֗ל בַּל־תִּמּֽוֹט׃
נָכ֣וֹן כִּסְאֲךָ֣ מֵאָ֑ז מֵעוֹלָ֣ם אָֽתָּה׃
נָשְׂא֤וּ נְהָר֨וֹת ׀ יְיָ֗ נָשְׂא֣וּ נְהָר֣וֹת קוֹלָ֑ם יִשְׂא֖וּ נְהָר֣וֹת דׇּכְיָֽם׃
מִקֹּל֨וֹת ׀ מַ֤יִם רַבִּ֗ים אַדִּירִ֣ים מִשְׁבְּרֵי־יָ֑ם אַדִּ֖יר בַּמָּר֣וֹם יְיָ׃
עֵֽדֹתֶ֨יךָ ׀ נֶאֶמְנ֬וּ מְאֹ֗ד לְבֵיתְךָ֥ נַאֲוָה־קֹ֑דֶשׁ יְ֝יָ֗ לְאֹ֣רֶךְ יָמִֽים׃

\mournerskaddish

\ifboolexpr{togl {includeshabbat}}{
\ssubsection{\adforn{48} במה מדליקין \adforn{22}}

\englishinst{The following chapter of Mishna is not recited on Festivals, including Shabbat \d{H}ol HaMo'ed.}
%\instruction{אין אומרים במה מדליקין בשבת חול המועד:}\\
\firstword{(א) בַּמֶּה מַדְלִיקִין}\source{שבת פרק ב}
וּבַמָּה אֵין מַדְלִיקִין׃ אֵין מַדְלִיקִין לֹא בְלֶֽכֶשׁ וְלֹא בְחֹֽסֶן וְלֹא בְכַלָּךְ וְלֹא בִּפְתִילַת הָאִידָן וְלֹא בִּפְתִילַת הַמִּדְבָּר וְלֹא בִּירוֹקָה שֶׁעַל פְּנֵי הַמָּֽיִם׃ לֹא בְזֶֽפֶת וְלֹא בְשַׁעֲוָה וְלֹא בְּשֶֽׁמֶן קִיק וְלֹא בְּשֶֽׁמֶן שְׂרֵפָה וְלֹא בְאַלְיָה וְלֹא בְחֵֽלֶב׃ נַחוּם הַמָּדִי אוֹמֵר׃ מַדְלִיקִין בְּחֵֽלֶב מְבֻשָּׁל׃ וַחֲכָמִים אוֹמְרִים׃ אֶחָד מְבֻשָּׁל וְאֶחָד שֶׁאֵינוֹ מְבֻשָּׁל אֵין מַדְלִיקִין בּוֹ׃

(ב) אֵין מַדְלִיקִין בְּשֶֽׁמֶן שְׂרֵפָה בְּיוֹם טוֹב׃ רַבִּי יִשְׁמָעֵאל אוֹמֵר׃ אֵין מַדְלִיקִין בְּעִטְרָן מִפְּנֵי כְּבוֹד הַשַּׁבָּת׃ וַחֲכָמִים מַתִּירִין בְּכׇל־הַשְּׁמָנִים׃ בְּשֶֽׁמֶן שֻׁמְשְׁמִין בְּשֶֽׁמֶן אֱגוֹזִים בְּשֶֽׁמֶן צְנוֹנוֹת בְּשֶֽׁמֶן דָּגִים בְּשֶֽׁמֶן פַּקֻּעוֹת בְּעִטְרָן וּבְנֵפְטְ׃ רַבִּי טַרְפוֹן אוֹמֵר׃ אֵין מַדְלִיקִין אֶלָּא בְּשֶֽׁמֶן זַֽיִת בִּלְבָד׃

(ג) כׇּל־הַיּוֹצֵא מִן הָעֵץ אֵין מַדְלִיקִין בּוֹ אֶלָּא פִשְׁתָּן׃ וְכׇל־הַיּוֹצֵא מִן הָעֵץ אֵינוֹ מִטַּמֵּא טֻמְאַת אֹהָלִים אֶלָּא פִשְׁתָּן׃ פְּתִילַת הַבֶּֽגֶד שֶׁקִּפְּלָהּ וְלֹא הִבְהֲבָהּ - רַבִּי אֱלִיעֶֽזֶר אוֹמֵר׃ טְמֵאָה הִיא וְאֵין מַדְלִיקִין בָּהּ׃ רַבִּי עֲקִיבָא אוֹמֵר׃ טְהוֹרָה הִיא וּמַדְלִיקִין בָּהּ׃

(ד) לֹא יִקּוֹב אָדָם שְׁפוֹפֶֽרֶת שֶׁל בֵּיצָה וִימַלְּאֶֽנָּה שֶֽׁמֶן וְיִתְּנֶֽנָּה עַל פִּי הַנֵּר בִּשְׁבִיל שֶׁתְּהֵא מְנַטֶּֽפֶת וַאֲפִילוּ הִיא שֶׁל חֶֽרֶס׃ וּרְבִי יְהוּדָה מַתִּיר אֲבָל אִם חִבְּרָהּ הַיּוֹצֵר מִתְּחִלָּה - מֻתָּר מִפְּנֵי שֶׁהוּא כְּלִי אֶחָד׃ לֹא יְמַלֵּא אָדָם קְעָרָה שֶֽׁמֶן וְיִתְּנֶֽנָּה בְּצַד הַנֵּר וְיִתֵּן רֹאשׁ הַפְּתִילָה בְּתוֹכָהּ בִּשְׁבִיל שֶׁתְּהֵא שׁוֹאָֽבֶת׃ וּרְבִי יְהוּדָה מַתִּיר׃

(ה) הַמְכַבֶּה אֶת־הַנֵּר מִפְּנֵי שֶׁהוּא מִתְיָרֵא מִפְּנֵי גוֹיִם מִפְּנֵי לִסְטִים מִפְּנֵי רֽוּחַ רָעָה אוֹ בִּשְׁבִיל הַחוֹלֶה שֶׁיִּישָׁן פָּטוּר׃ כְּחָס עַל הַנֵּר כְּחָס עַל הַשֶּֽׁמֶן כְּחָס עַל הַפְּתִילָה חַיָּב׃ רַבִּי יוֹסֵי פּוֹטֵר בְּכֻלָּן חוּץ מִן הַפְּתִילָה מִפְּנֵי שֶׁהוּא עוֹשָׂהּ פֶּחָם׃

(ו) עַל שָׁלֹשׁ עֲבֵרוֹת נָשִׁים מֵתוֹת בִּשְׁעַת לֵדָתָן׃ עַל שֶׁאֵינָן זְהִירוֹת בְּנִדָּה בְּחַלָּה וּבְהַדְלָקַת הַנֵּר׃

(ז) שְׁלֹשָׁה דְבָרִים צָרִיךְ אָדָם לוֹמַר בְּתוֹךְ בֵּיתוֹ עֶֽרֶב שַׁבָּת עִם חֲשֵׁכָה׃ עִשַׂרְתֶּם עֵרַבְתֶּם הַדְלִֽיקוּ אֶת־הַנֵּר׃ סָפֵק חֲשֵׁכָה סָפֵק אֵינָהּ חֲשֵׁכָה - אֵין מְעַשְּׂרִין אֶת־הַוַּדָּי וְאֵין מַטְבִּילִין אֶת־הַכֵּלִים וְאֵין מַדְלִיקִין אֶת־הַנֵּרוֹת׃ אֲבָל מְעַשְּׂרִין אֶת־הַדְּמָי וּמְעָרְבִין וְטוֹמְנִין אֶת־הַחַמִּין׃


\sofberakhot


\mournerskaddish}{}

\vspace{\baselineskip}

\ifboolexpr{togl {includeshabbat} and togl {includefestival}}{\chapter[ערבית לשבת ויו״ט]{\adforn{47} ערבית לשבת ויו״ט \adforn{19}}}{
	\ifboolexpr{togl {includeshabbat}}{\chapter[ערבית לשבת]{\adforn{47} ערבית לשבת \adforn{19}}}{}
	\ifboolexpr{togl {includefestival}}{\chapter[ערבית ליו״ט]{\adforn{47} ערבית ליו״ט \adforn{19}}}{}}

\barachu

\hamaarivaravim

\ahavasolam

\shema

\veahavta

\vehaya

\vayomer{}

\emesveemuna

\hashkiveinu{וּפְרוֹס עָלֵֽינוּ סֻכַּת שְׁלוֹמֶֽךָ׃ בָּרוּךְ אַתָּה יְיָ פּוֹרֵס סֻכַּת שָׁלוֹם עָלֵֽינוּ וְעַל כׇּל־עַמּוֹ יִשְׂרָאֵל וְעַל יְרוּשָׁלַ‍ִם׃}

\ifboolexpr{togl {includeshabbat}}{\instruction{הקהל ביחד}\\}{\instruction{הקהל ביחד בשבת}}
\veshameru

\ifboolexpr{togl {includefestival} and not togl {includeshabbat}}{\instruction{ברגלים:}}{}

\ifboolexpr{togl {includefestival}}{\textbf{
		וַיְדַבֵּ֣ר מֹשֶׁ֔ה אֶת־מֹעֲדֵ֖י יְיָ֑ אֶל־בְּנֵ֖י יִשְׂרָאֵֽל׃
	}\source{ןיקרא כג}}{}



\halfkaddish

\section[תפילת העמידה]{\adforn{53} תפילת העמידה \adforn{25}}

\amidaopening{\shabbosshuva}{}

%\shabboskiddushhashem
\ifboolexpr{togl {includeshabbat} and togl {includefestival}}{
	\englishinst{On festivals, continue on page \pageref{maarivyt}.}
}{}

\ifboolexpr{togl {includeshabbat}}{\firstword{אַתָּה קִדַּֽשְׁתָּ}
אֶת־יוֹם הַשְּׁבִיעִי לִשְׁמֶֽךָ תַּכְלִית מַעֲשֵׂה שָׁמַֽיִם וָאָֽרֶץ וּבֵרַכְתּוֹ מִכׇּל־הַיָּמִים וְקִדַּשְׁתּוֹ מִכׇּל־הַזְּמַנִּים וְכֵן כָּתוּב בְּתוֹרָתֶֽךָ׃

\firstword{וַיְכֻלּ֛וּ}\source{בראשית ב}
הַשָּׁמַ֥יִם וְהָאָ֖רֶץ וְכׇל־צְבָאָֽם׃ וַיְכַ֤ל אֱלֹהִים֙ בַּיּ֣וֹם הַשְּׁבִיעִ֔י מְלַאכְתּ֖וֹ אֲשֶׁ֣ר עָשָׂ֑ה וַיִּשְׁבֹּת֙ בַּיּ֣וֹם הַשְּׁבִיעִ֔י מִכׇּל־מְלַאכְתּ֖וֹ אֲשֶׁ֥ר עָשָֽׂה׃ וַיְבָ֤רֶךְ אֱלֹהִים֙ אֶת־י֣וֹם הַשְּׁבִיעִ֔י וַיְקַדֵּ֖שׁ אֹת֑וֹ כִּ֣י ב֤וֹ שָׁבַת֙ מִכׇּל־מְלַאכְתּ֔וֹ אֲשֶׁר־בָּרָ֥א אֱלֹהִ֖ים לַֽעֲשֽׂוֹת׃

\shabboskiddushhayom{}}{}

\ifboolexpr{togl {includeshabbat} and togl {includefestival}}{\instruction{רצה וכו׳}

\sepline}{}

\ifboolexpr{togl {includefestival}}{\label{maarivyt}
\ytkiddushhayom{\YTShabboshavdalah}}{}


\ifboolexpr{togl {includeshabbat} and togl {includefestival}}{\sepline}{}

\retzeh

\yaalehveyavo

\zion

\maarivmodim

\ifboolexpr{togl {includeshabbat}}{\shabboschanukah

\shabboshodos}{}

\shabbosshalomrav

\tachanunim

\ifboolexpr{togl {includefestival}}{\englishinst{On Shabbat, the congregation stands and says the following together:}}{\englishinst{The congregation stands and says the following together:}}
\label{vayachulu}

\firstword{וַיְכֻלּ֛וּ} \source{בראשית ב}
הַשָּׁמַ֥יִם וְהָאָ֖רֶץ וְכׇל־צְבָאָֽם׃ וַיְכַ֤ל אֱלֹהִים֙ בַּיּ֣וֹם הַשְּׁבִיעִ֔י מְלַאכְתּ֖וֹ אֲשֶׁ֣ר עָשָׂ֑ה וַיִּשְׁבֹּת֙ בַּיּ֣וֹם הַשְּׁבִיעִ֔י מִכׇּל־מְלַאכְתּ֖וֹ אֲשֶׁ֥ר עָשָֽׂה׃ וַיְבָ֤רֶךְ אֱלֹהִים֙ אֶת־י֣וֹם הַשְּׁבִיעִ֔י וַיְקַדֵּ֖שׁ אֹת֑וֹ כִּ֣י ב֤וֹ שָׁבַת֙ מִכׇּל־מְלַאכְתּ֔וֹ אֲשֶׁר־בָּרָ֥א אֱלֹהִ֖ים לַעֲשֽׂוֹת׃


\ssubsection{\adforn{48} מעין שבע \adforn{22}}\\
\englishinst{The following is not said on the first nights of Pesa\d{h}.}
בָּרוּךְ אַתָּה יְיָ אֱלֹהֵֽינוּ וֵאלֹהֵי אֲבוֹתֵֽינוּ \middot אֱלֹהֵי אַבְרָהָם אֱלֹהֵי יִצְחָק וֵאלֹהֵי יַעֲקֹב \middot הָאֵל הַגָּדוֹל הַגִּבּוֹר וְהַנּוֹרָא אֵל עֶלְיוֹן קֹנֵה שָׁמַֽיִם וָאָֽרֶץ׃\\
\englishinst{The congregation says the following paragraph, and the leader repeats it and continues with the paragraph after:}
מָגֵן אָבוֹת בִּדְבָרוֹ מְחַיֵּה מֵתִים בְּמַאֲמָרוֹ הָאֵל
(\instruction{בשבת שובה:} הַמֶּֽלֶךְ)
הַקָּדוֹשׁ שֶׁאֵין כָּמֽוֹהוּ הַמֵּנִֽיחַ לְעַמּוֹ בְּיוֹם שַׁבַּת קָדְשׁוֹ כִּי בָם רָצָה לְהָנִֽיחַ לָהֶם׃ לְפָנָיו נַעֲבוֹד בְּיִרְאָה וָפַֽחַד וְנוֹדֶה לִשְׁמוֹ בְּכׇל־יוֹם תָּמִיד מֵעֵין הַבְּרָכוֹת׃ אֵל הַהוֹדָאוֹת אֲדוֹן הַשָּׁלוֹם מְקַדֵּשׁ הַשַּׁבָּת וּמְבָרֵךְ שְׁבִיעִי וּמֵנִֽיחַ בִּקְדֻשָּׁה לְעַם מְדֻשְּׁנֵי עֹֽנֶג זֵֽכֶר לְמַעֲשֵׂה בְרֵאשִׁית׃

\shabboskiddushhayom{}

\fullkaddish

\newcommand{\kiddushshabbateve}{\firstword{בָּרוּךְ}
	אַתָּה יְיָ אֱלֹהֵֽינוּ מֶֽלֶךְ הָעוֹלָם אֲשֶׁר קִדְּשָֽׁנוּ בְּמִצְוֹתָיו וְרָֽצָה בָֽנוּ וְשַׁבַּת קָדְשׁוֹ בְּאַהֲבָה וּבְרָצוֹן הִנְחִילָֽנוּ זִכָּרוֹן לְמַעֲשֵׂה בְרֵאשִׁית׃ כִּי הוּא יוֹם תְּחִלָּה לְמִקְרָאֵי קֹֽדֶשׁ זֵֽכֶר לִיצִיאַת מִצְרָֽיִם׃ כִּי בָֽנוּ בָחַֽרְתָּ וְאוֹתָֽנוּ קִדַּֽשְׁתָּ מִכׇּל־הָעַמִּים׃ וְשַׁבַּת קׇדְשְׁךָ בְּאַהֲבָה וּבְרָצוֹן הִנְחַלְתָּֽנוּ׃ בָּרוּךְ אַתָּה יְיָ מְקַדֵּשׁ הַשַּׁבָּת׃}
\newcommand{\kiddushYTeve}{\firstword{בָּרוּךְ}
	אַתָּה יְיָ אֱלֹהֵֽינוּ מֶֽלֶךְ הָעוֹלָם אֲשֶׁר בָּֽחַר בָּֽנוּ מִכׇּל־עָם וְרוֹמְמָֽנוּ מִכׇּל־לָשׁוֹן וְקִדְּשָֽׁנוּ בְּמִצְוֹתָיו׃ וַתִּתֶּן לָֽנוּ יְיָ אֱלֹהֵֽינוּ בְּאַהֲבָה
	\shabaddition{שַׁבָּתוֹת לִמְנוּחָה וּ}
	מוֹעֲדִים לְשִׂמְחָה חַגִּים וּזְמַנִּים לְשָׂשׂוֹן׃ אֶת־יוֹם
	\shabaddition{הַשַּׁבָּת הַזֶּה וְאֶת יוֹם} \\
	\begin{tabular}{>{\centering\arraybackslash}m{.2\textwidth} | >{\centering\arraybackslash}m{.2\textwidth} | >{\centering\arraybackslash}m{.2\textwidth} | >{\centering\arraybackslash}m{.25\textwidth}}
		\instruction{לפסח} & \instruction{לשבעות} & \instruction{לסכות} &
		\instruction{לשמיני עצרת ולשמ״ת}
		\\
		חַג הַמַּצּוֹת הַזֶּה זְמַן חֵרוּתֵֽנוּ&
		חַג הַשָּׁבֻעוֹת הַזֶּה זְמַן מַתַּן תּוֹרָתֵֽנוּ&
		חַג הַסֻּכּוֹת הַזֶּה זְמַן שִׂמְחָתֵֽנוּ &
		שְׁמִינִי חַג הָעֲצֶֽרֶת הַזֶּה זְמַן שִׂמְחָתֵֽנוּ\\
		
	\end{tabular}
	
	\shabaddition{בְּאַהֲבָה}
	מִקְרָא קֹֽדֶשׁ זֵֽכֶר לִיצִיאַת מִצְרָֽיִם׃ כִּי בָֽנוּ בָחַֽרְתָּ וְאוֹתָֽנוּ קִדַּֽשְׁתָּ מִכׇּל־הָעַמִּים \shabaddition{וְשַׁבַּת} וּמוֹעֲדֵי קׇדְשֶֽׁךָ \shabaddition{בְּאַהֲבָה וּבְרָצוֹן} בְּשִׂמְחָה וּבְשָׂשׂוֹן הִנְחַלְתָּֽנוּ׃ בָּרוּךְ אַתָּה יְיָ מְקַדֵּשׁ \shabaddition{הַשַּׁבָּת וְ} יִשְׂרָאֵל וְהַזְּמַנִּים׃
	
	\begin{sometimes}
		
		\englishinst{On Saturday night, include Havdalah:}
		בָּרוּךְ אַתָּה יְיָ אֱלֹהֵֽינוּ מֶֽלֶךְ הָעוֹלָם בּוֹרֵא מְאוֹרֵי הָאֵשׁ׃
		
		בָּרוּךְ אַתָּה יְיָ אֱלֹהֵֽינוּ מֶֽלֶךְ הָעוֹלָם הַמַּבְדִיל בֵּין קֹֽדֶשׁ לְחוֹל בֵּין אוֹר לְחֹֽשֶׁךְ בֵּין יִשְׂרָאֵל לָעַמִּים בֵּין יוֹם הַשְּׁבִיעִי לְשֵֽׁשֶׁת יְמֵי הַמַּעֲשֶׂה׃ בֵּין קְדֻשַּׁת שַׁבָּת לִקְדֻשַּׁת יוֹם טוֹב הִבְדַּֽלְתָּ וְאֶת־יוֹם הַשְּׁבִיעִי מִשֵּֽׁשֶׁת יְמֵי הַמַּעֲשֶׂה קִדַּֽשְׁתָּ הִבְדַּֽלְתָּ וְקִדַּֽשְׁתָּ אֶת־עַמְּךָ יִשְׂרָאֵל בִּקְדֻשָּׁתֶֽךָ׃ בָּרוּךְ אַתָּה יְיָ הַמַּבְדִּיל בֵּין קֹֽדֶשׁ לְקֹֽדֶשׁ׃
		
\end{sometimes}}

\englishinst{In many congregations, the reader recites kiddush here.}
\firstword{בָּרוּךְ}
אַתָּה יְיָ אֱלֹהֵֽינוּ מֶֽלֶךְ הָעוֹלָם בּוֹרֵא פְּרִי הַגָּֽפֶן׃

\ifboolexpr{togl {includefestival} and togl {includeshabbat}}{\instruction{בשבת׃}}{}
\ifboolexpr{togl {includeshabbat}}{\kiddushshabbateve}{}

\ifboolexpr{togl {includefestival} and togl {includeshabbat}}{\instruction{ביו״ט׃}}{}
\ifboolexpr{togl {includefestival}}{\kiddushYTeve}{}

\englishinst{Between Pesa\d{h} and Shavu'ot, count the Omer on page \pageref{sefiras haomer}.}
%\instruction{בימי ספירה, סופרים כאן את העומר בעמ׳ \pageref{sefiras haomer}}

\aleinu

\ifboolexpr{togl {includeshabbat}}{\ledavid}{}
\\
\mournerskaddish
\nextpage

\firstword{יִגְדַּל}
אֱלהִים חַי וְיִשְׁתַּבַּח, \hfill נִמְצָא וְאֵין עֵת אֶל מְצִיאוּתוֹ׃ \\
אֶחָד וְאֵין יָחִיד כְּיִחוּדוׂ, \hfill נֶעְלָם וְגַם אֵין סוׂף לְאַחְדוּתוֹ׃ \\
אֵין לוׂ דְמוּת הַגּוּף וְאֵינוׂ גוּף,\hfill לׂא נַעֲרךְ אֵלָיו קְדֻשָּׁתוֹ׃ \\
קַדְמוׂן לְכׇל־דָּבָר אֲשֶׁר נִבְרָא,\hfill רִאשׁוׂן וְאֵין רֵאשִׁית לְרֵאשִׁיתוֹ׃ \\
הִנּו אֲדוׂן עוׂלָם לְכׇל־נוׂצָר,\hfill יוׂרֶה גְדֻלָּתוׂ וּמַלְכוּתוֹ׃ \\
שֶׁפַע נְבוּאָתוׂ נְתָנוׂ,\hfill אֶל אַנְשֵׁי סְגֻלָּתוׂ וְתִפְאַרְתּוֹ׃ \\
לׂא קָם בְּיִשרָאֵל כְּמשֶׁה עוׂד,\hfill נָבִיא וּמַבִּיט אֶת־תְּמוּנָתוֹ׃ \\
תּוׂרַת אֱמֶת נָתַן לְעַמּוׂ אֵל,\hfill עַל יַד נְבִיאוׂ נֶאֱמַן בֵּיתוֹ׃ \\
לׂא יַחֲלִיף הָאֵל וְלׂא יָמִיר דָּתוׂ,\hfill לְעוׂלָמִים לְזוּלָתוֹ׃ \\
צוׂפֶה וְיוׂדֵעַ סְתָרֵינוּ,\hfill מַבִּיט לְסוׂף דָּבָר בְְַּקַדְמָתוֹ׃ \\
גּוׂמֵל לְאִישׁ חֶסֶד כְּמִפְעָלוׂ,\hfill נוׂתֵן לְרָשָׁע רַע כְּרִשְׁעָתוֹ׃ \\
יִשְׁלַח לְקֵץ הַיָּמִין מְשִׁיחֵנוּ,\hfill לִפְדּות מְחַכֵּי קֵץ יְשׁוּעָתוֹ׃ \\
מֵתִים יְחַיֶּה אֵל בְּרב חַסְדּוׂ,\hfill בָּרוּךְ עֲדֵי עַד שֵׁם תְּהִלָּתוֹ׃\\

\vfill

\adforn{43}\quad\adforn{4}\quad\adforn{42}\\

\section[קידוש ליל שבת]{\adforn{47} סדר קידוש ליל שבת בבית \adforn{19}}



\medskip

\newcommand{\birkashabonim}{
\ssubsection{\adforn{18} ברכת הבנים \adforn{17}}

\begin{tabular}{>{\centering\arraybackslash}m{.4\textwidth} | >{\centering\arraybackslash}m{.4\textwidth}}
\instruction{לבנים} & \instruction{לבנות}\\
יְשִֽׂמְךָ֣ אֱלֹהִ֔ים כְּאֶפְרַ֖יִם וְכִמְנַשֶּׁ֑ה׃ \mdsource{בראשית מח}&
יְשִׂמֵךְ אֱלׂהִים כְּשָׂרָה, רִבְקָה, רָחֵל, וְלֵאָה [אֲשֶׁר בָּנוּ אֶת־בֵּית יִשְׂרָאֵל׃] \mdsource{רות ד}׃
\end{tabular}

יְבָרֶכְךָ֥ יְיָ֖ וְיִשְׁמְרֶֽךָ׃\\ \source{במידבר ו}
יָאֵ֨ר יְיָ֧ ׀ פָּנָ֛יו אֵלֶ֖יךָ וִֽיחֻנֶּֽךָּ׃\\
יִשָּׂ֨א יְיָ֤ ׀ פָּנָיו֙ אֵלֶ֔יךָ וְיָשֵׂ֥ם לְךָ֖ שָׁלֽוֹם׃ \\

}



\birkashabonim

\medskip

\ssubsection{\adforn{18} שלום עליכם \adforn{17}}


\firstword{שָׁלוֹם}
עֲלֵיכֶם מַלְאֲכֵי הַשָּׁרֵת מַלְאֲכֵי עֶלְיוֹן\\ מִמֶֽלֶךְ מַלְכֵי הַמְּלָכִים הַקָּדוֹשׁ בָּרוּךְ הוּא׃ \\
בּוֹאֲכֶם לְשָׁלוֹם מַלְאֲכֵי הַשָּׁלוֹם מַלְאֲכֵי עֶלְיוֹן\\ מִמֶֽלֶךְ מַלְכֵי הַמְּלָכִים הַקָּדוֹשׁ בָּרוּךְ הוּא׃\\
בָּרְכֽוּנִי לְשָׁלוֹם מַלְאֲכֵי הַשָּׁלוֹם מַלְאֲכֵי עֶלְיוֹן \\ מִמֶֽלֶךְ מַלְכֵי הַמְּלָכִים הַקָּדוֹשׁ בָּרוּךְ הוּא׃\\
צֵאתְכֶם לְשָׁלוֹם מַלְאֲכֵי הַשָּׁלוֹם מַלְאֲכֵי עֶלְיוֹן\\ מִמֶֽלֶךְ מַלְכֵי הַמְּלָכִים הַקָּדוֹשׁ בָּרוּךְ הוּא׃

\vfill
%\clearpage

\ssubsection{\adforn{18} אשת חיל \adforn{17}}

\firstword{אֵֽשֶׁת־חַ֭יִל}\source{משלי לא}
מִ֣י יִמְצָ֑א וְרָחֹ֖ק מִפְּנִינִ֣ים מִכְרָֽהּ׃ \hfill\break
בָּ֣טַח בָּ֭הּ לֵ֣ב בַּעְלָ֑הּ וְ֝שָׁלָ֗ל לֹ֣א יֶחְסָֽר׃ \hfill\break
גְּמָלַ֣תְהוּ ט֣וֹב וְלֹא־רָ֑ע כֹּ֗֝ל יְמֵ֣י חַיֶּֽיהָ׃ \hfill\break
דָּ֭רְשָׁה צֶ֣מֶר וּפִשְׁתִּ֑ים וַ֝תַּ֗עַשׂ בְּחֵ֣פֶץ כַּפֶּֽיהָ׃ \hfill\break
הָ֭יְתָה כׇּאֳנִיּ֣וֹת סוֹחֵ֑ר מִ֝מֶּרְחָ֗ק תָּבִ֥יא לַחְמָֽהּ׃ \hfill\break
וַתָּ֤קׇם ׀ בְּע֬וֹד לַ֗יְלָה וַתִּתֵּ֣ן טֶ֣רֶף לְבֵיתָ֑הּ וְ֝חֹ֗ק לְנַעֲרֹתֶֽיהָ׃ \hfill\break
זָֽמְמָ֣ה שָׂ֭דֶה וַתִּקָּחֵ֑הוּ מִפְּרִ֥י כַ֝פֶּ֗יהָ נָ֣טְעָה כָּֽרֶם׃ \hfill\break
חָֽגְרָ֣ה בְע֣וֹז מׇתְנֶ֑יהָ וַ֝תְּאַמֵּ֗ץ זְרוֹעֹתֶֽיהָ׃ \hfill\break
טָ֭עֲמָה כִּי־ט֣וֹב סַחְרָ֑הּ לֹא־יִכְבֶּ֖ה בַלַּ֣יְלָה נֵרָֽהּ׃ \hfill\break
יָ֭דֶיהָ שִׁלְּחָ֣ה בַכִּישׁ֑וֹר וְ֝כַפֶּ֗יהָ תָּ֣מְכוּ פָֽלֶךְ׃ \hfill\break
כַּ֭פָּהּ פָּֽרְשָׂ֣ה לֶעָנִ֑י וְ֝יָדֶ֗יהָ שִׁלְּחָ֥ה לָאֶבְיֽוֹן׃ \hfill\break
לֹא־תִירָ֣א לְבֵיתָ֣הּ מִשָּׁ֑לֶג כִּ֥י כׇל־בֵּ֝יתָ֗הּ לָבֻ֥שׁ שָׁנִֽים׃ \hfill\break
מַרְבַדִּ֥ים עָֽשְׂתָה־לָּ֑הּ שֵׁ֖שׁ וְאַרְגָּמָ֣ן לְבוּשָֽׁהּ׃ \hfill\break
נוֹדָ֣ע בַּשְּׁעָרִ֣ים בַּעְלָ֑הּ בְּ֝שִׁבְתּ֗וֹ עִם־זִקְנֵי־אָֽרֶץ׃ \hfill\break
סָדִ֣ין עָ֭שְׂתָה וַתִּמְכֹּ֑ר וַ֝חֲג֗וֹר נָתְנָ֥ה לַֽכְּנַעֲנִֽי׃ \hfill\break
עֹז־וְהָדָ֥ר לְבוּשָׁ֑הּ וַ֝תִּשְׂחַ֗ק לְי֣וֹם אַחֲרֽוֹן׃ \hfill\break
פִּ֭יהָ פָּתְחָ֣ה בְחׇכְמָ֑ה וְת֥וֹרַת חֶ֝֗סֶד עַל־לְשׁוֹנָֽהּ׃ \hfill\break
צ֭וֹפִיָּה הֲלִיכ֣וֹת בֵּיתָ֑הּ וְלֶ֥חֶם עַ֝צְל֗וּת לֹ֣א תֹאכֵֽל׃ \hfill\break
קָ֣מוּ בָ֭נֶיהָ וַֽיְאַשְּׁר֑וּהָ בַּ֝עְלָ֗הּ וַֽיְהַלְלָֽהּ׃ \hfill\break
רַבּ֣וֹת בָּ֭נוֹת עָ֣שׂוּ חָ֑יִל וְ֝אַ֗תְּ עָלִ֥ית עַל־כֻּלָּֽנָה׃ \hfill\break
שֶׁ֣קֶר הַ֭חֵן וְהֶ֣בֶל הַיֹּ֑פִי אִשָּׁ֥ה יִרְאַת־יְ֝יָ֗ הִ֣יא תִתְהַלָּֽל׃ \hfill\break
תְּנוּ־לָ֭הּ מִפְּרִ֣י יָדֶ֑יהָ וִיהַלְל֖וּהָ בַשְּׁעָרִ֣ים מַֽעֲשֶֽׂיהָ׃ \hfill\break


\section[קידוש ליל שבת]{\adforn{18} קידוש ליל שבת \adforn{17}}

\label{shabbatkiddush}
\begin{footnotesize}וַֽיְהִי־עֶ֥רֶב וַֽיְהִי־בֹ֖קֶר\end{footnotesize}
י֥וֹם הַשִּׁשִּֽׁי׃ וַיְכֻלּ֛וּ \source{בראשית ב}הַשָּׁמַ֥יִם וְהָאָ֖רֶץ וְכׇל־צְבָאָֽם׃ וַיְכַ֤ל אֱלֹהִים֙ בַּיּ֣וֹם הַשְּׁבִיעִ֔י מְלַאכְתּ֖וֹ אֲשֶׁ֣ר עָשָׂ֑ה וַיִּשְׁבֹּת֙ בַּיּ֣וֹם הַשְּׁבִיעִ֔י מִכׇּל־מְלַאכְתּ֖וֹ אֲשֶׁ֥ר עָשָֽׂה׃ וַיְבָ֤רֶךְ אֱלֹהִים֙ אֶת־י֣וֹם הַשְּׁבִיעִ֔י וַיְקַדֵּ֖שׁ אֹת֑וֹ כִּ֣י ב֤וֹ שָׁבַת֙ מִכׇּל־מְלַאכְתּ֔וֹ אֲשֶׁר־בָּרָ֥א אֱלֹהִ֖ים לַעֲשֽׂוֹת׃

\savri
\firstword{בָּרוּךְ}
אַתָּה יְיָ אֱלֹהֵֽינוּ מֶֽלֶךְ הָעוֹלָם בּוֹרֵא פְּרִי הַגָּֽפֶן׃

\kiddushshabbateve

\begin{sometimes}

\instruction{בחול המועד סוכות:}\\
בָּרוּךְ אַתָּה יְיָ אֱלֹהֵינוּ מֶלֶךְ הָעוֹלָם, אֲשֶׁר קִדְּשָׁנוּ בְּמִצְוֹתָיו, וְצִוָּנוּ לֵישֵׁב בַּסֻּכָּה׃

\end{sometimes}

\adforn{43}\quad\adforn{4}\quad\adforn{42}\\

\ifboolexpr{togl {includefestival}}{\section[קידוש ליל יום טוב]{\adforn{53} קידוש ליל יום טוב \adforn{25}}
\label{kiddush leil yom tov}
\instruction{בשבת׃}
\begin{footnotesize}וַֽיְהִי־עֶ֥רֶב וַֽיְהִי־בֹ֖קֶר\end{footnotesize}
י֥וֹם הַשִּׁשִּֽׁי׃ וַיְכֻלּ֛וּ \source{בראשית ב}הַשָּׁמַ֥יִם וְהָאָ֖רֶץ וְכׇל־צְבָאָֽם׃ וַיְכַ֤ל אֱלֹהִים֙ בַּיּ֣וֹם הַשְּׁבִיעִ֔י מְלַאכְתּ֖וֹ אֲשֶׁ֣ר עָשָׂ֑ה וַיִּשְׁבֹּת֙ בַּיּ֣וֹם הַשְּׁבִיעִ֔י מִכׇּל־מְלַאכְתּ֖וֹ אֲשֶׁ֥ר עָשָֽׂה׃ וַיְבָ֤רֶךְ אֱלֹהִים֙ אֶת־י֣וֹם הַשְּׁבִיעִ֔י וַיְקַדֵּ֖שׁ אֹת֑וֹ כִּ֣י ב֤וֹ שָׁבַת֙ מִכׇּל־מְלַאכְתּ֔וֹ אֲשֶׁר־בָּרָ֥א אֱלֹהִ֖ים לַעֲשֽׂוֹת׃

\sepline


\savri
\firstword{בָּרוּךְ}
אַתָּה יְיָ אֱלֹהֵֽינוּ מֶֽלֶךְ הָעוֹלָם בּוֹרֵא פְּרִי הַגָּֽפֶן׃


\kiddushYTeve



\vspace{-.5\baselineskip}
\instruction{בסוכת:}
בָּרוּךְ אַתָּה יְיָ אֱלֹהֵינוּ מֶלֶךְ הָעוֹלָם, אֲשֶׁר קִדְּשָׁנוּ בְּמִצְוֹתָיו, וְצִוָּנוּ לֵישֵׁב בַּסֻּכָּה׃

\firstword{בָּרוּךְ}
אַתָּה יְיָ אֱלֹהֵינוּ מֶלֶךְ הָעוֹלָם, שֶׁהֶחֱיָנוּ וְקִיְּמָנוּ וְהִגִּיעָנוּ לַזְמַן הַזֶּה׃

\adforn{43}\quad\adforn{4}\quad\adforn{42}\\
}


%\vspace{\baselineskip}
%{\let\clearpage\relax
%\chapter[קידוש ליל ראש השנה]{\adforn{53} קידוש ליל ראש השנה \adforn{25}}}
%\begin{footnotesize}
%סַבְרִי מָרָנָן וְרְבָּנָן וְרַבּוֹתַי\\
%\end{footnotesize}
%בָּרוּךְ אַתָּה יְיָ אֱלֹהֵֽינוּ מֶֽלֶךְ הָעוֹלָם בּוֹרֵא פְּרִי הַגָּֽפֶן׃
%
%בָּרוּךְ אַתָּה יְיָ אֱלֹהֵֽינוּ מֶֽלֶךְ הָעוֹלָם אֲשֶׁר בָּֽחַר בָּֽנוּ מִכׇּל־עָם וְרוֹמְמָֽנוּ מִכׇּל־לָשׁוֹן וְקִדְּשָֽׁנוּ בְּמִצְוֹתָיו׃ וַתִּתֶּן לָֽנוּ יְיָ אֱלֹהֵֽינוּ בְּאַהֲבָה יוֹם [הַשַּׁבָּת הַזֶּה וְאֶת יוֹם] הַזִכָּרוֹן הַזֶּה יוֹם [זִכְרוֹן] תְּרוּעָה [בְּאַהֲבָה] מִקְרָא קֹֽדֶשׁ זֵֽכֶר לִיצִיאַת מִצְרָֽיִם׃ כִּי בָֽנוּ בָחַֽרְתָּ וְאוֹתָֽנוּ קִדַּֽשְׁתָּ מִכׇּל־הָעַמִּים וּדְבָרְךָ מַלְכֵּֽנוּ אֱמֶת וְקַיָּם לָעַד׃ בָּרוּךְ אַתָּה יְיָ מֶֽלֶךְ עַל כׇּל־הָאָֽרֶץ מְקַדֵּשׁ [הַשַּׁבָּת וְ] יִשְׂרָאֵל וְיוֹם הַזִּכָּרוֹן׃
%
%
%
%\begin{sometimes}
%
%\instruction{במוצאי שבת אומרים הבדלה:}\\
%בָּרוּךְ אַתָּה יְיָ אֱלֹהֵֽינוּ מֶֽלֶךְ הָעוֹלָם בּוֹרֵא מְאוֹרֵי הָאֵשׁ׃
%
%בָּרוּךְ אַתָּה יְיָ אֱלֹהֵֽינוּ מֶֽלֶךְ הָעוֹלָם הַמַּבְדִיל בֵּין קֹֽדֶשׁ לְחוֹל בֵּין אוֹר לְחֹֽשֶׁךְ בֵּין יִשְׂרָאֵל לָעַמִּים בֵּין יוֹם הַשְּׁבִיעִי לְשֵֽׁשֶׁת יְמֵי הַמַּעֲשֶׂה׃ בֵּין קְדֻשַּׁת שַׁבָּת לִקְדֻשַּׁת יוֹם טוֹב הִבְדַּֽלְתָּ וְאֶת־יוֹם הַשְּׁבִיעִי מִשֵּֽׁשֶׁת יְמֵי הַמַּעֲשֶׂה קִדַּֽשְׁתָּ הִבְדַּֽלְתָּ וְקִדַּֽשְׁתָּ אֶת־עַמְּךָ יִשְׂרָאֵל בִּקְדֻשָּׁתֶֽךָ׃ בָּרוּךְ אַתָּה יְיָ הַמַּבְדִּיל בֵּין קֹֽדֶשׁ לְקֹֽדֶשׁ׃
%
%\end{sometimes}
%
%\firstword{בָּרוּךְ}
% אַתָּה יְיָ אֱלֹהֵינוּ מֶלֶךְ הָעוֹלָם, שֶׁהֶחֱיָנוּ וְקִיְּמָנוּ וְהִגִּיעָנוּ לַזְמַן הַזֶּה׃



