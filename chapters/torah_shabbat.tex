\vspace{0.25in}
\chapter[סדר קריאת התורה]{\adforn{53} סדר קריאת התורה \adforn{25}}

\longpesicha

\brikhshmei

\textbf{שְׁמַ֖ע יִשְׂרָאֵ֑ל יְיָ֥ אֱלֹהֵ֖ינוּ יְיָ֥ ׀ אֶחָֽד׃}

\textbf{אֶחָד אֱלֹהֵֽינוּ גָּדוֹל אֲדוֹנֵֽנוּ קָדוֹשׁ שְׁמוֹ׃}

\gadlu

\label{al hakol}
%\firstword{עַל הַכֹּל}
%יִתְגַּדַּל וְיִתְקַדַּשׁ וְיִשְׁתַּבַּח וְיִתְפָּאַר וְיִתְרוֹמַם וְיִתְנַשֵּׂא שְׁמוֹ שֶׁל מֶֽלֶךְ מַלְכֵי הַמְּלָכִים הַקְָּדוֹשׁ בָּרוּךְ הוּא בָּעוֹלָמוֹת שֶׁבָּרָא הָעוֹלָם הַזֶּה וְהָעוֹלָם הַבָּא כִּרְצוֹנוֹ וְכִרְצוֹן יְרֵאָיו וְכִרְצוֹן כׇּל־בֵּית יִשְׂרָאֵל׃ צוּר הָעוֹלָמִים אֲדוֹן כׇּל־הַבְּרִיּוֹת אֱלֽוֹהַּ כׇּל־הַנְּפָשׁוֹת הַיּוֹשֵׁב בְּמֶרְחֲבֵי מָרוֹם הַשּׁוֹכֵן בִּשְׁמֵי שְׁמֵי קֶֽדֶם׃ קְדֻשָּׁתוֹ עַל הַחַיּוֹת וּקְדֻשָּׁתוֹ עַל כִּסֵּא הַכָּבוֹד׃ וּבְכֵן יִתְקַדַּשׁ שִׁמְךָ בָּֽנוּ יְיָ אֱלֹהֵֽינוּ לְעֵינֵי כׇּל־חָי׃ וְנֹאמַר לְפָנָיו שִׁיר חָדָשׁ כַּכָּתוּב׃
%שִׁ֥ירוּ \source{תהלים סח}לֵֽאלֹהִֽ֘ים זַמְּר֢וּ שְׁ֫מ֥וֹ סֹ֡לּוּ לָֽרֹכֵ֣ב בָּֽ֭עֲרָבוֹת בְּיָ֥הּ שְׁ֝מ֗וֹ וְעִלְז֥וּ לְפָנָֽיו׃ וְנִרְאֵֽהוּ עַֽיִן בְּעַֽיִן בְּשׁוּבוֹ אֶל נָוֵֽהוּ כַּכָּתוּב׃
%\source{ישעיה נב}%
%כִּ֣י עַ֤יִן בְּעַ֨יִן֙ יִרְא֔וּ בְּשׁ֥וּב יְיָ֖ צִיּֽוֹן׃ וְנֶאֱמַר׃
%וְנִגְלָ֖ה \source{ישעיה מ}כְּב֣וֹד יְיָ֑ וְרָא֤וּ כׇל־בָּשָׂר֙ יַחְדָּ֔ו כִּ֛י פִּ֥י יְיָ֖ דִּבֵּֽר׃

\avharachamim

\vayaazor

\torahbarachu

\hagomel

\misheberakhbaby

%\misheberakhbarmitzva

%\instruction{מי שבירך לחתן:}\\
%מִי שֶׁבֵּרַךְ אֲבוֹתֵֽינוּ אַבְרָהָם יִצְחָק וְיַעֲקֹב הוּא יְבָרֵךְ אֶת־הֶחָתָן \instruction{(פלוני בן פלוני)} וְאֶת־כַּלָתוֹ \instruction{פלונית בת פלוני} בַּעֲבוּר שֶׁנָדְרוּ נִדְבַת לִבָּם... בִּשְׂכַר זֶה הַקָדוֹשׁ בָּרוּךְ הוּא יְבָרֵךְ אוֹתָם וְיִתֵּן לָהֶם בְּרָכָה וְהַצְלָחָה בְּכׇל־מַעֲשֵׂה יְדֵיהֶם וְיִזְכוּ לִבְנוֹת בַּֽיִת בְּיִשְׂרָאֵל לְשֵׁם וְלִתְהִלָה וְנֹאמַר אָמֵן:

\misheberakhcholim{‏שַׁבָּת הִיא מִלִּזְעוֹק}

%\misheberakholim{וְלִכְבוֹד הַשַּׁבָּת}
\englishinst{Half kaddish is recited before the Maftir is read.}
\halfkaddish

\hagbaha

\ssubsection{\adforn{18} סדר קריאת ההפטרה \adforn{17}}

\englishinst{Before reading the Haftara:}
\firstword{בָּר֙וּךְ}
אַתָּ֤ה יְ֙יָ אֱלֹ֙הֵֽינוּ֙ מֶ֣לֶךְ הָעוֹלָ֔ם אֲשֶׁ֤ר בָּחַר֙ בִּנְבִיאִ֣ים טוֹבִ֔ים וְרָצָ֥ה בְדִבְרֵיהֶ֖ם הַנֶּֽאֱמָרִ֣ים בֶּאֱמֶ֑ת בָּר֨וּךְ אַתָּ֜ה יְיָ֗ הַבּוֹחֵר֚ בַּתּוֹרָה֙ וּבְמֹשֶׁ֣ה עַבְדּ֔וֹ וּבְיִשְׂרָאֵ֣ל עַמּ֔וֹ וּבִנְבִיאֵ֥י הָֽאֱמֶ֖ת וְהַצֶֽדֶק׃

\englishinst{After reading the Haftara:}
\firstword{בָּרוּךְ}
אַתָּה יְיָ אֱלֹהֵֽינוּ מֶֽלֶךְ הָעוֹלָם צוּר כׇּל־הָעוֹלָמִים צַדִּיק בְּכׇל־הַדּוֹרוֹת הָאֵל הַנֶּאֱמָן הָאוֹמֵר וְעוֹשֶׂה הַמְדַבֵּר וּמְקַיֵּם שֶׁכׇּל־דְּבָרָיו אֱמֶת וָצֶֽדֶק׃ נֶאֱמָן אַתָּה הוּא יְיָ אֱלֹהֵֽינוּ וְנֶאֱמָנִים דְּבָרֶֽיךָ וְדָבָר אֶחָד מִדְּבָרֶֽיךָ אָחוֹר לֹא יָשׁוּב רֵיקָם כִּי אֵל מֶֽלֶךְ נֶאֱמָן וְרַחֲמָן אַֽתָּה׃ בָּרוּךְ אַתָּה יְיָ הָאֵל הַנֶּאֱמָן בְּכׇל־דְּבָרָיו׃

\firstword{רַחֵם}
עַל צִיּוֹן כִּי הִיא בֵּית חַיֵּֽינוּ וְלַעֲלֽוּבַת נֶֽפֶשׁ %תִּנְקֺם נָקָם
תּוֹשִֽׁיעַ בִּמְהֵרָה בְיָמֵֽינוּ׃ בָּרוּךְ אַתָּה יְיָ מְשַׂמֵּֽחַ צִיּוֹן בְּבָנֶֽיהָ׃

\firstword{שַׂמְּחֵֽנוּ}
יְיָ אֱלֹהֵֽינוּ בְּאֵלִיָּֽהוּ הַנָּבִיא עַבְדֶּֽךָ וּבְמַלְכוּת בֵּית דָּוִד מְשִׁיחֶֽךָ בִּמְהֵרָה יָבֹא וְיָגֵל לִבֵּֽנוּ עַל כִּסְאוֹ לֹא יֵשֵׁב זָר וְלֹא יִנְחֲלוּ עוֹד אֲחֵרִים אֶת־כְּבוֹדוֹ כִּי בְשֵׁם קׇדְשְׁךָ נִשְׁבַּעְתָּ לוֹ שֶׁלֹּא יִכְבֶּה נֵרוֹ לְעוֹלָם וָעֶד׃ בָּרוּךְ אַתָּה יְיָ מָגֵן דָּוִד׃

\firstword{עַל הַתּוֹרָה}
וְעַל הָעֲבוֹדָה וְעַל הַנְּבִיאִים וְעַל יוֹם הַשַּׁבָּת הַזֶּה שֶׁנָּתַֽתָּ לָֽנוּ יְיָ אֱלֹהֵֽינוּ לִקְדֻשָּׁה וְלִמְנוּחָה לְכָבוֹד וּלְתִפְאָֽרֶת׃ עַל הַכֹּל יְיָ אֱלֹהֵֽינוּ אָֽנוּ מוֹדִים לָךְ וּמְבָרְכִים אוֹתָךְ יִתְבָּרַךְ שִׁמְךָ בְּפִי כׇל־חַי תָּמִיד לְעוֹלָם וָעֶד׃ בָּרוּךְ אַתָּה יְיָ מְקַדֵּשׁ הַשַּׁבָּת׃


\ssubsection{\adforn{18} יקום פרקן \adforn{17}}

\yekumpurkans

\ssubsection{\adforn{18} ברכת החודש \adforn{17}}

יְהִי רָצוֹן מִלְּפָנֶֽיךָ יְיָ אֱלֹהֵֽינוּ וֵאלֹהֵי אֲבוֹתֵֽינוּ
שֶׁתְּחַדֵּשׁ עָלֵֽינוּ אֶת־הַחֹדֶשׁ הַזֶּה לְטוֹבָה וְלִבְרָכָה \middot
וְתִתֶּן־לָנוּ חַיִּים אֲרוּכִים,
חַיִּים שֶׁל שָׁלוֹם,
חַיִּים שֶׁל טוֹבָה,
חַיִּים שֶׁל בְּרָכָה,
חַיִּים שֶׁל פַּרְנָסָה,
חַיִּים שֶׁל חִלּוּץ עֲצָמוֹת,
חַיִּים שֶׁיֵשׁ בָּהֶם יִרְאַת שָׁמַֽיִם וְיִרְאַת חֵטְא,
חַיִּים שֶׁאֵין בָּהֶם בּוּשָׁה וּכְלִמָּה,
חַיִּים שֶׁל עֽשֶׁר וְכָבוֹד,
חַיִּים שֶׁתְּהֵא בָֽנוּ אַהֲבַת תּוֹרָה וְיִרְאַת שָׁמַֽיִם,
חַיִּים שֶׁיְּמַּלֵא יְיָ מִשְׁאֲלוֹת לִבֵּנוּ לְטוֹבָה \middot אָמֵן סֶלָה׃

\englishinst{Some announce the Molad at this point.}
\firstword{מִי שֶׁעָשָׂה}
נִסִּים לַאֲבוֹתֵֽינוּ וְגָאַל אוֹתָם מֵעַבְדוּת לְחֵרוּת \middot הוּא יִגְאַל אוֹתָנוּ בְּקָרוֹב וִיקַבֵּץ נִדָּחֵינוּ מֵאַרְבַּע כַּנְפוֹת הָאָֽרֶץ חֲבֵרִים כׇּל־יִשְׂרָאֵל \middot וְנֹאמַר אָמֵן׃

רֹאשׁ חֹדֶש ... יִהְיֶה בְּיוֹם ... הַבָּא עָלֵֽינוּ וְעַל כׇּל־יִשְׂרָאֵל לְטוֹבָה׃

יְחַדְּשֵׁהוּ הַקָּדוֹשׁ בָּרוּךְ הוּא, עָלֵֽינוּ וְעַל כׇּל־עַמּוֹ בֵּית יִשְׂרָאֵל, לְחַיִּים וּלְשָׁלוֹם, לְשָׂשׂוֹן וּלְשִׂמְחָה, לִישׁוּעָה וּלְנֶחָמָה, וְנֹאמַר אָמֵן׃

\label{avharachamim}
%\instruction{א״א אב הרחמים ביו״ט ובשבת מברכים (אבל אומרים בימי ספירה ובר״ח אב ובימים שאומרים יזכור)}\\
\englishinst{There are several customs as to when to say Av HaRa\d{h}amim. Most omit it on Shabbat Mevarekhim (except during the Omer and the Three Weeks, when it is said) or on other festive Shabbatot (such as the Four Parshiyot or Rosh \d{H}odesh). Some recite it only on the Shabbat before Shavu'ot and Shabbat \d{H}azon.}
\firstword{אַב הָרַחֲמִים}
שׁוֹכֵן מְרוֹמִים בְּרַחֲמָיו הָעֲצוּמִים הוּא יִפְקוֹד בְּרַחֲמִים הַחֲסִידִים וְהַיְשָׁרִים וְהַתְּמִימִים קְהִילּוֹת הַקֹּֽדֶשׁ שֶׁמָּסְרוּ נַפְשָׁם עַל קְדֻשַּׁת הַשֵּׁם \source{שמ״ב א}הַנֶּאֱהָבִ֤ים וְהַנְּעִימִם֙ בְּחַיֵּיהֶ֔ם וּבְמוֹתָ֖ם לֹ֣א נִפְרָ֑דוּ׃ מִנְּשָׁרִים קַֽלּוּ וּמֵאֲרָיוֹת גָּבֵֽרוּ לַעֲשׂוֹת רְצוֹן קוֹנָם וְחֵֽפֶץ צוּרָם׃ יִזְכְּרֵם אֱלֹהֵֽינוּ לְטוֹבָה עִם שְׁאָר צַדִּיקֵי עוֹלָם וְיִנְקוֹם בְּיָמֵֽינוּ לְעֵינֵֽינוּ נִקְמַת דַּם עֲבָדָיו הַשָּׁפוּךְ כַּכָּתוּב בְּתוֹרַת מֹשֶׁה אִישׁ הָאֱלֹהִים׃ \source{דברים לב}הַרְנִ֤ינוּ גוֹיִם֙ עַמּ֔וֹ כִּ֥י דַם־עֲבָדָ֖יו יִקּ֑וֹם וְנָקָם֙ יָשִׁ֣יב לְצָרָ֔יו וְכִפֶּ֥ר אַדְמָת֖וֹ עַמּֽוֹ׃ וְעַל יְדֵי עֲבָדֶֽיךָ הַנְּבִיאִים כָּתוּב לֵאמֹר׃ \source{יואל ד}וְנִקֵּ֖יתִי דָּמָ֣ם לֹֽא־נִקֵּ֑יתִי וַֽיְיָ֖ שֹׁכֵ֥ן בְּצִיּֽוֹן׃ וּבְכִתְבֵי הַקֹּֽדֶשׁ נֶאֱמַר׃ \source{תהלים עט}לָ֤מָּה יֹֽאמְר֣וּ הַגּוֹיִם֘ אַיֵּ֢ה אֱלֹֽהֵ֫יהֶ֥ם יִוָּדַ֣ע בַּגֹּייִ֣ם לְעֵינֵ֑ינוּ נִ֝קְמַ֗ת דַּם־עֲבָדֶ֥יךָ הַשָּׁפֽוּךְ׃ וְאוֹמֵר׃ \source{תהלים ט}כִּ֤י דֹרֵ֣שׁ דָּ֭מִים אֹתָ֣ם זָכָ֑ר לֹ֥א שָׁ֝כַ֗ח צַֽעֲקַ֥ת עֲנָוִֽים׃ וְאוֹמֵר׃ \source{תהלים קי}יָדִ֣ין בַּ֭גּוֹיִם מָלֵ֣א גְוִיּ֑וֹת מָ֥חַץ רֹ֝֗אשׁ עַל־אֶ֥רֶץ רַבָּֽה׃ מִ֭נַּחַל בַּדֶּ֣רֶךְ יִשְׁתֶּ֑ה עַל־כֵּ֝֗ן יָרִ֥ים רֹֽאשׁ׃

\ashrei

\yehalelu

\firstword{מִזְמ֗וֹר לְדָ֫וִ֥ד}\source{תהלים כט} %
הָב֣וּ לַ֭ייָ בְּנֵ֣י אֵלִ֑ים הָב֥וּ לַ֝ייָ֗ כָּב֥וֹד וָעֹֽז׃
הָב֣וּ לַ֭ייָ כְּב֣וֹד שְׁמ֑וֹ הִשְׁתַּחֲו֥וּ לַ֝ייָ֗ בְּהַדְרַת־קֹֽדֶשׁ׃
ק֥וֹל יְיָ֗ עַל־הַ֫מָּ֥יִם אֵֽל־הַכָּב֥וֹד הִרְעִ֑ים יְ֝יָ֗ עַל־מַ֥יִם רַבִּֽים׃
קוֹל־יְיָ֥ בַּכֹּ֑חַ ק֥וֹל יְ֝יָ֗ בֶּהָדָֽר׃
ק֣וֹל יְיָ֭ שֹׁבֵ֣ר אֲרָזִ֑ים וַיְשַׁבֵּ֥ר יְ֝יָ֗ אֶת־אַרְזֵ֥י הַלְּבָנֽוֹן׃
וַיַּרְקִידֵ֥ם כְּמוֹ־עֵ֑גֶל לְבָנ֥וֹן וְ֝שִׂרְיֹ֗ן כְּמ֣וֹ בֶן־רְאֵמִֽים׃
קוֹל־יְיָ֥ חֹצֵ֗ב לַהֲב֥וֹת אֵֽשׁ׃
ק֣וֹל יְיָ֭ יָחִ֣יל מִדְבָּ֑ר יָחִ֥יל יְ֝יָ֗ מִדְבַּ֥ר קָדֵֽשׁ׃
ק֤וֹל יְיָ֨ ׀ יְחוֹלֵ֣ל אַיָּלוֹת֮ וַֽיֶּחֱשֹׂ֢ף יְעָ֫ר֥וֹת וּבְהֵיכָל֑וֹ כֻּ֝לּ֗וֹ אֹמֵ֥ר כָּבֽוֹד׃
יְיָ֭ לַמַּבּ֣וּל יָשָׁ֑ב וַיֵּ֥שֶׁב יְ֝יָ֗ מֶ֣לֶךְ לְעוֹלָֽם׃
יְיָ֗ עֹ֭ז לְעַמּ֣וֹ יִתֵּ֑ן יְיָ֓ ׀ יְבָרֵ֖ךְ אֶת־עַמּ֣וֹ בַשָּׁלֽוֹם׃

\etzchaim

\halfkaddish\\
