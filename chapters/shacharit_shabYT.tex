\ifboolexpr{togl {includeshabbat} and togl {includefestival}}{\chapter[שחרית לשבת ויו״ט]{\adforn{47} שחרית לשבת ויו״ט \adforn{19}}}{
\ifboolexpr{togl {includeshabbat}}{\chapter[שחרית לשבת]{\adforn{47} שחרית לשבת \adforn{19}}}{}
\ifboolexpr{togl {includefestival}}{\chapter[שחרית ליו״ט]{\adforn{47} שחרית ליו״ט \adforn{19}}}{}}

\label{nishmas}
\nishmat

\hael

\shochenad

\yishtabach
\ifboolexpr{togl {includeshabbat}}{\mimaamakim}{}
\halfkaddish

\section[קריאת שמע וברכותיה]{\adforn{53} קריאת שמע וברכותיה \adforn{25}}

\barachu

\firstword{
	בָּרוּךְ אַתָּה יְיָ אֱלֹהֵֽינוּ מֶֽלֶךְ הָעוֹלָם \middot יוֹצֵר אוֹר וּבוֹרֵא חֹֽשֶׁךְ עֹשֶׂה שָׁלוֹם וּבוֹרֵא אֶת־הַכֹּל׃}

%\instruction{כשאומרים יוצרות:}
%אוֹר עוֹלָם בְּאוֹצַר חַיִּים אוֹרוֹת מֵאוֹפֶל אָמַר וַיֶהִי׃
\ifboolexpr{togl {includefestival} and togl {includeshabbat}}{

\englishinst{On a Festival occurring on a weekday:}
\hameir \instruction{תִּתְבָּרַךְ וכו׳ עמ׳ \pageref{tisbarach}}
}{}

\ifboolexpr{togl {includefestival} and not togl {includeshabbat}}{
\englishinst{On weekdays:}
\hameir
\instruction{תִּתְבָּרַךְ וכו׳ עמ׳ \pageref{tisbarach}}

\englishinst{On Shabbat:}}{}
\firstword{הַכֹּל יוֹדֽוּךָ}
וְהַכֹּל יְשַׁבְּחֽוּךָ \middot וְהַכֹּל יֹאמְרוּ אֵין קָדוֹשׁ כַּיָי׃ הַכֹּל יְרוֹמְמֽוּךָ סֶּֽלָה יוֹצֵר הַכֹּל \middot הָאֵל הַפּוֹתֵֽחַ בְּכׇל־יוֹם דַּלְתוֹת שַׁעֲרֵי מִזְרָח \middot וּבוֹקֵֽעַ חַלּוֹנֵי רָקִֽיעַ מוֹצִיא חַמָּה מִמְּקוֹמָהּ וּלְבָנָה מִמְּכוֹן שִׁבְתָּהּ׃ וּמֵאִיר לָעוֹלָם כֻּלּוֹ וּלְיוֹשְׁבָיו שֶׁבָּרָא בְּמִדַּת רַחֲמִים׃ 
הַמֵּאִיר לָאָֽרֶץ וְלַדָּרִים עָלֶֽיהָ בְּרַחֲמִים \middot וּבְטוּבוֹ מְחַדֵּשׁ בְּכׇל־יוֹם תָּמִיד מַעֲשֵׂה בְרֵאשִׁית׃
הַמֶּֽלֶךְ הַמְּרוֹמָם לְבַדּוֹ מֵאָז \middot הַמְשֻׁבָּח וְהַמְפֹאָר וְהַמִּתְנַשֵּׂא מִימוֹת עוֹלָם׃
אֱלֹהֵי עוֹלָם בְּרַחֲמֶֽיךָ הָרַבִּים רַחֵם עָלֵֽינוּ \middot אֲדוֹן עֻזֵּֽנוּ צוּר מִשְׂגַּבֵּֽנוּ מָגֵן יִשְׁעֵֽנוּ מִשְׂגָּב בַּעֲדֵֽנוּ׃
אֵין כְּעֶרְכֶּֽךָ וְאֵין זוּלָתֶֽךָ \middot אֶפֶס בִּלְתֶּֽךָ וּמִי דּֽוֹמֶה לָּךְ׃
אֵין כְּעֶרְכְּךָ יְיָ אֱלֹהֵֽינוּ בָּעוֹלָם הַזֶּה \middot וְאֵין זוּלָתְךָ מַלְכֵּֽנוּ לְחַיֵּי הָעוֹלָם הַבָּא׃
אֶֽפֶס בִּלְתְּךָ גּוֹאֲלֵֽנוּ לִימוֹת הַמָּשִֽׁיחַ \middot וְאֵין דּֽוֹמֶה לְּךָ מוֹשִׁיעֵֽנוּ לִתְחִיַּת הַמֵּתִים׃

\acrostic{אֵ}ל אָדוֹן עַל כׇּל־הַמַּעֲשִׂים \hfill \acrostic{בָּ}רוּךְ וּמְבֹרָךְ בְּפִי כׇּל־נְשָׁמָה׃ \\
\acrostic{גׇּ}דְלוֹ וְטוּבוֹ מָלֵא עוֹלָם \hfill \acrostic{דַּֽ}עַת וּתְבוּנָה סוֹבְבִים אוֹתוֹ׃

\acrostic{הַ}מִּתְגָּאֶה עַל חַיּוֹת הַקֹּֽדֶשׁ \hfill \acrostic{וְ}נֶהְדָּר בְּכָבוֹד עַל הַמֶּרְכָּבָה׃\\
\acrostic{זְ}כוּת וּמִישׁוֹר לִפְנֵי כִסְאוֹ \hfill \acrostic{חֶֽ}סֶד וְרַחֲמִים לִפְנֵי כְבוֹדוֹ׃

\acrostic{ט}וֹבִים מְאוֹרוֹת שֶׁבָּרָא אֱלֹהֵֽינוּ \hfill \acrostic{יְ}צָרָם בְּדַֽעַת בְּבִינָה וּבְהַשְׂכֵּל׃\\
\acrostic{כֹּֽ}חַ וּגְבוּרָה נָתַן בָּהֶם \hfill \acrostic{לִ}הְיוֹת מוֹשְׁלִים בְּקֶֽרֶב תֵּבֵל׃

\acrostic{מְ}לֵאִים זִיו וּמְפִיקִים נֹֽגַהּ \hfill \acrostic{נָ}אֶה זִיוָם בְּכׇל־הָעוֹלָם׃ \\
\acrostic{שְׂ}מֵחִים בְּצֵאתָם וְשָׂשִׂים בְּבוֹאָם \hfill \acrostic{ע}וֹשִׂים בְּאֵימָה רְצוֹן קוׂנָם׃

\acrostic{פְּ}אֵר וְכָבוֹד נוֹתְנִים לִשְׁמוֹ \hfill \acrostic{צׇ}הֳלָה וְרִנָּה לְזֵֽכֶר מַלְכוּתוֹ׃ \\
\acrostic{קָ}רָא לַשֶּֽׁמֶשׁ וַיִּזְרַח אוֹר \hfill \acrostic{רָ}אָה וְהִתְקִין צוּרַת הַלְּבָנָה׃

\acrostic{שֶֽׁ}בַח נוֹתְנִים לוֹ\hfill כׇּל־צְבָא מָרוֹם \\ \acrostic{תִּ}פְאֶֽרֶת וּגְדֻלָּה\hfill שְׂרָפִים וְאוֹפַנִּים וְחַיּוֹת הַקֹּֽדֶשׁ׃

\firstword{לָאֵל}
אֲשֶׁר שָׁבַת מִכׇּל־הַמַּעֲשִׂים בַּיּוֹם הַשְּׁבִיעִי נִתְעַלָּה וְיָשַׁב עַל כִּסֵּא כְבוֹדוֹ \middot תִּפְאֶֽרֶת עָטָה לְיוֹם הַמְּנוּחָה עֹֽנֶג קָרָא לְיוֹם הַשַּׁבָּת׃
זֶה שֶֽׁבַח שֶׁלַּיּוֹם הַשְּׁבִיעִי שֶׁבּוֹ שָֽׁבַת אֵל מִכׇּל־מְלַאכְתּוֹ׃ וְיוֹם הַשְּׁבִיעִי מְשַׁבֵּֽחַ וְאוֹמֵר׃
\source{תהלים צב}%
מִזְמ֥וֹר שִׁ֗יר לְי֣וֹם הַשַּׁבָּֽת׃ ט֗וֹב לְהֹד֥וֹת לַייָ֑
לְפִיכָךְ יְפָאֲרוּ וִיבָרְכוּ לָאֵל כׇּל־יְצוּרָיו \middot שֶֽׁבַח יְקָר וּגְדֻלָּה יִתְּנוּ לְאֵל מֶֽלֶךְ יוֹצֵר כֹּל \middot הַמַּנְחִיל מְנוּחָה לְעַמּוֹ יִשְׂרָאֵל בִּקְדֻשָּׁתוֹ בְּיוֹם שַׁבַּת קֹֽדֶשׁ׃
שִׁמְךָ יְיָ אֱלֹהֵֽינוּ יִתְקַדַּשׁ \middot וְזִכְרְךָ מַלְכֵּֽנוּ יִתְפָּאַר בַּשָּׁמַֽיִם מִמַּֽעַל וְעַל הָאָֽרֶץ מִתָּֽחַת׃ תִּתְבָּרַךְ מוֹשִׁיעֵֽנוּ עַל שֶֽׁבַח מַעֲשֵׂה יָדֶֽיךָ וְעַל מְאוֹרֵי אוֹר שֶׁעָשִֽׂיתָ יְפָאֲרֽוּךָ סֶּֽלָה׃


\label{tisbarach}
\yotzerhameoros

\ahavaraba

\ifboolexpr{not togl {includeweekday}}{\label{morningshema}}{}
\shema

\veahavta

\vehaya

\vayomer{}

\emesveyatziv

\ezrasavoseinu

\gaalyisroel\\

\englishinst{On Festivals, recite the Amidah on page \pageref{YTamidah}.}

\section[תפילת העמידה]{\adforn{53} תפילת העמידה \adforn{25}}

\amidaopening{\shabbosshuva}{\englishinst{During the repetition of the Amidah, Kedusha is said here}}
\nextpage
\begin{Center}\ssubsection{\adforn{48} קדושה \adforn{22}}\end{Center}

\begin{footnotesize}
\begin{longtable}{ l p{.8\textwidth} }

\shatz &
נְקַדֵּשׁ אֶת־שִׁמְךָ בָּעוֹלָם כְּשֵׁם שֶׁמַּקְדִּישִׁים אוֹתוֹ בִּשְׁמֵי מָרוֹם כַּכָּתוּב עַל יַד נְבִיאֶךָ קָרָ֨א זֶ֤ה אֶל־זֶה֙ וְאָמַ֔ר \\

\vshatzkahal &
\kadoshkadoshkadosh\\

\shatz &
אָז בְּקוֹל־רַֽעַשׁ גָּדוֹל אַדִּיר וְחָזָק מַשְׁמִיעִים קוֹל מִתְנַשְּׂאִים לְעֻמַּת שְׂרָפִים לְעֻמָּתָם בָּרוּךְ יֹאמֵֽרוּ׃ \\

\vshatzkahal &
\textbf{בָּר֥וּךְ כְּבוֹד־יְיָ֖ מִמְּקוֹמֽוֹ׃} \\

\shatz &
מִמְּקוֹמְךָ מַלְכֵּֽנוּ תוֹפִֽיעַ וְתִמְלֹךְ עָלֵֽינוּ כִּי מְחַכִּים אֲנַֽחְנוּ לָךְ׃ מָתַי תִּמְלֹךְ בְּצִיּוֹן בְּקָרוֹב בְּיָמֵֽינוּ לְעֹלָם וָעֶד תִּשְׁכּוֹן׃ תִּתְגַּדַּל וְתִתְקַדַּשׁ בְּתוֹךְ יְרוּשָׁלַֽיִם עִירְךָ לְדוֹר וָדוֹר וּלְנֵֽצַח נְצָחִים׃ וְעֵינֵֽינוּ תִרְאֶֽינָה מַלְכוּתְךָ כַּדָּבָר הָאָמוּר בְּשִׁירֵי עֻזֶּךָ עַל יְדֵי דָּוִד מְשִֽׁיחַ צִדְקֶֽךָ׃ \\

\vshatzkahal &
\textbf{יִמְלֹ֤ךְ יְיָ֨ לְֽעוֹלָ֗ם אֱלֹהַ֣יִךְ צִ֭יּוֹן לְדֹ֥ר וָ֝דֹ֗ר הַֽלְלוּיָֽהּ׃} \\

\shatz &
לְדוֹר וָדוֹר נַגִּיד גׇּדְלֶךָ וּלְנֵצַח נְצָחִים קְדֻשָּׁתְךָ נַקְדִּישׁ וְשִׁבְחֲךָ אֱלֹהֵֽינוּ מִפִּינוּ לֹא יָמוּשׁ לְעוֹלָם וָעֶד כִּי אֵל מֶלֶךְ גָּדוֹל וְקָדוֹשׁ אַֽתָּה׃ בָּרוּךְ אַתָּה יְיָ הָאֵל
(\instruction{בשבת שובה:} הַמֶּֽלֶךְ)
הַקָּדוֹשׁ׃
\end{longtable}
\end{footnotesize}
%\vspace{-24pt}
\firstword{יִשְׂמַח}
מֹשֶׁה בְּמַתְּנַת חֶלְקוֹ כִּי עֶבֶד נֶאֶמָן קָרָאתָ לּוֹ׃ כְּלִיל תִּפְאֶרֶת בְּרֹאשׁוֹ נָתַתָּ בְּעׇמְדוֹ לְפָנֶיךָ עַל הַר סִינַי׃ וּשְׁנֵי לֻחֹת אֲבָנִים הוֹרִיד בְּיָדוֹ וְכָתוּב בָּהֶם שְׁמִירַת שַׁבָּת וְכֵן כָּתוּב בְּתוֹרָתֶךָ׃

\veshameru

\firstword{וְלֹא נְתַתּוֹ}
יְיָ אֱלֹהֵינוּ לְגוֹיֵי הָאֲרָצוֹת וְלֹא הִנְחַלְתּוֹ מַלְכֵּֽנוּ לְעוֹבְדֵי פְסִילִים וְגַם בִּמְנוּחָתוֹ לֹא יִשְׁכְּנוּ עֲרֵלִים׃ כִּי לְיִשְׂרָאֵל עַמְּךָ נְתַתּוֹ בְּאַהֲבָה לְזֶֽרַע יַעֲקֹב אֲשֶׁר בָּם בָּחָֽרְתָּ׃ עַם מְקַדְּשֵׁי שְׁבִיעִי כֻּלָּם יִשְׂבְּעוּ וְיִתְעַנְּגוּ מִטּוּבֶֽךָ׃ וּבַשְּׁבִיעִי רָצִֽיתָ בּוֹ וְקִדַּשְׁתּוֹ חֶמְדַּת יָמִים אוֹתוֹ קָרָֽאתָ זֵֽכֶר לְמַעֲשֵׂה בְרֵאשִׁית׃

\shabboskiddushhayom{}
%\shabboskiddushhayom{

%\ifboolexpr{togl {includefestival} and togl {includeshabbat}}{%\instruction{ביו״ט ממשיכים בעמ׳ \pageref{ytshacharit}}
%	\englishinst{On Festivals, including when they fall on Shabbat, continue on page \pageref{ytshacharit}.}
%}{}
%\ifboolexpr{togl {includeshabbat}}{
%\firstword{יִשְׂמַח}
%מֹשֶׁה בְּמַתְּנַת חֶלְקוֹ כִּי עֶבֶד נֶאֶמָן קָרָאתָ לּוֹ׃ כְּלִיל תִּפְאֶרֶת בְּרֹאשׁוֹ נָתַתָּ בְּעׇמְדוֹ לְפָנֶיךָ עַל הַר סִינַי׃ וּשְׁנֵי לֻחֹת אֲבָנִים הוֹרִיד בְּיָדוֹ וְכָתוּב בָּהֶם שְׁמִירַת שַׁבָּת וְכֵן כָּתוּב בְּתוֹרָתֶךָ׃
%
%\veshameru
%
%\firstword{וְלֹא נְתַתּוֹ}
%יְיָ אֱלֹהֵינוּ לְגוֹיֵי הָאֲרָצוֹת וְלֹא הִנְחַלְתּוֹ מַלְכֵּנוּ לְעוֹבְדֵי פְסִילִים וְגַם בִּמְנוּחָתוֹ לֹא יִשְׁכְּנוּ עֲרֵלִים׃ כִּי לְיִשְׂרָאֵל עַמְּךָ נְתַתּוֹ בְּאַהֲבָה לְזֶרַע יַעֲקֹב אֲשֶׁר בָּם בָּחָֽרְתָּ׃ עַם מְקַדְּשֵׁי שְׁבִיעִי כֻּלָּם יִשְׂבְּעוּ וְיִתְעַנְּגוּ מִטּוּבֶךָ׃ וּבַשְּׁבִיעִי רָצִיתָ בּוֹ וְקִדַּשְׁתּוֹ חֶמְדַּת יָמִים אוֹתוֹ קָרָאתָ זֵֽכֶר לְמַעֲשֵׂה בְרֵאשִׁית׃
%
%%\shabboskiddushhayom{\footnote{\instruction{נ״א:} וְיָנוּחוּ בָוֹ}} \instruction{רצה וכו׳}
%\shabboskiddushhayom{}}{}
%
%\ifboolexpr{togl {includefestival} and togl {includeshabbat}}{\instruction{רצה וכו׳}
%
%\sepline
%
%\label{ytshacharit}
%
%\ytkiddushhayom
%%\vspace{16pt}
%\sepline
%
%}{}
%
%\ifboolexpr{togl {includefestival} and not togl {includeshabbat}}{\ytkiddushhayom{}}{}

\retzeh

\yaalehveyavo

\zion

\modim

\ifboolexpr{togl {includeshabbat}}{

\shabboschanukah

\shabboshodos

}{}

\shatzbirkaskohanim{בחזרת הש״ץ׃}
%\bircaskohanim{ בחזרת הש״ץ בארץ ישראל אם יש כוהנים, הם מברכים את הקהל׃}{בחזרת הש״ץ בחו״ל או בא״י עם אין כהנים׃ \vspace{.3\baselineskip}}

\shabbossimshalom

\tachanunim


\ifboolexpr{togl {includefestival} and togl {includeshabbat}}{\englishinst{Hallel is recited on Festivals, Rosh \d{H}odesh, and \d{H}anukka. On Sukkot (when not Shabbat) the Lulav is taken before Hallel.}

\ifboolexpr{togl {includefestival} or togl {includeChM}}{
\section[נטילת הלולב]{\adforn{53} נטילת הלולב \adforn{25}}
\label{lulav}

בָּרוּךְ אַתָּה יְיָ אֱלֹהֵינוּ מֶלֶךְ הָעוֹלָם אֲשֶׁר קִדְּשָׁנוּ בְּמִצְוֹתָיו וְצִוָּנוּ עַל נְטִילַת לוּלָב׃

\englishinst{When first taking Lulav, add the following blessing:}
בָּרוּךְ אַתָּה יְיָ אֱלֹהֵינוּ מֶלֶךְ הָעוֹלָם שֶׁהֶחֱיָנוּ וְקִיְּמָנוּ וְהִגִּיעָנוּ לַזְמַן הַזֶּה׃
}{}

\newcommand{\diluginst}{\englishinst{}}
\ifboolexpr{(togl {includeweekday} or togl {includeshabbat}) and not togl {includefestival} and not togl {includeChM}}{
	\renewcommand{\diluginst}{\englishinst{This section is skipped on Rosh \d{H}odesh (except on \d{H}anukka).}}
	\chapter[הלל‎]{\adforn{53} הלל‎ \adforn{25}}
}{}
\ifboolexpr{(togl {includefestival} or togl {includeChM}) and (togl {includeweekday} or togl {includeshabbat})}{
	\renewcommand{\diluginst}{\englishinst{This section is skipped on Rosh \d{H}odesh (except on \d{H}anukka) and the intermediate and final days of Passover.}}
	
\chapter[הלל‎]{\adforn{53} הלל‎ \adforn{25}}
}{}
\ifboolexpr{(togl {includefestival} or togl {includeChM}) and not togl {includeweekday} and not togl {includeshabbat}}{
	\renewcommand{\diluginst}{\englishinst{This section is skipped on the intermediate and final days of Passover.}}
	
	\section[הלל‎]{\adforn{53} הלל‎ \adforn{25}}
}{}

\label{hallel}

%\instruction{החזן אומר הברכה בקול רם, הקהל אומר אמן ואחר כך חוזרים ומברכים:}\\
\firstword{בָּרוּךְ}
אַתָּה יְיָ אֱלֹהֵֽינוּ מֶֽלֶךְ הָעוֹלָם אֲשֶׁר קִדְּשָֽׁנוּ בְּמִצְוֹתָיו וְצִוָּֽנוּ לִקְרֹא אֶת־הַהַלֵּל׃

\firstword{הַ֥לְלוּ יָ֨הּ}\source{תהלים קיג}
׀ הַ֭לְלוּ עַבְדֵ֣י יְיָ֑ הַֽ֝לְל֗וּ אֶת־שֵׁ֥ם יְיָ׃
יְהִ֤י שֵׁ֣ם יְיָ֣ מְבֹרָ֑ךְ מֵ֝עַתָּ֗ה וְעַד־עוֹלָֽם׃
מִמִּזְרַח־שֶׁ֥מֶשׁ עַד־מְבוֹא֑וֹ מְ֝הֻלָּ֗ל שֵׁ֣ם יְיָ׃
רָ֖ם עַל־כׇּל־גּוֹיִ֥ם ׀ יְיָ֑ עַ֖ל הַשָּׁמַ֣יִם כְּבוֹדֽוֹ׃
מִ֭י כַּייָ֣ אֱלֹהֵ֑ינוּ הַֽמַּגְבִּיהִ֥י לָשָֽׁבֶת׃
הַֽמַּשְׁפִּילִ֥י לִרְא֑וֹת בַּשָּׁמַ֥יִם וּבָאָֽרֶץ׃
מְקִ֥ימִ֣י מֵעָפָ֣ר דָּ֑ל מֵ֝אַשְׁפֹּ֗ת יָרִ֥ים אֶבְיֽוֹן׃
לְהוֹשִׁיבִ֥י עִם־נְדִיבִ֑ים עִ֗֝ם נְדִיבֵ֥י עַמּֽוֹ׃
מֽוֹשִׁיבִ֨י ׀ עֲקֶ֬רֶת הַבַּ֗יִת אֵֽם־הַבָּנִ֥ים שְׂמֵחָ֗ה הַֽלְלוּ־יָֽהּ׃

\firstword{בְּצֵ֣את יִ֭שְׂרָאֵל}\source{תהלים קיד}
מִמִּצְרָ֑יִם בֵּ֥ית יַ֝עֲקֹ֗ב מֵעַ֥ם לֹעֵֽז׃
הָיְתָ֣ה יְהוּדָ֣ה לְקׇדְשׁ֑וֹ יִ֝שְׂרָאֵ֗ל מַמְשְׁלוֹתָֽיו׃
הַיָּ֣ם רָ֭אָה וַיָּנֹ֑ס הַ֝יַּרְדֵּ֗ן יִסֹּ֥ב לְאָחֽוֹר׃
הֶ֭הָרִים רָקְד֣וּ כְאֵילִ֑ים גְּ֝בָע֗וֹת כִּבְנֵי־צֹֽאן׃
מַה־לְּךָ֣ הַ֭יָּם כִּ֣י תָנ֑וּס הַ֝יַּרְדֵּ֗ן תִּסֹּ֥ב לְאָחֽוֹר׃
הֶ֭הָרִים תִּרְקְד֣וּ כְאֵילִ֑ים גְּ֝בָע֗וֹת כִּבְנֵי־צֹֽאן׃
מִלִּפְנֵ֣י אָ֭דוֹן ח֣וּלִי אָ֑רֶץ מִ֝לִּפְנֵ֗י אֱל֣וֹהַּ יַעֲקֹֽב׃
הַהֹפְכִ֣י הַצּ֣וּר אֲגַם־מָ֑יִם חַ֝לָּמִ֗ישׁ לְמַעְיְנוֹ־מָֽיִם׃

\diluginst
\begin{narrow}\vspace{-12pt}
	\firstword{לֹ֤א לָ֥נוּ}\source{תהלים קטו}
יְיָ֗ לֹ֫א לָ֥נוּ כִּֽי־לְ֭שִׁמְךָ תֵּ֣ן כָּב֑וֹד עַל־חַ֝סְדְּךָ֗ עַל־אֲמִתֶּֽךָ׃
לָ֭מָּה יֹאמְר֣וּ הַגּוֹיִ֑ם אַיֵּה־נָ֗֝א אֱלֹהֵיהֶֽם׃
וֵאלֹהֵ֥ינוּ בַשָּׁמָ֑יִם כֹּ֖ל אֲשֶׁר־חָפֵ֣ץ עָשָֽׂה׃
עֲֽ֭צַבֵּיהֶם כֶּ֣סֶף וְזָהָ֑ב מַ֝עֲשֵׂ֗ה יְדֵ֣י אָדָֽם׃
פֶּֽה־לָ֭הֶם וְלֹ֣א יְדַבֵּ֑רוּ עֵינַ֥יִם לָ֝הֶ֗ם וְלֹ֣א יִרְאֽוּ׃
אׇזְנַ֣יִם לָ֭הֶם וְלֹ֣א יִשְׁמָ֑עוּ אַ֥ף לָ֝הֶ֗ם וְלֹ֣א יְרִיחֽוּן׃
יְדֵיהֶ֤ם ׀ וְלֹ֬א יְמִישׁ֗וּן רַ֭גְלֵיהֶם וְלֹ֣א יְהַלֵּ֑כוּ לֹא־יֶ֝הְגּ֗וּ בִּגְרוֹנָֽם׃
כְּ֭מוֹהֶם יִהְי֣וּ עֹשֵׂיהֶ֑ם כֹּ֖ל אֲשֶׁר־בֹּטֵ֣חַ בָּהֶֽם׃
יִ֭שְׂרָאֵל בְּטַ֣ח בַּייָ֑ עֶזְרָ֖ם וּמָגִנָּ֣ם הֽוּא׃
בֵּ֣ית אַ֭הֲרֹן בִּטְח֣וּ בַייָ֑ עֶזְרָ֖ם וּמָגִנָּ֣ם הֽוּא׃
יִרְאֵ֣י יְיָ֭ בִּטְח֣וּ בַייָ֑ עֶזְרָ֖ם וּמָגִנָּ֣ם הֽוּא׃
\end{narrow}
\firstword{יְיָ זְכָרָ֢נוּ}
יְ֭בָרֵךְ אֶת־בֵּ֣ית יִשְׂרָאֵ֑ל יְ֝בָרֵ֗ךְ אֶת־בֵּ֥ית אַהֲרֹֽן׃
יְ֭בָרֵךְ יִרְאֵ֣י יְיָ֑ הַ֝קְּטַנִּ֗ים עִם־הַגְּדֹלִֽים׃
יֹסֵ֣ף יְיָ֣ עֲלֵיכֶ֑ם עֲ֝לֵיכֶ֗ם וְעַל־בְּנֵיכֶֽם׃
בְּרוּכִ֣ים אַ֭תֶּם לַייָ֑ עֹ֝שֵׂ֗ה שָׁמַ֥יִם וָאָֽרֶץ׃
הַשָּׁמַ֣יִם שָׁ֭מַיִם לַייָ֑ וְ֝הָאָ֗רֶץ נָתַ֥ן לִבְנֵי־אָדָֽם׃
לֹ֣א הַ֭מֵּתִים יְהַֽלְלוּ־יָ֑הּ וְ֝לֹ֗א כׇּל־יֹרְדֵ֥י דוּמָֽה׃
וַאֲנַ֤חְנוּ ׀ נְבָ֘רֵ֤ךְ יָ֗הּ מֵעַתָּ֥ה וְעַד־עוֹלָ֗ם הַֽלְלוּ־יָֽהּ׃

\diluginst
\begin{narrow}\vspace{-12pt}
	\firstword{אָ֭הַבְתִּי}\source{תהלים קטז}
כִּי־יִשְׁמַ֥ע ׀ יְיָ֑ אֶת־ק֝וֹלִ֗י תַּחֲנוּנָֽי׃
כִּי־הִטָּ֣ה אׇזְנ֣וֹ לִ֑י וּבְיָמַ֥י אֶקְרָֽא׃
אֲפָפ֤וּנִי ׀ חֶבְלֵי־מָ֗וֶת וּמְצָרֵ֣י שְׁא֣וֹל מְצָא֑וּנִי צָרָ֖ה וְיָג֣וֹן אֶמְצָֽא׃
וּבְשֵֽׁם־יְיָ֥ אֶקְרָ֑א אָנָּ֥ה יְ֝יָ֗ מַלְּטָ֥ה נַפְשִֽׁי׃
חַנּ֣וּן יְיָ֣ וְצַדִּ֑יק וֵ֖אלֹהֵ֣ינוּ מְרַחֵֽם׃
שֹׁמֵ֣ר פְּתָאיִ֣ם יְיָ֑ דַּ֝לֹּתִ֗י וְלִ֣י יְהוֹשִֽׁיעַ׃
שׁוּבִ֣י נַ֭פְשִׁי לִמְנוּחָ֑יְכִי כִּֽי־יְ֝יָ֗ גָּמַ֥ל עָלָֽיְכִי׃
כִּ֤י חִלַּ֥צְתָּ נַפְשִׁ֗י מִ֫מָּ֥וֶת אֶת־עֵינִ֥י מִן־דִּמְעָ֑ה אֶת־רַגְלִ֥י מִדֶּֽחִי׃
אֶ֭תְהַלֵּךְ לִפְנֵ֣י יְיָ֑ בְּ֝אַרְצ֗וֹת הַחַיִּֽים׃
הֶ֭אֱמַנְתִּי כִּ֣י אֲדַבֵּ֑ר אֲ֝נִ֗י עָנִ֥יתִי מְאֹֽד׃
אֲ֭נִי אָמַ֣רְתִּי בְחׇפְזִ֑י כׇּֽל־הָאָדָ֥ם כֹּזֵֽב׃
\end{narrow}

\firstword{מָה־אָשִׁ֥יב לַייָ֑}
כׇּֽל־תַּגְמוּל֥וֹהִי עָלָֽי׃
כּוֹס־יְשׁוּע֥וֹת אֶשָּׂ֑א וּבְשֵׁ֖ם יְיָ֣ אֶקְרָֽא׃
נְ֭דָרַי לַייָ֣ אֲשַׁלֵּ֑ם נֶגְדָה־נָּ֗֝א לְכׇל־עַמּֽוֹ׃
יָ֭קָר בְּעֵינֵ֣י יְיָ֑ הַ֝מָּ֗וְתָה לַחֲסִידָֽיו׃
אָנָּ֣ה יְיָ כִּֽי־אֲנִ֢י עַ֫בְדֶּ֥ךָ אֲנִי־עַ֭בְדְּךָ בֶּן־אֲמָתֶ֑ךָ פִּ֝תַּ֗חְתָּ לְמֽוֹסֵרָֽי׃
לְֽךָ־אֶ֭זְבַּח זֶ֣בַח תּוֹדָ֑ה וּבְשֵׁ֖ם יְיָ֣ אֶקְרָֽא׃
נְ֭דָרַי לַייָ֣ אֲשַׁלֵּ֑ם נֶגְדָה־נָּ֗֝א לְכׇל־עַמּֽוֹ׃
בְּחַצְר֤וֹת ׀ בֵּ֤ית יְיָ֗ בְּֽת֘וֹכֵ֤כִי יְֽרוּשָׁלָ֗‍ִם הַֽלְלוּ־יָֽהּ׃

\firstword{הַֽלְל֣וּ}\source{תהלים קיז}
אֶת־יְיָ֭ כׇּל־גּוֹיִ֑ם שַׁ֝בְּח֗וּהוּ כׇּל־הָאֻמִּֽים׃
כִּ֥י גָ֘בַ֤ר עָלֵ֨ינוּ ׀ חַסְדּ֗וֹ וֶאֱמֶת־יְיָ֥ לְעוֹלָ֗ם הַֽלְלוּ־יָֽהּ׃

\shatz \source{תהלים קיח}הוֹד֣וּ לַייָ֣ כִּי־ט֑וֹב כִּ֖י לְעוֹלָ֣ם חַסְדּֽוֹ׃ \hfill \break
\kahal \begin{small}הוֹד֣וּ לַייָ֣ כִּי־ט֑וֹב כִּ֖י לְעוֹלָ֣ם חַסְדּֽוֹ׃ \end{small}\\
\shatz יֹאמַר־נָ֥א יִשְׂרָאֵ֑ל כִּ֖י לְעוֹלָ֣ם חַסְדּֽוֹ׃\hfill \break
\kahal \begin{small}הוֹד֣וּ לַייָ֣ כִּי־ט֑וֹב כִּ֖י לְעוֹלָ֣ם חַסְדּֽוֹ׃ \end{small}\\
\shatz יֹאמְרוּ־נָ֥א בֵֽית־אַהֲרֹ֑ן כִּ֖י לְעוֹלָ֣ם חַסְדּֽוֹ׃ \hfill \break
\kahal \begin{small}הוֹד֣וּ לַייָ֣ כִּי־ט֑וֹב כִּ֖י לְעוֹלָ֣ם חַסְדּֽוֹ׃ \end{small}\\
\shatz יֹאמְרוּ־נָ֭א יִרְאֵ֣י יְיָ֑ כִּ֖י לְעוֹלָ֣ם חַסְדּֽוֹ׃\hfill \break
\kahal \begin{small}הוֹד֣וּ לַייָ֣ כִּי־ט֑וֹב כִּ֖י לְעוֹלָ֣ם חַסְדּֽוֹ׃ \end{small}\\
\firstword{מִֽן־הַ֭מֵּצַר}\source{תהלים קיח}
קָרָ֣אתִי יָּ֑הּ עָנָ֖נִי בַמֶּרְחָ֣ב יָֽהּ׃
יְיָ֣ לִ֭י לֹ֣א אִירָ֑א מַה־יַּעֲשֶׂ֖ה לִ֣י אָדָֽם׃
יְיָ֣ לִ֭י בְּעֹזְרָ֑י וַ֝אֲנִ֗י אֶרְאֶ֥ה בְשֹׂנְאָֽי׃
ט֗וֹב לַחֲס֥וֹת בַּייָ֑ מִ֝בְּטֹ֗חַ בָּאָדָֽם׃
ט֗וֹב לַחֲס֥וֹת בַּייָ֑ מִ֝בְּטֹ֗חַ בִּנְדִיבִֽים׃
כׇּל־גּוֹיִ֥ם סְבָב֑וּנִי בְּשֵׁ֥ם יְ֝יָ֗ כִּ֣י אֲמִילַֽם׃
סַבּ֥וּנִי גַם־סְבָב֑וּנִי בְּשֵׁ֥ם יְ֝יָ֗ כִּ֣י אֲמִילַֽם׃
סַבּ֤וּנִי כִדְבוֹרִ֗ים דֹּ֭עֲכוּ כְּאֵ֣שׁ קוֹצִ֑ים בְּשֵׁ֥ם יְ֝יָ֗ כִּ֣י אֲמִילַֽם׃
דַּחֹ֣ה דְחִיתַ֣נִי לִנְפֹּ֑ל וַ֖ייָ֣ עֲזָרָֽנִי׃
עָזִּ֣י וְזִמְרָ֣ת יָ֑הּ וַֽיְהִי־לִ֗֝י לִישׁוּעָֽה׃
ק֤וֹל ׀ רִנָּ֬ה וִישׁוּעָ֗ה בְּאׇהֳלֵ֥י צַדִּיקִ֑ים יְמִ֥ין יְ֝יָ֗ עֹ֣שָׂה חָֽיִל׃
יְמִ֣ין יְיָ֭ רוֹמֵמָ֑ה יְמִ֥ין יְ֝יָ֗ עֹ֣שָׂה חָֽיִל׃
לֹא־אָמ֥וּת כִּֽי־אֶחְיֶ֑ה וַ֝אֲסַפֵּ֗ר מַעֲשֵׂ֥י יָֽהּ׃
יַסֹּ֣ר יִסְּרַ֣נִּי יָּ֑הּ וְ֝לַמָּ֗וֶת לֹ֣א נְתָנָֽנִי׃
פִּתְחוּ־לִ֥י שַׁעֲרֵי־צֶ֑דֶק אָבֹא־בָ֗֝ם אוֹדֶ֥ה יָֽהּ׃
זֶה־הַשַּׁ֥עַר לַייָ֑ צַ֝דִּיקִ֗ים יָבֹ֥אוּ בֽוֹ׃\\
א֭וֹדְךָ כִּ֣י עֲנִיתָ֑נִי וַתְּהִי־לִ֗֝י לִישׁוּעָֽה׃ \\
\scriptsize{ א֭וֹדְךָ כִּ֣י עֲנִיתָ֑נִי וַתְּהִי־לִ֗֝י לִישׁוּעָֽה׃ \\}\normalsize{}
אֶ֭בֶן מָאֲס֣וּ הַבּוֹנִ֑ים הָ֝יְתָ֗ה לְרֹ֣אשׁ פִּנָּֽה׃ \\
\scriptsize{ אֶ֭בֶן מָאֲס֣וּ הַבּוֹנִ֑ים הָ֝יְתָ֗ה לְרֹ֣אשׁ פִּנָּֽה׃ \\}\normalsize{}
מֵאֵ֣ת יְיָ֭ הָ֣יְתָה זֹּ֑את הִ֖יא נִפְלָ֣את בְּעֵינֵֽינוּ׃ \\
\scriptsize{ מֵאֵ֣ת יְיָ֭ הָ֣יְתָה זֹּ֑את הִ֖יא נִפְלָ֣את בְּעֵינֵֽינוּ׃ \\}\normalsize{}
זֶה־הַ֭יּוֹם עָשָׂ֣ה יְיָ֑ נָגִ֖ילָה וְנִשְׂמְחָ֣ה בֽוֹ׃ \\
\scriptsize{ זֶה־הַ֭יּוֹם עָשָׂ֣ה יְיָ֑ נָגִ֖ילָה וְנִשְׂמְחָ֣ה בֽוֹ׃ } \normalsize{}


\instruction{ש״ץ ואח״כ הקהל׃}\\
אָנָּ֣א יְיָ֭ הוֹשִׁ֘יעָ֥ה נָּ֑א \hfill אָנָּ֣א יְיָ֭ הוֹשִׁ֘יעָ֥ה נָּ֑א\\
אָנָּ֥א יְ֝יָ֗ הַצְלִ֘יחָ֥ה נָּֽא׃ \hfill אָנָּ֥א יְ֝יָ֗ הַצְלִ֘יחָ֥ה נָּֽא׃\\
בָּר֣וּךְ הַ֭בָּא בְּשֵׁ֣ם יְיָ֑ בֵּ֝רַ֥כְנוּכֶ֗ם מִבֵּ֥ית יְיָ׃\\
\scriptsize{בָּר֣וּךְ הַ֭בָּא בְּשֵׁ֣ם יְיָ֑ בֵּ֝רַ֥כְנוּכֶ֗ם מִבֵּ֥ית יְיָ׃}\\
\normalsize{אֵ֤ל ׀ יְיָ וַיָּ֢אֶ֫ר לָ֥נוּ אִסְרוּ־חַ֥ג בַּעֲבֹתִ֑ים עַד־קַ֝רְנ֗וֹת הַמִּזְבֵּֽחַ׃}\\
\scriptsize{אֵ֤ל ׀ יְיָ וַיָּ֢אֶ֫ר לָ֥נוּ אִסְרוּ־חַ֥ג בַּעֲבֹתִ֑ים עַד־קַ֝רְנ֗וֹת הַמִּזְבֵּֽחַ׃}\\
\normalsize{אֵלִ֣י אַתָּ֣ה וְאוֹדֶ֑ךָּ אֱ֝לֹהַ֗י אֲרוֹמְמֶֽךָּ׃}\\
\scriptsize{אֵלִ֣י אַתָּ֣ה וְאוֹדֶ֑ךָּ אֱ֝לֹהַ֗י אֲרוֹמְמֶֽךָּ׃}\\
\normalsize{הוֹד֣וּ לַייָ֣ כִּי־ט֑וֹב כִּ֖י לְעוֹלָ֣ם חַסְדּֽוֹ׃}\\
\scriptsize{הוֹד֣וּ לַייָ֣ כִּי־ט֑וֹב כִּ֖י לְעוֹלָ֣ם חַסְדּֽוֹ׃} \\
\normalsize{}



\negline

\firstword{יְהַלְלֽוּךָ}
יְיָ אֱלֹהֵֽינוּ כׇּל־מַעֲשֶֽׂיךָ וַחֲסִידֶֽיךָ צַדִּיקִים עוֹשֵׂי רְצֹנֶֽךָ וְכׇל־עַמְּךָ בֵּית יִשְׂרָאֵל בְּרִנָּה יוֹדוּ וִיבָרְכוּ וִישַׁבְּחוּ וִיפָאֲרוּ וִירוֹמֲמוּ וְיַעֲרִֽיצוּ וְיַקְדִּֽישׁוּ וְיַמְלִֽיכוּ אֶת־שִׁמְךָ מַלְכֵּֽנוּ כִּי לְךָ טוֹב לְהוֹדוֹת וּלְשִׁמְךָ נָאֶה לְזַמֵּר כִּי מֵעוֹלָם וְעַד עוֹלָם אַתָּה אֵל׃ בָּרוּךְ אַתָּה יְיָ מֶֽלֶךְ מְהֻלָּל בַּתִּשְׁבָּחוֹת׃\\

%\ifboolexpr{togl {includeshabbat} and togl {includeweekday} and not togl {includeChM}}{\englishinst{On Shabbat, continue with Full Kaddish on page \pageref{shacharitShabbatYTtitkabel}. On weekday Rosh \d{H}odesh, continue with Full Kaddish on page \pageref{end of shacharis}, followed by the Psalm of the Day as relevant on page \pageref{shir_shel_yom}. On \d{H}anukka that is not Rosh \d{H}odesh, continue with Half Kaddish on page \pageref{hatzi_kaddish}.}}{}

\englishinst{On Shabbat, Festivals, and Rosh \d{H}odesh, recite Full Kaddish followed by the Psalm of the Day. Then Mourner's Kaddish is read, followed by the Torah service on page \pageref{shabYTtorah} for Shabbat and Festivals and page \pageref{weekday torah} for Intermediate Festival Days and Rosh \d{H}odesh.}


}{}
\ifboolexpr{togl {includeshabbat} and not togl {includeweekday}}{\englishinst{On \d{H}anukka and Rosh \d{H}odesh Hallel is recited.  On other days, continue with Full Kaddish on page \pageref{shacharitShabbatYTtitkabel}.}
	
\ifboolexpr{togl {includefestival} or togl {includeChM}}{
\section[נטילת הלולב]{\adforn{53} נטילת הלולב \adforn{25}}
\label{lulav}

בָּרוּךְ אַתָּה יְיָ אֱלֹהֵינוּ מֶלֶךְ הָעוֹלָם אֲשֶׁר קִדְּשָׁנוּ בְּמִצְוֹתָיו וְצִוָּנוּ עַל נְטִילַת לוּלָב׃

\englishinst{When first taking Lulav, add the following blessing:}
בָּרוּךְ אַתָּה יְיָ אֱלֹהֵינוּ מֶלֶךְ הָעוֹלָם שֶׁהֶחֱיָנוּ וְקִיְּמָנוּ וְהִגִּיעָנוּ לַזְמַן הַזֶּה׃
}{}

\newcommand{\diluginst}{\englishinst{}}
\ifboolexpr{(togl {includeweekday} or togl {includeshabbat}) and not togl {includefestival} and not togl {includeChM}}{
	\renewcommand{\diluginst}{\englishinst{This section is skipped on Rosh \d{H}odesh (except on \d{H}anukka).}}
	\chapter[הלל‎]{\adforn{53} הלל‎ \adforn{25}}
}{}
\ifboolexpr{(togl {includefestival} or togl {includeChM}) and (togl {includeweekday} or togl {includeshabbat})}{
	\renewcommand{\diluginst}{\englishinst{This section is skipped on Rosh \d{H}odesh (except on \d{H}anukka) and the intermediate and final days of Passover.}}
	
\chapter[הלל‎]{\adforn{53} הלל‎ \adforn{25}}
}{}
\ifboolexpr{(togl {includefestival} or togl {includeChM}) and not togl {includeweekday} and not togl {includeshabbat}}{
	\renewcommand{\diluginst}{\englishinst{This section is skipped on the intermediate and final days of Passover.}}
	
	\section[הלל‎]{\adforn{53} הלל‎ \adforn{25}}
}{}

\label{hallel}

%\instruction{החזן אומר הברכה בקול רם, הקהל אומר אמן ואחר כך חוזרים ומברכים:}\\
\firstword{בָּרוּךְ}
אַתָּה יְיָ אֱלֹהֵֽינוּ מֶֽלֶךְ הָעוֹלָם אֲשֶׁר קִדְּשָֽׁנוּ בְּמִצְוֹתָיו וְצִוָּֽנוּ לִקְרֹא אֶת־הַהַלֵּל׃

\firstword{הַ֥לְלוּ יָ֨הּ}\source{תהלים קיג}
׀ הַ֭לְלוּ עַבְדֵ֣י יְיָ֑ הַֽ֝לְל֗וּ אֶת־שֵׁ֥ם יְיָ׃
יְהִ֤י שֵׁ֣ם יְיָ֣ מְבֹרָ֑ךְ מֵ֝עַתָּ֗ה וְעַד־עוֹלָֽם׃
מִמִּזְרַח־שֶׁ֥מֶשׁ עַד־מְבוֹא֑וֹ מְ֝הֻלָּ֗ל שֵׁ֣ם יְיָ׃
רָ֖ם עַל־כׇּל־גּוֹיִ֥ם ׀ יְיָ֑ עַ֖ל הַשָּׁמַ֣יִם כְּבוֹדֽוֹ׃
מִ֭י כַּייָ֣ אֱלֹהֵ֑ינוּ הַֽמַּגְבִּיהִ֥י לָשָֽׁבֶת׃
הַֽמַּשְׁפִּילִ֥י לִרְא֑וֹת בַּשָּׁמַ֥יִם וּבָאָֽרֶץ׃
מְקִ֥ימִ֣י מֵעָפָ֣ר דָּ֑ל מֵ֝אַשְׁפֹּ֗ת יָרִ֥ים אֶבְיֽוֹן׃
לְהוֹשִׁיבִ֥י עִם־נְדִיבִ֑ים עִ֗֝ם נְדִיבֵ֥י עַמּֽוֹ׃
מֽוֹשִׁיבִ֨י ׀ עֲקֶ֬רֶת הַבַּ֗יִת אֵֽם־הַבָּנִ֥ים שְׂמֵחָ֗ה הַֽלְלוּ־יָֽהּ׃

\firstword{בְּצֵ֣את יִ֭שְׂרָאֵל}\source{תהלים קיד}
מִמִּצְרָ֑יִם בֵּ֥ית יַ֝עֲקֹ֗ב מֵעַ֥ם לֹעֵֽז׃
הָיְתָ֣ה יְהוּדָ֣ה לְקׇדְשׁ֑וֹ יִ֝שְׂרָאֵ֗ל מַמְשְׁלוֹתָֽיו׃
הַיָּ֣ם רָ֭אָה וַיָּנֹ֑ס הַ֝יַּרְדֵּ֗ן יִסֹּ֥ב לְאָחֽוֹר׃
הֶ֭הָרִים רָקְד֣וּ כְאֵילִ֑ים גְּ֝בָע֗וֹת כִּבְנֵי־צֹֽאן׃
מַה־לְּךָ֣ הַ֭יָּם כִּ֣י תָנ֑וּס הַ֝יַּרְדֵּ֗ן תִּסֹּ֥ב לְאָחֽוֹר׃
הֶ֭הָרִים תִּרְקְד֣וּ כְאֵילִ֑ים גְּ֝בָע֗וֹת כִּבְנֵי־צֹֽאן׃
מִלִּפְנֵ֣י אָ֭דוֹן ח֣וּלִי אָ֑רֶץ מִ֝לִּפְנֵ֗י אֱל֣וֹהַּ יַעֲקֹֽב׃
הַהֹפְכִ֣י הַצּ֣וּר אֲגַם־מָ֑יִם חַ֝לָּמִ֗ישׁ לְמַעְיְנוֹ־מָֽיִם׃

\diluginst
\begin{narrow}\vspace{-12pt}
	\firstword{לֹ֤א לָ֥נוּ}\source{תהלים קטו}
יְיָ֗ לֹ֫א לָ֥נוּ כִּֽי־לְ֭שִׁמְךָ תֵּ֣ן כָּב֑וֹד עַל־חַ֝סְדְּךָ֗ עַל־אֲמִתֶּֽךָ׃
לָ֭מָּה יֹאמְר֣וּ הַגּוֹיִ֑ם אַיֵּה־נָ֗֝א אֱלֹהֵיהֶֽם׃
וֵאלֹהֵ֥ינוּ בַשָּׁמָ֑יִם כֹּ֖ל אֲשֶׁר־חָפֵ֣ץ עָשָֽׂה׃
עֲֽ֭צַבֵּיהֶם כֶּ֣סֶף וְזָהָ֑ב מַ֝עֲשֵׂ֗ה יְדֵ֣י אָדָֽם׃
פֶּֽה־לָ֭הֶם וְלֹ֣א יְדַבֵּ֑רוּ עֵינַ֥יִם לָ֝הֶ֗ם וְלֹ֣א יִרְאֽוּ׃
אׇזְנַ֣יִם לָ֭הֶם וְלֹ֣א יִשְׁמָ֑עוּ אַ֥ף לָ֝הֶ֗ם וְלֹ֣א יְרִיחֽוּן׃
יְדֵיהֶ֤ם ׀ וְלֹ֬א יְמִישׁ֗וּן רַ֭גְלֵיהֶם וְלֹ֣א יְהַלֵּ֑כוּ לֹא־יֶ֝הְגּ֗וּ בִּגְרוֹנָֽם׃
כְּ֭מוֹהֶם יִהְי֣וּ עֹשֵׂיהֶ֑ם כֹּ֖ל אֲשֶׁר־בֹּטֵ֣חַ בָּהֶֽם׃
יִ֭שְׂרָאֵל בְּטַ֣ח בַּייָ֑ עֶזְרָ֖ם וּמָגִנָּ֣ם הֽוּא׃
בֵּ֣ית אַ֭הֲרֹן בִּטְח֣וּ בַייָ֑ עֶזְרָ֖ם וּמָגִנָּ֣ם הֽוּא׃
יִרְאֵ֣י יְיָ֭ בִּטְח֣וּ בַייָ֑ עֶזְרָ֖ם וּמָגִנָּ֣ם הֽוּא׃
\end{narrow}
\firstword{יְיָ זְכָרָ֢נוּ}
יְ֭בָרֵךְ אֶת־בֵּ֣ית יִשְׂרָאֵ֑ל יְ֝בָרֵ֗ךְ אֶת־בֵּ֥ית אַהֲרֹֽן׃
יְ֭בָרֵךְ יִרְאֵ֣י יְיָ֑ הַ֝קְּטַנִּ֗ים עִם־הַגְּדֹלִֽים׃
יֹסֵ֣ף יְיָ֣ עֲלֵיכֶ֑ם עֲ֝לֵיכֶ֗ם וְעַל־בְּנֵיכֶֽם׃
בְּרוּכִ֣ים אַ֭תֶּם לַייָ֑ עֹ֝שֵׂ֗ה שָׁמַ֥יִם וָאָֽרֶץ׃
הַשָּׁמַ֣יִם שָׁ֭מַיִם לַייָ֑ וְ֝הָאָ֗רֶץ נָתַ֥ן לִבְנֵי־אָדָֽם׃
לֹ֣א הַ֭מֵּתִים יְהַֽלְלוּ־יָ֑הּ וְ֝לֹ֗א כׇּל־יֹרְדֵ֥י דוּמָֽה׃
וַאֲנַ֤חְנוּ ׀ נְבָ֘רֵ֤ךְ יָ֗הּ מֵעַתָּ֥ה וְעַד־עוֹלָ֗ם הַֽלְלוּ־יָֽהּ׃

\diluginst
\begin{narrow}\vspace{-12pt}
	\firstword{אָ֭הַבְתִּי}\source{תהלים קטז}
כִּי־יִשְׁמַ֥ע ׀ יְיָ֑ אֶת־ק֝וֹלִ֗י תַּחֲנוּנָֽי׃
כִּי־הִטָּ֣ה אׇזְנ֣וֹ לִ֑י וּבְיָמַ֥י אֶקְרָֽא׃
אֲפָפ֤וּנִי ׀ חֶבְלֵי־מָ֗וֶת וּמְצָרֵ֣י שְׁא֣וֹל מְצָא֑וּנִי צָרָ֖ה וְיָג֣וֹן אֶמְצָֽא׃
וּבְשֵֽׁם־יְיָ֥ אֶקְרָ֑א אָנָּ֥ה יְ֝יָ֗ מַלְּטָ֥ה נַפְשִֽׁי׃
חַנּ֣וּן יְיָ֣ וְצַדִּ֑יק וֵ֖אלֹהֵ֣ינוּ מְרַחֵֽם׃
שֹׁמֵ֣ר פְּתָאיִ֣ם יְיָ֑ דַּ֝לֹּתִ֗י וְלִ֣י יְהוֹשִֽׁיעַ׃
שׁוּבִ֣י נַ֭פְשִׁי לִמְנוּחָ֑יְכִי כִּֽי־יְ֝יָ֗ גָּמַ֥ל עָלָֽיְכִי׃
כִּ֤י חִלַּ֥צְתָּ נַפְשִׁ֗י מִ֫מָּ֥וֶת אֶת־עֵינִ֥י מִן־דִּמְעָ֑ה אֶת־רַגְלִ֥י מִדֶּֽחִי׃
אֶ֭תְהַלֵּךְ לִפְנֵ֣י יְיָ֑ בְּ֝אַרְצ֗וֹת הַחַיִּֽים׃
הֶ֭אֱמַנְתִּי כִּ֣י אֲדַבֵּ֑ר אֲ֝נִ֗י עָנִ֥יתִי מְאֹֽד׃
אֲ֭נִי אָמַ֣רְתִּי בְחׇפְזִ֑י כׇּֽל־הָאָדָ֥ם כֹּזֵֽב׃
\end{narrow}

\firstword{מָה־אָשִׁ֥יב לַייָ֑}
כׇּֽל־תַּגְמוּל֥וֹהִי עָלָֽי׃
כּוֹס־יְשׁוּע֥וֹת אֶשָּׂ֑א וּבְשֵׁ֖ם יְיָ֣ אֶקְרָֽא׃
נְ֭דָרַי לַייָ֣ אֲשַׁלֵּ֑ם נֶגְדָה־נָּ֗֝א לְכׇל־עַמּֽוֹ׃
יָ֭קָר בְּעֵינֵ֣י יְיָ֑ הַ֝מָּ֗וְתָה לַחֲסִידָֽיו׃
אָנָּ֣ה יְיָ כִּֽי־אֲנִ֢י עַ֫בְדֶּ֥ךָ אֲנִי־עַ֭בְדְּךָ בֶּן־אֲמָתֶ֑ךָ פִּ֝תַּ֗חְתָּ לְמֽוֹסֵרָֽי׃
לְֽךָ־אֶ֭זְבַּח זֶ֣בַח תּוֹדָ֑ה וּבְשֵׁ֖ם יְיָ֣ אֶקְרָֽא׃
נְ֭דָרַי לַייָ֣ אֲשַׁלֵּ֑ם נֶגְדָה־נָּ֗֝א לְכׇל־עַמּֽוֹ׃
בְּחַצְר֤וֹת ׀ בֵּ֤ית יְיָ֗ בְּֽת֘וֹכֵ֤כִי יְֽרוּשָׁלָ֗‍ִם הַֽלְלוּ־יָֽהּ׃

\firstword{הַֽלְל֣וּ}\source{תהלים קיז}
אֶת־יְיָ֭ כׇּל־גּוֹיִ֑ם שַׁ֝בְּח֗וּהוּ כׇּל־הָאֻמִּֽים׃
כִּ֥י גָ֘בַ֤ר עָלֵ֨ינוּ ׀ חַסְדּ֗וֹ וֶאֱמֶת־יְיָ֥ לְעוֹלָ֗ם הַֽלְלוּ־יָֽהּ׃

\shatz \source{תהלים קיח}הוֹד֣וּ לַייָ֣ כִּי־ט֑וֹב כִּ֖י לְעוֹלָ֣ם חַסְדּֽוֹ׃ \hfill \break
\kahal \begin{small}הוֹד֣וּ לַייָ֣ כִּי־ט֑וֹב כִּ֖י לְעוֹלָ֣ם חַסְדּֽוֹ׃ \end{small}\\
\shatz יֹאמַר־נָ֥א יִשְׂרָאֵ֑ל כִּ֖י לְעוֹלָ֣ם חַסְדּֽוֹ׃\hfill \break
\kahal \begin{small}הוֹד֣וּ לַייָ֣ כִּי־ט֑וֹב כִּ֖י לְעוֹלָ֣ם חַסְדּֽוֹ׃ \end{small}\\
\shatz יֹאמְרוּ־נָ֥א בֵֽית־אַהֲרֹ֑ן כִּ֖י לְעוֹלָ֣ם חַסְדּֽוֹ׃ \hfill \break
\kahal \begin{small}הוֹד֣וּ לַייָ֣ כִּי־ט֑וֹב כִּ֖י לְעוֹלָ֣ם חַסְדּֽוֹ׃ \end{small}\\
\shatz יֹאמְרוּ־נָ֭א יִרְאֵ֣י יְיָ֑ כִּ֖י לְעוֹלָ֣ם חַסְדּֽוֹ׃\hfill \break
\kahal \begin{small}הוֹד֣וּ לַייָ֣ כִּי־ט֑וֹב כִּ֖י לְעוֹלָ֣ם חַסְדּֽוֹ׃ \end{small}\\
\firstword{מִֽן־הַ֭מֵּצַר}\source{תהלים קיח}
קָרָ֣אתִי יָּ֑הּ עָנָ֖נִי בַמֶּרְחָ֣ב יָֽהּ׃
יְיָ֣ לִ֭י לֹ֣א אִירָ֑א מַה־יַּעֲשֶׂ֖ה לִ֣י אָדָֽם׃
יְיָ֣ לִ֭י בְּעֹזְרָ֑י וַ֝אֲנִ֗י אֶרְאֶ֥ה בְשֹׂנְאָֽי׃
ט֗וֹב לַחֲס֥וֹת בַּייָ֑ מִ֝בְּטֹ֗חַ בָּאָדָֽם׃
ט֗וֹב לַחֲס֥וֹת בַּייָ֑ מִ֝בְּטֹ֗חַ בִּנְדִיבִֽים׃
כׇּל־גּוֹיִ֥ם סְבָב֑וּנִי בְּשֵׁ֥ם יְ֝יָ֗ כִּ֣י אֲמִילַֽם׃
סַבּ֥וּנִי גַם־סְבָב֑וּנִי בְּשֵׁ֥ם יְ֝יָ֗ כִּ֣י אֲמִילַֽם׃
סַבּ֤וּנִי כִדְבוֹרִ֗ים דֹּ֭עֲכוּ כְּאֵ֣שׁ קוֹצִ֑ים בְּשֵׁ֥ם יְ֝יָ֗ כִּ֣י אֲמִילַֽם׃
דַּחֹ֣ה דְחִיתַ֣נִי לִנְפֹּ֑ל וַ֖ייָ֣ עֲזָרָֽנִי׃
עָזִּ֣י וְזִמְרָ֣ת יָ֑הּ וַֽיְהִי־לִ֗֝י לִישׁוּעָֽה׃
ק֤וֹל ׀ רִנָּ֬ה וִישׁוּעָ֗ה בְּאׇהֳלֵ֥י צַדִּיקִ֑ים יְמִ֥ין יְ֝יָ֗ עֹ֣שָׂה חָֽיִל׃
יְמִ֣ין יְיָ֭ רוֹמֵמָ֑ה יְמִ֥ין יְ֝יָ֗ עֹ֣שָׂה חָֽיִל׃
לֹא־אָמ֥וּת כִּֽי־אֶחְיֶ֑ה וַ֝אֲסַפֵּ֗ר מַעֲשֵׂ֥י יָֽהּ׃
יַסֹּ֣ר יִסְּרַ֣נִּי יָּ֑הּ וְ֝לַמָּ֗וֶת לֹ֣א נְתָנָֽנִי׃
פִּתְחוּ־לִ֥י שַׁעֲרֵי־צֶ֑דֶק אָבֹא־בָ֗֝ם אוֹדֶ֥ה יָֽהּ׃
זֶה־הַשַּׁ֥עַר לַייָ֑ צַ֝דִּיקִ֗ים יָבֹ֥אוּ בֽוֹ׃\\
א֭וֹדְךָ כִּ֣י עֲנִיתָ֑נִי וַתְּהִי־לִ֗֝י לִישׁוּעָֽה׃ \\
\scriptsize{ א֭וֹדְךָ כִּ֣י עֲנִיתָ֑נִי וַתְּהִי־לִ֗֝י לִישׁוּעָֽה׃ \\}\normalsize{}
אֶ֭בֶן מָאֲס֣וּ הַבּוֹנִ֑ים הָ֝יְתָ֗ה לְרֹ֣אשׁ פִּנָּֽה׃ \\
\scriptsize{ אֶ֭בֶן מָאֲס֣וּ הַבּוֹנִ֑ים הָ֝יְתָ֗ה לְרֹ֣אשׁ פִּנָּֽה׃ \\}\normalsize{}
מֵאֵ֣ת יְיָ֭ הָ֣יְתָה זֹּ֑את הִ֖יא נִפְלָ֣את בְּעֵינֵֽינוּ׃ \\
\scriptsize{ מֵאֵ֣ת יְיָ֭ הָ֣יְתָה זֹּ֑את הִ֖יא נִפְלָ֣את בְּעֵינֵֽינוּ׃ \\}\normalsize{}
זֶה־הַ֭יּוֹם עָשָׂ֣ה יְיָ֑ נָגִ֖ילָה וְנִשְׂמְחָ֣ה בֽוֹ׃ \\
\scriptsize{ זֶה־הַ֭יּוֹם עָשָׂ֣ה יְיָ֑ נָגִ֖ילָה וְנִשְׂמְחָ֣ה בֽוֹ׃ } \normalsize{}


\instruction{ש״ץ ואח״כ הקהל׃}\\
אָנָּ֣א יְיָ֭ הוֹשִׁ֘יעָ֥ה נָּ֑א \hfill אָנָּ֣א יְיָ֭ הוֹשִׁ֘יעָ֥ה נָּ֑א\\
אָנָּ֥א יְ֝יָ֗ הַצְלִ֘יחָ֥ה נָּֽא׃ \hfill אָנָּ֥א יְ֝יָ֗ הַצְלִ֘יחָ֥ה נָּֽא׃\\
בָּר֣וּךְ הַ֭בָּא בְּשֵׁ֣ם יְיָ֑ בֵּ֝רַ֥כְנוּכֶ֗ם מִבֵּ֥ית יְיָ׃\\
\scriptsize{בָּר֣וּךְ הַ֭בָּא בְּשֵׁ֣ם יְיָ֑ בֵּ֝רַ֥כְנוּכֶ֗ם מִבֵּ֥ית יְיָ׃}\\
\normalsize{אֵ֤ל ׀ יְיָ וַיָּ֢אֶ֫ר לָ֥נוּ אִסְרוּ־חַ֥ג בַּעֲבֹתִ֑ים עַד־קַ֝רְנ֗וֹת הַמִּזְבֵּֽחַ׃}\\
\scriptsize{אֵ֤ל ׀ יְיָ וַיָּ֢אֶ֫ר לָ֥נוּ אִסְרוּ־חַ֥ג בַּעֲבֹתִ֑ים עַד־קַ֝רְנ֗וֹת הַמִּזְבֵּֽחַ׃}\\
\normalsize{אֵלִ֣י אַתָּ֣ה וְאוֹדֶ֑ךָּ אֱ֝לֹהַ֗י אֲרוֹמְמֶֽךָּ׃}\\
\scriptsize{אֵלִ֣י אַתָּ֣ה וְאוֹדֶ֑ךָּ אֱ֝לֹהַ֗י אֲרוֹמְמֶֽךָּ׃}\\
\normalsize{הוֹד֣וּ לַייָ֣ כִּי־ט֑וֹב כִּ֖י לְעוֹלָ֣ם חַסְדּֽוֹ׃}\\
\scriptsize{הוֹד֣וּ לַייָ֣ כִּי־ט֑וֹב כִּ֖י לְעוֹלָ֣ם חַסְדּֽוֹ׃} \\
\normalsize{}



\negline

\firstword{יְהַלְלֽוּךָ}
יְיָ אֱלֹהֵֽינוּ כׇּל־מַעֲשֶֽׂיךָ וַחֲסִידֶֽיךָ צַדִּיקִים עוֹשֵׂי רְצֹנֶֽךָ וְכׇל־עַמְּךָ בֵּית יִשְׂרָאֵל בְּרִנָּה יוֹדוּ וִיבָרְכוּ וִישַׁבְּחוּ וִיפָאֲרוּ וִירוֹמֲמוּ וְיַעֲרִֽיצוּ וְיַקְדִּֽישׁוּ וְיַמְלִֽיכוּ אֶת־שִׁמְךָ מַלְכֵּֽנוּ כִּי לְךָ טוֹב לְהוֹדוֹת וּלְשִׁמְךָ נָאֶה לְזַמֵּר כִּי מֵעוֹלָם וְעַד עוֹלָם אַתָּה אֵל׃ בָּרוּךְ אַתָּה יְיָ מֶֽלֶךְ מְהֻלָּל בַּתִּשְׁבָּחוֹת׃\\

%\ifboolexpr{togl {includeshabbat} and togl {includeweekday} and not togl {includeChM}}{\englishinst{On Shabbat, continue with Full Kaddish on page \pageref{shacharitShabbatYTtitkabel}. On weekday Rosh \d{H}odesh, continue with Full Kaddish on page \pageref{end of shacharis}, followed by the Psalm of the Day as relevant on page \pageref{shir_shel_yom}. On \d{H}anukka that is not Rosh \d{H}odesh, continue with Half Kaddish on page \pageref{hatzi_kaddish}.}}{}

\englishinst{On Shabbat, Festivals, and Rosh \d{H}odesh, recite Full Kaddish followed by the Psalm of the Day. Then Mourner's Kaddish is read, followed by the Torah service on page \pageref{shabYTtorah} for Shabbat and Festivals and page \pageref{weekday torah} for Intermediate Festival Days and Rosh \d{H}odesh.}

}{}

\ifboolexpr{togl {includeshabbat} and togl {includeweekday} and not togl {includefestival}}{\englishinst{On \d{H}anukka and Rosh \d{H}odesh Hallel is recited on page \pageref{hallel}.}}{}

%\vfill
\label{shacharitShabbatYTtitkabel}
\fullkaddish
\section[שיר של יום]{\adforn{53} שיר של יום‎ \adforn{25}}

\englishinst{Mourner's kaddish follows the Psalm of the Day.}
%\ifboolexpr{togl {includefestival}}{\weekdayshir}{}
\begin{small}הַיּוֹם שַּׁבָּת קֹֽדֶשׁ שֶׁבּוֹ הָיוּ הַלְוִיִּם אוֹמְרִים בְּבֵית־הַמִּקְדָּשׁ׃\end{small}\\
\mizmorshabbat\\

%\ifboolexpr{togl {includeshabbat}}{
%\RChBarekhi

%\instruction{בחנכה׃}
%\chanukat

\ledavid
%}{}

%\mournerskaddish