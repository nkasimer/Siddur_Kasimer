\chapter[ברכת המזון]{\adforn{47} ברכת המזון \adforn{19}}

%\source{תהלים קלז}
%
%\columnratio{0.63}
%\begin{paracol}{2}
%\instruction{בימים שיש בהם תחנון:}\\
%\firstword{עַ֥ל נַהֲר֨וֹת}
% בָּבֶ֗ל שָׁ֣ם יָ֭שַׁבְנוּ גַּם־בָּכִ֑ינוּ בְּ֝זׇכְרֵ֗נוּ אֶת־צִיּֽוֹן׃ עַֽל־עֲרָבִ֥ים בְּתוֹכָ֑הּ תָּ֝לִ֗ינוּ כִּנֹּרוֹתֵֽינוּ׃ כִּ֤י שָׁ֨ם שְֽׁאֵל֪וּנוּ שׁוֹבֵ֡ינוּ דִּבְרֵי־שִׁ֭יר וְתוֹלָלֵ֣ינוּ שִׂמְחָ֑ה שִׁ֥ירוּ לָ֝֗נוּ מִשִּׁ֥יר צִיּֽוֹן׃ אֵ֗יךְ נָשִׁ֥יר אֶת־שִׁיר־יְיָ֑ עַ֝֗ל אַדְמַ֥ת נֵכָֽר׃ אִֽם־אֶשְׁכָּחֵ֥ךְ יְֽרוּשָׁלִָ֗ם תִּשְׁכַּ֥ח יְמִינִֽי׃ תִּדְבַּ֥ק־לְשׁוֹנִ֨י לְחִכִּי֮ אִם־לֹ֪א אֶ֫זְכְּרֵ֥כִי אִם־לֹ֣א אַ֭עֲלֶה אֶת־יְרוּשָׁלִַ֑ם עַ֝֗ל רֹ֣אשׁ שִׂמְחָתִֽי׃ זְכֹ֤ר יְיָ֨ לִבְנֵ֬י אֱד֗וֹם אֵת֮ י֤וֹם יְֽרוּשָׁ֫לִָ֥ם הָ֭אֹ֣מְרִים עָ֤רוּ | עָ֑רוּ עַ֝֗ד הַיְס֥וֹד בָּֽהּ׃ בַּת־בָּבֶ֗ל הַשְּׁד֫וּדָ֥ה אַשְׁרֵ֥י שֶׁיְשַׁלֶּם־לָ֑ךְ אֶת־גְּ֝מוּלֵ֗ךְ שֶׁגָּמַ֥לְתְּ לָֽנוּ׃ אַשְׁרֵ֤י שֶׁיֹּאחֵ֓ז וְנִפֵּ֬ץ אֶֽת־עֹ֝לָלַ֗יִךְ אֶל־הַסָּֽלַע׃
%
%\switchcolumn

\ifboolexpr{togl {includeweekday}}{\englishinst{The following is added on festive occasions:}}{}
\firstword{שִׁ֗יר הַֽמַּֽ֫עֲל֥וֹת}\source{תהלים קכו}
בְּשׁ֣וּב יְ֖יָ אֶת־שִׁיבַ֣ת צִיּ֑וֹן הָ֝יִ֗ינוּ כְּחֹלְמִֽים׃ אָ֤ז יִמָּלֵ֢א שְׂחֹ֡ק פִּינוּ֘ וּלְשׁוֹנֵ֢נוּ רִ֫נָּ֥ה אָ֭ז יֹֽאמְר֣וּ בַגּוֹיִ֑ם הִגְדִּ֥יל יְ֜יָ֗ לַֽעֲשׂ֥וֹת עִם־אֵֽלֶּה׃ הִגְדִּ֥יל יְ֖יָ לַֽעֲשׂ֣וֹת עִמָּ֑נוּ הָ֜יִ֗ינוּ שְׂמֵחִֽים׃ שׁוּבָ֣ה יְ֖יָ אֶת־שְׁבִיתֵ֑נוּ כַּֽאֲפִיקִ֥ים בַּנֶּֽגֶב׃ הַזֹּֽרְעִ֥ים בְּדִמְעָ֗ה בְּרִנָּ֥ה יִקְצֹֽרוּ׃ הָ֘ל֤וֹךְ יֵלֵ֨ךְ וּבָכֹה֘ נֹשֵׂ֢א מֶֽשֶׁךְ־הַ֫זָּ֥רַע בֹּֽא־יָבֹ֥א בְרִנָּ֗ה נֹשֵׂ֥א אֲלֻמֹּתָֽיו׃
%\end{paracol}

\englishinst{Three or more who ate together have one person invite the others to bless with a Zimmun.}
\begin{small}
\begin{tabular}{l p{.8\textwidth}}

\instruction{המזמן:} &
רַבּוֹתַי נְבָרֵךְ! \instruction{או} רַבּוֹתַי מיר וועלן בענטשן! \instruction{או} הַב לָן וְנִבְרִךְ!\\
\instruction{כולם:} &
יְהִ֤י שֵׁ֣ם יְיָ֣ מְבֹרָ֑ךְ מֵֽ֝עַתָּ֗ה וְעַד־עוֹלָֽם׃\\
\instruction{המזמן:} &
בִּרְשׁוּת ... נְבָרֵךְ (\instruction{בעשרה} אֱלֹהֵֽינוּ) שֶׁאָכַלְנוּ מִשֶּׁלּוֹ:\\
\instruction{כולם:} &
בָּרוּךְ (\instruction{בעשרה:} אֱלֹהֵֽינוּ) שֶׁאָכַֽלְנוּ מִשֶּׁלּוֹ וּבְטוּבוֹ חָיִֽינוּ:\\
(\instruction{מי שלא אכל:} &
בָּרוּךְ וּמְבֹרָךְ שְׁמוֹ תָּמִיד לְעוֹלָם וָעֶד׃)\\
\instruction{המזמן:} &
בָּרוּךְ (\instruction{בעשרה:} אֱלֹהֵֽינוּ) שֶׁאָכַֽלְנוּ מִשֶּׁלּוֹ וּבְטוּבוֹ חָיִֽינוּ:
\end{tabular}

בָּרוּךְ הוּא וּבָרוּךְ שְׁמוֹ׃\\
\end{small}

\firstword{בָּרוּךְ}
אַתָּה יְיָ אֱלֹהֵֽינוּ מֶֽלֶךְ הָעוֹלָם הַזָּן אֶת־הָעוֹלָם כֻּלּוֹ בְּטוּבוֹ בְּחֵן בְּחֶֽסֶד וּבְרַחֲמִים הוּא נֹתֵ֣ן \source{תהלים קלו}לֶ֭חֶם לְכׇל־בָּשָׂ֑ר כִּ֖י לְעוֹלָ֣ם חַסְדּֽוֹ׃ וּבְטוּבוֹ הַגָּדוֹל תָּמִיד לֹא חָסַר לָֽנוּ וְאַל יֶחְסַר לָֽנוּ מָזוֹן לְעוֹלָם וָעֶד׃ בַּעֲבוּר שְׁמוֹ הַגָּדוֹל כִּי הוּא זָן וּמְפַרְנֵס לַכֹּל וּמֵטִיב לַכֹּל וּמֵכִין מָזוֹן לְכׇל־בְּרִיּוֹתָיו אֲשֶׁר בָּרָא׃ בָּרוּךְ אַתָּה יְיָ הַזָּן אֶת־הַכֹּל׃



\firstword{נוֹדֶה}
לְךָ יְיָ אֱלֹהֵֽינוּ עַל שֶׁהִנְחַֽלְתָּ לַאֲבוֹתֵֽינוּ אֶֽרֶץ חֶמְדָה טוֹבָה וּרְחָבָה׃ וְעַל שֶׁהוֹצֵאתָֽנוּ יְיָ אֱלֹהֵֽינוּ מֵאֶֽרֶץ מִצְרַֽיִם וּפְדִיתָֽנוּ מִבֵּית עֲבָדִים וְעַל בְּרִיתְךָ שֶׁחָתַֽמְתָּ בִּבְשָׂרֵֽנוּ וְעַל תּוֹרָתְךָ שֶׁלִּמַּדְתָּֽנוּ וְעַל חֻקֶּֽיךָ שֶׁהוֹדַעְתָּֽנוּ וְעַל חַיִּים חֵן וָחֶֽסֶד שֶׁחוֹנַנְתָּֽנוּ וְעַל אֲכִילַת מָזוֹן שָׁאַתָּה זָן וּמְפַרְנֵס אוֹתָֽנוּ תָּמִיד בְּכׇל־יוֹם וּבְכׇל־עֵת וּבְכׇל־שָׁעָה׃


\alhanisim

\firstword{וְעַל הַכֹּל}
יְיָ אֱלֹהֵֽינוּ אֲנַֽחְנוּ מוֹדִים לָךְ וּמְבָרְכִים אוֹתָךְ יִתְבָּרַךְ שִׁמְךָ בְּפִי כׇל־חַי תָּמִיד לְעוֹלָם וָעֶד׃ כַּכָּתוּב׃ \source{דברים ח}%
וְאָכַלְתָּ֖ וְשָׂבָ֑עְתָּ וּבֵֽרַכְתָּ֙ אֶת־יְיָ֣ אֱלֹהֶ֔יךָ עַל־הָאָ֥רֶץ הַטֹּבָ֖ה אֲשֶׁ֥ר נָֽתַן־לָֽךְ׃
בָּרוּךְ אַתָּה יְיָ עַל הָאָֽרֶץ וְעַל הַמָּזוֹן׃



\firstword{רַחֵם}
יְיָ אֱלֹהֵֽינוּ עָלֵֽינוּ וְעַל יִשְׂרָאֵל עַמֶּךָ וְעַל יְרוּשָׁלַ‍ִם עִירֶֽךָ וְעַל צִיּוֹן מִשְׁכַּן כְּבוֹדֶֽךָ וְעַל מַלְכוּת בֵּית דָּוִד מְשִׁיחֶֽךָ וְעַל הַבַּֽיִת הַגָּדוֹל וְהַקָּדוֹשׁ שֶׁנִּקְרָא שִׁמְךָ עָלָיו׃ אֱלֹהֵֽינוּ אָבִֽינוּ רְעֵֽנוּ זוּנֵֽנוּ פַרְנְסֵֽנוּ וְכַלְכְּלֵֽנוּ וְהַרְוִיחֵֽנוּ וְהַרְוַח לָֽנוּ יְיָ אֱלֹהֵֽינוּ מְהֵרָה מִכׇּל־צָרוֹתֵֽינוּ׃ וְנָא אַל תַּצְרִיכֵֽנוּ יְיָ אֱלֹהֵֽינוּ לֹא לִידֵי מַתְּנַת בָּשָׂר וָדָם וְלֹא לִידֵי הַלְוָאָתָם. כִּי אִם לְיָדְךָ הַמְּלֵאָה הַפְּתוּחָה הַקְּדוֹשָׁה וְהָרְחָבָה שֶׁלֹּא נֵבוֹשׁ וְלֹא נִכָּלֵם לְעוֹלָם וָעֶד׃

%\enlargethispage{\baselineskip}
%
%\vspace{-.25\baselineskip}
\ifboolexpr{togl {includefestival} or (togl {includeshabbat} and togl {includeweekday})}{
\begin{sometimes}

\shabbos
רְצֵה וְהַחֲלִיצֵֽנוּ יְיָ אֱלֹהֵֽינוּ בְּמִצְוֹתֶֽיךָ וּבְמִצְוַת יוֹם הַשְּׁבִיעִי הַשַּׁבָּת הַגָּדוֹל וְהַקָּדוֹשׁ הַזֶּה כִּי יוֹם זֶה גָּדוֹל וְקָדוֹשׁ הוּא לְפָנֶֽיךָ לִשְׁבׇּת־בּוֹ וְלָנֽוּחַ־בּוֹ בְּאַהֲבָה כְּמִצְוַת רְצוֹנֶֽךָ׃ בִּרְצוֹנְךָ הָנִֽיחַ לָֽנוּ יְיָ אֱלֹהֵֽינוּ שֶׁלֹא תְהֵי צָרָה וְיָגוֹן וַאֲנָחָה בְּיוֹם מְנוּחָתֵֽנוּ וְהַרְאֵֽנוּ יְיָ אֱלֹהֵֽינוּ בְּנֶחָמוֹת צִיּוֹן עִירֶֽךָ וּבְבִנְיַן יְרוּשָׁלַ‍ִם עִיר קׇדְשֶֽׁךָ כִּי אַתָּה הוּא בַּֽעַל הַיְשׁוּעוֹת וּבַֽעַל הַנֶּחָמוֹת׃


\sepline %These are really two "sometimes's". Sepline to separate them

\vspace{-.25\baselineskip}
\instruction{בראש חודש ומועדים:}\\
אֱלֹהֵֽינוּ וֵאלֹהֵי אֲבוֹתֵֽינוּ יַעֲלֶה וְיָבֹא וְיַגִּיעַ וְיֵרָאֶה וְיֵרָצֶה וְיִשָּׁמַע וְיִפָּקֵד וְיִזָּכֵר זִכְרוֹנֵֽנוּ וּפִקְדּוֹנֵֽנוּ וְזִכְרוֹן אֲבוֹתֵֽינוּ וְזִכְרוֹן מָשִׁיחַ בֶּן דָּוִד עַבְדֶּֽךָ וְזִכְרוֹן יְרוּשָׁלַ‍ִם עִיר קׇדְשֶֽׁךָ וְזִכְרוֹן כׇּל־עַמְּךָ בֵּית יִשְׂרָאֵל לְפָנֶיךָ לִפְלֵיטָה וּלְטוֹבָה וּלְחֵן וּלְחֶֽסֶד וּלְרַחֲמִים וּלְחַיִּים וּלְשָׁלוֹם בְּיוֹם\\
\begin{tabular}{c|c|c}
רֹאשׁ הַחֹֽדֶשׁ & חַג הַמַּצוֹת & חַג הַשָּׁבֻעוֹת\\ \hline
\end{tabular}\\
\begin{tabular}{c|c|c}
הַזִּכָּרוֹן & חַג הַסֻּכּוֹת & שְׁמִינִי חַג הָעֲצֶֽרֶת
\end{tabular}\\
הַזֶּה זׇכְרֵֽנּוּ יְיָ אֱלֹהֵֽינוּ בּוֹ לְטוֹבָה וּפׇקְדֵֽנוּ בוֹ לִבְרָכָה וְהוֹשִׁיעֵֽנוּ בוֹ לְחַיִּים וּבִדְבַר יְשׁוּעָה וְרַחֲמִים חוּס וְחׇׇׇׇנֵּנוּ וְרַחֵם עָלֵֽינוּ וְהוֹשִׁיעֵֽנוּ כִּי אֵלֶֽיךָ עֵינֵֽינוּ כִּי אֵל מֶֽלֶךְ חַנּוּן וְרַחוּם אַֽתָּה׃

\end{sometimes}}{
\ifboolexpr{togl {includeshabbat}}{
	רְצֵה וְהַחֲלִיצֵֽנוּ יְיָ אֱלֹהֵֽינוּ בְּמִצְוֹתֶֽיךָ וּבְמִצְוַת יוֹם הַשְּׁבִיעִי הַשַּׁבָּת הַגָּדוֹל וְהַקָּדוֹשׁ הַזֶּה כִּי יוֹם זֶה גָּדוֹל וְקָדוֹשׁ הוּא לְפָנֶֽיךָ לִשְׁבׇּת־בּוֹ וְלָנֽוּחַ־בּוֹ בְּאַהֲבָה כְּמִצְוַת רְצוֹנֶךָ׃ בִּרְצוֹנְךָ הָנִֽיחַ לָֽנוּ יְיָ אֱלֹהֵֽינוּ שֶׁלֹא תְהֵי צָרָה וְיָגוֹן וַאֲנָחָה בְּיוֹם מְנוּחָתֵֽנוּ וְהַרְאֵֽנוּ יְיָ אֱלֹהֵֽינוּ בְּנֶחָמוֹת צִיּוֹן עִירֶֽךָ וּבְבִנְיַן יְרוּשָׁלַ‍ִם עִיר קׇדְשֶֽׁךָ כִּי אַתָּה הוּא בַּֽעַל הַיְשׁוּעוֹת וּבַֽעַל הַנֶּחָמוֹת׃
	
	\instruction{בר״ח׃}
	\yaalehveyavotemplate{רֹאשׁ הַחֹֽדֶשׁ}
}{}
}

\firstword{וּבְנֵה}
יְרוּשָׁלַ‍ִם עִיר הַקֹּֽדֶשׁ בִּמְהֵרָה בְּיָמֵֽינוּ׃ בָּרוּךְ אַתָּה יְיָ בֹּֽנֶה בְרַחֲמָיו יְרוּשָׁלַ‍ִם אָמֵן׃

%\begin{sometimes}
%
%\instruction{אם שכח רצה או יעלה ויבא:}\\
%בָּרוּךְ אַתָּה יְיָ אֱלֹהֵֽינוּ מֶֽלֶךְ הָעוֹלָם אֲשֶׁר נָתַן (שַׁבָּתוֹת לִמְנוּחָה לְעַמּוֹ יִשְׂרָאֵל בְּאַהֲבָה לְאוֹת וְלִבְרִית)
%(וְיָמִים טוֹבִים לְשָׂשׂוֹן וּלְשִׂמְחָה אֶת־יוֹם חַג ... הַזֶּה)(וְרָאשֵׁי חֳדָשִׁים לְזִכָּרוֹן \instruction{מסיים כאן בחול}):
%בָּרוּךְ אַתָּה יְיָ מְקַדֵּשׁ (הַשַּׁבָּת) ([וְ]יִשְׂרָאֵל וְהַזְּמַנִּים)(וְיִשְׂרָאֵל וְרָאשֵׁי חֳדָשִׁים׃)׃
%
%\end{sometimes}


\firstword{בָּרוּךְ}
אַתָּה יְיָ אֱלֹהֵֽינוּ מֶֽלֶךְ הָעוֹלָם הָאֵל אָבִֽינוּ מַלְכֵּֽנוּ אַדִּירֵֽנוּ בּוֹרְאֵֽנוּ גֹאֲלֵֽנוּ יוֹצְרֵֽנוּ קְדוֹשֵֽׁנוּ קְדוֹשׁ יַעֲקֹב רוֹעֵֽנוּ רוֹעֵה יִשְׂרָאֵל הַמֶּֽלֶךְ הַטּוֹב וְהַמֵּטִיב לַכֹּל שֶׁבְּכׇל־יוֹם וָיוֹם הוּא הֵטִיב הוּא מֵטִיב הוּא יֵיטִיב לָֽנוּ׃ הוּא גְמָלָֽנוּ הוּא גוֹמְלֵנוּ הוּא יִגְמְלֵנוּ לָעַד לְחֵן לְחֶֽסֶד וּלְרַחֲמִים וּלְרֶֽוַח הַצָּלָה וְהַצְלָחָה בְּרָכָה וִישׁוּעָה נֶחָמָה פַּרְנָסָה וְכַלְכָּלָה וְרַחֲמִים וְחַיִּים וְשָׁלוֹם וְכׇל־טוֹב וּמִכׇּל־טוֹב אַל יְחַסְּרֵֽנוּ׃

\firstword{הָרַחֲמָן}
הוּא יִמְלֹךְ עָלֵֽינוּ לְעוֹלָם וָעֶד׃
\firstword{הָרַחֲמָן}
הוּא יִתְבָּרַךְ בַּשָּׁמַֽיִם וּבָאָֽרֶץ׃
\firstword{הָרַחֲמָן}
הוּא יִשְׁתַּבַּח לְדוֹר דּוֹרִים וְיִתְפָּֽאַר בָּֽנוּ לָנֵֽצַח נְצָחִים
וְיִתְהַדַּר בָּֽנוּ לָעַד וּלְעוֹלְמֵי עוֹלָמִים׃
\firstword{הָרַחֲמָן}
הוּא יְפַרְנְסֵֽנוּ בְּכָבוֹד׃
\firstword{הָרַחֲמָן}
הוּא יִשְׁבּוֹר עֻלֵּֽנוּ מֵעַל צַוָּארֵֽנוּ וְהוּא יוֹלִיכֵֽנוּ קוֹמְמִיּוּת לְאַרְצֵֽנוּ׃
\firstword{הָרַחֲמָן}
הוּא יִשְׁלַח בְּרָכָה מְרֻבָּה בְּבַֽיִת זֶה וְעַל שֻׁלְחָן זֶה שֶׁאָכַֽלְנוּ עָלָיו׃
\firstword{הָרַחֲמָן}
הוּא יִשְׁלַח לָֽנוּ אֶת־אֵלִיָּֽהוּ הַנָּבִיא זָכוּר לַטּוֹב וִיבַשֵּׂר לָנוּ בְּשׂוֹרוֹת טוֹבוֹת יְשׁוּעוֹת וְנֶחָמוֹת׃


\begin{footnotesize}
\instruction{אורחים אומרים:}\\
יְהִי רָצוֹן שֶׁלֹא יֵבוֹשׁ בַּעַל הַבַּיִת בָּעוֹלָם הַזֶּה וְלֹא יִכָּלֵם לָעוֹלָם הַבָּא וְיִצְלַח מְאֹד בְּכׇל־נְכָסָיו וְיִהְיוּ נְכָסָיו מֻצְלָחִים וּקְרוֹבִים לָעִיר וְאַל יִשְׁלוֹט שָׂטָן לֹא בְּמַעֲשֵׂי יָדָיו וְלֹא בְּמַעֲשֵׂי יָדֵינוּ וְאַל יִזְדַקֵּר לֹא לְפָנָיו וְלֹא לְפָנֵינוּ שׁוּם דְבַר הִרְהוּר חֵטְא וַעֲבֵרָה וְעָוֹן מֵעַתָּה וְעַד עוֹלָם׃

\end{footnotesize}

\firstword{הָרַחֲמָן}
הוּא יְבָרֵךְ אֶת־[אָבִי מוֹרִי] בַּעַל הַבַּֽיִת הַזֶּה וְאֶת־[אִמִּי מוֹרָתִי] בַּעֲלַת הַבַּֽיִת הַזֶּה׃ אוֹתָם וְאֶת־בֵּיתָם וְאֶת־זַרְעָם וְאֶת־כׇּל־אַשֶׁר לָהֶם, אוֹתָנוּ וְאֶת־כׇּל־אַשֶׁר לָֽנוּ כְּמוֹ שֶׁנִּתְבָּרְכוּ אֲבוֹתֵֽינוּ אַבְרָהָם יִצְחָק וְיַעֲקֹב בַּכֹּל מִכֹּל כֹּל כֵּן יְבָרֵךְ אוֹתָֽנוּ כֻּלָּנוּ יַֽחַד בִּבְרָכָה שְׁלֵמָה וְנֹאמַר אָמֵן׃

\begin{sometimes}

\englishinst{At the meal following a Berit Mila:}\nopagebreak
\begin{center}
\textbf{הָרַחֲמָן}
הוּא אֲשֶׁר חָנַן אֶת־הַיֶּלֶד הַזֶּה לְאָבִיו וּלְאִמּוֹ הוּא יָגֵן עָלָיו מִמְּרוֹמוֹ וּבְשָׁלוֹם יָבֹא עַל־מְקוֹמוֹ וִיהִי אֱלֹהָיו עִמּוֹ וּכְאֶפְרַיִם וְכִמְנַשֶּׁה לְשׂוּמוֹ ְויִקָּרֵא בְיִשְׂרָאֵל שְׁמוֹ׃

\textbf{הָרַחֲמָן}
הוּא פָּקוֹד יִפְקְדֵהוּ בְּרַחֲמִים לַהֲבִינוֹ בְּדָת חִכּוּמִים וִיבַלֶּה בַטּוֹב יָמִים וּשְׁנוֹתָיו בַּנְּעִימִים יַעַבְדוּהוּ עַמִּים וְיִשְׁתַּחֲווּ לוֹ לְאֻמִּים׃

\textbf{הָרַחֲמָן}
הוּא רַבּוֹת שָׁנִים יְחַיֵּהוּ צֶדֶק לְרַגְלָיו יִקְרָאֵהוּ וְנֶחָמַת צִיּוֹן יַרְאֵהוּ וְאֶת־עַמּוֹ לְשָׁלוֹם יְבָרְכֵהוּ וִיעוֹרֵר חֲסָדָיו לְרַחֲמֵהוּ כִּי חָפֵץ חֶסֶד הוּא׃

\textbf{הָרַחֲמָן}
הוּא יְבָרֵךְ אֶת־הֶחָתָן הַזֶּה וּבַעַל בְּרִיתוֹ יִשְׂמַח אָבִיו וְתָגֵל יוֹלַדְתּוֹ וְיִתְבָּרַכוּ הַמְסֻבִּים בִּסְעוּדָתוֹ וְכֵן יִזְכּוּ שֶׁיִּשְׂמְחוּ בַּחֲתֻנָּתוֹ בְּנֵי בָנִים לְהַרְאוֹתוֹ וְיֵשׁ תִּקְוָה לְאַחֲרִיתוֹ׃

\textbf{הָרַחֲמָן}
הוּא מְהֵרָה יִזְכֹּר זֹאת מִצְוָתוֹ בָּהּ יִפְדֶה אֲיֻמָּתוֹ רַחֲמִים יְעוֹרֵר לַעֲדָתוֹ קְהָלָיו יְקַבֵּץ בְּחֶמְלָתוֹ בְּהַרְאֹתוֹ אֶת־עֹשֶׁר כְּבוֹד מַלְכוּתוֹ וְאֶת־יְקָר תִּפְאֶרֶת גְּדֻלָּתוֹ׃

\textbf{הָרַחֲמָן}
הוּא יְבָרֵךְ אֶת־הֶחָתָן הַזֶּה וּבַּעַל בְּרִיתוֹ וְאֶת אָבִיו וְאֶת אִמּוֹ וְאֶת רַבּוֹתֵינוּ וְאֶת־אַחֵינוּ הַיּוֹשְׁבִים פֹּה כְּמוֹ שֶׁנִתְבָּרְכוּ אֲבוֹתֵֽינוּ אַבְרָהָם יִצְחָק וְיַעֲקֹב בַּכֹּל מִכֹּל כֹּל כֵּן יְבָרֵךְ אוֹתָֽנוּ כֻּלָּנוּ יַֽחַד בִּבְרָכָה שְׁלֵמָה וְנֹאמַר אָמֵן׃
\end{center}
\end{sometimes}

\begin{center}
\firstword{בַּמָּרוֹם}
יְלַמְּדוּ עֲלֵיהֶם וְעָלֵֽינוּ זְכוּת שֶׁתְּהֵא לְמִשְׁמֶֽרֶת שָׁלוֹם׃ וְנִשָּׂא בְרָכָה מֵאֵת יְיָ וּצְדָקָה מֵאֱלֹהֵי יִשְׁעֵנוּ וְנִמְצָא חֵן וְשֵֽׂכֶל טוֹב בְּעֵינֵי אֱלֹהִים וְאָדָם׃
\end{center}

\begin{longtable}{l p{.8\textwidth}}

\shabbos &
הָרַחֲמָן הוּא יַנְחִילֵֽנוּ לְיּוֹם שֶׁכֻּלּוֹ שַׁבָּת וּמְנוּחָה לְחַיֵּי הָעוֹלָמִים׃ \\

\instruction{בראש חודש:} &
הָרַחֲמָן הוּא יְחַדֵּשׁ עָלֵֽינוּ אֶת־הַחֹֽדֶשׁ הַזֶּה לְטוֹבָה וְלִבְרָכָה׃ \\

\instruction{בשלש רגלים:} &
הָרַחֲמָן הוּא יַנְחִילֵֽנוּ לְיּוֹם שֶׁכֻּלּוֹ טוֹב׃ \\

\instruction{בראש השנה:} &
הָרַחֲמָן הוּא יְחַדֵּשׁ עָלֵֽינוּ אֶת־הַשָּׁנָה הַזֹּאת לְטוֹבָה וְלִבְרָכָה׃ \\

\instruction{בסכות:} &
הָרַחֲמָן הוּא יָקִים לָֽנוּ אֶת־סֻכַּ֥ת דָּוִ֖יד הַנֹּפֶ֑לֶת׃ \mdsource{עמוס ט}

\end{longtable}

\firstword{הָרַחֲמָן}
הוּא יְזַכֵּֽנוּ לִימוֹת הַמָּשִֽׁיחַ וּלְחַיֵּי עוֹלָם הַבָּא׃

\firstword{מַגְדִּיל֘}\source{תהלים יח}
(\instruction{בשבת, יו״ט, ור״ח׃ }
מִגְדּ֖וֹל)
יְשׁוּע֢וֹת מַ֫לְכּ֥וֹ וְעֹ֤שֶׂה חֶ֨סֶד ׀ לִמְשִׁיח֗וֹ לְדָוִ֥ד וּלְזַרְע֗וֹ עַד־עוֹלָֽם׃
עֹשֶׂה שָׁלוֹם בִּמְרוֹמָיו הוּא יַעֲשֶׂה שָׁלוֹם עָלֵֽינוּ וְעַל כׇּל־יִשְׂרָאֵל וְאִמְרוּ אָמֵן׃\\
יְר֣אוּ \source{תהלים לד}אֶת־יְיָ֣ קְדֹשָׁ֑יו כִּי־אֵ֥ין מַ֝חְס֗וֹר לִירֵאָֽיו׃ כְּ֭פִירִים רָשׁ֣וּ וְרָעֵ֑בוּ
וְדֹרְשֵׁ֥י יְ֝יָ֗ לֹא־יַחְסְר֥וּ כׇל־טֽוֹב׃
הוֹד֣וּ לַֽיְיָ֑ \source{תהלים קיח}כִּי־ט֑וֹב כִּ֖י לְעוֹלָ֣ם חַסְדּֽוֹ׃
פּוֹתֵ֥חַ \source{תהלים קמה}אֶת־יָדֶ֑ךָ וּמַשְׂבִּ֖יעַ לְכׇל־חַ֣י רָצֽוֹן׃
בָּר֣וּךְ \source{ירמיהו יז}הַגֶּ֔בֶר אֲשֶׁ֥ר יִבְטַ֖ח בַּייָ֑ וְהָיָ֥ה יְיָ֖ מִבְטַחֽוֹ׃
נַ֤עַר \source{תהלים לז}׀ הָיִ֗יתִי גַּם־זָ֫קַ֥נְתִּי וְֽלֹא־רָ֭אִיתִי צַדִּ֣יק נֶעֱזָ֑ב וְ֝זַרְע֗וֹ מְבַקֶּשׁ־לָֽחֶם׃
יְיָ֗ \source{תהלים כט}עֹ֭ז לְעַמּ֣וֹ יִתֵּ֑ן יְיָ֓ ׀ יְבָרֵ֖ךְ אֶת־עַמּ֣וֹ בַשָּׁלֽוֹם׃

\bigskip

\sepline

\bigskip

\instruction{המזמן:}
בָּרוּךְ אַתָּה יְיָ אֱלֹהֵֽינוּ מֶֽלֶךְ הָעוֹלָם בּוֹרֵא פְּרִי הַגָּֽפֶן׃

\vfill
\sepline

\ifboolexpr{togl {includeweekday}}{
\section[ברכת המזון בבית אבל]{\adforn{53} ברכת המזון בבית אבל \adforn{25}}

\instruction{זימון בבית־אבל}

\begin{small}
	\begin{tabular}{l p{.8\textwidth}}
		
		\instruction{המזמן:} &
		רַבּוֹתַי נְבָרֵךְ! \instruction{או} רַבּוֹתַי מיר וועלן בענטשן! \instruction{או} הַב לָן וְנִבְרִךְ!\\
		\instruction{כולם:} &
		יְהִ֤י שֵׁ֣ם יְיָ֣ מְבֹרָ֑ךְ מֵֽ֝עַתָּ֗ה וְעַד־עוֹלָֽם׃\\
		\instruction{המזמן:} &
		בִּרְשׁוּת ... נְבָרֵךְ (\instruction{בעשרה} אֱלֹהֵֽינוּ) מְנַחֵם אֲבֵלִים שֶׁאָכַלְנוּ מִשֶּׁלּוֹ:\\
		\instruction{כולם:} &
		בָּרוּךְ (\instruction{בעשרה:} אֱלֹהֵֽינוּ) מְנַחֵם אֲבֵלִים שֶׁאָכַֽלְנוּ מִשֶּׁלּוֹ וּבְטוּבוֹ חָיִֽינוּ:\\
		(\instruction{מי שלא אכל:} &
		בָּרוּךְ מְנַחֵם אֲבֵלִים וּמְבֹרָךְ שְׁמוֹ תָּמִיד לְעוֹלָם וָעֶד׃)\\
		\instruction{המזמן:} &
		בָּרוּךְ (\instruction{בעשרה:} אֱלֹהֵֽינוּ) מְנַחֵם אֲבֵלִים שֶׁאָכַֽלְנוּ מִשֶּׁלּוֹ וּבְטוּבוֹ חָיִֽינוּ:
	\end{tabular}

בָּרוּךְ הוּא וּבָרוּךְ שְׁמוֹ׃

\end{small}

נַחֵם יְיָ אֱלֹהֵינוּ אֶת אֲבֵלֵי יְרוּשָׁלַיִם. וְאֶת הָאֲבֵלִים הַמִּתְאַבְּלִים בָּאֵֽבֶל הַזֶּה. נְחַמֵּם מֵאֶבְלָם וְשֶׁמֵּחָם מִיגוֹנָם כָּאָמוּר׃\source{ישעיה סו} כְּאִ֕ישׁ אֲשֶׁ֥ר אִמּ֖וֹ תְּנַחֲמֶ֑נּוּ כֵּ֤ן אָֽנֹכִי֙ אֲנַ֣חֶמְכֶ֔ם וּבִירֽוּשָׁלַ֖͏ִם תְּנֻחָֽמוּ׃ בָּרוּךְ אַתָּה יְיָ מְנַחֵם צִיּוֹן בְּבִנְיַן יְרוּשָׁלַיִם׃

\firstword{בָּרוּךְ}
אַתָּה יְיָ אֱלֹהֵֽינוּ מֶֽלֶךְ הָעוֹלָם הָאֵל אָבִֽינוּ מַלְכֵּֽנוּ אַדִּירֵֽנוּ בּוֹרְאֵֽנוּ גֹאֲלֵֽנוּ יוֹצְרֵֽנוּ קְדוֹשֵֽׁנוּ קְדוֹשׁ יַעֲקֹב. הַמֶּלֶךְ הַחַי הַטּוֹב וְהַמֵּטִיב. אֵל אֱמֶת, דַּיַּן אֱמֶת, שׁוֹפֵט בְּצֶדֶק, לוֹקֵחַ נְפָשׁוֹת בַּמִּשְׁפָּט. שַׁלִּיט בְּעוֹלָמוֹ לַעֲשׂוֹת בּוֹ כִּרְצוֹנוֹ כִּי כׇל־דְּרָכָיו בַּמִּשְׁפָּט, וַאֲנַחְנוּ עַמּוֹ וַעֲבָדָיו, וּבַכֹּל אֲנַחְנוּ חַיָּבִים לְהוֹדוֹת לוֹ וּלְבָרְכוֹ. גּוֹדֵר פְּרָצוֹת יִשְׂרָאֵל הוּא יִגְדֹּר הַפִּרְצָה הַזֹּאת מֵעָלֵינוּ וּמֵעַל אֲבָל זֶה לְחַיִּים וּלְשָׁלוֹם. הוּא יִגְמְלֵנוּ לָעַד לְחֵן לְחֶֽסֶד וּלְרַחֲמִים וּלְרֶֽוַח הַצָּלָה וְהַצְלָחָה בְּרָכָה וִישׁוּעָה נֶחָמָה פַּרְנָסָה וְכַלְכָּלָה וְרַחֲמִים וְחַיִּים וְשָׁלוֹם וְכׇל־טוֹב וּמִכׇּל־טוֹב אַל יְחַסְּרֵֽנוּ׃}{}


\section[ברכה מעין שלש]{\adforn{53} ברכה מעין שלש \adforn{25}}

\englishinst{After eating foods made from the five grains (besides bread), grapes, figs, pomegranate, olives, dates, or drinking wine, recite the following blessing.}
\firstword{בָּרוּךְ}
אַתָּה יְיָ אֱלֹהֵֽינוּ מֶֽלֶךְ הָעוֹלָם עַל

\begin{tabular}{>{\centering\arraybackslash}m{.3\textwidth} | >{\centering\arraybackslash}m{.3\textwidth} | >{\centering\arraybackslash}m{.3\textwidth}}

הָעֵץ וְעַל פְּרִי הָעֵץ
&
הַמִּחְיָה וְעַל הַכַּלְכָּלָה
&
הַגֶּֽפֶן וְעַל פְּרִי הַגֶּֽפֶן \\

\end{tabular}

וְעַל תְּנוּבַת הַשָּׂדֶה וְעַל אֶֽרֶץ חֶמְדָּה טוֹבָה וּרְחָבָה
שֶׁרָצִֽיתָ וְהִנְחַֽלְתָּ לַאֲבוֹתֵֽינוּ לֶאֱכוֹל מִפִּרְיָהּ וְלִשְׂבּֽוֹעַ מִטּוּבָהּ׃
רַחֶם יְיָ אֱלֹהֵֽינוּ עַל יִשְׂרָאֵל עַמֶּֽךָ וְעַל יְרוּשָׁלַֽיִם עִירֶֽךָ וְעַל צִיּוֹן מִשְׁכַּן כְּבוֹדֶֽךָ וְעַל מִזְבַּחֲךָ וְעַל הֵיכָלֶֽךָ׃ וּבְנֵה יְרוּשָׁלַֽיִם עִיר הַקֹּדֶשׁ בִּמְהֵרָה בְּיָמֵֽינוּ וְהַעֲלֵֽנוּ לְתוֹכָהּ וְשַׂמְּחֵֽנוּ בְּבִנְיָנָהּ וְנֹאכַל מִפִּרְיָהּ וְנִשְׂבַּע מִטּוּבָהּ וּנְבָרֶכְךָ עָלֶיהָ בִּקְדֻשָּׁה וּבְטׇהֳרָה׃

\begin{small}

\begin{tabular}{l p{.7\textwidth}}
\instruction{שבת:}&
וּרְצֵה וְהַחֲלִיצֵֽנוּ בְּיוֹם הַשַּׁבָּת הַזֶּה׃ \\


\instruction{ראש חודש:}&
וְזׇכְרֵֽנוּ לְטוֹבָה
בְּיוֹם רֹאשׁ הַחֹֽדֶשׁ הַזֶּה׃ \\

\instruction{שלוש רגלים:}&
וְשַׂמְּחֵֽנוּ בְּיוֹם
חַג הַמַּצּוֹת \textbackslash \space הַשָּׁבֻעוֹת \textbackslash \space הַסֻּכּוֹת \textbackslash \space שְׁמִינִי חַג הָעֲצֶֽרֶת הַזֶּה׃\\


\instruction{ראש השנה:}&
וְזׇכְרֵֽנוּ לְטוֹבָה בְּיוֹם חַזִּכָּרוֹן הַזֶּה׃\\

\end{tabular}

\end{small}

כִּי אַתָּה טוֹב וּמֵטִיב לַכֹּל וְנוֹדֶה לְךָ עַל הָאָֽרֶץ,

\begin{tabular}{c|c|c}
וְעַל הַפֵּרוֹת & וְעַל הַמִּחְיָה & וְעַל פְּרִי הַגָּֽפֶן
\end{tabular}

בָּרוּךְ אַתָּה יְיָ עַל הָאָֽרֶץ

\begin{tabular}{c|c|c}
וְעַל הַפֵּרוֹת׃ & וְעַל הַמִּחְיָה׃ & וְעַל פְּרִי הַגָּֽפֶן׃
\end{tabular}

\englishinst{After all other foods:}
\firstword{בָּרוּךְ}
אַתָּה יְיָ אֱלֹהֵֽינוּ מֶֽלֶךְ הָעוֹלָם בּוֹרֵא נְפָשׁוֹת רַבּוֹת וְחֶסְרוֹנָן
עַל כׇּל־מַה שֶּׁבָּרָא לְהַחֲיוֹת בָּהֶם נֶֽפֶשׁ כׇּל־חָי׃ בָּרוּךְ חַי הָעוֹלָמִים׃\\

\chapter[ברכות]{\adforn{47} ברכות \adforn{19}}

\newcommand{\berakha}[2]{\englishinst{#1}
בָּרוּךְ אַתָּה יְיָ אֱלֺהֵֽינוּ מֶֽלֶךְ הָעוֹלָם #2׃}
\newcommand{\berakhamitzva}[2]{\berakha{#1}{אֲשֶׁר קִדְּשָֽׁנוּ בְּמִצְוֺתָיו וְצִוָּֽנוּ #2}}


\ssubsection{\adforn{18} ברכות על אכילה \adforn{17}}

\ifboolexpr{not togl {includeweekday}}{
\berakhamitzva{On washing hands before eating bread:}{עַל נְטִילַת יָדָיִם}
\berakha{Before eating bread:}{הַמּֽוֹצִיא לֶֽחֶם מִן הָאָֽרֶץ}
}{
\berakha{Before eating bread, wash hands and recite its blessing in the following section, then recite the following blessing before eating:}{הַמּֽוֹצִיא לֶֽחֶם מִן הָאָֽרֶץ}
}

\berakha{On food made from grains besides bread (e.g. crackers or cake):}{בּוֹרֵא מִינֵי מְזוֹנוֹת}

\berakha{Before drinking wine:}{בּוֹרֵא פְּרִי הַגָּֽפֶן}

\berakha{Before eating fruit that grows on trees:}{בּוֹרֵא פְּרִי הָעֵץ}

\berakha{Before eating fruits or vegetables:}{בּוֹרֵא פְּרִי הָאֲדָמָה}

\berakha{On all other foods:}{שֶׁהַכֹּל נִהְיֶה בִּדְבָרוֹ}

\ssubsection{\adforn{18} ברכות הריח \adforn{17}}

\berakha{On pleasant-smelling products from trees:}{בּוֹרֵא עֲצֵי בְשָׂמִים}

\berakha{On fragrant shrubs and grasses:}{בּוֹרֵא עִשְׂבֵי בְשָׂמִים}

\berakha{On fragrant fruits:}{הַנּוֹתֵן רֵֽיחַ טוֹב בַּפֵּרוֹת}

\berakha{On balsam oil:}{בּוֹרֵא שֶֽׁמֶן עָרֵב}

\berakha{On other pleasant scents:}{בּוֹרֵא מִינֵי בְשָׂמִים}

\ssubsection{\adforn{18} ברכות הראייה והשמיעה \adforn{17}}

\berakha{On seeing natural phenomena, such as lightning, mountains, or canyons:}{עֹשֶׂה מַעֲשֵׂה בְרֵאשִׁית}

\berakha{On hearing thunder or experiencing very strong winds:}{שֶׁכֹּחוֹ וּגְבוּרָתוֹ מָלֵא עוֹלָם}

\berakha{On seeing a rainbow:}{זוֹכֵר הַבְּרִית וְנֶאֱמָן בִּבְרִיתוֹ וְקַיָּם בְּמַאֲמָרוֹ}

\berakha{On seeing the ocean, for the first time in thirty days:}{שֶׁעָשָׂה אֶת הַיָּם הַגָּדוֹל}

\berakha{On seeing beautiful animals:}{שֶׁכָּֽכָה לּוֹ בְּעוֹלָמוֹ}

\berakha{On seeing trees blooming for the first time in spring:}{שֶׁלֹּא חִסַּר בְּעוֹלָמוֹ דָּבָר, וּבָרָא בוֹ בְּרִיּוֹת טוֹבוֹת וְאִילָנוֹת טוֹבִים לְהַנּוֹת בָּהֶם בְּנֵי אָדָם}

\berakha{On seeing unusual animals:}{מְשַׁנֶּה הַבְּרִיּוֹת}

\berakha{On seeing a monarch:}{שֶׁנָּתַן מִכְּבוֹדוֹ לַבָּשָׂר וָדָם}

\berakha{On seeing a Torah scholar:}{שֶׁחָלַק מֵחׇכְמָתוֹ לִירֵאָיו}

\berakha{On seeing scholars of other subjects:}{שֶׁנָּתַן מֵחׇכְמָתוֹ לְבָשָׂר וָדָם}

\berakha{On news that is good for multiple people:}{הַטוֹב וְהַמֵּטִיב}

\berakha{On good news for an individual:}{שֶׁהֶחֱיָֽנוּ וְקִיְּמָֽנוּ וְהִגִּיעָֽנוּ לַזְּמַן הַזֶּה}

\berakha{On hearing bad news:}{דַּיַּן הָאֱמֶת}

\berakha{On wearing a significant item of new clothing:}{מַלְבִּישׁ עַרֻמִּים}

\ifboolexpr{togl {includeweekday}}{
\ssubsection{\adforn{18} ברכות המצוה \adforn{17}}

\berakhamitzva{On washing hands before eating bread:}{עַל נְטִילַת יָדָיִם}

\berakhamitzva{On ritual immersion:}{עַל הַטְּבִילָה}

\berakhamitzva{On immersion of vessels:}{עַל טְבִילַת כֵּלִים}

\berakhamitzva{On fixing a mezuza to a doorpost:}{לִקְבּֽוֹעַ מְזוּזָה}

\berakhamitzva{On performing she\d{h}ita:}{עַל הַשְּׁחִיטָה}

\berakhamitzva{On covering the blood of she\d{h}ted fowl or deer:}{עַל כִּיסוּי הַדָּם}

\section[תפלת הדרך]{\adforn{47} תפלת הדרך \adforn{19}}

\englishinst{On embarking on a journey:}
יְהִי רָצוֹן מִלְּפָנֶֽיךָ יְיָ אֱלֹהֵֽינוּ וֵאלֹהֵי אֲבוֹתֵֽינוּ שֶׁתּוֹלִיכֵֽנוּ לְשָׁלוֹם וְתַצְעִידֵֽנוּ לְשָׁלוֹם וְתַדְרִיכֵֽנוּ לְשָׁלוֹם׃ וְתַגִּיעֵֽנוּ לִמְחוֹז חֶפְצֵֽנוּ לְחַיִּים וּלְשִׂמְחָה וּלְשָׁלוֹם וְתַצִּילֵֽנוּ מִכַּף כׇּל־אוֹיֵב וְאוֹרֵב בַּדֶּֽרֶךְ וְתִתְּנֵֽנוּ לְחֵן וּלְחֶֽסֶד וּלְרַחֲמִים בְּעֵינֶֽיךָ וּבְעֵינֵי כׇל־רוֹאֵֽנוּ וְתִשְׁמַע קוֹל תַּחֲנוּנֵֽינוּ כִּי אֵל שׁוֹמֵֽעַ תְּפִלָה וְתַחֲנוּן אַֽתָּה׃ בָּרוּךְ אַתָּה יְיָ שׁוֹמֵֽעַ תְּפִלָּה׃}{}
