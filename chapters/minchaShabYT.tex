\vspace{0.2in}
\ifboolexpr{togl {includeshabbat} and togl {includefestival}}{\chapter[מנחה לשבת ויו״ט]{\adforn{47} מנחה לשבת ויו״ט \adforn{19}}}{
	\ifboolexpr{togl {includeshabbat}}{\chapter[מנחה לשבת]{\adforn{47} מנחה לשבת \adforn{19}}}{}
	\ifboolexpr{togl {includefestival}}{\chapter[מנחה ליו״ט]{\adforn{47} מנחה ליו״ט \adforn{19}}}{}}
	
\vspace{0.5in}
\ashrei

\uvaletzion

\halfkaddish

\ifboolexpr{togl {includeshabbat}}{
\ifboolexpr{togl {includefestival}}{\instruction{ביום טוב שאין בשבת אומרים תפילת יום טוב עמ׳ \pageref{shabYTamidah}}}{}

\textbf{
וַאֲנִ֤י תְפִלָּֽתִי־לְךָ֨ ׀ יְיָ֡ עֵ֤ת רָצ֗וֹן אֱלֹהִ֥ים בְּרׇב־חַסְדֶּ֑ךָ עֲ֝נֵ֗נִי בֶּאֱמֶ֥ת יִשְׁעֶֽךָ׃ } \source{תהלים סט}

\section[סדר קריאת התורה]{\adforn{53} סדר קריאת התורה \adforn{25}}


\pesicha

\brikhshmei

\gadlu

\avharachamim

\vesigale

ֺ%\instruction{קריאת התורה בעמוד \pageref{torah}}

\torahbarachu

%\hagomel

\hagbaha

\englishinst{Some chant the following Psalm while the Torah is being wrapped:}

\instruction{יש אומרים בזמן גלילת התורה׃}

\begin{footnotesize}
	\mizmorshabbat
\end{footnotesize}

\yehalelu

\instruction{כשחוזרים הספר תורה לארון:}\\
\kafdalet

\nextpage
\etzchaim

\instruction{סוגרים הארון}}{}
\label{shabYTamidah}
\halfkaddish

\section[תפילת העמידה]{\adforn{53} תפילת העמידה \adforn{25}}


\amidaopening{\shabbosshuva}{\englishinst{During the repetition of the Amidah, Kedusha is said here}}

\weekdaysakedusha

\sepline

\ifboolexpr{togl {includeshabbat}}{
	\ifboolexpr{togl {includefestival}}{\instruction{ביו״ט ממשיכים בעמ׳ \pageref{ytmincha}}}{}

\firstword{אַתָּה}
אֶחָד וְשִׁמְךָ אֶחָד \source{דה״א יז}וּמִי֙ כְּעַמְּךָ֣ יִשְׂרָאֵ֔ל גּ֥וֹי אֶחָ֖ד בָּאָ֑רֶץ תִּפְאֶֽרֶת גְּדֻלָּה וַעֲטֶֽרֶת יְשׁוּעָה, יוֹם מְנוּחָה וּקְדֻשָּׁה לְעַמְּךָ נָתַֽתָּ, אַבְרָהָם יָגֵל יִצְחָק יְרַנֵּן יַעֲקֹב וּבָנָיו יָנֽוּחוּ בוֹ מְנוּחַת אַהֲבָה וּנְדָבָה מְנוּחַת אֱמֶת וֶאֱמוּנָה מְנוּחַת שָׁלוֹם וְשַׁלְוָה וְהַשְׁקֵט וָבֶֽטַח מְנוּחָה שְׁלֵמָה שָׁאַתָּה רוֹצֶה בָּהּ יַכִּֽירוּ בָנֶֽיךָ וְיֵדְעוּ כִּי מֵאִתְּךָ הִיא מְנוּחָתָם וְעַל מְנוּחָתָם יַקְדִּֽישׁוּ אֶת־שְׁמֶֽךָ׃

%\shabboskiddushhayom{\footnote{\instruction{נ״א:} (שַׁבּתוֹת קׇדְשֶׁךָ) וְיָנוּחוּ בָם}} \instruction{רצה וכו׳}
\shabboskiddushhayom{} \ifboolexpr{togl {includefestival}}{\instruction{רצה וכו׳}}{}


\sepline}{}

\ifboolexpr{togl {includefestival}}{\label{ytmincha}
\ytkiddushhayom{}

\ifboolexpr{togl {includeshabbat}}{\sepline}{}
}{}

\retzeh

\yaalehveyavo

\zion

\modim

\ifboolexpr{togl {includeshabbat}}{\shabboschanukah

\shabboshodos

\shabbossimshalom

\tachanunim

\instruction{בימי שאין אומרים תחנון בחול אין אומרים צו״צ בשבת.}\\
צִדְקָתְךָ֣ \source{תהלים קיט}צֶ֣דֶק לְעוֹלָ֑ם וְֽתוֹרָתְךָ֥ אֱמֶֽת׃ \source{תהלים עא}וְצִדְקָתְךָ֥ אֱלֹהִ֗ים עַד־מָ֫ר֥וֹם אֲשֶׁר־עָשִׂ֥יתָ גְדֹל֑וֹת אֱ֝לֹהִ֗ים מִ֣י כָמֽוֹךָ׃ \source{תהלים לו}צִדְקָתְךָ֨ ׀ כְּֽהַרְרֵי־אֵ֗ל מִ֭שְׁפָּטֶיךָ תְּה֣וֹם רַבָּ֑ה אָ֤דָֽם וּבְהֵמָ֖ה תוֹשִׁ֣יעַ יְיָ׃
}{
\simshalomplain

\tachanunim
}

\fullkaddish

\aleinu

\englishinst{At the third meal, if wine is drunk, some recite verses before its blessing:}
\begin{footnotesize}
	וַיֹּ֤אמֶר\source{שמות טז} מֹשֶׁה֙ אִכְלֻ֣הוּ הַיּ֔וֹם כִּֽי־שַׁבָּ֥ת הַיּ֖וֹם לַייָ֑ הַיּ֕וֹם לֹ֥א תִמְצָאֻ֖הוּ בַּשָּׂדֶֽה׃ שֵׁ֥שֶׁת יָמִ֖ים תִּלְקְטֻ֑הוּ וּבַיּ֧וֹם הַשְּׁבִיעִ֛י שַׁבָּ֖ת לֹ֥א יִֽהְיֶה־בּֽוֹ׃ רְא֗וּ כִּֽי־יְיָ נָתַ֣ן לָכֶ֣ם הַשַּׁבָּת֒ עַל־כֵּ֠ן ה֣וּא נֹתֵ֥ן לָכֶ֛ם בַּיּ֥וֹם הַשִּׁשִּׁ֖י לֶ֣חֶם יוֹמָ֑יִם שְׁב֣וּ ׀ אִ֣ישׁ תַּחְתָּ֗יו אַל־יֵ֥צֵא אִ֛ישׁ מִמְּקֹמ֖וֹ בַּיּ֥וֹם הַשְּׁבִיעִֽי׃ וַיִּשְׁבְּת֥וּ הָעָ֖ם בַּיּ֥וֹם הַשְּׁבִעִֽי׃
\end{footnotesize}


עַל־כֵּ֗ן בֵּרַ֧ךְ יְיָ֛ אֶת־י֥וֹם הַשַּׁבָּ֖ת וַֽיְקַדְּשֵֽׁהוּ׃

\savri
\firstword{בָּרוּךְ}
אַתָּה יְיָ אֱלֹהֵֽינוּ מֶֽלֶךְ הָעוֹלָם בּוֹרֵא פְּרִי הַגָֽפֶן׃\\

\adforn{43}\quad\adforn{4}\quad\adforn{42}\\