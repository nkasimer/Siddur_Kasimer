%\chapter[שחרית לחול Mornings Weekday ]{\adforn{47} Blessings its and Shem\ayin \adforn{19}\\קריאת שמע וברכותיה}
\chapter[שחרית לחול]{\adforn{47} קריאת שמע וברכותיה \adforn{19}}

\shacharitinstruction

\barachu

%\textbf{
בָּרוּךְ אַתָּה יְיָ אֱלֹהֵֽינוּ מֶֽלֶךְ הָעוֹלָם \middot יוֹצֵר אוֹר וּבוֹרֵא חֹֽשֶׁךְ עֹשֶׂה שָׁלוֹם וּבוֹרֵא אֶת־הַכֹּל׃
%}



\hameir

\yotzerhameoros

\ahavaraba

\label{morningshema}

\shema

\englishinst{Touch and kiss the tefillin each time they are mentioned.}
\veahavta

\vehaya

\englishinst{Kiss the tzitzi\thav\space each time they are mentioned.}
\vayomer

\englishinst{Join this paragraph with the end of the previous one with the word \hebineng{אמת}. Drop the tzitzi\thav\space at \hebineng{עולמים עלמי לעד}.}
\emesveyatziv

\ezrasavoseinu

\gaalyisroel\\
%\vspace{\baselineskip}

\AMamidainst

\section[תפילת העמידה]{\adforn{53} תפילת העמידה \adforn{25}}
\label{amidahshacharitchol}


\amidaopening{\ayt}{\englishinst{During the repetition of the Amidah, Kedusha is said here}}

\weekdaysakedusha
\sepline

\weekdaysabinah

\weekdaysateshuva

\weekdaysaselichah

\weekdaysageulah

\weekdaysaanneinu

\weekdaysarefuah

\weekdaysaberacha

\weekdaysashofar

\weekdaysamishpat

\weekdaysaminim

\weekdaysatzadikim

\ifboolexpr{togl {minchainshacharit}}{\yerushwithnachem}{\weekdaysayerushelayim}

\weekdaysamalchus

\ifboolexpr{togl {minchainshacharit}}{\weekdaysashemakoleinu{\footnote{
			\englishinst{On a public fast during the silent Amidah, the following is added:}
			%\instruction{בת״צ במנחה היחיד מוסיף׃}
			\\\aneinubasetext}}}{\weekdaysashemakoleinu{}}

\retzeh

\yaalehveyavo

\enlargethispage{\baselineskip}

\zion

\modim

\alhanisim

\weekdaysahodos

%\bircaskohanim{ בחזרת הש״ץ בארץ ישראל אם יש כוהנים, הם מברכים את הקהל׃}{בחזרת הש״ץ בחו״ל או בא״י עם אין כהנים׃} \label{bk}
%\simshalom{\ayt}
\ifboolexpr{togl {minchainshacharit}}{
	\shatzbirkaskohanim{בחזרת הש״ץ בשחרית ומנחה בת״צ׃}
	
	\simshalomrav}{
	\shatzbirkaskohanim{בחזרת הש״ץ׃}
	
	\simshalom{\ayt}}

\tachanunim

\begin{footnotesize}
\sepline

\englishinst{If it is impossible to recite a complete Amidah, recite the first three blessings, the following text, and the last three blessings. This text cannot be used during the rainy season, or at the conclusion of Shabbat or Festivals.}
\havineinu
\instruction{רצה...}

\sepline

\personalfast

\sepline
\end{footnotesize}

\enlargethispage{\baselineskip}

%\instruction{אין אומרים תחנון בבית הכנסת ביום המילה, בבית אבל, כשיש חתן או כלה בבהכ״נ, ראש חודש, חדש ניסן, פסח שני, יום העצמאות, ל״ג בעומר, יום ירושלים, ר״ח סיון עד י״ב סיון, ט׳ באב, ט״ו באב, ערב ראש השנה, מערב יום כפור עד חדש מרחשון, חנוכה, פורים, שושן פורים, פורים קטן, ושושן פורים קטן, ושאר ימי שמחה}
%\instruction{בימים אלו אומרים קדיש עמ׳ \pageref{hatzi_kaddish}}\\
%\instruction{כשאומרים תחנון ביום ב׳ וביום ה׳ אומרים תחנון עמ׳ \pageref{tachanun mon thurs}},
%\instruction{ביום א׳ ג׳ ד׳ ו׳ נופלים על פניהם עמ׳ \pageref{nefilas_apayim}}\\
%\ifboolexpr{togl{includeChM}}{\instruction{בסוכות נוטלים הלולב עמ׳ \pageref{lulav}}\\}{}
%\instruction{ בחול המועד, חנוכה, וראש חודש אומרים הלל עמ׳\pageref{hallel}}\\
%\instruction{ בעשי״ת אומרים אבינו מלכנו עמ׳ \pageref{avinu malkeinu}}\\
%\instruction{ בת״צ אומרים סליחות ואח״כ אבינו מלכנו עמ׳ \pageref{avinu malkeinu}}
\longenginst{Ta\d{h}anun is not recited on the day of a bri\thav\space mila, in a house of mourning, or if a bride or groom is present. It is omitted on Rosh \d{H}odesh, the entire month of Nisan, Pesa\d{h} Sheini, Israeli Independence Day, Lag Ba\bigAyin omer, Jerusalem Reunification Day, from Rosh \d{H}odesh Sivan until the 12th of Sivan, the 9th of Av, the 15th of Av, the day before Rosh HaShana, from the day before Yom Kippur until the beginning of Mar\d{h}eshvan, \d{H}anukka, Purim, Shushan Purim, Purim Katan, Shushan Purim Katan, and other festive occasions.  On all those days, say half-kaddish on page \pageref{hatzi_kaddish}.

\ifboolexpr{togl {includefestival} or togl {includeChM}}{On Sukkot the Lulav is taken on page \pageref{lulav}, followed by Hallel.
}{}
During the Ten Penitential Days, Avinu Malkeinu is recited on page \pageref{avinu_malkeinu}.  On minor fasts, most congregations recite seli\d{h}ot now, followed by Avinu Malkeinu.

\ifboolexpr{togl {includeChM}}{On Intermediate festival days, \d{H}anukka, and Rosh \d{H}odesh, recite Hallel on page \pageref{hallel}.}{On \d{H}anukka and Rosh \d{H}odesh, recite Hallel on page \pageref{hallel}.}

When ta\d{h}anun is recited on Monday and Thursday, continue on page \pageref{tachanun mon thurs}.  On other days, continue on page \pageref{nefilas_apayim}.	
}

\ifboolexpr{not togl{includeshabbat} and not togl{includefestival}}{
%\let\clearpage\relax{
	\ifboolexpr{togl {includefestival} or togl {includeChM}}{
\section[נטילת הלולב]{\adforn{53} נטילת הלולב \adforn{25}}
\label{lulav}

בָּרוּךְ אַתָּה יְיָ אֱלֹהֵינוּ מֶלֶךְ הָעוֹלָם אֲשֶׁר קִדְּשָׁנוּ בְּמִצְוֹתָיו וְצִוָּנוּ עַל נְטִילַת לוּלָב׃

\englishinst{When first taking Lulav, add the following blessing:}
בָּרוּךְ אַתָּה יְיָ אֱלֹהֵינוּ מֶלֶךְ הָעוֹלָם שֶׁהֶחֱיָנוּ וְקִיְּמָנוּ וְהִגִּיעָנוּ לַזְמַן הַזֶּה׃
}{}

\newcommand{\diluginst}{\englishinst{}}
\ifboolexpr{(togl {includeweekday} or togl {includeshabbat}) and not togl {includefestival} and not togl {includeChM}}{
	\renewcommand{\diluginst}{\englishinst{This section is skipped on Rosh \d{H}odesh (except on \d{H}anukka).}}
	\chapter[הלל‎]{\adforn{53} הלל‎ \adforn{25}}
}{}
\ifboolexpr{(togl {includefestival} or togl {includeChM}) and (togl {includeweekday} or togl {includeshabbat})}{
	\renewcommand{\diluginst}{\englishinst{This section is skipped on Rosh \d{H}odesh (except on \d{H}anukka) and the intermediate and final days of Passover.}}
	
\chapter[הלל‎]{\adforn{53} הלל‎ \adforn{25}}
}{}
\ifboolexpr{(togl {includefestival} or togl {includeChM}) and not togl {includeweekday} and not togl {includeshabbat}}{
	\renewcommand{\diluginst}{\englishinst{This section is skipped on the intermediate and final days of Passover.}}
	
	\section[הלל‎]{\adforn{53} הלל‎ \adforn{25}}
}{}

\label{hallel}

%\instruction{החזן אומר הברכה בקול רם, הקהל אומר אמן ואחר כך חוזרים ומברכים:}\\
\firstword{בָּרוּךְ}
אַתָּה יְיָ אֱלֹהֵֽינוּ מֶֽלֶךְ הָעוֹלָם אֲשֶׁר קִדְּשָֽׁנוּ בְּמִצְוֹתָיו וְצִוָּֽנוּ לִקְרֹא אֶת־הַהַלֵּל׃

\firstword{הַ֥לְלוּ יָ֨הּ}\source{תהלים קיג}
׀ הַ֭לְלוּ עַבְדֵ֣י יְיָ֑ הַֽ֝לְל֗וּ אֶת־שֵׁ֥ם יְיָ׃
יְהִ֤י שֵׁ֣ם יְיָ֣ מְבֹרָ֑ךְ מֵ֝עַתָּ֗ה וְעַד־עוֹלָֽם׃
מִמִּזְרַח־שֶׁ֥מֶשׁ עַד־מְבוֹא֑וֹ מְ֝הֻלָּ֗ל שֵׁ֣ם יְיָ׃
רָ֖ם עַל־כׇּל־גּוֹיִ֥ם ׀ יְיָ֑ עַ֖ל הַשָּׁמַ֣יִם כְּבוֹדֽוֹ׃
מִ֭י כַּייָ֣ אֱלֹהֵ֑ינוּ הַֽמַּגְבִּיהִ֥י לָשָֽׁבֶת׃
הַֽמַּשְׁפִּילִ֥י לִרְא֑וֹת בַּשָּׁמַ֥יִם וּבָאָֽרֶץ׃
מְקִ֥ימִ֣י מֵעָפָ֣ר דָּ֑ל מֵ֝אַשְׁפֹּ֗ת יָרִ֥ים אֶבְיֽוֹן׃
לְהוֹשִׁיבִ֥י עִם־נְדִיבִ֑ים עִ֗֝ם נְדִיבֵ֥י עַמּֽוֹ׃
מֽוֹשִׁיבִ֨י ׀ עֲקֶ֬רֶת הַבַּ֗יִת אֵֽם־הַבָּנִ֥ים שְׂמֵחָ֗ה הַֽלְלוּ־יָֽהּ׃

\firstword{בְּצֵ֣את יִ֭שְׂרָאֵל}\source{תהלים קיד}
מִמִּצְרָ֑יִם בֵּ֥ית יַ֝עֲקֹ֗ב מֵעַ֥ם לֹעֵֽז׃
הָיְתָ֣ה יְהוּדָ֣ה לְקׇדְשׁ֑וֹ יִ֝שְׂרָאֵ֗ל מַמְשְׁלוֹתָֽיו׃
הַיָּ֣ם רָ֭אָה וַיָּנֹ֑ס הַ֝יַּרְדֵּ֗ן יִסֹּ֥ב לְאָחֽוֹר׃
הֶ֭הָרִים רָקְד֣וּ כְאֵילִ֑ים גְּ֝בָע֗וֹת כִּבְנֵי־צֹֽאן׃
מַה־לְּךָ֣ הַ֭יָּם כִּ֣י תָנ֑וּס הַ֝יַּרְדֵּ֗ן תִּסֹּ֥ב לְאָחֽוֹר׃
הֶ֭הָרִים תִּרְקְד֣וּ כְאֵילִ֑ים גְּ֝בָע֗וֹת כִּבְנֵי־צֹֽאן׃
מִלִּפְנֵ֣י אָ֭דוֹן ח֣וּלִי אָ֑רֶץ מִ֝לִּפְנֵ֗י אֱל֣וֹהַּ יַעֲקֹֽב׃
הַהֹפְכִ֣י הַצּ֣וּר אֲגַם־מָ֑יִם חַ֝לָּמִ֗ישׁ לְמַעְיְנוֹ־מָֽיִם׃

\diluginst
\begin{narrow}\vspace{-12pt}
	\firstword{לֹ֤א לָ֥נוּ}\source{תהלים קטו}
יְיָ֗ לֹ֫א לָ֥נוּ כִּֽי־לְ֭שִׁמְךָ תֵּ֣ן כָּב֑וֹד עַל־חַ֝סְדְּךָ֗ עַל־אֲמִתֶּֽךָ׃
לָ֭מָּה יֹאמְר֣וּ הַגּוֹיִ֑ם אַיֵּה־נָ֗֝א אֱלֹהֵיהֶֽם׃
וֵאלֹהֵ֥ינוּ בַשָּׁמָ֑יִם כֹּ֖ל אֲשֶׁר־חָפֵ֣ץ עָשָֽׂה׃
עֲֽ֭צַבֵּיהֶם כֶּ֣סֶף וְזָהָ֑ב מַ֝עֲשֵׂ֗ה יְדֵ֣י אָדָֽם׃
פֶּֽה־לָ֭הֶם וְלֹ֣א יְדַבֵּ֑רוּ עֵינַ֥יִם לָ֝הֶ֗ם וְלֹ֣א יִרְאֽוּ׃
אׇזְנַ֣יִם לָ֭הֶם וְלֹ֣א יִשְׁמָ֑עוּ אַ֥ף לָ֝הֶ֗ם וְלֹ֣א יְרִיחֽוּן׃
יְדֵיהֶ֤ם ׀ וְלֹ֬א יְמִישׁ֗וּן רַ֭גְלֵיהֶם וְלֹ֣א יְהַלֵּ֑כוּ לֹא־יֶ֝הְגּ֗וּ בִּגְרוֹנָֽם׃
כְּ֭מוֹהֶם יִהְי֣וּ עֹשֵׂיהֶ֑ם כֹּ֖ל אֲשֶׁר־בֹּטֵ֣חַ בָּהֶֽם׃
יִ֭שְׂרָאֵל בְּטַ֣ח בַּייָ֑ עֶזְרָ֖ם וּמָגִנָּ֣ם הֽוּא׃
בֵּ֣ית אַ֭הֲרֹן בִּטְח֣וּ בַייָ֑ עֶזְרָ֖ם וּמָגִנָּ֣ם הֽוּא׃
יִרְאֵ֣י יְיָ֭ בִּטְח֣וּ בַייָ֑ עֶזְרָ֖ם וּמָגִנָּ֣ם הֽוּא׃
\end{narrow}
\firstword{יְיָ זְכָרָ֢נוּ}
יְ֭בָרֵךְ אֶת־בֵּ֣ית יִשְׂרָאֵ֑ל יְ֝בָרֵ֗ךְ אֶת־בֵּ֥ית אַהֲרֹֽן׃
יְ֭בָרֵךְ יִרְאֵ֣י יְיָ֑ הַ֝קְּטַנִּ֗ים עִם־הַגְּדֹלִֽים׃
יֹסֵ֣ף יְיָ֣ עֲלֵיכֶ֑ם עֲ֝לֵיכֶ֗ם וְעַל־בְּנֵיכֶֽם׃
בְּרוּכִ֣ים אַ֭תֶּם לַייָ֑ עֹ֝שֵׂ֗ה שָׁמַ֥יִם וָאָֽרֶץ׃
הַשָּׁמַ֣יִם שָׁ֭מַיִם לַייָ֑ וְ֝הָאָ֗רֶץ נָתַ֥ן לִבְנֵי־אָדָֽם׃
לֹ֣א הַ֭מֵּתִים יְהַֽלְלוּ־יָ֑הּ וְ֝לֹ֗א כׇּל־יֹרְדֵ֥י דוּמָֽה׃
וַאֲנַ֤חְנוּ ׀ נְבָ֘רֵ֤ךְ יָ֗הּ מֵעַתָּ֥ה וְעַד־עוֹלָ֗ם הַֽלְלוּ־יָֽהּ׃

\diluginst
\begin{narrow}\vspace{-12pt}
	\firstword{אָ֭הַבְתִּי}\source{תהלים קטז}
כִּי־יִשְׁמַ֥ע ׀ יְיָ֑ אֶת־ק֝וֹלִ֗י תַּחֲנוּנָֽי׃
כִּי־הִטָּ֣ה אׇזְנ֣וֹ לִ֑י וּבְיָמַ֥י אֶקְרָֽא׃
אֲפָפ֤וּנִי ׀ חֶבְלֵי־מָ֗וֶת וּמְצָרֵ֣י שְׁא֣וֹל מְצָא֑וּנִי צָרָ֖ה וְיָג֣וֹן אֶמְצָֽא׃
וּבְשֵֽׁם־יְיָ֥ אֶקְרָ֑א אָנָּ֥ה יְ֝יָ֗ מַלְּטָ֥ה נַפְשִֽׁי׃
חַנּ֣וּן יְיָ֣ וְצַדִּ֑יק וֵ֖אלֹהֵ֣ינוּ מְרַחֵֽם׃
שֹׁמֵ֣ר פְּתָאיִ֣ם יְיָ֑ דַּ֝לֹּתִ֗י וְלִ֣י יְהוֹשִֽׁיעַ׃
שׁוּבִ֣י נַ֭פְשִׁי לִמְנוּחָ֑יְכִי כִּֽי־יְ֝יָ֗ גָּמַ֥ל עָלָֽיְכִי׃
כִּ֤י חִלַּ֥צְתָּ נַפְשִׁ֗י מִ֫מָּ֥וֶת אֶת־עֵינִ֥י מִן־דִּמְעָ֑ה אֶת־רַגְלִ֥י מִדֶּֽחִי׃
אֶ֭תְהַלֵּךְ לִפְנֵ֣י יְיָ֑ בְּ֝אַרְצ֗וֹת הַחַיִּֽים׃
הֶ֭אֱמַנְתִּי כִּ֣י אֲדַבֵּ֑ר אֲ֝נִ֗י עָנִ֥יתִי מְאֹֽד׃
אֲ֭נִי אָמַ֣רְתִּי בְחׇפְזִ֑י כׇּֽל־הָאָדָ֥ם כֹּזֵֽב׃
\end{narrow}

\firstword{מָה־אָשִׁ֥יב לַייָ֑}
כׇּֽל־תַּגְמוּל֥וֹהִי עָלָֽי׃
כּוֹס־יְשׁוּע֥וֹת אֶשָּׂ֑א וּבְשֵׁ֖ם יְיָ֣ אֶקְרָֽא׃
נְ֭דָרַי לַייָ֣ אֲשַׁלֵּ֑ם נֶגְדָה־נָּ֗֝א לְכׇל־עַמּֽוֹ׃
יָ֭קָר בְּעֵינֵ֣י יְיָ֑ הַ֝מָּ֗וְתָה לַחֲסִידָֽיו׃
אָנָּ֣ה יְיָ כִּֽי־אֲנִ֢י עַ֫בְדֶּ֥ךָ אֲנִי־עַ֭בְדְּךָ בֶּן־אֲמָתֶ֑ךָ פִּ֝תַּ֗חְתָּ לְמֽוֹסֵרָֽי׃
לְֽךָ־אֶ֭זְבַּח זֶ֣בַח תּוֹדָ֑ה וּבְשֵׁ֖ם יְיָ֣ אֶקְרָֽא׃
נְ֭דָרַי לַייָ֣ אֲשַׁלֵּ֑ם נֶגְדָה־נָּ֗֝א לְכׇל־עַמּֽוֹ׃
בְּחַצְר֤וֹת ׀ בֵּ֤ית יְיָ֗ בְּֽת֘וֹכֵ֤כִי יְֽרוּשָׁלָ֗‍ִם הַֽלְלוּ־יָֽהּ׃

\firstword{הַֽלְל֣וּ}\source{תהלים קיז}
אֶת־יְיָ֭ כׇּל־גּוֹיִ֑ם שַׁ֝בְּח֗וּהוּ כׇּל־הָאֻמִּֽים׃
כִּ֥י גָ֘בַ֤ר עָלֵ֨ינוּ ׀ חַסְדּ֗וֹ וֶאֱמֶת־יְיָ֥ לְעוֹלָ֗ם הַֽלְלוּ־יָֽהּ׃

\shatz \source{תהלים קיח}הוֹד֣וּ לַייָ֣ כִּי־ט֑וֹב כִּ֖י לְעוֹלָ֣ם חַסְדּֽוֹ׃ \hfill \break
\kahal \begin{small}הוֹד֣וּ לַייָ֣ כִּי־ט֑וֹב כִּ֖י לְעוֹלָ֣ם חַסְדּֽוֹ׃ \end{small}\\
\shatz יֹאמַר־נָ֥א יִשְׂרָאֵ֑ל כִּ֖י לְעוֹלָ֣ם חַסְדּֽוֹ׃\hfill \break
\kahal \begin{small}הוֹד֣וּ לַייָ֣ כִּי־ט֑וֹב כִּ֖י לְעוֹלָ֣ם חַסְדּֽוֹ׃ \end{small}\\
\shatz יֹאמְרוּ־נָ֥א בֵֽית־אַהֲרֹ֑ן כִּ֖י לְעוֹלָ֣ם חַסְדּֽוֹ׃ \hfill \break
\kahal \begin{small}הוֹד֣וּ לַייָ֣ כִּי־ט֑וֹב כִּ֖י לְעוֹלָ֣ם חַסְדּֽוֹ׃ \end{small}\\
\shatz יֹאמְרוּ־נָ֭א יִרְאֵ֣י יְיָ֑ כִּ֖י לְעוֹלָ֣ם חַסְדּֽוֹ׃\hfill \break
\kahal \begin{small}הוֹד֣וּ לַייָ֣ כִּי־ט֑וֹב כִּ֖י לְעוֹלָ֣ם חַסְדּֽוֹ׃ \end{small}\\
\firstword{מִֽן־הַ֭מֵּצַר}\source{תהלים קיח}
קָרָ֣אתִי יָּ֑הּ עָנָ֖נִי בַמֶּרְחָ֣ב יָֽהּ׃
יְיָ֣ לִ֭י לֹ֣א אִירָ֑א מַה־יַּעֲשֶׂ֖ה לִ֣י אָדָֽם׃
יְיָ֣ לִ֭י בְּעֹזְרָ֑י וַ֝אֲנִ֗י אֶרְאֶ֥ה בְשֹׂנְאָֽי׃
ט֗וֹב לַחֲס֥וֹת בַּייָ֑ מִ֝בְּטֹ֗חַ בָּאָדָֽם׃
ט֗וֹב לַחֲס֥וֹת בַּייָ֑ מִ֝בְּטֹ֗חַ בִּנְדִיבִֽים׃
כׇּל־גּוֹיִ֥ם סְבָב֑וּנִי בְּשֵׁ֥ם יְ֝יָ֗ כִּ֣י אֲמִילַֽם׃
סַבּ֥וּנִי גַם־סְבָב֑וּנִי בְּשֵׁ֥ם יְ֝יָ֗ כִּ֣י אֲמִילַֽם׃
סַבּ֤וּנִי כִדְבוֹרִ֗ים דֹּ֭עֲכוּ כְּאֵ֣שׁ קוֹצִ֑ים בְּשֵׁ֥ם יְ֝יָ֗ כִּ֣י אֲמִילַֽם׃
דַּחֹ֣ה דְחִיתַ֣נִי לִנְפֹּ֑ל וַ֖ייָ֣ עֲזָרָֽנִי׃
עָזִּ֣י וְזִמְרָ֣ת יָ֑הּ וַֽיְהִי־לִ֗֝י לִישׁוּעָֽה׃
ק֤וֹל ׀ רִנָּ֬ה וִישׁוּעָ֗ה בְּאׇהֳלֵ֥י צַדִּיקִ֑ים יְמִ֥ין יְ֝יָ֗ עֹ֣שָׂה חָֽיִל׃
יְמִ֣ין יְיָ֭ רוֹמֵמָ֑ה יְמִ֥ין יְ֝יָ֗ עֹ֣שָׂה חָֽיִל׃
לֹא־אָמ֥וּת כִּֽי־אֶחְיֶ֑ה וַ֝אֲסַפֵּ֗ר מַעֲשֵׂ֥י יָֽהּ׃
יַסֹּ֣ר יִסְּרַ֣נִּי יָּ֑הּ וְ֝לַמָּ֗וֶת לֹ֣א נְתָנָֽנִי׃
פִּתְחוּ־לִ֥י שַׁעֲרֵי־צֶ֑דֶק אָבֹא־בָ֗֝ם אוֹדֶ֥ה יָֽהּ׃
זֶה־הַשַּׁ֥עַר לַייָ֑ צַ֝דִּיקִ֗ים יָבֹ֥אוּ בֽוֹ׃\\
א֭וֹדְךָ כִּ֣י עֲנִיתָ֑נִי וַתְּהִי־לִ֗֝י לִישׁוּעָֽה׃ \\
\scriptsize{ א֭וֹדְךָ כִּ֣י עֲנִיתָ֑נִי וַתְּהִי־לִ֗֝י לִישׁוּעָֽה׃ \\}\normalsize{}
אֶ֭בֶן מָאֲס֣וּ הַבּוֹנִ֑ים הָ֝יְתָ֗ה לְרֹ֣אשׁ פִּנָּֽה׃ \\
\scriptsize{ אֶ֭בֶן מָאֲס֣וּ הַבּוֹנִ֑ים הָ֝יְתָ֗ה לְרֹ֣אשׁ פִּנָּֽה׃ \\}\normalsize{}
מֵאֵ֣ת יְיָ֭ הָ֣יְתָה זֹּ֑את הִ֖יא נִפְלָ֣את בְּעֵינֵֽינוּ׃ \\
\scriptsize{ מֵאֵ֣ת יְיָ֭ הָ֣יְתָה זֹּ֑את הִ֖יא נִפְלָ֣את בְּעֵינֵֽינוּ׃ \\}\normalsize{}
זֶה־הַ֭יּוֹם עָשָׂ֣ה יְיָ֑ נָגִ֖ילָה וְנִשְׂמְחָ֣ה בֽוֹ׃ \\
\scriptsize{ זֶה־הַ֭יּוֹם עָשָׂ֣ה יְיָ֑ נָגִ֖ילָה וְנִשְׂמְחָ֣ה בֽוֹ׃ } \normalsize{}


\instruction{ש״ץ ואח״כ הקהל׃}\\
אָנָּ֣א יְיָ֭ הוֹשִׁ֘יעָ֥ה נָּ֑א \hfill אָנָּ֣א יְיָ֭ הוֹשִׁ֘יעָ֥ה נָּ֑א\\
אָנָּ֥א יְ֝יָ֗ הַצְלִ֘יחָ֥ה נָּֽא׃ \hfill אָנָּ֥א יְ֝יָ֗ הַצְלִ֘יחָ֥ה נָּֽא׃\\
בָּר֣וּךְ הַ֭בָּא בְּשֵׁ֣ם יְיָ֑ בֵּ֝רַ֥כְנוּכֶ֗ם מִבֵּ֥ית יְיָ׃\\
\scriptsize{בָּר֣וּךְ הַ֭בָּא בְּשֵׁ֣ם יְיָ֑ בֵּ֝רַ֥כְנוּכֶ֗ם מִבֵּ֥ית יְיָ׃}\\
\normalsize{אֵ֤ל ׀ יְיָ וַיָּ֢אֶ֫ר לָ֥נוּ אִסְרוּ־חַ֥ג בַּעֲבֹתִ֑ים עַד־קַ֝רְנ֗וֹת הַמִּזְבֵּֽחַ׃}\\
\scriptsize{אֵ֤ל ׀ יְיָ וַיָּ֢אֶ֫ר לָ֥נוּ אִסְרוּ־חַ֥ג בַּעֲבֹתִ֑ים עַד־קַ֝רְנ֗וֹת הַמִּזְבֵּֽחַ׃}\\
\normalsize{אֵלִ֣י אַתָּ֣ה וְאוֹדֶ֑ךָּ אֱ֝לֹהַ֗י אֲרוֹמְמֶֽךָּ׃}\\
\scriptsize{אֵלִ֣י אַתָּ֣ה וְאוֹדֶ֑ךָּ אֱ֝לֹהַ֗י אֲרוֹמְמֶֽךָּ׃}\\
\normalsize{הוֹד֣וּ לַייָ֣ כִּי־ט֑וֹב כִּ֖י לְעוֹלָ֣ם חַסְדּֽוֹ׃}\\
\scriptsize{הוֹד֣וּ לַייָ֣ כִּי־ט֑וֹב כִּ֖י לְעוֹלָ֣ם חַסְדּֽוֹ׃} \\
\normalsize{}



\negline

\firstword{יְהַלְלֽוּךָ}
יְיָ אֱלֹהֵֽינוּ כׇּל־מַעֲשֶֽׂיךָ וַחֲסִידֶֽיךָ צַדִּיקִים עוֹשֵׂי רְצֹנֶֽךָ וְכׇל־עַמְּךָ בֵּית יִשְׂרָאֵל בְּרִנָּה יוֹדוּ וִיבָרְכוּ וִישַׁבְּחוּ וִיפָאֲרוּ וִירוֹמֲמוּ וְיַעֲרִֽיצוּ וְיַקְדִּֽישׁוּ וְיַמְלִֽיכוּ אֶת־שִׁמְךָ מַלְכֵּֽנוּ כִּי לְךָ טוֹב לְהוֹדוֹת וּלְשִׁמְךָ נָאֶה לְזַמֵּר כִּי מֵעוֹלָם וְעַד עוֹלָם אַתָּה אֵל׃ בָּרוּךְ אַתָּה יְיָ מֶֽלֶךְ מְהֻלָּל בַּתִּשְׁבָּחוֹת׃\\

%\ifboolexpr{togl {includeshabbat} and togl {includeweekday} and not togl {includeChM}}{\englishinst{On Shabbat, continue with Full Kaddish on page \pageref{shacharitShabbatYTtitkabel}. On weekday Rosh \d{H}odesh, continue with Full Kaddish on page \pageref{end of shacharis}, followed by the Psalm of the Day as relevant on page \pageref{shir_shel_yom}. On \d{H}anukka that is not Rosh \d{H}odesh, continue with Half Kaddish on page \pageref{hatzi_kaddish}.}}{}

\englishinst{On Shabbat, Festivals, and Rosh \d{H}odesh, recite Full Kaddish followed by the Psalm of the Day. Then Mourner's Kaddish is read, followed by the Torah service on page \pageref{shabYTtorah} for Shabbat and Festivals and page \pageref{weekday torah} for Intermediate Festival Days and Rosh \d{H}odesh.}

%}
}{}

%\section[אבינו מלכנו Malkeinu Avinu]{\adforn{53} Malkeinu Avinu \adforn{25}\\אבינו מלכנו}
\section[אבינו מלכנו]{\adforn{53} אבינו מלכנו \adforn{25}}
\label{avinu malkeinu}

%\instruction{פותחים הארון}
\englishinst{Avinu Malkeinu is recited standing. The ark is opened.}
\avinumalkeinu

%\instruction{סגורים הארון}\\
\englishinst{The ark is closed.}  

%\section[תחנון Ta\d{h}anun]{\adforn{53} תחנון Ta\d{h}anun \adforn{25}}
\section[תחנון]{\adforn{53} תחנון \adforn{25}}
\label{tachanun mon thurs}

\englishinst{On Monday and Thursday, ta\d{h}anun begins here. This portion of ta\d{h}anun is recited standing.}
\firstword{וְה֤וּא רַח֨וּם}
׀ יְכַפֵּ֥ר\source{תהלים עח} עָוֺן֮ וְֽלֹא־יַֽ֫שְׁחִ֥ית וְ֭הִרְבָּה לְהָשִׁ֣יב אַפּ֑וֹ וְלֹא־יָ֝עִ֗יר כׇּל־חֲמָתֽוֹ׃
אַתָּה יְיָ לֹא תִכְלָא רַחֲמֶיךָ מִמֶּֽנּוּ חַסְדְּךָ וַאֲמִתְּךָ תָּמִיד יִצְּרֽוּנוּ׃
 הוֹשִׁיעֵ֨נוּ\source{תהילים קו} ׀ יְ֘יָ֤ אֱלֹהֵ֗ינוּ וְקַבְּצֵנוּ֮ מִֽן־הַגּ֫וֹיִ֥ם לְ֭הֹדוֹת לְשֵׁ֣ם קׇדְשֶׁ֑ךָ לְ֝הִשְׁתַּבֵּ֗חַ בִּתְהִלָּתֶֽךָ׃
 אִם־עֲוֺנ֥וֹת\source{תהילים קל} תִּשְׁמׇר־יָ֑הּ אֲ֝דֹנָ֗י מִ֣י יַעֲמֹֽד׃ כִּֽי־עִמְּךָ֥ הַסְּלִיחָ֑ה לְ֝מַ֗עַן תִּוָּרֵֽא׃
לֹא כַחֲטָאֵֽינוּ תַּעֲשֶׂה־לָּנוּ וְלֹא כַעֲוֹנוֹתֵֽינוּ תִּגְמוֹל עָלֵינוּ׃
 אִם־עֲוֺנֵ֙ינוּ֙\source{ירמיה יד} עָ֣נוּ בָ֔נוּ יְיָ֕ עֲשֵׂ֖ה לְמַ֣עַן שְׁמֶ֑ךָ זְכֹר־רַחֲמֶ֣יךָ\source{תהילים כה} יְיָ֭ וַחֲסָדֶ֑יךָ כִּ֖י מֵעוֹלָ֣ם הֵֽמָּה׃
יַעֲנֵֽנוּ יְיָ בְּיוֹם צָרָה יְשַׂגְּבֵֽנוּ שֵׁם אֱלֹהֵי יַעֲקֹב׃
 יְיָ֥\source{תהילים כ} הוֹשִׁ֑יעָה הַ֝מֶּ֗לֶךְ יַעֲנֵ֥נוּ בְיוֹם־קׇרְאֵֽנוּ׃
אָבִֽינוּ מַלְכֵּֽנוּ חׇנֵּֽנוּ וַעֲנֵֽנוּ כִּי אֵין בָּֽנוּ מַעֲשִׂים צְדָקָה עֲשֵׂה עִמָּֽנוּ לְמַעַן שְׁמֶךָ׃
אֲדוֹנֵֽינוּ אֱלֹהֵֽינוּ שְׁמַע קוֹל תַּחֲנוּנֵֽינוּ וּזְכׇר־לָנוּ אֶת־בְּרִית אֲבוֹתֵֽינוּ וְהוֹשִׁיעֵֽנוּ לְמַֽעַן שְׁמֶךָ׃
וְעַתָּ֣ה\source{דניאל ט} ׀ אֲדֹנָ֣י אֱלֹהֵ֗ינוּ אֲשֶׁר֩ הוֹצֵ֨אתָ אֶֽת־עַמְּךָ֜ מֵאֶ֤רֶץ מִצְרַ֙יִם֙ בְּיָ֣ד חֲזָקָ֔ה וַתַּֽעַשׂ־לְךָ֥ שֵׁ֖ם כַּיּ֣וֹם הַזֶּ֑ה חָטָ֖אנוּ רָשָֽׁעְנוּ׃ אֲדֹנָ֗י כְּכׇל־צִדְקֹתֶ֙ךָ֙ יָֽשׇׁב־נָ֤א אַפְּךָ֙ וַחֲמָ֣תְךָ֔ מֵעִֽירְךָ֥ יְרוּשָׁלַ֖͏ִם הַר־קׇדְשֶׁ֑ךָ כִּ֤י בַחֲטָאֵ֙ינוּ֙ וּבַעֲוֺנ֣וֹת אֲבֹתֵ֔ינוּ יְרוּשָׁלַ֧͏ִם וְעַמְּךָ֛ לְחֶרְפָּ֖ה לְכׇל־סְבִיבֹתֵֽינוּ׃ וְעַתָּ֣ה ׀ שְׁמַ֣ע אֱלֹהֵ֗ינוּ אֶל־תְּפִלַּ֤ת עַבְדְּךָ֙ וְאֶל־תַּ֣חֲנוּנָ֔יו וְהָאֵ֣ר פָּנֶ֔יךָ עַל־מִקְדָּשְׁךָ֖ הַשָּׁמֵ֑ם לְמַ֖עַן אֲדֹנָֽי׃

הַטֵּ֨ה\source{דניאל ט} אֱלֹהַ֥י ׀ אׇזְנְךָ֮ וּֽשְׁמָע֒ \qk{פְּקַ֣ח}{פקחה} עֵינֶ֗יךָ וּרְאֵה֙ שֹֽׁמְמֹתֵ֔ינוּ וְהָעִ֕יר אֲשֶׁר־נִקְרָ֥א שִׁמְךָ֖ עָלֶ֑יהָ כִּ֣י ׀ לֹ֣א עַל־צִדְקֹתֵ֗ינוּ אֲנַ֨חְנוּ מַפִּילִ֤ים תַּחֲנוּנֵ֙ינוּ֙ לְפָנֶ֔יךָ כִּ֖י עַל־רַחֲמֶ֥יךָ הָרַבִּֽים׃ אֲדֹנָ֤י ׀ שְׁמָ֙עָה֙ אֲדֹנָ֣י ׀ סְלָ֔חָה אֲדֹנָ֛י הַֽקְשִׁ֥יבָה וַעֲשֵׂ֖ה אַל־תְּאַחַ֑ר לְמַֽעַנְךָ֣ אֱלֹהַ֔י כִּֽי־שִׁמְךָ֣ נִקְרָ֔א עַל־עִירְךָ֖ וְעַל־עַמֶּֽךָ׃
אָבִֽינוּ אָב הָרַחֲמָן הַרְאֵֽנוּ אוֹת לְטוֹבָה וְקַבֵּץ נְפוּצוֹתֵֽינוּ מֵאַרְבַּע כַּנְפוֹת הָאָרֶץ יַכִּירוּ וְיֵדְעוּ כׇּל־הַגּוֹיִם כִּי אַתָּה יְיָ אֱלֹהֵֽינוּ׃
וְעַתָּ֥ה\source{ישעיה סד} יְיָ֖ אָבִ֣ינוּ אָ֑תָּה אֲנַ֤חְנוּ הַחֹ֙מֶר֙ וְאַתָּ֣ה יֹצְרֵ֔נוּ וּמַעֲשֵׂ֥ה יָדְךָ֖ כֻּלָּֽנוּ׃ הוֹשִׁיעֵֽנוּ לְמַעַן שְׁמֶךָ צוּרֵֽנוּ מַלְכֵּֽנוּ וְגוֹאֲלֵֽנוּ׃
ח֧וּסָה \source{יואל ב}יְיָ֣ עַל־עַמֶּ֗ךָ וְאַל־תִּתֵּ֨ן נַחֲלָתְךָ֤ לְחֶרְפָּה֙ לִמְשׇׁל־בָּ֣ם גּוֹיִ֔ם לָ֚מָּה יֹאמְר֣וּ בָעַמִּ֔ים אַיֵּ֖ה אֱלֹהֵיהֶֽם׃
יָדַֽעְנוּ כִּי חָטָֽאנוּ וְאֵין מִי יַעֲמֹד בַּעֲדֵֽנוּ שִׁמְךָ הַגָּדוֹל יַעֲמׇד־לָֽנוּ בְּעֵת צָרָה׃ יָדַֽעְנוּ כִּי אֵין בָּֽנוּ מַעֲשִׂים צְדָקָה עֲשֵׂה עִמָּֽנוּ לְמַעַן שְׁמֶֽךָ׃ כְּרַחֵם אָב עַל בָּנִים כֵּן תְּרַחֵם יְיָ עָלֵינוּ וְהוֹשִׁעֵֽנוּ לְמַעַן שְׁמֶךָ׃ חֲמוֹל עַל עַמֶּֽךָ רַחֵם עַל נַחֲלָתֶֽךָ חֽוּסָה נָּא כְּרוֹב רַחֲמֶיךָ׃ חׇנֵּֽנוּ וַעֲנֵֽנוּ כִּי לְךָ יְיָ הַצְּדָקָה עֹשֵׂה נִפְלָאוֹת בְּכׇל־עֵת׃\\
\firstword{הַבֶּט־נָא}
רַחֶם־נָא עַל עַמְּךָ מְהֵרָה לְמַֽעַן שְׁמֶךָ׃ בְּרַחֲמֶֽיךָ הָרַבִּים יְיָ אֱלֹהֵֽינוּ חוּס וְרַחֵם וְהוֹשִֽׁיעָה צֹאן מַרְעִיתֶֽךָ וְאַל יִמְשׇׁל־בָּֽנוּ קֶֽצֶף כִּי לְךָ עֵינֵֽינוּ תְלוּיוֹת׃ הוֹשִׁיעֵֽנוּ לְמַֽעַן שְׁמֶֽךָ רַחֵם עָלֵֽינוּ לְמַֽעַן בְּרִיתֶֽךָ׃ הַבִּֽיטָה וַעֲנֵֽנוּ בְּעֵת צָרָה כִּי לְךָ יְיָ הַיְשׁוּעָה׃ בְּךָ תוֹחַלְתֵּֽנוּ אֱלֽוֹהַּ סְלִיחוֹת אָנָּא סְלַח־נָא אֵל טוֹב וְסַלָח כִּי אֵל מֶֽלֶךְ חַנּוּן וְרַחוּם אַֽתָּה׃\\
\firstword{אָנָּא}
מֶֽלֶךְ חַנּוּן וְרַחוּם זְכוֹר וְהַבֵּט לִבְרִית בֵּין הַבְּתָרִים וְתֵרָאֶה לְפָנֶֽיךָ עֲקֵדַת יָחִיד לְמַעַן יִשְׂרָאֵל׃ אָבִֽינוּ מַלְכֵּֽנוּ חׇנֵּֽנוּ וְעֲנֵֽנוּ כִּי שִׁמְךָ הַגָּדוֹל נִקְרָא עָלֵֽינוּ׃ עֹשֵׂה נִפְלָאוֹת בְּכׇל־עֵת עֲשֵׂה עִמָּֽנוּ כְּחַסְדֶּֽךָ׃ חַנּוּן וְרַחוּם הַבִּֽיטָה וַעֲנֵֽנוּ בְּעֵת צָרָה כִּי לְךָ יְיָ הַיְשׁוּעָה׃ אָבִֽינוּ מַלְכֵּֽנוּ מַחֲסֵֽנוּ אַל תַּֽעַשׂ עִמָּֽנוּ כְּרֹֽעַ מַעֲלָלֵֽינוּ׃ זְכֹר רַחֲמֶֽיךָ יְיָ וְחֲסָדֶֽיךָ וּכְרֹב טוּבְךָ הוֹשִׁיעֵֽנוּ וַחֲמׇל־נָא עָלֵֽינוּ כִּי אֵין לָֽנוּ אֱלֽוֹהַּ אַחֵר מִבַּלְעָדֶיךָ צוּרֵֽנוּ׃ אַל תַּעַזְבֵֽנוּ יְיָ אֱלֹהֵֽינוּ וְאַל תִּרְחַק מִמֶּנּוּ כִּי נַפְשֵֽׁנוּ קְצָרָה מֵחֶֽרֶב וּמִשְּׁבִי וּמִדֶּֽבֶר וּמִמַּגֵּפָה וּמִכׇּל־צָרָה וְיָגוֹן׃ הַצִּילֵֽנוּ כִּי לְךָ קִוִּֽינוּ וְאַל תַּכְלִימֵֽנוּ יְיָ אֱלֹהֵֽינוּ׃ וְהָאֵר פָּנֶֽיךָ בָּֽנוּ וּזְכׇר־לָֽנוּ אֶת־בְּרִית אֲבוֹתֵֽינוּ וְהוֹשִׁיעֵֽנוּ לְמַֽעַן שְׁמֶֽךָ׃ רְאֵה בְצָרוֹתֵֽינוּ וּשְׁמַע קוֹל תְּפִלָּתֵֽנוּ כִּי אַתָּה שׁוֹמֵֽעַ תְּפִלַּת כׇּל־פֶּה׃\\
\firstword{אֵל רַחוּם}
וְחַנּוּן רַחֵם עָלֵֽינוּ וְעַל כׇּל־מַעֲשֶֽׂיךָ כִּי אֵין כָּמֽוֹךָ׃ יְיָ אֱלֹהֵֽינוּ אָנָּא שָׂא־נָא פְשָׁעֵֽינוּ׃ אָבִינוּ מַלְכֵּֽנוּ צוּרֵֽנוּ וְגוֹאֲלֵֽנוּ אֵל חַי וְקַיָּם הַחֲסִין בַּכֹּֽחַ חָסִיד וָטוֹב עַל כׇּל־מַעֲשֶֽׂיךָ כִּי אַתָּה הוּא יְיָ אֱלֹהֵֽינוּ׃ אֵל אֶֽרֶךְ אַפַּֽיִם וּמָלֵא רַחֲמִים עֲשֵׂה עִמָּֽנוּ כְּרֹב רַחֲמֶֽיךָ וְהוֹשִׁיעֵֽנוּ לְמַֽעַן שְׁמֶֽךָ׃ שְׁמַע מַלְכֵּֽנוּ תְּפִלָּתֵֽנוּ וּמִיַד אוֹיְבֵֽינוּ הַצִּילֵֽנוּ׃ שְׁמַע מַלְכֵּֽנוּ תְּפִלָּתֵֽנוּ וּמִכׇּל־צָרָה וְיָגוֹן הַצִּילֵֽנוּ׃ אָבִֽינוּ מַלְכֵּֽנוּ אַֽתָּה \source{ירמיה יד}וְשִׁמְךָ֛ עָלֵ֥ינוּ נִקְרָ֖א אַל־תַּנִּחֵֽנוּ׃
אַל תַּעַזְבֵֽנוּ אָבִֽינוּ וְאַל תִּטְּשֵֽׁנוּ בּוֹרְאֵֽנוּ וְאַל תִּשְׁכָּחֵֽנוּ יוֹצְרֵֽנוּ כִּי אֵל מֶֽלֶךְ חַנּוּן וְרַחוּם אַֽתָּה׃\\
\firstword{אֵין כָּמֽוֹךָ}
חַנּוּן וְרַחוּם יְיָ אֱלֹהֵֽינוּ אֵין כָּמֽוֹךָ אֵל אֶֽרֶךְ אַפַּֽיִם וְרַב חֶֽסֶד וֶאֶמֶת׃ הוֹשִׁיעֵֽנוּ בְּרַחֲמֶֽיךָ הָרַבִּים מֵרַֽעַשׁ וּמֵרֹֽגֶז הַצִּילֵֽנוּ׃ זְכֹר לַעֲבָדֶֽיךָ לְאַבְרָהָם לְיִצְחָק וּלְיַעֲקֹב אַל תֵּֽפֶן אֶל קׇשְׁיֵֽנוּ וְאֶל רִשְׁעֵֽנוּ וְאֶל חַטָּאתֵֽנוּ׃
שׁ֚וּב \source{שמות לב}מֵחֲר֣וֹן אַפֶּ֔ךָ וְהִנָּחֵ֥ם עַל־הָרָעָ֖ה לְעַמֶּֽךָ׃
וְהָסֵר מִמֶּֽנּוּ מַכַּת הַמָּֽוֶת כִּי רַחוּם אַֽתָּה כִּי כֵן דַּרְכֶּֽךָ עֹֽשֶׂה חֶֽסֶד חִנָּם בְּכׇל־דוֹר וָדוֹר׃ חֽוּסָה יְיָ עַל עַמֶּֽךָ וְהַצִּילֵֽנוּ מִזַּעְמֶּֽךָ וְהָסֵר מִמֶּֽנּוּ מַכַּת הַמַּגֵּפָה וּגְזֵרָה קָשָׁה כִּי אַתָּה שׁוֹמֵר יִשְׂרָאֵל׃
לְךָ֤ \source{דניאל ט}אֲדֹנָי֙ הַצְּדָקָ֔ה וְלָ֛נוּ בֹּ֥שֶׁת הַפָּנִ֖ים
מַה נִּתְאוֹנֵן וּמַה נֹּאמַר מַה נְּדַבֵּר וּמַה נִּצְטַדָּק׃ נַחְפְּשָׂה דְרָכֵֽינוּ וְנַחְקֹֽרָה וְנָשֽׁוּבָה אֵלֶֽיךָ כִּי יְמִינְךָ פְשׁוּטָה לְקַבֵּל שָׁבִים׃
אָנָּ֣א \source{תהלים קיח}יְ֭יָ הוֹשִׁ֘יעָ֥ה נָּ֑א אָנָּ֥א יְ֝יָ֗ הַצְלִ֘יחָ֥ה נָּֽא׃
אָנָּא יְיָ עֲנֵֽנוּ בְּיוֹם קׇרְאֵֽנוּ׃ לְךָ יְיָ חִכִּֽינוּ לְךָ יְיָ קִוִּֽינוּ לְךָ יְיָ נְיַחֵל אַל תֶּחֱשֶׁה וּתְעַנֵּֽנוּ כִּי נָאֲמוּ גוֹיִם אָבְדָה תִקְוָתָם כׇּל־בֶּֽרֶךְ וְכׇל־קוֹמָה לְךָ לְבַד תִּשְׁתַּחֲוֶה׃\\
\firstword{הַפּוֹתֵחַ}
יָד בִּתְשׁוּבָה לְקַבֵּל פּוֹשְׁעִים וְחַטָּאִים נִבְהֲלָה נַפְשֵֽׁנוּ מֵרֹב עִצְּבוֹנֵֽנוּ וְאַל תִּשְׁכָּחֵֽנוּ נֶֽצַח׃ קֽוּמָה וְהוֹשִׁיעֵֽנוּ כִּי חָסִֽינוּ בָךְ׃ אָבִֽינוּ מַלְכֵּֽנוּ אִם אֵין בָּֽנוּ צְדָקָה וּמַעֲשִׂים טוֹבִים זְכׇר־לָֽנוּ אֶת־בְּרִית אֲבוֹתֵֽינוּ וְעֵדוֹתֵֽינוּ בְּכׇל־יוֹם יְיָ אֶחָד׃ הַבִּֽיטָה בְעׇנְיֵֽנוּ כִּי רַֽבּוּ מַכְאוֹבֵֽינוּ וְצָרוֹת לְבָבֵֽנוּ׃ חֽוּסָה יְיָ עָלֵֽינוּ בְּאֶֽרֶץ שִׁבְיֵֽנוּ וְאַל תִּשְׁפּוֹךְ חֲרוֹנְךָ עָלֵֽינוּ כִּי אֲנַֽחְנוּ עַמְּךָ בְּנֵי בְרִיתֶֽךָ׃ אֵל הַבִּיטָה דַּל כְּבוֹדֵֽנוּ בַּגּוֹיִם וְשִׁקְּצֽוּנוּ כְּטֻמְאַת הַנִּדָּה׃ עַד מָתַי עֻזְּךָ לַשְּׁבִי וְתִפְאַרְתְּךָ בְּיַד צָר׃ עוֹרְרָה גְבוּרָתְךָ וְקִנְאָתְךָ עַל אוֹיְבֶֽיךָ הֵם יֵבֽוֹשׁוּ וְיֵחַֽתּוּ מִגְּבוּרָתָם וְאַל יִמְעֲטוּ לְפָנֶֽיךָ תְּלָאוֹתֵֽינוּ׃ מַהֵר יְקַדְּמֽוּנוּ רַחֲמֶֽיךָ בְּיוֹם צָרָתֵֽינוּ וְאִם לֹא לְמַעֲנֵֽנוּ לְמַעַנְךָ פְעַל וְאַל תַּשְׁחִית זֵֽכֶר שְׁאֵרִיתֵֽנוּ׃ וְחֹן אֹם הַמְיַחֲדִים שִׁמְךָ פַּעֲמַֽיִם בְּכׇל־יוֹם תָּמִיד בְּאַהֲבָה וְאוֹמְרִים׃
שְׁמַ֖ע \source{דברים ו}יִשְׂרָאֵ֑ל יְיָ֥ אֱלֹהֵ֖ינוּ יְיָ֥ ׀ אֶחָֽד׃

\subsection[נפילת אפים]{\adforn{18} נפילת אפים \adforn{17}}

\englishinst{On Sunday, Tuesday, Wednesday, and Friday, ta\d{h}anun begins here.}
\nefilasapayim \label{nefilas_apayim}

\begin{sometimes}

\instruction{בשני וחמישי:}
\setlength{\LTpost}{0pt}

יְיָ אֱלֹהֵי יִשְׂרָאֵל \source{שמות לב} שׁ֚וּב מֵֽחֲר֣וֹן אַפֶּ֔ךָ וְהִנָּחֵ֥ם עַל־הָֽרָעָ֖ה לְעַמֶּֽךָ׃\\
הַבֵּט מִשָׁמַיִם וּרְאֵה כִּי הָיִינוּ לַעַג וָקֶלֶס בַּגּוֹיִם נֶחְשַׁבְנוּ כְּצֹאן לַטֶּבַח יוּבָל לַהֲרוֹג וּלְאַבֵּד וּלְמַכָּה וּלְחֶרְפָּה׃ וּבְכׇל־זֹאת שִׁמְךָ לֹא שָׁכָחְנוּ נָא אַל תִּשְׁכָּחֵנוּ׃\\
יְיָ אֱלֹהֵי יִשְׂרָאֵל שׁ֚וּב מֵֽחֲר֣וֹן אַפֶּ֔ךָ וְהִנָּחֵ֥ם עַל־הָֽרָעָ֖ה לְעַמֶּֽךָ׃\\
זָרִים אוֹמְרִים אֵין תּוֹחֶלֶת וְתִקְוָה חֹן אֹם לְשִׁמְךָ מְקַוָּה טָהוֹר יְשׁוּעָתֵנוּ קָרְבָה יָגַ֖עְנוּ וְלֹ֥א הֽוּנַֽח־לָֽנוּ רַחֲמֶיךָ יִכְבְּשׁוּ אֶת־כַּעַסְךָ מֵעָלֵינוּ׃ אָנָא שׁוּב מֵחֲרוֹנְךָ וְרַחֵם סְגֻלָּה אֲשֶׁר בָּחָרְתָּ\\
יְיָ אֱלֹהֵי יִשְׂרָאֵל שׁ֚וּב מֵֽחֲר֣וֹן אַפֶּ֔ךָ וְהִנָּחֵ֥ם עַל־הָֽרָעָ֖ה לְעַמֶּֽךָ׃\\
חוּסָה יְיָ עָלֵינוּ בְּרַחֲמֶיךָ וְאַל תִּתְּנֵֽנוּ בִּידֵי אַכְזָרִים׃
לָ֭מָּה יֹֽאמְר֣וּ הַגּוֹיִ֑ם אַיֵּה־נָ֝֗א אֱלֹֽהֵיהֶֽם׃
לְמַעַנְךָ עֲשֵׂה עִמָּנוּ חֶסֶד וְאַל תְּאַחַר׃
אָנָא שׁוּב מֵחֲרוֹנְךָ וְרַחֵם סְגֻלָּה אֲשֶׁר בָּחָֽרְתָּ\\
יְיָ אֱלֹהֵי יִשְׂרָאֵל שׁ֚וּב מֵֽחֲר֣וֹן אַפֶּ֔ךָ וְהִנָּחֵ֥ם עַל־הָֽרָעָ֖ה לְעַמֶּֽךָ׃\\
קוֹלֵנוּ תִשְׁמַע וְתָחֹן וְאַל תִּטְּשֵׁנוּ בְּיַד אוֹיְבֵינוּ לִמְחוֹת אֶת־שְׁמֵנוּ׃
זְכֹר אֲשֶׁר נִשְׁבַּעְתָּ לַאֲבוֹתֵינוּ כְּכוֹכְבֵי הַשָּׁמַיִם אַרְבֶּה אֶת־זַרְעֲכֶם וְעַתָּה נִשְׁאַרְנוּ מְעַט מֵהַרְבֵּה׃
וּבְכׇל־זֹאת שִׁמְךָ לֹא שָׁכָחְנוּ נָא אַל תִּשְׁכָּחֵנוּ׃\\
יְיָ אֱלֹהֵי יִשְׂרָאֵל שׁ֚וּב מֵֽחֲר֣וֹן אַפֶּ֔ךָ וְהִנָּחֵ֥ם עַל־הָֽרָעָ֖ה לְעַמֶּֽךָ׃\\
\source{תהלים עט}
עׇזְרֵ֤נוּ ׀ אֱלֹ֘הֵ֤י יִשְׁעֵ֗נוּ עַֽל־דְּבַ֥ר כְּבֽוֹד־שְׁמֶ֑ךָ וְהַצִּילֵ֥נוּ וְכַפֵּ֥ר עַל־חַ֝טֹּאתֵ֗ינוּ לְמַ֣עַן שְׁמֶֽךָ׃\\
יְיָ אֱלֹהֵי יִשְׂרָאֵל שׁ֚וּב מֵֽחֲר֣וֹן אַפֶּ֔ךָ וְהִנָּחֵ֥ם עַל־הָֽרָעָ֖ה לְעַמֶּֽךָ׃

\end{sometimes}


\enlargethispage{\baselineskip}

\shomeryisroel



\label{hatzi_kaddish}
\halfkaddish

\ifboolexpr{togl {includeChM}}{\englishinst{On Monday, Thursday, Rosh \d{H}odesh, \d{H}anukka, Purim, and Intermediate Festival Days the Torah is read.  On other days, continue with Ashrei on page \pageref{ashrei}.}}{
\englishinst{On Monday, Thursday, Rosh \d{H}odesh, \d{H}anukka, and Purim, the Torah is read.  On other days, continue with Ashrei on page \pageref{ashrei}.}
}

%\instruction{
%ביום ב׳ ה׳, ראש חודש, חנוכה, פורים, וחול המועד קוראים התורה.
%בשאר ימים ממשיכים באשרי עמ׳
%\pageref{ashrei}}

\section[סדר קריאת התורה]{\adforn{53} סדר קריאת התורה \adforn{25}}
%\section[סדר קריאת התורה Torah Service]{\adforn{53} Service Torah \adforn{25}\\ סדר קריאת התורה}
\label{weekday torah}

\ifboolexpr{togl {includeChM}}{\englishinst{The following is not said on Rosh \d{H}odesh, \d{H}anukka, or Intermediate Festival Days.  Most congregations say only one paragraph.}}{
\englishinst{The following is not said on Rosh \d{H}odesh or \d{H}anukka.  Most congregations say only one paragraph.}}

\instruction{אאאא״א בראש חודש, חנוכה, וחול המועד:}\\
\firstword{אֵל אֶֽרֶךְ}
אַפַּֽיִם וְרַב חֶֽסֶד וֶאֱמֶת אַל בְּאַפְּךָ תּוֹכִיחֵֽנוּ: 
ח֧וּסָה יְיָ֣ עַל־עַמֶּ֗ךָ וְהוֹשִׁיעֵֽנוּ מִכָּל־רָע:
 חָטָֽאנוּ לְךָ אָדוֹן סְלַח־נָא כְּרֹב רַחֲמֶֽיךָ אֵל:
 
\firstword{אֵל אֶֽרֶךְ}
אַפַּֽיִם וְרַב חֶֽסֶד וֶאֱמֶת אַל תַּסְתֵּר פָּנֶֽיךָ מִמֶּֽנּוּ׃
ח֧וּסָה יְיָ֣ עַל־יִשְׂרָאֵל עַמֶּֽךָ וְהַצּילֵֽנוּ מִכׇּל־רָע׃
חָטָֽאנוּ לְךָ אָדוֹן סְלַח־נָא כְּרֹב רַחֲמֶֽיךָ אֵל׃



\pesicha

%\brikhshmei

\gadlu

\avharachamim

\vesigale

\torahbarachu

\hagomel


\vspace{\baselineskip}
\begin{small}
	
\misheberakhcholim{}

\misheberakhbaby

%\misheberakhbarmitzva


\end{small}



\sepline

\englishinst{After the Torah reading, Half Kaddish is recited:}
\halfkaddish


\hagbaha
יְהִי רָצוֹן מִלִּפְנֵי אָבִֽינוּ שֶׁבַּשָּׁמַֽיִם לְכוֹנֵן אֶת־בֵּית חַיֵּֽינוּ. וּלְהָשִׁיב אֶת־שְׁכִינָתוֹ בְּתוֹכֵֽנוּ בִּמְהֵרָה בְיָמֵֽינוּ. וְנֹאמַר אָמֵן׃\\
יְהִי רָצוֹן מִלִּפְנֵי אָבִֽינוּ שֶׁבַּשָּׁמַֽיִם לְרַחֵם עָלֵֽינוּ וְעַל פְּלֵיטָתֵֽנוּ. וְלִמְנֹֽעַ מַשְׁחִית וּמַגֵּפָה מֵעָלֵֽינוּ וּמֵעַל כׇּל עַמּוֹ בֵּית יִשְׂרָאֵל. וְנֹאמַר אָמֵן׃\\
יְהִי רָצוֹן מִלִּפְנֵי אָבִֽינוּ שֶׁבַּשָּׁמַֽיִם לְקַיֶּם־בָּֽנוּ חַכְמֵי יִשְׂרָאֵל. הֵם וּמִשְפְּחוֹתֵיהֶם. וְתַלְמִידֵיהֶם וְתַלְמִידֵי תַלְמִידֵיהֶם. בְּכׇל־מְקוֹמוֹת מֹושְׁבוֹתֵיהֶם. וְנֹאמַר אָמֵן׃\\
יְהִי רָצוֹן מִלִּפְנֵי אָבִֽינוּ שֶׁבַּשָּׁמַֽיִם שֶׁנִּשְׁמַע וְנִתְבַּשֵּׂר בְּשׂוֹרוֹת טוֹבוֹת יְשׁוּעוֹת וְנֶחָמוֹת. וִיקַבֵּץ נִדָּחֵֽינוּ מֵאַרְבַּע כַּנְפוֹת הָאָֽרֶץ. וְנֹאמַר אָמֵן׃\\
אַחֵֽינוּ כׇּל בֵּית יִשְׂרָאֵל הַנְּתוּנִים בְּצָּרָה וּבְשִּׁבְיָה. הָעוֹמְדִים בֵּין בַּיָּם וּבֵין בַּיַּבָּשָׁה. הַמָּקוֹם יְרַחֵם עֲלֵיהֶם וְיוֹצִיאֵם מִצָּרָה לִרְוָחָה וּמֵאֲפֵלָה לְאוֹרָה וּמִשִּׁעְבּוּד לִגְאֻלָּה הַשְׁתָּא בַּעֲגָלָא וּבִזְמַן קָרִיב. וְנֹאמַר אָמֵן׃\\
\yehalelu

\englishinst{As the Torah is returned to the ark:}
\kafdalet

\etzchaim
\englishinst{The ark is closed.}

\section[סיום התפילה]{\adforn{53} סיום התפילה \adforn{25}}
%\section[סיום התפילה Prayers Concluding]{\adforn{53} סיום התפילה Prayers Concluding \adforn{25}}
\label{ashrei}
\ashrei

%\instruction{אין אומרים למנצח בראש חודש, חנוכה, י״ד וט״ו אדר א׳, פורים ושושן פורים, ערב פסח, חול המועד, אסרו חג, תשעה באב, ערב יום כפור, ובבית אבל}
\englishinst{The following Psalm is omitted on Rosh \d{H}odesh, \d{H}anukka, Purim and Shushan Purim, Purim Katan and Shushan Purim Katan, the day before Passover and Yom Kippur, Intermediate Festival Days, the day following festivals, the 9th of Av, or in a house of mourning.}
\firstword{לַמְנַצֵּ֗חַ מִזְמ֥וֹר לְדָוִֽד׃}\source{תהלים כ}
יַֽעַנְךָ֣ יְיָ֭ בְּי֣וֹם צָרָ֑ה יְ֝שַׂגֶּבְךָ֗ שֵׁ֤ם ׀ אֱלֹהֵ֬י יַעֲקֹֽב׃
יִשְׁלַֽח־עֶזְרְךָ֥ מִקֹּ֑דֶשׁ וּ֝מִצִּיּ֗וֹן יִסְעָדֶֽךָּ׃
יִזְכֹּ֥ר כׇּל־מִנְחֹתֶ֑ךָ וְעוֹלָתְךָ֖ יְדַשְּׁנֶ֣ה סֶֽלָה׃
יִֽתֶּן־לְךָ֥ כִלְבָבֶ֑ךָ וְֽכׇל־עֲצָתְךָ֥ יְמַלֵּֽא׃
נְרַנְּנָ֤ה ׀ בִּ֘ישׁ֤וּעָתֶ֗ךָ וּבְשֵֽׁם־אֱלֹהֵ֥ינוּ נִדְגֹּ֑ל יְמַלֵּ֥א יְ֝יָ֗ כׇּל־מִשְׁאֲלוֹתֶֽיךָ׃
עַתָּ֤ה יָדַ֗עְתִּי כִּ֤י הוֹשִׁ֥יעַ ׀ יְיָ֗ מְשִׁ֫יח֥וֹ יַ֭עֲנֵהוּ מִשְּׁמֵ֣י קׇדְשׁ֑וֹ בִּ֝גְבֻר֗וֹת יֵ֣שַׁע יְמִינֽוֹ׃
אֵ֣לֶּה בָ֭רֶכֶב וְאֵ֣לֶּה בַסּוּסִ֑ים וַאֲנַ֓חְנוּ ׀ בְּשֵׁם־יְיָ֖ אֱלֹהֵ֣ינוּ נַזְכִּֽיר׃
הֵ֭מָּה כָּרְע֣וּ וְנָפָ֑לוּ וַאֲנַ֥חְנוּ קַּ֝֗מְנוּ וַנִּתְעוֹדָֽד׃
יְיָ֥ הוֹשִׁ֑יעָה הַ֝מֶּ֗לֶךְ יַעֲנֵ֥נוּ בְיוֹם־קׇרְאֵֽנוּ׃

%\instruction{אין אומרים ואני זאת בריתי בבית אבל או בט״ב}
\englishinst{The line \hebineng{וכו׳ בריתי זאת ואני} is omitted in a house of mourning and on the 9th of Av.}
\uvaletzion

%ֺ\instruction{בראש חדש אומרים כאן חצי קדיש ומוסף עמ׳ \pageref{musaphrh}}\\
\englishinst{On Rosh \d{H}odesh, Half-Kaddish is recited followed by Musaf on page \pageref{musaphrh}.}
%\ifboolexpr{togl {includeChM}}{\instruction{בחול המועד אומרים כאן חצי קדיש ומוסף עמ׳ \pageref{musaphregel}}}{}\\
\ifboolexpr{togl {includeChM}}{\englishinst{On Intermediate Festival Days, Half-Kaddish is recited followed by Musaf on page \pageref{musaphregel}.}}{}\\


\label{end of shacharis}
\fullkaddish

\aleinu

\section[שיר של יום]{\adforn{53} שיר של יום \adforn{25}}
%\section[שיר של יום Psalms Daily]{\adforn{53} שיר של יום Psalms Daily \adforn{25}}
\label{shir_shel_yom}


\weekdayshir



%\instruction{באלול תוקעים בשופר, אבל לא בערה״ש:}
\englishinst{During the month of Elul the Shofar is blown here, except on the day before Rosh HaShana.}
תקיעה שברים תרועה תקיעה\\
\ledavid\\

\englishinst{On \d{H}anukka most congregations recite:}
\instruction{בחנכה׃}
\chanukat

\label{kaddish_yasom_shacharis}
\mournerskaddish

\begin{sometimes}

% \instruction{בבית אבל אומרים׃ }
\englishinst{In a house of mourning, on days Ta\d{h}anun is said:}
\source{תהלים מט}\firstword{לַמְנַצֵּ֬חַ ׀ לִבְנֵי־קֹ֬רַח מִזְמֽוֹר׃}
שִׁמְעוּ־זֹ֭את כׇּל־הָעַמִּ֑ים הַ֝אֲזִ֗ינוּ כׇּל־יֹ֥שְׁבֵי חָֽלֶד׃
גַּם־בְּנֵ֣י אָ֭דָם גַּם־בְּנֵי־אִ֑ישׁ יַ֗֝חַד עָשִׁ֥יר וְאֶבְיֽוֹן׃
פִּ֭י יְדַבֵּ֣ר חׇכְמ֑וֹת וְהָג֖וּת לִבִּ֣י תְבוּנֽוֹת׃
אַטֶּ֣ה לְמָשָׁ֣ל אׇזְנִ֑י אֶפְתַּ֥ח בְּ֝כִנּ֗וֹר חִידָתִֽי׃
לָ֣מָּה אִ֭ירָא בִּ֣ימֵי רָ֑ע עֲוֺ֖ן עֲקֵבַ֣י יְסוּבֵּֽנִי׃
הַבֹּטְחִ֥ים עַל־חֵילָ֑ם וּבְרֹ֥ב עׇ֝שְׁרָ֗ם יִתְהַלָּֽלוּ׃
אָ֗ח לֹא־פָדֹ֣ה יִפְדֶּ֣ה אִ֑ישׁ לֹא־יִתֵּ֖ן לֵאלֹהִ֣ים כׇּפְרֽוֹ׃
וְ֭יֵקַר פִּדְי֥וֹן נַפְשָׁ֗ם וְחָדַ֥ל לְעוֹלָֽם׃
וִיחִי־ע֥וֹד לָנֶ֑צַח לֹ֖א יִרְאֶ֣ה הַשָּֽׁחַת׃
כִּ֤י יִרְאֶ֨ה ׀ {{ר0|רווח=לא}}חֲכָ֘מִ֤ים יָמ֗וּתוּ יַ֤חַד כְּסִ֣יל וָבַ֣עַר יֹאבֵ֑דוּ וְעָזְב֖וּ לַאֲחֵרִ֣ים חֵילָֽם׃
קִרְבָּ֤ם בָּתֵּ֨ימוֹ ׀ לְֽעוֹלָ֗ם מִ֭שְׁכְּנֹתָם לְד֣וֹר וָדֹ֑ר קָרְא֥וּ בִ֝שְׁמוֹתָ֗ם עֲלֵ֣י אֲדָמֽוֹת׃
וְאָדָ֣ם בִּ֭יקָר בַּל־יָלִ֑ין נִמְשַׁ֖ל כַּבְּהֵמ֣וֹת נִדְמֽוּ׃
זֶ֣ה דַ֭רְכָּם כֵּ֣סֶל לָ֑מוֹ וְאַחֲרֵיהֶ֓ם ׀ בְּפִיהֶ֖ם יִרְצ֣וּ סֶֽלָה׃
כַּצֹּ֤אן ׀ לִ֥שְׁא֣וֹל שַׁתּוּ֮ מָ֤וֶת יִ֫רְעֵ֥ם וַיִּרְדּ֘וּ בָ֤ם יְשָׁרִ֨ים ׀ לַבֹּ֗קֶר ְ֭צוּרָם לְבַלּ֥וֹת שְׁא֗וֹל מִזְּבֻ֥ל לֽוֹ׃
אַךְ־אֱלֹהִ֗ים יִפְדֶּ֣ה נַ֭פְשִׁי מִֽיַּד־שְׁא֑וֹל כִּ֖י יִקָּחֵ֣נִי סֶֽלָה׃
אַל־תִּ֭ירָא כִּֽי־יַעֲשִׁ֣ר אִ֑ישׁ כִּי־יִ֝רְבֶּ֗ה כְּב֣וֹד בֵּיתֽוֹ׃
כִּ֤י לֹ֣א בְ֭מוֹתוֹ יִקַּ֣ח הַכֹּ֑ל לֹֽא־יֵרֵ֖ד אַחֲרָ֣יו כְּבוֹדֽוֹ׃
כִּֽי־נַ֭פְשׁוֹ בְּחַיָּ֣יו יְבָרֵ֑ךְ וְ֝יוֹדֻ֗ךָ כִּי־תֵיטִ֥יב לָֽךְ׃
תָּ֭בוֹא עַד־דּ֣וֹר אֲבוֹתָ֑יו עַד־נֵ֗֝צַח לֹ֣א יִרְאוּ־אֽוֹר׃
אָדָ֣ם בִּ֭יקָר וְלֹ֣א יָבִ֑ין נִמְשַׁ֖ל כַּבְּהֵמ֣וֹת נִדְמֽוּ׃


\sepline

%\instruction{בבית אבל בימים שאין אומרים תחנון׃}
\englishinst{In a house of mourning, on days Ta\d{h}anun is omitted:}
\source{תהלים טז}\firstword{מִכְתָּ֥ם לְדָוִ֑ד}
שׇֽׁמְרֵ֥נִי אֵ֝֗ל כִּֽי־חָסִ֥יתִי בָֽךְ׃
אָמַ֣רְתְּ לַֽייָ֭ אֲדֹנָ֣י אָ֑תָּה ט֝וֹבָתִ֗י בַּל־עָלֶֽיךָ׃
לִ֭קְדוֹשִׁים אֲשֶׁר־בָּאָ֣רֶץ הֵ֑מָּה וְ֝אַדִּירֵ֗י כׇּל־חֶפְצִי־בָֽם׃
יִרְבּ֥וּ עַצְּבוֹתָם֮ אַחֵ֢ר מָ֫הָ֥רוּ בַּל־אַסִּ֣יךְ נִסְכֵּיהֶ֣ם מִדָּ֑ם וּֽבַל־אֶשָּׂ֥א אֶת־שְׁ֝מוֹתָ֗ם עַל־שְׂפָתָֽי׃
יְיָ֗ מְנָת־חֶלְקִ֥י וְכוֹסִ֑י אַ֝תָּ֗ה תּוֹמִ֥יךְ גּוֹרָלִֽי׃
חֲבָלִ֣ים נָֽפְלוּ־לִ֭י בַּנְּעִמִ֑ים אַף־נַ֝חֲלָ֗ת שָֽׁפְרָ֥ה עָלָֽי׃
אֲבָרֵ֗ךְ אֶת־יְיָ֭ אֲשֶׁ֣ר יְעָצָ֑נִי אַף־לֵ֝יל֗וֹת יִסְּר֥וּנִי כִלְיוֹתָֽי׃
שִׁוִּ֬יתִי יְיָ֣ לְנֶגְדִּ֣י תָמִ֑יד כִּ֥י מִֽ֝ימִינִ֗י בַּל־אֶמּֽוֹט׃
לָכֵ֤ן ׀ שָׂמַ֣ח לִ֭בִּי וַיָּ֣גֶל כְּבוֹדִ֑י אַף־בְּ֝שָׂרִ֗י יִשְׁכֹּ֥ן לָבֶֽטַח׃
כִּ֤י ׀ לֹא־תַעֲזֹ֣ב נַפְשִׁ֣י לִשְׁא֑וֹל לֹֽא־תִתֵּ֥ן חֲ֝סִידְךָ֗ לִרְא֥וֹת שָֽׁחַת׃
תּֽוֹדִיעֵנִי֮ אֹ֤רַח חַ֫יִּ֥ים שֹׂ֣בַע שְׂ֭מָחוֹת אֶת־פָּנֶ֑יךָ נְעִמ֖וֹת בִּימִינְךָ֣ נֶֽצַח׃

\end{sometimes}


%\instruction{בצאתו מבית הכנסת:}\\
%\source{תהלים ה} יְיָ֤ נְחֵ֬נִי בְצִדְקָתֶ֗ךָ לְמַ֥עַן שׁוֹרְרָ֑י הַיְשַׁ֖ר לְפָנַ֣י דַּרְכֶּֽךָ׃

\adforn{43}\quad\adforn{4}\quad\adforn{42}\\

%\chapter[מוסף לראש חודש Months New for Musaf]{\adforn{47} Months New for Musaf \adforn{19}\\מוסף לראש חודש בחול}
\chapter[מוסף לראש חודש]{\adforn{47} מוסף לראש חודש בחול \adforn{19}}
\label{musaphrh}

\specialsaavos

\specialsameisim

\englishinst{During the repetition of the Amidah, Kedusha is said here}
\instruction{בחזרת הש״ץ אומרים קדושה כאן}

\firstword{אַתָּה}
קָדוֹשׁ וְשִׁמְךָ קָדוֹשׁ וּקְדוֹשִׁים בְּכׇל־יוֹם יְהַלְלוּךָ סֶּֽלָה׃ בָּרוּךְ אַתָּה יְיָ הָאֵל הַקָּדוֹשׁ׃\\

\kedusmusafchol{קדושה}{}

\firstword{רָאשֵׁי חֳדָשִׁים}
לְעַמְּךָ נָתַתָּ זְמַן כַּפָּרָה לְכׇל־תּוֹלְדוֹתָם בִּהְיוֹתָם מַקְרִיבִים לְפָנֶֽיךָ זִבְחֵי רָצוֹן וּשְׂעִירֵי חַטָּאת לְכַפֵּר בַּעֲדָם׃ זִכָּרוֹן לְכֻלָּם יִהְיוּ תְּשׁוּעַת נַפְשָׁם מִיַּד שׂוֹנֵא׃ מִזְבֵּֽחַ חָדָשׁ בְּצִיּוֹן תָּכִין וְעוֹלַת רֹאשׁ חֹֽדֶשׁ נַעֲלֶה עָלָיו וּשְׂעִירֵי עִזִּים נַעֲשֶׂה בְרָצוֹן׃ וּבַעֲבוֹדַת בֵּית הַמִּקְדָּשׁ נִשְׂמַח כֻּלָּֽנוּ וְשִׁירֵי דָוִד עַבְדֶּֽךָ נִּשְׁמָעִים בְּעִירֶֽךָ הָאֲמוּרִים לִפְנֵי מִזְבְּחֶֽךָ׃ אַהֲבַת עוֹלָם תָּבִיא לָהֶם וּבְרִית אָבוֹת לַבָּנִים תִּזְכּוֹר׃ וַהֲבִיאֵֽנוּ לְצִיּוֹן עִירְךָ בְּרִנָּה וְלִירוּשָׁלַ‍ִם בֵּית מִקְדָשְׁךָ בְּשִׂמְחַת עוֹלָם׃ וְשָׁם נַעֲשֶׂה לְפָנֶֽיךָ אֶת־קׇרְבְּנוֹת חוֹבוֹתֵֽינוּ תְּמִידִים כְּסִדְרָם וּמוּסָפִים כְּהִלְכָתָם׃

\firstword{וְאֶת מוּסַף}
יוֹם רֹאשׁ הַחֹֽדֶשׁ
הַזֶּה נַעֲשֶׂה וְנַקְרִיב לְפָנֶֽיךָ בְּאַהֲבָה כְּמִצְוַת רְצוֹנֶֽךָ כְּמוֹ שֶׁכָּתַֽבְתָּ עָלֵֽינוּ בְּתוֹרָתֶֽךָ עַל יְדֵי מֹשֶׁה עַבְדְּךָ מִפִּי כְבוֹדֶֽךָ כָּאָמוּר׃\\
\firstword{וּבְרָאשֵׁי֙ חׇדְשֵׁיכֶ֔ם}\source{במדבר כח}
תַּקְרִ֥יבוּ עֹלָ֖ה לַייָ֑ פָּרִ֨ים בְּנֵֽי־בָקָ֤ר שְׁנַ֙יִם֙ וְאַ֣יִל אֶחָ֔ד כְּבָשִׂ֧ים בְּנֵי־שָׁנָ֛ה שִׁבְעָ֖ה תְּמִימִֽם׃ וּמִנְחָתָם וְנִסְכֵּיהֶם כִּמְדֻבָּר שְׁלֹשָׁה עֶשְׂרֹנִים לַפָּר וּשְׁנֵי עֶשְׂרֹנִים לָאָֽיִל וְעִשָּׂרוֹן לַכֶּֽבֶשׂ וְיַֽיִן כְּנִסְכּוֹ וְשָׂעִיר לְכַפֵּר וּשְׁנֵי תְמִידִים כְּהִלְכָתָם׃

\firstword{אֱלֹהֵֽינוּ}
וֵאלֹהֵי אֲבוֹתֵֽינוּ חַדֵּשׁ עָלֵֽינוּ אֶת־הַחֹֽדֶשׁ הַזֶּה לְטוֹבָה וְלִבְרָכָה לְשָׂשׂוֹן וּלְשִׂמְחָה לִישׁוּעָה וּלְנֶחָמָה לְפַרְנָסָה וּלְכַלְכָּלָה לְחַיִּים וּלְשָׁלוֹם לִמְחִֽילַת חֵטְא וְלִסְלִיחַת עָוֹן [\instruction{בשנה העבור עד בכלל ר״ח אדר ב׳:}
וּלְכַפָּרַת פָּֽשַׁע]׃ כִּי בְעַמְּךָ יִשְׂרָאֵל בָּחַֽרְתָּ מִכׇּל־הָאֻמּוֹת וְחֻקֵּי רָאשֵׁי חֳדָשִׁים לָהֶם קָבָֽעְתָּ׃ בָּרוּךְ אַתָּה יְיָ מְקַדֵּשׁ יִשְׂרָאֵל וְרָאשֵׁי חֳדָשִׁים׃

\firstword{רְצֵה}
יְיָ אֱלֹהֵֽינוּ בְּעַמְּךָ יִשְׂרָאֵל וּבִתְפִלָּתָם וְהָשֵׁב הָעֲבוֹדָה לִדְבִיר בֵּיתֶֽךָ׃ וְאִשֵּׁי יִשְׂרָאֵל וּתְפִלָּתָם בְּאַהֲבָה תְקַבֵּל בְּרָצוֹן וּתְהִי לְרָצוֹן תָּמִיד עֲבוֹדַת יִשְׂרָאֵל עַמֶּֽךָ׃ וְתֶחֱזֶֽינָה עֵינֵֽינוּ בְּשׁוּבְךָ לְצִיּוֹן בְּרַחֲמִים׃ בָּרוּךְ אַתָּה יְיָ הַמַּחֲזִיר שְׁכִינָתוֹ לְצִיּוֹן׃

\modim

\enlargethispage{\baselineskip}

\begin{sometimes}

\instruction{בחנוכה:}
\textbf{עַל הַנִּסִּים}
וְעַל הַפֻּרְקָן וְעַל הַגְּבוּרוֹת וְעַל הַתְּשׁוּעוֹת וְעַל הַמִּלְחָמוֹת
שֶׁעָשִֽׂיתָ לַאֲבוֹתֵֽינוּ בַּיָּמִים הָהֵם בַּזְּמַן הַזֶּה׃
בִּימֵי מַתִּתְיָֽהוּ בֶּן יוֹחָנָן כֹּהֵן גָּדוֹל חַשְׁמֹנַי וּבָנָיו כְּשֶׁעָמְדָה מַלְכוּת יָוָן הָרְשָׁעָה עַל עַמְּךָ יִשְׂרָאֵל לְהַשְׁכִּיחָם תּוֹרָתֶֽךָ וּלְהַעֲבִירָם מֵחֻקֵּי רְצוֹנֶֽךָ׃ וְאַתָּה בְּרַחֲמֶֽיךָ הָרַבִּים עָמַֽדְתָּ לָהֶם בְּעֵת צָרָתָם רַֽבְתָּ אֶת־רִיבָם דַּֽנְתָּ אֶת־דִּינָם נָקַֽמְתָּ אֶת־נִקְמָתָם׃ מָסַֽרְתָּ גִּבּוֹרִים בְּיַד חַלָּשִׁים וְרַבִּים בְּיַד מְעַטִּים וּטְמֵאִים בְּיַד טְהוֹרִים וּרְשָׁעִים בְּיַד צַדִּיקִים וְזֵדִים בְּיַד עוֹסְקֵי תוֹרָתֶֽךָ׃ וּלְךָ עָשִֽׂיתָ שֵׁם גָּדוֹל וְקָדוֹשׁ בְּעוֹלָמֶֽךָ וּלְעַמְּךָ יִשְׂרָאֵל עָשִֽׂיתָ תְּשׁוּעָה גְדוֹלָה וּפֻרְקָן כְּהַיּוֹם הַזֶּה׃ וְאַֽחַר כַּךְ בָּֽאוּ בָנֶֽיךָ לִדְבִיר בֵּיתֶֽךָ וּפִנּוּ אֶת־הֵיכָלֶֽךָ וְטִהֲרוּ אֶת־מִקְדָּשֶֽׁךָ וְהִדְלִֽיקוּ נֵרוֹת בְּחַצְרוֹת קׇדְּֿשֶֽׁךָ וְקָבְעוּ שְׁמוֹנַת יְמֵי חֲנֻכָּה אֵֽלּוּ לְהוֹדוֹת לְהַלֵּל לְשִׁמְךָ הַגָּדוֹל׃

\end{sometimes}

\firstword{וְעַל כֻּלָּם}
יִתְבָּרַךְ וְיִתְרוֹמַם שִׁמְךָ מַלְכֵּֽנוּ תָּמִיד לְעוֹלָם וָעֶד׃
וְכׇל־הַחַיִּים יוֹדֽוּךָ סֶּֽלָה וִיהַלְלוּ אֶת־שִׁמְךָ בֶּאֱמֶת הָאֵל יְשׁוּעָתֵֽנוּ וְעֶזְרָתֵֽנוּ סֶֽלָה׃ בָּרוּךְ אַתָּה יְיָ הַטּוֹב שִׁמְךָ וּלְךָ נָאֶה לְהוֹדוֹת׃

\shatzbirkaskohanim\\
\simshalomplain\space\vetov\space
בָּרוּךְ אַתָּה יְיָ הַמְבָרֵךְ אֶת־עַמּוֹ יִשְׂרָאֵל בַּשָּׁלוֹם׃

\tachanunim

\fullkaddish

\aleinu

\mournerskaddish

%\chapter[תפילות נוספות Prayers Other]{\adforn{53} Prayers Other \adforn{25}\\ תפילות נוספות }
\chapter[תפילות נוספות]{\adforn{53} תפילות נוספות \adforn{25}\\}

\section[עשרת הדברות]{עשרת הדברות}

\begin{footnotesize}
	אָֽנֹכִי֙ יְיָ֣ אֱלֹהֶ֔יךָ אֲשֶׁ֧ר הוֹצֵאתִ֛יךָ מֵאֶ֥רֶץ מִצְרַ֖יִם מִבֵּ֣ית עֲבָדִ֑ים לֹֽא־יִהְיֶ֥ה לְךָ֛ אֱלֹהִ֥ים אֲחֵרִ֖ים עַל־פָּנָֽי׃ לֹֽא־תַעֲשֶׂ֨ה לְךָ֥ פֶ֙סֶל֙ וְכׇל־תְּמוּנָ֔ה אֲשֶׁ֤ר בַּשָּׁמַ֙יִם֙ מִמַּ֔עַל וַֽאֲשֶׁ֥ר בָּאָ֖רֶץ מִתָּ֑חַת וַאֲשֶׁ֥ר בַּמַּ֖יִם מִתַּ֥חַת לָאָֽרֶץ׃ לֹֽא־תִשְׁתַּחֲוֶ֥ה לָהֶ֖ם וְלֹ֣א תׇעׇבְדֵ֑ם כִּ֣י אָֽנֹכִ֞י יְיָ֤ אֱלֹהֶ֙יךָ֙ אֵ֣ל קַנָּ֔א פֹּ֠קֵ֠ד עֲוֺ֨ן אָבֹ֧ת עַל־בָּנִ֛ים עַל־שִׁלֵּשִׁ֥ים וְעַל־רִבֵּעִ֖ים לְשֹׂנְאָֽי׃ וְעֹ֥שֶׂה חֶ֖סֶד לַאֲלָפִ֑ים לְאֹהֲבַ֖י וּלְשֹׁמְרֵ֥י מִצְוֺתָֽי׃	לֹ֥א תִשָּׂ֛א אֶת־שֵֽׁם־יְיָ֥ אֱלֹהֶ֖יךָ לַשָּׁ֑וְא כִּ֣י לֹ֤א יְנַקֶּה֙ יְיָ֔ אֵ֛ת אֲשֶׁר־יִשָּׂ֥א אֶת־שְׁמ֖וֹ לַשָּֽׁוְא׃\hfill\break
	זָכ֛וֹר אֶת־י֥וֹם הַשַּׁבָּ֖ת לְקַדְּשֽׁוֹ׃ שֵׁ֤שֶׁת יָמִים֙ תַּֽעֲבֹ֔ד וְעָשִׂ֖יתָ כׇּל־מְלַאכְתֶּֽךָ׃ וְיוֹם֙ הַשְּׁבִיעִ֔י שַׁבָּ֖ת לַייָ֣ אֱלֹהֶ֑יךָ לֹֽא־תַעֲשֶׂ֨ה כׇל־מְלָאכָ֜ה אַתָּ֣ה ׀ וּבִנְךָ֣ וּבִתֶּ֗ךָ עַבְדְּךָ֤ וַאֲמָֽתְךָ֙ וּבְהֶמְתֶּ֔ךָ וְגֵרְךָ֖ אֲשֶׁ֥ר בִּשְׁעָרֶֽיךָ׃ כִּ֣י שֵֽׁשֶׁת־יָמִים֩ עָשָׂ֨ה יְיָ֜ 
	אֶת־הַשָּׁמַ֣יִם וְאֶת־הָאָ֗רֶץ אֶת־הַיָּם֙ וְאֶת־כׇּל־אֲשֶׁר־בָּ֔ם וַיָּ֖נַח בַּיּ֣וֹם הַשְּׁבִיעִ֑י עַל־כֵּ֗ן בֵּרַ֧ךְ יְיָ֛  
	אֶת־י֥וֹם הַשַּׁבָּ֖ת וַֽיְקַדְּשֵֽׁהוּ׃	כַּבֵּ֥ד אֶת־אָבִ֖יךָ וְאֶת־אִמֶּ֑ךָ לְמַ֙עַן֙ יַאֲרִכ֣וּן יָמֶ֔יךָ עַ֚ל הָאֲדָמָ֔ה אֲשֶׁר־יְיָ֥  
	אֱלֹהֶ֖יךָ נֹתֵ֥ן לָֽךְ׃ לֹ֥א תִרְצָ֖ח לֹ֣א תִנְאָ֑ף לֹ֣א תִגְנֹ֔ב לֹֽא־תַעֲנֶ֥ה בְרֵעֲךָ֖ עֵ֥ד שָֽׁקֶר׃ לֹ֥א תַחְמֹ֖ד בֵּ֣ית רֵעֶ֑ךָ לֹֽא־תַחְמֹ֞ד אֵ֣שֶׁת רֵעֶ֗ךָ וְעַבְדּ֤וֹ וַאֲמָתוֹ֙ וְשׁוֹר֣וֹ וַחֲמֹר֔וֹ וְכֹ֖ל אֲשֶׁ֥ר לְרֵעֶֽךָ׃ 
\end{footnotesize}


\section[העקדה]{העקדה}

\begin{footnotesize}
וַיְהִ֗י\source{ברא׳ כב} אַחַר֙ הַדְּבָרִ֣ים הָאֵ֔לֶּה וְהָ֣אֱלֹהִ֔ים נִסָּ֖ה אֶת־אַבְרָהָ֑ם וַיֹּ֣אמֶר אֵלָ֔יו אַבְרָהָ֖ם וַיֹּ֥אמֶר הִנֵּֽנִי׃ וַיֹּ֡אמֶר קַח־נָ֠א אֶת־בִּנְךָ֨ אֶת־יְחִֽידְךָ֤ אֲשֶׁר־אָהַ֙בְתָּ֙ אֶת־יִצְחָ֔ק וְלֶ֨ךְ־לְךָ֔ אֶל־אֶ֖רֶץ הַמֹּרִיָּ֑ה וְהַעֲלֵ֤הוּ שָׁם֙ לְעֹלָ֔ה עַ֚ל אַחַ֣ד הֶֽהָרִ֔ים אֲשֶׁ֖ר אֹמַ֥ר אֵלֶֽיךָ׃ וַיַּשְׁכֵּ֨ם אַבְרָהָ֜ם בַּבֹּ֗קֶר וַֽיַּחֲבֹשׁ֙ אֶת־חֲמֹר֔וֹ וַיִּקַּ֞ח אֶת־שְׁנֵ֤י נְעָרָיו֙ אִתּ֔וֹ וְאֵ֖ת יִצְחָ֣ק בְּנ֑וֹ וַיְבַקַּע֙ עֲצֵ֣י עֹלָ֔ה וַיָּ֣קׇם וַיֵּ֔לֶךְ אֶל־הַמָּק֖וֹם אֲשֶׁר־אָֽמַר־ל֥וֹ הָאֱלֹהִֽים׃ בַּיּ֣וֹם הַשְּׁלִישִׁ֗י וַיִּשָּׂ֨א אַבְרָהָ֧ם אֶת־עֵינָ֛יו וַיַּ֥רְא אֶת־הַמָּק֖וֹם מֵרָחֹֽק׃ וַיֹּ֨אמֶר אַבְרָהָ֜ם אֶל־נְעָרָ֗יו שְׁבוּ־לָכֶ֥ם פֹּה֙ עִֽם־הַחֲמ֔וֹר וַאֲנִ֣י וְהַנַּ֔עַר נֵלְכָ֖ה עַד־כֹּ֑ה וְנִֽשְׁתַּחֲוֶ֖ה וְנָשׁ֥וּבָה אֲלֵיכֶֽם׃ וַיִּקַּ֨ח אַבְרָהָ֜ם אֶת־עֲצֵ֣י הָעֹלָ֗ה וַיָּ֙שֶׂם֙ עַל־יִצְחָ֣ק בְּנ֔וֹ וַיִּקַּ֣ח בְּיָד֔וֹ אֶת־הָאֵ֖שׁ וְאֶת־הַֽמַּאֲכֶ֑לֶת וַיֵּלְכ֥וּ שְׁנֵיהֶ֖ם יַחְדָּֽו׃ וַיֹּ֨אמֶר יִצְחָ֜ק אֶל־אַבְרָהָ֤ם אָבִיו֙ וַיֹּ֣אמֶר אָבִ֔י וַיֹּ֖אמֶר הִנֶּ֣נִּֽי בְנִ֑י וַיֹּ֗אמֶר הִנֵּ֤ה הָאֵשׁ֙ וְהָ֣עֵצִ֔ים וְאַיֵּ֥ה הַשֶּׂ֖ה לְעֹלָֽה׃ וַיֹּ֙אמֶר֙ אַבְרָהָ֔ם אֱלֹהִ֞ים יִרְאֶה־לּ֥וֹ הַשֶּׂ֛ה לְעֹלָ֖ה בְּנִ֑י וַיֵּלְכ֥וּ שְׁנֵיהֶ֖ם יַחְדָּֽו׃ וַיָּבֹ֗אוּ אֶֽל־הַמָּקוֹם֮ אֲשֶׁ֣ר אָֽמַר־ל֣וֹ הָאֱלֹהִים֒ וַיִּ֨בֶן שָׁ֤ם אַבְרָהָם֙ אֶת־הַמִּזְבֵּ֔חַ וַֽיַּעֲרֹ֖ךְ אֶת־הָעֵצִ֑ים וַֽיַּעֲקֹד֙ אֶת־יִצְחָ֣ק בְּנ֔וֹ וַיָּ֤שֶׂם אֹתוֹ֙ עַל־הַמִּזְבֵּ֔חַ מִמַּ֖עַל לָעֵצִֽים׃ וַיִּשְׁלַ֤ח אַבְרָהָם֙ אֶת־יָד֔וֹ וַיִּקַּ֖ח אֶת־הַֽמַּאֲכֶ֑לֶת לִשְׁחֹ֖ט אֶת־בְּנֽוֹ׃ וַיִּקְרָ֨א אֵלָ֜יו מַלְאַ֤ךְ יְיָ֙ מִן־הַשָּׁמַ֔יִם וַיֹּ֖אמֶר אַבְרָהָ֣ם ׀ אַבְרָהָ֑ם וַיֹּ֖אמֶר הִנֵּֽנִי׃ וַיֹּ֗אמֶר אַל־תִּשְׁלַ֤ח יָֽדְךָ֙ אֶל־הַנַּ֔עַר וְאַל־תַּ֥עַשׂ ל֖וֹ מְא֑וּמָה כִּ֣י ׀ עַתָּ֣ה יָדַ֗עְתִּי כִּֽי־יְרֵ֤א אֱלֹהִים֙ אַ֔תָּה וְלֹ֥א חָשַׂ֛כְתָּ אֶת־בִּנְךָ֥ אֶת־יְחִידְךָ֖ מִמֶּֽנִּי׃ וַיִּשָּׂ֨א אַבְרָהָ֜ם אֶת־עֵינָ֗יו וַיַּרְא֙ וְהִנֵּה־אַ֔יִל אַחַ֕ר נֶאֱחַ֥ז בַּסְּבַ֖ךְ בְּקַרְנָ֑יו וַיֵּ֤לֶךְ אַבְרָהָם֙ וַיִּקַּ֣ח אֶת־הָאַ֔יִל וַיַּעֲלֵ֥הוּ לְעֹלָ֖ה תַּ֥חַת בְּנֽוֹ׃ וַיִּקְרָ֧א אַבְרָהָ֛ם שֵֽׁם־הַמָּק֥וֹם הַה֖וּא יְיָ֣ ׀ יִרְאֶ֑ה אֲשֶׁר֙ יֵאָמֵ֣ר הַיּ֔וֹם בְּהַ֥ר יְיָ֖ יֵרָאֶֽה׃ וַיִּקְרָ֛א מַלְאַ֥ךְ יְיָ֖ אֶל־אַבְרָהָ֑ם שֵׁנִ֖ית מִן־הַשָּׁמָֽיִם׃ וַיֹּ֕אמֶר בִּ֥י נִשְׁבַּ֖עְתִּי נְאֻם־יְיָ֑ כִּ֗י יַ֚עַן אֲשֶׁ֤ר עָשִׂ֙יתָ֙ אֶת־הַדָּבָ֣ר הַזֶּ֔ה וְלֹ֥א חָשַׂ֖כְתָּ אֶת־בִּנְךָ֥ אֶת־יְחִידֶֽךָ׃ כִּֽי־בָרֵ֣ךְ אֲבָרֶכְךָ֗ וְהַרְבָּ֨ה אַרְבֶּ֤ה אֶֽת־זַרְעֲךָ֙ כְּכוֹכְבֵ֣י הַשָּׁמַ֔יִם וְכַח֕וֹל אֲשֶׁ֖ר עַל־שְׂפַ֣ת הַיָּ֑ם וְיִרַ֣שׁ זַרְעֲךָ֔ אֵ֖ת שַׁ֥עַר אֹיְבָֽיו׃ וְהִתְבָּרְכ֣וּ בְזַרְעֲךָ֔ כֹּ֖ל גּוֹיֵ֣י הָאָ֑רֶץ עֵ֕קֶב אֲשֶׁ֥ר שָׁמַ֖עְתָּ בְּקֹלִֽי׃ וַיָּ֤שׇׁב אַבְרָהָם֙ אֶל־נְעָרָ֔יו וַיָּקֻ֛מוּ וַיֵּלְכ֥וּ יַחְדָּ֖ו אֶל־בְּאֵ֣ר שָׁ֑בַע וַיֵּ֥שֶׁב אַבְרָהָ֖ם בִּבְאֵ֥ר שָֽׁבַע׃\hfill\break
\end{footnotesize}


\section[תפילת חנה]{תפילת חנה}

\begin{footnotesize}
וַתִּתְפַּלֵּ֤ל\source{ש״ב ב} חַנָּה֙ וַתֹּאמַ֔ר עָלַ֤ץ לִבִּי֙ בַּייָ֔ רָ֥מָה קַרְנִ֖י בַּייָ֑ רָ֤חַב פִּי֙ עַל־א֣וֹיְבַ֔י כִּ֥י שָׂמַ֖חְתִּי בִּישׁוּעָתֶֽךָ׃ אֵין־קָד֥וֹשׁ כַּייָ֖ כִּ֣י אֵ֣ין בִּלְתֶּ֑ךָ וְאֵ֥ין צ֖וּר כֵּאלֹהֵֽינוּ׃ אַל־תַּרְבּ֤וּ תְדַבְּרוּ֙ גְּבֹהָ֣ה גְבֹהָ֔ה יֵצֵ֥א עָתָ֖ק מִפִּיכֶ֑ם כִּ֣י אֵ֤ל דֵּעוֹת֙ יְיָ֔ \qk{וְל֥וֹ}{ולא} נִתְכְּנ֖וּ עֲלִלֽוֹת׃ קֶ֥שֶׁת גִּבֹּרִ֖ים חַתִּ֑ים וְנִכְשָׁלִ֖ים אָ֥זְרוּ חָֽיִל׃ שְׂבֵעִ֤ים בַּלֶּ֙חֶם֙ נִשְׂכָּ֔רוּ וּרְעֵבִ֖ים חָדֵ֑לּוּ עַד־עֲקָרָה֙ יָלְדָ֣ה שִׁבְעָ֔ה וְרַבַּ֥ת בָּנִ֖ים אֻמְלָֽלָה׃ יְיָ֖ מֵמִ֣ית וּמְחַיֶּ֑ה מוֹרִ֥יד שְׁא֖וֹל וַיָּֽעַל׃ יְיָ֖ מוֹרִ֣ישׁ וּמַעֲשִׁ֑יר מַשְׁפִּ֖יל אַף־מְרוֹמֵֽם׃ מֵקִ֨ים מֵעָפָ֜ר דָּ֗ל מֵֽאַשְׁפֹּת֙ יָרִ֣ים אֶבְי֔וֹן לְהוֹשִׁיב֙ עִם־נְדִיבִ֔ים וְכִסֵּ֥א כָב֖וֹד יַנְחִלֵ֑ם כִּ֤י לַֽייָ֙ מְצֻ֣קֵי אֶ֔רֶץ וַיָּ֥שֶׁת עֲלֵיהֶ֖ם תֵּבֵֽל׃ רַגְלֵ֤י חֲסִידָו֙ יִשְׁמֹ֔ר וּרְשָׁעִ֖ים בַּחֹ֣שֶׁךְ יִדָּ֑מּוּ כִּי־לֹ֥א בְכֹ֖חַ יִגְבַּר־אִֽישׁ׃ יְיָ֞ יֵחַ֣תּוּ מְרִיבָ֗ו עָלָו֙ בַּשָּׁמַ֣יִם יַרְעֵ֔ם יְיָ֖ יָדִ֣ין אַפְסֵי־אָ֑רֶץ וְיִתֶּן־עֹ֣ז לְמַלְכּ֔וֹ וְיָרֵ֖ם קֶ֥רֶן מְשִׁיחֽוֹ׃\hfill\break
\end{footnotesize}

\section[פרשת המן]{פרשת המן}

\begin{footnotesize}
	
וַיֹּ֤אמֶר\source{שמות טז} יְיָ֙ אֶל־מֹשֶׁ֔ה הִנְנִ֨י מַמְטִ֥יר לָכֶ֛ם לֶ֖חֶם מִן־הַשָּׁמָ֑יִם וְיָצָ֨א הָעָ֤ם וְלָֽקְטוּ֙ דְּבַר־י֣וֹם בְּיוֹמ֔וֹ לְמַ֧עַן אֲנַסֶּ֛נּוּ הֲיֵלֵ֥ךְ בְּתוֹרָתִ֖י אִם־לֹֽא׃ וְהָיָה֙ בַּיּ֣וֹם הַשִּׁשִּׁ֔י וְהֵכִ֖ינוּ אֵ֣ת אֲשֶׁר־יָבִ֑יאוּ וְהָיָ֣ה מִשְׁנֶ֔ה עַ֥ל אֲשֶֽׁר־יִלְקְט֖וּ י֥וֹם ׀ יֽוֹם׃ וַיֹּ֤אמֶר מֹשֶׁה֙ וְאַהֲרֹ֔ן אֶֽל־כׇּל־בְּנֵ֖י יִשְׂרָאֵ֑ל עֶ֕רֶב וִֽידַעְתֶּ֕ם כִּ֧י יְיָ֛ הוֹצִ֥יא אֶתְכֶ֖ם מֵאֶ֥רֶץ מִצְרָֽיִם׃ וּבֹ֗קֶר וּרְאִיתֶם֙ אֶת־כְּב֣וֹד יְיָ֔ בְּשׇׁמְע֥וֹ אֶת־תְּלֻנֹּתֵיכֶ֖ם עַל־יְיָ֑ וְנַ֣חְנוּ מָ֔ה כִּ֥י \qk{תַלִּ֖ינוּ}{תלונו} עָלֵֽינוּ׃ וַיֹּ֣אמֶר מֹשֶׁ֗ה בְּתֵ֣ת יְיָ֩ לָכֶ֨ם בָּעֶ֜רֶב בָּשָׂ֣ר לֶאֱכֹ֗ל וְלֶ֤חֶם בַּבֹּ֙קֶר֙ לִשְׂבֹּ֔עַ בִּשְׁמֹ֤עַ יְיָ֙ אֶת־תְּלֻנֹּ֣תֵיכֶ֔ם אֲשֶׁר־אַתֶּ֥ם מַלִּינִ֖ם עָלָ֑יו וְנַ֣חְנוּ מָ֔ה לֹא־עָלֵ֥ינוּ תְלֻנֹּתֵיכֶ֖ם כִּ֥י עַל־יְיָ׃ וַיֹּ֤אמֶר מֹשֶׁה֙ אֶֽל־אַהֲרֹ֔ן אֱמֹ֗ר אֶֽל־כׇּל־עֲדַת֙ בְּנֵ֣י יִשְׂרָאֵ֔ל קִרְב֖וּ לִפְנֵ֣י יְיָ֑ כִּ֣י שָׁמַ֔ע אֵ֖ת תְּלֻנֹּתֵיכֶֽם׃ וַיְהִ֗י כְּדַבֵּ֤ר אַהֲרֹן֙ אֶל־כׇּל־עֲדַ֣ת בְּנֵֽי־יִשְׂרָאֵ֔ל וַיִּפְנ֖וּ אֶל־הַמִּדְבָּ֑ר וְהִנֵּה֙ כְּב֣וֹד יְיָ֔ נִרְאָ֖ה בֶּעָנָֽן׃ \hfill\break
וַיְדַבֵּ֥ר יְיָ֖ אֶל־מֹשֶׁ֥ה לֵּאמֹֽר׃ שָׁמַ֗עְתִּי אֶת־תְּלוּנֹּת֮ בְּנֵ֣י יִשְׂרָאֵל֒ דַּבֵּ֨ר אֲלֵהֶ֜ם לֵאמֹ֗ר בֵּ֤ין הָֽעַרְבַּ֙יִם֙ תֹּאכְל֣וּ בָשָׂ֔ר וּבַבֹּ֖קֶר תִּשְׂבְּעוּ־לָ֑חֶם וִֽידַעְתֶּ֕ם כִּ֛י אֲנִ֥י יְיָ֖ אֱלֹהֵיכֶֽם׃ וַיְהִ֣י בָעֶ֔רֶב וַתַּ֣עַל הַשְּׂלָ֔ו וַתְּכַ֖ס אֶת־הַֽמַּחֲנֶ֑ה וּבַבֹּ֗קֶר הָֽיְתָה֙ שִׁכְבַ֣ת הַטַּ֔ל סָבִ֖יב לַֽמַּחֲנֶֽה׃ וַתַּ֖עַל שִׁכְבַ֣ת הַטָּ֑ל וְהִנֵּ֞ה עַל־פְּנֵ֤י הַמִּדְבָּר֙ דַּ֣ק מְחֻסְפָּ֔ס דַּ֥ק כַּכְּפֹ֖ר עַל־הָאָֽרֶץ׃ וַיִּרְא֣וּ בְנֵֽי־יִשְׂרָאֵ֗ל וַיֹּ֨אמְר֜וּ אִ֤ישׁ אֶל־אָחִיו֙ מָ֣ן ה֔וּא כִּ֛י לֹ֥א יָדְע֖וּ מַה־ה֑וּא וַיֹּ֤אמֶר מֹשֶׁה֙ אֲלֵהֶ֔ם ה֣וּא הַלֶּ֔חֶם אֲשֶׁ֨ר נָתַ֧ן יְיָ֛ לָכֶ֖ם לְאׇכְלָֽה׃ זֶ֤ה הַדָּבָר֙ אֲשֶׁ֣ר צִוָּ֣ה יְיָ֔ לִקְט֣וּ מִמֶּ֔נּוּ אִ֖ישׁ לְפִ֣י אׇכְל֑וֹ עֹ֣מֶר לַגֻּלְגֹּ֗לֶת מִסְפַּר֙ נַפְשֹׁ֣תֵיכֶ֔ם אִ֛ישׁ לַאֲשֶׁ֥ר בְּאׇהֳל֖וֹ תִּקָּֽחוּ׃ וַיַּעֲשׂוּ־כֵ֖ן בְּנֵ֣י יִשְׂרָאֵ֑ל וַֽיִּלְקְט֔וּ הַמַּרְבֶּ֖ה וְהַמַּמְעִֽיט׃ וַיָּמֹ֣דּוּ בָעֹ֔מֶר וְלֹ֤א הֶעְדִּיף֙ הַמַּרְבֶּ֔ה וְהַמַּמְעִ֖יט לֹ֣א הֶחְסִ֑יר אִ֥ישׁ לְפִֽי־אׇכְל֖וֹ לָקָֽטוּ׃ וַיֹּ֥אמֶר מֹשֶׁ֖ה אֲלֵהֶ֑ם אִ֕ישׁ אַל־יוֹתֵ֥ר מִמֶּ֖נּוּ עַד־בֹּֽקֶר׃ וְלֹא־שָׁמְע֣וּ אֶל־מֹשֶׁ֗ה וַיּוֹתִ֨רוּ אֲנָשִׁ֤ים מִמֶּ֙נּוּ֙ עַד־בֹּ֔קֶר וַיָּ֥רֻם תּוֹלָעִ֖ים וַיִּבְאַ֑שׁ וַיִּקְצֹ֥ף עֲלֵהֶ֖ם מֹשֶֽׁה׃ וַיִּלְקְט֤וּ אֹתוֹ֙ בַּבֹּ֣קֶר בַּבֹּ֔קֶר אִ֖ישׁ כְּפִ֣י אׇכְל֑וֹ וְחַ֥ם הַשֶּׁ֖מֶשׁ וְנָמָֽס׃ וַיְהִ֣י ׀ בַּיּ֣וֹם הַשִּׁשִּׁ֗י לָֽקְט֥וּ לֶ֙חֶם֙ מִשְׁנֶ֔ה שְׁנֵ֥י הָעֹ֖מֶר לָאֶחָ֑ד וַיָּבֹ֙אוּ֙ כׇּל־נְשִׂיאֵ֣י הָֽעֵדָ֔ה וַיַּגִּ֖ידוּ לְמֹשֶֽׁה׃ וַיֹּ֣אמֶר אֲלֵהֶ֗ם ה֚וּא אֲשֶׁ֣ר דִּבֶּ֣ר יְיָ֔ שַׁבָּת֧וֹן שַׁבַּת־קֹ֛דֶשׁ לַֽייָ֖ מָחָ֑ר אֵ֣ת אֲשֶׁר־תֹּאפ֞וּ אֵפ֗וּ וְאֵ֤ת אֲשֶֽׁר־תְּבַשְּׁלוּ֙ בַּשֵּׁ֔לוּ וְאֵת֙ כׇּל־הָ֣עֹדֵ֔ף הַנִּ֧יחוּ לָכֶ֛ם לְמִשְׁמֶ֖רֶת עַד־הַבֹּֽקֶר׃ וַיַּנִּ֤יחוּ אֹתוֹ֙ עַד־הַבֹּ֔קֶר כַּאֲשֶׁ֖ר צִוָּ֣ה מֹשֶׁ֑ה וְלֹ֣א הִבְאִ֔ישׁ וְרִמָּ֖ה לֹא־הָ֥יְתָה בּֽוֹ׃ וַיֹּ֤אמֶר מֹשֶׁה֙ אִכְלֻ֣הוּ הַיּ֔וֹם כִּֽי־שַׁבָּ֥ת הַיּ֖וֹם לַייָ֑ הַיּ֕וֹם לֹ֥א תִמְצָאֻ֖הוּ בַּשָּׂדֶֽה׃ שֵׁ֥שֶׁת יָמִ֖ים תִּלְקְטֻ֑הוּ וּבַיּ֧וֹם הַשְּׁבִיעִ֛י שַׁבָּ֖ת לֹ֥א יִֽהְיֶה־בּֽוֹ׃ וַֽיְהִי֙ בַּיּ֣וֹם הַשְּׁבִיעִ֔י יָצְא֥וּ מִן־הָעָ֖ם לִלְקֹ֑ט וְלֹ֖א מָצָֽאוּ׃  \hfill וַיֹּ֥אמֶר יְיָ֖ אֶל־מֹשֶׁ֑ה עַד־אָ֙נָה֙ מֵֽאַנְתֶּ֔ם לִשְׁמֹ֥ר מִצְוֺתַ֖י וְתוֹרֹתָֽי׃ רְא֗וּ כִּֽי־יְיָ נָתַ֣ן לָכֶ֣ם הַשַּׁבָּת֒ עַל־כֵּ֠ן ה֣וּא נֹתֵ֥ן לָכֶ֛ם בַּיּ֥וֹם הַשִּׁשִּׁ֖י לֶ֣חֶם יוֹמָ֑יִם שְׁב֣וּ ׀ אִ֣ישׁ תַּחְתָּ֗יו אַל־יֵ֥צֵא אִ֛ישׁ מִמְּקֹמ֖וֹ בַּיּ֥וֹם הַשְּׁבִיעִֽי׃ וַיִּשְׁבְּת֥וּ הָעָ֖ם בַּיּ֥וֹם הַשְּׁבִעִֽי׃ וַיִּקְרְא֧וּ בֵֽית־יִשְׂרָאֵ֛ל אֶת־שְׁמ֖וֹ מָ֑ן וְה֗וּא כְּזֶ֤רַע גַּד֙ לָבָ֔ן וְטַעְמ֖וֹ כְּצַפִּיחִ֥ת בִּדְבָֽשׁ׃ וַיֹּ֣אמֶר מֹשֶׁ֗ה זֶ֤ה הַדָּבָר֙ אֲשֶׁ֣ר צִוָּ֣ה יְיָ֔ מְלֹ֤א הָעֹ֙מֶר֙ מִמֶּ֔נּוּ לְמִשְׁמֶ֖רֶת לְדֹרֹתֵיכֶ֑ם לְמַ֣עַן ׀ יִרְא֣וּ אֶת־הַלֶּ֗חֶם אֲשֶׁ֨ר הֶאֱכַ֤לְתִּי אֶתְכֶם֙ בַּמִּדְבָּ֔ר בְּהוֹצִיאִ֥י אֶתְכֶ֖ם מֵאֶ֥רֶץ מִצְרָֽיִם׃ וַיֹּ֨אמֶר מֹשֶׁ֜ה אֶֽל־אַהֲרֹ֗ן קַ֚ח צִנְצֶ֣נֶת אַחַ֔ת וְתֶן־שָׁ֥מָּה מְלֹֽא־הָעֹ֖מֶר מָ֑ן וְהַנַּ֤ח אֹתוֹ֙ לִפְנֵ֣י יְיָ֔ לְמִשְׁמֶ֖רֶת לְדֹרֹתֵיכֶֽם׃ כַּאֲשֶׁ֛ר צִוָּ֥ה יְיָ֖ אֶל־מֹשֶׁ֑ה וַיַּנִּיחֵ֧הוּ אַהֲרֹ֛ן לִפְנֵ֥י הָעֵדֻ֖ת לְמִשְׁמָֽרֶת׃ וּבְנֵ֣י יִשְׂרָאֵ֗ל אָֽכְל֤וּ אֶת־הַמָּן֙ אַרְבָּעִ֣ים שָׁנָ֔ה עַד־בֹּאָ֖ם אֶל־אֶ֣רֶץ נוֹשָׁ֑בֶת אֶת־הַמָּן֙ אָֽכְל֔וּ עַד־בֹּאָ֕ם אֶל־קְצֵ֖ה אֶ֥רֶץ כְּנָֽעַן׃ וְהָעֹ֕מֶר עֲשִׂרִ֥ית הָאֵיפָ֖ה הֽוּא׃\hfill\break

עֶ֭זְרִי \source{תהילים קכא} מֵעִ֣ם יְיָ֑ עֹ֝שֵׂ֗ה שָׁמַ֥יִם וָאָֽרֶץ׃
הַשְׁלֵ֤ךְ \source{תהילים נה} עַל־יְיָ֨ ׀ יְהָבְךָ֮ וְה֢וּא יְכַ֫לְכְּלֶ֥ךָ לֹא־יִתֵּ֖ן לְעוֹלָ֥ם מ֗וֹט לַצַּדִּֽיק׃
שְׁמׇר־תָּ֭ם \source{תהילים לז} וּרְאֵ֣ה יָשָׁ֑ר כִּֽי־אַחֲרִ֖ית לְאִ֣ישׁ שָׁלֽוֹם: 
בְּטַ֣ח בַּייָ֭ וַעֲשֵׂה־ט֑וֹב שְׁכׇן־אֶ֝֗רֶץ וּרְעֵ֥ה אֱמוּנָֽה:
הִנֵּ֨ה \source{ישעיה יב} אֵ֧ל יְשׁוּעָתִ֛י אֶבְטַ֖ח וְלֹ֣א אֶפְחָ֑ד כִּֽי־עׇזִּ֤י וְזִמְרָת֙ יָ֣הּ יְיָ֔ וַֽיְהִי־לִ֖י לִֽישׁוּעָֽה׃
רִבּוֹנוֹ שֶׁל עוֹלָם בְּדִבְרֵי קָדְשְׁךָ כָּתוּב לֵאמר׃
...הַבּוֹטֵ֥חַ\source{תהילים לב} בַּייָ֑ חֶ֝֗סֶד יְסוֹבְבֶֽנּוּ׃
  וּכְתִיב׃ 
  וְאַתָּ֖ה\source{נחמיה ט} מְחַיֶּ֣ה אֶת־כֻּלָּ֑ם׃
יְיָ אֱלהִים אֱמֶת תֵּן לִי בְּרָכָה וְהַצְלָחָה בְּכׇל־מַעֲשֵׂה יָדַי. כִּי בָטַחְתִּי בְךָ שֶׁעַל יְדֵי מְלַאכְתִּי וּמַשָּׂא וּמַתָּן וַעֲסָקִים שֶׁלִּי תִּשְׁלַח־לִי בְּרָכָה שֶׁאוּכַל לְפַרְנֵס אֶת־עַצְמִי וּבְנֵי בֵיתִי בְּנַחַת וְלא בְצַֽעַר בְּהֶתֵּר וְלא בְאִסּוּר לְחַיִּים וּלְשָׁלוֹם וִיקוּיַם בִּי מִקְרָא שֶׁכָּתוּב׃
  הַשְׁלֵ֤ךְ\source{תהילים נה} עַל־יְיָ֨ ׀ יְהָבְךָ֮ וְה֢וּא יְכַ֫לְכְּלֶ֥ךָ׃
	
\end{footnotesize}

%\section[קדיש דרבנן]{\adforn{53} קדיש דרבנן \adforn{25}}
\section[קדיש דרבנן]
{קדיש דרבנן}
\label{kaddish derabonan}

\englishinst{After learning Torah with a minyan, it is customary to recite either the end of Gemara Berakhot or Avot 6:11 before the Rabbi's Kaddish.}
\sofberakhot

\firstword{רַבִּי חֲנַנְיָה בֶּן עֲקַשְׁיָא אוֹמֵר׃ }\source{אבות ו}
רָצָה הַקָּדוֹשׁ בָּרוּךְ הוּא לְזַכּוֹת אֶת־יִשְׂרָאֵל, לְפִיכָךְ הִרְבָּה לָהֶם תּוֹרָה וּמִצְוֹת, שֶׁנֶּאֱמַר׃
יְיָ \source{ישעיה מב}חָפֵ֖ץ לְמַ֣עַן צִדְק֑וֹ יַגְדִּ֥יל תּוֹרָ֖ה וְיַאְדִּֽיר׃

	\rabbiskaddish


\section[סיום הספר]{\adforn{53} סיום הספר \adforn{25}}

\englishinst{Upon completing a tractate of Gemara, or an order of Mishna, the person learning recites as follows:}
הֲדְרָן עֲלָךְ מַסֶּכֶת/סֵדֶר...וְהֲדְרָךְ עֲלָן, דַּעְתָּן עֲלָךְ מַסֶּכֶת/סֵדֶר...וְדַעְתָּךְ עֲלָן. לָא נִתֽנְשֵׁי מִינָךְ מַסֶּכֶת/סֵדֶר...וְלֹא תִתְנְשֵׁי מִינָן, לָא בְּעָלְמָא הָדֵין וְלֹא בְּעָלְמָא דְאַָתֵי:

יְהִי רָצוֹן מִלְּפָנֶֽיךָ יְיָ אֱלֹהֵֽינוּ וֶאֱלֹהֵי אַבוֹתֵֽינוּ שֶׁתְּהֵא תוֹרָתְךָ אֻמָּנוּתֵֽנוּ בָּעוֹלָם הַזֶּה ותְהֵא עִמָּֽנוּ לָעוֹלָם הַבָּא. חֲנִינָא בַּר פָּפָּא, רָמִי בַּר פָּפָּא, נַחְמָן בַּר פָּפָּא, אַחָאי בַּר פָּפָּא, אַבָּא בַּר פָּפָּא, רַפֽרָם בַּר פָּפָּא, רָכִישׁ בַּר פָּפָּא, סוּרְחָב בַּר פָּפָּא, אַדָּא בַּר פָּפָּא, דָּרוּ בַּר פָּפָּא:

הַעֲרֵב־נָא יְיָ אֱלֹהֵינוּ, אֶת דִּבְרֵי תּוֹרָתְךָ בְּֽפִינוּ וּבְפִיפִיּוֹת עַמְּךָ בֵּית יִשְׂרָאֵל, וְנִהְיֶה אֲנַחְנוּ כֻּֽלָּנוּ וְצֶאֱצָאֵינוּ וְצֶאֱצָאֵי עַמְּךָ בֵּית יִשְׂרָאֵל, כֻּלָּֽנוּ יוֹדְעֵי שְׁמֶךָ וְלוֹמְדֵי תּוֹרָתְךָ לִשְׁמָהּ׃\source{תהלים קיט}%
מֵֽ֭אֹיְבַי תְּחַכְּמֵ֣נִי מִצְוֺתֶ֑ךָ כִּ֖י לְעוֹלָ֣ם הִיא־לִֽי׃ יְהִי־לִבִּ֣י תָמִ֣ים בְּחֻקֶּ֑יךָ לְ֝מַ֗עַן לֹ֣א אֵבֽוֹשׁ׃ לְ֭עוֹלָם לֹא־אֶשְׁכַּ֣ח פִּקּוּדֶ֑יךָ כִּ֥י בָ֗֝ם חִיִּיתָֽנִי׃ בָּר֖וּךְ אַתָּ֥ה יְיָ֗ לַמְּדֵ֥נִי חֻקֶּֽיךָ׃

מוֹדִים אֲנַחְנוּ לְפָנֶֽיךָ יְיָ אֱלֹהֵֽינוּ וֶאֱלֹהֵי אַבוֹתֵֽינוּ שֶׁשַּׂמְתָּ חֶלְקֵֽנוּ מִיּוֹשְׁבֵי בֵּית הַמִּדְרָשׁ, וְלֹא שַׂמְתָּ חֶלְקֵֽנוּ מִיּוֹשְׁבֵי קְרָנוֹת. שֶׁאָנוּ מַשְׁכִּימִים וְהֵם מַשְׁכִּימִים אָנוּ מַשְׁכִּימִים לְדִבְרֵי תּוֹרָה וְהֵם מַשְׁכִּימִים לִדְבָרִים בְּטֵלִים. אָנוּ עֲמֵלִים וְהֵם עֲמֵלִים. אָנוּ עֲמֵלִים וּמְקַבְּלִים שָׂכָר וְהֵם עֲמֵלִים וְאֵינָם מְקַבְּלִים שָׂכָר. אָנוּ רָצִים וְהֵם רָצִים. אָנוּ רָצִים לְחַיֵּי הָעוֹלָם הַבָּא, וְהֵם רָצִים לִבְאֵר שַׁחַת. שֶׁנֱאמַר:\source{תהלים נה}
וְאַתָּ֤ה אֱלֹהִ֨ים ׀ תּוֹרִדֵ֬ם ׀ לִבְאֵ֬ר שַׁ֗חַת אַנְשֵׁ֤י דָמִ֣ים וּ֭מִרְמָה לֹא־יֶחֱצ֣וּ יְמֵיהֶ֑ם וַ֝אֲנִ֗י אֶבְטַח־בָּֽךְ׃

יְהִי רָצוֹן מִלְּפָנֶֽיךָ יְיָ אֱלֹהֵי, כְּשֵׁם שֶׁעֲזַרֽתַּנִי לְסַיֵים מַסֶּכֶת/סֵדֶר..., כֵּן תְּעַזְרֵֽנִי לְהַתְחִיל מְסֶכְתוֹת וּסְפָרִים אַחֵרים וּלְסַיֵימָם, לִלְמֹד וּלְלַמֵּד, לִשְׁמֹר וְלַעֲשׂוֹת וּלְקַיֵּם אֶת כׇּל־דִּבְרֵי תַלְמוּד תּוֹרָתְךָ בְּאַהֲבָה, וּזְכוּת כֹֹּל הַתְנָאִים וְאָמוֹרָאִים וּתַּלְמִידֵי חֲכָמִים יַעֲמוֹד לִי וּלְזַרְעִי שֶׁלֹא תָּמוּש הַתּוֹרָה מִפִּי וּמִפִּי זַרְעִי עַד עוֹלָם. וַיִתְקַיֵים בִּי: בְּהִתְהַלֶּכְךָ תַּנְחֶה אֹתָךְ בְּשׇׁכְבְּךָ תִּשְׁמֹר עָלֶיךָ וַהֲקִיצוֹתָ הִיא תְשִׂיחֶךָ. כִּי־בִי יִרְבּוּ יָמֶיךָ; וְיוֹסִיפוּ לְּךָ, שְׁנוֹת חַיִּים אֹרֶךְ יָמִים בִּימִינָהּ. בִּשְׂמֹאולָהּ עֹשֶׁר וְכָבוֹד׃\source{תהלים כט}
יְיָ֗ עֹ֭ז לְעַמּ֣וֹ יִתֵּ֑ן יְיָ֓ ׀ יְבָרֵ֖ךְ אֶת־עַמּ֣וֹ בַשָּׁלֽוֹם׃

\englishinst{This is followed by Kaddish De'it\d{h}adta:}
\itchadtastart
\englishinst{Continue as in the Rabbi's Kaddish above.}
