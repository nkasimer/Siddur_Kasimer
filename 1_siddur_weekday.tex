%% This work is licensed under a Creative Commons Attribution-ShareAlike 4.0 International License. %%


\documentclass[twoside, openany, parskip=half, 11pt]{book}

\usepackage[paperheight=8.5in,paperwidth=5.5in,top=.7in,bottom=.5in, inner=.875in, outer=.75in, marginparsep=.1in, headsep=16pt]{geometry}
\setlength{\marginparwidth}{.5in}

\usepackage{siddur}

\usepackage[backend=bibtex,style=verbose-trad2]{biblatex}
\bibliography{referenced_texts}

%\usepackage{newclude}

\begin{document}

\title{ \adforn{54} סדור \adforn{26}\\
לשון ישרים
\vspace{.5in}
%\includegraphics[scale=.5]{wolf_shofar_bitmap.png}
}

\author{מסודר ע״י
\\
\textbf{ﭏיעזר בן זאב וואלף קזימיר}}
\date{נוסח אשכנז}

\maketitle

\begin{minipage}{\textwidth}
\begin{english}
\raggedright

©Nathan Kasimer, 20XX (578X). Shared under a Creative Commons Attribution 4.0 license.\\
\textcolor{blue}{https׃//creativecommons.org/licenses/by/4.0/}\\ \vspace{\baselineskip}


Shlomo font created by Shlomo Orbach and Ralph Hancock.\\ \textcolor{blue}{https׃//sites.google.com/site/orlaeinayim/download}\\ \vspace{\baselineskip}

Full \XeLaTeX \quad source and PDF can be found at\\ \textcolor{blue}{link tbd}\\ \vspace{\baselineskip}


\end{english}
\end{minipage}

\begin{minipage}{\textwidth}

\begin{english}
\begin{center} %regular chapter (which I don't want in the TOC anyways) doesn't center.
\begin{LARGE}
Introduction
\end{LARGE}
\end{center}

This siddur is an attempt to make a siddur that reflects the liturgical practice I've adopted from various communities I've lived in--a mish-mash American pan-Eastern Ashkenaz tradition. This siddur is a work in progress. I hope to add a bit more instructions in the future, including instructions in English.  This siddur omits some customary texts which are now recited privately and can be found easily in other texts, such as Pirkei Avot.

The most important acknowledgment for this siddur is to Aaron Wolf, for his \textit{Siddur Olas Tamid}. This is a derivative of his work, with various changes to reflect the mostly Eastern Ashkenazi liturgical tradition I've received. His work, in turn, is largely derived from Rabbi Rallis Weisenthal's work \textit{Siddur Sefas Yisroel}. All biblical quotations are from Miqra Al Pi Masora, a remarkable project whose accurate and free text of Tanakh has helped me a great deal. I am also grateful to the Opensiddur community for supporting opensource liturgical ventures. The individuals behind these project have helped out the world of Jewish text distribution enormously. I would also like to thank Rabbanit Leah Sarna, for both highlighting the need for siddurim to include the text of Hatarat Nedarim for women, and for furnishing me with the text of it used in this siddur.\\

This work would not have been possible without the help of other individuals who assisted me with the technical aspects of creating this siddur. Noah Liebman, Benjamin Epstein, and Steven DuBois helped me learn the Python and \LaTeX{} skills necessary for this work. I would also like to thank my father, for bringing me to shul growing up, and for giving me an appreciation for the Jewish liturgical tradition. And my wife, for supporting my siddur-making interests. I hope this siddur is useful for others to daven from and repurpose.

\end{english}

\end{minipage}

\renewcommand{\contentsname}{}
\tableofcontents

\clearpage

\pagenumbering{arabic}

\setstretch{1.5}

\setcounter{page}{1}

\topskip0pt
\vspace*{\fill}

\thispagestyle{empty}
\begin{Large}
\begin{center}
\begin{tikzpicture}
\draw[-latex,white,postaction={decorate},decoration={text along path,
text={תמאב והארקי רשא לכל ויארק־לכל יי בורק},text align=center}]
(4,0) arc [start angle=180,end angle=0,radius=4];
\end{tikzpicture}
\end{center}
\end{Large}

\newcommand{\mizmorshabbat}{%
מִזְמ֥וֹר\source{תהילים צב} שִׁ֗יר לְי֣וֹם הַשַּׁבָּֽת׃ ט֗וֹב לְהֹד֥וֹת לַייָ֑ וּלְזַמֵּ֖ר לְשִׁמְךָ֣ עֶלְיֽוֹן׃ לְהַגִּ֣יד בַּבֹּ֣קֶר חַסְדֶּ֑ךָ וֶ֝אֱמ֥וּנָתְךָ֗ בַּלֵּילֽוֹת׃ עֲֽלֵי־עָ֭שׂוֹר וַעֲלֵי־נָ֑בֶל עֲלֵ֖י הִגָּי֣וֹן בְּכִנּֽוֹר׃ כִּ֤י שִׂמַּחְתַּ֣נִי יְיָ֣ בְּפׇעֳלֶ֑ךָ בְּֽמַעֲשֵׂ֖י יָדֶ֣יךָ אֲרַנֵּֽן׃ מַה־גָּדְל֣וּ מַעֲשֶׂ֣יךָ יְיָ֑ מְ֝אֹ֗ד עָמְק֥וּ מַחְשְׁבֹתֶֽיךָ׃ אִֽישׁ־בַּ֭עַר לֹ֣א יֵדָ֑ע וּ֝כְסִ֗יל לֹא־יָבִ֥ין אֶת־זֹֽאת׃ בִּפְרֹ֤חַ רְשָׁעִ֨ים ׀ כְּמ֥וֹ־עֵ֗שֶׂב וַ֭יָּצִיצוּ כׇּל־פֹּ֣עֲלֵי אָ֑וֶן לְהִשָּׁמְדָ֥ם עֲדֵי־עַֽד׃ וְאַתָּ֥ה מָר֗וֹם לְעֹלָ֥ם יְיָ׃ כִּ֤י הִנֵּ֪ה אֹיְבֶ֡יךָ ׀ יְיָ֗ כִּֽי־הִנֵּ֣ה אֹיְבֶ֣יךָ יֹאבֵ֑דוּ יִ֝תְפָּרְד֗וּ כׇּל־פֹּ֥עֲלֵי אָֽוֶן׃ וַתָּ֣רֶם כִּרְאֵ֣ים קַרְנִ֑י בַּ֝לֹּתִ֗י בְּשֶׁ֣מֶן רַעֲנָֽן׃ וַתַּבֵּ֥ט עֵינִ֗י בְּשׁ֫וּרָ֥י בַּקָּמִ֖ים עָלַ֥י מְרֵעִ֗ים תִּשְׁמַ֥עְנָה אׇזְנָֽי׃ צַ֭דִּיק כַּתָּמָ֣ר יִפְרָ֑ח כְּאֶ֖רֶז בַּלְּבָנ֣וֹן יִשְׂגֶּֽה׃ שְׁ֭תוּלִים בְּבֵ֣ית יְיָ֑ בְּחַצְר֖וֹת אֱלֹהֵ֣ינוּ יַפְרִֽיחוּ׃ ע֭וֹד יְנוּב֣וּן בְּשֵׂיבָ֑ה דְּשֵׁנִ֖ים וְֽרַעֲנַנִּ֣ים יִהְיֽוּ׃ לְ֭הַגִּיד כִּֽי־יָשָׁ֣ר יְיָ֑ צ֝וּרִ֗י וְֽלֹא־\qk{עַוְלָ֥תָה}{עלתה} בּֽוֹ׃}
\newcommand{\chanukat}{%
מִזְמ֡וֹר\source{תהילים ל} שִׁיר־חֲנֻכַּ֖ת הַבַּ֣יִת לְדָוִֽד׃ אֲרוֹמִמְךָ֣ יְיָ֭ כִּ֣י דִלִּיתָ֑נִי וְלֹֽא־שִׂמַּ֖חְתָּ אֹיְבַ֣י לִֽי׃ יְיָ֥ אֱלֹהָ֑י שִׁוַּ֥עְתִּי אֵ֝לֶ֗יךָ וַתִּרְפָּאֵֽנִי׃ יְיָ֗ הֶעֱלִ֣יתָ מִן־שְׁא֣וֹל נַפְשִׁ֑י חִ֝יִּיתַ֗נִי \qk{מִיׇּֽרְדִי־}{מיורדי}בֽוֹר׃ זַמְּר֣וּ לַייָ֣ חֲסִידָ֑יו וְ֝הוֹד֗וּ לְזֵ֣כֶר קׇדְשֽׁוֹ׃ כִּ֤י רֶ֨גַע ׀ בְּאַפּוֹ֮ חַיִּ֢ים בִּרְצ֫וֹנ֥וֹ בָּ֭עֶרֶב יָלִ֥ין בֶּ֗כִי וְלַבֹּ֥קֶר רִנָּֽה׃ וַ֭אֲנִי אָמַ֣רְתִּי בְשַׁלְוִ֑י בַּל־אֶמּ֥וֹט לְעוֹלָֽם׃ יְיָ֗ בִּרְצוֹנְךָ֮ הֶעֱמַ֢דְתָּה לְֽהַרְרִ֫י־עֹ֥ז הִסְתַּ֥רְתָּ פָנֶ֗יךָ הָיִ֥יתִי נִבְהָֽל׃ אֵלֶ֣יךָ יְיָ֣ אֶקְרָ֑א וְאֶל־אֲ֝דֹנָ֗י אֶתְחַנָּֽן׃ מַה־בֶּ֥צַע בְּדָמִי֮ בְּרִדְתִּ֢י אֶ֫ל־שָׁ֥חַת הֲיוֹדְךָ֥ עָפָ֑ר הֲיַגִּ֥יד אֲמִתֶּֽךָ׃ שְׁמַע־יְיָ֥ וְחׇנֵּ֑נִי יְ֝יָ֗ הֱֽיֵה־עֹזֵ֥ר לִֽי׃ הָפַ֣כְתָּ מִסְפְּדִי֮ לְמָח֢וֹל לִ֥י פִּתַּ֥חְתָּ שַׂקִּ֑י וַֽתְּאַזְּרֵ֥נִי שִׂמְחָֽה׃ לְמַ֤עַן ׀ יְזַמֶּרְךָ֣ כָ֭בוֹד וְלֹ֣א יִדֹּ֑ם יְיָ֥ אֱ֝לֹהַ֗י לְעוֹלָ֥ם אוֹדֶֽךָּ׃}
\newcommand{\lamenatzeachbinginot}{%
לַמְנַצֵּ֥חַ\source{תהילים סז} בִּנְגִינֹ֗ת מִזְמ֥וֹר שִֽׁיר׃ אֱֽלֹהִ֗ים יְחׇנֵּ֥נוּ וִיבָרְכֵ֑נוּ יָ֤אֵֽר פָּנָ֖יו אִתָּ֣נוּ סֶֽלָה׃ לָדַ֣עַת בָּאָ֣רֶץ דַּרְכֶּ֑ךָ בְּכׇל־גּ֝וֹיִ֗ם יְשׁוּעָתֶֽךָ׃ יוֹד֖וּךָ עַמִּ֥ים ׀ אֱלֹהִ֑ים י֝וֹד֗וּךָ עַמִּ֥ים כֻּלָּֽם׃ יִ֥שְׂמְח֥וּ וִירַנְּנ֗וּ לְאֻ֫מִּ֥ים כִּֽי־תִשְׁפֹּ֣ט עַמִּ֣ים מִישֹׁ֑ר וּלְאֻמִּ֓ים ׀ בָּאָ֖רֶץ תַּנְחֵ֣ם סֶֽלָה׃ יוֹד֖וּךָ עַמִּ֥ים ׀ אֱלֹהִ֑ים י֝וֹד֗וּךָ עַמִּ֥ים כֻּלָּֽם׃ אֶ֭רֶץ נָתְנָ֣ה יְבוּלָ֑הּ יְ֝בָרְכֵ֗נוּ אֱלֹהִ֥ים אֱלֹהֵֽינוּ׃ יְבָרְכֵ֥נוּ אֱלֹהִ֑ים וְיִֽירְא֥וּ א֝וֹת֗וֹ כׇּל־אַפְסֵי־אָֽרֶץ׃}

\newcommand{\kaddishamein}{(\nolinebreak\kahal\textbf{אָמֵן})\quad}

\newcommand{\kaddishtitle}[1]{\vspace{0.6\baselineskip}\instruction{#1}\nopagebreak\vspace{-0.6\baselineskip}\nopagebreak}

\newcommand{\kaddishstart}{

יִתְגַּדַּל וְיִתְקַדַּשׁ שְׁמֵיהּ רַבָּא \kaddishamein
בְּעָלְמָא דִּי בְרָא כִרְעוּתֵהּ וְיַמְלִיךְ מַלְכוּתֵהּ בְּחַיֵּיכוֹן וּבְיוֹמֵיכוֹן וּבְחַיֵּי דְכׇל־בֵּית יִשְׂרָאֵל בַּעֲגָלָא וּבִזְמַן קָרִיב׃ וְאִמְרוּ אָמֵן׃\\
\textbf{\kahal
אָמֵן׃ יְהֵא שְׁמֵיהּ רַבָּא מְבָרַךְ לְעָלַם וּלְעָלְמֵי עָלְמַיָּא׃}\\
יִתְבָּרַךְ וְיִשְׁתַּבַּח וְיִתְפָּאַר וְיִתְרוֹמַם וְיִתְנַשֵּׂא וְיִתְהַדַּר וְיִתְעַלֶּה וְיִתְהַלַּל שְׁמֵיהּ דְּקֻדְשָׁא בְּרִיךְ הוּא׃ *לְעֵֽלָּא מִן כׇּל־בִּרְכָתָא
(*\instruction{בעשי״ת:}
לְעֵֽלָּא לְעֵֽלָּא מִכׇּל־בִּרְכָתָא) וְשִׁירָתָא תֻּשְׁבְּחָתָא וְנֶחָמָתָא דַּאֲמִירָן בְּעָלְמָא וְאִמְרוּ אָמֵן׃
\kaddishamein
}

\newcommand{\osehshalom}{
עֹשֶׂה שָׁלוֹם בִּמְרוֹמָיו הוּא יַעֲשֶׂה שָׁלוֹם עָלֵֽינוּ וְעַל כׇּל־יִשְׂרָאֵל׃ וְאִמְרוּ אָמֵן׃
%עֹשֶׂה *שָׁלוֹם (*\instruction{בעשי״ת:} הַשָּׁלוֹם)
}

\newcommand{\halfkaddish}{
\kaddishtitle{חצי קדיש}
\footnotesize{\kaddishstart}
\normalsize
}

\newcommand{\fullkaddish}{\footnotesize{
\kaddishtitle{קדיש תתקבל}
\kaddishstart
% (\kahal
% \begin{footnotesize}
% קַבֵּל בְּרַחֲמִים וּבְרָצוֹן אֶת־תְּפִלָּתֵֽינוּ׃)\\
% \end{footnotesize}
תִּתְקַבַּל צְלוֹתְהוֹן וּבָעוּתְהוֹן דְּכׇל־יִשְׂרָאֵל קֳדָם אֲבוּהוֹן דִּי בִשְׁמַיָּא׃ וְאִמְרוּ אָמֵן׃
\kaddishamein
% (\kahal
% \begin{footnotesize}
%יְהִ֤י
% \source{תהלים קיג}%
% שֵׁ֣ם יְיָ֣ מְבֹרָ֑ךְ מֵֽ֝עַתָּ֗ה וְעַד־עוֹלָֽם׃)\\
% \end{footnotesize}
יְהֵא שְׁלָמָא רַבָּא מִן שְׁמַיָּא וְחַיִּים עָלֵֽינוּ וְעַל כׇּל־יִשְׂרָאֵל׃ וְאִמְרוּ אָמֵן׃
\kaddishamein
% (\kahal
% \begin{footnotesize}
% עֶ֭זְרִי
% \source{תהלים קכא}%
%מֵעִ֣ם יְיָ֑ עֹ֝שֵׂ֗ה שָׁמַ֥יִם וָאָֽרֶץ׃) \\
% \end{footnotesize}
\osehshalom
\kaddishamein
}\normalsize}

\newcommand{\mournerskaddish}{\footnotesize{
\kaddishtitle{קדיש יתום}
\kaddishstart
יְהֵא שְׁלָמָא רַבָּא מִן שְׁמַיָּא וְחַיִּים עָלֵֽינוּ וְעַל כׇּל־יִשְׂרָאֵל׃ וְאִמְרוּ אָמֵן׃
\kaddishamein
\osehshalom
\kaddishamein
\\}\normalsize
}

\newcommand{\alrabbanan}{
עַל יִשְׂרָאֵל וְעַל רַבָּנָן וְעַל תַּלְמִידֵיהוֹן וְעַל כׇּל־תַלְמִידֵי תַלְמִידֵיהוֹן וְעַל כׇּל־מָאן דְּעָסְקִין בְּאוֹרַיְתָא דִּי בְאַתְרָא הָדֵן וְדִי בְּכׇל־אֲתַר וַאֲתַר. יְהֵא לְהוֹן וּלְכוֹן שְׁלָמָא רַבָּא חִנָּא וְחִסְדָּא וְרַחֲמִין וְחַיִּין אֲרִיכִין וּמְזוֹנֵי רְוִיחֵי וּפֻרְקָנָא מִן־קֳדָם אֲבוּהוׂן דִּי בִשְׁמַיָּא׃ וְאִמְרוּ אָמֵן׃}

\newcommand{\rabbiskaddish}{\footnotesize{
\instruction{קדיש דרבנן}
\kaddishstart
\alrabbanan
\kaddishamein
יְהֵא שְׁלָמָא רַבָּא מִן שְׁמַיָּא וְחַיִּים עָלֵֽינוּ וְעַל כׇּל־יִשְׂרָאֵל׃ וְאִמְרוּ אָמֵן׃
\kaddishamein
\osehshalom
\kaddishamein
\\}\normalsize}

\newcommand{\itchadtastart}{
יִתְגַּדַּל וְיִתְקַדַּשׁ שְׁמֵיהּ רַבָּא 
\kaddishamein
בְּעָלְמָא דִּי הוּא עָתִיד לְאִתְחַדְתָּא \middot וּלְאַחֲיָאה מֵתַיָּא \middot וּלְאַסָּקָא יַתְּהוֹן לְחַיֵּי עָלְמָא \middot וּלְמִבְנָא קַרְתָּא דִּי יְרוּשְלֵם \middot וּלְשַׁכְלְלָא הֵיכָלֵהּ בְּגַוָּהּ \middot וּלְמֶעֱקַר פּוּלְחָנָא נוּכְרָאָה מִן אַרְעָה \middot וּלְאָתָבָא פּוּלְחָנָא דִּי שְׁמַיָּא לְאַתְרָהּ \middot וְיַמְלִיךְ קוּדְשָׁא בְּרִיךְ הוּא בּמַלְכוּתֵהּ וִיקָרֵהּ בְּחַיֵּיכוֹן וּבְיוֹמֵיכוֹן וּבְחַיֵּי דְכׇל־בֵּית יִשְׂרָאֵל בַּעֲגָלָא וּבִזְמַן קָרִיב׃ וְאִמְרוּ אָמֵן׃\\
}

\newcommand{\kaddishitchadeta}{
\itchadtastart
\textbf{\kahal
אָמֵן׃ יְהֵא שְׁמֵיהּ רַבָּא מְבָרַךְ לְעָלַם וּלְעָלְמֵי עָלְמַיָּא}\\
יִתְבָּרַךְ וְיִשְׁתַּבַּח וְיִתְפָּאַר וְיִתְרוֹמַם וְיִתְנַשֵּׂא וְיִתְהַדַּר וְיִתְעַלֶּה וְיִתְהַלַּל שְׁמֵיהּ דְּקֻדְשָׁא
\textbf{בְּרִיךְ הוּא}
׃ *לְעֵֽלָּא מִן כׇּל־בִּרְכָתָא
(*\instruction{בעשי״ת:}
לְעֵֽלָּא לְעֵֽלָּא מִכׇּל־בִּרְכָתָא) וְשִׁירָתָא תֻּשְׁבְּחָתָא וְנֶחָמָתָא דַּאֲמִירָן בְּעָלְמָא וְאִמְרוּ אָמֵן׃
\kaddishamein
\alrabbanan
\kaddishamein
יְהֵא שְׁלָמָא רַבָּא מִן שְׁמַיָּא וְחַיִּים עָלֵֽינוּ וְעַל כׇּל־יִשְׂרָאֵל׃ וְאִמְרוּ אָמֵן׃
\kaddishamein
\osehshalom
\kaddishamein}

\newcommand{\adonolam}{
	
	\firstword{אֲדוֹן עוֹלָם}
	אֲשֶׁר מָלַךְ \hfill בְּטֶֽרֶם כׇּל־יְצִיר נִבְרָא׃ \\
	לְעֵת נַעֲשָׂה בְחֶפְצוֹ כֹּל \hfill אֲזַי מֶֽלֶךְ שְׁמוֹ נִקְרָא׃\\
	וְאַֽחֲרֵי כִּכְלוֹת הַכֹּל \hfill לְבַדּוֹ יִמְלֹךְ נוֹרָא׃ \\
	וְהוּא הָיָה וְהוּא הוֶֹה \hfill וְהוּא יִהְיֶה בְּתִפְאָרָה׃ \\
	וְהוּא אֶחָד וְאֵין שֵׁנִי \hfill לְהַמְשִׁיל לוֹ לְהַחְבִּֽירָה׃ \\
	בְּלִי רֵאשִׁית בְּלִי תַכְלִית \hfill וְלוֹ הָעֹז וְהַמִּשְׂרָה׃ \\
	וְהוּא אֵלִי וְחַי גוֹאֲלִי \hfill וְצוּר חֶבְלִי בְּעֵת צָרָה׃ \\
	וְהוּא נִסִּי וּמָנוֹסִי \hfill מְנָת כּוֹסִי בְּיוֹם אֶקְרָא׃ \\
	בְּיָדוֹ אַפְקִיד רוּחִי \hfill בְּעֵת אִישַׁן וְאָעִֽירָה׃ \\
	וְעִם רוּחִי גְּוִיָּתִי \hfill אֲדוֹנָי לִי וְלֹא אִירָא׃\\
}

\newcommand{\tamid}{
	%יְהִי רָצוֹן מִלְּפָנֶֽיךָ יְיָ אֱלֹהֵינוּ וֵאלֹהֵי אֲבוֹתֵינוּ, שֶׁתְּרַחֵם עָלֵינוּ וְתִמְחוֹל לָנוּ עַל כׇּל־חַטֹּאתֵינוּ וּתְכַפֶּר לָנוּ עַל כׇּל־עֲוֹנוֹתֵינוּ וְתִמְחוֹל וְתִסְלַח לָנוּ עַל כׇּל־פְּשָׁעֵינוּ וְשֶׁיִבָּנֶה בֵּית הַמִּקְדָּשׁ בִּמְהֵרָה בְיָמֵינוּ וְנַקְרִיב לְפָנֶיךָ קׇרְבַּן הַתָּמִיד שֶׁיְּכַפֵּר בַּעֲדֵנוּ כְּמוֹ שֶׁכָּתַבְתָּ עָלֵינוּ בְּתוֹרָתֶךָ עַל יְדֵי משֶׁה עַבְדֶּךָ מִפִּי כְבוֹדֶךָ כָּאָמוּר׃\\
וַיְדַבֵּ֥ר\source{במדבר כח} יְיָ֖ אֶל־מֹשֶׁ֥ה לֵּאמֹֽר׃ צַ֚ו אֶת־בְּנֵ֣י יִשְׂרָאֵ֔ל וְאָמַרְתָּ֖ אֲלֵהֶ֑ם אֶת־קׇרְבָּנִ֨י לַחְמִ֜י לְאִשַּׁ֗י רֵ֚יחַ נִֽיחֹחִ֔י תִּשְׁמְר֕וּ לְהַקְרִ֥יב לִ֖י בְּמוֹעֲדֽוֹ׃ וְאָמַרְתָּ֣ לָהֶ֔ם זֶ֚ה הָֽאִשֶּׁ֔ה אֲשֶׁ֥ר תַּקְרִ֖יבוּ לַייָ֑ כְּבָשִׂ֨ים בְּנֵֽי־שָׁנָ֧ה תְמִימִ֛ם שְׁנַ֥יִם לַיּ֖וֹם עֹלָ֥ה תָמִֽיד׃ אֶת־הַכֶּ֥בֶשׂ אֶחָ֖ד תַּעֲשֶׂ֣ה בַבֹּ֑קֶר וְאֵת֙ הַכֶּ֣בֶשׂ הַשֵּׁנִ֔י תַּעֲשֶׂ֖ה בֵּ֥ין הָֽעַרְבָּֽיִם׃ וַעֲשִׂירִ֧ית הָאֵיפָ֛ה סֹ֖לֶת לְמִנְחָ֑ה בְּלוּלָ֛ה בְּשֶׁ֥מֶן כָּתִ֖ית רְבִיעִ֥ת הַהִֽין׃ עֹלַ֖ת תָּמִ֑יד הָעֲשֻׂיָה֙ בְּהַ֣ר סִינַ֔י לְרֵ֣יחַ נִיחֹ֔חַ אִשֶּׁ֖ה לַֽייָ׃ וְנִסְכּוֹ֙ רְבִיעִ֣ת הַהִ֔ין לַכֶּ֖בֶשׂ הָאֶחָ֑ד בַּקֹּ֗דֶשׁ הַסֵּ֛ךְ נֶ֥סֶךְ שֵׁכָ֖ר לַייָ׃ וְאֵת֙ הַכֶּ֣בֶשׂ הַשֵּׁנִ֔י תַּעֲשֶׂ֖ה בֵּ֣ין הָֽעַרְבָּ֑יִם כְּמִנְחַ֨ת הַבֹּ֤קֶר וּכְנִסְכּוֹ֙ תַּעֲשֶׂ֔ה אִשֵּׁ֛ה רֵ֥יחַ נִיחֹ֖חַ לַייָ׃	
וְשָׁחַ֨ט\source{ויקרא א} אֹת֜וֹ עַ֣ל יֶ֧רֶךְ הַמִּזְבֵּ֛חַ צָפֹ֖נָה לִפְנֵ֣י יְיָ֑ וְזָרְק֡וּ בְּנֵי֩ אַהֲרֹ֨ן הַכֹּהֲנִ֧ים אֶת־דָּמ֛וֹ עַל־הַמִּזְבֵּ֖חַ סָבִֽיב׃	%יְהִי רָצוֹן מִלְּפָנֶֽיךָ, יְיָ אֱלֹהֵֽינוּ וֵאלֹהֵי אֲבוֹתֵֽינוּ, שֶׁתְּהֵא אֲמִירָה זוֹ חֲשׁוּבָה וּמְקֻבֶּֽלֶת וּמְרֻצָּה לְפָנֶֽיךָ, כְּאִלּוּ הִקְרַֽבְנוּ קׇרְבַּן הַתָּמִיד בְּמוֹעֲדוֹ וּבִמְקוֹמוֹ וּכְהִלְכָתוֹ.
}

\newcommand{\ketoret}{
	\firstword{אַתָּה הוּא יְיָ אֱלֹהֵינוּ שֶׁהִקְטִירוּ}
	אֲבוֹתֵינוּ לְפָנֶיךָ אֶת־קְטֹרֶת הַסַּמִּים בִּזְמַן שֶׁבֵּית הַמִּקְדָּשׁ קַיָּם. כַּאֲשֶׁר צִוִּיתָ אוֹתָם עַל יְדֵי מֹשֶׁה נְבִיאֶךָ. כַּכָּתוּב בְּתוֹרָתֶךָ׃ \source{שמות ל} 
	וַיֹּ֩אמֶר֩ יְיָ֨ אֶל־מֹשֶׁ֜ה קַח־לְךָ֣ סַמִּ֗ים נָטָ֤ף ׀ וּשְׁחֵ֙לֶת֙ וְחֶלְבְּנָ֔ה סַמִּ֖ים וּלְבֹנָ֣ה זַכָּ֑ה בַּ֥ד בְּבַ֖ד יִהְיֶֽה׃ וְעָשִׂ֤יתָ אֹתָהּ֙ קְטֹ֔רֶת רֹ֖קַח מַעֲשֵׂ֣ה רוֹקֵ֑חַ מְמֻלָּ֖ח טָה֥וֹר קֹֽדֶשׁ׃ וְשָֽׁחַקְתָּ֣ מִמֶּ֘נָּה֮ הָדֵק֒ וְנָתַתָּ֨ה מִמֶּ֜נָּה לִפְנֵ֤י הָעֵדֻת֙ בְּאֹ֣הֶל מוֹעֵ֔ד אֲשֶׁ֛ר אִוָּעֵ֥ד לְךָ֖ שָׁ֑מָּה קֹ֥דֶשׁ קׇֽדָשִׁ֖ים תִּהְיֶ֥ה לָכֶֽם׃\\
	וְנֶאֱמַר׃ וְהִקְטִ֥יר עָלָ֛יו אַהֲרֹ֖ן קְטֹ֣רֶת סַמִּ֑ים בַּבֹּ֣קֶר בַּבֹּ֗קֶר בְּהֵיטִיב֛וֹ אֶת־הַנֵּרֹ֖ת יַקְטִירֶֽנָּה׃ וּבְהַעֲלֹ֨ת אַהֲרֹ֧ן אֶת־הַנֵּרֹ֛ת בֵּ֥ין הָעַרְבַּ֖יִם יַקְטִירֶ֑נָּה קְטֹ֧רֶת תָּמִ֛יד לִפְנֵ֥י יְיָ֖ לְדֹרֹתֵיכֶֽם׃
}

\newcommand{\melekhmalakhyimlokh}{יְיָ מֶֽלֶךְ \middot יְיָ מָלָךְ \middot יְיָ יִמְלֹךְ לְעוֹלָם וָעֶד׃}

\newcommand{\ashrei}{
	
	\firstword{אַ֭שְׁרֵי יוֹשְׁבֵ֣י בֵיתֶ֑ךָ ע֗֝וֹד יְֽהַלְל֥וּךָ סֶּֽלָה׃ }\source{תהלים פד}\\
	\textbf{אַשְׁרֵ֣י הָ֭עָם שֶׁכָּ֣כָה לּ֑וֹ אַֽשְׁרֵ֥י הָ֝עָ֗ם שֱׁייָ֥ אֱלֹהָֽיו׃ }\source{תהלים קמד}\\
	%used to have dots for the letters of the alphabet, but it makes it too messy with the trop marks
	תְּהִלָּ֗ה לְדָ֫וִ֥ד\source{תהלים קמה}
	אֲרוֹמִמְךָ֣ אֱלוֹהַ֣י הַמֶּ֑לֶךְ וַאֲבָרְכָ֥ה שִׁ֝מְךָ֗ לְעוֹלָ֥ם וָעֶֽד׃
	בְּכׇל־י֥וֹם אֲבָֽרְכֶ֑ךָּ וַאֲהַֽלְלָ֥ה שִׁ֝מְךָ֗ לְעוֹלָ֥ם וָעֶֽד׃
	גָּ֘ד֤וֹל יְיָ֣ וּמְהֻלָּ֣ל מְאֹ֑ד וְ֝לִגְדֻלָּת֗וֹ אֵ֣ין חֵֽקֶר׃
	דּ֣וֹר לְ֭דוֹר יְשַׁבַּ֣ח מַעֲשֶׂ֑יךָ וּגְב֖וּרֹתֶ֣יךָ יַגִּֽידוּ׃
	הֲ֭דַר כְּב֣וֹד הוֹדֶ֑ךָ וְדִבְרֵ֖י נִפְלְאֹתֶ֣יךָ אָשִֽׂיחָה׃
	וֶעֱז֣וּז נֽוֹרְאֹתֶ֣יךָ יֹאמֵ֑רוּ \qk{וּגְדֻלָּתְךָ֥}{וגדלותיך} אֲסַפְּרֶֽנָּה׃
	זֵ֣כֶר רַב־טוּבְךָ֣ יַבִּ֑יעוּ וְצִדְקָתְךָ֥ יְרַנֵּֽנוּ׃
	חַנּ֣וּן וְרַח֣וּם יְיָ֑ אֶ֥רֶךְ אַ֝פַּ֗יִם וּגְדׇל־חָֽסֶד׃
	טוֹב־יְיָ֥ לַכֹּ֑ל וְ֝רַחֲמָ֗יו עַל־כׇּל־מַעֲשָֽׂיו׃
	יוֹד֣וּךָ יְיָ֭ כׇּל־מַעֲשֶׂ֑יךָ וַ֝חֲסִידֶ֗יךָ יְבָרְכֽוּכָה׃
	כְּב֣וֹד מַלְכוּתְךָ֣ יֹאמֵ֑רוּ וּגְבוּרָתְךָ֥ יְדַבֵּֽרוּ׃
	לְהוֹדִ֤יעַ ׀ לִבְנֵ֣י הָ֭אָדָם גְּבוּרֹתָ֑יו וּ֝כְב֗וֹד הֲדַ֣ר מַלְכוּתֽוֹ׃
	מַֽלְכוּתְךָ֗ מַלְכ֥וּת כׇּל־עֹלָמִ֑ים וּ֝מֶֽמְשַׁלְתְּךָ֗ בְּכׇל־דּ֥וֹר וָדֹֽר׃
	סוֹמֵ֣ךְ יְיָ֭ לְכׇל־הַנֹּפְלִ֑ים וְ֝זוֹקֵ֗ף לְכׇל־הַכְּפוּפִֽים׃
	עֵֽינֵי־כֹ֭ל אֵלֶ֣יךָ יְשַׂבֵּ֑רוּ וְאַתָּ֤ה נֽוֹתֵן־לָהֶ֖ם אֶת־אׇכְלָ֣ם בְּעִתּֽוֹ׃
	פּוֹתֵ֥חַ אֶת־יָדֶ֑ךָ וּמַשְׂבִּ֖יעַ לְכׇל־חַ֣י רָצֽוֹן׃
	צַדִּ֣יק יְיָ֭ בְּכׇל־דְּרָכָ֑יו וְ֝חָסִ֗יד בְּכׇל־מַעֲשָֽׂיו׃
	קָר֣וֹב יְיָ֭ לְכׇל־קֹרְאָ֑יו לְכֹ֤ל אֲשֶׁ֖ר יִקְרָאֻ֣הוּ בֶֽאֱמֶֽת׃ 
	רְצוֹן־יְרֵאָ֥יו יַעֲשֶׂ֑ה וְֽאֶת־שַׁוְעָתָ֥ם יִ֝שְׁמַ֗ע וְיוֹשִׁיעֵֽם׃ 
	שׁוֹמֵ֣ר יְיָ֭ אֶת־כׇּל־אֹהֲבָ֑יו וְאֵ֖ת כׇּל־הָרְשָׁעִ֣ים יַשְׁמִֽיד׃ 
	תְּהִלַּ֥ת יְיָ֗ יְֽדַבֶּ֫ר פִּ֥י וִיבָרֵ֣ךְ כׇּל־בָּ֭שָׂר שֵׁ֥ם קׇדְשׁ֗וֹ לְעוֹלָ֥ם וָעֶֽד׃
	\source{תהלים קטו}וַאֲנַ֤חְנוּ ׀ נְבָ֘רֵ֤ךְ יָ֗הּ מֵעַתָּ֥ה וְעַד־עוֹלָ֗ם הַֽלְלוּ־יָֽהּ׃
	
}

\newcommand{\mimaamakim}{
	
	
	\begin{sometimes}\\
		
		\englishinst{During the Ten Penitential Days and Hoshana Rabba, the ark is opened for the following Psalm. The reader chants each verse, and the congregation repeats it.}
		%\instruction{בעשי״ת והושענא רבא, פותחים הארון. הש״ץ קורא בפסוק בפסוק והקהל חוזר׃}\\
		\firstword{שִׁ֥יר הַֽמַּעֲל֑וֹת מִמַּעֲמַקִּ֖ים}\source{תהלים קל}
		קְרָאתִ֣יךָ יְיָ׃\ \hfill\break
		אֲדֹנָי֮ שִׁמְעָ֢ה בְק֫וֹלִ֥י תִּהְיֶ֣ינָה אׇ֭זְנֶיךָ קַשֻּׁב֑וֹת לְ֝ק֗וֹל תַּחֲנוּנָֽי׃\hfill \break
		אִם־עֲוֺנ֥וֹת תִּשְׁמׇר־יָ֑הּ אֲ֝דֹנָ֗י מִ֣י יַעֲמֹֽד׃\hfill \break
		כִּֽי־עִמְּךָ֥ הַסְּלִיחָ֑ה לְ֝מַ֗עַן תִּוָּרֵֽא׃\hfill \break
		קִוִּ֣יתִי יְיָ֭ קִוְּתָ֣ה נַפְשִׁ֑י וְֽלִדְבָר֥וֹ הוֹחָֽלְתִּי׃\hfill \break
		נַפְשִׁ֥י לַאדֹנָ֑י מִשֹּׁמְרִ֥ים לַ֝בֹּ֗קֶר שֹׁמְרִ֥ים לַבֹּֽקֶר׃\hfill \break
		יַחֵ֥ל יִשְׂרָאֵ֗ל אֶל־יְ֫יָ֥ כִּֽי־עִם־יְיָ֥ הַחֶ֑סֶד וְהַרְבֵּ֖ה עִמּ֣וֹ פְדֽוּת׃\hfill \break
		וְ֭הוּא יִפְדֶּ֣ה אֶת־יִשְׂרָאֵ֑ל מִ֝כֹּ֗ל עֲוֺנֹתָֽיו׃ \instruction{סוגרים הארון}\hfill \break
		
	\end{sometimes}
	
}

\newcommand{\tzadialeph}{
יֹ֭שֵׁב\source{תהילים צא} בְּסֵ֣תֶר עֶלְי֑וֹן בְּצֵ֥ל שַׁ֝דַּ֗י יִתְלוֹנָֽן׃ אֹמַ֗ר לַ֭ייָ מַחְסִ֣י וּמְצוּדָתִ֑י אֱ֝לֹהַ֗י אֶבְטַח־בּֽוֹ׃ כִּ֤י ה֣וּא יַ֭צִּילְךָ מִפַּ֥ח יָק֗וּשׁ מִדֶּ֥בֶר הַוּֽוֹת׃ בְּאֶבְרָת֨וֹ ׀ יָ֣סֶךְ לָ֭ךְ וְתַחַת־כְּנָפָ֣יו תֶּחְסֶ֑ה צִנָּ֖ה וְסֹחֵרָ֣ה אֲמִתּֽוֹ׃ לֹֽא־תִ֭ירָא מִפַּ֣חַד לָ֑יְלָה מֵ֝חֵ֗ץ יָע֥וּף יוֹמָֽם׃ מִ֭דֶּבֶר בָּאֹ֣פֶל יַהֲלֹ֑ךְ מִ֝קֶּ֗טֶב יָשׁ֥וּד צׇהֳרָֽיִם׃ יִפֹּ֤ל מִצִּדְּךָ֨ ׀ אֶ֗לֶף וּרְבָבָ֥ה מִימִינֶ֑ךָ אֵ֝לֶ֗יךָ לֹ֣א יִגָּֽשׁ׃ רַ֭ק בְּעֵינֶ֣יךָ תַבִּ֑יט וְשִׁלֻּמַ֖ת רְשָׁעִ֣ים תִּרְאֶֽה׃ כִּֽי־אַתָּ֣ה יְיָ֣ מַחְסִ֑י עֶ֝לְי֗וֹן שַׂ֣מְתָּ מְעוֹנֶֽךָ׃ לֹא־תְאֻנֶּ֣ה אֵלֶ֣יךָ רָעָ֑ה וְ֝נֶ֗גַע לֹא־יִקְרַ֥ב בְּאׇהֳלֶֽךָ׃ כִּ֣י מַ֭לְאָכָיו יְצַוֶּה־לָּ֑ךְ לִ֝שְׁמׇרְךָ֗ בְּכׇל־דְּרָכֶֽיךָ׃ עַל־כַּפַּ֥יִם יִשָּׂא֑וּנְךָ פֶּן־תִּגֹּ֖ף בָּאֶ֣בֶן רַגְלֶֽךָ׃ עַל־שַׁ֣חַל וָפֶ֣תֶן תִּדְרֹ֑ךְ תִּרְמֹ֖ס כְּפִ֣יר וְתַנִּֽין׃ כִּ֤י בִ֣י חָ֭שַׁק וַאֲפַלְּטֵ֑הוּ אֲ֝שַׂגְּבֵ֗הוּ כִּֽי־יָדַ֥ע שְׁמִֽי׃ יִקְרָאֵ֨נִי ׀ וְֽאֶעֱנֵ֗הוּ עִמּֽוֹ־אָנֹכִ֥י בְצָרָ֑ה אֲ֝חַלְּצֵ֗הוּ וַאֲכַבְּדֵֽהוּ׃ אֹ֣רֶךְ יָ֭מִים אַשְׂבִּיעֵ֑הוּ וְ֝אַרְאֵ֗הוּ בִּישׁוּעָתִֽי׃	\scriptsize{אֹ֣רֶךְ יָ֭מִים אַשְׂבִּיעֵ֑הוּ וְ֝אַרְאֵ֗הוּ בִּישׁוּעָתִֽי׃}
	\normalsize{}
}

\newcommand{\nishmat}{
\firstword{נִשְׁמַת}
כׇּל־חַי תְּבָרֵךְ אֶת־שִׁמְךָ יְיָ אֱלֹהֵֽינוּ \middot וְרֽוּחַ כׇּל־בָּשָׂר תְּפָאֵר וּתְרוֹמֵם זִכְרְךָ מַלְכֵּֽנוּ תָּמִיד \middot מִן הָעוֹלָם וְעַד הָעוֹלָם אַתָּה אֵל וּמִבַּלְעָדֶֽיךָ אֵין לָֽנּוּ מֶֽלֶךְ גּוֹאֵל וּמוֹשִֽׁיעַ פּוֹדֶה וּמַצִיל וּמְפַרְנֵס וּמְרַחֵם בְּכׇל־עֵת צָרָה וְצוּקָה \middot אֵין לָֽנוּ מֶֽלֶךְ אֶלָּא אַֽתָּה׃ אֱלֹהֵי הָרִאשׁוֹנִים וְהָאַחֲרוֹנִים \middot אֱלֽוֹהַּ כׇּל־בְּרִיּוֹת \middot אֲדוֹן כׇּל־תּוֹלָדוֹת הַמְהֻלָּל בְּרֹב תִּשְׁבָּחוֹת \middot הַמְנַהֵג עוֹלָמוֹ בְּחֶֽסֶד וּבְרִיּוֹתָיו בְּרַחֲמִים׃ וַייָ לֹא יָנוּם וְלֹא יִישָׁן \middot הַמְעוֹרֵר יְשֵׁנִים וְהַמֵּקִיץ רְדוּמִים וְהַמֵּשִֽׂיחַ אִלְּמִים וְהַמַּתִּיר אֲסוּרִים וְהַסּוֹמֵךְ נוֹפְלִים וְהַזּוֹקֵף כְּפוּפִים \middot לְךָ לְבַדְּךָ אֲנַֽחְנוּ מוֹדִים׃ 
אִֽלּוּ פִֽינוּ מָלֵא שִׁירָה כַּיָּם וּלְשׁוֹנֵֽנוּ רִנָּה כַּהֲמוֹן גַּלָּיו וְשִׂפְתוֹתֵֽינוּ שֶֽׁבַח כְּמֶרְחֲבֵי רָקִֽיעַ וְעֵינֵֽינוּ מְאִירוֹת כַּשֶּֽׁמֶשׁ וְכַיָּרֵֽחַ וְיָדֵֽינוּ פְּרוּשׂוֹת כְּנִשְׁרֵי שָּׁמָֽיִם וְרַגְלֵֽינוּ קַלּוֹת כָּאַיָּלוֹת \middot אֵין אָֽנוּ מַסְפִּיקִים לְהוֹדוֹת לְךָ יְיָ אֱלֹהֵֽינוּ וֵאלֹהֵי אֲבוֹתֵֽינוּ וּלְבָרֵךְ אֶת־שְׁמֶֽךָ \middot עַל אַֽחַת מֵאָֽלֶף אֶֽלֶף אַלְפֵי אֲלָפִים וְרִבֵּי רְבָבוֹת פְּעָמִים הַטּוֹבוֹת שֶׁעָשִֽׂיתָ עִם אֲבוֹתֵֽינוּ וְעִמָּֽנוּ׃ 
מִמִּצְרַֽיִם גְּאַלְתָּֽנוּ יְיָ אֱלֹהֵֽינוּ וּמִבֵּית עֲבָדִים פְּדִיתָֽנוּ \middot בָּרָעָב זַנְתָּֽנוּ וּבְשָׂבָע כִּלְכַּלְתָּֽנוּ מֵחֶֽרֶב הִצַּלְתָּֽנּוּ וּמִדֶּֽבֶר מִלַּטְתָּֽנוּ וּמֵחֳלָיִם רָעִים וְנֶאֱמָנִים דִּלִּיתָֽנוּ \middot עַד הֵֽנָּה עֲזָרֽוּנוּ רַחֲמֶֽיךָ וְלֹא עֲזָבֽוּנוּ חֲסָדֶֽיךָ \middot וְאַל תִּטְּשֵׁנוּ יְיָ אֱלֹהֵֽינוּ לָנֶֽצַח׃ עַל כֵּן אֵבָרִים שֶׁפִּלַּגְתָּ בָּֽנוּ וְרֽוּחַ וּנְשָׁמָה שֶׁנָּפַֽחְתָּ בְּאַפֵּֽינוּ וְלָשׁוֹן אֲשֶׁר שַֽׂמְתָּ בְּפִֽינוּ \middot הֵן הֵם יוֹדוּ וִיבָרְכוּ וִישַׁבְּחוּ וִיפָאֲרוּ וִירוֹמֲמוּ וְיַעֲרִֽיצוּ וְיַקְדִּֽישׁוּ וְיַמְלִֽיכוּ אֶת־שִׁמְךָ מַלְכֵּֽנוּ׃ כִּי כׇל־פֶּה לְךָ יוֹדֶה וְכׇל־לָשׁוֹן לְךָ תִּשָּׁבַע וְכׇל־בֶּֽרֶךְ לְךָ תִּכְרַע וְכׇל־קוֹמָה לְפָנֶֽיךָ תִשְׁתַּחֲוֶה וְכׇל־לְבָבוֹת יִירָאֽוּךָ וְכׇל־קֶֽרֶב וּכְלָיוֹת יְזַמְּרוּ לִשְׁמֶֽךָ \middot כַּדָּבָר שֶׁכָּתוּב׃
כׇּ‍֥ל \source{תהלים לה}עַצְמוֹתַ֨י ׀ תֹּאמַרְנָה֮ יְיָ֗ מִ֥י כָ֫מ֥וֹךָ מַצִּ֣יל עָ֭נִי מֵחָזָ֣ק מִמֶּ֑נּוּ וְעָנִ֥י וְ֝אֶבְי֗וֹן מִגֹּֽזְלֽוֹ׃
מִי יִדְמֶה־לָּךְ וּמִי יִשְׁוֶה־לָּךְ וּמִי יַעֲרׇךְ־לָּךְ׃ הָאֵל הַגָּדוֹל הַגִּבּוֹר וְהַנּוֹרָא אֵל עֶלְיוֹן קֹנֵה שָׁמַֽיִם וָאָֽרֶץ׃
נְהַלֶּלְךָ וּנְשַׁבֵּחֲךָ וּנְפָאֶרְךָ וּנְבָרֵךְ אֶת־שֵׁם קׇדְשֶׁךָ כָּאָמוּר׃
לְדָוִ֨ד \source{תהלים קג} ׀ בָּרְכִ֣י נַ֭פְשִׁי אֶת־יְיָ֑ וְכׇל־קְ֝רָבַ֗י אֶת־שֵׁ֥ם קׇדְשֽׁוֹ׃
}

\newcommand{\hael}{
\firstword{הָאֵל}
בְּתַעֲצֻמוֹת עֻזֶּֽךָ \middot \\
\firstword{הַגָּדוֹל}
בִּכְבוֹד שְׁמֶֽךָ \middot \\
\firstword{הַגִּבּוֹר}
לָנֶֽצַח וְהַנּוֹרָא בְּנוֹרְאוֹתֶֽיךָ \middot\\
\firstword{הַמֶּֽלֶךְ}
הַיּוֹשֵׁב עַל כִּסֵּא רָם וְנִשָּׂא׃
}

\newcommand{\shochenad}{
\firstword{שׁוֹכֵן}
עַד מָרוֹם וְקָדוֹשׁ שְׁמוֹ \middot וְכָתוּב׃ רַֽנֲנ֣וּ \source{תהלים לג}צַ֭דִּיקִים בַּֽיְיָ֑ לַ֝יְשָׁרִ֗ים נָאוָ֥ה תְהִלָּֽה׃ בְּפִי יְשָׁרִים תִּתְהַלָּל \middot וּבְדִבְרֵי צַדִּיקִים תִּתְבָּרַךְ \middot וּבִלְשׁוֹן חֲסִידִים תִּתְרוֹמָם \middot וּבְקֶֽרֶב קְדוֹשִׁים תִּתְקַדָּשׁ׃

\firstword{וּבְמַקְהֲלוֹת}
רִבְבוֹת עַמְּךָ בֵּית יִשְׂרָאֵל בְּרִנָּה יִתְפָּאַר שִׁמְךָ מַלְכֵּֽנוּ בְּכׇל־דּוֹר וָדוֹר \middot שֶׁכֵּן חוֹבַת כׇּל־הַיְצוּרִים לְפָנֶֽיךָ יְיָ אֱלֹהֵֽינוּ וֵאלֹהֵי אֲבוֹתֵֽינוּ לְהוֹדוֹת לְהַלֵּל לְשַׁבֵּֽחַ לְפָאֵר לְרוֹמֵם לְהַדֵּר לְבָרֵךְ לְעַלֵּה וּלְקַלֵּס עַל כׇּל־דִּבְרֵי שִׁירוֹת וְתֻשְׁבְּחוֹת דָּוִד בֶּן־יִשַׁי עַבְדְּךָ מְשִׁיחֶֽךָ׃
}

\newcommand{\barachu}{
	%\begin{wrapfigure}[5]{I}{0.6\textwidth}
	\begin{minipage}{0.8\textwidth}
		\leftskip=0pt plus-.5fil
		\rightskip=0pt plus.5fil
		\parfillskip=0pt plus1fil
		\begin{large}
			
			\shatz
			\begin{Large}\textbf{בָּרְכוּ אֶת־יְיָ הַמְבֹרָךְ׃}\end{Large}
		\end{large}
		
		\vspace{12pt}
		
		\shatzvkahal
		בָּרוּךְ יְיָ הַמְבֹרָךְ לְעוֹלָם וָעֶד׃
	\end{minipage}
	ֺֺ%\end{wrapfigure}
	
	%\begin{footnotesize}
	%יִתְבָּרַךְ וְיִשְׁתַּבַּח וְיִתְפָּאַר וְיִתְרוֹמַם וְיִתְנַשֵּׂא שְׁמוֹ שֶׁל מֶֽלֶךְ מַלְכֵי הַמְּלָכִים הַקָּדוֹשׁ בָּרוּךְ הוּא׃ שֶׁהוּא רִאשׁוֹן וְהוּא אַחֲרוֹן וּמִבַּלְעָדָיו אֵין אֱלֹהִים׃ \source{תהלים סח}סֹ֡לּוּ לָרֹכֵ֣ב בָּ֭עֲרָבוֹת בְּיָ֥הּ שְׁמ֗וֹ וְעִלְז֥וּ לְפָנָֽיו׃ וּשְׁמוֹ מְרוֹמָם עַל־כׇּל־בְּרָכָה וּתְהִלָּה׃ בָּרוּךְ שֵׁם כְּבוֺד מַלְכוּתוֺ לְעוֺלָם וָעֶד׃ \source{תהלים קיג}יְהִ֤י שֵׁ֣ם יְיָ֣ מְבֹרָ֑ךְ מֵ֝עַתָּ֗ה וְעַד־עוֹלָֽם׃
	%
	%\end{footnotesize}
}

\newcommand{\hameir}{
	\firstword{הַמֵּאִיר}
	לָאָֽרֶץ וְלַדָּרִים עָלֶֽיהָ בְּרַחֲמִים \middot וּבְטוּבוֹ מְחַדֵּשׁ בְּכׇל־יוֹם תָּמִיד מַעֲשֵׂה בְרֵאשִׁית׃ 
	\ifboolexpr{togl {includeshabbat} and togl {includefestival}}{\mdsource{תהלים קד}}{\source{תהלים קד}}
	מָה־רַבּ֬וּ מַעֲשֶׂ֨יךָ ׀ יְיָ֗ כֻּ֭לָּם בְּחׇכְמָ֣ה עָשִׂ֑יתָ מָלְאָ֥ה הָ֝אָ֗רֶץ קִנְיָנֶֽךָ׃ הַמֶּֽלֶךְ הַמְרוֹמָם לְבַדּוֹ מֵאָז \middot הַמְשֻׁבָּח וְהַמְפֹאָר וְהַמִּתְנַשֵּׂא מִימוֹת עוֹלָם׃ אֱלֹהֵי עוֹלָם בְּרַחֲמֶיךָ הָרַבִּים רַחֵם עָלֵינוּ \middot אֲדוֹן עֻזֵּֽנוּ צוּר מִשְׂגַּבֵּנוּ מָגֵן יִשְׁעֵֽנוּ מִשְׂגָּב בַּעֲדֵֽנוּ׃ אֵ֗ל בָּ֗רוּךְ גְּ֗דוֹל דֵּ֗עָה \middot הֵ֗כִין וּ֗פָעַל זׇ֗הֳרֵי חַ֗מָּה \middot ט֗וֹב יָ֗צַר כָּ֗בוֹד לִ֗שְׁמוֹ \middot מְ֗אוֹרוֹת נָ֗תַן סְ֗בִיבוֹת עֻ֗זּוֹ \middot פִּ֗נּוֹת צְ֗בָאָיו קְ֗דוֹשִׁים ר֗וֹמֲמֵי שַׁ֗דַּי \middot תָּ֗מִיד מְסַפְּרִים כְּבוֹד־אֵל וּקְדֻשָׁתוֹ׃ תִּתְבָּרַךְ יְיָ אֱלֹהֵֽינוּ עַל־שֶׁבַח מַעֲשֵׂי יָדֶֽיךָ \middot וְעַל־מְאֽוֹרֵי־אוֹר שֶׁעָשִֽׂיתָ יְפָאֲרֽוּךָ סֶּֽלָה׃
}

\newcommand{\kadoshbase}{\textbf{קָד֧וֹשׁ ׀ קָד֛וֹשׁ קָד֖וֹשׁ יְיָ֣ צְבָא֑וֹת מְלֹ֥א כׇל־הָאָ֖רֶץ כְּבוֹדֽוֹ׃}}
\newcommand{\barukhbase}{\textbf{בָּר֥וּךְ כְּבוֹד־יְיָ֖ מִמְּקוֹמֽוֹ׃}}
\newcommand{\yimlokhbase}{\textbf{יִמְלֹ֤ךְ יְיָ֨ ׀ לְעוֹלָ֗ם אֱלֹהַ֣יִךְ צִ֭יּוֹן לְדֹ֥ר וָ֝דֹ֗ר הַֽלְלוּיָֽהּ׃}}

\newcommand{\kadoshkadoshkadosh}{\kadoshbase\mdsource{ישעיה ו}} %\source{ישעיה ו}
\newcommand{\barukhhashem}{\barukhbase\mdsource{יחזקאל ג}}%\source{יחזקאל ג}
\newcommand{\yimloch}{\yimlokhbase\mdsource{תהלים קמו}}%\source{תהלים קמו}

\newcommand{\kadoshkadoshkadoshsource}{\kadoshbase\source{ישעיה ו}} %\source{ישעיה ו}
\newcommand{\barukhhashemsource}{\barukhbase\source{יחזקאל ג}}%\source{יחזקאל ג}
\newcommand{\yimlochsource}{\yimlokhbase\source{תהלים קמו}}%\source{תהלים קמו}

\newcommand{\hashemyimloch}{\textbf{יְיָ֥ ׀ יִמְלֹ֖ךְ לְעֹלָ֥ם וָעֶֽד׃} \source{שמות טו}}

\newcommand{\yotzerhameoros}{
	תִּתְבָּרַךְ צוּרֵֽנוּ מַלְכֵּֽנוּ וְגוֹאֲלֵֽנוּ בּוֹרֵא קְדוֹשִׁים \middot יִשְׁתַּבַּח שִׁמְךָ לָעַד מַלְכֵּֽנוּ יוֹצֵר מְשָׁרְתִים \middot וַאֲשֶׁר מְשָׁרְתָיו כֻּלָּם עוֹמְדִים בְּרוּם עוֹלָם \middot וּמַשְׁמִיעִים בְּיִרְאָה יַֽחַד בְּקוֹל \middot דִּבְרֵי אֱלֹהִים חַיִּים וּמֶֽלֶךְ עוֹלָם׃ כֻּלָּם אֲהוּבִים כֻּלָּם בְּרוּרִים כֻּלָּם גִּבּוֹרִים וְכֻלָּם עֹשִׂים בְּאֵימָה וּבְיִרְאָה רְצוֹן קוׂנָם \middot וְכֻלָּם פּוֹתְחִים אֶת־פִּיהֶם בִּקְדֻשָּׁה וּבְטׇהֳרָה בְּשִׁירָה וּבְזִמְרָה וּמְבָרְכִים וּמְשַׁבְּחִים וּמְפָאֲרִים וּמַעֲרִיצִים וּמַקְדִּישִׁים וּמַמְלִיכִים
	
	\kahal אֶת־שֵׁם הָאֵל הַמֶּֽלֶךְ הַגָּדוֹל הַגִּבּוֹר וְהַנּוֹרָא קָדוֹשׁ הוּא׃
	
	וְכֻלָּם מְקַבְּלִים עֲלֵיהֶם עֹל מַלְכוּת שָׁמַֽיִם זֶה מִזֶּה וְנוֹתְנִים רְשׁוּת זֶה לָזֶה \middot לְהַקְדִּישׁ לְיוֹצְרָם בְּנַֽחַת רֽוּחַ בְּשָׂפָה בְרוּרָה וּבִנְעִימָה \middot קְדֻשָּׁה כֻּלָּם כְּאֶחָד עוֹנִים וְאוֹמְרִים בְּיִרְאָה׃
	
	\kahal\kadoshkadoshkadoshsource
	
	וְהָאוֹפַנִּים וְחַיּוֹת הַקֹּֽדֶשׁ בְּרַֽעַשׁ גָּדוֹל מִתְנַשְּׂאִים לְעֻמַּת שְׂרָפִים לְעֻמָּתָם מְשַׁבְּחִים וְאוֹמְרִים׃
	
	\kahal\barukhhashemsource
	
	\firstword{לְאֵל}
	בָּרוּךְ נְעִימוֹת יִתֵּֽנוּ \middot לְמֶֽלֶךְ אֵל חַי וְקַיָּם זְמִירוֹת יֹאמֵֽרוּ וְתֻשְׁבָּחוֹת יַשְׁמִֽיעוּ \middot כִּי הוּא לְבַדּוֹ פּוֹעֵל גְּבוּרוֹת עֹשֶׂה חֲדָשׁוֹת בַּֽעַל מִלְחָמוֹת זוֹרֵֽעַ צְדָקוֹת מַצְמִֽיחַ יְשׁוּעוֹת בּוֹרֵא רְפוּאוֹת נוֹרָא תְהִלּוֹת אֲדוֹן הַנִּפְלָאוֹת \middot הַמְחַדֵּשׁ בְּטוּבוֹ בְּכׇל־יוֹם תָּמִיד מַעֲשֵׂה בְרֵאשִׁית׃ כָּאָמוּר׃ \source{תהלים קלו}לְ֭עֹשֵׂה אוֹרִ֣ים גְּדֹלִ֑ים כִּ֖י לְעוֹלָ֣ם חַסְדּֽוֹ׃ אוֹר חָדָשׁ עַל־צִיּוֹן תָּאִיר וְנִזְכֶּה כֻלָּֽנוּ מְהֵרָה לְאוֹרוֹ׃ בָּרוּךְ אַתָּה יְיָ יוֹצֵר הַמְּאוֹרוֹת׃
}

\newcommand{\ahavaraba}{
	\englishinst{At \hebineng{קנפות ארבע} gather the front two tzitzi\thav.}
	\firstword{אַהֲבָה רַבָּה}
	אֲהַבְתָּֽנוּ יְיָ אֱלֹהֵֽינוּ חֶמְלָה גְּדוֹלָה וִיתֵירָה חָמַֽלְתָּ עָלֵֽינוּ׃ אָבִֽינוּ מַלְכֵּֽנוּ בַּעֲבוּר אֲבוֹתֵֽינוּ שֶׁבָּטְחוּ בְךָ וַתְּלַמְּדֵם חֻקֵּי חַיִּים כֵּן תְּחׇנֵּֽנוּ וּתְלַמְּדֵֽנוּ׃ אָבִֽינוּ הָאָב הָרַחֲמָן הַמְרַחֵם רַחֵם עָלֵֽינוּ וְתֵן בְּלִבֵּֽנוּ לְהָבִין וּלְהַשְׂכִּיל לִשְׁמֹֽעַ לִלְמֹד וּלְלַמֵּד לִשְׁמֹר וְלַעֲשׂוֹת וּלְקַיֵּם אֶת־כׇּל־דִּבְרֵי תַלְמוּד תּוֹרָתֶֽךָ בְּאַהֲבָה׃ וְהָאֵר עֵינֵֽינוּ בְּתוֹרָתֶֽךָ וְדַבֵּק לִבֵּֽנוּ בְּמִצְוֹתֶֽיךָ וְיַחֵד לְבָבֵֽנוּ לְאַהֲבָה וּלְיִרְאָה אֶת־שְׁמֶֽךָ וְלֹא נֵבוֹשׁ לְעוֹלָם וָעֶד׃ כִּי בְשֵׁם קׇדְשְׁךָ הַגָּדוֹל וְהַנּוֹרָא בָּטָֽחְנוּ נָגִֽילָה וְנִשְׂמְחָה בִּישׁוּעָתֶֽךָ׃ וַהֲבִיאֵֽנוּ לְשָׁלוֹם מֵאַרְבַּע כַּנְפוֹת הָאָֽרֶץ וְתוֹלִיכֵֽנוּ קוֹמְמִיּוּת לְאַרְצֵֽנוּ׃ כִּי אֵל פּוֹעֵל יְשׁוּעוֹת אַֽתָּה וּבָֽנוּ בָחַֽרְתָּ מִכׇּל־עַם וְלָשׁוֹן וְקֵרַבְתָּֽנוּ לְשִׁמְךָ הַגָּדוֹל סֶֽלָה בֶּאֱמֶת \middot לְהוֹדוֹת לְךָ וּלְיַחֶדְךָ בְּאַהֲבָה׃ בָּרוּךְ אַתָּה יְיָ הַבּוֹחֵר בְּעַמּוֹ יִשְׂרָאֵל בְּאַהֲבָה׃
}

\newcommand{\shema}{
	
	\englishinst{Recite the line below when praying privately:}
	(\instruction{יחיד אומר׃}
	 אֵל מֶֽלֶךְ נֶאֱמָן
	 )
	\\
	\begin{Large}
		\textbf{
			שְׁמַ֖ע יִשְׂרָאֵ֑ל יְיָ֥ אֱלֹהֵ֖ינוּ יְיָ֥ ׀ אֶחָֽד׃} \source{דברים ו}\\
	\end{Large}
	\begin{large}
		\instruction{בלחש:} \textbf{בָּרוּךְ שֵׁם כְּבוֹד מַלְכוּתוֹ לְעוֹלָם וָעֶד:}
	\end{large}
}

\newcommand{\veahavta}{
	\firstword{וְאָ֣הַבְתָּ֔}\source{דברים ו}
	אֵ֖ת יְיָ֣ אֱלֹהֶ֑יךָ בְּכׇל־לְבָבְךָ֥ וּבְכׇל־נַפְשְׁךָ֖ וּבְכׇל־מְאֹדֶֽךָ׃
	וְהָי֞וּ הַדְּבָרִ֣ים הָאֵ֗לֶּה אֲשֶׁ֨ר אָנֹכִ֧י מְצַוְּךָ֛ הַיּ֖וֹם עַל־לְבָבֶֽךָ׃
	וְשִׁנַּנְתָּ֣ם לְבָנֶ֔יךָ וְדִבַּרְתָּ֖ בָּ֑ם בְּשִׁבְתְּךָ֤ בְּבֵיתֶ֙ךָ֙ וּבְלֶכְתְּךָ֣ בַדֶּ֔רֶךְ וּֽבְשׇׁכְבְּךָ֖ וּבְקוּמֶֽךָ׃
	וּקְשַׁרְתָּ֥ם לְא֖וֹת עַל־יָדֶ֑ךָ וְהָי֥וּ לְטֹטָפֹ֖ת בֵּ֥ין עֵינֶֽיךָ׃
	וּכְתַבְתָּ֛ם עַל־מְזֻז֥וֹת בֵּיתֶ֖ךָ וּבִשְׁעָרֶֽיךָ׃
}

\newcommand{\vehaya}{
	\firstword{וְהָיָ֗ה}\source{דברים יא}
	אִם־שָׁמֹ֤עַ תִּשְׁמְעוּ֙ אֶל־מִצְוֺתַ֔י אֲשֶׁ֧ר אָנֹכִ֛י מְצַוֶּ֥ה אֶתְכֶ֖ם הַיּ֑וֹם לְאַהֲבָ֞ה אֶת־יְיָ֤ אֱלֹֽהֵיכֶם֙ וּלְעׇבְד֔וֹ בְּכׇל־לְבַבְכֶ֖ם וּבְכׇל־נַפְשְׁכֶֽם׃
	וְנָתַתִּ֧י מְטַֽר־אַרְצְכֶ֛ם בְּעִתּ֖וֹ יוֹרֶ֣ה וּמַלְק֑וֹשׁ וְאָסַפְתָּ֣ דְגָנֶ֔ךָ וְתִֽירֹשְׁךָ֖ וְיִצְהָרֶֽךָ׃
	וְנָתַתִּ֛י עֵ֥שֶׂב בְּשָׂדְךָ֖ לִבְהֶמְתֶּ֑ךָ וְאָכַלְתָּ֖ וְשָׂבָֽעְתָּ׃
	הִשָּֽׁמְר֣וּ לָכֶ֔ם פֶּ֥ן יִפְתֶּ֖ה לְבַבְכֶ֑ם וְסַרְתֶּ֗ם וַעֲבַדְתֶּם֙ אֱלֹהִ֣ים אֲחֵרִ֔ים וְהִשְׁתַּחֲוִיתֶ֖ם לָהֶֽם׃
	וְחָרָ֨ה אַף־יְיָ֜ בָּכֶ֗ם וְעָצַ֤ר אֶת־הַשָּׁמַ֙יִם֙ וְלֹֽא־יִהְיֶ֣ה מָטָ֔ר וְהָ֣אֲדָמָ֔ה לֹ֥א תִתֵּ֖ן אֶת־יְבוּלָ֑הּ וַאֲבַדְתֶּ֣ם מְהֵרָ֗ה מֵעַל֙ הָאָ֣רֶץ הַטֹּבָ֔ה אֲשֶׁ֥ר יְיָ֖ נֹתֵ֥ן לָכֶֽם׃
	וְשַׂמְתֶּם֙ אֶת־דְּבָרַ֣י אֵ֔לֶּה עַל־לְבַבְכֶ֖ם וְעַֽל־נַפְשְׁכֶ֑ם וּקְשַׁרְתֶּ֨ם אֹתָ֤ם לְאוֹת֙ עַל־יֶדְכֶ֔ם וְהָי֥וּ לְטוֹטָפֹ֖ת בֵּ֥ין עֵינֵיכֶֽם׃
	וְלִמַּדְתֶּ֥ם אֹתָ֛ם אֶת־בְּנֵיכֶ֖ם לְדַבֵּ֣ר בָּ֑ם בְּשִׁבְתְּךָ֤ בְּבֵיתֶ֙ךָ֙ וּבְלֶכְתְּךָ֣ בַדֶּ֔רֶךְ וּֽבְשׇׁכְבְּךָ֖ וּבְקוּמֶֽךָ׃
	וּכְתַבְתָּ֛ם עַל־מְזוּז֥וֹת בֵּיתֶ֖ךָ וּבִשְׁעָרֶֽיךָ׃
	לְמַ֨עַן יִרְבּ֤וּ יְמֵיכֶם֙ וִימֵ֣י בְנֵיכֶ֔ם עַ֚ל הָֽאֲדָמָ֔ה אֲשֶׁ֨ר נִשְׁבַּ֧ע יְיָ֛ לַאֲבֹתֵיכֶ֖ם לָתֵ֣ת לָהֶ֑ם כִּימֵ֥י הַשָּׁמַ֖יִם עַל־הָאָֽרֶץ׃
}

\newcommand{\vayomer}{
	\firstword{וַיֹּ֥אמֶר}\source{במדבר טו}
	יְיָ֖ אֶל־מֹשֶׁ֥ה לֵּאמֹֽר׃
	דַּבֵּ֞ר אֶל־בְּנֵ֤י יִשְׂרָאֵל֙ וְאָמַרְתָּ֣ אֲלֵהֶ֔ם וְעָשׂ֨וּ לָהֶ֥ם צִיצִ֛ת עַל־כַּנְפֵ֥י בִגְדֵיהֶ֖ם לְדֹרֹתָ֑ם וְנָ֥תְנ֛וּ עַל־צִיצִ֥ת הַכָּנָ֖ף פְּתִ֥יל תְּכֵֽלֶת׃
	וְהָיָ֣ה לָכֶם֮ לְצִיצִת֒ וּרְאִיתֶ֣ם אֹת֗וֹ וּזְכַרְתֶּם֙ אֶת־כׇּל־מִצְוֺ֣ת יְיָ֔ וַעֲשִׂיתֶ֖ם אֹתָ֑ם וְלֹֽא־תָת֜וּרוּ אַחֲרֵ֤י לְבַבְכֶם֙ וְאַחֲרֵ֣י עֵֽינֵיכֶ֔ם אֲשֶׁר־אַתֶּ֥ם זֹנִ֖ים אַחֲרֵיהֶֽם׃
	לְמַ֣עַן תִּזְכְּר֔וּ וַעֲשִׂיתֶ֖ם אֶת־כׇּל־מִצְוֺתָ֑י וִהְיִיתֶ֥ם קְדֹשִׁ֖ים לֵאלֹֽהֵיכֶֽם׃
	אֲנִ֞י יְיָ֣ אֱלֹֽהֵיכֶ֗ם אֲשֶׁ֨ר הוֹצֵ֤אתִי אֶתְכֶם֙ מֵאֶ֣רֶץ מִצְרַ֔יִם לִהְי֥וֹת לָכֶ֖ם לֵאלֹהִ֑ים אֲנִ֖י יְיָ֥ אֱלֹהֵיכֶֽם׃
}

\newcommand{\emesveyatziv}{
	\firstword{אֱמֶת}
	וְיַצִּיב וְנָכוֹן וְקַיָּם וְיָשָׁר וְנֶאֱמָן וְאָהוּב וְחָבִיב וְנֶחְמָד וְנָעִים וְנוֹרָא וְאַדִּיר וּמְתֻקָּן וּמְקֻבָּל וְטוֹב וְיָפֶה הַדָּבָר הַזֶּה עָלֵֽינוּ לְעוֹלָם וָעֶד׃ אֱמֶת אֱלֹהֵי עוֹלָם מַלְכֵּֽנוּ צוּר יַעֲקֹב מָגֵן יִשְׁעֵֽנוּ׃ לְדוֹר וָדוֹר הוּא קַיָּם וּשְׁמוֹ קַיָּם וְכִסְאוֹ נָכוֹן וּמַלְכוּתוֹ וֶאֱמוּנָתוֹ לָעַד קַיָּֽמֶת׃ וּדְבָרָיו חָיִים וְקַיָּמִים נֶאֱמָנִים וְנֶחֱמָדִים לָעַד וּלְעוֹלְמֵי עוֹלָמִים׃ עַל־אֲבוֹתֵֽינוּ וְעָלֵֽינוּ עַל־בָּנֵֽינוּ וְעַל־דּוֹרוֹתֵֽינוּ וְעַל כׇּל־דּוֹרוֹת זֶֽרַע יִשְׂרָאֵל עֲבָדֶֽיךָ׃
	
	\firstword{עַל־הָרִאשׁוֹנִים}
	וְעַל־הָאַחֲרוֹנִים דָּבָר טוֹב וְקַיָּם לְעוֹלָם וָעֶד \middot אֱמֶת וֶאֱמוּנָה חֹק וְלֹא יַעֲבוֹר׃ אֱמֶת שֶׁאַתָּה הוּא יְיָ אֱלֹהֵֽינוּ וֵאלֹהֵי אֲבוֹתֵֽינוּ מַלְכֵּֽנוּ מֶֽלֶךְ אֲבוֹתֵֽינוּ גּוֹאֲלֵֽנוּ גּוֹאֵל אֲבוֹתֵֽינוּ \middot יוֹצְרֵֽנוּ צוּר יְשׁוּעָתֵֽנוּ פּוֹדֵֽנוּ וּמַצִּילֵֽנוּ מֵעוֹלָם שְׁמֶֽךָ \middot אֵין אֱלֹהִים זוּלָתֶֽךָ׃
}


\newcommand{\ezrasavoseinu}{
	\firstword{עֶזְרַת}
	אֲבוֹתֵֽינוּ אַתָּה הוּא מֵעוֹלָם מָגֵן וּמוֹשִֽׁיעַ לִבְנֵיהֶם אַחֲרֵיהֶם בְּכׇל־דּוֹר וָדוֹר׃ בְּרוּם עוֹלָם מוֹשָׁבֶֽךָ וּמִשְׁפָּטֶֽיךָ וְצִדְקָתְךָ עַד־אַפְסֵי אָֽרֶץ׃ אַשְׁרֵי אִישׁ שֶׁיִּשְׁמַע לְמִצְוֹתֶֽיךָ וְתוֹרָתְךָ וּדְבָרְךָ יָשִׂים עַל־לִבּוֹ׃ אֱמֶת אַתָּה הוּא אָדוֹן לְעַמֶּֽךָ וּמֶֽלֶךְ גִּבּוֹר לָרִיב רִיבָם׃ אֱמֶת אַתָּה הוּא רִאשׁוֹן וְאַתָּה הוּא אַחֲרוֹן וּמִבַּלְעָדֶֽיךָ אֵין לָֽנוּ מֶֽלֶךְ גּוֹאֵל וּמוֹשִֽׁיעַ׃ מִמִּצְרַֽיִם גְּאַלְתָּֽנוּ יְיָ אֱלֹהֵֽינוּ וּמִבֵּית עֲבָדִים פְּדִיתָֽנוּ \middot כׇּל־בְּכוֹרֵיהֶם הָרַֽגְתָּ וּבְכוֹרְךָ גָּאַֽלְתָּ וְיַם־סוּף בָּקַֽעְתָּ וְזֵדִים טִבַּֽעְתָּ וִידִידִים הֶעֱבַֽרְתָּ
	\source{תהלים קו}וַיְכַסּוּ־מַ֥יִם צָרֵיהֶ֑ם אֶחָ֥ד מֵ֝הֶ֗ם לֹ֣א נוֹתָֽר׃
	עַל־זֹאת שִׁבְּחוּ אֲהוּבִים וְרוֹמְמוּ אֵל וְנָתְנוּ יְדִידִים זְמִירוֹת שִׁירוֹת וְתֻשְׁבָּחוֹת בְּרָכוֹת וְהוֹדָאוֹת לְמֶלֶךְ אֵל חַי וְקַיָּם. רָם וְנִשָּׂא גָּדוֹל וְנוֹרָא מַשְׁפִּיל גֵּאִים וּמַגְבִּֽיהַּ שְׁפָלִים מוֹצִיא אֲסִירִים וּפוֹדֶה עֲנָוִים וְעוֹזֵר דַּלִּים וְעוֹנֶה לְעַמּוֹ בְּעֵת שַׁוְּעָם אֵלָיו׃
	\englishinst{Stand here and take three steps back.}
	 תְּהִלּוֹת לְאֵל עֶלְיוֹן בָּרוּךְ הוּא וּמְבֹרָךְ \middot מֹשֶׁה וּבְנֵי יִשְׂרָאֵל לְךָ עָנוּ שִׁירָה בְּשִׂמְחָה רַבָּה וְאָמְרוּ כֻלָּם׃
}

\newcommand{\gaalyisroel}{
	\kahal \source{שמות טו}\textbf{מִֽי־כָמֹ֤כָה בָּֽאֵלִם֙ יְיָ֔ מִ֥י כָּמֹ֖כָה נֶאְדָּ֣ר בַּקֹּ֑דֶשׁ נוֹרָ֥א תְהִלֹּ֖ת עֹ֥שֵׂה פֶֽלֶא׃}
	
	שִׁירָה חֲדָשָׁה שִׁבְּחוּ גְאוּלִים לְשִׁמְךָ עַל־שְׂפַת הַיָּם יַֽחַד כֻּלָּם הוֹדוּ וְהִמְלִֽיכוּ וְאָמְרוּ׃
	
	\kahal \hashemyimloch\\
	\englishinst{Recite the blessing of \hebineng{ישראל גאל} together with the reader.}
	צוּר יִשְׂרָאֵל קֽוּמָה בְּעֶזְרַת יִשְׂרָאֵל וּפְדֵה כִנְאֻמֶֽךָ יְהוּדָה וְיִשְׂרָאֵל׃ גֹּאֲלֵ֕נוּ\source{ישעיהו מז} יְיָ֥ צְבָא֖וֹת שְׁמ֑וֹ קְד֖וֹשׁ יִשְׂרָאֵֽל׃ בָּרוּךְ אַתָּה יְיָ גָּאַל יִשְׂרָאֵל׃
}

%\newcommand{\ayt}{ בעשי״ת׃}
\newcommand{\ayt}{ בעשי״ת}

\newcommand{\amidaopening}[2]{
	\englishinst{Take three steps forward while reciting the following verse.  Bend the knees and bow for the words \hebineng{ברוך} and \hebineng{אתה} respectively at the beginning and end of the first blessing.}
	אֲ֭דֹנָי\source{תהלים נא} שְׂפָתַ֣י תִּפְתָּ֑ח וּ֝פִ֗י יַגִּ֥יד תְּהִלָּתֶֽךָ׃\\
	\firstword{בָּרוּךְ}
	אַתָּה יְיָ אֱלֹהֵֽינוּ וֵאלֹהֵי אֲבוֹתֵֽינוּ \middot אֱלֹהֵי אַבְרָהָם אֱלֹהֵי יִצְחָק וֵאלֹהֵי יַעֲקֹב \middot הָאֵל הַגָּדוֹל הַגִּבּוֹר וְהַנּוֹרָא אֵל עֶלְיוֹן גּוֹמֵל חֲסָדִים טוֹבִים וְקוֹנֵה הַכֹּל \middot
	וְזוֹכֵר חַסְדֵי אָבוֹת וּמֵבִיא גוֹאֵל לִבְנֵי בְנֵיהֶם לְמַֽעַן שְׁמוֹ בְּאַהֲבָה׃
	\vspace{-12pt}
\begin{footnotesize}\begin{Center}
	(\instruction{#1} 
זׇכְרֵֽנוּ לְחַיִּים מֶֽלֶךְ חָפֵץ בַּחַיִּים \middot\\
וְכׇתְבֵֽנוּ בְּסֵפֶר הַחַיִּים לְמַעַנְךָ אֱלֹהִים חַיִּים׃)
\end{Center}\end{footnotesize}
\vspace{-24pt}
\begin{Center}
\firstword{מֶֽלֶךְ}
עוֹזֵר וּמוֹשִֽׁיעַ וּמָגֵן׃ בָּרוּךְ אַתָּה יְיָ מָגֵן אַבְרָהָם׃
	
\firstword{אַתָּה}
גִּבּוֹר לְעוֹלָם אֲדֹנָי מְחַיֵּה מֵתִים אַתָּה רַב לְהוֹשִֽׁיעַ \middot
\end{Center}
\vspace{-16pt}
\begin{Center}
	\englishinst{From Shemini Atzeret till Pesach:}
	%\instruction{ממוסף של שמיני עצרת עד מוסף יום א׳ של פסח אומרים׃}\\
מַשִּׁיב הָרוּחַ וּמורִיד הַגָּֽשֶׁם \middot
\end{Center}
\begin{justify}
\firstword{מְכַלְכֵּל}
חַיִּים בְּחֶֽסֶד מְחַיֵּה מֵתִים בְּרַחֲמִים רַבִּים סוֹמֵךְ נוֹפְלִים וְרוֹפֵא חוֹלִים וּמַתִּיר אֲסוּרִים וּמְקַיֵּם אֱמוּנָתוֹ לִישֵׁנֵי עָפָר \middot מִי כָמֽוֹךָ בַּֽעַל גְּבוּרוֹת וּמִי דּֽוֹמֶה לָּךְ מֶֽלֶךְ מֵמִית וּמְחַיֶּה וּמַצְמִֽיחַ יְשׁוּעָה׃\end{justify}
	\begin{small}(\instruction{#1}
		מִי כָמֽוֹךָ אַב הָרַחֲמִים זוֹכֵר יְצוּרָיו לְחַיִּים בְּרַחֲמִים׃)\end{small}\\
\firstword{וְנֶאֱמָן}
	אַתָּה לְהַחֲיוֹת מֵתִים׃ בָּרוּךְ אַתָּה יְיָ מְחַיֵּה הַמֵּתִים׃
\begin{Center} 
#2
\firstword{אַתָּה}
	קָדוֹשׁ וְשִׁמְךָ קָדוֹשׁ וּקְדוֹשִׁים בְּכׇל־יוֹם יְהַלְלוּךָ סֶּֽלָה׃ בָּרוּךְ אַתָּה יְיָ *הָאֵל
(*\instruction{#1}
	הַמֶּֽלֶךְ)
	הַקָּדוֹשׁ׃\end{Center}
}

\newcommand{\weekdaysakedusha}{
\begin{Center}\ssubsection{\adforn{48} קדושה \adforn{22}}\end{Center}
	
	\begin{small}
		%\setlength{\LTpost}{0pt}
		\begin{longtable}{l p{.90\textwidth}}
			
			\shatz &
			נְקַדֵּשׁ אֶת־שִׁמְךָ בָּעוֹלָם כְּשֵׁם שֶׁמַּקְדִּישִׁים אוֹתוֹ בִּשְׁמֵי מָרוֹם כַּכָּתוּב עַל יַד נְבִיאֶךָ קָרָ֨א זֶ֤ה אֶל־זֶה֙ וְאָמַ֔ר׃\\
			
			\shatzvkahal &
			\kadoshkadoshkadosh\\
			
			\shatz &
			לְעֻמָּתָם בָּרוּךְ יֹאמֵרוּ׃\\
			
			\shatzvkahal &
			\barukhhashem\\
			
			\shatz &
			וּבְדִבְרֵי קׇדְשְׁךָ כָּתוּב לֵאמֹר׃ \\
			
			\shatzvkahal &
			\yimloch\\
			
			\shatz &
			לְדוֹר וָדוֹר נַגִּיד גׇּדְלֶךָ וּלְנֵצַח נְצָחִים קְדֻשָּׁתְךָ נַקְדִּישׁ \middot וְשִׁבְחֲךָ אֱלֹהֵֽינוּ מִפִּינוּ לֹא יָמוּשׁ לְעוֹלָם וָעֶד \middot כִּי אֵל מֶלֶךְ גָּדוֹל וְקָדוֹשׁ אַֽתָּה: בָּרוּךְ אַתָּה יְיָ *הָאֵל (*\instruction{בעשי״ת:} הַמֶּֽלֶךְ) הַקָּדוֹשׁ:
		\end{longtable}	\end{small}\vspace{-\baselineskip}
}


\newcommand{\weekdaysakiddushhashem}{
	\firstword{אַתָּה}
	קָדוֹשׁ וְשִׁמְךָ קָדוֹשׁ וּקְדוֹשִׁים בְּכׇל־יוֹם יְהַלְלוּךָ סֶּֽלָה׃ בָּרוּךְ אַתָּה יְיָ *הָאֵל
	(*\instruction{בעשי״ת:}
	הַמֶּֽלֶךְ)
	הַקָּדוֹשׁ׃
}


\newcommand{\weekdaysabinah}{
	\firstword{אַתָּה חוֹנֵן}
	לְאָדָם דַּֽעַת וּמְלַמֵּד לֶאֱנוֹשׁ בִּינָה \middot חׇנֵּֽנוּ מֵאִתְּךָ בִּינָה דֵּעָה וְהַשְׂכֵּל׃ בָּרוּךְ אַתָּה יְיָ חוֹנֵן הַדָּֽעַת׃
}

\newcommand{\weekdaysateshuva}{
	\firstword{הֲשִׁיבֵֽנוּ}
	אָבִֽינוּ לְתוֹרָתֶֽךָ וְקָרְבֵֽנוּ מַלְכֵּֽנוּ לַעֲבוֹדָתֶֽךָ וְהַחֲזִירֵֽנוּ בִתְשׁוּבָה שְׁלֵמָה לְפָנֶֽיךָ׃ בָּרוּךְ אַתָּה יְיָ הָרוֹצֶה בִּתְשׁוּבָה׃
}

\newcommand{\weekdaysaselichah}{
	\englishinst{Strike the chest with the right fist at the words \hebineng{חטאנו} and \hebineng{פשענו}.}
	\firstword{סְלַח}
	לָֽנוּ אָבִֽינוּ כִּי חָטָֽאנוּ מְחַל לָֽנוּ מַלְכֵּֽנוּ כִּי פָשָֽׁעְנוּ \middot כִּי מוֹחֵל וְסוֹלֵֽחַ אַֽתָּה׃ בָּרוּךְ אַתָּה יְיָ חַנּוּן הַמַּרְבֶּה לִסְלֽוֹחַ׃
}

\newcommand{\weekdaysageulah}{
	\firstword{רְאֵה}
	בְעׇנְיֵֽנוּ וְרִיבָה רִיבֵֽנוּ וּגְאָלֵֽנוּ מְהֵרָה לְמַֽעַן שְׁמֶֽךָ כִּי גוֹאֵל חָזָק אַֽתָּה׃ בָּרוּךְ אַתָּה יְיָ גּוֹאֵל יִשְׂרָאֵל׃
}

\newcommand{\aneinubasetext}{
	עֲנֵֽנוּ יְיָ עֲנֵֽנוּ בְּיוֹם צוֹם תַּעֲנִיתֵֽנוּ כִּי בְצָרָה גְדוֹלָה אֲנָֽחְנוּ \middot אַל תֵּֽפֶן אֶל רִשְׁעֵֽנוּ וְאַל תַּסְתֵּר פָּנֶֽיךָ מִמֶּֽנּוּ וְאַל תִּתְעַלַּם מִתְּחִנָּתֵֽנוּ׃ הֱיֵה־נָא קָרוֹב לְשַׁוְעָתֵֽנוּ \middot יְהִי־נָא חַסְדְּךָ לְנַחֲמֵֽנוּ טֶֽרֶם נִקְרָא אֵלֶֽיךָ עֲנֵֽנוּ כַּדָּבָר שֶׁנֶּאֱמַר׃ \mdsource{ישעיה סה}וְהָיָ֥ה טֶֽרֶם־יִקְרָ֖אוּ וַֽאֲנִ֣י אֶעֱנֶ֑ה ע֛וֹד הֵ֥ם מְדַבְּרִ֖ים וַֽאֲנִ֥י אֶשְׁמָֽע׃ כִּי אַתָּה יְיָ הָעוֹנֶה בְּעֵת צָרָה פּוֹדֶה וּמַצִּיל בְּכׇל־עֵת צָרָה וְצוּקָה׃}

\newcommand{\weekdaysaanneinu}{
	\begin{sometimes}
		
		\begin{small}
			\englishinst{On a public fast during the repetition of the Amidah, the leader includes the following:}
			%\instruction{בת״צ הש״ץ אומר עננו:}
			\aneinubasetext
			 בָּרוּךְ אַתָּה יְיָ הָעוֹנֶה בְּעֵת צָרָה׃
			
		\end{small}
	\end{sometimes}
}


\newcommand{\weekdaysarefuah}{
	\firstword{רְפָאֵֽנוּ}
	יְיָ וְנֵרָפֵא הוֹשִׁיעֵֽנוּ וְנִוָּשֵֽׁעָה כִּי תְהִלָּתֵֽנוּ אַֽתָּה \middot וְהַעֲלֵה רְפוּאָה שְׁלֵמָה לְכׇל־מַכּוֹתֵֽינוּ
	\footnote{
		\englishinst{In private prayer, individuals can add a prayer for a specific sick person:}
		יהִי רָצוֹן מִלְּפָנֶֽיךָ יְיָ אֱלֹהֵֽינוּ שֶׁתִּשְׁלַח מְהֵרָה רְפוּאָה שְׁלֵמָה מִן הַשָּׁמַֽיִם רְפוּאַת הַנֶּֽפֶשׁ וּרְפוּאַת הַגּוּף לְחוֹלֶה/לְחוֹלָה/לְחוֹלִים (\instruction{פב״פ}) בְּתוֹךְ שְׁאָר חוֹלֵי יִשְׂרָאֵל:\\
	}
	כִּי אֵל מֶֽלֶךְ רוֹפֵא נֶאֱמָן וְרַחֲמָן אַֽתָּה׃ בָּרוּךְ אַתָּה יְיָ רוֹפֵא חוֹלֵי עַמּוֹ יִשְׂרָאֵל׃
}

\newcommand{\weekdaysaberacha}{
	\firstword{בָּרֵךְ}
	עָלֵֽינוּ יְיָ אֱלֹהֵֽינוּ אֶת־הַשָּׁנָה הַזֹּאת וְאֶת־כׇּל־מִינֵי תְבוּאָתָהּ לְטוֹבָה \middot וְתֵן (\instruction{בקיץ:}
	\textbf{בְּרָכָה})
	(\instruction{בחורף:}
	\textbf{טַל וּמָטָר לִבְרָכָה})
	עַל פְּנֵי הָאֲדָמָה וְשַׂבְּעֵֽנוּ מִטּוּבָהּ וּבָרֵךְ שְׁנוֹתֵֽנוּ כַּשָּׁנִים הַטּוֹבוֹת: בָּרוּךְ אַתָּה יְיָ מְבָרֵךְ הַשָּׁנִים:
}

%\footnote{\instruction{נ״א}: מִטּוּבֶֽךָ}
\newcommand{\weekdaysashofar}{
	\firstword{תְּקַע}
	בְּשׁוֹפָר גָּדוֹל לְחֵרוּתֵֽנוּ וְשָׂא נֵס לְקַבֵּץ גָּלֻיּוֹתֵֽינוּ \middot וְקַבְּצֵֽנוּ יַֽחַד מֵאַרְבַּע כַּנְפוֹת הָאָֽרֶץ׃ בָּרוּךְ אַתָּה יְיָ מְקַבֵּץ נִדְחֵי עַמּוֹ יִשְׂרָאֵל׃
}

\newcommand{\weekdaysamishpat}{
	\firstword{הָשִֽׁיבָה}
	שׁוֹפְטֵֽינוּ כְּבָרִאשׁוֹנָה וְיוֹעֲצֵֽינוּ כְּבַתְּחִלָּה וְהָסֵר מִמֶּֽנּוּ יָגוֹן וַאֲנָחָה \middot וּמְלוֹךְ עָלֵֽינוּ אַתָּה יְיָ לְבַדְּךָ בְּחֶֽסֶד וּבְרַחֲמִים וְצַדְּקֵֽנוּ בַּמִּשְׁפָּט׃ בָּרוּךְ אַתָּה יְיָ *מֶֽלֶךְ אוֹהֵב צְדָקָה וּמִשְׁפָּט
	(*\instruction{בעשי״ת:}
	הַמֶּֽלֶךְ הַמִּשְׁפָּט)׃
}


%וְלַמְשֻׁמָּדִים אַל תְּהִי תִקְוָה וְכׇל־הַמִּינִים כְּרֶֽגַע יֹאבֵֽדוּ וְכׇל־אוֹיְבֵי עַמְּךָ מְהֵרָה יִכָּרֵֽתוּ \middot וּמַלְכוּת זָדוֹן מְהֵרָה תְעַקֵּר וּתְשַׁבֵּר וּתְמַגֵּר וְתַכְנִֽיעַ כׇּל־אוֹיְבֵינוּ בִּמְהֵרָה בְיָמֵֽינוּ׃


\newcommand{\weekdaysaminim}{
	\firstword{וְלַמַּלְשִׁינִים}
	אַל תְּהִי תִקְוָה וְכׇל־הָרִשְׁעָה כְּרֶגַע תֺּאבֵד וְכׇל־אוֹיְבֶֽיךָ מְהֵרָה יִכָּרֵתוּ \middot וְהַזֵדִים מְהֵרָה תְעַקֵּר וּתְשַׁבֵּר וּתְמַגֵּר וְתַכְנִיעַ בִּמְהֵרָה בְיָמֵינוּ׃ בָּרוּךְ אַתָּה יְיָ שׁוֹבֵר אוֹיְבִים וּמַכְנִיעַ זֵדִים׃
}



\newcommand{\weekdaysatzadikim}{
	\firstword{עַל הַצַּדִּיקִים}
	וְעַל הַחֲסִידִים וְעַל זִקְנֵי עַמְּךָ בֵּית יִשְׂרָאֵל וְעַל פְּלֵיטַת סוֹפְרֵיהֶם וְעַל גֵּרֵי הַצֶּֽדֶק וְעָלֵֽינוּ \middot יֶהֱמוּ־נָא רַחֲמֶֽיךָ יְיָ אֱלֹהֵֽינוּ \middot וְתֵן שָׂכָר טוֹב לְכֹל הַבּוֹטְחִים בְּשִׁמְךָ בֶּאֱמֶת וְשִׂים חֶלְקֵֽנוּ עִמָּהֶם \middot וּלְעוֹלָם לֹא נֵבוֹשׁ כִּי בְךָ בָּטָֽחְנוּ׃ בָּרוּךְ אַתָּה יְיָ מִשְׁעָן וּמִבְטָח לַצַּדִּיקִים׃
}

\newcommand{\weekdaysayerushelayim}{
	\firstword{וְלִיְרוּשָׁלַ‍ִם}
	עִירְךָ בְּרַחֲמִים תָּשׁוּב וְתִשְׁכּוֹן בְּתוֹכָהּ כַּאֲשֶׁר דִּבַּֽרְתָּ \middot וּבְנֵה אוֹתָהּ בְּקָרוֹב בְּיָמֵֽינוּ בִּנְיַן עוֹלָם וְכִסֵּא דָוִד מְהֵרָה לְתוֹכָהּ תָּכִין׃
	בָּרוּךְ אַתָּה יְיָ בּוֹנֵה יְרוּשָׁלַ‍ִם׃
}

\newcommand{\yerushwithnachem}{\firstword{וְלִירוּשָׁלַ‍ִם}
	עִירְךָ בְּרַחֲמִים תָּשׁוּב וְתִשְׁכּוֹן בְּתוֹכָהּ כַּאֲשֶׁר דִּבַּֽרְתָּ \middot וּבְנֵה אוֹתָהּ בְּקָרוֹב בְּיָמֵֽינוּ בִּנְיַן עוֹלָם וְכִסֵּא דָוִד מְהֵרָה לְתוֹכָהּ תָּכִין׃
	\footnote{
		\instruction{בתשעה באב במנחה׃ }
		\textbf{נַחֵם}
		יְיָ אֱלֹהֵֽינוּ אֶת־אֲבֵלֵי צִיּוֺן וְאֶת־אֲבֵלֵי יְרוּשָׁלַֽ֔֗͏ִם וְאֶת־הָעִיר הָאֲבֵלָה וְהֶחֳרֵבָה וְהַבְּזוּיָה וְהַשׁוֺמֵמָה \middot הָאֲבֵלָה מִבְּלִי בָנֱיהָ וְהֶחֳרֵבָה מִמְּעוֺנוֺתֶֽיהָ וְהַבְּזוּיָה מִכְּבוֺדָהּ וְהַשׁוֺמֵמָה מֵאֵין יוֺשֵׁב \middot וְהִיא יוֺשֶֽׁבֶת וְרֹאשָׁה חָפוּי בְּאִשָׁה עֲקַרָה שֶׁלֹּא יָלֳדָה \middot וַיְבַלְְּעֽוּהָ לִגְיוֺנוֺת וַיְּירָשׁוּהָ עוֺבְדֵי פְסִילִים \middot וַיָטִֽילוּ אֶת־עַמְּךָ יִשְׂרָאֵל לֶחָרֱֽב וַיַּהַרְגוּ בְזָדוֺן חֲסִידֵי עֶלְיוֺן \middot עַל־כֵּן צִיּוֺן בְּמַר תִּבְכֶּה וִירוּשָׁלַֽ͏ִם תִּתֵּן קוֺלָהּ \middot לִבִּי לִבִּי עַל חַלְלֵיהֶם מֵעַי מֵעַי עַל חַלְלֵיהֶם \middot כִּי אַתָּה יְיָ בָּאֵשׁ הִצַּתָּהּ וּבָאֵשׁ אַתָּה עָתִיד לִבְנוֺתָה, כָּאָמוּר׃
		\mdsource{זכריה ב}%
		וַֽאֲנִ֤י אֶֽהְיֶה־לָּהּ֙ נְאֻם־יְיָ֔ ח֥וֹמַת אֵ֖שׁ סָבִ֑יב וּלְכָב֖וֹד אֶֽהְיֶ֥ה בְתוֹכָֽהּ׃
		בָּרוּךְ אַתָּה יְיָ מְנַחֵם צִיוֺן וּבֹנֵה יְרוּשָׁלַֽ͏ִם׃ \instruction{את צמח וכו׳}\\
	}
	בָּרוּךְ אַתָּה יְיָ בּוֹנֵה יְרוּשָׁלַ‍ִם׃}

\newcommand{\weekdaysamalchus}{
	\firstword{אֶת צֶֽמַח}
	דָּוִד עַבְדְּךָ מְהֵרָה תַצְמִֽיחַ וְקַרְנוֹ תָּרוּם בִּישׁוּעָתֶֽךָ \middot כִּי לִישׁוּעָתְךָ קִוִּֽינוּ כׇּל־הַיּוֹם׃ בָּרוּךְ אַתָּה יְיָ מַצְמִֽיחַ קֶֽרֶן יְשׁוּעָה׃
}

\newcommand{\weekdaysashemakoleinu}[1]{
	\firstword{שְׁמַע קוֹלֵֽנוּ}
	יְיָ אֱלֹהֵֽינוּ חוּס וְרַחֵם עָלֵֽינוּ וְקַבֵּל בְּרַחֲמִים וּבְרָצוֹן אֶת־תְּפִלָּתֵֽנוּ \middot כִּי אֵל שׁוֹמֵעַ תְּפִלּוֹת וְתַחֲנוּנִים אַֽתָּה וּמִלְּפָנֶֽיךָ מַלְכֵּֽנוּ רֵיקָם אַל תְּשִׁיבֵֽנוּ #1
	כִּי אַתָּה שׁוֹמֵֽעַ תְּפִלַּת עַמְּךָ יִשְׂרָאֵל בְּרַחֲמִים׃ בָּרוּךְ אַתָּה יְיָ שׁוֹמֵֽעַ תְּפִלָּה׃
}

\newcommand{\retzeh}{
	\firstword{רְצֵה}
	יְיָ אֱלֹהֵֽינוּ בְּעַמְּךָ יִשְׂרָאֵל וּבִתְפִלָּתָם וְהָשֵׁב הָעֲבוֹדָה לִדְבִיר בֵּיתֶֽךָ \middot וְאִשֵּׁי יִשְׂרָאֵל וּתְפִלָּתָם בְּאַהֲבָה תְקַבֵּל בְּרָצוֹן וּתְהִי לְרָצוֹן תָּמִיד עֲבוֹדַת יִשְׂרָאֵל עַמֶּֽךָ \middot
}

\newcommand{\yaalehveyavo}{
	
	\begin{sometimes}
		
		\instruction{בראש חודש וחול המועד:}\\
		\yaalehveyavotemplate{\\\begin{tabular}{c|c|c}
				רֹאשׁ הַחֹֽדֶשׁ & חַג הַמַּצוֹת & חַג הַסֻּכּוֹת
		\end{tabular}\\}
	
	\end{sometimes}
	
}

\newcommand{\yaalehveyavotemplate}[1]{
\firstword{אֱלֹהֵֽינוּ}
וֵאלֹהֵי אֲבוֹתֵֽינוּ \middot יַעֲלֶה וְיָבֹא וְיַגִּיעַ וְיֵרָאֶה וְיֵרָצֶה וְיִשָּׁמַע וְיִפָּקֵד וְיִזָּכֵר זִכְרוֹנֵֽנוּ וּפִקְדּוֹנֵֽנוּ וְזִכְרוֹן אֲבוֹתֵֽינוּ \middot וְזִכְרוֹן מָשִׁיחַ בֶּן דָּוִד עַבְדֶּֽךָ וְזִכְרוֹן יְרוּשָׁלַ‍ִם עִיר קׇדְשֶֽׁךָ וְזִכְרוֹן כׇּל־עַמְּךָ בֵּית יִשְׂרָאֵל לְפָנֶיךָ \middot לִפְלֵיטָה וּלְטוֹבָה וּלְחֵן וּלְחֶֽסֶד וּלְרַחֲמִים וּלְחַיִּים וּלְשָׁלוֹם בְּיוֹם #1 הַזֶּה \middot זׇכְרֵֽנּוּ יְיָ אֱלֹהֵֽינוּ בּוֹ לְטוֹבָה וּפׇקְדֵֽנוּ בוֹ לִבְרָכָה וְהוֹשִׁיעֵֽנוּ בוֹ לְחַיִּים \middot וּבִדְבַר יְשׁוּעָה וְרַחֲמִים חוּס וְחׇׇׇׇנֵּנוּ וְרַחֵם עָלֵֽינוּ וְהוֹשִׁיעֵֽנוּ כִּי אֵלֶֽיךָ עֵינֵֽינוּ כִּי אֵל מֶֽלֶךְ חַנּוּן וְרַחוּם אַֽתָּה׃
}


\newcommand{\zion}{
	\firstword{וְתֶחֱזֶֽינָה}
	עֵינֵֽינוּ בְּשׁוּבְךָ לְצִיּוֹן בְּרַחֲמִים׃ בָּרוּךְ אַתָּה יְיָ הַמַּחֲזִיר שְׁכִינָתוֹ לְצִיּוֹן׃
}

\newcommand{\modim}{
	\columnratio{0.43}
	\setlength{\columnsep}{2em}
	
	\smallskip
	\englishinst{Bow for the first five words of the following paragraph.  Recite the paragraph on the right during private prayer.  During the repetition of the Amidah, the congregation recites the paragraph on the left while the reader says the paragraph on the right.}
	\begin{paracol}{2}
		
		\begin{small}
			מוֹדִים אֲנַֽחְנוּ לָךְ שָׁאַתָּה הוּא יְיָ אֱלֹהֵֽינוּ וֵאלֹהֵי אֲבוֹתֵֽינוּ \middot אֱלֹהֵי כׇל־בָּשָׂר יוֹצְרֵֽנוּ יוֹצֵר בְּרֵאשִׁית \middot בְּרָכוֹת וְהוֹדָאוֹת לְשִׁמְךָ הַגָּדוֹל וְהַקָּדוֹשׁ עַל שֶׁהֶחֱיִיתָֽנוּ וְקִיַּמְתָּֽנוּ \middot כֵּן תְּחַיֵּֽנוּ וּתְקַיְּמֵֽנוּ וְתֶאֱסוֹף גָּלֻיּוֹתֵֽינוּ לְחַצְרֹת קׇדְשֶֽׁךָ לִשְׁמֹר חֻקֶּֽיךָ וְלַעֲשׂוֹת רְצֹנֶֽךָ וּלְעׇבְדְּךָ בְּלֵבָב שָׁלֵם \middot עַל שֶׁאֲנַֽחְנוּ מוֹדִים לָךְ \middot בָּרוּךְ אֵל הַהוֹדָאוֹת׃
			
		\end{small}
		
		\switchcolumn
		
		\firstword{מוֹדִים}
		אֲנַֽחְנוּ לָךְ שָׁאַתָּה הוּא יְיָ אֱלֹהֵֽינוּ וֵאלֹהֵי אֲבוֹתֵֽינוּ לְעוֹלָם וָעֶד \middot צוּר חַיֵּֽינוּ מָגֵן יִשְׁעֵֽנוּ אַתָּה הוּא לְדוֹר וָדוֹר \middot נוֹדֶה לְךָ וּנְסַפֵּר תְּהִלָּתֶֽךָ עַל חַיֵּֽינוּ הַמְּסוּרִים בְּיָדֶֽךָ וְעַל נִשְׁמוֹתֵֽינוּ הַפְּקוּדוֹת לָךְ \middot וְעַל נִסֶּֽיךָ שֶׁבְּכׇל־יוֹם עִמָּֽנוּ וְעַל נִפְלְאוֹתֶֽיךָ וְטוֹבוֹתֶֽיךָ שֶׁבְּכׇל־עֵת עֶֽרֶב וָבֹֽקֶר וְצׇהֳרָֽיִם \middot הַטּוֹב כִּי לֹא כָלוּ רַחֲמֶֽיךָ וְהַמְרַחֵם כִּי לֹא תַֽמּוּ חֲסָדֶֽיךָ מֵעוֹלָם קִוִֽינוּ לָךְ׃
		
	\end{paracol}
}

\newcommand{\bimeimatityahu}{
		בִּימֵי מַתִּתְיָֽהוּ בֶּן יוֹחָנָן כֹּהֵן גָּדוֹל חַשְׁמֹנַי וּבָנָיו \middot כְּשֶׁעָמְדָה מַלְכוּת יָוָן הָרְשָׁעָה עַל עַמְּךָ יִשְׂרָאֵל לְהַשְׁכִּיחָם תּוֹרָתֶֽךָ וּלְהַעֲבִירָם מֵחֻקֵּי רְצוֹנֶֽךָ׃ וְאַתָּה בְּרַחֲמֶֽיךָ הָרַבִּים עָמַֽדְתָּ לָהֶם בְּעֵת צָרָתָם רַֽבְתָּ אֶת־רִיבָם דַּֽנְתָּ אֶת־דִּינָם נָקַֽמְתָּ אֶת־נִקְמָתָם \middot מָסַֽרְתָּ גִּבּוֹרִים בְּיַד חַלָּשִׁים וְרַבִּים בְּיַד מְעַטִּים וּטְמֵאִים בְּיַד טְהוֹרִים וּרְשָׁעִים בְּיַד צַדִּיקִים וְזֵדִים בְּיַד עוֹסְקֵי תוֹרָתֶֽךָ׃ וּלְךָ עָשִֽׂיתָ שֵׁם גָּדוֹל וְקָדוֹשׁ בְּעוֹלָמֶֽךָ וּלְעַמְּךָ יִשְׂרָאֵל עָשִֽׂיתָ תְּשׁוּעָה גְדוֹלָה וּפֻרְקָן כְּהַיּוֹם הַזֶּה׃ וְאַֽחַר כַּךְ בָּֽאוּ בָנֶֽיךָ לִדְבִיר בֵּיתֶֽךָ וּפִנּוּ אֶת־הֵיכָלֶֽךָ וְטִהֲרוּ אֶת־מִקְדָּשֶֽׁךָ \middot וְהִדְלִֽיקוּ נֵרוֹת בְּחַצְרוֹת קׇדְּשֶֽׁךָ וְקָבְעוּ שְׁמוֹנַת יְמֵי חֲנֻכָּה אֵֽלּוּ לְהוֹדוֹת לְהַלֵּל לְשִׁמְךָ הַגָּדוֹל׃
}

\newcommand{\alhanisim}{
	
	\begin{sometimes}
		
		%\vspace{-7mm}
		\columnratio{0.62}
		\begin{paracol}{2}[
			\englishinst{On \d{H}anukka and Purim:}
			\firstword{עַל הַנִּסִּים}
			וְעַל הַפֻּרְקָן וְעַל הַגְּבוּרוֹת וְעַל הַתְּשׁוּעוֹת וְעַל הַמִּלְחָמוֹת שֶׁעָשִֽׂיתָ לַאֲבוֹתֵֽינוּ בַּיָּמִים הָהֵם בַּזְּמַן הַזֶּה׃
			]
			\instruction{בחנוכה:} \bimeimatityahu
		
			\switchcolumn
			\instruction{בפורים:}
			בִּימֵי מׇרְדְּכַי וְִאֶסְתֵּר בְּשׁוּשַׁן הַבִּירָה כְּשֶׁעָמַד עֲלֵיהֶם הָמָן הָרָשָׁע בִּקֵּשׁ
			\mdsource{אסתר ג}
			לְהַשְׁמִ֡יד לַֽהֲרֹ֣ג וּלְאַבֵּ֣ד אֶת־כׇּל־הַ֠יְּהוּדִים מִנַּ֨עַר וְעַד־זָקֵ֨ן טַ֤ף וְנָשִׁים֙ בְּי֣וֹם אֶחָ֔ד בִּשְׁלוֹשָׁ֥ה עָשָׂ֛ר לְחֹ֥דֶשׁ שְׁנֵים־עָשָׂ֖ר הוּא־חֹ֣דֶשׁ אֲדָ֑ר וּשְׁלָלָ֖ם לָבֽוֹז׃ וְאַתָּה בְּרַחֲמֶֽיךָ הָרַבִּים הֵפַֽרְתָּ אֶת־עֲצָתוֹ וְקִלְקַלְתָּ אֶת־מַחֲשַׁבְתּוֹ וַהֲשֵׁבֽוֹתָ גְּמוּלוֹ בְּרֹאשׁוֹ וְתָלוּ אוֹתוֹ וְאֶת־בָּנָיו עַל הָעֵץ׃
		\end{paracol}
	\end{sometimes}
}



\newcommand{\weekdaysahodos}{
	\firstword{וְעַל כֻּלָּם}
	יִתְבָּרַךְ וְיִתְרוֹמַם שִׁמְךָ מַלְכֵּֽנוּ תָּמִיד לְעוֹלָם וָעֶד \middot
	
	\begin{small}
		
		\instruction{בעשי״ת:}
		וּכְתוֹב לְחַיִּים טוֹבִים בְּנֵי בְרִיתֶֽךָ \middot
		
	\end{small}
	\englishinst{Bend the knees and bow when saying \hebineng{ברוך} and \hebineng{אתה} respectively for the blessing in the following paragraph}
	\firstword{וְכׇל־הַחַיִּים}
	יוֹדֽוּךָ סֶּֽלָה וִיהַלְלוּ אֶת־שִׁמְךָ בֶּאֱמֶת \middot הָאֵל יְשׁוּעָתֵֽנוּ וְעֶזְרָתֵֽנוּ סֶֽלָה׃ בָּרוּךְ אַתָּה יְיָ הַטּוֹב שִׁמְךָ וּלְךָ נָאֶה לְהוֹדוֹת׃
	
	
	
	%\columnratio{0.52}
	%\begin{paracol}{2}
	%\instruction{כהנים לעצמם בלחש:}\\
	%יְהִי רָצוׂן מִלְּפָנֶֽךָ יְיָ אֱלֹהֵֽינוּ וֵאלֹהֵי אֲבוׂתֵֽינוּ שֶׁתְּהִי הַבְּרָכָה הַזֹּאת שֶׁצִּוִּיתָנוּ לְבָרֵךְ אֶת־עַמְּךָ יִשׂרָאֵל בְּרָכָה שְׁלֵמָה. וְלֹא יִהְיֶה בָּה מִכְשׁוֹל וְעָוׂן מֵעַתָּה וְעַד עוׂלָם׃
	
	%\switchcolumn
}




\newcommand{\bircaskohanimnl}[2]{
	\begin{small}
		
		\vspace{-.5\baselineskip}\rule[-0.5ex]{2in}{1pt}
		
		\instruction{#1}
		
		%\columnratio{0.52}
		%\begin{paracol}{2}
		%\instruction{כהנים לעצמם בלחש:}\\
		%יְהִי רָצוׂן מִלְּפָנֶֽךָ יְיָ אֱלֹהֵֽינוּ וֵאלֹהֵי אֲבוׂתֵֽינוּ שֶׁתְּהִי הַבְּרָכָה הַזֹּאת שֶׁצִּוִּיתָנוּ לְבָרֵךְ אֶת־עַמְּךָ יִשׂרָאֵל בְּרָכָה שְׁלֵמָה. וְלֹא יִהְיֶה בָּה מִכְשׁוֹל וְעָוׂן מֵעַתָּה וְעַד עוׂלָם׃
		
		%\switchcolumn
		
		\shatz \firstword{אֱלֹהֵֽינוּ}
		וֵאלֹהֵי אֲבוֹתֵֽינוּ בָּרְכֵֽנוּ בַּבְּרָכָה הַמְשֻׁלֶּֽשֶׁת בַּתּוֹרָה \middot הַכְּתוּבָה עַל יְדֵי מֹשֶׁה עַבְדֶּֽךָ \middot הָאֲמוּרָה מִפִּי אַהֲרֹן וּבָנָיו כֹּהֲנִים
		
		\instruction{ש״ץ וקהל:}
		\textbf{עַם קְדוֹשֶֽׁךָ כָּאָמוּר׃}
		%\end{paracol}
		
		\instruction{כהנים:}
		בָּרוּךְ אַתָּה יְיָ אֱלֹהֵֽינוּ מֶֽלֶךְ הָעוֹלָם \middot אֲשֶׁר קִדְּשָֽׁנוּ בִּקְדֻשָּׁתוֹ שֶׁל אַהֲרֹן וְצִוָּֽנוּ לְבָרֵךְ אֶת־עַמּוֹ יִשְׂרָאֵל בְּאַהֲבָה׃

		\begin{large}
			
			\textbf{
				יְבָֽרֶכְךָ֥\source{במדבר ו} יְיָ֖ וְיִשְׁמְרֶֽךָ׃\\
				יָאֵ֨ר יְיָ֧ פָּנָ֛יו אֵלֶ֖יךָ וִֽיחֻנֶּֽךָּ׃\\
				יִשָּׂ֨א יְיָ֤ פָּנָיו֙ אֵלֶ֔יךָ וְיָשֵׂ֥ם לְךָ֖ שָׁלֽוֹם׃
			}
			
		\end{large}
		
		\columnratio{0.6}
		\begin{paracol}{2}
			\instruction{כהנים:}
			רִבּוֹן הָעוֹלָם עָשִֽׂינוּ מַה שֶּׁגָּזַֽרְתָּ עָלֵֽינוּ אַף אַתָּה עֲשֵׂה עִמָּֽנוּ כַּאֲשֶׁר הִבְטַחְתָּֽנוּ׃ הַשְׁקִ֩יפָה֩ מִמְּע֨וֹן קׇדְשְׁךָ֜ מִן־הַשָּׁמַ֗יִם וּבָרֵ֤ךְ אֶֽת־עַמְּךָ֙ אֶת־יִשְׂרָאֵ֔ל וְאֵת֙ הָֽאֲדָמָ֔ה אֲשֶׁ֥ר נָתַ֖תָּה לָ֑נוּ כַּֽאֲשֶׁ֤ר נִשְׁבַּ֨עְתָּ֙ לַֽאֲבֹתֵ֔ינוּ אֶ֛רֶץ זָבַ֥ת חָלָ֖ב וּדְבָֽשׁ׃
			
			\switchcolumn
			
			\kahal
			אַדִּיר בַּמָּרוֹם שׁוֹכֵן בִּגְבוּרָה אַתָּה שָׁלוֹם וְשִׁמְךָ שָׁלוֹם׃ יְהִי רָצוֹן שֶׁתָּשִׂים עָלֵֽינוּ וְעַל כׇּל־עַמְּךָ בֵּית יִשְׂרָאֵל חַיִּים וּבְרָכָה לְמִשְׁמֶֽרֶת שָׁלוֹם׃
		\end{paracol}
		
		\sepline
		
		\instruction{#2}\\
		אֱלֹהֵֽינוּ וֵאלֹהֵי אֲבוֹתֵֽינוּ בָּרְכֵֽנוּ בַּבְּרָכָה הַמְשֻׁלֶּֽשֶׁת בַּתּוֹרָה \middot
		הַכְּתוּבָה עַל יְדֵי מֹשֶׁה עַבְדֶּֽךָ \middot הָאֲמוּרָה מִפִּי אַהֲרֹן וּבָנָיו כֹּהֲנִים עַם קְדוֹשֶֽׁךָ כָּאָמוּר׃
		
		יְבָֽרֶכְךָ֥\source{במדבר ו} יְיָ֖ וְיִשְׁמְרֶֽךָ׃ \hfill \kahal כֵּן יְהִי רָצוׂן \\
		יָאֵ֨ר יְיָ֧ פָּנָ֛יו אֵלֶ֖יךָ וִֽיחֻנֶּֽךָּ׃ \hfill \kahal כֵּן יְהִי רָצוׂן \\
		יִשָּׂ֨א יְיָ֤ פָּנָיו֙ אֵלֶ֔יךָ וְיָשֵׂ֥ם לְךָ֖ שָׁלֽוֹם׃ \hfill \kahal כי״ר
		
	\end{small}
}

\newcommand{\tachanunim}{
	\firstword{אֱלֹהַי}
	נְצֹר לְשׁוֹנִי מֵרָע וּשְׂפָתַי מִדַּבֵּר מִרְמָה וְלִמְקַלְלַי נַפְשִׁי תִדּוֹם וְנַפְשִׁי כֶּעָפָר לַכֹּל תִּהְיֶה׃ פְּתַח לִבִּי בְּתוֹרָתֶֽךָ וּבְמִצְוֹתֶֽיךָ תִּרְדּוֹף נַפְשִׁי \middot וְכֹל הַחוֹשְׁבִים עָלַי רָעָה מְהֵרָה הָפֵר עֲצָתָם וְקַלְקֵל מַחֲשַׁבְתָם׃ עֲשֵׂה לְמַֽעַן שְׁמֶֽךָ עֲשֵׂה לְמַֽעַן יְמִינֶֽךָ עֲשֵׂה לְמַֽעַן קְדֻשָּׁתֶֽךָ עֲשֵׂה לְמַֽעַן תּוֹרָתֶֽךָ׃ לְ֭מַעַן \source{תהלים ס}יֵחָֽלְצ֥וּן יְדִידֶ֑יךָ הֽוֹשִׁ֖יעָה יְמִֽינְךָ֣ וַֽעֲנֵֽנִי׃ \source{תהלים יט}יִֽהְי֥וּ לְרָצ֨וֹן אִמְרֵי־פִ֡י וְהֶגְי֣וֹן לִבִּ֣י לְפָנֶ֑יךָ יְ֜יָ֗ צוּרִ֥י וְגֹֽאֲלִֽי׃
	\osehshalom
	
	
	\begin{small}
		
		יְהִי רָצוֹן מִלְּפָנֶֽיךָ יְיָ אֱלֹהֵֽינוּ וִֵאלֹהֵי אֲבוֹתֵֽינוּ שֶׁיִבָּנֶה בֵּית הַמִּקְדָּשׁ בִּמְהֵרָה בְיָמֵֽינוּ וְתֵן חֶלְקֵֽנוּ בְּתוֹרָתֶֽךָ׃ וְשָׁם נַעֲבׇדְךָ בְּיִרְאָה כִּימֵי עוֹלָם וּכְשָׁנִים קַדְמֹנִיּוֹת׃
		וְעָֽרְבָה֙ \source{מלאכי ג}לַֽיְיָ֔ מִנְחַ֥ת יְהוּדָ֖ה וִירוּשָׁלָ֑םִ כִּימֵ֣י עוֹלָ֔ם וּכְשָׁנִ֖ים קַדְמֹֽנִיּֽוֹת׃
		
		
	\end{small}
}

\newcommand{\shatzbirkaskohanim}[1]{
	
	\begin{narrow}
		
		\instruction{#1}
		אֱלֹהֵֽינוּ וֵאלֹהֵי אֲבוֹתֵֽינוּ בָּרְכֵֽנוּ בַּבְּרָכָה הַמְשֻׁלֶּֽשֶׁת בַּתּוֹרָה
		הַכְּתוּבָה עַל יְדֵי מֹשֶׁה עַבְדֶּֽךָ הָאֲמוּרָה מִפִּי אַהֲרֹן וּבָנָיו כֹּהֲנִים עַם קְדוֹשֶֽׁךָ כָּאָמוּר׃
		
		יְבָֽרֶכְךָ֥\source{במדבר ו} יְיָ֖ וְיִשְׁמְרֶֽךָ׃ \hfill \kahal כֵּן יְהִי רָצוׂן \\
		יָאֵ֨ר יְיָ֧ ׀ פָּנָ֛יו אֵלֶ֖יךָ וִֽיחֻנֶּֽךָּ׃ \hfill \kahal כֵּן יְהִי רָצוׂן \\
		יִשָּׂ֨א יְיָ֤ ׀ פָּנָיו֙ אֵלֶ֔יךָ וְיָשֵׂ֥ם לְךָ֖ שָׁלֽוֹם׃ \hfill \kahal כי״ר
	\end{narrow}
}

\newcommand{\mishnahtamid}{
\firstword{הַשִּׁיר שֶׁהַלְוִיִּם}\source{תמיד פ״ז}
הָיוּ אוֹמְרִים בְּבֵית הַמִּקְדָּשׁ׃
בַּיּוֹם הַרִאשׁוֹן הָיוּ אוֹמְרִים \source{תהלים כד}%
לְדָוִ֗ד מִ֫זְמ֥וֹר לַֽייָ֭ הָאָ֣רֶץ וּמְלוֹאָ֑הּ תֵּ֝בֵ֗ל וְיֹ֣שְׁבֵי בָֽהּ׃
בַּשֵּׁנִי הָיוּ אוֹמְרִים \source{תהלים מח}
גָּ֘ד֤וֹל יְיָ֣ וּמְהֻלָּ֣ל מְאֹ֑ד בְּעִ֥יר אֱ֝לֹהֵ֗ינוּ הַר־קׇדְשֽׁוֹ׃
בַּשְּׁלִישִׁי הָיוּ אוֹמְרִים \source{תהלים פב}
אֱֽלֹהִ֗ים נִצָּ֥ב בַּעֲדַת־אֵ֑ל בְּקֶ֖רֶב אֱלֹהִ֣ים יִשְׁפֹּֽט׃
בָּרְבִיעִי הָיוּ אוֹמְרִים \source{תהלים צד}
אֵל־נְקָמ֥וֹת יְיָ֑ אֵ֖ל נְקָמ֣וֹת הוֹפִֽיעַ׃
בַּחֲמִישִׁי הָיוּ אוֹמְרִים \source{תהלים פא}
הַ֭רְנִינוּ לֵאלֹהִ֣ים עוּזֵּ֑נוּ הָ֝רִ֗יעוּ לֵאלֹהֵ֥י יַעֲקֹֽב׃
בַּשִּׁשִּׁי הָיוּ אוֹמְרִים \source{תהלים צג}
יְיָ֣ מָלָךְ֮ גֵּא֢וּת לָ֫בֵ֥שׁ לָבֵ֣שׁ יְיָ֭ עֹ֣ז הִתְאַזָּ֑ר אַף־תִּכּ֥וֹן תֵּ֝בֵ֗ל בַּל־תִּמּֽוֹט׃
בַּשַׁבָּת הָיוּ אוֹמְרִים \source{תהלים צב}
מִזְמ֥וֹר שִׁ֗יר לְי֣וֹם הַשַּׁבָּֽת׃
מִזְמוֹר שִׁיר לֶעָתִיד לָבוֹא לְיוֹם שֶׁכֻּלּוֹ שַׁבָּת וּמְנוּחָה לְחַיֵּי הָעוֹלָמִים׃
}

\newcommand{\einkeloheinu}{
 	אֵין כֵּאלֹהֵֽינוּ\hfill אֵין כַּאדוֹנֵֽנוּ \hfill אֵין כְּמַלְכֵּֽנוּ \hfill אֵין כְּמוֹשִׁיעֵֽנוּ׃\\
 	מִי כֵאלֹהֵֽינוּ \hfill מִי כַאדוֹנֵֽנוּ \hfill מִי כְמַלְכֵּֽנוּ \hfill מִי כְמוֹשִׁיעֵֽנוּ׃\\
 	נוֹדֶה לֵאלֹהֵֽינוּ \hfill נוֹדֶה לַאדוֹנֵֽנוּ \hfill נוֹדֶה לְמַלְכֵּֽנוּ \hfill נוֹדֶה לְמוֹשִׁיעֵֽנוּ׃\\
 	בָּרוּךְ אֱלֹהֵֽינוּ \hfill בָּרוּךְ אֲדוֹנֵֽנוּ \hfill בָּרוּךְ מַלְכֵּֽנוּ \hfill בָּרוּךְ מוֹשִׁיעֵֽנוּ׃\\
 	\hfill 
 	אַתָּה הוּא אֱלֹהֵֽינוּ\hfill אַתָּה הוּא אֲדוֹנֵֽנוּ\hfill\\\hfill אַתָּה הוּא מַלְכֵּֽנוּ\hfill אַתָּה הוּא מוֹשִׁיעֵֽנוּ׃\hfill 
 	אַתָּה הוּא שֶׁהִקְטִֽירוּ אֲבוֹתֵֽינוּ לְפָנֶֽיךָ אֶת־קְטֹֽרֶת הַסַּמִּים׃
}

\newcommand{\conclusionshabYT}{
\einkeloheinu
	
\pitumhaketoret

\mishnahtamid

\sofberakhot

\rabbiskaddish

\aleinu
}

\newcommand{\sofberakhot}{
	\firstword{אָמַר רַבִּי אֱלְעָזָר}\source{ברכות סד}
	אָמַר רַבִּי חֲנִינָא׃ תַּלְמִידֵי חֲכָמִים מַרְבִּים שָׁלוֹם בָּעוֹלָם שֶׁנֶּאֱמַר׃ וְכׇל־בָּנַ֖יִךְ \source{ישעיה נד}לִמּוּדֵ֣י יְיָ֑ וְרַ֖ב שְׁל֥וֹם בָּנָֽיִךְ׃ אַל תִּקְרָא בָּנַֽיִךְ אֶלָּא בּוֹנַֽיִךְ׃ שָׁל֣וֹם רָ֭ב \source{תהלים קיט}לְאֹהֲבֵ֣י תוֹרָתֶ֑ךָ וְאֵֽין־לָ֥מוֹ מִכְשֽׁוֹל׃ יְהִי־שָׁל֥וֹם \source{תהלים קכב}בְּחֵילֵ֑ךְ שַׁ֝לְוָ֗ה בְּאַרְמְנוֹתָֽיִךְ׃ לְ֭מַעַן אַחַ֣י וְרֵעָ֑י אֲדַבְּרָה־נָּ֖א שָׁל֣וֹם בָּֽךְ׃ לְ֭מַעַן בֵּית־יְיָ֣ אֱלֹהֵ֑ינוּ אֲבַקְשָׁ֖ה ט֣וֹב לָֽךְ׃ יְיָ֗ \source{תהלים כט} עֹ֭ז לְעַמּ֣וֹ יִתֵּ֑ן יְיָ֓ ׀ יְבָרֵ֖ךְ אֶת־עַמּ֣וֹ בַשָּׁלֽוֹם׃}

\newcommand{\barukhbayom}{}

\newcommand{\yerueinnu}{
	
	\firstword{יִרְאוּ}
	עֵינֵֽינוּ וְיִשְׂמַח לִבֵּֽנוּ וְתָגֵל נַפְשֵֽׁנוּ בִּישׁוּעָתְךָ בֶּאֱמֶת בֶּאֱמֹר לְצִיּוֹן מָלַךְ אֱלֹהָֽיִךְ׃
	\melekhmalakhyimlokh
	כִּי הַמַּלְכוּת שֶׁלְּךָ הִיא וּלְעֽוֹלְמֵי עַד תִּמְלֹךְ בְּכָבוֹד כִּי אֵין לָֽנוּ מֶֽלֶךְ אֶלָּא אַֽתָּה׃
}

\newcommand{\boruchhashemleolam}{
	\firstword{בָּר֖וּךְ} \source{תהלים פט}יְיָ֥ לְ֝עוֹלָ֗ם אָ֘מֵ֥ן ׀ וְאָמֵֽן׃
	בָּ֘ר֤וּךְ \source{תהלים קלה}יְיָ֨ ׀ מִצִּיּ֗וֹן שֹׁ֘כֵ֤ן יְֽרוּשָׁלָ֗‍ִם הַֽלְלוּ־יָֽהּ׃
	בָּר֤וּךְ \source{תהלים עב}׀ יְיָ֣ אֱ֭לֹהִים אֱלֹהֵ֣י יִשְׂרָאֵ֑ל עֹשֵׂ֖ה נִפְלָא֣וֹת לְבַדּֽוֹ׃ וּבָר֤וּךְ ׀ שֵׁ֥ם כְּבוֹד֗וֹ לְע֫וֹלָ֥ם וְיִמָּלֵ֣א כְ֭בוֹדוֹ אֶת־כֹּ֥ל הָאָ֗רֶץ אָ֘מֵ֥ן ׀ וְאָמֵֽן׃
	יְהִ֤י \source{תהלים קד}כְב֣וֹד יְיָ֣ לְעוֹלָ֑ם יִשְׂמַ֖ח יְיָ֣ בְּמַעֲשָֽׂיו׃
	יְהִ֤י \source{תהלים קיג}שֵׁ֣ם יְיָ֣ מְבֹרָ֑ךְ מֵ֝עַתָּ֗ה וְעַד־עוֹלָֽם׃
	כִּ֠י \source{שמ״א יב}לֹֽא־יִטֹּ֤שׁ יְיָ֙ אֶת־עַמּ֔וֹ בַּעֲב֖וּר שְׁמ֣וֹ הַגָּד֑וֹל כִּ֚י הוֹאִ֣יל יְיָ֔ לַעֲשׂ֥וֹת אֶתְכֶ֛ם ל֖וֹ לְעָֽם׃
	וַיַּרְא֙ \source{מ״א יח}כׇּל־הָעָ֔ם וַֽיִּפְּל֖וּ עַל־פְּנֵיהֶ֑ם וַיֹּ֣אמְר֔וּ יְיָ֙ ה֣וּא הָאֱלֹהִ֔ים יְיָ֖ ה֥וּא הָאֱלֹהִֽים׃
	\source{זכריה יד}וְהָיָ֧ה יְיָ֛ לְמֶ֖לֶךְ עַל־כׇּל־הָאָ֑רֶץ בַּיּ֣וֹם הַה֗וּא יִהְיֶ֧ה יְיָ֛ אֶחָ֖ד וּשְׁמ֥וֹ אֶחָֽד׃
	\source{תהלים לג}יְהִי־חַסְדְּךָ֣ יְיָ֣ עָלֵ֑ינוּ כַּ֝אֲשֶׁ֗ר יִחַ֥לְנוּ לָֽךְ׃
	\source{תהלים קו}הוֹשִׁיעֵ֨נוּ ׀ יְ֘יָ֤ אֱלֹהֵ֗ינוּ וְקַבְּצֵנוּ֮ מִֽן־הַגּ֫וֹיִ֥ם לְ֭הֹדוֹת לְשֵׁ֣ם קׇדְשֶׁ֑ךָ לְ֝הִשְׁתַּבֵּ֗חַ בִּתְהִלָּתֶֽךָ׃
	כׇּל־גּוֹיִ֤ם \source{תהלים פו}׀ אֲשֶׁ֥ר עָשִׂ֗יתָ יָב֤וֹאוּ ׀ וְיִשְׁתַּֽחֲו֣וּ לְפָנֶ֣יךָ אֲדֹנָ֑י וִ֖יכַבְּד֣וּ לִשְׁמֶֽךָ׃ כִּֽי־גָד֣וֹל אַ֭תָּה וְעֹשֵׂ֣ה נִפְלָא֑וֹת אַתָּ֖ה אֱלֹהִ֣ים לְבַדֶּֽךָ׃
	וַאֲנַ֤חְנוּ \source{תהלים עט}עַמְּךָ֨ ׀ וְצֹ֥אן מַרְעִיתֶךָ֮ נ֤וֹדֶ֥ה לְּךָ֗ לְע֫וֹלָ֥ם לְד֥וֹר וָדֹ֑ר נְ֝סַפֵּ֗ר תְּהִלָּתֶֽךָ׃
	בָּרוּךְ יְיָ בַּיּוֹם. בָּרוּךְ יְיָ בַּלָּֽיְלָה׃ בָּרוּךְ יְיָ בְּשׇׁכְבֵֽנוּ. בָּרוּךְ יְיָ בְּקוּמֵֽנוּ׃ כִּי בְיָדְךָ נַפְשׁוֹת הַחַיִּים וְהַמֵּתִים׃
	אֲשֶׁ֣ר \source{איוב יב}בְּ֭יָדוֹ נֶ֣פֶשׁ כׇּל־חָ֑י וְ֝ר֗וּחַ כׇּל־בְּשַׂר־אִֽישׁ׃
	בְּיָדְךָ֮ \source{תהלים לא}אַפְקִ֢יד ר֫וּחִ֥י פָּדִ֖יתָ אוֹתִ֥י יְיָ֗ אֵ֣ל אֱמֶֽת׃
	אֱלֹהֵֽינוּ שֶׁבַּשָּׁמַֽיִם יַחֵד שִׁמְךָ וְקַיֵּם מַלְכוּתְךָ תָּמִיד וּמְלֹךְ עָלֵֽינוּ לְעוֹלָם וָעֶד׃
	\yerueinnu
	בָּרוּךְ אַתָּה יְיָ הַמֶּֽלֶךְ בִּכְבוֹדוֹ תָּמִיד יִמְלוֹךְ עָלֵֽינוּ לְעוֹלָם וָעֶד וְעַל כׇּל־מַעֲשָׂיו׃
}

\newcommand{\aleinu}{
	\firstword{עָלֵֽינוּ}
	לְשַׁבֵּחַ לַאֲדוֹן הַכֹּל \middot לָתֵת גְּדֻלָּה לְיוֹצֵר בְּרֵאשִׁית׃ שֶׁלֹּא עָשָׂנוּ כְּגוֹיֵי הָאֲרָצוֹת \middot וְלֹא שָׂמָנוּ כְּמִשְׁפְּחוֹת הָאֲדָמָה׃ שֶׁלֹּא שָׂם חֶלְקֵנוּ כָּהֶם \middot וְגוֹרָלֵנוּ כְּכׇל־הֲמוֹנָם׃ [שֶׁהֵם מִשְׁתַּחֲוִים לְהֶבֶל וָרִיק \middot וּמִתְפַּלֲּלִים אֶל אֵל לֹא יוֹשִׁיעַ׃] וַאֲנַחְנוּ כּוֹרְעִים וּמִשְׁתַּחֲוִים וּמוֹדִים לִפְנֵי מֶלֶךְ מַלְכֵי הַמְּלָכִים הַקָּדוֹשׁ בָּרוּךְ הוּא׃ שֶׁהוּא נוֹטֶה שָׁמַיִם וְיֹסֵד אָרֶץ \middot וּמוֹשַׁב יְקָרוֹ בַּשָּׁמַיִם מִמַּעַל \middot וּשְׁכִינַת עֻזּוֹ בְּגׇבְהֵי מְרוֹמִים׃ הוּא אֱלֹהֵינוּ אֵין עוֹד \middot אֱמֶת מַלְכֵּנוּ אֶפֶס זוּלָתוֹ׃ כַּכָּתוּב בְּתּוֹרָתוֹ׃ וְיָדַעְתָּ֣
	\source{דברים ד}
	הַיּ֗וֹם וַהֲשֵׁבֹתָ֮ אֶל־לְבָבֶ֒ךָ֒ כִּ֤י יְיָ֙ ה֣וּא הָֽאֱלֹהִ֔ים בַּשָּׁמַ֣יִם מִמַּ֔עַל וְעַל־הָאָ֖רֶץ מִתָּ֑חַת אֵ֖ין עֽוֹד׃\\
	עַל כֵּן נְקַוֶּה לְךָ יְיָ אֱלֹהֵינוּ לִרְאוֹת מְהֵרָה בְּתִפְאֶרֶת עֻזֶּךָ \middot לְהַעֲבִיר גִּלּוּלִים מִן הָאָרֶץ וְהָאֱלִילִים כָּרוֹת יִכָּרֵתוּן \middot לְתַקֵּן עוֹלָם בְּמַלְכוּת שַׁדַּי, וְכׇל־בְּנֵי בָשָׂר יִקְרְאוּ בִשְׁמֶךָ \middot לְהַפְנוֹת אֵלֶיךָ כָּל־רִשְׁעֵי אָרֶץ \middot יַכִּירוּ וְיֵדְעוּ כָּל־יוֹשְׁבֵי תֵבֵל \middot כִּי לְךָ תִכְרַע כָּל־בֶּרֶךְ תִּשָּׁבַע כָּל־לָשׁוֹן׃ לְפָנֶיךָ יְיָ אֱלֹהֵינוּ יִכְרְעוּ וְיִפֹּלוּ וְלִכְבוֹד שִׁמְךָ יְקָר יִתֵּנוּ \middot וִיקַבְּלוּ כֻלָּם אֶת־עֹל מַלְכוּתֶךָ וְתִמְלֹךְ עֲלֵיהֶם מְהֵרָה לְעוֹלָם וָעֶד׃ כִּי הַמַּלְכוּת שֶׁלְּךָ הִיא וּלְעוֹלְמֵי עַד תִּמְלֹךְ בְּכָבוֹד׃ כַּכָּתוּב בְּתוֹרָתֶךָ׃\source{שמות טו} יְיָ֥ ׀ יִמְלֹ֖ךְ לְעֹלָ֥ם וָעֶֽד׃ וְנֶאֱמַר׃\source{זכריה יד} וְהָיָ֧ה יְיָ֛ לְמֶ֖לֶךְ עַל־כׇּל־הָאָ֑רֶץ בַּיּ֣וֹם הַה֗וּא יִהְיֶ֧ה יְיָ֛ אֶחָ֖ד וּשְׁמ֥וֹ אֶחָֽד׃
	%\firstword{אַל־תִּ֭ירָא}\source{משלי ג}
	%מִפַּ֣חַד פִּתְאֹ֑ם וּמִשֹּׁאַ֥ת רְ֝שָׁעִ֗ים כִּ֣י תָבֹֽא׃\source{ישעיה ח}
	%עֻ֥צוּ עֵצָ֖ה וְתֻפָ֑ר דַּבְּר֤וּ דָבָר֙ וְלֹ֣א יָק֔וּם כִּ֥י עִמָּ֖נוּ אֵֽל׃\source{ישעיה מו}
	%וְעַד־זִקְנָה֙ אֲנִ֣י ה֔וּא וְעַד־שֵׂיבָ֖ה אֲנִ֣י אֶסְבֹּ֑ל אֲנִ֤י עָשִׂ֙יתִי֙ וַאֲנִ֣י אֶשָּׂ֔א וַאֲנִ֥י אֶסְבֹּ֖ל וַאֲמַלֵּֽט׃
	
}

\newcommand{\shabboshodos}{
	\firstword{וְעַל כֻּלָּם}
	יִתְבָּרַךְ וְיִתְרוֹמַם שִׁמְךָ מַלְכֵּֽנוּ תָּמִיד לְעוֹלָם וָעֶד׃
	
	\instruction{בשבת שובה:}
	וּכְתוֹב לְחַיִּים טוֹבִים בְּנֵי בְרִיתֶֽךָ׃
	
	\firstword{וְכׇל־הַחַיִּים}
	יוֹדֽוּךָ סֶּֽלָה וִיהַלְלוּ אֶת־שִׁמְךָ בֶּאֱמֶת הָאֵל יְשׁוּעָתֵֽנוּ וְעֶזְרָתֵֽנוּ סֶֽלָה׃ בָּרוּךְ אַתָּה יְיָ הַטּוֹב שִׁמְךָ וּלְךָ נָאֶה לְהוֹדוֹת׃
}

\newcommand{\hamaarivaravim}{
	\firstword{בָּרוּךְ}
	אַתָּה יְיָ אֱלֹהֵֽינוּ מֶֽלֶךְ הָעוֹלָם אֲשֶׁר בִּדְבָרוֹ מַעֲרִיב עֲרָבִים בְּחׇכְמָה פּוֹתֵֽחַ שְׁעָרִים וּבִתְבוּנָה מְשַׁנֶּה עִתִּים וּמַחֲלִיף אֶת־הַזְּמַנִּים וּמְסַדֵּר אֶת־הַכּוֹכָבִים בְּמִשְׁמְרוֹתֵֽיהֶם בָּרָקִֽיעַ כִּרְצוֹנוֹ׃ בּוֹרֵא יוֹם וָלָֽיְלָה גּוֹלֵל אוֹר מִפְּנֵי חֹֽשֶׁךְ וְחֹֽשֶׁךְ מִפְּנֵי אוֹר׃ וּמַעֲבִיר יוֹם וּמֵֽבִיא לָֽיְלָה וּמַבְדִּיל בֵּין יוֹם וּבֵין לָֽיְלָה יְיָ צְבָאוֹת שְׁמוֹ׃ אֵל חַי וְקַיָּם תָּמִיד יִמְלוֹךְ עָלֵֽינוּ לְעוֹלָם וָעֶד׃ בָּרוּךְ אַתָּה יְיָ הַמַּעֲרִיב עֲרָבִים׃
}

\newcommand{\ahavasolam}{
	\firstword{אַהֲבַת}
	עוֹלָם בֵּית יִשְׂרָאֵל עַמְּךָ אָהַבְתָּ \middot תּוֹרָה וּמִצְוֹת חֻקִּים וּמִשְׁפָּטִים אוֹתָֽנוּ לִמַֽדְתָּ׃ עַל כֵּן יְיָ אֱלֹהֵֽינוּ בְּשׇׁכְבֵּֽנוּ וּבְקוּמֵֽנוּ נָשִֽׂיחַ בְּחֻקֶּיךָ וְנִשְׂמַח בְּדִבְרֵי תוֹרָתֶֽךָ וּבְמִצְוֹתֶֽיךָ לְעוֹלָם וָעֶד׃ כִּי הֵם חַיֵּֽינוּ וְאֹֽרֶךְ יָמֵֽינוּ וּבָהֶם נֶהְגֶּה יוֹמָם וָלָֽיְלָה \middot וְאַהֲבָתְךָ אַל תָּסִיר מִמֶּֽנּוּ לְעוֹלָמִים׃ בָּרוּךְ אַתָּה יְיָ אוֹהֵב עַמּוֹ יִשְׂרָאֵל׃
}

\newcommand{\emesveemuna}{
	%\instruction{הש״ץ אומר אמת בקול רם:}\\
	\firstword{אֱמֶת}
	וֶאֱמוּנָה כׇּל־זֹאת וְקַיָּם עָלֵֽינוּ כִּי הוּא יְיָ אֱלֹהֵֽינוּ וְאֵין זוּלָתוֹ וַאֲנַֽחְנוּ יִשְׂרָאֵל עַמּוֹ׃ הַפּוֹדֵֽנוּ מִיַּד מְלָכִים מַלְכֵּֽנוּ הַגּוֹאֲלֵֽנוּ מִכַּף כׇּל־הֶעָרִיצִים \middot הָאֵל הַנִּפְרָע לָֽנוּ מִצָּרֵֽנוּ וְהַמְשַׁלֵּם גְּמוּל לְכׇל־אוֹיְבֵי נַפְשֵֽׁנוּ׃ \source{איוב ט}עֹשֶׂ֣ה גְ֭דֹלוֹת עַד־אֵ֣ין חֵ֑קֶר וְנִפְלָא֗וֹת עַד־אֵ֥ין מִסְפָּֽר׃ \source{תהלים סו}הַשָּׂ֣ם נַ֭פְשֵׁנוּ בַּחַיִּ֑ים וְלֹֽא־נָתַ֖ן לַמּ֣וֹט רַגְלֵֽנוּ׃ הַמַּדְרִיכֵֽנוּ עַל בָּמוֹת אוֹיְבֵֽינוּ וַיָּֽרֶם קַרְנֵֽנוּ עַל כׇּל־שׂנְאֵֽינוּ׃ הָעֹֽשֶׂה לָּֽנוּ נִסִּים וּנְקָמָה בְּפַרְעֹה אוֹתֹת וּמוֹפְתִים בְּאַדְמַת בְּנֵי חָם \middot הַמַּכֶּה בְעֶבְרָתוֹ כׇּל־בְּכוֹרֵי מִצְרָֽיִם וַיּוֹצֵא אֶת־עַמּוֹ יִשְׂרָאֵל מִתּוֹכָם לְחֵרוּת עוֹלָם׃ הַמַּעֲבִיר בָּנָיו בֵּין גִּזְרֵי יַם סוּף אֶת־רוֹדְפֵיהֶם וְאֶת־שׂוֹנְאֵיהֶם בִּתְהוֹמוֹת טִבַּע׃ וְרָאוּ בָנָיו גְּבוּרָתוֹ שִׁבְּחוּ וְהוֹדוּ לִשְׁמוֹ׃ וּמַלְכוּתוֹ בְּרָצוֹן קִבְּלוּ עַלֵיהֶם \middot מֹשֶׁה וּבְנֵי יִשְׂרָאֵל לְךָ עָנוּ שִׁירָה בְּשִׂמְחָה רַבָּה וְאָמְרוּ כֻלָּם׃
	
	
	\kahal\source{שמות טו}\textbf{%
		מִֽי־כָמֹ֤כָה בָּֽאֵלִם֙ יְיָ֔ מִ֥י כָּמֹ֖כָה נֶאְדָּ֣ר בַּקֹּ֑דֶשׁ נוֹרָ֥א תְהִלֹּ֖ת עֹ֥שֵׂה פֶֽלֶא׃
	}
	
	
	מַלְכוּתְךָ רָאוּ בָנֶֽיךָ בּוֹקֵֽעַ יָם לִפְנֵי משֶׁה זֶ֤ה אֵלִי֙ עָנוּ וְאָמְרוּ׃
	
	\kahal \hashemyimloch
	
	
	וְנֶאֱמַר׃ \source{ירמיה לא}כִּֽי־פָדָ֥ה יְיָ֖ אֶֽת־יַעֲקֹ֑ב וּגְאָל֕וֹ מִיַּ֖ד חָזָ֥ק מִמֶּֽנּוּ׃ בָּרוּךְ אַתָּה יְיָ גָּאַל יִשְׂרָאֵל׃
}

\newcommand{\maarivmodim}{
	\firstword{מוֹדִים}
	אֲנַֽחְנוּ לָךְ שָׁאַתָּה הוּא יְיָ אֱלֹהֵֽינוּ וֵאלֹהֵי אֲבוֹתֵֽינוּ לְעוֹלָם וָעֶד צוּר חַיֵּֽינוּ מָגֵן יִשְׁעֵֽנוּ אַתָּה הוּא לְדוֹר וָדוֹר׃ נוֹדֶה לְךָ וּנְסַפֵּר תְּהִלָּתֶֽךָ עַל חַיֵּֽינוּ הַמְּסוּרִים בְּיָדֶֽךָ וְעַל נִשְׁמוֹתֵֽינוּ הַפְּקוּדוֹת לָךְ וְעַל נִסֶּֽיךָ שֶׁבְּכׇל־יוֹם עִמָּֽנוּ וְעַל נִפְלְאוֹתֶֽיךָ וְטוֹבוֹתֶֽיךָ שֶׁבְּכׇל־עֵת עֶֽרֶב וָבֹֽקֶר וְצׇהֳרָֽיִם׃ הַטּוֹב כִּי לֹא כָלוּ רַחֲמֶֽיךָ וְהַמְרַחֵם כִּי לֹא תַֽמּוּ חֲסָדֶֽיךָ מֵעוֹלָם קִוִֽינוּ לָךְ׃
}


\newcommand{\hashkiveinu}[1]{
	\firstword{הַשְׁכִּיבֵֽנוּ}
	יְיָ אֱלֹהֵֽינוּ לְשָׁלוֹם \middot וְהַעֲמִידֵֽנוּ מַלְכֵּֽנוּ לְחַיִּים \middot וּפְרוֹשׂ עָלֵֽינוּ סֻכַּת שְׁלוֹמֶֽךָ \middot וְתַקְּנֵֽנוּ בְּעֵצָה טוֹבָה מִלְּפָנֶֽיךָ וְהוֹשִׁיעֵֽנוּ לְמַֽעַן שְׁמֶֽךָ׃ וְהָגֵן בַּעֲדֵֽנוּ וְהָסֵר מֵעָלֵֽינוּ אוֹיֵב דֶּֽבֶר וְחֶֽרֶב וְרָעָב וְיָגוֹן \middot וְהָסֵר שָׂטָן מִלְּפָנֵֽינוּ וּמֵאַחֲרֵֽנוּ וּבְצֵל כְּנָפֶֽיךָ תַּסְתִּירֵֽנוּ׃ כִּי אֵל שׁוֹמְרֵֽנוּ וּמַצִּילֵֽנוּ אַֽתָּה \middot כִּי אֵל מֶֽלֶךְ חַנּוּן וְרַחוּם אַֽתָּה׃ וּשְׁמוֹר צֵאתֵֽנוּ וּבוֹאֵֽנוּ לְחַיִּים וּלְשָׁלוֹם מֵעַתָּה וְעַד עוֹלָם׃ #1
}

\newcommand{\avinumalkeinu}{
	
	אָבִֽינוּ מַלְכֵּֽנוּ חָטָאנוּ לְפָנֶיךָ׃\hfill \break
	אָבִֽינוּ מַלְכֵּֽנוּ אֵין לָנוּ מֶֽלֶךְ אֶלָּא אַֽתָּה׃ \hfill \break
	אָבִֽינוּ מַלְכֵּֽנוּ עֲשֵׂה עִמָֽנוּ לְמַעַן שְׁמֶךָ׃\hfill \break
	אָבִֽינוּ מַלְכֵּֽנוּ (\instruction{בת״צ:} בָּרֵךְ)(\instruction{בעשי״ת:} חַדֵּשׁ) עָלֵינוּ שָׁנָה טוֹבָה:\hfill \break
	אָבִֽינוּ מַלְכֵּֽנוּ בַּטֵל מֵעָלֵינוּ כׇּל־גְּזֵּרוֹת קָשׁוֹת׃\hfill \break
	אָבִֽינוּ מַלְכֵּֽנוּ בַּטֵל מַחְשְׁבוֹת שׂוֹנְאֵֽינוּ׃\hfill \break
	אָבִֽינוּ מַלְכֵּֽנוּ הָפֵר עֲצַת אוֹיְּבֵֽינוּ׃\hfill \break
	אָבִֽינוּ מַלְכֵּֽנוּ כַּלֵה כׇּל־צָר וּמַשְׂטִין מֵעָלֵֽינוּ׃\hfill \break
	אָבִֽינוּ מַלְכֵּֽנוּ סְתוֹם פִּיּוֹת מַשְׂטִינֵֽנוּ וּמְקַטְרְגֵֽינוּ׃\hfill \break
	אָבִֽינוּ מַלְכֵּֽנוּ כַּלֵּה דֶּבֶר וְחֶרֶב וְרָעָב וּשְׁבִי וּמַשְׁחִית וּמַגֵּפָה מִבְּנֵי בְּרִיתֶֽךָ׃\hfill \break
	אָבִֽינוּ מַלְכֵּֽנוּ מְנַע מַגֵּפָה מִנַּחֲלָתֶֽךָ׃\hfill \break
	אָבִֽינוּ מַלְכֵּֽנוּ סְלַח וּמְחַל לְכׇל־עֲוֹנוֹתֵֽינוּ׃\hfill \break
	אָבִֽינוּ מַלְכֵּֽנוּ מְחֵה וְהַעֲבֵר פְּשָׁעֵֽינוּ וְחַטֹּאתֵֽינוּ מִנֶּגֶד עֵינֶֽיךָ׃\hfill \break
	אָבִֽינוּ מַלְכֵּֽנוּ מְחוֹק בְּרַחֲמֶיךָ הָרַבִּים כׇּל־שִׁטְרֵי חוֹבוֹתֵֽינוּ׃\hfill \break
	אָבִֽינוּ מַלְכֵּֽנוּ הַחֲזִירֵֽנוּ בִּתְשׁוּבָה שְׁלֵמָה לְפָנֶיךָ׃\hfill \break
	אָבִֽינוּ מַלְכֵּֽנוּ שְׁלַח רְפוּאָה שְׁלֵמָה לְחוֹלֵי עַמֶּךָ׃\hfill \break
	אָבִֽינוּ מַלְכֵּֽנוּ קְרַע רֽוֹעַ גְּזַר דִּינֵֽנוּ׃\hfill \break
	אָבִֽינוּ מַלְכֵּֽנוּ זׇכְרֵֽנוּ בְּזִכְרוׂן טוֹב לְפָנֶיךָ׃\hfill \break
	\begin{longtable}{>{\centering\arraybackslash}m{.48\textwidth} | >{\centering\arraybackslash}m{.48\textwidth}}
		
		\instruction{בעשי״ת:} & \instruction{בת״צ:}\\
		אָבִֽינוּ מַלְכֵּֽנוּ כׇּתְבֵֽנוּ בְּסֵפֶר חַיִּים טוֹבִים׃ & אָבִֽינוּ מַלְכֵּֽנוּ זׇכְרֵֽנוּ בְּסֵפֶר חַיִּים טוֹבִים׃\\
		אָבִֽינוּ מַלְכֵּֽנוּ כׇּתְבֵֽנוּ בְּסֵפֶר זָכִיּוֹת׃ & אָבִֽינוּ מַלְכֵּֽנוּ זׇכְרֵֽנוּ בְּסֵפֶר זָכִיּוֹת׃\\
		אָבִֽינוּ מַלְכֵּֽנוּ כׇּתְבֵֽנוּ בְּסֵפֶר פַּרְנָסָה וְכַלְכָּלָה׃ & אָבִֽינוּ מַלְכֵּֽנוּ זׇכְרֵֽנוּ בְּסֵפֶר פַּרְנָסָה וְכַלְכָּלָה׃\\
		אָבִֽינוּ מַלְכֵּֽנוּ כׇּתְבֵֽנוּ בְּסֵפֶר גְּאֻלָה וִישׁוּעָה׃ & אָבִֽינוּ מַלְכֵּֽנוּ זׇכְרֵֽנוּ בְּסֵפֶר גְּאֻלָה וִישׁוּעָה׃\\
		אָבִֽינוּ מַלְכֵּֽנוּ כׇּתְבֵֽנוּ בְּסֵפֶר מְחִילָה וּסְלִיחָה׃ & אָבִֽינוּ מַלְכֵּֽנוּ זׇכְרֵֽנוּ בְּסֵפֶר מְחִילָה וּסְלִיחָה׃\\
		%אָבִֽינוּ מַלְכֵּֽנוּ חַדֵּשׁ עָלֵינוּ שָׁנָה טוֹבָה׃ & אָבִֽינוּ מַלְכֵּֽנוּ בָּרֵךְ עָלֵינוּ שָׁנָה טוֹבָה׃
	\end{longtable}
	
	אָבִֽינוּ מַלְכֵּֽנוּ הַצְמַח לָֽנוּ יְשׁוּעָה בְּקָרוֹב׃\hfill \break
	אָבִֽינוּ מַלְכֵּֽנוּ הָרֵם קֶֽרֶן יִשְׂרָאֵל עַמֶּךָ׃\hfill \break
	אָבִֽינוּ מַלְכֵּֽנוּ הָרֵם קֶרֶן מְשִׁיחֶֽךָ׃\hfill \break
	אָבִֽינוּ מַלְכֵּֽנוּ מַלֵּא יָדֵֽינוּ מִבִּרְכוֹתֶֽיךָ׃\hfill \break
	אָבִֽינוּ מַלְכֵּֽנוּ מַלֵּא אֲסָמֵֽינוּ שָׂבָע׃\hfill \break
	אָבִֽינוּ מַלְכֵּֽנוּ שְׁמַע קּוֹלֵֽנוּ חוּס וְרַחֵם עָלֵֽינוּ׃\hfill \break
	אָבִֽינוּ מַלְכֵּֽנוּ קַבֵּל בְּרַחֲמִים וּבְרָצוֹן אֶת־תְּפִלָּתֵֽינוּ׃\hfill \break
	אָבִֽינוּ מַלְכֵּֽנוּ פְּתַח שַׁעֲרֵי שָׁמַֽיִם לִתְפִלָּתֵֽנוּ׃\hfill \break
	אָבִֽינוּ מַלְכֵּֽנוּ זְכוֹר כִּי עָפָר אֲנָֽחְנוּ׃\hfill \break
	אָבִֽינוּ מַלְכֵּֽנוּ נָא אַל תְּשִׁיבֵֽנּוּ רֵיקָם מִלְּפָנֶיךָ׃\hfill \break
	אָבִֽינוּ מַלְכֵּֽנוּ תְּהֵא הַשָּׁעָה הַזֹּאת שְׁעַת רַחֲמִים וְעֵת רָצוֹן מִלְּפָנֶֽיךָ׃\hfill \break
	אָבִֽינוּ מַלְכֵּֽנוּ חֲמוֹל עָלֵֽינוּ וְעַל עוֹלָלֵֽינוּ וְטַפֵּֽנוּ׃\hfill \break
	אָבִֽינוּ מַלְכֵּֽנוּ עֲשֵׂה לְמַעַן הֲרוּגִים עַל־שֵׁם קׇדְשֶׁךָ׃\hfill \break
	אָבִֽינוּ מַלְכֵּֽנוּ עֲשֵׂה לְמַעַן טְבוּחִים עַל־יִחוּדֶֽךָ׃\hfill \break
	אָבִֽינוּ מַלְכֵּֽנוּ עֲשֵׂה לְמַעַן בָּאֵי בָאֵשׁ וּבַמַּיִם עַל־קִּדוּשׁ שְׁמֶךָ׃\hfill \break
	אָבִֽינוּ מַלְכֵּֽנוּ נְקוֹם לְעֵינֵֽינוּ נִקְמַת דַּם עֲבָדֶיךָ הַשָׁפוּךְ׃\hfill \break
	אָבִֽינוּ מַלְכֵּֽנוּ עֲשֵׂה לְמַעַנְךָ אִם־לֹא־לְמַעֲנֵֽנוּ׃\hfill \break
	אָבִֽינוּ מַלְכֵּֽנוּ עֲשֵׂה לְמַעַנְךָ וְהוֹשִׁיעֵֽנוּ׃\hfill \break
	אָבִֽינוּ מַלְכֵּֽנוּ עֲשֵׂה לְמַעַן רַחֲמֶיךָ הָרַבִּים׃\hfill \break
	אָבִֽינוּ מַלְכֵּֽנוּ עֲשֵׂה לְמַעַן שִׁמְךָ הַגָּדוֹל הַגִּבּוֹר וְהַנוֹרָא שֶׁנִקְרָא עָלֵינוּ׃\hfill \break
	אָבִֽינוּ מַלְכֵּֽנוּ חׇנֵּנוּ וַעֲנֵנוּ כִּי אֵין בָּנוּ מַעֲשִׂים עֲשֵׂה עִמָּנוּ צְדָקָה וָחֶסֶד וְהוֹשִׁיעֵנוּ׃
}

\newcommand{\pesicha}{
	\englishinst{The ark is opened.}
	%\instruction{פותחים הארון}\\
	\firstword{וַיְהִ֛י}\source{במדבר י}
	בִּנְסֹ֥עַ הָאָרֹ֖ן וַיֹּ֣אמֶר מֹשֶׁ֑ה קוּמָ֣ה ׀ יְיָ֗ וְיָפֻ֙צוּ֙ אֹֽיְבֶ֔יךָ וְיָנֻ֥סוּ מְשַׂנְאֶ֖יךָ מִפָּנֶֽיךָ׃
	כִּ֤י \source{ישעיה ב}מִצִּיּוֹן֙ תֵּצֵ֣א תוֹרָ֔ה וּדְבַר־יְיָ֖ מִירוּשָׁלָֽ‍ִם׃
	בָּרוּךְ שֶׁנָּתַן תּוֹרָה לְעַמּוֹ יִשְׂרָאֵל בִּקְדֻשָּׁתוֹ׃
}


\newcommand{\brikhshmei}{\begin{small}
		בְּרִיךְ שְׁמֵהּ דְּמָרֵא עָלְמָא בְּרִיךְ כִּתְרָךְ וְאַתְרָךְ \middot יְהֵא רְעוּתָךְ עִם עַמָּךְ יִשְׂרָאֵל לְעָלַם וּפוּרְקַן יְמִינָךְ אַחֲזֵי לְעַמָּךְ בְּבֵית מַקְדְּשָׁךְ \middot וּלְאַמְטוּיֵי לָנָא מִטּוּב נְהוֹרָךְ וּלְקַבֵּל צְלוֹתָנָא בְּרַחֲמִין׃ יְהֵא רַעֲוָא קֳדָמָךְ דְּתוֹרִיךְ לָן חַיִּין בְּטִיבוּתָא \middot וְלֶהֱוֵי אֲנָא פְקִידָא בְּגוֹ צַדִּיקַיָּא לְמִרְחַם עָלַי וּלְמִנְטַר יָתִי וְיַת כׇּל־דִּי לִי וְדִי לְעַמָּךְ יִשְׂרָאֵל׃ אַנְתְּ הוּא זָן לְכֹלָּא וּמְפַרְנֵס לְכֹלָּא אַנְתְּ הוּא שַׁלִּיט עַל כֹּלָּא אַנְתְּ הוּא דְשַׁלִּיט עַל מַלְכַיָּא וּמַלְכוּתָא דִּילָךְ הִיא׃ אֲנָא עַבְדָּא דְקֻדְשָׁא בְּרִיךְ הוּא דְּסָגִידְנָא קַמֵּהּ וּמִקַּמֵּי דִּיקַר אוֹרַיְתֵהּ בְּכׇל־עִדָּן וְעִדָּן \middot לָא עַל אֱנָשׁ רָחִיצְנָא וְלָא עַל בַּר אֱלָהִין סָמִיכְנָא \middot אֶלָּא בֶּאֱלָהָא דִשְׁמַיָּא דְּהוּא אֱלָהָא קְשׁוֹט וְאוֹרַיְתֵהּ קְשׁוֹט וּנְבִיאוֹהִי קְשׁוֹט וּמַסְגֵּא לְמֶעְבַּד טַבְוָן וּקְשׁוֹט׃ בֵּהּ אֲנָא רְחִיץ וְלִשְׁמֵהּ קַדִּישָׁא יַקִּירָא אֲנָא אֵמַר תֻּשְׁבְּחָן \middot יְהֵא רַעֲוָא קֳדָמָךְ דְּתִפְתַּח לִבָּאִי בְּאוֹרַיְתָא וְתַשְׁלִים מִשְׁאֲלִין דְּלִבָּאִי וְלִבָּא דְכׇל־עַמָּךְ יִשְׂרָאֵל \middot לְטַב וּלְחַיִּין וְלִשְלָם. אָמֵן׃
	\end{small}
}

\newcommand{\gadlu}{
	\shatz \begin{large}\textbf{ֽגַּדְּל֣וּ לַייָ֣ אִתִּ֑י וּנְרוֹמְמָ֖ה שְׁמ֣וֹ יַחְדָּֽו׃} \source{תהלים לד}\end{large}
	
	
	\instruction{כולם׃}
	\firstword{לְךָ֣ יְ֠יָ֠}\source{דה״א כט}
	הַגְּדֻלָּ֨ה וְהַגְּבוּרָ֤ה וְהַתִּפְאֶ֙רֶת֙ וְהַנֵּ֣צַח וְהַה֔וֹד כִּי־כֹ֖ל בַּשָּׁמַ֣יִם וּבָאָ֑רֶץ לְךָ֤ יְיָ֙ הַמַּמְלָכָ֔ה וְהַמִּתְנַשֵּׂ֖א לְכֹ֥ל ׀ לְרֹֽאשׁ׃
	רוֹמְמ֡וּ\source{תהלים צט} יְ֘יָ֤ אֱלֹהֵ֗ינוּ וְֽ֭הִשְׁתַּחֲווּ לַהֲדֹ֥ם רַגְלָ֗יו קָד֥וֹשׁ הֽוּא׃
	רוֹמְמ֡וּ יְ֘יָ֤ אֱלֹהֵ֗ינוּ וְֽ֭הִשְׁתַּחֲווּ לְהַ֣ר קׇדְשׁ֑וֹ כִּי־קָ֝ד֗וֹשׁ יְיָ֥ אֱלֹהֵֽינוּ׃
}
\newcommand{\avharachamim}{
	\firstword{אַב הָרַחֲמִים}
	הוּא יְרַחֵם עַם עֲמוּסִים וְיִזְכּוֹר בְּרִית אֵיתָנִים וְיַצִּיל נַפְשׁוֹתֵֽינוּ מִן הַשָּׁעוֹת הָרָעוֹת וְיִגְעַר בְּיֵֽצֶר הָרַע מִן הַנְּשׂוּאִים וְיָחוֹן עָלֵֽינוּ לִפְלֵיטַת עוֹלָמִים וִימַלֵּא מִשְׁאֲלוֹתֵֽינוּ בְּמִדָּה טוֹבָה יְשׁוּעָה וְרַחֲמִים׃
}


\newcommand{\vesigale}{
	\englishinst{As the Torah is placed on the reading desk:}
	\instruction{גבאי:}
	\firstword{וְתִגָּלֶה}
	וְתֵרָאֶה מַלְכוּתוֹ עָלֵֽינוּ בִּזְמַן קָרוֹב וְיָחֹן פְּלֵטָתֵֽנוּ וּפְלֵטַת עַמּוֹ בֵּית יִשְׂרָאֵל לְחֵן וּלְחֶֽסֶד וּלְרַחֲמִים וּלְרָצוֹן׃ וְנֹאמַר אָמֵן׃
	הַכֹּל הָבוּ גוֹדֶל לֵאלֹהֵֽינוּ וּתְנוּ כָבוֹד לַתּוֹרָה׃ כֹּהֵן קְרָב יַעֲמוֹד \ תַּעֲמוֹד \instruction{(פב״פ)} הַכֹּהֵן.
	(\instruction{עם אין כהן׃ }
		אֵין כַּאן כֹּהֵן, יַעֲמוֹד \ תַּעֲמוֹד 
		\instruction{(פב״פ)}
		בִּמְקוׂם כֹּהֵן)
 בָּרוּךְ שֶׁנָּתַן תּוֹרָה לְעַמּוֹ יִשְׂרָאֵל בִּקְדֻשָּׁתוֹ׃
	\instruction{קהל ואח״כ הגבאי׃}
	\textbf{וְאַתֶּם֙ הַדְּבֵקִ֔ים בַּייָ֖ אֱלֹהֵיכֶ֑ם חַיִּ֥ים כֻּלְּכֶ֖ם הַיּֽוֹם׃} \source{דברים ד}
}



\newcommand{\torahbarachu}{
	\instruction{עולה:}
	\begin{large}
		\firstword{בָּרְכוּ אֶת־יְיָ הַמְבֹרָךְ׃}\\
		\instruction{עולה וקהל:}\firstword{בָּרוּךְ יְיָ הַמְבֹרָךְ לְעוֹלָם וָעֶד:}\\
	\end{large}
	\instruction{עולה:}
	בָּרוּךְ אַתָּה יְיָ אֱלֹהֵֽינוּ מֶֽלֶךְ הָעוֹלָם אֲשֶׁר בָּֽחַר בָּֽנוּ מִכׇּל־הָעַמִּים
	וְנָֽתַן לָֽנוּ אֶת־תּוֹרָתוֹ׃ בָּרוּךְ אַתָּה יְיָ נוֹתֵן הַתּוֹרָה׃
	
	\instruction{אחר הקריאה:}\\
	בָּרוּךְ אַתָּה יְיָ אֱלֹהֵֽינוּ מֶֽלֶךְ הָעוֹלָם אֲשֶׁר נָֽתַן לָֽנוּ תּוֹרַת אֱמֶת
	וְחַיֵּי עוֹלָם נָטַע בְּתוֹכֵֽנוּ׃ בָּרוּךְ אַתָּה יְיָ נוֹתֵן הַתּוֹרָה׃
}

\newcommand{\hagomel}{
	\begin{sometimes}
		
		\instruction{ברכת הגומל:}\\
		בָּרוּךְ אַתָּה יְיָ אֱלֹהֵֽינוּ מֶֽלֶךְ הָעוֹלָם הַגּוֹמֵל לְחַיָּבִים טוֹבוֹת שֶׁגְּמָלַֽנִי כׇּל־טוֹב׃\\
		\kahal
		מִי שֶׁגְּמׇלְךָ/שֶׁגְּמָלֵךְ כׇּל־טוֹב הוּא יִגְמׇלְךָ/שֶׁגְּמָלֵךְ כׇּל־טוֹב סֶֽלָה׃
		
\end{sometimes}}

\newcommand{\misheberakholim}[1]{\instruction{מי שברך לעולֶה:}\\
	\firstword{מִי שֶׁבֵּרַךְ}
	אֲבוֹתֵֽינוּ אַבְרָהָם יִצְחָק וְיַעֲקֹב שָׂרָה רִבְקָה רָחֵל וְלֵאָה, הוּא יְבָרֵךְ אֶת
	\instruction{(פב״פ)}
	בַּעֲבוּר שֶׁעָלָה לִכְבוֹד הַמָּקוֹם וְלִכְבוֹד הַתּוֹרָה #1 בִּשְׂכַר זֶה הַקָּדוֹשׁ בָּרוּךְ הוּא יִשְׁמְרֵֽהוּ וְיַצִּילֵֽהוּ מִכׇּל־צָרָה וְצוּקָה וְיִשְׁלַח בְּרָכָה וְהַצְלָחָה בְּכׇל־מַעֲשֵׂה יָדָיו עִם כׇּל־יִשְׂרָאֵל אֶחָיו׃ וְנֹאמַר אָמֵן׃
	
	\instruction{מי שברך לעולָה:}\\
	\firstword{מִי שֶׁבֵּרַךְ}
	אֲבוֹתֵֽינוּ אַבְרָהָם יִצְחָק וְיַעֲקֹב שָׂרָה רִבְקָה רָחֵל וְלֵאָה, הוּא יְבָרֵךְ אֶת
	\instruction{(פב״פ)}
	בַּעֲבוּר שֶׁעָלְתָה לִכְבוֹד הַמָּקוֹם וְלִכְבוֹד הַתּוֹרָה #1 בִּשְׂכַר זֶה הַקָּדוֹשׁ בָּרוּךְ הוּא יִשְׁמְרֶֽהָ וְיַצִּילֶֽהָ מִכׇּל־צָרָה וְצוּקָה, וּמִכׇּל־נֶֽגַע וּמַחֲלָה, וְיִשְׁלַח בְּרָכָה וְהַצְלָחָה בְּכׇל־מַעֲשֵׂה יָדֶֽיהָ עִם כׇּל־יִשְׂרָאֵל אֲחֶֽיהָ׃ וְנֹאמַר אָמֵן׃
	
	\instruction{מי שברך לעולים:}\\
	\firstword{מִי שֶׁבֵּרַךְ}
	אֲבוֹתֵֽינוּ אַבְרָהָם יִצְחָק וְיַעֲקֹב שָׂרָה רִבְקָה רָחֵל וְלֵאָה, הוּא יְבָרֵךְ אֶת
	בַּעֲבוּר שֶׁעָלָה לִכְבוֹד הַמָּקוֹם וְלִכְבוֹד הַתּוֹרָה #1
	בִּשְׂכַר זֶה הַקָּדוֹשׁ בָּרוּךְ הוּא יִשְׁמְרֵֽהוּ וְיַצִּילֵֽהוּ מִכׇּל־צָרָה וְצוּקָה וְיִשְׁלַח בְּרָכָה וְהַצְלָחָה בְּכׇל־מַעֲשֵׂה יָדָיו עִם כׇּל־יִשְׂרָאֵל אֶחָיו׃ וְנֹאמַר אָמֵן׃
}

\vspace{-\baselineskip}
\newcommand{\misheberakhcholimmulti}[1]{
	
	\firstword{מִי שֶׁבֵּרַךְ}
	אֲבוֹתֵינוּ אַבְרָהָם יִצְחָק וְיַעֲקֹב, שָׂרָה רִבְקָה רָחֵל וְלֵאָה, הוּא יְבָרֵךְ וִירַפֵּא
	
	\setcolumnwidth{1.4in,1.4in,1.4in}
	\begin{paracol}{3}
		\instruction{לחולֶה}\\
		אֶת־הַחוֹלֶה \instruction{פלוני בן פלונית} עֲבוּר שֶׁאָנוּ מִתְפַּלְלִים בַּעֲבוּרוֹ. בִּשְׂכַר זֶה, הַקָּדוֹשׁ בָּרוּךְ הוּא יִמָּלֵא רַחֲמִים עָלָיו לְהַחֲלִימוֹ וּלְרַפֹּאתוֹ, לְהַחֲזִיקוֹ וּלְהַחֲיוֹתוֹ, וְיִשְׁלַח לוֹ מְהֵרָה רְפוּאָה שְׁלֵמָה, רְפוּאַת הַנֶּֽפֶשׁ וּרְפוּאַת הַגּוּף
		\switchcolumn
		\instruction{לחולָה}\\
		אֶת־הַחוֹלָה \instruction{פלונית בת פלונית} בַּעֲבוּר שֶׁאָנוּ מִתְפַּלְלִים בַּעֲבוּרָהּ. בִּשְׂכַר זֶה, הַקָּדוֹשׁ בָּרוּךְ הוּא יִמָּלֵא רַחֲמִים עָלֶיהָ לְהַחֲלִימָהּ וּלְרַפֹּאתָהּ, לְהַחֲזִיקָהּ וּלְהַחֲיוֹתָהּ, וְיִשְׁלַח לָהּ מְהֵרָה רְפוּאָה שְׁלֵמָה, רְפוּאַת הַנֶּֽפֶשׁ וּרְפוּאַת הַגּוּף
		\switchcolumn
		\instruction{לחולים}\\
		אֶת־הַחוׂלִים...בַּעֲבוּר שֶׁאָנוּ מִתְפַּלְלִים בַּעֲבוּרָם. בִּשְׂכַר זֶה, הַקָּדוֹשׁ בָּרוּךְ הוּא יִמָּלֵא רַחֲמִים עָלֵיהֶם, לְהַחֲלִימָם וּלְרַפְּאֹתָם וּלְהַחֲזִיקָם וּלְהַחֲיוֹתָם, וְיִשְׁלַח לָהֶם מְהֵרָה רְפוּאָה שְׁלֵמָה מִן הַשָּׁמַיִם לְכׇל־אֵבָרֵֽיהֶם וְגִידֵֽיהֶם,
		
	\end{paracol}
	בְּתוֹךְ שְׁאָר חוֹלֵי יִשְׂרָאֵל, רְפוּאַת הַנֶּֽפֶשׁ וּרְפוּאַת הַגּוּף #1 וּרְפוּאָה קְרוֹבָה לָבוֹא, הַשְׁתָּא בַּעֲגָלָא וּבִזְמַן קָרִיב. וְנֹאמַר: אָמֵן׃
}

\newcommand{\misheberakhcholim}[1]{
\firstword{מִי שֶׁבֵּרַךְ}
אֲבוֹתֵינוּ אַבְרָהָם יִצְחָק וְיַעֲקֹב שָׂרָה רִבְקָה רָחֵל וְלֵאָה \middot הוּא יְבָרֵךְ וִירַפֵּאאֶת־הַחוׂלִים...בַּעֲבוּר שֶׁאָנוּ מִתְפַּלְלִים בַּעֲבוּרָם׃ בִּשְׂכַר זֶה, הַקָּדוֹשׁ בָּרוּךְ הוּא יִמָּלֵא רַחֲמִים עָלֵיהֶם \middot לְהַחֲלִימָם וּלְרַפְּאֹתָם וּלְהַחֲזִיקָם וּלְהַחֲיוֹתָם \middot וְיִשְׁלַח לָהֶם מְהֵרָה רְפוּאָה שְׁלֵמָה מִן הַשָּׁמַיִם לְכׇל־אֵבָרֵֽיהֶם וְגִידֵֽיהֶם בְּתוֹךְ שְׁאָר חוֹלֵי יִשְׂרָאֵל רְפוּאַת הַנֶּֽפֶשׁ וּרְפוּאַת הַגּוּף \middot #1 וּרְפוּאָה קְרוֹבָה לָבוֹא \middot הַשְׁתָּא בַּעֲגָלָא וּבִזְמַן קָרִיב. וְנֹאמַר: אָמֵן׃
}

\newcommand{\misheberakhbaby}{\instruction{מי שבירך ליולדת בן:}\\
	\firstword{מִי שֶׁבֵּרַךְ}
	אֲבוֹתֵֽינוּ אַבְרָהָם יִצְחָק וְיַעֲקֹב, שָׂרָה רִבְקָה רָחֵל וְלֵאָה, הוּא יְבָרֵךְ אֶת־הָאִשָׁה הַיוֹלֶֽדֶת
	\instruction{(פלונית בת פלונית)}
	עִם בְּנָהּ הַנוֹלָד בְּמַזָל טוֹב בַּעֲבוּר שֶׁבַּעֲלָהּ נָדַר...בַּעֲדָם. בִּשְׂכַר זֶה הַקָדוֹשׁ בָּרוּךְ הוּא יְהִי בְּעֶזְרָם וְיִשְׁמְרֵם וִיזַכֶּה אֶת־הָאֵם לְגַדֵל אֶת־בְּנָהּ בַּטוֹב וּבַנְעִימִים וּלְהַדְרִיכוֹ בְּאֹֽרַח מִישׁוֹר לַתּוֹרָה לְחֻפָּה וּלְמַעֲשִׂים טוֹבִים. וְנֹאמַר אָמֵן׃
	
	\instruction{מי שבירך ליולדת בת:}\\
	\firstword{מִי שֶׁבֵּרַךְ}
	אֲבוֹתֵֽינוּ אַבְרָהָם יִצְחָק וְיַעֲקֹב הוּא יְבָרֵךְ אֶת־הָאִשָׁה הַיוֹלֶֽדֶת
	\instruction{(פלונית בת פלונית)}
	עִם בִּתָּהּ הַנוֹלֶֽדֶת בְּמַזָל טוֹב (וְיִקָרֵא שְׁמָהּ בְּיִשְׂרָאֵל...) בַּעֲבוּר שֶׁבַּעֲלָהּ נָדַר...בַּעֲדָן בִּשְׂכַר זֶה הַקָדוֹשׁ בָּרוּךְ הוּא יְהִי בְּעֶזְרָן וְיִשְׁמְרֵן וִיזַכֶּה אֶת־הָאֵם לְגַדֵל אֶת־בִּתָּהּ בַּטוֹב ובַנְעִימִים וּלְהַדְרִיכָה בְּאֹֽרַח מִישׁוֹר לְמִצְוֹת לְחֻפָּה וּלְמַעֲשִׂים טוֹבִים. וְנֹאמַר אָמֵן׃
}

\newcommand{\misheberakhbarmitzva}{\instruction{מי שבירך לבר מצוה:}\\
	\firstword{מִי שֶׁבֵּרַךְ}
	אֲבוֹתֵֽינוּ אַבְרָהָם יִצְחָק וְיַעֲקֹב הוּא יְבָרֵךְ אֶת
	\instruction{(פב״פ)}
	שֶׁהִגִֽיעוּ יָמָיו לִהְיוֹת בַּר מִצְוָה וְעָלָה הַיוֹם לַתּוֹרָה בַּפַּֽעַם הָרִאשׁוֹנָה לָתֵת שֶֽׁבַח וְהוֹדָאָה לְהַשֵׁם יִתְבָּרַךְ עַל כׇּל־הַטוֹבָה אֲשֶׁר עָשָׂה לוֹ (וְנָדַר...) בִּשְׂכַר זֶה הַקָדוֹשׁ בָּרוּךְ הוּא יִשְׁמְרֵֽהוּ וִיחַיֵֽהוּ וִיכוֹנֵן אֶת־לִבּוֹ לִהְיוֹת שָׁלֵם עִם יְיָ וְלָלֶֽכֶת בִּדְרָכָיו וְלִשְׁמֹר מִצְוֹתָיו כׇּל־הַיָמִים וְנֹאמַר אָמֵן׃}
\newcommand{\hagbaha}{
	\instruction{הגבה:}
	\firstword{וְזֹ֖את הַתּוֹרָ֑ה}\source{דברים ד}
	אֲשֶׁר־שָׂ֣ם מֹשֶׁ֔ה לִפְנֵ֖י בְּנֵ֥י יִשְׂרָאֵֽל׃
	עַל־פִּ֥י \source{במדבר ט}יְיָ֖ בְּיַד־מֹשֶֽׁה׃
	עֵץ־חַיִּ֣ים \source{משלי ג} הִ֭יא לַמַּחֲזִיקִ֣ים בָּ֑הּ וְֽתֹמְכֶ֥יהָ מְאֻשָּֽׁר׃
	דְּרָכֶ֥יהָ דַרְכֵי־נֹ֑עַם וְֽכׇל־נְתִ֖יבוֹתֶ֣יהָ שָׁלֽוֹם׃
	אֹ֣רֶךְ יָ֭מִים בִּֽימִינָ֑הּ בִּ֝שְׂמֹאולָ֗הּ עֹ֣שֶׁר וְכָבֽוֹד׃
	יְיָ֥ \source{ישעיה מב}חָפֵ֖ץ לְמַ֣עַן צִדְק֑וֹ יַגְדִּ֥יל תּוֹרָ֖ה וְיַאְדִּֽיר׃
	
}


\newcommand{\yehalelu}{
	\shatz
	\begin{large}
		\textbf{יְהַלְל֤וּ ׀ אֶת־שֵׁ֬ם יְיָ֗ כִּֽי־נִשְׂגָּ֣ב שְׁמ֣וֹ לְבַדּ֑וֹ}\source{תהלים קמח}
	\end{large}
	
	\kahal
	ה֝וֹד֗וֹ עַל־אֶ֥רֶץ וְשָׁמָֽיִם׃ וַיָּ֤רֶם קֶ֨רֶן ׀ לְעַמּ֡וֹ תְּהִלָּ֤ה לְֽכׇל־חֲסִידָ֗יו לִבְנֵ֣י יִ֭שְׂרָאֵל עַ֥ם קְרֹב֗וֹ הַֽלְלוּ־יָֽהּ׃
}


\newcommand{\kafdalet}{
	
	\firstword{לְדָוִ֗ד מִ֫זְמ֥וֹר}\source{תהלים כד}
	לַֽייָ֭ הָאָ֣רֶץ וּמְלוֹאָ֑הּ תֵּ֝בֵ֗ל וְיֹ֣שְׁבֵי בָֽהּ׃
	כִּי־ה֭וּא עַל־יַמִּ֣ים יְסָדָ֑הּ וְעַל־נְ֝הָר֗וֹת יְכוֹנְנֶֽהָ׃
	מִֽי־יַעֲלֶ֥ה בְהַר־יְיָ֑ וּמִי־יָ֝קוּם בִּמְק֥וֹם קׇדְשֽׁוֹ׃
	נְקִ֥י כַפַּ֗יִם וּֽבַר־לֵ֫בָ֥ב אֲשֶׁ֤ר ׀ לֹא־נָשָׂ֣א לַשָּׁ֣וְא נַפְשִׁ֑י וְלֹ֖א נִשְׁבַּ֣ע לְמִרְמָֽה׃
	יִשָּׂ֣א בְ֭רָכָה מֵאֵ֣ת יְיָ֑ וּ֝צְדָקָ֗ה מֵאֱלֹהֵ֥י יִשְׁעֽוֹ׃
	זֶ֭ה דּ֣וֹר דֹּרְשָׁ֑ו מְבַקְשֵׁ֨י פָנֶ֖יךָ יַעֲקֹ֣ב סֶֽלָה׃
	שְׂא֤וּ שְׁעָרִ֨ים ׀ רָֽאשֵׁיכֶ֗ם וְֽ֭הִנָּשְׂאוּ פִּתְחֵ֣י עוֹלָ֑ם וְ֝יָב֗וֹא מֶ֣לֶךְ הַכָּבֽוֹד׃
	מִ֥י זֶה֮ מֶ֤לֶךְ הַכָּ֫ב֥וֹד יְיָ֭ עִזּ֣וּז וְגִבּ֑וֹר יְ֝יָ֗ גִּבּ֥וֹר מִלְחָמָֽה׃
	שְׂא֤וּ שְׁעָרִ֨ים ׀ רָֽאשֵׁיכֶ֗ם וּ֭שְׂאוּ פִּתְחֵ֣י עוֹלָ֑ם וְ֝יָבֹ֗א מֶ֣לֶךְ הַכָּבֽוֹד׃
	מִ֤י ה֣וּא זֶה֮ מֶ֤לֶךְ הַכָּ֫ב֥וֹד יְיָ֥ צְבָא֑וֹת ה֤וּא מֶ֖לֶךְ הַכָּב֣וֹד סֶֽלָה׃
}

\newcommand{\etzchaim}{
	\firstword{וּבְנֻחֹ֖ה יֹאמַ֑ר}\source{במדבר י}
	שׁוּבָ֣ה יְיָ֔ רִֽבְב֖וֹת אַלְפֵ֥י יִשְׂרָאֵֽל׃
	קוּמָ֣ה \source{תהלים קלב}יְ֖יָ לִמְנוּחָתֶ֑ךָ אַ֝תָּ֗ה וַאֲר֥וֹן עֻזֶּֽךָ׃
	כֹּהֲנֶ֥יךָ יִלְבְּשׁוּ־צֶ֑דֶק וַחֲסִידֶ֥יךָ יְרַנֵּֽנוּ׃
	בַּֽ֭עֲבוּר דָּוִ֣ד עַבְדֶּ֑ךָ אַל־תָּ֝שֵׁ֗ב פְּנֵ֣י מְשִׁיחֶֽךָ׃
	כִּ֤י \source{משלי ד}לֶ֣קַח ט֭וֹב נָתַ֣תִּי לָכֶ֑ם תּ֝וֹרָתִ֗י אַֽל־תַּעֲזֹֽבוּ׃
	עֵץ־חַיִּ֣ים \source{משלי ג}הִ֭יא לַמַּחֲזִיקִ֣ים בָּ֑הּ וְֽתֹמְכֶ֥יהָ מְאֻשָּֽׁר׃
	דְּרָכֶ֥יהָ דַרְכֵי־נֹ֑עַם וְֽכׇל־נְתִ֖יבוֹתֶ֣יהָ שָׁלֽוֹם׃
	הֲשִׁיבֵ֨נוּ \source{איכה ה}יְיָ֤ ׀ אֵלֶ֙יךָ֙ וְֽנָשׁ֔וּבָ חַדֵּ֥שׁ יָמֵ֖ינוּ כְּקֶֽדֶם׃
}

\newcommand{\kedushadesidra}{
	וְאַתָּ֥ה
	\source{תהלים כב}%
	קָד֑וֹשׁ י֝וֹשֵׁ֗ב תְּהִלּ֥וֹת יִשְׂרָאֵֽל׃
	\source{ישעיה ו}%
	וְקָרָ֨א זֶ֤ה אֶל־זֶה֙ וְאָמַ֔ר\\
	\kahal
	\kadoshbase\\
	וּמְקַבְּלִין דֵּין מִן דֵּין וְאָמְרִין׃ קַדִּישׁ בִּשְׁמֵי מְרוֹמָא עִלָּאָה בֵּית שְׁכִינְתֵּהּ קַדִּישׁ עַל אַרְעָא עוֹבַד גְּבוּרְתֵּהּ קַדִּישׁ לְעָלַם וּלְעָלְמֵי עָלְמַיָּא יְיָ צְבָאוֹת מַלְיָא כׇל־אַרְעָא זִיו יְקָרֵהּ׃
	
	וַתִּשָּׂאֵ֣נִי ר֔וּחַ וָאֶשְׁמַ֣ע אַחֲרַ֔י ק֖וֹל רַ֣עַשׁ גָּד֑וֹל\\
	\kahal
	\barukhhashemsource\\
	וּנְטָלַֽתְנִי רוּחָא וּשְׁמָעֵת בָּתְרַי קָל זִֽיעַ סַגִּיא דִּמְשַׁבְּחִין וְאָמְרִין׃ בְּרִיךְ יְקָרָא דַיְיָ מֵאֲתַר בֵּית שְׁכִינְתֵּהּ׃\\
	\kahal
	\hashemyimloch\\
	יְיָ מַלְכוּתֵהּ [קָאֵם] לְעָלַם וּלְעַלְמֵי עָלְמַיָּא׃ יְיָ֗
	\source{דה״א כט}%
	אֱ֠לֹהֵ֠י אַבְרָהָ֞ם יִצְחָ֤ק וְיִשְׂרָאֵל֙ אֲבֹתֵ֔ינוּ שׇׁמְרָה־זֹּ֣את לְעוֹלָ֔ם לְיֵ֥צֶר מַחְשְׁב֖וֹת לְבַ֣ב עַמֶּ֑ךָ וְהָכֵ֥ן לְבָבָ֖ם אֵלֶֽיךָ׃ וְה֤וּא
	\source{תהלים עח}%
	רַח֨וּם ׀ יְכַפֵּ֥ר עָוֺן֮ וְֽלֹא־יַֽ֫שְׁחִ֥ית וְ֭הִרְבָּה לְהָשִׁ֣יב אַפּ֑וֹ וְלֹא־יָ֝עִ֗יר כׇּל־חֲמָתֽוֹ׃ כִּֽי־אַתָּ֣ה
	\source{תהלים פו}%
	אֲ֭דֹנָי ט֣וֹב וְסַלָּ֑ח וְרַב־חֶ֗֝סֶד לְכׇל־קֹֽרְאֶֽיךָ׃ צִדְקָתְךָ֣
	\source{תהלים קיט}%
	צֶ֣דֶק לְעוֹלָ֑ם וְֽתוֹרָתְךָ֥ אֱמֶֽת׃ תִּתֵּ֤ן
	\source{מיכה ז}%
	אֱמֶת֙ לְיַֽעֲקֹ֔ב חֶ֖סֶד לְאַבְרָהָ֑ם אֲשֶׁר־נִשְׁבַּ֥עְתָּ לַאֲבֹתֵ֖ינוּ מִ֥ימֵי קֶֽדֶם׃ בָּ֤ר֣וּךְ
	\source{תהלים סח}%
	אֲדֹנָי֮ י֤וֹם ׀ י֥֫וֹם יַעֲמׇס־לָ֗נוּ הָ֘אֵ֤ל יְֽשׁוּעָתֵ֬נוּ סֶֽלָה׃ יְיָ֣
	\source{תהלים מו}%
	צְבָא֣וֹת עִמָּ֑נוּ מִשְׂגָּֽב־לָ֨נוּ אֱלֹהֵ֖י יַֽעֲקֹ֣ב סֶֽלָה׃ יְיָ֥
	\source{תהלים פד}%
	צְבָא֑וֹת אַֽשְׁרֵ֥י אָ֝דָ֗ם בֹּטֵ֥חַ בָּֽךְ׃ יְיָ֥
	\source{תהלים כ}%
	הוֹשִׁ֑יעָה הַ֝מֶּ֗לֶךְ יַעֲנֵ֥נוּ בְיוֹם־קׇרְאֵֽנוּ׃ \\
	בָּרוּךְ הוּא אֱלֹהֵֽינוּ שֶׁבְּרָאָֽנוּ לִכְבוֹדוֹ וְהִבְדִּילָֽנוּ מִן הַתּוֹעִים וְנָֽתַן לָֽנוּ תּוֹרַת אֱמֶת וְחַיֵּי עוֹלָם נָטַע בְּתוֹכֵֽנוּ הוּא יִפְתַּח לִבֵּֽנוּ בְּתוֹרָתוֹ וְיָשֵׂם בְּלִבֵּֽנוּ אַהֲבָתוֹ וְיִרְאָתוֹ וְלַעֲשׂוֹת רְצוֹנוֹ וּלְעׇבְדוֹ בְלֵבָב שָׁלֵם לְמַֽעַן לֹא נִיגַע לָרִיק וְלֹא נֵלֵד לַבֶּהָלָה׃\\
	יְהִי רָצוֹן מִלְּפָנֶֽיךָ יְיָ אֱלֹהֵֽינוּ וֵאלֹהֵי אֲבוֹתֵֽינוּ שֶׁנִּשְׁמוֹר חֻקֶּֽיךָ בָּעוֹלָם הַזֶּה וְנִזְכֶּה וְנִחְיֶה וְנִרְאֶה וְנִירַשׁ טוֹבָה וּבְרָכָה לִשְׁנֵי יְמוֹת הַמָּשִֽׁיחַ וּלְחַיֵּי הָעוֹלָם הַבָּא׃ לְמַ֤עַן\source{תהלים ל} יְזַמֶּרְךָ֣ כָ֭בוֹד וְלֹ֣א יִדֹּ֑ם יְיָ֥ אֱ֝לֹהַ֗י לְעוֹלָ֣ם אוֹדֶֽךָּ׃ בָּר֣וּךְ\source{ירמיה יז} הַגֶּ֔בֶר אֲשֶׁ֥ר יִבְטַ֖ח בַּייָ֑ וְהָיָ֥ה יְיָ֖ מִבְטַחֽוֹ׃ בִּטְח֥וּ
	\source{ישעיה כו}%
	בַֽייָ֖ עֲדֵי־עַ֑ד כִּ֚י בְּיָ֣הּ יְיָ֔ צ֖וּר עוֹלָמִֽים׃ וְיִבְטְח֣וּ
	\source{תהלים ט}%
	בְ֭ךָ יוֹדְעֵ֣י שְׁמֶ֑ךָ כִּ֤י לֹֽא־עָזַ֖בְתָּ דֹרְשֶׁ֣יךָ יְיָ׃ יְיָ֥
	\source{ישעיה מב}%
	חָפֵ֖ץ לְמַ֣עַן צִדְק֑וֹ יַגְדִּ֥יל תּוֹרָ֖ה וְיַאְדִּֽיר׃
}


\newcommand{\uvaletzion}{
	\firstword{וּבָ֤א לְצִיּוֹן֙}\source{ישעיה נט}
	גּוֹאֵ֔ל וּלְשָׁבֵ֥י פֶ֖שַׁע בְּיַֽעֲקֹ֑ב נְאֻ֖ם יְיָ׃ וַאֲנִ֗י זֹ֣את בְּרִיתִ֤י אוֹתָם֙ אָמַ֣ר יְיָ֔ רוּחִי֙ אֲשֶׁ֣ר עָלֶ֔יךָ וּדְבָרַ֖י אֲשֶׁר־שַׂ֣מְתִּי בְּפִ֑יךָ לֹֽא־יָמ֡וּשׁוּ מִפִּ֩יךָ֩ וּמִפִּ֨י זַרְעֲךָ֜ וּמִפִּ֨י זֶ֤רַע זַרְעֲךָ֙ אָמַ֣ר יְיָ֔ מֵעַתָּ֖ה וְעַד־עוֹלָֽם׃
	\kedushadesidra
}

\newcommand{\shirshelyomintro}[1]{\begin{small}הַיּוֹם יוֹם #1 שֶׁבּוֹ הָיוּ הַלְוִיִּם אוֹמְרִים בְּבֵית־הַמִּקְדָּשׁ׃ \end{small}\vspace{-0.7\baselineskip}}

\newcommand{\weekdayshir}{
	\shirshelyomintro{רִאשׁוֹן בַּשַּׁבָּת}
	\begin{narrow}\kafdalet\end{narrow}
	
	\shirshelyomintro{שֵׁנִי בַּשַּׁבָּת}\begin{narrow}
		\source{תהלים מח}%
		שִׁ֥יר מִ֝זְמ֗וֹר לִבְנֵי־קֹֽרַח׃
		גָּ֘ד֤וֹל יְיָ֣ וּמְהֻלָּ֣ל מְאֹ֑ד בְּעִ֥יר אֱ֝לֹהֵ֗ינוּ הַר־קׇדְשֽׁוֹ׃
		יְפֵ֥ה נוֹף֮ מְשׂ֢וֹשׂ כׇּל־הָ֫אָ֥רֶץ הַר־צִ֭יּוֹן יַרְכְּתֵ֣י צָפ֑וֹן קִ֝רְיַ֗ת מֶ֣לֶךְ רָֽב׃
		אֱלֹהִ֥ים בְּאַרְמְנוֹתֶ֗יהָ נוֹדַ֥ע לְמִשְׂגָּֽב׃
		כִּֽי־הִנֵּ֣ה הַ֭מְּלָכִים נ֥וֹעֲד֑וּ עָבְר֥וּ יַחְדָּֽו׃
		הֵ֣מָּה רָ֭אוּ כֵּ֣ן תָּמָ֑הוּ נִבְהֲל֥וּ נֶחְפָּֽזוּ׃
		רְ֭עָדָה אֲחָזָ֣תַם שָׁ֑ם חִ֗֝יל כַּיּוֹלֵדָֽה׃
		בְּר֥וּחַ קָדִ֑ים תְּ֝שַׁבֵּ֗ר אֳנִיּ֥וֹת תַּרְשִֽׁישׁ׃
		כַּאֲשֶׁ֤ר שָׁמַ֨עְנוּ ׀ כֵּ֤ן רָאִ֗ינוּ בְּעִיר־יְיָ֣ צְ֭בָאוֹת בְּעִ֣יר אֱלֹהֵ֑ינוּ אֱלֹ֘הִ֤ים יְכוֹנְנֶ֖הָ עַד־עוֹלָ֣ם סֶֽלָה׃
		דִּמִּ֣ינוּ אֱלֹהִ֣ים חַסְדֶּ֑ךָ בְּ֝קֶ֗רֶב הֵיכָלֶֽךָ׃
		כְּשִׁמְךָ֤ אֱלֹהִ֗ים כֵּ֣ן תְּ֭הִלָּתְךָ עַל־קַצְוֵי־אֶ֑רֶץ צֶ֗֝דֶק מָלְאָ֥ה יְמִינֶֽךָ׃
		יִשְׂמַ֤ח ׀ הַר־צִיּ֗וֹן תָּ֭גֵלְנָה בְּנ֣וֹת יְהוּדָ֑ה לְ֝מַ֗עַן מִשְׁפָּטֶֽיךָ׃
		סֹ֣בּוּ צִ֭יּוֹן וְהַקִּיפ֑וּהָ סִ֝פְר֗וּ מִגְדָּלֶֽיהָ׃
		שִׁ֤יתוּ לִבְּכֶ֨ם ׀ לְֽחֵילָ֗ה פַּסְּג֥וּ אַרְמְנוֹתֶ֑יהָ לְמַ֥עַן תְּ֝סַפְּר֗וּ לְד֣וֹר אַֽחֲרֽוֹן׃
		כִּ֤י זֶ֨ה ׀ אֱלֹהִ֣ים אֱ֭לֹהֵינוּ עוֹלָ֣ם וָעֶ֑ד ה֖וּא יְנַהֲגֵ֣נוּ עַל־מֽוּת׃
	\end{narrow}
	
	\enlargethispage{\baselineskip}
	\shirshelyomintro{שְׁלִישִׁי בַּשַּׁבָּת}
	\begin{narrow}
		\source{תהלים פב}%
		מִזְמ֗וֹר לְאָ֫סָ֥ף אֱֽלֹהִ֗ים נִצָּ֥ב בַּעֲדַת־אֵ֑ל בְּקֶ֖רֶב אֱלֹהִ֣ים יִשְׁפֹּֽט׃
		עַד־מָתַ֥י תִּשְׁפְּטוּ־עָ֑וֶל וּפְנֵ֥י רְ֝שָׁעִ֗ים תִּשְׂאוּ־סֶֽלָה׃
		שִׁפְטוּ־דַ֥ל וְיָת֑וֹם עָנִ֖י וָרָ֣שׁ הַצְדִּֽיקוּ׃
		פַּלְּטוּ־דַ֥ל וְאֶבְי֑וֹן מִיַּ֖ד רְשָׁעִ֣ים הַצִּֽילוּ׃
		לֹ֤א יָדְע֨וּ ׀ וְלֹ֥א יָבִ֗ינוּ בַּחֲשֵׁכָ֥ה יִתְהַלָּ֑כוּ יִ֝מּ֗וֹטוּ כׇּל־מ֥וֹסְדֵי אָֽרֶץ׃
		אֲֽנִי־אָ֭מַרְתִּי אֱלֹהִ֣ים אַתֶּ֑ם וּבְנֵ֖י עֶלְי֣וֹן כֻּלְּכֶֽם׃
		אָ֭כֵן כְּאָדָ֣ם תְּמוּת֑וּן וּכְאַחַ֖ד הַשָּׂרִ֣ים תִּפֹּֽלוּ׃
		קוּמָ֣ה אֱ֭לֹהִים שׇׁפְטָ֣ה הָאָ֑רֶץ כִּֽי־אַתָּ֥ה תִ֝נְחַ֗ל בְּכׇל־הַגּוֹיִֽם׃
		
	\end{narrow}
	
	\shirshelyomintro{רְבִיעִי בַּשַּׁבָּת}
	\begin{narrow}
		\source{תהלים צד}%
		אֵל־נְקָמ֥וֹת יְיָ֑ אֵ֖ל נְקָמ֣וֹת הוֹפִֽיעַ׃
		הִ֭נָּשֵׂא שֹׁפֵ֣ט הָאָ֑רֶץ הָשֵׁ֥ב גְּ֝מ֗וּל עַל־גֵּאִֽים׃
		עַד־מָתַ֖י רְשָׁעִ֥ים ׀ יְיָ֑ עַד־מָ֝תַ֗י רְשָׁעִ֥ים יַעֲלֹֽזוּ׃
		יַבִּ֣יעוּ יְדַבְּר֣וּ עָתָ֑ק יִ֝תְאַמְּר֗וּ כׇּל־פֹּ֥עֲלֵי אָֽוֶן׃
		עַמְּךָ֣ יְיָ֣ יְדַכְּא֑וּ וְֽנַחֲלָתְךָ֥ יְעַנּֽוּ׃
		אַלְמָנָ֣ה וְגֵ֣ר יַהֲרֹ֑גוּ וִ֖יתוֹמִ֣ים יְרַצֵּֽחוּ׃
		וַ֭יֹּ֣אמְרוּ לֹ֣א יִרְאֶה־יָּ֑הּ וְלֹא־יָ֝בִ֗ין אֱלֹהֵ֥י יַעֲקֹֽב׃
		בִּ֭ינוּ בֹּעֲרִ֣ים בָּעָ֑ם וּ֝כְסִילִ֗ים מָתַ֥י תַּשְׂכִּֽילוּ׃
		הֲנֹ֣טַֽע אֹ֭זֶן הֲלֹ֣א יִשְׁמָ֑ע אִֽם־יֹ֥צֵֽר עַ֗֝יִן הֲלֹ֣א יַבִּֽיט׃
		הֲיֹסֵ֣ר גּ֭וֹיִם הֲלֹ֣א יוֹכִ֑יחַ הַֽמְלַמֵּ֖ד אָדָ֣ם דָּֽעַת׃
		יְיָ֗ יֹ֭דֵעַ מַחְשְׁב֣וֹת אָדָ֑ם כִּי־הֵ֥מָּה הָֽבֶל׃
		אַשְׁרֵ֤י ׀ הַגֶּ֣בֶר אֲשֶׁר־תְּיַסְּרֶ֣נּוּ יָּ֑הּ וּֽמִתּוֹרָתְךָ֥ תְלַמְּדֶֽנּוּ׃
		לְהַשְׁקִ֣יט ל֭וֹ מִ֣ימֵי רָ֑ע עַ֤ד יִכָּרֶ֖ה לָרָשָׁ֣ע שָֽׁחַת׃
		כִּ֤י ׀ לֹא־יִטֹּ֣שׁ יְיָ֣ עַמּ֑וֹ וְ֝נַחֲלָת֗וֹ לֹ֣א יַעֲזֹֽב׃
		כִּֽי־עַד־צֶ֭דֶק יָשׁ֣וּב מִשְׁפָּ֑ט וְ֝אַחֲרָ֗יו כׇּל־יִשְׁרֵי־לֵֽב׃
		מִֽי־יָק֣וּם לִ֭י עִם־מְרֵעִ֑ים מִי־יִתְיַצֵּ֥ב לִ֗֝י עִם־פֹּ֥עֲלֵי אָֽוֶן׃
		לוּלֵ֣י יְיָ֭ עֶזְרָ֣תָה לִּ֑י כִּמְעַ֓ט ׀ שָׁכְנָ֖ה דוּמָ֣ה נַפְשִֽׁי׃
		אִם־אָ֭מַרְתִּי מָ֣טָה רַגְלִ֑י חַסְדְּךָ֥ יְ֝יָ֗ יִסְעָדֵֽנִי׃
		בְּרֹ֣ב שַׂרְעַפַּ֣י בְּקִרְבִּ֑י תַּ֝נְחוּמֶ֗יךָ יְֽשַׁעַשְׁע֥וּ נַפְשִֽׁי׃
		הַֽ֭יְחׇבְרְךָ כִּסֵּ֣א הַוּ֑וֹת יֹצֵ֖ר עָמָ֣ל עֲלֵי־חֹֽק׃
		יָ֭גוֹדּוּ עַל־נֶ֣פֶשׁ צַדִּ֑יק וְדָ֖ם נָקִ֣י יַרְשִֽׁיעוּ׃
		וַיְהִ֬י יְיָ֣ לִ֣י לְמִשְׂגָּ֑ב וֵ֝אלֹהַ֗י לְצ֣וּר מַחְסִֽי׃
		וַיָּ֤שֶׁב עֲלֵיהֶ֨ם ׀ אֶת־אוֹנָ֗ם וּבְרָעָתָ֥ם יַצְמִיתֵ֑ם יַ֝צְמִיתֵ֗ם יְיָ֥ אֱלֹהֵֽינוּ׃\\
		\source{תהלים צה}%
		לְ֭כוּ נְרַנְּנָ֣ה לַייָ֑ נָ֝רִ֗יעָה לְצ֣וּר יִשְׁעֵֽנוּ׃
		נְקַדְּמָ֣ה פָנָ֣יו בְּתוֹדָ֑ה בִּ֝זְמִר֗וֹת נָרִ֥יעַֽ לֽוֹ׃
		כִּ֤י אֵ֣ל גָּד֣וֹל יְיָ֑ וּמֶ֥לֶךְ גָּ֝ד֗וֹל עַל־כׇּל־אֱלֹהִֽים׃
	\end{narrow}
	
	\shirshelyomintro{חַמִישִׁי בַּשַּׁבָּת}
	\begin{narrow}
		\source{תהלים פא}%
		לַמְנַצֵּ֬חַ ׀ עַֽל־הַגִּתִּ֬ית לְאָסָֽף׃
		הַ֭רְנִינוּ לֵאלֹהִ֣ים עוּזֵּ֑נוּ הָ֝רִ֗יעוּ לֵאלֹהֵ֥י יַעֲקֹֽב׃
		שְֽׂאוּ־זִ֭מְרָה וּתְנוּ־תֹ֑ף כִּנּ֖וֹר נָעִ֣ים עִם־נָֽבֶל׃
		תִּקְע֣וּ בַחֹ֣דֶשׁ שׁוֹפָ֑ר בַּ֝כֵּ֗סֶה לְי֣וֹם חַגֵּֽנוּ׃
		כִּ֤י חֹ֣ק לְיִשְׂרָאֵ֣ל ה֑וּא מִ֝שְׁפָּ֗ט לֵאלֹהֵ֥י יַעֲקֹֽב׃
		עֵ֤דוּת ׀ בִּיה֘וֹסֵ֤ף שָׂמ֗וֹ בְּ֭צֵאתוֹ עַל־אֶ֣רֶץ מִצְרָ֑יִם שְׂפַ֖ת לֹא־יָדַ֣עְתִּי אֶשְׁמָֽע׃
		הֲסִיר֣וֹתִי מִסֵּ֣בֶל שִׁכְמ֑וֹ כַּ֝פָּ֗יו מִדּ֥וּד תַּעֲבֹֽרְנָה׃
		בַּצָּרָ֥ה קָרָ֗אתָ וָאֲחַ֫לְּצֶ֥ךָּ אֶ֭עֶנְךָ בְּסֵ֣תֶר רַ֑עַם אֶבְחׇנְךָ֨ עַל־מֵ֖י מְרִיבָ֣ה סֶֽלָה׃
		שְׁמַ֣ע עַ֭מִּי וְאָעִ֣ידָה בָּ֑ךְ יִ֝שְׂרָאֵ֗ל אִם־תִּֽשְׁמַֽע־לִֽי׃
		לֹא־יִהְיֶ֣ה בְ֭ךָ אֵ֣ל זָ֑ר וְלֹ֥א תִ֝שְׁתַּחֲוֶ֗ה לְאֵ֣ל נֵכָֽר׃
		אָֽנֹכִ֨י ׀ יְ֘יָ֤ אֱלֹהֶ֗יךָ הַֽ֭מַּעַלְךָ מֵאֶ֣רֶץ מִצְרָ֑יִם הַרְחֶב־פִּ֗֝יךָ וַאֲמַלְאֵֽהוּ׃
		וְלֹֽא־שָׁמַ֣ע עַמִּ֣י לְקוֹלִ֑י וְ֝יִשְׂרָאֵ֗ל לֹא־אָ֥בָה לִֽי׃
		וָ֭אֲשַׁלְּחֵהוּ בִּשְׁרִיר֣וּת לִבָּ֑ם יֵ֝לְכ֗וּ בְּֽמוֹעֲצ֖וֹתֵיהֶֽם׃
		ל֗וּ עַ֭מִּי שֹׁמֵ֣עַֽ לִ֑י יִ֝שְׂרָאֵ֗ל בִּדְרָכַ֥י יְהַלֵּֽכוּ׃
		כִּ֭מְעַט אוֹיְבֵיהֶ֣ם אַכְנִ֑יעַ וְעַ֥ל צָ֝רֵיהֶ֗ם אָשִׁ֥יב יָדִֽי׃
		מְשַׂנְאֵ֣י יְיָ֭ יְכַחֲשׁוּ־ל֑וֹ וִיהִ֖י עִתָּ֣ם לְעוֹלָֽם׃
		וַֽ֭יַּאֲכִילֵהוּ מֵחֵ֣לֶב חִטָּ֑ה וּ֝מִצּ֗וּר דְּבַ֣שׁ אַשְׂבִּיעֶֽךָ׃
	\end{narrow}
	
	\shirshelyomintro{שִׁשִּׁי בַּשַּׁבָּת}
	\begin{narrow}
		\source{תהלים צג}%
		יְיָ֣ מָלָךְ֮ גֵּא֢וּת לָ֫בֵ֥שׁ לָבֵ֣שׁ יְיָ֭ עֹ֣ז הִתְאַזָּ֑ר אַף־תִּכּ֥וֹן תֵּ֝בֵ֗ל בַּל־תִּמּֽוֹט׃
		נָכ֣וֹן כִּסְאֲךָ֣ מֵאָ֑ז מֵעוֹלָ֣ם אָֽתָּה׃
		נָשְׂא֤וּ נְהָר֨וֹת ׀ יְיָ֗ נָשְׂא֣וּ נְהָר֣וֹת קוֹלָ֑ם יִשְׂא֖וּ נְהָר֣וֹת דׇּכְיָֽם׃
		מִקֹּל֨וֹת ׀ מַ֤יִם רַבִּ֗ים אַדִּירִ֣ים מִשְׁבְּרֵי־יָ֑ם אַדִּ֖יר בַּמָּר֣וֹם יְיָ׃
		עֵֽדֹתֶ֨יךָ ׀ נֶאֶמְנ֬וּ מְאֹ֗ד לְבֵיתְךָ֥ נַאֲוָה־קֹ֑דֶשׁ יְ֝יָ֗ לְאֹ֣רֶךְ יָמִֽים׃
	\end{narrow}
}

\newcommand{\RChBarekhi}{
	\instruction{בראש חדש׃}\space
	\firstword{בָּרְכִ֥י נַפְשִׁ֗י } \source{תהלים קד}
	אֶת־יְ֫יָ֥ יְיָ֣ אֱ֭לֹהַי גָּדַ֣לְתָּ מְּאֹ֑ד ה֖וֹד וְהָדָ֣ר לָבָֽשְׁתָּ׃
	עֹֽטֶה־א֭וֹר כַּשַּׂלְמָ֑ה נוֹטֶ֥ה שָׁ֝מַ֗יִם כַּיְרִיעָֽה׃
	הַ֥מְקָרֶ֥ה בַמַּ֗יִם עֲֽלִיּ֫וֹתָ֥יו הַשָּׂם־עָבִ֥ים רְכוּב֑וֹ הַֽ֝מְהַלֵּ֗ךְ עַל־כַּנְפֵי־רֽוּחַ׃
	עֹשֶׂ֣ה מַלְאָכָ֣יו רוּח֑וֹת מְ֝שָׁרְתָ֗יו אֵ֣שׁ לֹהֵֽט׃
	יָֽסַד־אֶ֭רֶץ עַל־מְכוֹנֶ֑יהָ בַּל־תִּ֝מּ֗וֹט עוֹלָ֥ם וָעֶֽד׃
	תְּ֭הוֹם כַּלְּב֣וּשׁ כִּסִּית֑וֹ עַל־הָ֝רִ֗ים יַ֖עַמְדוּ מָֽיִם׃
	מִן־גַּעֲרָ֣תְךָ֣ יְנוּס֑וּן מִן־ק֥וֹל רַֽ֝עַמְךָ֗ יֵחָפֵזֽוּן׃
	יַעֲל֣וּ הָ֭רִים יֵרְד֣וּ בְקָע֑וֹת אֶל־מְ֝ק֗וֹם זֶ֤ה ׀ יָסַ֬דְתָּ לָהֶֽם׃
	גְּֽבוּל־שַׂ֭מְתָּ בַּל־יַעֲבֹר֑וּן בַּל־יְ֝שֻׁב֗וּן לְכַסּ֥וֹת הָאָֽרֶץ׃
	הַֽמְשַׁלֵּ֣חַ מַ֭עְיָנִים בַּנְּחָלִ֑ים בֵּ֥ין הָ֝רִ֗ים יְהַלֵּכֽוּן׃
	יַ֭שְׁקוּ כׇּל־חַיְת֣וֹ שָׂדָ֑י יִשְׁבְּר֖וּ פְרָאִ֣ים צְמָאָֽם׃
	עֲ֭לֵיהֶם עוֹף־הַשָּׁמַ֣יִם יִשְׁכּ֑וֹן מִבֵּ֥ין עֳ֝פָאיִ֗ם יִתְּנוּ־קֽוֹל׃
	מַשְׁקֶ֣ה הָ֭רִים מֵעֲלִיּוֹתָ֑יו מִפְּרִ֥י מַ֝עֲשֶׂ֗יךָ תִּשְׂבַּ֥ע הָאָֽרֶץ׃
	מַצְמִ֤יחַ חָצִ֨יר ׀ לַבְּהֵמָ֗ה וְ֭עֵשֶׂב לַעֲבֹדַ֣ת הָאָדָ֑ם לְה֥וֹצִיא לֶ֗֝חֶם מִן־הָאָֽרֶץ׃
	וְיַ֤יִן ׀ יְשַׂמַּ֬ח לְֽבַב־אֱנ֗וֹשׁ לְהַצְהִ֣יל פָּנִ֣ים מִשָּׁ֑מֶן וְ֝לֶ֗חֶם לְֽבַב־אֱנ֥וֹשׁ יִסְעָֽד׃
	יִ֭שְׂבְּעוּ עֲצֵ֣י יְיָ֑ אַֽרְזֵ֥י לְ֝בָנ֗וֹן אֲשֶׁ֣ר נָטָֽע׃
	אֲשֶׁר־שָׁ֭ם צִפֳּרִ֣ים יְקַנֵּ֑נוּ חֲ֝סִידָ֗ה בְּרוֹשִׁ֥ים בֵּיתָֽהּ׃
	הָרִ֣ים הַ֭גְּבֹהִים לַיְּעֵלִ֑ים סְ֝לָעִ֗ים מַחְסֶ֥ה לַֽשְׁפַנִּֽים׃
	עָשָׂ֣ה יָ֭רֵחַ לְמוֹעֲדִ֑ים שֶׁ֝֗מֶשׁ יָדַ֥ע מְבוֹאֽוֹ׃
	תָּֽשֶׁת־חֹ֭שֶׁךְ וִ֣יהִי לָ֑יְלָה בּוֹ־תִ֝רְמֹ֗שׂ כׇּל־חַיְתוֹ־יָֽעַר׃
	הַ֭כְּפִירִים שֹׁאֲגִ֣ים לַטָּ֑רֶף וּלְבַקֵּ֖שׁ מֵאֵ֣ל אׇכְלָֽם׃
	תִּזְרַ֣ח הַ֭שֶּׁמֶשׁ יֵאָסֵפ֑וּן וְאֶל־מְ֝עוֹנֹתָ֗ם יִרְבָּצֽוּן׃
	יֵצֵ֣א אָדָ֣ם לְפׇעֳל֑וֹ וְֽלַעֲבֹ֖דָת֣וֹ עֲדֵי־עָֽרֶב׃
	מָה־רַבּ֬וּ מַעֲשֶׂ֨יךָ ׀ יְיָ֗ כֻּ֭לָּם בְּחׇכְמָ֣ה עָשִׂ֑יתָ מָלְאָ֥ה הָ֝אָ֗רֶץ קִנְיָנֶֽךָ׃
	זֶ֤ה ׀ הַיָּ֥ם גָּדוֹל֮ וּרְחַ֢ב יָ֫דָ֥יִם שָֽׁם־רֶ֭מֶשׂ וְאֵ֣ין מִסְפָּ֑ר חַיּ֥וֹת קְ֝טַנּ֗וֹת עִם־גְּדֹלֽוֹת׃
	שָׁ֭ם אֳנִיּ֣וֹת יְהַלֵּכ֑וּן לִ֝וְיָתָ֗ן זֶֽה־יָצַ֥רְתָּ לְשַֽׂחֶק־בּֽוֹ׃
	כֻּ֭לָּם אֵלֶ֣יךָ יְשַׂבֵּר֑וּן לָתֵ֖ת אׇכְלָ֣ם בְּעִתּֽוֹ׃
	תִּתֵּ֣ן לָ֭הֶם יִלְקֹט֑וּן תִּפְתַּ֥ח יָ֝דְךָ֗ יִשְׂבְּע֥וּן טֽוֹב׃
	תַּסְתִּ֥יר פָּנֶיךָ֮ יִֽבָּהֵ֫ל֥וּן תֹּסֵ֣ף ר֭וּחָם יִגְוָע֑וּן וְֽאֶל־עֲפָרָ֥ם יְשׁוּבֽוּן׃
	תְּשַׁלַּ֣ח ר֭וּחֲךָ יִבָּרֵא֑וּן וּ֝תְחַדֵּ֗שׁ פְּנֵ֣י אֲדָמָֽה׃
	יְהִ֤י כְב֣וֹד יְיָ֣ לְעוֹלָ֑ם יִשְׂמַ֖ח יְיָ֣ בְּמַעֲשָֽׂיו׃
	הַמַּבִּ֣יט לָ֭אָרֶץ וַתִּרְעָ֑ד יִגַּ֖ע בֶּהָרִ֣ים וְֽיֶעֱשָֽׁנוּ׃
	אָשִׁ֣ירָה לַייָ֣ בְּחַיָּ֑י אֲזַמְּרָ֖ה לֵאלֹהַ֣י בְּעוֹדִֽי׃
	יֶעֱרַ֣ב עָלָ֣יו שִׂיחִ֑י אָ֝נֹכִ֗י אֶשְׂמַ֥ח בַּייָ׃
	יִתַּ֤מּוּ חַטָּאִ֨ים ׀ מִן־הָאָ֡רֶץ וּרְשָׁעִ֤ים ׀ ע֤וֹד אֵינָ֗ם בָּרְכִ֣י נַ֭פְשִׁי אֶת־יְיָ֗ הַֽלְלוּ־יָֽהּ׃
}

\newcommand{\ledavid}{
	\englishinst{The following Psalm is said from Rosh \d{H}odesh Elul until Hoshana Rabba (some say only until Yom Kippur):}
	\firstword{לְדָוִ֨ד ׀ יְיָ֤ ׀ אוֹרִ֣י וְ֭יִשְׁעִי}\source{תהלים בז}
	מִמִּ֣י אִירָ֑א יְיָ֥ מָעוֹז־חַ֝יַּ֗י מִמִּ֥י אֶפְחָֽד׃
	בִּקְרֹ֤ב עָלַ֨י ׀ מְרֵעִים֮ לֶאֱכֹ֢ל אֶת־בְּשָׂ֫רִ֥י צָרַ֣י וְאֹיְבַ֣י לִ֑י הֵ֖מָּה כָשְׁל֣וּ וְנָפָֽלוּ׃
	אִם־תַּחֲנֶ֬ה עָלַ֨י ׀ מַחֲנֶה֮ לֹא־יִירָ֢א לִ֫בִּ֥י אִם־תָּק֣וּם עָ֭לַי מִלְחָמָ֑ה בְּ֝זֹ֗את אֲנִ֣י בוֹטֵֽחַ׃
	אַחַ֤ת ׀ שָׁאַ֣לְתִּי מֵֽאֵת־יְיָ אוֹתָ֢הּ אֲבַ֫קֵּ֥שׁ שִׁבְתִּ֣י בְּבֵית־יְיָ֭ כׇּל־יְמֵ֣י חַיַּ֑י לַחֲז֥וֹת בְּנֹעַם־יְ֝יָ֗ וּלְבַקֵּ֥ר בְּהֵֽיכָלֽוֹ׃
	כִּ֤י יִצְפְּנֵ֨נִי ׀ בְּסֻכֹּה֮ בְּי֢וֹם רָ֫עָ֥ה יַ֭סְתִּרֵנִי בְּסֵ֣תֶר אׇהֳל֑וֹ בְּ֝צ֗וּר יְרוֹמְמֵֽנִי׃
	וְעַתָּ֨ה יָר֪וּם רֹאשִׁ֡י עַ֤ל אֹיְבַ֬י סְֽבִיבוֹתַ֗י וְאֶזְבְּחָ֣ה בְ֭אׇהֳלוֹ זִבְחֵ֣י תְרוּעָ֑ה אָשִׁ֥ירָה וַ֝אֲזַמְּרָ֗ה לַֽייָ׃
	שְׁמַע־יְיָ֖ קוֹלִ֥י אֶקְרָ֗א וְחׇנֵּ֥נִי וַֽעֲנֵֽנִי׃
	לְךָ֤ ׀ אָמַ֣ר לִ֭בִּי בַּקְּשׁ֣וּ פָנָ֑י אֶת־פָּנֶ֖יךָ יְיָ֣ אֲבַקֵּֽשׁ׃
	אַל־תַּסְתֵּ֬ר פָּנֶ֨יךָ ׀ מִמֶּנִּי֮ אַ֥ל תַּט־בְּאַ֗ף עַ֫בְדֶּ֥ךָ עֶזְרָתִ֥י הָיִ֑יתָ אַֽל־תִּטְּשֵׁ֥נִי וְאַל־תַּ֝עַזְבֵ֗נִי אֱלֹהֵ֥י יִשְׁעִֽי׃
	כִּֽי־אָבִ֣י וְאִמִּ֣י עֲזָב֑וּנִי וַֽייָ֣ יַאַסְפֵֽנִי׃
	ה֤וֹרֵ֥נִי יְיָ֗ דַּ֫רְכֶּ֥ךָ וּ֭נְחֵנִי בְּאֹ֣רַח מִישׁ֑וֹר לְ֝מַ֗עַן שֽׁוֹרְרָֽי׃
	אַֽל־תִּ֭תְּנֵנִי בְּנֶ֣פֶשׁ צָרָ֑י כִּ֥י קָמוּ־בִ֥י עֵדֵי־שֶׁ֝֗קֶר וִיפֵ֥חַ חָמָֽס׃
	לׅׄוּׅׄלֵׅׄ֗אׅׄ הֶ֭אֱמַנְתִּי לִרְא֥וֹת בְּֽטוּב־יְיָ֗ בְּאֶ֣רֶץ חַיִּֽים׃
	קַוֵּ֗ה אֶל־יְ֫יָ֥ חֲ֭זַק וְיַאֲמֵ֣ץ לִבֶּ֑ךָ וְ֝קַוֵּ֗ה אֶל־יְיָ׃
}

\newcommand{\specialsaavos}{
	\begin{small}
		אֲ֭דֹנָי שְׂפָתַ֣י תִּפְתָּ֑ח וּ֝פִ֗י יַגִּ֥יד תְּהִלָּתֶֽךָ׃
		\source{תהלים נא}\\
	\end{small}
	\firstword{בָּרוּךְ}
	אַתָּה יְיָ אֱלֹהֵֽינוּ וֵאלֹהֵי אֲבוֹתֵֽינוּ אֱלֹהֵי אַבְרָהָם אֱלֹהֵי יִצְחָק וֵאלֹהֵי יַעֲקֹב הָאֵל הַגָּדוֹל הַגִּבּוֹר וְהַנּוֹרָא אֵל עֶלְיוֹן גּוֹמֵל חֲסָדִים טוֹבִים וְקוֹנֵה הַכֹּל וְזוֹכֵר חַסְדֵי אָבוֹת וּמֵבִיא גוֹאֵל לִבְנֵי בְנֵיהֶם לְמַֽעַן שְׁמוֹ בְּאַהֲבָה׃ מֶֽלֶךְ עוֹזֵר וּמוֹשִֽׁיעַ וּמָגֵן׃ בָּרוּךְ אַתָּה יְיָ מָגֵן אַבְרָהָם׃
}

\newcommand{\specialsameisim}{
	\firstword{אַתָּה}
	גִּבּוֹר לְעוֹלָם אֲדֹנָי מְחַיֵּה מֵתִים אַתָּה רַב לְהוֹשִֽׁיעַ׃
	
	\englishinst{From Shemini Atzeret till Pesach:}
	%\instruction{ממוסף של שמיני עצרת עד מוסף יום א׳ של פסח אומרים:}\\
	מַשִּׁיב הָרֽוּחַ וּמוֹרִיד הַגָּֽשֶׁם:
	%\footnote{\instruction{נ״א}: הַגֶּֽשֶׁם}
	
	\firstword{מְכַלְכֵּל}
	חַיִּים בְּחֶֽסֶד מְחַיֵּה מֵתִים בְּרַחֲמִים רַבִּים סוֹמֵךְ נוֹפְלִים וְרוֹפֵא חוֹלִים וּמַתִּיר אֲסוּרִים וּמְקַיֵּם אֱמוּנָתוֹ לִישֵׁנֵי עָפָר׃ מִי כָמֽוֹךָ בַּֽעַל גְּבוּרוֹת וּמִי דּֽוֹמֶה לָּךְ מֶֽלֶךְ מֵמִית וּמְחַיֶּה וּמַצְמִֽיחַ יְשׁוּעָה׃ וְנֶאֱמָן אַתָּה לְהַחֲיוֹת מֵתִים׃ בָּרוּךְ אַתָּה יְיָ מְחַיֵּה הַמֵּתִים׃
}

\newcommand{\kedusmusafchol}[2]{
	\ssubsection{\adforn{48} #1 \adforn{22}}
	
	\begin{small}
		%\setlength{\LTpost}{0pt}
		\begin{tabular}{l p{.85\textwidth}}
			
			\shatz &
			נְקַדֵּשׁ אֶת־שִׁמְךָ בָּעוֹלָם כְּשֵׁם שֶׁמַּקְדִּישִׁים אוֹתוֹ בִּשְׁמֵי מָרוֹם כַּכָּתוּב עַל יַד נְבִיאֶךָ קָרָ֨א זֶ֤ה אֶל־זֶה֙ וְאָמַ֔ר׃\\
			
			\shatzvkahal &
			\kadoshkadoshkadosh \\
			
			\shatz &
			לְעֻמָּתָם בָּרוּךְ יֹאמֵרוּ׃\\
			
			\shatzvkahal &
			\barukhhashem\\
			
			\shatz &
			וּבְדִבְרֵי קׇדְשְׁךָ כָּתוּב לֵאמֹר׃ \\
			
			\shatzvkahal &
			\yimloch\\
			
			\shatz &
			לְדוֹר וָדוֹר נַגִּיד גׇּדְלֶךָ וּלְנֵצַח נְצָחִים קְדֻשָּׁתְךָ נַקְדִּישׁ וְשִׁבְחֲךָ אֱלֹהֵֽינוּ מִפִּינוּ לֹא יָמוּשׁ לְעוֹלָם וָעֶד כִּי אֵל מֶלֶךְ גָּדוֹל וְקָדוֹשׁ אַֽתָּה׃ בָּרוּךְ אַתָּה יְיָ הָאֵל הַקָּדוֹשׁ׃ #2\\
		\end{tabular}
		
\end{small}}

\newcommand{\longpesicha}{
אֵין־כָּמ֖וֹךָ\source{תהלים פו} בָאֱלֹהִ֥ים ׀ אֲדֹנָ֗י וְאֵ֣ין כְּֽמַעֲשֶֽׂיךָ׃
מַֽלְכוּתְךָ֗ \source{תהלים קמה}מַלְכ֥וּת כׇּל־עֹלָמִ֑ים וּ֝מֶֽמְשַׁלְתְּךָ֗ בְּכׇל־דּ֥וֹר וָדֹֽר׃
\melekhmalakhyimlokh 
יְיָ֗ \source{תהלים כט}עֹ֭ז לְעַמּ֣וֹ יִתֵּ֑ן יְיָ֓ ׀ יְבָרֵ֖ךְ אֶת־עַמּ֣וֹ בַשָּׁלֽוֹם׃\\
אַב הָרַחֲמִים\source{תהלים נא} הֵיטִ֣יבָה בִ֭רְצוֹנְךָ אֶת־צִיּ֑וֹן תִּ֝בְנֶ֗ה חוֹמ֥וֹת יְרוּשָׁלָֽ‍ִם׃
כִּי בְךָ לְבַד בָּטָֽחְנוּ מֶֽלֶךְ אֵל רָם וְנִשָּׂא אֲדוֹן עוֹלָמִים׃

\pesicha
}

\newcommand{\yekumpurkans}{
\firstword{יְקוּם פֻּרְקָן}
מִן שְׁמַיָּא חִנָּא וְחִסְדָּא וְרַחֲמֵי וְחַיֵּי אֲרִיכֵי וּמְזוֹנֵי רְוִיחֵי וְסִיַּעְתָּא דִשְּׁמַיָּא וּבַרְיוּת גּוּפָא וּנְהוֹרָא מַעַלְיָא׃ זַרְעָא חַיָּא וְקַיָּמָא זַרְעָא דִּי לֹא יִפְסוּק וְדִי לָא יִבְטוּל מִפִּתְגָּמֵי אוֹרַיְתָא׃ לְמָרָנָן וְרַבָּנָן חֲבוּרָתָא קַדִּישָׁתָא דִּי בְאַרְעָא דְיִשְׂרָאֵל וְדִי בְּבָבֶל לְרֵישֵׁי כַלֵּי וּלְרֵישֵׁי גַלְוָתָא וּלְרֵישֵׁי מְתִיבָתָא וּלְדַיָּנֵי דִי בָבָא׃ לְכׇל־תַּלְמִידֵיהוֹן וּלְכׇל־תַּלְמִידֵי תַלְמִידֵיהוֹן וּלְכׇל־מָן דְּעָסְקִין בְּאוֹרַיְתָא׃ מַלְכָּא דְעָלְמָא יְבָרֵךְ יַתְהוֹן יַפִּישׁ חַיֵּיהוֹן וְיַסְגֵּא יוֹמֵיהוֹן וְיִתֵּן אָרְכָה לִשְׁנֵיהוֹן וְיִתְפָּרְקוּן וְיִשְׁתֵּזְבוּן מִן כׇּל־עָקָא וּמִן כׇּל־מַרְעִין בִּישִׁין מָרָן דִּי בִשְׁמַיָּא יְהֵא בְּסַעְדְּהוֹן כׇּל־זְמַן וְעִדָּן׃ וְנֹאמַר אָמֵן׃



\firstword{יְקוּם פֻּרְקָן}
מִן שְׁמַיָּא חִנָּא וְחִסְדָּא וְרַחֲמֵי וְחַיֵּי אֲרִיכֵי וּמְזוֹנֵי רְוִיחֵי וְסִיַּעְתָּא דִּשְׁמַיָּא וּבַרְיוּת גּוּפָא וּנְהוֹרָא מַעַלְיָא׃ זַרְעָא חַיָּא וְקַיָּמָא זַרְעָא דִּי לָא יִפְסוּק וְדִי לָא יִבְטוּל מִפִּתְגָּמֵי אוֹרַיְתָא׃ לְכׇל־קְהָלָא קַדִּישָׁא הָדֵין רַבְרְבַיָּא עִם זְעֵרַיָּא טַפְלָא וּנְשַׁיָּא׃ מַלְכָּא דְעָלְמָא יְבָרֵךְ יָתְכוֹן יַפִּישׁ חַיֵּיכוֹן וְיַסְגֵּא יוֹמֵיכוֹן וְיִתֵּן אָרְכָה לִשְׁנֵיכוֹן וְתִתְפָּרְקוּן וְתִשְׁתֵּזְבוּן מִן כׇּל־עָקָא וּמִן כׇּל־מַרְעִין בִּישִׁין מָרָן דִּי בִשְׁמַיָּא יְהֵא בְּסַעְדְּכוֹן כׇּל־זְמַן וְעִדָּן׃ וְנֹאמַר אָמֵן׃

\firstword{מִי שֶׁבֵּירַךְ}
אֲבוֹתֵֽינוּ אַבְרָהָם יִצְחָק וְיַעֲקֹב הוּא יְבָרֵךְ אֶת־כׇּל־הַקָּהָל הַקָּדוֹשׁ הַזֶּה עִם כׇּל־קְהִילּוֹת הַקּוֹדֶשׁ הֵם וּמִשְׁפְּחוֹתֵיהֶם וְכׇל־אֲשֶׁר לָהֶם׃ וּמִי שֶׁמְּיַחֲדִים בָּתֵּי־כְנֵסִיּוֹת לִתְפִלָּה וּמִי שֶׁבָּאִים בְּתוֹכָם לְהִתְפַּלֵּל וּמִי שֶׁנּוֹתְנִים נֵר לַמָּאוֹר וְיַֽיִן לְקִדּוּשׁ וּלְהַבְדָּלָה וּפַת לְאוֹרְחִים וּצְדָקָה לַעֲנִיִּים וְכׇל־מִי שֶׁעוֹסְקִים בְּצׇרְכֵי צִבּוּר בֶּאֱמוּנָה הַקָּדוֹשׁ בָּרוּךְ הוּא יְשַׁלֵם שְׂכָרָם וְיָסִיר מֵהֶם כׇּל־מַחֲלָה וְיִרְפָּא לְכׇל־גּוּפָם וְיִסְלַח לְכׇל־עֲוֹנָם וְיִשְׁלַח בְּרָכָה וְהַצְלָחָה בְּכׇל־מַעֲשֵׂה יְדֵיהֶם עִם כׇּל־יִשְׂרָאֵל אֲחֵיהֶם וְנֹאמַר אָמֵן׃
}

\newcommand{\shomeryisroel}{
	\englishinst{Sit upright for this paragraph.}
	\textbf{שׁוֹמֵר יִשְׂרָאֵל}
	שְׁמוֹר שְׁאֵרִית יִשְׂרָאֵל וְאַל יֹאבַד יִשְׂרָאֵל הָאוֹמְרִים שְׁמַע יִשְׂרָאֵל׃
	שׁוֹמֵר גּוֹי אֶחָד שְׁמוֹר שְׁאֵרִית עַם אֶחָד וְאַל יֹאבַד גּוֹי אֶחָד
	הַמְיַחֲדִים שִׁמְךָ יְיָ אֱלֹהֵֽינוּ יְיָ אֶחָד׃
	שׁוֹמֵר גּוֹי קָדוֹשׁ שְׁמוֹר שְׁאֵרִית עַם קָדוֹשׁ
	וְאַל יֹאבַד גּוֹי קָדוֹשׁ הַמְשַׁלְּשִׁים בְּשָׁלוֹשׁ קְדֻשּׁוֹת לְקָדוֹשׁ׃
	מִתְרַצֶּה בְּרַחֲמִים וּמִתְפַּיֵּס בְּתַחֲנוּנִים הִתְרַצֶּה וְהִתְפַּיֵּס לְדוֹר עָנִי כִּי אֵין עוֹזֵר׃
	אָבִינוּ מַלְכֵּנוּ חׇנֵּנוּ וַעֲנֵנוּ כִּי אֵין בָּנוּ מַעֲשִׂים עֲשֵׂה עִמָּנוּ צְדָקָה וָחֶסֶד וְהוֹשִׁיעֵנוּ׃\\
	\englishinst{Stand for the recitation of the following paragraph.}
	\firstword{וַאֲנַ֗חְנוּ}\source{דה״ב כ}
	לֹ֤א נֵדַע֙ מַֽה־נַּעֲשֶׂ֔ה כִּ֥י עָלֶ֖יךָ עֵינֵֽינוּ׃
	זְכֹר־רַחֲמֶ֣יךָ
	\source{תהלים כה}%
	יְיָ֭ וַחֲסָדֶ֑יךָ כִּ֖י מֵעוֹלָ֣ם הֵֽמָּה׃
	יְהִי־חַסְדְּךָ֣ \source{תהלים לג}יְיָ֣ עָלֵ֑ינוּ כַּֽ֝אֲשֶׁ֗ר יִחַ֥לְנוּ לָֽךְ׃
	אַֽל־תִּזְכׇּר־לָנוּ֮ \source{תהלים עט}עֲוֺנֹ֢ת רִאשֹׁ֫נִ֥ים מַ֭הֵר יְקַדְּמ֣וּנוּ רַחֲמֶ֑יךָ כִּ֖י דַלּ֣וֹנוּ מְאֹֽד׃
	חׇנֵּ֣נוּ \source{תהלים קכג}יְיָ֣ חׇנֵּ֑נוּ כִּי־רַ֗֝ב שָׂבַ֥עְנוּ בֽוּז׃
	בְּרֹ֖גֶז \source{חבקוק ג}רַחֵ֥ם תִּזְכּֽוֹר׃
	כִּי־ה֭וּא \source{תהלים קג} יָדַ֣ע יִצְרֵ֑נוּ זָ֝כ֗וּר כִּי־עָפָ֥ר אֲנָֽחְנוּ׃
	עׇזְרֵ֤נוּ \source{תהלים עט}
	׀ אֱלֹ֘הֵ֤י יִשְׁעֵ֗נוּ עַֽל־דְּבַ֥ר כְּבֽוֹד־שְׁמֶ֑ךָ וְהַצִּילֵ֥נוּ וְכַפֵּ֥ר עַל־חַ֝טֹּאתֵ֗ינוּ לְמַ֣עַן שְׁמֶֽךָ׃
}

\newcommand{\nefilasapayim}{
	
	\englishinst{In a place with a Sefer Torah, the following is recited leaning the head on the left forearm, unless tefillin are worn on that arm in which case it is recited leaning on the right arm.}
	וַיֹּ֧אמֶר דָּוִ֛ד אֶל־גָּ֖ד\source{שמ״ב כד} צַר־לִ֣י מְאֹ֑ד נִפְּלָה־נָּ֤א בְיַד־יְיָ֙ כִּֽי־רַבִּ֣ים רַֽחֲמָ֔ו וּבְיַד־אָדָ֖ם אַל־אֶפֹּֽלָה׃\\
	\firstword{רַחוּם וְחַנּוּן,}
	חָטָֽאתִי לְפָנֶֽיךָ יְיָ מָלֵא רַחֲמִים רַחֵם עָלַי וְקַבֵּל תַּחֲנוּנָי׃
	יְיָ֗ \source{תהלים ו}אַל־בְּאַפְּךָ֥ תוֹכִיחֵ֑נִי וְֽאַל־בַּחֲמָתְךָ֥ תְיַסְּרֵֽנִי׃
	חׇנֵּ֥נִי יְיָ כִּ֤י אֻמְלַ֫ל אָ֥נִי רְפָאֵ֥נִי יְיָ֑ כִּ֖י נִבְהֲל֣וּ עֲצָמָֽי׃
	וְ֭נַפְשִׁי נִבְהֲלָ֣ה מְאֹ֑ד וְאַתָּ֥ה יְ֝יָ֗ עַד־מָתָֽי׃
	שׁוּבָ֣ה יְיָ֭ חַלְּצָ֣ה נַפְשִׁ֑י ה֝וֹשִׁיעֵ֗נִי לְמַ֣עַן חַסְדֶּֽךָ׃
	כִּ֤י אֵ֣ין בַּמָּ֣וֶת זִכְרֶ֑ךָ בִּ֝שְׁא֗וֹל מִ֣י יֽוֹדֶה־לָּֽךְ׃
	יָגַ֤עְתִּי ׀ בְּֽאַנְחָתִ֗י אַשְׂחֶ֣ה בְכׇל־לַ֭יְלָה מִטָּתִ֑י בְּ֝דִמְעָתִ֗י עַרְשִׂ֥י אַמְסֶֽה׃
	עָשְׁשָׁ֣ה מִכַּ֣עַס עֵינִ֑י עָ֝תְקָ֗ה בְּכׇל־צוֹרְרָֽי׃
	ס֣וּרוּ מִ֭מֶּנִּי כׇּל־פֹּ֣עֲלֵי אָ֑וֶן כִּֽי־שָׁמַ֥ע יְ֝יָ֗ ק֣וֹל בִּכְיִֽי׃
	שָׁמַ֣ע יְיָ֭ תְּחִנָּתִ֑י יְ֝יָ֗ תְּֽפִלָּתִ֥י יִקָּֽח׃
	יֵבֹ֤שׁוּ ׀ וְיִבָּהֲל֣וּ מְ֭אֹד כׇּל־אֹיְבָ֑י יָ֝שֻׁ֗בוּ יֵבֹ֥שׁוּ רָֽגַע׃
}

\newcommand{\yishtabach}{\firstword{יִשְׁתַּבַּח}
שִׁמְךָ לָעַד מַלְכֵּֽנוּ הָאֵל הַמֶּֽלֶךְ הַגָּדוֹל וְהַקָּדוֹשׁ בַּשָׁמַֽיִם וּבָאָֽרֶץ \middot כִּי לְךָ נָאֶה יְיָ אֱלֹהֵֽינוּ וֵאלֹהֵי אֲבוֹתֵֽינוּ שִׁיר וּשְׁבָחָה הַלֵּל וְזִמְרָה עֹז וּמֶמְשָׁלָה נֶֽצַח גְּדֻלָּה וּגְבוּרָה תְּהִלָּה וְתִפְאֶֽרֶת קְדֻשָּׁה וּמַלְכוּת בְּרָכוֹת וְהוֹדָאוֹת מֵעַתָּה וְעַד עוֹלָם׃ בָּרוּךְ אַתָּה יְיָ אֵל מֶֽלֶךְ גָּדוֹל בַּתֻּשְׁבָּחוֹת אֵל הַהוֹדָאוֹת אֲדוֹן הַנִּפְלָאוֹת הַבּוֹחֵר בְּשִׁירֵי זִמְרָה מֶֽלֶךְ אֵל חֵי הָעוֹלָמִים׃}

\newcommand{\shabmusafpesukim}{\firstword{וּבְיוֹם֙ הַשַּׁבָּ֔ת }\source{במדבר כח}
	שְׁנֵֽי־כְבָשִׂ֥ים בְּנֵֽי־שָׁנָ֖ה תְּמִימִ֑ם וּשְׁנֵ֣י עֶשְׂרֹנִ֗ים סֹ֧לֶת מִנְחָ֛ה בְּלוּלָ֥ה בַשֶּׁ֖מֶן וְנִסְכּֽוֹ׃
	עֹלַ֥ת שַׁבַּ֖ת בְּשַׁבַּתּ֑וֹ עַל־עֹלַ֥ת הַתָּמִ֖יד וְנִסְכָּֽהּ׃}

\newcommand{\shabbosshuva}{בשבת שובה׃}


\newcommand{\shabboskiddushhashem}{
	\firstword{אַתָּה קָדוֹשׁ}
	וְשִׁמְךָ קָדוֹשׁ וּקְדוֹשִׁים בְּכׇל־יוֹם יְהַלְלוּךָ סֶּֽלָה׃ בָּרוּךְ אַתָּה יְיָ *הָאֵל
	(*\instruction{בשבת שובה:}
	הַמֶּֽלֶךְ)
	הַקָּדוֹשׁ׃
}


\newcommand{\shabboskiddushhayom}[1]{{
		\firstword{אֱלֹהֵינוּ}
		וֵאלֹהֵי אֲבוֹתֵינוּ רְצֵה בִמְנוּחָתֵנוּ קַדְּשֵׁנוּ בְּמִצְוֹתֶיךָ וְתֵן חֶלְקֵנוּ בְּתוֹרָתֶךָ \middot שַׂבְּעֵנוּ מִטּוּבֶךָ וְשַׂמְּחֵנוּ בִּישׁוּעָתֶךָ וְטַהֵר לִבֵּנוּ לְעׇבְדְּךָ בֶּאֱמֶת׃ וְהַנְחִילֵנוּ יְיָ אֱלֹהֵינוּ בְּאַהֲבָה וּבְרָצוֹן שַׁבַּת קׇדְשֶׁךָ \middot וְיָנוּחוּ בָהּ#1 יִשְׂרָאֵל מְקַדְּשֵׁי שְׁמֶךָ׃
		בָּרוּךְ אַתָּה יְיָ מְקַדֵּשׁ הַשַּׁבָּת׃
}}

\newcommand{\YTShabboshavdalah}{
	
	\begin{sometimes}
		
		\instruction{במוצאי שבת:}
		וַתּוֹדִיעֵֽנוּ יְיָ אֱלֹהֵֽינוּ אֶת־מִשְׁפְּטֵי צִדְקֶֽךָ וַתְּלַמְּדֵֽנוּ לַעֲשׂוֹת חֻקֵּי רְצוֹנֶֽךָ וַתִּתֶּן־לָֽנוּ יְיָ אֱלֹהֵֽינוּ מִשְׁפָּטִים יְשָׁרִים וְתוֹרוֹת אֱמֶת חֻקִּים וּמִצְוֹת טוֹבִים׃ וַתַּנְחִילֵֽנוּ זְמַנֵּי שָׂשׂוֹן וּמֽוֹעֲדֵי קֹֽדֶשׁ וְחַגֵּי נְדָבָה׃ וַתּוֹרִישֵֽׁנוּ קְדֻשַּׁת שַׁבָּת וּכְבוֹד מוֹעֵד וַחֲגִיגַת הָרֶֽגֶל׃ וַתַּבְדִּילֵֽנוּ יְיָ אֱלֹהֵֽינוּ בֵּין קֹֽדֶשׁ לְחוֹל בֵּין אוֹר לְחֹֽשֶׁךְ בֵּין יִשְׂרָאֵל לָעַמִּים בֵּין יוֹם הַשְּׁבִיעִי לְשֵֽׁשֶׁת יְמֵי הַמַּעֲשֶׂה׃ בֵּין קְדֻשַּׁת שַׁבָּת לִקְדֻשַּׁת יוֹם טוֹב הִבְדַּֽלְתָּ וְאֶת־יוֹם הַשְּׁבִיעִי מִשֵּֽׁשֶׁת יְמֵי הַמַּעֲשֶׂה קִדַּֽשְׁתָּ הִבְדַּֽלְתָּ וְקִדַּֽשְׁתָּ אֶת־עַמְּךָ יִשְׂרָאֵל בִּקְדֻשָּׁתֶֽךָ׃
		
	\end{sometimes}
	
}

\newcommand{\atavechartanu}{\firstword{אַתָּה בְחַרְתָּֽנוּ}
	מִכׇּל־הָעַמִּים אָהַֽבְתָּ אוֹתָֽנוּ וְרָצִֽיתָ בָּֽנוּ וְרוֹמַמְתָּֽנוּ מִכׇּל־הַלְּשׁוֹנוֹת וְקִדַּשְׁתָּֽנוּ בְּמִצְוֹתֶֽיךָ וְקֵרַבְתָּֽנוּ מַלְכֵּֽנוּ לַעֲבוֹדָתֶֽךָ וְשִׁמְךָ הַגָּדוֹל וְהַקָּדוֹשׁ עָלֵֽינוּ קָרָֽאתָ׃}

\newcommand{\ytkiddushhayom}[1]{{
		\atavechartanu
		
		\enlargethispage{\baselineskip}
		
		#1
		
		\firstword{וַתִּתֶּן}
		לָֽנוּ יְיָ אֱלֹהֵֽינוּ בְּאַהֲבָה
		\shabaddition{שַׁבָּתוֹת לִמְנוּחָה וּ}		]
		מוֹעֲדִים
		לְשִׂמְחָה חַגִּים וּזְמַנִּים לְשָׂשׂוֹן אֶת־יוֹם
		\shabaddition{הַשַּׁבָּת הַזֶּה וְאֶת־יוֹם}
		
		
		\begin{tabular}{>{\centering\arraybackslash}m{.2\textwidth} | >{\centering\arraybackslash}m{.2\textwidth} | >{\centering\arraybackslash}m{.2\textwidth} | >{\centering\arraybackslash}m{.24\textwidth}}
			
			\instruction{לפסח} & \instruction{לשבעות} & \instruction{לסכות} & \instruction{לשמיני עצרת} \\
			
			חַג הַמַּצּוֹת הַזֶּה זְמַן חֵרוּתֵֽנוּ & חַג הַשָּׁבֻעוֹת הַזֶּה זְמַן מַתַּן תּוֹרָתֵֽנוּ & חַג הַסֻּכּוֹת הַזֶּה זְמַן שִׂמְחָתֵֽנוּ & שְׁמִינִי חַג הָעֲצֶֽרֶת הַזֶּה זְמַן שִׂמְחָתֵֽנוּ
		\end{tabular}
		
		\shabaddition{בְּאַהֲבָה}
		מִקְרָא קֹֽדֶשׁ זֵֽכֶר לִיצִיאַת מִצְרָֽיִם׃
		
		\yaalehveyavotemplate{
		\begin{tabular}{>{\centering\arraybackslash}m{.2\textwidth} | >{\centering\arraybackslash}m{.2\textwidth} | >{\centering\arraybackslash}m{.2\textwidth} | >{\centering\arraybackslash}m{.24\textwidth}}
			
			\instruction{לפסח} & \instruction{לשבעות} & \instruction{לסכות} & \instruction{לשמיני עצרת} \\
			
			חַג הַמַּצּוֹת הַזֶּה & חַג הַשָּׁבֻעוֹת הַזֶּה & חַג הַסֻּכּוֹת הַזֶּה & שְׁמִינִי חַג הָעֲצֶֽרֶת הַזֶּה
		\end{tabular}}
		
		\firstword{וְהַשִּׂיאֵֽנוּ}
		יְיָ אֱלֹהֵֽינוּ אֶת־בִּרְכַּת מוֹעֲדֶֽיךָ לְחַיִּים וּלְשָׁלוֹם לְשִׂמְחָה וּלְשָׂשׂוֹן כַּאֲשֶׁר רָצִֽיתָ וְאָמַֽרְתָּ לְבָרְכֵֽנוּ׃ [\shabbos%
		אֱלֹהֵֽינוּ וֵאלֹהֵי אֲבוֹתֵֽינוּ רְצֵה בִמְנוּחָתֵֽנוּ] קַדְּשֵֽׁנוּ בְּמִצְוֹתֶֽיךָ וְתֵן חֶלְקֵֽנוּ בְּתוֹרָתֶֽךָ שַׂבְּעֵֽנוּ מִטּוּבֶֽךָ וְשַׂמְּחֵֽנוּ בִּישׁוּעָתֶֽךָ וְטַהֵר לִבֵּֽנוּ לְעׇבְדְּךָ בֶּאֱמֶת וְהַנְחִילֵֽנוּ יְיָ אֱלֹהֵֽינוּ \shabaddition{בְּאַהֲבָה וּבְרָצוֹן} בְּשִׂמְחָה וּבְשָׂשׂוֹן
		\shabaddition{שַׁבַּת וּ}
		מוֹעֲדֵי קׇדְשֶֽׁךָ וְיִשְׂמְחוּ בְךָ יִשְׂרָאֵל מְקַדְּשֵׁי שְׁמֶךָ׃ בָּרוּךְ אַתָּה יְיָ מְקַדֵּשׁ
		\shabaddition{הַשַּׁבָּת וְ}
		 יִשְׂרָאֵל וְהַזְּמַנִּים׃
}}

\newcommand{\shabboschanukah}{
	\begin{sometimes}
		
		\instruction{בחנוכה:}
		עַל הַנִּסִּים וְעַל הַפֻּרְקָן וְעַל הַגְּבוּרוֹת וְעַל הַתְּשׁוּעוֹת וְעַל הַמִּלְחָמוֹת
		שֶׁעָשִֽׂיתָ לַאֲבוֹתֵֽינוּ בַּיָּמִים הָהֵם בַּזְּמַן הַזֶּה׃
		\bimeimatityahu
		
	\end{sometimes}
}

\newcommand{\veshameru}{
	\source{שמות לא}\firstword{וְשָׁמְר֥וּ}
	בְנֵֽי־יִשְׂרָאֵ֖ל אֶת־הַשַּׁבָּ֑ת לַעֲשׂ֧וֹת אֶת־הַשַּׁבָּ֛ת לְדֹרֹתָ֖ם בְּרִ֥ית עוֹלָֽם׃ בֵּינִ֗י וּבֵין֙ בְּנֵ֣י יִשְׂרָאֵ֔ל א֥וֹת הִ֖וא לְעֹלָ֑ם כִּי־שֵׁ֣שֶׁת יָמִ֗ים עָשָׂ֤ה יְיָ֙ אֶת־הַשָּׁמַ֣יִם וְאֶת־הָאָ֔רֶץ וּבַיּוֹם֙ הַשְּׁבִיעִ֔י שָׁבַ֖ת וַיִּנָּפַֽשׁ׃}

\newcommand{\personalfast}{
		
\englishinst{If a person wants to accept a voluntary fast, they say the following text at Min\d{h}a the preceding day:}
		רִבּוֹן הָעוֹלָמִים הֲרֵי אֲנִי לְפָנֶיךָ בְּתַעֲנִית נְדָבָה לְמָחָר׃\\
\englishinst{If they plan to fast a partial day, add:}
		עַד חֲצִי הַיּוֹם: \instruction{או:} עַד אַחֲרֵי תְּפִלַת מִנְחָה:\\
		יְהִי רָצוֹן מִלְּפָנֶֽיךָ יְיָ אֱלֹהַי וֵאלֹהֵי אֲבוֹתַי שֶׁתְּקַבְּלֵֽנִי בְּאַהֲבָה וּבְרָצוֹן וְתָבֹא לְפָנֶיךָ תְּפִלָתִי
		וְתַעֲנֶה עֲתִירָתִי בְּרַחֲמֶֽיךָ הָרַבִּים: כִּי אַתָּה שׁוֹמֵֽעַ תְּפִלַת כׇּל־פֶּה: \instruction{יהיו לרצון ...}
			
\englishinst{At min\d{h}a on the day of a personal fast say:}
		רִבּוֹן הָעוֹלָמִים גָּלוּי וְיָדֽוּעַ לְפָנֶיךָ בִּזְמַן שֶׁבֵּית הַמִּקְדָּשׁ קַיָּם אָדָם חוֹטֵא מַקְרִיב קׇרְבָּן וְאֵין מַקְרִיבִין מִמֶּֽנּוּ אֶלָּא חֶלְבּוֹ וְדָמוֹ וְאַתָּה בְּרַחֲמֶֽיךָ הָרַבִּים מְכַפֵּר \middot וְעַכְשָׁיו יָשַֽׁבְתִּי בְּתַעֲנִית וְנִתְמַעֵט חֶלְבִּי וְדָמִי׃ יְהִי רָצוֹן מִלְּפָנֶֽיךָ שֶׁיְּהִי חֶלְבִּי וְדָמִי שֶׁנִּתְמַעַט הַיּוֹם כְּאִילּוּ הִקְרַבְתִּיו לְפָנֶֽיךָ עַל גַּבֵּי הַמִּזְבֵּֽחַ וְתִרְצֵֽנִי׃}

\newcommand{\shalomravbase}{\firstword{שָׁלוֹם}
	רָב עַל יִשְׂרָאֵל עַמְּךָ תָּשִׂים לְעוֹלָם \middot כִּי אַתָּה הוּא מֶֽלֶךְ אָדוֹן לְכׇל־הַשָּׁלוֹם׃}

\newcommand{\simshalombase}{\firstword{שִׂים שָׁלוֹם}
	טוֹבָה וּבְרָכָה חֵן וָחֶֽסֶד וְרַחֲמִים עָלֵֽינוּ וְעַל כׇּל־יִשְׂרָאֵל עַמֶּֽךָ \middot בָּרְכֵֽנוּ אָבִֽינוּ כֻּלָּֽנוּ כְּאֶחָד בְּאוֹר פָּנֶֽיךָ \middot כִּי בְאוֹר פָּנֶֽיךָ נָתַֽתָּ לָֽנוּ יְיָ אֱלֹהֵֽינוּ תּוֹרַת חַיִּים וְאַהֲבַת חֶֽסֶד וּצְדָקָה וּבְרָכָה וְרַחֲמִים וְחַיִּים וְשָׁלוֹם \middot}

\newcommand{\vetov}{וְטוֹב בְּעֵינֶֽיךָ לְבָרֵךְ אֶת־עַמְּךָ יִשְׂרָאֵל בְּכׇל־עֵת וּבְכׇל־שָׁעָה בִּשְׁלוֹמֶֽךָ׃}

\newcommand{\shalomendingAYT}[1]{
\columnratio{0.7}
\begin{paracol}{2}
	
	\instruction{#1}
	\begin{small}
		בְּסֵֽפֶר חַיִּים בְּרָכָה וְשָׁלוֹם וּפַרְנָסָה טוֹבָה \middot נִזָּכֵר וְנִכָּתֵב לְפָנֶֽיךָ אָֽנוּ וְכׇל־עַמְּךָ בֵּית יִשְׂרָאֵל לְחַיִּים וּלְשָׁלוֹם׃ בָּרוּךְ אַתָּה יְיָ עוֹשֵׂה הַשָּׁלוֹם׃
		
	\end{small}
	\switchcolumn
	בָּרוּךְ אַתָּה יְיָ הַמְבָרֵךְ אֶת־עַמּוֹ יִשְׂרָאֵל בַּשָּׁלוֹם׃
\end{paracol}
}

\newcommand{\simshalom}[1]{
	\simshalombase 
\vetov
	\shalomendingAYT{בעשי״ת׃}
}

\newcommand{\shabbossimshalom}{
	\simshalombase
\vetov
	
	\shalomendingAYT{בשבת שובה׃}
}

\newcommand{\shabbosshalomrav}{
	\shalomravbase
\vetov

\shalomendingAYT{בשבת שובה׃}
}

\newcommand{\simshalomrav}{
\columnratio{0.3}
\begin{paracol}{2}
	\instruction{במנחה׃}\\
	\shalomravbase
	\switchcolumn
	\instruction{בשחרית ובמנחה בת״צ׃}\\
	\simshalombase
	
\end{paracol}

\vetov

\shalomendingAYT{בעשי״ת׃}
}

\newcommand{\simshalomplain}{\simshalombase}

\newcommand{\savri}{\instruction{סַבְרִי מָרָנָן וְרְבָּנָן וְרַבּוֹתַי}\\}

\newcommand{\pitumhaketoret}{
	\firstword{פִּטּוּם הַקְּטֹֽרֶת׃}\source{מסכת כריתות}
	(א) הַצֳּרִי (ב) וְהַצִּפֹּֽרֶן (ג) וְהַחֶלְבְּנָה (ד) וְהַלְּבוֹנָה מִשְׁקַל שִׁבְעִים שִׁבְעִים מָנֶה (ה) מֹר (ו) וּקְצִיעָה (ז) שִׁבֹּֽלֶת נֵרְדְּ (ח) וְכַרְכֹּם מִשְׁקַל שִׁשָּׁה עָשָׂר שִׁשָּׁה עָשָׂר מָנֶה (ט) הַקֹּשְׁטְ שְׁנֵים עָשָׂר (י) וְקִלּוּפָה שְׁלֹשָׁה (יא) וְקִנָּמוֹן תִּשְׁעָה׃ בֹּרִית כַּרְשִׁינָה תִּשְׁעָה קַבִּין יֵין קַפְרִיסִין סְאִין תְּלָתָא וְקַבִּין תְּלָתָא וְאִם אֵין לוֹ יֵין קַפְרִיסִין מֵבִיא חֲמַר חִוַּרְיָן עַתִּיק מֶֽלַח סְדוֹמִית רֹבַע [הַקָּב] מַעֲלֶה עָשָׁן כׇּל־שֶׁהוּא׃ רַבִּי נָתָן אוֹמֵר׃ אַף כִּפַּת הַיַּרְדֵּן כׇּל־שֶׁהוּא וְאִם נָתַן בָּהּ דְּבַשׁ פְּסָלָהּ׃ וְאִם חִסַּר אַחַת מִכׇּל־סַמָּנֶֽיהָ חַיַּב מִיתָה׃ רַבָּן שִׁמְעוֹן בֶּן גַּמְלִיאֵל אוֹמֵר׃ הַצֳּרִי אֵינוֹ אֶלָּא שְׂרָף הַנּוֹטֵף מֵעֲצֵי הַקְּטָף׃ בֹּרִית כַּרְשִׁינָה שֶׁשָּׁפִין בָּהּ אֶת־הַצִּפֹּֽרֶן כְּדֵי שֶׁתְּהֵא נָאָה׃ יֵין קַפְרִיסִין שֶׁשּׁוֹרִין בּוֹ אֶת־הַצִּפֹּֽרֶן כְּדֵי שֶׁתְּהֵא עַזָּה וַהֲלֹא מֵי רַגְלַֽיִם יָפִין לָהּ אֶלָּא שֶׁאֵין מַכְנִיסִין מֵי רַגְלַֽיִם בָּעֲזָרָה מִפְּנֵי הַכָּבוֹד׃
}

\newcommand{\AVHHN}{
	\begin{large}
	\textbf{אֲנִי וָהוֹ הוֹשִֽׁיעָה נָּא׃}
\end{large}

\begin{small}
	כְּהוֹשַֽׁעְתָּ אֵלִים בְּלוּד עִמָּךְ\hfill\break\hfill בְּצֵאתְךָ לְיֵֽשַׁע עַמָּךְ \hfill כֵּן הוֹשַׁע נָא׃ \\
	כְּהוֹשַֽׁעְתָּ גּוֹי וֵאלֹהִים\hfill\break\hfill דְּרוּשִׁים לְיֵֽשַׁע אֱלֹהִים \hfill כֵּן הוֹשַׁע נָא׃ \\
	כְּהוֹשַֽׁעְתָּ הֲמוֹן צְבָאוֹת\hfill\break\hfill וְעִמָּם מַלְאֲכֵי צְבָאוֹת \hfill כֵּן הוֹשַׁע נָא׃ \\
	כְּהוֹשַֽׁעְתָּ זַכִּים מִבֵּית עֲבָדִים\hfill\break\hfill חַנּוּן בְּיָדָם מַעֲבִידִים \hfill כֵּן הוֹשַׁע נָא׃ \\
	כְּהוֹשַֽׁעְתָּ טְבוּעִים בְּצוּל גְּזָרִים\hfill\break\hfill יְקָרְךָ עִמָּם מַעֲבִירִים \hfill כֵּן הוֹשַׁע נָא׃ \\
	כְּהוֹשַֽׁעְתָּ כַּנָּה מְשׁוֹרֶֽרֶת וַיּֽוֹשַׁע\hfill\break\hfill לְגוֹחָהּ מְצֻיֶּנֶת וַיִוָּֽשַׁע \hfill כֵּן הוֹשַׁע נָא׃ \\
	כְּהוֹשַֽׁעְתָּ מַאֲמַר וְהוֹצֵאתִי אֶתְכֶם\hfill\break\hfill נָקוּב וְהוּצֵאתִי אִתְּכֶם \hfill כֵּן הוֹשַׁע נָא׃\\
	כְּהוֹשַֽׁעְתָּ סוֹבְבֵי מִזְבֵּֽחַ\hfill\break\hfill עוֹמְסֵי עֲרָבָה לְהַקִּיף מִזְבֵּֽחַ \hfill כֵּן הוֹשַׁע נָא׃ \\
	כְּהוֹשַֽׁעְתָּ פִּלְאֵי אָרוֹן כְּהֻפְשַׁע\hfill\break\hfill צִעֵר פְּלֶֽשֶׁת בַּחֲרוֹן אַף וְנוֹשַׁע \hfill כֵּן הוֹשַׁע נָא׃\\
	כְּהוֹשַֽׁעְתָּ קְהִלּוֹת בָּבֶֽלָה שִׁלַּֽחְתָּ\hfill\break\hfill רַחוּם לְמַעֲנָם שֻׁלַּחְתָּ \hfill כֵּן הוֹשַׁע נָא׃\\
	כְּהוֹשַֽׁעְתָּ שְׁבוּת שִׁבְטֵי יַעֲקֹב\hfill\break\hfill תָּשׁוּב וְתָשִׁיב שְׁבוּת אׇהֳלֵי יַעֲקֹב \hfill וְהוֹשִׁיעָה נָּא׃\\
	כְּהוֹשַֽׁעְתָּ שׁ֗וֹמְרֵי מִ֗צְווֹת וְ֗חוֹכֵי יְשׁוּעוֹת\hfill\break\hfill אֵ֗ל֗ לְמוֹשָׁעוֹת \hfill וְהוֹשִׁיעָה נָּא׃
	
\end{small}

\begin{large}
	\textbf{אֲנִי וָהוֹ הוֹשִֽׁיעָה נָּא׃}
\end{large}
}

\newcommand{\havineinu}{
הֲבִינֵֽנוּ יְיָ אֱלֹהֵֽינוּ לָדַֽעַת דְּרָכֶיךָ. וּמוֹל אֶת־לְבָבֵֽנוּ לְיִרְאָתֶֽךָ. וְתִסְלַח לָֽנוּ לִהְיוֹת גְּאוּלִים. וְרַחֲקֵנוּ מִמַּכְאוֹב. וְדַשְּׁנֵֽנוּ בִּנְאוֹת אַרְצֶֽךָ. וּנְפוּצוֹתֵֽינוּ מֵאַרְבַּע כַּנְפוֹת הָאָֽרֶץ תְּקַבֵּץ. וְהַתּוֹעִים עַל דַּעְתְּךָ יִשָׁפֵֽטוּ. וְעַל הַרְשָׁעִים תָּנִיף יָדֶֽךָ. וְיִשְׂמְחוּ צַדִיקִים בְּבִנְיַן עִירֶֽךָ. וּבְתִקּוּן הֵיכָלֶֽךָ. וּבִצְמִֽיחַת קֶֽרֶן לְדָוִד עַבְדֶּֽךָ. וּבְעֲרִֽיכַת נֵר לְבֶן יִשַׁי מְשִׁיחֶֽךָ. טֶֽרֶם נִקְרָא אַתָּה תַעֲנֶה׃ בָּרוּךְ אַתָּה יְיָ שׁוֹמֵֽעַ תְּפִלָּה׃}

\newcommand{\shacharitinstruction}{\longenginst{The following begins the formal morning prayer. Barekhu is recited only in public prayer. The reader bends their knees and bows while saying the word \hebineng{ברכו} and straightens for the divine name, and the congregation does the same while saying \hebineng{ברוך}.}}

\newcommand{\maarivinst}{}

\newcommand{\AMamidainst}{}

\vspace*{\fill}

\setstretch{1.5}

\centerlast

\renewcommand{\thefootnote}{\roman{footnote}} % makes footnote lower-case Roman Numeral
\setlength{\parskip}{0.75em}

\newcommand{\halfline}{\vspace{0.5\baselineskip}}

\newtoggle{includeshabbat}
\newtoggle{includefestival}
\newtoggle{includeChM}
\newtoggle{includeweekday}
\newtoggle{includeRCh}
\newtoggle{includeAYT}
\togglefalse{includeshabbat}
\toggletrue{includeweekday}
\toggletrue{includeRCh}
\toggletrue{includeAYT}
\togglefalse{includefestival}
\togglefalse{includeChM}

\let\clearpage\relax{

\chapter[ברכות השחר]{\adforn{47} ברכות השחר \adforn{19}}
%\chapter[ברכות השחר Blessings Morning]{\adforn{47} Blessings Morning \adforn{19}\\ ברכות השחר }

\englishinst{Upon waking:}
%\instruction{כשמתעורר בבוקר׃}
\firstword{מוֹדֶה/מוֹדָה}
אֲנִי לְפָנֶיךָ מֶלֶךְ חַי וְקַיָּם \middot שֶׁהֶחֱזַרְתָּ בִּי נִשְׁמָתִי בְּחֶמְלָה \middot רַבָּה אֱמוּנָתֶךָ׃\\
\englishinst{On washing hands:}
\firstword{בָּרוּךְ}
אַתָּה יְיָ אֱלֹהֵֽינוּ מֶֽלֶךְ הָעוֹלָם \middot אֲשֶׁר קִדְּשָֽׁנוּ בְּמִצְוֹתָיו וְצִוָּֽנוּ עַל נְטִילַת יָדָֽיִם׃

\firstword{בָּרוּךְ}
אַתָּה יְיָ אֱלֹהֵֽינוּ מֶֽלֶךְ הָעוֹלָם אֲשֶׁר יָצַר אֶת־הָאָדָם בְּחׇכְמָה וּבָרָא בוֹ נְקָבִים נְקָבִים חֲלוּלִים חֲלוּלִים \middot גָּלוּי וְיָדֽוּעַ לִפְנֵי כִסֵּא כְבוֹדֶֽךָ שֶׁאִם יִפָּתֵֽחַ אֶחָד מֵהֶם אוֹ יִסָּתֵם אֶחָד מֵהֶם אִי אֶפְשַׁר לְהִתְקַיֵּם וְלַעֲמוֹד לְפָנֶֽיךָ׃ בָּרוּךְ אַתָּה יְיָ רוֹפֵא כׇל־בָּשָׂר וּמַפְלִיא לַעֲשׂוֹת׃


\firstword{אֱלֹהַי}
נְשָׁמָה שֶׁנָּתַֽתָּ בִּי טְהוֹרָה הִיא \middot אַתָּה בְרָאתָהּ אַתָּה יְצַרְתָּהּ אַתָּה נְפַחְתָּהּ בִּי וְאַתָּה מְשַׁמְּרָהּ בְּקִרְבִּי וְאַתָּה עָתִיד לִטְּלָהּ מִמֶּֽנִּי וּלְהַחֲזִירָהּ בִּי לֶעָתִיד לָבוֹא \middot כׇּל־זְמַן שֶׁהַנְּשָׁמָה בְּקִרְבִּי מוֹדֶה/מוֹדָה אֲנִי לְפָנֶֽיךָ יְיָ אֱלֹהַי וֵאלֹהֵי אֲבוֹתַי רִבּוֹן כׇּל־הַמַּעֲשִׂים אֲדוֹן כׇּל־הַנְּשָׁמוֹת׃ בָּרוּךְ אַתָּה יְיָ הַמַּחֲזִיר נְשָׁמוֹת לִפְגָרִים מֵתִים׃

\englishinst{One who does not wear a talli\thav\space gadol and wears a talli\thav\space katan blesses as follows before putting it on:}
%\instruction{מי שלא לובש טלית גדול מתלבש בטלית קטן ומברך׃}\\
\firstword{בָּרוּךְ}
אַתָּה יְיָ אֱלֹהֵֽינוּ מֶלֶךְ הָעוֹלָם אֲשֶׁר קִדְּשָׁנוּ בְּמִצְוֹתָיו וְצִוָּנוּ עַל מִצְוַת צִיצִית׃

\newcommand{\birkothatorah}{
	בָּרוּךְ אַתָּה יְיָ אֱלֹהֵֽינוּ מֶֽלֶךְ הָעוֹלָם אֲשֶׁר קִדְּשָֽׁנוּ בְּמִצְוֹתָיו וְצִוָּֽנוּ לַעֲסוֹק בְּדִבְרֵי תוֹרָה׃ וְהַעֲרֶב־נָא יְיָ אֱלֹהֵֽינוּ אֶת־דִּבְרֵי תוֹרָתְךָ בְּפִֽינוּ וּבְפִיפִיּוֹת עַמְּךָ בֵּית יִשְׂרָאֵל \middot וְנִהְיֶה אֲנַֽחְנוּ וְצֶאֱצָאֵֽינוּ וְצֶאֱצָאֵי עַמְּךָ בֵּית יִשְׂרָאֵל כֻּלָּֽנוּ יוֹדְעֵי שְׁמֶֽךָ וְלוֹמְדֵי תוֹרָתֶֽךָ לִשְׁמָהּ׃ בָּרוּךְ אַתָּה יְיָ הַמְלַמֵּד תּוֹרָה לְעַמּוֹ יִשְׂרָאֵל׃
	
	בָּרוּךְ אַתָּה יְיָ אֱלֹהֵֽינוּ מֶֽלֶךְ הָעוֹלָם אֲשֶׁר בָּֽחַר־בָּֽנוּ מִכׇּל־הָעַמִּים וְנָֽתַן־לָֽנוּ אֶת־תּוֹרָתוֹ׃ בָּרוּךְ אַתָּה יְיָ נוֹתֵן הַתּוֹרָה׃\\}

\englishinst{Some postpone recitation of the blessings on the Torah below and recite them before the sacrificial readings on page \pageref{korbanos}.}
%\instruction{יש אומרים ברכות התורה לפני אמירת הקרבנות}
\birkothatorah
יְבָֽרֶכְךָ֥ יְיָ֖ וְיִשְׁמְרֶֽךָ׃ \source{במדבר ו}יָאֵ֨ר יְיָ֧ פָּנָ֛יו אֵלֶ֖יךָ וִֽיחֻנֶּֽךָּ׃ יִשָּׂ֨א יְיָ֤ פָּנָיו֙ אֵלֶ֔יךָ וְיָשֵׂ֥ם לְךָ֖ שָׁלֽוֹם׃\\
אֵֽלּוּ דְבָרִים שֶׁאֵין לָהֶם שִׁעוּר׃ \source{משנה פאה א}הַפֵּאָה וְהַבִּכּוּרִים וְהָרֵאָיוֹן וּגְמִילוּת חֲסָדִים וְתַלְמוּד תּוֹרָה׃\\
אֵֽלּוּ \source{שבת קכז}דְבָרִים שֶׁאָדָם אוֹכֵל פֵּירוֹתֵיהֶם בָּעוֹלָם הַזֶּה וְהַקֶּֽרֶן קַיֶּֽמֶת לוֹ לָעוֹלָם הַבָּא׃ וְאֵֽלּוּ הֵן - כִּבּוּד אָב וָאֵם וּגְמִילוּת חֲסָדִים וְהַשְׁכָּמַת בֵּית הַמִּדְרָשׁ שַׁחֲרִית וְעַרְבִית וְהַכְנָסַת אוֹרְחִים וּבִקּוּר חוֹלִים וְהַכְנָסַת כַּלָּה וְהַלְוָיַת הַמֵּת וְעִיּוּן תְּפִלָּה וַהֲבָאַת שָׁלוֹם בֵּין אָדָם לַחֲבֵרוֹ - וְתַלְמוּד תּוֹרָה כְּנֶֽגֶד כֻּלָּם׃

\englishinst{Stand for recitation of the following blessings:}
\firstword{בָּרוּךְ}
אַתָּה יְיָ אֱלֹהֵֽינוּ מֶֽלֶךְ הָעוֹלָם אֲשֶׁר נָתַן לַשֶּֽׂכְוִי בִינָה לְהַבְחִין בֵּין יוֹם וּבֵין לָֽיְלָה׃\hfill \break
%\firstword{בָּרוּךְ}
%אַתָּה יְיָ אֱלֹהֵֽינוּ מֶֽלֶךְ הָעוֹלָם שֶׁעָשַֽׂנִי יִשְׂרָאֵל׃\hfill\break
\firstword{בָּרוּךְ}
אַתָּה יְיָ אֱלֹהֵֽינוּ מֶֽלֶךְ הָעוֹלָם...\hfill\break
%\begin{small}
	\begin{tabular}{>{\centering\arraybackslash}m{.45\textwidth} | >{\centering\arraybackslash}m{.45\textwidth}}
		
		\instruction{גברים׃} & \instruction{נשים׃}\\% & \instruction{נוסח שויוני׃}\\
		שֶׁלֹּא עָשַֽׂנִי גּוֹי׃
		&
		שֶׁלֹּא עָשַֽׂנִי גּוֹיָה׃
		%& 
		%שֶׁעָשַֽׂנִי בְּצַלְמוֹ׃
		\\
		
		שֶׁלֹּא עָשַׂנִי עְָבֶד׃
		&
		שֶׁלֹּא עָשַׂנִי שִׁפְחָה׃
	%	&
		%\begin{large}שֶׁעָשַֽׂנִי יִשְׂרָאֵל׃\end{large}
		\\
		
		שֶׁלֹּא עָשַֽׂנִי אִשָּׁה׃
		&
		שֶׁעָשַֽׂנִי כִּרְצוֹנוֹ׃
	%	&
		%שֶׁעָשַֽׂנִי בֶּן־/בַּת־חוֹרִין׃
	\end{tabular}
%\end{small}
%(\instruction{יש אומרים:}
%\firstword{בָּרוּךְ}
%אַתָּה יְיָ אֱלֹהֵֽינוּ מֶֽלֶךְ הָעוֹלָם מַגְבִּֽיהַּ שְׁפָלִים׃)\hfill \break
\firstword{בָּרוּךְ}
אַתָּה יְיָ אֱלֹהֵֽינוּ מֶֽלֶךְ הָעוֹלָם פּוֹקֵֽחַ עִוְרִים׃\hfill \break
\firstword{בָּרוּךְ}
אַתָּה יְיָ אֱלֹהֵֽינוּ מֶֽלֶךְ הָעוֹלָם מַלְבִּישׁ עַרֻמִּים׃\hfill \break
\firstword{בָּרוּךְ}
אַתָּה יְיָ אֱלֹהֵֽינוּ מֶֽלֶךְ הָעוֹלָם מַתִּיר אֲסוּרִים׃\hfill \break
\firstword{בָּרוּךְ}
אַתָּה יְיָ אֱלֹהֵֽינוּ מֶֽלֶךְ הָעוֹלָם זוֹקֵף כְּפוּפִים׃\hfill \break
\firstword{בָּרוּךְ}
אַתָּה יְיָ אֱלֹהֵֽינוּ מֶֽלֶךְ הָעוֹלָם רוֹקַע הָאָֽרֶץ עַל־הַמָּֽיִם׃\hfill \break
\firstword{בָּרוּךְ}
אַתָּה יְיָ אֱלֹהֵֽינוּ מֶֽלֶךְ הָעוֹלָם הַמֵּכִין מִצְעֲדֵי־גָֽבֶר׃\hfill \break
\firstword{בָּרוּךְ}
אַתָּה יְיָ אֱלֹהֵֽינוּ מֶֽלֶךְ הָעוֹלָם שֶׁעָֽשָׂה לִי כׇּל־צׇרְכִּי׃\hfill \break
\firstword{בָּרוּךְ}
אַתָּה יְיָ אֱלֹהֵֽינוּ מֶֽלֶךְ הָעוֹלָם אוֹזֵר יִשְׂרָאֵל בִּגְבוּרָה׃\hfill \break
\firstword{בָּרוּךְ}
אַתָּה יְיָ אֱלֹהֵֽינוּ מֶֽלֶךְ הָעוֹלָם עוֹטֵר יִשְׂרָאֵל בְּתִפְאָרָה׃\hfill \break
\firstword{בָּרוּךְ}
אַתָּה יְיָ אֱלֹהֵֽינוּ מֶֽלֶךְ הָעוֹלָם הַנּוֹתֵן לַיָּעֵף כֹּֽחַ׃\hfill

\firstword{בָּרוּךְ}
אַתָּה יְיָ אֱלֹהֵֽינוּ מֶֽלֶךְ הָעוֹלָם הַמַּעֲבִיר שֵׁנָה מֵעֵינָי וּתְנוּמָה מֵעַפְעַפָּי \middot וִיהִי רָצוֹן מִלְּפָנֶֽיךָ יְיָ אֱלֹהֵֽינוּ וֵאלֹהֵי אֲבוֹתֵֽינוּ שֶׁתַּרְגִּילֵֽנוּ בְּתוֹרָתֶֽךָ וְתַדְבִּיקֵֽנוּ בְּמִצְוׂתֶֽיךָ וְאַל תְּבִיאֵֽנוּ לֹא לִידֵי חֵטְא וְלֹא לִידֵי עֲבֵרָה וְעָוׂן וְלֹא לִידֵי נִסָּיוֹן וְלֹא לִידֵי בִזָּיוֹן וְאַל תַּשְׁלֶט־בָּנוּ יֵֽצֶר הָרָע וְהַרְחִיקֵֽנוּ מֵאָדָם רָע וּמֵחָבֵר רָע וְדַבְּקֵֽנוּ בְּיֵֽצֶר טוֹב וּבְמַעֲשִׂים טוֹבִים וְכֹף אֶת־יִצְרֵֽנוּ לְהִשְׁתַּעְבֶּד־לָךְ \middot וּתְנֵנוּ הַיּוֹם וּבְכׇל־יוֹם לְחֵן וּלְחֶסֶד וּלְרַחֲמִים בְּעֵינֶיךָ וּבְעֵינֵי כׇל־רוֹאֵינוּ וְתִגְמְלֵנוּ חֲסָדִים טוֹבִים׃ בָּרוּךְ אַתָּה יְיָ גּוֹמֵל חֲסָדִים טוֹבִים לְעַמּוֹ יִשְׂרָאֵל׃\\
יְהִי רָצוֹן מִלְּפָנֶֽיךָ יְיָ אֱלֹהַי וֵאלֹהֵי אֲבוֹתַי שֶׁתַּצִּילֵֽנִי הַיּוֹם וּבְכׇל־יוֹם מֵעַזֵּי פָנִים וּמֵעַזּוּת פָּנִים מֵאָדָם רַע וּמֵחָבֵר רַע וּמִשָּׁכֵן רַע וּמִפֶּֽגַע רַע וּמִשָּׂטָן הַמַּשְׁחִית מִדִּין קָשֶׁה וּמִבַּֽעַל דִּין קָשֶׁה בֵּין שְׁהוּא בֶן־בְּרִית וּבֵין שֶׁאֵינוֹ בֶן־בְּרִית׃

%\ssubsection{\adforn{18} קבלת עול מלכות שמים \adforn{17}}\\
\instruction{לְעוֹלָם יְהֵא אָדָם יְרֵא שָׁמַיִם בַּסֵּתֶר וּבַגָּלוּי וּמוֹדֶה עַל הָאֱמֶת וְדוֹבֵר אֱמֶת בִּלְבָבוֹ וְיַשְׁכֵּם וְיֹאמַר׃}

\firstword{רִבּוֹן}
כׇּל־הָעוֹלָמִים לֹא עַל־צִדְקוֹתֵֽינוּ אֲנַֽחְנוּ מַפִּילִים תַּחֲנוּנֵֽינוּ לְפָנֶֽיךָ כִּי עַל רַחֲמֶֽיךָ הָרַבִּים \middot מָה אֲנַחְנוּ מֶה חַיֵּֽינוּ מֶה חַסְדֵּֽנוּ מַה־צִּדְקֵֽנוּ מַה־יְּשׁוּעָתֵֽנוּ מַה־כֹּחֵֽנוּ מַה־גְּבוּרָתֵֽנוּ \middot מַה־נֹּאמַר לְפָנֶֽיךָ יְיָ אֱלֹהֵֽינוּ וֵאלֹהֵי אֲבוֹתֵֽינוּ הֲלֹא כׇל־הַגִּבּוֹרִים כְּאַֽיִן לְפָנֶֽיךָ וְאַנְשֵׁי הַשֵּׁם כְּלֹא הָיוּ וַחֲכָמִים כִּבְלִי מַדָּע וּנְבוֹנִים כִּבְלִי הַשְׂכֵּל \middot כִּי רֹב מַעֲשֵׂיהֶם תֹּֽהוּ וִימֵי חַיֵּיהֶם הֶֽבֶל לְפָנֶֽיךָ
וּמוֹתַ֨ר \source{קהלת ג}הָֽאָדָ֤ם מִן־הַבְּהֵמָה֙ אָֽ֔יִן כִּ֥י הַכֹּ֖ל הָֽבֶל׃ \\
\firstword{אֲבָל}
אֲנַֽחְנוּ עַמְּךָ בְּנֵי בְרִיתֶֽךָ בְּנֵי אַבְרָהָם אֹהַבְךָ שֶׁנִּשְׁבַּֽעְתָּ לּוֹ בְּהַר הַמֹּרִיָּה זֶֽרַע יִצְחָק יְחִידוֹ שֶׁנֶּעֱקַד עַל גַּבֵּי הַמִּזְבֵּֽחַ עֲדַת יַעֲקֹב בִּנְךָ בְּכוֹרֶֽךָ שֶׁמֵּאַהֲבָתְךָ שֶׁאָהַֽבְתָּ אֹתוֹ וּמִשִּׂמְחָתְךָ שֶׁשָּׂמַֽחְתָּ בּוֹ קָרָֽאתָ אֶת־שְׁמוֹ יִשְׂרָאֵל וִישֻׁרוּן׃ \\
\firstword{לְפִיכָךְ}
אֲנַֽחְנוּ חַיָּבִים לְהוֹדוֹת לְךָ וּלְשַׁבֵּחֲךָ וּלְפָאֶרְךָ וּלְבָרֵךְ וּלְקַדֵּשׁ וְלִתֵּן־שֶֽׁבַח וְהוֹדָיָה לִשְׁמֶֽךָ \middot אַשְׁרֵֽינוּ מַה־טּוֹב חֶלְקֵֽנוּ וּמַה־נָּעִים גּוֹרָלֵֽינוּ וּמַה־יָּפָה יְרֻשָּׁתֵֽינוּ \middot אַשְׁרֵֽינוּ שֶׁאֲנַֽחְנוּ מַשְׁכִּימִים וּמַעֲרִיבִים עֶֽרֶב וָבֹֽקֶר וְאוֹמְרִים פַּעֲמַֽיִם בְּכׇל־יוֹם׃

\englishinst{If the communal recitation of morning Shema\ayin\space will be after its time, recite the full Shema\ayin\space on page \pageref{morningshema}.}
%\instruction{אם השעה מאוחר קוראים כאן ק״ש בעמ׳ \pageref{morningshema}}
\begin{Large}
	\textbf{שְׁמַ֖ע יִשְׂרָאֵ֑ל יְיָ֥ אֱלֹהֵ֖ינוּ יְיָ֥ ׀ אֶחָֽד׃}
\end{Large}\source{דברים ו}

\instruction{בלחש׃}
\textbf{%
	בָּרוּךְ שֵׁם כְּבוֹד מַלְכוּתוֹ לְעוֹלָם וָעֶד׃
}


\firstword{אַתָּה}
הוּא עַד שֶׁלֹּא נִבְרָא הָעוֹלָם אַתָּה הוּא מִשֶּׁנִּבְרָא הָעוֹלָם \middot אַתָּה הוּא בָּעוֹלָם הַזֶּה וְאַתָּה הוּא לָעוֹלָם הַבָּא \middot קַדֵּשׁ אֶת־שִׁמְךָ עַל מַקְדִּישֵׁי שְׁמֶֽךָ וְקַדֵּשׁ אֶת־שִׁמְךָ בְּעוֹלָמֶֽךָ וּבִישׁוּעָתְךָ תָּרוּם וְתַגְבִּֽיהַּ קַרְנֵֽנוּ׃ בָּרוּךְ אַתָּה יְיָ מְקַדֵּשׁ אֶת־שִׁמְךָ בָּרַבִּים׃

\firstword{אַתָּה}
הוּא יְיָ אֱלֹהֵֽינוּ בַּשָּׁמַֽיִם וּבָאָֽרֶץ וּבִשְׁמֵי הַשָּׁמַֽיִם הָעֶלְיוֹנִים׃ אֱמֶת אַתָּה הוּא רִאשׁוֹן וְאַתָּה הוּא אַחֲרוֹן וּמִבַּלְעָדֶֽיךָ אֵין אֱלֹהִים׃ קַבֵּץ קֹוֶֽיךָ מֵאַרְבַּע כַּנְפוֹת הָאָֽרֶץ׃ יַכִּֽירוּ וְיֵדְעוּ כׇּל־בָּאֵי עוֹלָם
כִּי \source{מל״ב יט}אַתָּה־ה֤וּא הָאֱלֹהִים֙ לְבַדְּךָ֔ לְכֹ֖ל מַמְלְכ֣וֹת הָאָ֑רֶץ אַתָּ֣ה עָשִׂ֔יתָ אֶת־הַשָּׁמַ֖יִם וְאֶת־הָאָֽרֶץ׃ \source{שמות כ}אֶת־הַיָּם וְאֶת־כׇּל־אֲשֶׁר־בָּם׃
וּמִי בְּכׇל־מַעֲשֵׂה יָדֶֽיךָ בָּעֶלְיוֹנִים אוֹ בַתַּחְתּוֹנִים שֶׁיֹּאמַר לְךָ מַה תַּעֲשֶׂה׃\\
אָבִֽינוּ שֶׁבַּשָּׁמַֽיִם עֲשֵׂה עִמָּֽנוּ חֶֽסֶד בַּעֲבוּר שִׁמְךָ הַגָּדוֹל שֶׁנִּקְרָא עָלֵֽינוּ וְקַיֶּם לָֽנוּ יְיָ אֱלֹהֵֽינוּ מַה־שֶׁכָּתוּב׃ \source{צפניה ד}%
בָּעֵ֤ת הַהִיא֙ אָבִ֣יא אֶתְכֶ֔ם וּבָעֵ֖ת קַבְּצִ֣י אֶתְכֶ֑ם כִּֽי־אֶתֵּ֨ן אֶתְכֶ֜ם לְשֵׁ֣ם וְלִתְהִלָּ֗ה בְּכֹל֙ עַמֵּ֣י הָאָֽ֔רֶץ בְּשׁוּבִ֧י אֶת־שְׁבֽוּתֵיכֶ֛ם לְעֵֽינֵיכֶ֖ם אָמַ֥ר יְיָ׃

%\section[קרבנות Sacrifices]{\adforn{18} קרבנות Sacrifices \adforn{17}}
\section[קרבנות]{\adforn{18} קרבנות \adforn{17}}

%\source{שמות ל}
%וַיְדַבֵּ֥ר יְיָ֖ אֶל־מֹשֶׁ֥ה לֵּאמֹֽר׃ וְעָשִׂ֜יתָ כִּיּ֥וֹר נְחֹ֛שֶׁת וְכַנּ֥וֹ נְחֹ֖שֶׁת לְרׇחְצָ֑ה וְנָתַתָּ֣ אֹת֗וֹ בֵּֽין־אֹ֤הֶל מוֹעֵד֙ וּבֵ֣ין הַמִּזְבֵּ֔חַ וְנָתַתָּ֥ שָׁ֖מָּה מָֽיִם׃ וְרָחֲצ֛וּ אַהֲרֹ֥ן וּבָנָ֖יו מִמֶּ֑נּוּ אֶת־יְדֵיהֶ֖ם וְאֶת־רַגְלֵיהֶֽם׃ בְּבֹאָ֞ם אֶל־אֹ֧הֶל מוֹעֵ֛ד יִרְחֲצוּ־מַ֖יִם וְלֹ֣א יָמֻ֑תוּ א֣וֹ בְגִשְׁתָּ֤ם אֶל־הַמִּזְבֵּ֙חַ֙ לְשָׁרֵ֔ת לְהַקְטִ֥יר אִשֶּׁ֖ה לַֽייָ׃ וְרָחֲצ֛וּ יְדֵיהֶ֥ם וְרַגְלֵיהֶ֖ם וְלֹ֣א יָמֻ֑תוּ וְהָיְתָ֨ה לָהֶ֧ם חׇק־עוֹלָ֛ם ל֥וֹ וּלְזַרְע֖וֹ לְדֹרֹתָֽם׃
%\source{ויקרא ו}
%וַיְדַבֵּ֥ר יְיָ֖ אֶל־מֹשֶׁ֥ה לֵּאמֹֽר׃ צַ֤ו אֶֽת־אַהֲרֹן֙ וְאֶת־בָּנָ֣יו לֵאמֹ֔ר זֹ֥את תּוֹרַ֖ת הָעֹלָ֑ה הִ֣וא הָעֹלָ֡ה עַל֩ מוֹקְדָ֨ה עַל־הַמִּזְבֵּ֤חַ כׇּל־הַלַּ֙יְלָה֙ עַד־הַבֹּ֔קֶר וְאֵ֥שׁ הַמִּזְבֵּ֖חַ תּ֥וּקַד בּֽוֹ׃ וְלָבַ֨שׁ הַכֹּהֵ֜ן מִדּ֣וֹ בַ֗ד וּמִֽכְנְסֵי־בַד֮ יִלְבַּ֣שׁ עַל־בְּשָׂרוֹ֒ וְהֵרִ֣ים אֶת־הַדֶּ֗שֶׁן אֲשֶׁ֨ר תֹּאכַ֥ל הָאֵ֛שׁ אֶת־הָעֹלָ֖ה עַל־הַמִּזְבֵּ֑חַ וְשָׂמ֕וֹ אֵ֖צֶל הַמִּזְבֵּֽחַ׃ וּפָשַׁט֙ אֶת־בְּגָדָ֔יו וְלָבַ֖שׁ בְּגָדִ֣ים אֲחֵרִ֑ים וְהוֹצִ֤יא אֶת־הַדֶּ֙שֶׁן֙ אֶל־מִח֣וּץ לַֽמַּחֲנֶ֔ה אֶל־מָק֖וֹם טָהֽוֹר׃ וְהָאֵ֨שׁ עַל־הַמִּזְבֵּ֤חַ תּֽוּקַד־בּוֹ֙ לֹ֣א תִכְבֶּ֔ה וּבִעֵ֨ר עָלֶ֧יהָ הַכֹּהֵ֛ן עֵצִ֖ים בַּבֹּ֣קֶר בַּבֹּ֑קֶר וְעָרַ֤ךְ עָלֶ֙יהָ֙ הָֽעֹלָ֔ה וְהִקְטִ֥יר עָלֶ֖יהָ חֶלְבֵ֥י הַשְּׁלָמִֽים׃ אֵ֗שׁ תָּמִ֛יד תּוּקַ֥ד עַל־הַמִּזְבֵּ֖חַ לֹ֥א תִכְבֶּֽה׃

%\footnote{
	\englishinst{If the blessings on the Torah were not said previously, they are recited now.}%\instruction{אם לא בירך ברכות התורה, חייב לברך קודם אמירת הקרבנות }
{\footnotesize \birkothatorah}
%}
\label{korbanos}
\tamid

%\ketoret
%
%\pitumhaketoret
%תַּנְיָא, רַבִּי נָתָן אוֹמֵר: כְּשֶׁהוּא שׁוֹחֵק, אוֹמֵר הָדֵק הֵיטֵב, הֵיטֵב הָדֵק, מִפְּנֵי שֶׁהַקּוֹל יָפֶה לַבְּשָׂמִים. פִּטְּמָהּ לַחֲצָאִין, כְּשֵׁרָה; לִשְׁלִישׁ וְלִרְבִֽיעַ, לֹא שָׁמַֽעְנוּ. אָמַר רַבִּי יְהוּדָה: זֶה הַכְּלָל: אִם כְּמִדָּתָהּ, כְּשֵׁרָה לַחֲצָאִין; וְאִם חִסַּר אַחַת מִכׇּל־סַמָּנֶֽיהָ, חַיָּב מִיתָה׃

%תַּנְיָא, בַּר קַפָּרָא אוֹמֵר: אַחַת לְשִׁשִּׁים אוֹ לְשִׁבְעִים שָׁנָה הָיְתָה בָאָה שֶׁל שִׁירַֽיִם לַחֲצָאִין. וְעוֹד תָּנֵי בַּר קַפָּרָא: אִלּוּ הָיָה נוֹתֵן בָּהּ קֹרְטוֹב שֶׁל דְּבַשׁ, אֵין אָדָם יָכוֹל לַעֲמֹד מִפְּנֵי רֵיחָהּ; וְלָמָה אֵין מְעָרְבִין בָּהּ דְּבַשׁ, מִפְּנֵי שֶׁהַתּוֹרָה אָמְרָה: כִּ֤י כׇל־שְׂאֹר֙ וְכׇל־דְּבַ֔שׁ לֹֽא־תַקְטִ֧ירוּ מִמֶּ֛נּוּ אִשֶּׁ֖ה לַֽייָֽ׃

%\firstword{אַבַּיֵּי הֲוָה מְסַדֵּר סֵדֶר הַמַּעֲרָכָה} \source{יומא לג} מִשְּׁמָא דִגְמָרָא, וְאַלִּבָּא דְאַבָּא שָׁאוּל, מַעֲרָכָה גְדוֹלָה קוֹדֶמֶת לְמַעֲרָכָה שְׁנִיָּה שֶׁל קְטֹרֶת, וּמַעֲרָכָה שְׁנִיָּה שֶׁל קְטֹרֶת קוֹדֶמֶת לְסִדּוּר שְׁנֵי גִזְרֵי עֵצִים, וְסִדּוּר שְׁנֵי גִזְרֵי עֵצִים קוֹדֶם לְדִשּׁוּן מִזְבֵּחַ הַפְּנִימִי, וְדִשּׁוּן מִזְבֵּחַ הַפְּנִימִי קוֹדֶם לְהַטָבַת חָמֵשׁ נֵרוֹת, וְהַטָבַת חָמֵשׁ נֵרוֹת קוֹדֶמֶת לְדַם הַתָּמִיד, וְדַם הַתָּמִיד קוֹדֶם לְהַטָבַת שְׁתֵּי נֵרוֹת, וְהַטָבַת שְׁתֵּי נֵרוֹת קוֹדֶמֶת לִקְטֹרֶת, וּקְטֹרֶת קוֹדֶמֶת לְאֵבָרִים, וְאֵבָרִים לְמִנְחָה, וּמִנְחָה לַחֲבִתִּין, וַחֲבִתִּין לִנְסָכִין, וּנְסָכִין לְמוּסָפִין, וּמוּסָפִין לְבָזִיכִין, וּבָזִיכִין קוֹדְמִין לְתָמִיד שֶׁל בֵּין הָעַרְבָּיִם. שֶׁנֶּאֱמַר, וְעָרַ֤ךְ \source{ויקרא ו} עָלֶ֙יהָ֙ הָֽעֹלָ֔ה וְהִקְטִ֥יר עָלֶ֖יהָ חֶלְבֵ֥י הַשְּׁלָמִֽים׃ עָלֶיהָ הַשְׁלֵם כׇּל־הַקׇּרְבָּנוֹת כֻּלָּם׃

%\firstword{רִבּוֹן הָעוֹלָמִים,} אַתָּה צִוִּיתָֽנוּ לְהַקְרִיב קׇרְבַּן הַתָּמִיד בְּמוֹעֲדוֹ, וְלִהְיוֹת כֹּהֲנִים בַּעֲבוֹדָתָם, וּלְוִיִּם בְּדוּכָנָם, וְיִשְׂרָאֵל בְּמַעֲמָדָם; וְעַתָּה בַּעֲוֺנוֹתֵֽינוּ חָרֵב בֵּית הַמִּקְדָּשׁ וּבָטֵל הַתָּמִיד, וְאֵין לָֽנוּ לֹא כֹהֵן בַּעֲבוֹדָתוֹ, וְלֹא לֵוִי בְּדוּכָנוֹ, וְלֹא יִשְׂרָאֵל בְּמַעֲמָדוֹ. וְאַתָּה אָמַרְתָּ׃ וּֽנְשַׁלְּמָ֥ה \source{הושע יד} פָרִ֖ים שְׂפָתֵֽינוּ׃ לָכֵן יְהִי רָצוֹן מִלְּפָנֶֽיךָ, יְיָ אֱלֹהֵֽינוּ וֵאלֹהֵי אֲבוֹתֵֽינוּ, שֶׁיְּהֵא שִֽׂיחַ שִׂפְתוֹתֵֽינוּ חָשׁוּב וּמְקֻבָּל וּמְרֻצֶּה לְפָנֶֽיךָ כְּאִלּוּ הִקְרַֽבְנוּ קׇרְבַּן הַתָּמִיד בְּמוֹעֲדוֹ וְעָמַֽדְנוּ עַל מַעֲמָדוֹ.\\
\ifboolexpr{togl {includeshabbat} and (togl {includeweekday} or togl {includefestival})}{\shabbos
}{}
\ifboolexpr{togl {includeshabbat} or togl {includefestival}}{\shabmusafpesukim}{}

\ifboolexpr{togl {includeRCh}}{
\instruction{בראש חודש׃}
\firstword{וּבְרָאשֵׁי֙ חׇדְשֵׁיכֶ֔ם}\source{במדבר כח}
תַּקְרִ֥יבוּ עֹלָ֖ה לַייָ֑ פָּרִ֨ים בְּנֵֽי־בָקָ֤ר שְׁנַ֙יִם֙ וְאַ֣יִל אֶחָ֔ד כְּבָשִׂ֧ים בְּנֵי־שָׁנָ֛ה שִׁבְעָ֖ה תְּמִימִֽם׃
וּשְׁלֹשָׁ֣ה עֶשְׂרֹנִ֗ים סֹ֤לֶת מִנְחָה֙ בְּלוּלָ֣ה בַשֶּׁ֔מֶן לַפָּ֖ר הָאֶחָ֑ד וּשְׁנֵ֣י עֶשְׂרֹנִ֗ים סֹ֤לֶת מִנְחָה֙ בְּלוּלָ֣ה בַשֶּׁ֔מֶן לָאַ֖יִל הָֽאֶחָֽד׃
וְעִשָּׂרֹ֣ן עִשָּׂר֗וֹן סֹ֤לֶת מִנְחָה֙ בְּלוּלָ֣ה בַשֶּׁ֔מֶן לַכֶּ֖בֶשׂ הָאֶחָ֑ד עֹלָה֙ רֵ֣יחַ נִיחֹ֔חַ אִשֶּׁ֖ה לַייָ׃
וְנִסְכֵּיהֶ֗ם חֲצִ֣י הַהִין֩ יִהְיֶ֨ה לַפָּ֜ר וּשְׁלִישִׁ֧ת הַהִ֣ין לָאַ֗יִל וּרְבִיעִ֥ת הַהִ֛ין לַכֶּ֖בֶשׂ יָ֑יִן זֹ֣את עֹלַ֥ת חֹ֙דֶשׁ֙ בְּחׇדְשׁ֔וֹ לְחׇדְשֵׁ֖י הַשָּׁנָֽה׃
וּשְׂעִ֨יר עִזִּ֥ים אֶחָ֛ד לְחַטָּ֖את לַייָ֑ עַל־עֹלַ֧ת הַתָּמִ֛יד יֵעָשֶׂ֖ה וְנִסְכּֽוֹ׃}{}

\firstword{(א) אֵיזֶהוּ מְקוֹמָן}
\source{זבחים פ״ה}%
שֶׁל זְבָחִים? קׇדְשֵׁי קׇדָשִׁים - שְׁחִיטָתָן בַּצָּפוֹן׃ פָּר וְשָׂעִיר שֶׁל יוֹם הַכִּפּוּרִים - שְׁחִיטָתָן בַּצָּפוֹן \middot וְקִבּוּל דָּמָן בִּכְלֵי שָׁרֵת בַּצָּפוֹן \middot וְדָמָן טָעוּן הַזָּיָה עַל־בֵּין הַבַּדִּים וְעַל הַפָּרֹֽכֶת וְעַל־מִזְבַּח הַזָּהָב׃ מַתָּנָה אַֽחַת מֵהֶן מְעַכֶּֽבֶת׃ שְׁיָרֵי הַדָּם הָיָה שׁוֹפֵךְ עַל יְסוֹד מַעֲרָבִי שֶׁל מִזְבֵּחַ הַחִיצוֹן׃ אִם לֹא נָתַן - לֹא עִכֵּב׃

(ב) פָּרִים הַנִּשְׂרָפִים וּשְׂעִירִים הַנִּשְׂרָפִים - שְׁחִיטָתָן בַּצָּפוֹן \middot וְקִבּוּל דָּמָן בִּכְלִי שָׁרֵת בַּצָּפוֹן \middot וְדָמָן טָעוּן הַזָּיָה עַל־הַפָּרֹֽכֶת וְעַל־מִזְבַּח הַזָּהָב׃ מַתָּנָה אַֽחַת מֵהֶן מְעַכֶּֽבֶת׃ שִׁיְרֵי הַדָּם הָיָה שׁוֹפֵךְ עַל יְסוֹד מַעֲרָבִי שֶׁל מִזְבֵּחַ הַחִיצוֹן אִם לֹא נָתַן לֹא עִכֵּב׃ אֵֽלּוּ וָאֵֽלּוּ נִשְׂרָפִין בְּבֵית הַדֶּֽשֶׁן׃

(ג) חַטֹּאת הַצִּבּוּר וְהַיָּחִיד - אֵֽלּוּ הֵן חַטֹּאת הַצִּבּוּר׃ שְׂעִירֵי רָאשֵׁי חֳדָשִׁים וְשֶׁל מוׁעֲדוׂת - שְׁחִיטָתָן בַּצָּפוֹן \middot וְקִבּוּל דָּמָן בִּכְלִי שָׁרֵת בַּצָּפוֹן \middot וְדָמָן טָעוּן אַרְבַּע מַתָּנוֹת עַל אַרְבַּע קְרָנוֹת׃ כֵּיצַד? עָלָה בַכֶּֽבֶשׁ \middot וּפָנָה לַסּוֹבֵב \middot וּבָא לוֹ לְקֶֽרֶן דְּרוֹמִית מִזְרָחִית, מִזְרָחִית צְפוֹנִית, צְפוֹנִית מַעֲרָבִית, מַעֲרָבִית דְּרוֹמִית׃ שִׁיְּרֵי הַדָּם הָיָה שֹׁפֵךְ עַל יְסוֹד דְּרוֹמִי׃ וְנֶאֱכָלִין לִפְנִים מִן־הַקְּלָעִים לְזִכְרֵי כְהֻנָּה בְּכׇל־מַאֲכָל לְיוֹם וָלַֽיְלָה עַד חֲצוֹת׃

(ד) הָעוֹלָה - קֹֽדֶשׁ קׇדָשִׁים - שְׁחִיטָתָהּ בַּצָּפוֹן \middot וְקִבּוּל דָּמָהּ בִּכְלִי שָׁרֵת בַּצָּפוֹן \middot וְדָמָהּ טָעוּן שְׁתֵּי מַתָּנוֹת שֶׁהֵן אַרְבַּע \middot וּטְעוּנָה הֶפְשֵׁט וְנִתּֽוּחַ וְכָלִיל לָאִשִּׁים׃

(ה) זִבְחֵי שַׁלְמֵי צִבּוּר וַאֲשָׁמוֹת - אֵֽלוּ הֵן אֲשָׁמוֹת׃ אֲשַׁם גְּזֵלוֹת, אֲשַׁם מְעִילוֹת, אֲשַׁם שִׁפְחָה חֲרוּפָה, אֲשַׁם נָזִיר, אֲשַׁם מְצוֹרָע, אָשָׁם תָּלוּי - שְׁחִיטָתָן בַּצָּפוֹן \middot וְקִבּוּל דָּמָן בִּכְלִי שָׁרֵת בַּצָּפוֹן \middot וְדָמָן טָעוּן שְׁתֵּי מַתָּנוֹת שֶׁהֵן אַרְבַּע \middot וְנֶאֱכָלִין לִפְנִים מִן הַקְּלָעִים לְזִכְרֵי כְהֻנָּה בְּכׇל־מַאֲכָל לְיוֹם וָלַֽיְלָה עַד חֲצוֹת׃

(ו) הַתּוֹדָה וְאֵיל נָזִיר - קׇדָשִׁים קַלִּים - שְׁחִיטָתָן בְּכׇל־מָקוֹם בָּעֲזָרָה \middot וְדָמָן טָעוּן שְׁתֵּי מַתָּנוֹת שֶׁהֵן אַרְבַּע \middot וְנֶאֱכָלִין בְּכׇל־הָעִיר לְכׇל־אָדָם בְּכׇל־מַאֲכָל לְיוֹם וָלַֽיְלָה עַד חֲצוֹת׃ הַמּוּרָם מֵהֶם - כַּיּוֹצֵא בָהֶם \middot אֶלָּא שֶׁהַמּוּרָם נֶאֱכָל לַכֹּהֲנִים לִנְשֵׁיהֶם וְלִבְנֵיהֶם וּלְעַבְדֵיהֶם׃

(ז) שְׁלָמִים - קׇדָשִׁים קַלִּים - שְׁחִיטָתָן בְּכׇל־מָקוֹם בָּעֲזָרָה \middot וְדָמָן טָעוּן שְׁתֵּי מַתָּנוֹת שֶׁהֵן אַרְבַּע \middot וְנֶאֱכָלִין בְּכׇל־הָעִיר לְכׇל־אָדָם בְּכׇל־מַאֲכָל לִשְׁנֵי יָמִים וְלַֽיְלָה אֶחָד׃ הַמּוּרָם מֵהֶם - כַּיּוֹצֵא בָהֶן \middot אֶלָּא שֶׁהַמּוּרָם נֶאֱכָל לַכֹּהֲנִים לִנְשֵׁיהֶם וְלִבְנֵיהֶם וּלְעַבְדֵיהֶם׃

(ח) הַבְּכוֹר וְהַמַּעֲשֵׂר וְהַפֶּֽסַח - קׇדָשִׁים קַלִּים - שְׁחִיטָתָן בְּכׇל־מָקוֹם בָּעֲזָרָה \middot וְדָמָן טָעוּן מַתָּנָה אֶחָת - וּבִלְבָד שֶׁיִּתֵּן כְּנֶֽגֶד הַיְסוֹד׃ שִׁנָּה בַאֲכִילָתָן \middot הַבְּכוֹר נֶאֱכָל לַכֹּהֲנִים \middot וְהַמַּעֲשֵׂר לְכׇל־אָדָם \middot וְנֶּאֱכָלִין בְּכׇל־הָעִיר בְּכׇל־מַאֲכָל לִשְׁנֵי יָמִים וְלַֽיְלָה אֶחָד׃ הַפֶּֽסַח אֵינוֹ נֶאֱכָל אֶלָּא בַלַּֽיְלָה וְאֵינוֹ נֶאֱכָל אֶלָּא עַד־חֲצוֹת וְאֵינוֹ נֶאֱכָל אֶלָּא לִמְנוּיָיו וְאֵינוֹ נֶאֱכָל אֶלָּא צָלִי׃

\firstword{רַבִּי יִשְׁמָעֵאל אוֹמֵר}\source{ספרא ויקרא}
בִּשְׁלֹשׁ עֶשְׂרֵה מִדּוֹת הַתּוֹרָה נִדְרֶֽשֶׁת׃\hfill \break
(א) מִקַּל וָחֹמֶר (ב) וּמִגְּזֵרָה שָׁוָה (ג) מִבִּנְיַן אָב מִכָּתוּב אֶחָד \middot וּמִבִּנְיַן אָב מִשְּׁנֵי כְתוּבִים (ד) מִכְּלָל וּפְרָט (ה) וּמִפְּרָט וּכְלָל (ו) כְּלָל וּפְרָט וּכְלָל \middot אֵי אַתָּה דָן אֶלָּא כְּעֵין הַפְּרָט (ז) מִכְּלָל שֶׁהוּא צָרִיךְ לִפְרָט \middot וּמִפְּרָט שֶׁהוּא צָרִיךְ לִכְלָל (ח) כׇּל־דָּבָר שֶׁהָיָה בִּכְלָל וְיָצָא מִן הַכְּלָל לְלַמֵּד \middot לֹא לְלַמֵּד עַל עַצְמוֹ יָצָא אֶלָּא לְלַמֵּד עַל הַכְּלָל כֻּלּוֹ יָצָא (ט) כׇּל־דָּבָר שֶׁהָיָה בִּכְלָל וְיָצָא לִטְעוֹן טַעַן אֶחָד שֶׁהוּא כְעִנְיָנוֹ \middot יָצָא לְהָקֵל וְלֹא לְהַחֲמִיר (י) כׇּל־דָּבָר שֶׁהָיָה בִּכְלָל וְיָצָא לִטְעוֹן טַעַן אַחֵר שֶׁלֹּא כְעִנְיָנוֹ \middot יָצָא לְהָקֵל וּלְהַחֲמִיר (יא) כׇּל־דָּבָר שֶׁהָיָה בִּכְלָל וְיָצָא לִדּוֹן בְּדָבָר חָדָשׁ \middot אֵי אַתָּה יָכוֹל לְהַחֲזִירוֹ לִכְלָלוֹ עַד שֶׁיַּחֲזִירֶנּוּ הַכָּתוּב לִכְלָלוֹ בְּפֵרוּשׁ (יב) דָּבָר הַלָּמֵד מֵעִנְיָנוֹ \middot וְדָבָר הַלָּמֵד מִסּוֹפוֹ (יג) וְכֵן שְׁנֵי כְתוּבִים הַמַּכְחִישִׁים זֶה אֶת־זֶה \middot עַד שֶׁיָּבוֹא הַכָּתוּב הַשְּׁלִישִׁי וְיַכְרִיעַ בֵּינֵיהֶם׃

יְהִי רָצוֹן מִלְּפָנֶֽיךָ יְיָ אֱלֹהֵֽינוּ וֵאלֹהֵי אֲבוֹתֵֽינוּ שֶׁיִּבָּנֶה בֵּית הַמִּקְדָּשׁ בִּמְהֵרָה בְיָמֵֽינוּ וְתֵן חֶלְקֵֽנוּ בְּתוֹרָתֶֽךָ \middot וְשָׁם נַעֲבׇדְךָ בְּיִרְאָה כִּימֵי עוֹלָם וּכְשָׁנִים קַדְמֹנִיּוֹת׃

\rabbiskaddish

%\section[כניסה לבהכ״נ Synagogue at Arriving]{\adforn{53} Synagogue at Arriving \adforn{25}\\כניסה לבהכ״נ }
\section[כניסה לבהכ״נ]{\adforn{53} כניסה לבהכ״נ \adforn{25}}

\englishinst{On entering the synagogue:}
\firstword{מַה־טֹּ֥בוּ}\source{במדבר נה}
אֹהָלֶ֖יךָ יַעֲקֹ֑ב מִשְׁכְּנֹתֶ֖יךָ יִשְׂרָאֵֽל׃
וַאֲנִ֗י\source{תהלים ה}
בְּרֹ֣ב חַ֭סְדְּךָ אָב֣וֹא בֵיתֶ֑ךָ אֶשְׁתַּחֲוֶ֥ה אֶל־הֵיכַל־קׇ֝דְשְׁךָ֗ בְּיִרְאָתֶֽךָ׃\\
\source{תהלים כו}
יְיָ֗ אָ֭הַבְתִּי מְע֣וֹן בֵּיתֶ֑ךָ וּ֝מְק֗וֹם מִשְׁכַּ֥ן כְּבוֹדֶֽךָ׃
וַאֲנִי אֶשְׁתַּחֲוֶה וְאֶכְרָֽעָה אֶבְרְכָה לִפְנֵי־יְיָ עֹשִׂי׃
וַאֲנִ֤י
\source{תהלים סט}%
תְפִלָּֽתִי־לְךָ֨ ׀ יְיָ֡ עֵ֤ת רָצ֗וֹן אֱלֹהִ֥ים בְּרׇב־חַסְדֶּ֑ךָ עֲ֝נֵ֗נִי בֶּאֱמֶ֥ת יִשְׁעֶֽךָ׃

{\footnotesize 
שִׁ֥יר\source{תהילים קכב} הַֽמַּעֲל֗וֹת לְדָ֫וִ֥ד שָׂ֭מַחְתִּי בְּאֹמְרִ֣ים לִ֑י בֵּ֖ית יְיָ֣ נֵלֵֽךְ׃ עֹ֭מְדוֹת הָי֣וּ רַגְלֵ֑ינוּ בִּ֝שְׁעָרַ֗יִךְ יְרוּשָׁלָֽ͏ִם׃ יְרוּשָׁלַ֥͏ִם הַבְּנוּיָ֑ה כְּ֝עִ֗יר שֶׁחֻבְּרָה־לָּ֥הּ יַחְדָּֽו׃ שֶׁשָּׁ֨ם עָל֪וּ שְׁבָטִ֡ים שִׁבְטֵי־יָ֭הּ עֵד֣וּת לְיִשְׂרָאֵ֑ל לְ֝הֹד֗וֹת לְשֵׁ֣ם יְיָ׃ כִּ֤י שָׁ֨מָּה ׀ יָשְׁב֣וּ כִסְא֣וֹת לְמִשְׁפָּ֑ט כִּ֝סְא֗וֹת לְבֵ֣ית דָּוִֽד׃ שַׁ֭אֲלוּ שְׁל֣וֹם יְרוּשָׁלָ֑͏ִם יִ֝שְׁלָ֗יוּ אֹהֲבָֽיִךְ׃ יְהִי־שָׁל֥וֹם בְּחֵילֵ֑ךְ שַׁ֝לְוָ֗ה בְּאַרְמְנוֹתָֽיִךְ׃ לְ֭מַעַן אַחַ֣י וְרֵעָ֑י אֲדַבְּרָה־נָּ֖א שָׁל֣וֹם בָּֽךְ׃ לְ֭מַעַן בֵּית־יְיָ֣ אֱלֹהֵ֑ינוּ אֲבַקְשָׁ֖ה ט֣וֹב לָֽךְ׃}


\ssubsection{\adforn{18} עטיפת טלית \adforn{17}}
\\
\englishinst{On putting on the talli\thav\space gadol:}
\firstword{בָּרוּךְ}
אַתָּה יְיָ אֱלֹהֵֽינוּ מֶֽלֶךְ הָעוֹלָם \middot אֲשֶׁר קִדְּשָֽׁנוּ בְּמִצְוֹתָיו וְצִוָּֽנוּ לְהִתְעַטֵּף בַּצִּיצִת׃\\
\begin{footnotesize}
	מַה־יָּקָ֥ר\source{תהילים לו} חַסְדְּךָ֗ אֱלֹ֫הִ֥ים וּבְנֵ֥י אָדָ֑ם בְּצֵ֥ל כְּ֝נָפֶ֗יךָ יֶחֱסָיֽוּן׃ יִ֭רְוְיֻן מִדֶּ֣שֶׁן בֵּיתֶ֑ךָ וְנַ֖חַל עֲדָנֶ֣יךָ תַשְׁקֵֽם׃ כִּֽי־עִ֭מְּךָ מְק֣וֹר חַיִּ֑ים בְּ֝אוֹרְךָ֗ נִרְאֶה־אֽוֹר׃ מְשֹׁ֣ךְ חַ֭סְדְּךָ לְיֹדְעֶ֑יךָ וְ֝צִדְקָֽתְךָ֗ לְיִשְׁרֵי־לֵֽב׃\\
	יְהִי רָצוֹן מִלְּפָנֶֽיךָ יְיָ אֱלֹהֵֽינוּ וֵאלֹהֵי אֲבוֹתֵֽינוּ. שֶׁתְּהֵא חֲשׁוּבָה מִצְוַת צִיצִת זוֹ כְּאִלּוּ קִיַּמְתִּֽיהָ בְּכׇל־פְּרָטֶֽיהָ וְדִקְדּוּקֶֽיהָ וְכַוָּנוֹתֶֽיהָ וְתַרְיָ"ג מִצְוֺת הַתְּלוּיִם בָּהּ. אָמֵן סֶֽלָה. 
\end{footnotesize}

\ifboolexpr{togl {includeweekday}}{
\ssubsection{\adforn{18} הנחת תפילין \adforn{17}}\\
\englishinst{Before laying on tefillin on the arm:}
%\instruction{לפני הנחת תפילין של יד׃ }
\firstword{בָּרוּךְ}
אַתָּה יְיָ אֱלֹהֵֽינוּ מֶֽלֶךְ הָעוֺלָם \middot אֲשֶׁר קִדְּשָֽׁנוּ בְּמִצְוֹתָיו וְצִוָֽנוּ לְהַנִּֽיחַ תְּפִלִּין׃

\englishinst{Before laying on the tefillin on the head:}
%\instruction{לפני הנחת תפילין של ראש׃ }
\firstword{בָּרוּךְ}
אַתָּה יְיָ, אֱלֹהֵֽינוּ מֶֽלֶךְ הָעוֺלָם \middot אֲשֶׁר קִדְּשָֽׁנוּ בְּמִצְוֹתָיו וְצִוָֽנוּ עַל־מִצְוַת תְּפִלִּין׃

\englishinst{After the tefillin is secured on the head:}
%\instruction{אחרי הנחת תפילין של ראש׃ }
בָּרוּךְ שֵׁם כְּבוֺד מַלְכוּתוֺ לְעוֺלָם וָעֶד׃

\englishinst{As the tefillin is wrapped around the fingers:}
%\instruction{כשקושר הרצועה על היד׃ }\\
וְאֵרַשְׂתִּ֥יךְ \source{הושע ב}לִ֖י לְעוֹלָ֑ם וְאֵרַשְׂתִּ֥יךְ לִי֙ בְּצֶ֣דֶק וּבְמִשְׁפָּ֔ט וּבְחֶ֖סֶד וּֽבְרַחֲמִֽים׃ וְאֵרַשְׂתִּ֥יךְ לִ֖י בֶּאֱמוּנָ֑ה וְיָדַ֖עַתְּ אֶת־יְיָ׃

\englishinst{Some say the following after laying on the tefillin:}
%\instruction{יש אומרים אחר הנחת תפילין:}\\
יְהִי רָצוֹן מִלְּפָנֶֽיךָ יְיָ אֱלֹהֵֽינוּ וֵאלֹהֵי אֲבוֹתֵֽינוּ שֶׁתְּהֵא חֲשׁוּבָה מִצְוַת הֲנָחַת תְּפִלִּין זוֹ כְּאִלּוּ קִיַּמְתִּֽיהָ בְּכׇל־פְּרָטֶֽיהָ וְדִקְדּוּקֶֽיהָ וְכַוׇּנוֹתֶֽיהָ וְתַרְיַ"ג מִצְוֺת הַתְּלוּיִם בָּהּ׃ אָמֵן סֶֽלָה׃

\begin{footnotesize}	
וַיְדַבֵּ֥ר יְיָ֖ אֶל־מֹשֶׁ֥ה לֵּאמֹֽר׃ \source{שמות יג}קַדֶּשׁ־לִ֨י כׇל־בְּכ֜וֹר פֶּ֤טֶר כׇּל־רֶ֙חֶם֙ בִּבְנֵ֣י יִשְׂרָאֵ֔ל בָּאָדָ֖ם וּבַבְּהֵמָ֑ה לִ֖י הֽוּא׃ וַיֹּ֨אמֶר מֹשֶׁ֜ה אֶל־הָעָ֗ם זָכ֞וֹר אֶת־הַיּ֤וֹם הַזֶּה֙ אֲשֶׁ֨ר יְצָאתֶ֤ם מִמִּצְרַ֙יִם֙ מִבֵּ֣ית עֲבָדִ֔ים כִּ֚י בְּחֹ֣זֶק יָ֔ד הוֹצִ֧יא יְיָ֛ אֶתְכֶ֖ם מִזֶּ֑ה וְלֹ֥א יֵאָכֵ֖ל חָמֵֽץ׃ הַיּ֖וֹם אַתֶּ֣ם יֹצְאִ֑ים בְּחֹ֖דֶשׁ הָאָבִֽיב׃ וְהָיָ֣ה כִֽי־יְבִיאֲךָ֣ יְיָ֡ אֶל־אֶ֣רֶץ הַֽ֠כְּנַעֲנִ֠י וְהַחִתִּ֨י וְהָאֱמֹרִ֜י וְהַחִוִּ֣י וְהַיְבוּסִ֗י אֲשֶׁ֨ר נִשְׁבַּ֤ע לַאֲבֹתֶ֙יךָ֙ לָ֣תֶת לָ֔ךְ אֶ֛רֶץ זָבַ֥ת חָלָ֖ב וּדְבָ֑שׁ וְעָבַדְתָּ֛ אֶת־הָעֲבֹדָ֥ה הַזֹּ֖את בַּחֹ֥דֶשׁ הַזֶּֽה׃ שִׁבְעַ֥ת יָמִ֖ים תֹּאכַ֣ל מַצֹּ֑ת וּבַיּוֹם֙ הַשְּׁבִיעִ֔י חַ֖ג לַייָ׃ מַצּוֹת֙ יֵֽאָכֵ֔ל אֵ֖ת שִׁבְעַ֣ת הַיָּמִ֑ים וְלֹֽא־יֵרָאֶ֨ה לְךָ֜ חָמֵ֗ץ וְלֹֽא־יֵרָאֶ֥ה לְךָ֛ שְׂאֹ֖ר בְּכׇל־גְּבֻלֶֽךָ׃ וְהִגַּדְתָּ֣ לְבִנְךָ֔ בַּיּ֥וֹם הַה֖וּא לֵאמֹ֑ר בַּעֲב֣וּר זֶ֗ה עָשָׂ֤ה יְיָ֙ לִ֔י בְּצֵאתִ֖י מִמִּצְרָֽיִם׃ וְהָיָה֩ לְךָ֨ לְא֜וֹת עַל־יָדְךָ֗ וּלְזִכָּרוֹן֙ בֵּ֣ין עֵינֶ֔יךָ לְמַ֗עַן תִּהְיֶ֛ה תּוֹרַ֥ת יְיָ֖ בְּפִ֑יךָ כִּ֚י בְּיָ֣ד חֲזָקָ֔ה הוֹצִֽאֲךָ֥ יְיָ֖ מִמִּצְרָֽיִם׃ וְשָׁמַרְתָּ֛ אֶת־הַחֻקָּ֥ה הַזֹּ֖את לְמוֹעֲדָ֑הּ מִיָּמִ֖ים יָמִֽימָה׃\hfill\break
וְהָיָ֞ה כִּֽי־יְבִאֲךָ֤ יְיָ֙ אֶל־אֶ֣רֶץ הַֽכְּנַעֲנִ֔י כַּאֲשֶׁ֛ר נִשְׁבַּ֥ע לְךָ֖ וְלַֽאֲבֹתֶ֑יךָ וּנְתָנָ֖הּ לָֽךְ׃ וְהַעֲבַרְתָּ֥ כׇל־פֶּֽטֶר־רֶ֖חֶם לַֽייָ֑ וְכׇל־פֶּ֣טֶר ׀ שֶׁ֣גֶר בְּהֵמָ֗ה אֲשֶׁ֨ר יִהְיֶ֥ה לְךָ֛ הַזְּכָרִ֖ים לַייָ׃ וְכׇל־פֶּ֤טֶר חֲמֹר֙ תִּפְדֶּ֣ה בְשֶׂ֔ה וְאִם־לֹ֥א תִפְדֶּ֖ה וַעֲרַפְתּ֑וֹ וְכֹ֨ל בְּכ֥וֹר אָדָ֛ם בְּבָנֶ֖יךָ תִּפְדֶּֽה׃ וְהָיָ֞ה כִּֽי־יִשְׁאָלְךָ֥ בִנְךָ֛ מָחָ֖ר לֵאמֹ֣ר מַה־זֹּ֑את וְאָמַרְתָּ֣ אֵלָ֔יו בְּחֹ֣זֶק יָ֗ד הוֹצִיאָ֧נוּ יְיָ֛ מִמִּצְרַ֖יִם מִבֵּ֥ית עֲבָדִֽים׃ וַיְהִ֗י כִּֽי־הִקְשָׁ֣ה פַרְעֹה֮ לְשַׁלְּחֵ֒נוּ֒ וַיַּהֲרֹ֨ג יְיָ֤ כׇּל־בְּכוֹר֙ בְּאֶ֣רֶץ מִצְרַ֔יִם מִבְּכֹ֥ר אָדָ֖ם וְעַד־בְּכ֣וֹר בְּהֵמָ֑ה עַל־כֵּן֩ אֲנִ֨י זֹבֵ֜חַ לַֽייָ֗ כׇּל־פֶּ֤טֶר רֶ֙חֶם֙ הַזְּכָרִ֔ים וְכׇל־בְּכ֥וֹר בָּנַ֖י אֶפְדֶּֽה׃ וְהָיָ֤ה לְאוֹת֙ עַל־יָ֣דְכָ֔ה וּלְטוֹטָפֹ֖ת בֵּ֣ין עֵינֶ֑יךָ כִּ֚י בְּחֹ֣זֶק יָ֔ד הוֹצִיאָ֥נוּ יְיָ֖ מִמִּצְרָֽיִם׃
\end{footnotesize}

}{}

\vspace{0.4in}
\chapter[פסוקי דזמרא]{\adforn{47} פסוקי דזמרא \adforn{19}}
\vspace{0.15in}
\newcommand{\longPDZ}{
	לַמְנַצֵּ֗חַ\source{תהילים יט} מִזְמ֥וֹר לְדָוִֽד׃ הַשָּׁמַ֗יִם מְֽסַפְּרִ֥ים כְּבֽוֹד־אֵ֑ל וּֽמַעֲשֵׂ֥ה יָ֝דָ֗יו מַגִּ֥יד הָרָקִֽיעַ׃ י֣וֹם לְ֭יוֹם יַבִּ֣יעַֽ אֹ֑מֶר וְלַ֥יְלָה לְּ֝לַ֗יְלָה יְחַוֶּה־דָּֽעַת׃ אֵֽין־אֹ֭מֶר וְאֵ֣ין דְּבָרִ֑ים בְּ֝לִ֗י נִשְׁמָ֥ע קוֹלָֽם׃ בְּכׇל־הָאָ֨רֶץ ׀ יָ֘צָ֤א קַוָּ֗ם וּבִקְצֵ֣ה תֵ֭בֵל מִלֵּיהֶ֑ם לַ֝שֶּׁ֗מֶשׁ שָֽׂם־אֹ֥הֶל בָּהֶֽם׃ וְה֗וּא כְּ֭חָתָן יֹצֵ֣א מֵחֻפָּת֑וֹ יָשִׂ֥ישׂ כְּ֝גִבּ֗וֹר לָר֥וּץ אֹֽרַח׃ מִקְצֵ֤ה הַשָּׁמַ֨יִם ׀ מֽוֹצָא֗וֹ וּתְקוּפָת֥וֹ עַל־קְצוֹתָ֑ם וְאֵ֥ין נִ֝סְתָּ֗ר מֵחַמָּתֽוֹ׃ תּ֘וֹרַ֤ת יְיָ֣ תְּ֭מִימָה מְשִׁ֣יבַת נָ֑פֶשׁ עֵד֥וּת יְיָ֥ נֶ֝אֱמָנָ֗ה מַחְכִּ֥ימַת פֶּֽתִי׃ פִּקּ֘וּדֵ֤י יְיָ֣ יְ֭שָׁרִים מְשַׂמְּחֵי־לֵ֑ב מִצְוַ֥ת יְיָ֥ בָּ֝רָ֗ה מְאִירַ֥ת עֵינָֽיִם׃ יִרְאַ֤ת יְיָ֨ ׀ טְהוֹרָה֮ עוֹמֶ֢דֶת לָ֫עַ֥ד מִֽשְׁפְּטֵי־יְיָ֥ אֱמֶ֑ת צָֽדְק֥וּ יַחְדָּֽו׃ הַֽנֶּחֱמָדִ֗ים מִ֭זָּהָב וּמִפַּ֣ז רָ֑ב וּמְתוּקִ֥ים מִ֝דְּבַ֗שׁ וְנֹ֣פֶת צוּפִֽים׃ גַּֽם־עַ֭בְדְּךָ נִזְהָ֣ר בָּהֶ֑ם בְּ֝שׇׁמְרָ֗ם עֵ֣קֶב רָֽב׃ שְׁגִיא֥וֹת מִֽי־יָבִ֑ין מִֽנִּסְתָּר֥וֹת נַקֵּֽנִי׃ גַּ֤ם מִזֵּדִ֨ים ׀ חֲשֹׂ֬ךְ עַבְדֶּ֗ךָ אַֽל־יִמְשְׁלוּ־בִ֣י אָ֣ז אֵיתָ֑ם וְ֝נִקֵּ֗יתִי מִפֶּ֥שַֽׁע רָֽב׃ יִ֥הְיֽוּ לְרָצ֨וֹן ׀ אִמְרֵי־פִ֡י וְהֶגְי֣וֹן לִבִּ֣י לְפָנֶ֑יךָ יְ֝יָ֗ צוּרִ֥י וְגֹאֲלִֽי׃
	
	
	\enlargethispage{\baselineskip}
	
	לְדָוִ֗ד\source{תהילים לד} בְּשַׁנּוֹת֣וֹ אֶת־טַ֭עְמוֹ לִפְנֵ֣י אֲבִימֶ֑לֶךְ וַ֝יְגָרְשֵׁ֗הוּ וַיֵּלַֽךְ׃ אֲבָרְכָ֣ה אֶת־יְיָ֣ בְּכׇל־עֵ֑ת תָּ֝מִ֗יד תְּֽהִלָּת֥וֹ בְּפִֽי׃ בַּייָ֭ תִּתְהַלֵּ֣ל נַפְשִׁ֑י יִשְׁמְע֖וּ עֲנָוִ֣ים וְיִשְׂמָֽחוּ׃ גַּדְּל֣וּ לַייָ֣ אִתִּ֑י וּנְרוֹמְמָ֖ה שְׁמ֣וֹ יַחְדָּֽו׃ דָּרַ֣שְׁתִּי אֶת־יְיָ֣ וְעָנָ֑נִי וּמִכׇּל־מְ֝גוּרוֹתַ֗י הִצִּילָֽנִי׃ הִבִּ֣יטוּ אֵלָ֣יו וְנָהָ֑רוּ וּ֝פְנֵיהֶ֗ם אַל־יֶחְפָּֽרוּ׃ זֶ֤ה עָנִ֣י קָ֭רָא וַייָ֣ שָׁמֵ֑עַ וּמִכׇּל־צָ֝רוֹתָ֗יו הוֹשִׁיעֽוֹ׃ חֹנֶ֤ה מַלְאַךְ־יְיָ֓ סָ֘בִ֤יב לִירֵאָ֗יו וַֽיְחַלְּצֵֽם׃ טַעֲמ֣וּ וּ֭רְאוּ כִּֽי־ט֣וֹב יְיָ֑ אַֽשְׁרֵ֥י הַ֝גֶּ֗בֶר יֶחֱסֶה־בּֽוֹ׃ יְר֣אוּ אֶת־יְיָ֣ קְדֹשָׁ֑יו כִּי־אֵ֥ין מַ֝חְס֗וֹר לִירֵאָֽיו׃ כְּ֭פִירִים רָשׁ֣וּ וְרָעֵ֑בוּ וְדֹרְשֵׁ֥י יְ֝יָ֗ לֹא־יַחְסְר֥וּ כׇל־טֽוֹב׃ לְֽכוּ־בָ֭נִים שִׁמְעוּ־לִ֑י יִֽרְאַ֥ת יְ֝יָ֗ אֲלַמֶּדְכֶֽם׃ מִֽי־הָ֭אִישׁ הֶחָפֵ֣ץ חַיִּ֑ים אֹהֵ֥ב יָ֝מִ֗ים לִרְא֥וֹת טֽוֹב׃ נְצֹ֣ר לְשׁוֹנְךָ֣ מֵרָ֑ע וּ֝שְׂפָתֶ֗יךָ מִדַּבֵּ֥ר מִרְמָֽה׃ ס֣וּר מֵ֭רָע וַעֲשֵׂה־ט֑וֹב בַּקֵּ֖שׁ שָׁל֣וֹם וְרׇדְפֵֽהוּ׃ עֵינֵ֣י יְיָ֭ אֶל־צַדִּיקִ֑ים וְ֝אׇזְנָ֗יו אֶל־שַׁוְעָתָֽם׃ פְּנֵ֣י יְיָ֭ בְּעֹ֣שֵׂי רָ֑ע לְהַכְרִ֖ית מֵאֶ֣רֶץ זִכְרָֽם׃ צָ֭עֲקוּ וַייָ֣ שָׁמֵ֑עַ וּמִכׇּל־צָ֝רוֹתָ֗ם הִצִּילָֽם׃ קָר֣וֹב יְיָ֭ לְנִשְׁבְּרֵי־לֵ֑ב וְֽאֶת־דַּכְּאֵי־ר֥וּחַ יוֹשִֽׁיעַ׃ רַ֭בּוֹת רָע֣וֹת צַדִּ֑יק וּ֝מִכֻּלָּ֗ם יַצִּילֶ֥נּוּ יְיָ׃ שֹׁמֵ֥ר כׇּל־עַצְמוֹתָ֑יו אַחַ֥ת מֵ֝הֵ֗נָּה לֹ֣א נִשְׁבָּֽרָה׃ תְּמוֹתֵ֣ת רָשָׁ֣ע רָעָ֑ה וְשֹׂנְאֵ֖י צַדִּ֣יק יֶאְשָֽׁמוּ׃ פֹּדֶ֣ה יְיָ֭ נֶ֣פֶשׁ עֲבָדָ֑יו וְלֹ֥א יֶ֝אְשְׁמ֗וּ כׇּֽל־הַחֹסִ֥ים בּֽוֹ׃
	
	
	תְּפִלָּה֮\source{תהילים צ} לְמֹשֶׁ֢ה אִֽישׁ־הָאֱלֹ֫הִ֥ים אֲֽדֹנָ֗י מָע֣וֹן אַ֭תָּה הָיִ֥יתָ לָּ֗נוּ בְּדֹ֣ר וָדֹֽר׃ בְּטֶ֤רֶם ׀ הָ֘רִ֤ים יֻלָּ֗דוּ וַתְּח֣וֹלֵֽל אֶ֣רֶץ וְתֵבֵ֑ל וּֽמֵעוֹלָ֥ם עַד־ע֝וֹלָ֗ם אַתָּ֥ה אֵֽל׃ תָּשֵׁ֣ב אֱ֭נוֹשׁ עַד־דַּכָּ֑א וַ֝תֹּ֗אמֶר שׁ֣וּבוּ בְנֵֽי־אָדָֽם׃ כִּ֤י אֶ֪לֶף שָׁנִ֡ים בְּֽעֵינֶ֗יךָ כְּי֣וֹם אֶ֭תְמוֹל כִּ֣י יַֽעֲבֹ֑ר וְאַשְׁמוּרָ֥ה בַלָּֽיְלָה׃ זְ֭רַמְתָּם שֵׁנָ֣ה יִהְי֑וּ בַּ֝בֹּ֗קֶר כֶּחָצִ֥יר יַחֲלֹֽף׃ בַּ֭בֹּקֶר יָצִ֣יץ וְחָלָ֑ף לָ֝עֶ֗רֶב יְמוֹלֵ֥ל וְיָבֵֽשׁ׃ כִּֽי־כָלִ֥ינוּ בְאַפֶּ֑ךָ וּֽבַחֲמָתְךָ֥ נִבְהָֽלְנוּ׃ שַׁתָּ֣ עֲוֺנֹתֵ֣ינוּ לְנֶגְדֶּ֑ךָ עֲ֝לֻמֵ֗נוּ לִמְא֥וֹר פָּנֶֽיךָ׃ כִּ֣י כׇל־יָ֭מֵינוּ פָּנ֣וּ בְעֶבְרָתֶ֑ךָ כִּלִּ֖ינוּ שָׁנֵ֣ינוּ כְמוֹ־הֶֽגֶה׃ יְמֵֽי־שְׁנוֹתֵ֨ינוּ בָהֶ֥ם שִׁבְעִ֪ים שָׁנָ֡ה וְאִ֤ם בִּגְבוּרֹ֨ת ׀ שְׁמ֘וֹנִ֤ים שָׁנָ֗ה וְ֭רׇהְבָּם עָמָ֣ל וָאָ֑וֶן כִּי־גָ֥ז חִ֝֗ישׁ וַנָּעֻֽפָה׃ מִֽי־י֭וֹדֵעַ עֹ֣ז אַפֶּ֑ךָ וּ֝כְיִרְאָתְךָ֗ עֶבְרָתֶֽךָ׃ לִמְנ֣וֹת יָ֭מֵינוּ כֵּ֣ן הוֹדַ֑ע וְ֝נָבִ֗א לְבַ֣ב חׇכְמָֽה׃ שׁוּבָ֣ה יְיָ֭ עַד־מָתָ֑י וְ֝הִנָּחֵ֗ם עַל־עֲבָדֶֽיךָ׃ שַׂבְּעֵ֣נוּ בַבֹּ֣קֶר חַסְדֶּ֑ךָ וּֽנְרַנְּנָ֥ה וְ֝נִשְׂמְחָ֗ה בְּכׇל־יָמֵֽינוּ׃ שַׂ֭מְּחֵנוּ כִּימ֣וֹת עִנִּיתָ֑נוּ שְׁ֝נ֗וֹת רָאִ֥ינוּ רָעָֽה׃ יֵרָאֶ֣ה אֶל־עֲבָדֶ֣יךָ פׇעֳלֶ֑ךָ וַ֝הֲדָרְךָ֗ עַל־בְּנֵיהֶֽם׃ וִיהִ֤י ׀ נֹ֤עַם אֲדֹנָ֥י אֱלֹהֵ֗ינוּ עָ֫לֵ֥ינוּ וּמַעֲשֵׂ֣ה יָ֭דֵינוּ כּוֹנְנָ֥ה עָלֵ֑ינוּ וּֽמַעֲשֵׂ֥ה יָ֝דֵ֗ינוּ כּוֹנְנֵֽהוּ׃
	
	\tzadialeph
	
	הַ֥לְלוּ־יָ֨הּ\source{תהילים קלה} ׀ הַֽ֭לְלוּ אֶת־שֵׁ֣ם יְיָ֑ הַֽ֝לְל֗וּ עַבְדֵ֥י יְיָ׃ שֶׁ֣֭עֹמְדִים בְּבֵ֣ית יְיָ֑ בְּ֝חַצְר֗וֹת בֵּ֣ית אֱלֹהֵֽינוּ׃ הַֽלְלוּ־יָ֭הּ כִּֽי־ט֣וֹב יְיָ֑ זַמְּר֥וּ לִ֝שְׁמ֗וֹ כִּ֣י נָעִֽים׃ כִּֽי־יַעֲקֹ֗ב בָּחַ֣ר ל֣וֹ יָ֑הּ יִ֝שְׂרָאֵ֗ל לִסְגֻלָּתֽוֹ׃ כִּ֤י אֲנִ֣י יָ֭דַעְתִּי כִּֽי־גָד֣וֹל יְיָ֑ וַ֝אֲדֹנֵ֗ינוּ מִכׇּל־אֱלֹהִֽים׃ כֹּ֤ל אֲשֶׁר־חָפֵ֥ץ יְיָ֗ עָ֫שָׂ֥ה בַּשָּׁמַ֥יִם וּבָאָ֑רֶץ בַּ֝יַּמִּ֗ים וְכׇל־תְּהֹמֽוֹת׃ מַעֲלֶ֣ה נְשִׂאִים֮ מִקְצֵ֢ה הָ֫אָ֥רֶץ בְּרָקִ֣ים לַמָּטָ֣ר עָשָׂ֑ה מֽוֹצֵא־ר֝֗וּחַ מֵאֽוֹצְרוֹתָֽיו׃ שֶׁ֭הִכָּה בְּכוֹרֵ֣י מִצְרָ֑יִם מֵ֝אָדָ֗ם עַד־בְּהֵמָֽה׃ שָׁלַ֤ח ׀ אוֹתֹ֣ת וּ֭מֹפְתִים בְּתוֹכֵ֣כִי מִצְרָ֑יִם בְּ֝פַרְעֹ֗ה וּבְכׇל־עֲבָדָֽיו׃ שֶׁ֭הִכָּה גּוֹיִ֣ם רַבִּ֑ים וְ֝הָרַ֗ג מְלָכִ֥ים עֲצוּמִֽים׃ לְסִיח֤וֹן ׀ מֶ֤לֶךְ הָאֱמֹרִ֗י וּ֭לְעוֹג מֶ֣לֶךְ הַבָּשָׁ֑ן וּ֝לְכֹ֗ל מַמְלְכ֥וֹת כְּנָֽעַן׃ וְנָתַ֣ן אַרְצָ֣ם נַחֲלָ֑ה נַ֝חֲלָ֗ה לְיִשְׂרָאֵ֥ל עַמּֽוֹ׃ יְיָ֭ שִׁמְךָ֣ לְעוֹלָ֑ם יְ֝יָ֗ זִכְרְךָ֥ לְדֹר־וָדֹֽר׃ כִּֽי־יָדִ֣ין יְיָ֣ עַמּ֑וֹ וְעַל־עֲ֝בָדָ֗יו יִתְנֶחָֽם׃ עֲצַבֵּ֣י הַ֭גּוֹיִם כֶּ֣סֶף וְזָהָ֑ב מַ֝עֲשֵׂ֗ה יְדֵ֣י אָדָֽם׃ פֶּֽה־לָ֭הֶם וְלֹ֣א יְדַבֵּ֑רוּ עֵינַ֥יִם לָ֝הֶ֗ם וְלֹ֣א יִרְאֽוּ׃ אׇזְנַ֣יִם לָ֭הֶם וְלֹ֣א יַאֲזִ֑ינוּ אַ֝֗ף אֵין־יֶשׁ־ר֥וּחַ בְּפִיהֶֽם׃ כְּ֭מוֹהֶם יִהְי֣וּ עֹשֵׂיהֶ֑ם כֹּ֖ל אֲשֶׁר־בֹּטֵ֣חַ בָּהֶֽם׃ בֵּ֣ית יִ֭שְׂרָאֵל בָּרְכ֣וּ אֶת־יְיָ֑ בֵּ֥ית אַ֝הֲרֹ֗ן בָּרְכ֥וּ אֶת־יְיָ׃ בֵּ֣ית הַ֭לֵּוִי בָּרְכ֣וּ אֶת־יְיָ֑ יִֽרְאֵ֥י יְ֝יָ֗ בָּרְכ֥וּ אֶת־יְיָ׃ בָּ֘ר֤וּךְ יְיָ֨ ׀ מִצִּיּ֗וֹן שֹׁ֘כֵ֤ן יְֽרוּשָׁלָ֗͏ִם הַֽלְלוּ־יָֽהּ׃
	
	
	הוֹד֣וּ לַייָ֣ כִּי־ט֑וֹב\source{תהלים קלו} \hfill
	כִּ֖י לְעוֹלָ֣ם חַסְדּֽוֹ׃\\
	ה֭וֹדוּ לֵאלֹהֵ֣י הָאֱלֹהִ֑ים \hfill כִּ֖י לְעוֹלָ֣ם חַסְדּֽוֹ׃\\
	ה֭וֹדוּ לַאֲדֹנֵ֣י הָאֲדֹנִ֑ים \hfill כִּ֖י לְעוֹלָ֣ם חַסְדּֽוֹ׃\\
	לְעֹ֘שֵׂ֤ה נִפְלָא֣וֹת גְּדֹל֣וֹת לְבַדּ֑וֹ \hfill כִּ֖י לְעוֹלָ֣ם חַסְדּֽוֹ׃\\
	לְעֹשֵׂ֣ה הַ֭שָּׁמַיִם בִּתְבוּנָ֑ה \hfill כִּ֖י לְעוֹלָ֣ם חַסְדּֽוֹ׃\\
	לְרֹקַ֣ע הָ֭אָרֶץ עַל־הַמָּ֑יִם \hfill כִּ֖י לְעוֹלָ֣ם חַסְדּֽוֹ׃\\
	לְ֭עֹשֵׂה אוֹרִ֣ים גְּדֹלִ֑ים \hfill כִּ֖י לְעוֹלָ֣ם חַסְדּֽוֹ׃\\
	אֶת־הַ֭שֶּׁמֶשׁ לְמֶמְשֶׁ֣לֶת בַּיּ֑וֹם \hfill כִּ֖י לְעוֹלָ֣ם חַסְדּֽוֹ׃\\
	אֶת־הַיָּרֵ֣חַ וְ֭כוֹכָבִים \hfill\break לְמֶמְשְׁל֣וֹת בַּלָּ֑יְלָה \hfill כִּ֖י לְעוֹלָ֣ם חַסְדּֽוֹ׃\\
	לְמַכֵּ֣ה מִ֭צְרַיִם בִּבְכוֹרֵיהֶ֑ם \hfill כִּ֖י לְעוֹלָ֣ם חַסְדּֽוֹ׃\\
	וַיּוֹצֵ֣א יִ֭שְׂרָאֵל מִתּוֹכָ֑ם \hfill כִּ֖י לְעוֹלָ֣ם חַסְדּֽוֹ׃\\
	בְּיָ֣ד חֲ֭זָקָה וּבִזְר֣וֹעַ נְטוּיָ֑ה \hfill כִּ֖י לְעוֹלָ֣ם חַסְדּֽוֹ׃\\
	לְגֹזֵ֣ר יַם־ס֭וּף לִגְזָרִ֑ים \hfill כִּ֖י לְעוֹלָ֣ם חַסְדּֽוֹ׃\\
	וְהֶעֱבִ֣יר יִשְׂרָאֵ֣ל בְּתוֹכ֑וֹ \hfill כִּ֖י לְעוֹלָ֣ם חַסְדּֽוֹ׃\\
	וְנִ֘עֵ֤ר פַּרְעֹ֣ה וְחֵיל֣וֹ בְיַם־ס֑וּף \hfill כִּ֖י לְעוֹלָ֣ם חַסְדּֽוֹ׃\\
	לְמוֹלִ֣יךְ עַ֭מּוֹ בַּמִּדְבָּ֑ר \hfill כִּ֖י לְעוֹלָ֣ם חַסְדּֽוֹ׃\\
	לְ֭מַכֵּה מְלָכִ֣ים גְּדֹלִ֑ים \hfill כִּ֖י לְעוֹלָ֣ם חַסְדּֽוֹ׃\\
	וַֽ֭יַּהֲרֹג מְלָכִ֣ים אַדִּירִ֑ים \hfill כִּ֖י לְעוֹלָ֣ם חַסְדּֽוֹ׃\\
	לְ֭סִיחוֹן מֶ֣לֶךְ הָאֱמֹרִ֑י \hfill כִּ֖י לְעוֹלָ֣ם חַסְדּֽוֹ׃\\
	וּ֭לְעוֹג מֶ֣לֶךְ הַבָּשָׁ֑ן \hfill כִּ֖י לְעוֹלָ֣ם חַסְדּֽוֹ׃\\
	וְנָתַ֣ן אַרְצָ֣ם לְנַחֲלָ֑ה \hfill כִּ֖י לְעוֹלָ֣ם חַסְדּֽוֹ׃\\
	נַ֭חֲלָה לְיִשְׂרָאֵ֣ל עַבְדּ֑וֹ \hfill כִּ֖י לְעוֹלָ֣ם חַסְדּֽוֹ׃\\
	שֶׁ֭בְּשִׁפְלֵנוּ זָ֣כַר לָ֑נוּ \hfill כִּ֖י לְעוֹלָ֣ם חַסְדּֽוֹ׃\\
	וַיִּפְרְקֵ֥נוּ מִצָּרֵ֑ינוּ \hfill כִּ֖י לְעוֹלָ֣ם חַסְדּֽוֹ׃\\
	נֹתֵ֣ן לֶ֭חֶם לְכׇל־בָּשָׂ֑ר \hfill כִּ֖י לְעוֹלָ֣ם חַסְדּֽוֹ׃\\
	ה֭וֹדוּ לְאֵ֣ל הַשָּׁמָ֑יִם \hfill כִּ֖י לְעוֹלָ֣ם חַסְדּֽוֹ׃\\
	
	
	רַנְּנ֣וּ\source{תהילים לג} צַ֭דִּיקִים בַּייָ֑ לַ֝יְשָׁרִ֗ים נָאוָ֥ה תְהִלָּֽה׃ הוֹד֣וּ לַייָ֣ בְּכִנּ֑וֹר בְּנֵ֥בֶל עָ֝שׂ֗וֹר זַמְּרוּ־לֽוֹ׃ שִֽׁירוּ־ל֭וֹ שִׁ֣יר חָדָ֑שׁ הֵיטִ֥יבוּ נַ֝גֵּ֗ן בִּתְרוּעָֽה׃ כִּֽי־יָשָׁ֥ר דְּבַר־יְיָ֑ וְכׇל־מַ֝עֲשֵׂ֗הוּ בֶּאֱמוּנָֽה׃ אֹ֭הֵב צְדָקָ֣ה וּמִשְׁפָּ֑ט חֶ֥סֶד יְ֝יָ֗ מָלְאָ֥ה הָאָֽרֶץ׃ בִּדְבַ֣ר יְיָ֭ שָׁמַ֣יִם נַעֲשׂ֑וּ וּבְר֥וּחַ פִּ֝֗יו כׇּל־צְבָאָֽם׃ כֹּנֵ֣ס כַּ֭נֵּד מֵ֣י הַיָּ֑ם נֹתֵ֖ן בְּאוֹצָר֣וֹת תְּהוֹמֽוֹת׃ יִֽירְא֣וּ מֵ֭ייָ כׇּל־הָאָ֑רֶץ מִמֶּ֥נּוּ יָ֝ג֗וּרוּ כׇּל־יֹשְׁבֵ֥י תֵבֵֽל׃ כִּ֤י ה֣וּא אָמַ֣ר וַיֶּ֑הִי הֽוּא־צִ֝וָּ֗ה וַֽיַּעֲמֹֽד׃ יְיָ֗ הֵפִ֥יר עֲצַת־גּוֹיִ֑ם הֵ֝נִ֗יא מַחְשְׁב֥וֹת עַמִּֽים׃ עֲצַ֣ת יְיָ֭ לְעוֹלָ֣ם תַּעֲמֹ֑ד מַחְשְׁב֥וֹת לִ֝בּ֗וֹ לְדֹ֣ר וָדֹֽר׃ אַשְׁרֵ֣י הַ֭גּוֹי אֲשֶׁר־יְיָ֣ אֱלֹהָ֑יו הָעָ֓ם ׀ בָּחַ֖ר לְנַחֲלָ֣ה לֽוֹ׃ מִ֭שָּׁמַיִם הִבִּ֣יט יְיָ֑ רָ֝אָ֗ה אֶֽת־כׇּל־בְּנֵ֥י הָאָדָֽם׃ מִֽמְּכוֹן־שִׁבְתּ֥וֹ הִשְׁגִּ֑יחַ אֶ֖ל כׇּל־יֹשְׁבֵ֣י הָאָֽרֶץ׃ הַיֹּצֵ֣ר יַ֣חַד לִבָּ֑ם הַ֝מֵּבִ֗ין אֶל־כׇּל־מַעֲשֵׂיהֶֽם׃ אֵֽין־הַ֭מֶּלֶךְ נוֹשָׁ֣ע בְּרׇב־חָ֑יִל גִּ֝בּ֗וֹר לֹא־יִנָּצֵ֥ל בְּרׇב־כֹּֽחַ׃ שֶׁ֣קֶר הַ֭סּוּס לִתְשׁוּעָ֑ה וּבְרֹ֥ב חֵ֝יל֗וֹ לֹ֣א יְמַלֵּֽט׃ הִנֵּ֤ה עֵ֣ין יְיָ֭ אֶל־יְרֵאָ֑יו לַֽמְיַחֲלִ֥ים לְחַסְדּֽוֹ׃ לְהַצִּ֣יל מִמָּ֣וֶת נַפְשָׁ֑ם וּ֝לְחַיּוֹתָ֗ם בָּרָעָֽב׃ נַ֭פְשֵׁנוּ חִכְּתָ֣ה לַֽייָ֑ עֶזְרֵ֖נוּ וּמָגִנֵּ֣נוּ הֽוּא׃ כִּי־ב֭וֹ יִשְׂמַ֣ח לִבֵּ֑נוּ כִּ֤י בְשֵׁ֖ם קׇדְשׁ֣וֹ בָטָֽחְנוּ׃ יְהִי־חַסְדְּךָ֣ יְיָ֣ עָלֵ֑ינוּ כַּ֝אֲשֶׁ֗ר יִחַ֥לְנוּ לָֽךְ׃
	
	\mizmorshabbat
	
	יְיָ֣ מָלָךְ֮ גֵּא֢וּת לָ֫בֵ֥שׁ\source{תהלים צג}
	לָבֵ֣שׁ יְיָ֭ עֹ֣ז הִתְאַזָּ֑ר אַף־תִּכּ֥וֹן תֵּ֝בֵ֗ל בַּל־תִּמּֽוֹט׃
	נָכ֣וֹן כִּסְאֲךָ֣ מֵאָ֑ז מֵעוֹלָ֣ם אָֽתָּה׃
	נָשְׂא֤וּ נְהָר֨וֹת ׀ יְיָ֗ נָשְׂא֣וּ נְהָר֣וֹת קוֹלָ֑ם יִשְׂא֖וּ נְהָר֣וֹת דׇּכְיָֽם׃
	מִקֹּל֨וֹת ׀ מַ֤יִם רַבִּ֗ים אַדִּירִ֣ים מִשְׁבְּרֵי־יָ֑ם אַדִּ֖יר בַּמָּר֣וֹם יְיָ׃
	עֵֽדֹתֶ֨יךָ ׀ נֶאֶמְנ֬וּ מְאֹ֗ד לְבֵיתְךָ֥ נַאֲוָה־קֹ֑דֶשׁ יְ֝יָ֗ לְאֹ֣רֶךְ יָמִֽים׃
}

\newcommand{\todah}{
מִזְמ֥וֹר\source{תהילים ק} לְתוֹדָ֑ה הָרִ֥יעוּ לַ֝ייָ֗ כׇּל־הָאָֽרֶץ׃ עִבְד֣וּ אֶת־יְיָ֣ בְּשִׂמְחָ֑ה בֹּ֥אוּ לְ֝פָנָ֗יו בִּרְנָנָֽה׃ דְּע֗וּ כִּֽי־יְיָ ה֤וּא אֱלֹ֫הִ֥ים הֽוּא־עָ֭שָׂנוּ (ולא) [וְל֣וֹ] אֲנַ֑חְנוּ עַ֝מּ֗וֹ וְצֹ֣אן מַרְעִיתֽוֹ׃ בֹּ֤אוּ שְׁעָרָ֨יו ׀ בְּתוֹדָ֗ה חֲצֵרֹתָ֥יו בִּתְהִלָּ֑ה הוֹדוּ־ל֝֗וֹ בָּרְכ֥וּ שְׁמֽוֹ׃ כִּי־ט֣וֹב יְיָ֭ לְעוֹלָ֣ם חַסְדּ֑וֹ וְעַד־דֹּ֥ר וָ֝דֹ֗ר אֱמוּנָתֽוֹ׃}

\instruction{יש אומרים מזמור זו ואח״כ קדיש יתום}\\\chanukat

\firstword{בָּרוּךְ שֶׁאָמַר}
וְהָיָה הָעוֹלָם בָּרוּךְ הוּא׃
בָּרוּךְ עוֹשֶׂה בְרֵאשִׁית בָּרוּךְ אוֹמֵר וְעוֹשֶׂה׃
בָּרוּךְ גּוֹזֵר וּמְקַיֵּם בָּרוּךְ מְרַחֵם עַל הָאָֽרֶץ׃
בָּרוּךְ מְרַחֵם עַל הַבְּרִיּוֹת בָּרוּךְ מְשַׁלֵּם שָׂכָר טוֹב לִירֵאָיו׃
בָּרוּךְ חַי לָעַד וְקַיָּם לָנֶֽצַח בָּרוּךְ פּוֹדֶה וּמַצִּיל בָּרוּךְ שְׁמוֹ׃
בָּרוּךְ אַתָּה יְיָ אֱלֹהֵֽינוּ מֶֽלֶךְ הָעוֹלָם הָאֵל אָב הָרַחֲמָן הַמְהֻלָּל בְּפִי עַמּוֹ מְשֻׁבָּח וּמְפֹאָר בִּלְשׁוֹן חֲסִידָיו וַעֲבָדָיו וּבְשִׁירֵי דָוִד עַבְדֶּֽךָ נְהַלֶּלְךָ יְיָ אֱלֹהֵֽינוּ בִּשְׁבָחוֹת וּבִזְמִירוֹת׃ נְגַדֶּלְךָ וּנְשַׁבֵּחֲךָ וּנְפָאֶרְךָ וְנַמְלִיכְךָ וְנַזְכִּיר שִׁמְךָ מַלְכֵּֽנוּ אֱלֹהֵֽינוּ׃
יָחִיד חֵי הָעוֹלָמִים מֶֽלֶךְ מְשֻׁבָּח וּמְפֹאָר עֲדֵי עַד שְׁמוֹ הַגָּדוֹל׃ בָּרוּךְ אַתָּה יְיָ מֶֽלֶךְ מְהֻלָּל בַּתֻּשְׁבָּחוֹת׃

\firstword{הוֹד֤וּ}
לַֽייָ֙ קִרְא֣וּ בִשְׁמ֔וֹ\source{דה״א טז}
הוֹדִ֥יעוּ בָעַמִּ֖ים עֲלִילֹתָֽיו׃
שִׁ֤ירוּ לוֹ֙ זַמְּרוּ־ל֔וֹ שִׂ֖יחוּ בְּכׇל־נִפְלְאֹתָֽיו׃
הִֽתְהַלְלוּ֙ בְּשֵׁ֣ם קׇדְשׁ֔וֹ יִשְׂמַ֕ח לֵ֖ב מְבַקְשֵׁ֥י יְיָ׃
דִּרְשׁ֤וּ יְיָ֙ וְעֻזּ֔וֹ בַּקְּשׁ֥וּ פָנָ֖יו תָּמִֽיד׃
זִכְר֗וּ נִפְלְאֹתָיו֙ אֲשֶׁ֣ר עָשָׂ֔ה מֹפְתָ֖יו וּמִשְׁפְּטֵי־פִֽיהוּ׃
זֶ֚רַע יִשְׂרָאֵ֣ל עַבְדּ֔וֹ בְּנֵ֥י יַעֲקֹ֖ב בְּחִירָֽיו׃
ה֚וּא יְיָ֣ אֱלֹהֵ֔ינוּ בְּכׇל־הָאָ֖רֶץ מִשְׁפָּטָֽיו׃
זִכְר֤וּ לְעוֹלָם֙ בְּרִית֔וֹ דָּבָ֥ר צִוָּ֖ה לְאֶ֥לֶף דּֽוֹר׃
אֲשֶׁ֤ר כָּרַת֙ אֶת־אַבְרָהָ֔ם וּשְׁבוּעָת֖וֹ לְיִצְחָֽק׃
וַיַּעֲמִידֶ֤הָ לְיַֽעֲקֹב֙ לְחֹ֔ק לְיִשְׂרָאֵ֖ל בְּרִ֥ית עוֹלָֽם׃
לֵאמֹ֗ר לְךָ֙ אֶתֵּ֣ן אֶֽרֶץ־כְּנָ֔עַן חֶ֖בֶל נַחֲלַתְכֶֽם׃
בִּהְיֽוֹתְכֶם֙ מְתֵ֣י מִסְפָּ֔ר כִּמְעַ֖ט וְגָרִ֥ים בָּֽהּ׃
וַיִּֽתְהַלְּכוּ֙ מִגּ֣וֹי אֶל־גּ֔וֹי וּמִמַּמְלָכָ֖ה אֶל־עַ֥ם אַחֵֽר׃
לֹֽא־הִנִּ֤יחַ לְאִישׁ֙ לְעׇשְׁקָ֔ם וַיּ֥וֹכַח עֲלֵיהֶ֖ם מְלָכִֽים׃
אַֽל־תִּגְּעוּ֙ בִּמְשִׁיחָ֔י וּבִנְבִיאַ֖י אַל־תָּרֵֽעוּ׃
שִׁ֤ירוּ לַֽייָ֙ כׇּל־הָאָ֔רֶץ בַּשְּׂר֥וּ מִיּֽוֹם־אֶל־י֖וֹם יְשׁוּעָתֽוֹ׃
סַפְּר֤וּ בַגּוֹיִם֙ אֶת־כְּבוֹד֔וֹ בְּכׇל־הָעַמִּ֖ים נִפְלְאֹתָֽיו׃
כִּי֩ גָד֨וֹל יְיָ֤ וּמְהֻלָּל֙ מְאֹ֔ד וְנוֹרָ֥א ה֖וּא עַל־כׇּל־אֱלֹהִֽים׃
כִּ֠י כׇּל־אֱלֹהֵ֤י הָֽעַמִּים֙ אֱלִילִ֔ים...וַייָ֖ שָׁמַ֥יִם עָשָֽׂה׃

ה֤וֹד וְהָדָר֙ לְפָנָ֔יו עֹ֥ז וְחֶדְוָ֖ה בִּמְקֹמֽוֹ׃
הָב֤וּ לַֽייָ֙ מִשְׁפְּח֣וֹת עַמִּ֔ים הָב֥וּ לַייָ֖ כָּב֥וֹד וָעֹֽז׃
הָב֥וּ לַֽייָ֖ כְּב֣וֹד שְׁמ֑וֹ שְׂא֤וּ מִנְחָה֙ וּבֹ֣אוּ לְפָנָ֔יו
הִשְׁתַּחֲו֥וּ לַֽייָ֖ בְּהַדְרַת־קֹֽדֶשׁ׃ חִ֤ילוּ מִלְּפָנָיו֙ כׇּל־הָאָ֔רֶץ
אַף־תִּכּ֥וֹן תֵּבֵ֖ל בַּל־תִּמּֽוֹט׃ יִשְׂמְח֤וּ הַשָּׁמַ֙יִם֙ וְתָגֵ֣ל הָאָ֔רֶץ
וְיֹאמְר֥וּ בַגּוֹיִ֖ם יְיָ֥ מָלָֽךְ׃ יִרְעַ֤ם הַיָּם֙ וּמְלוֹא֔וֹ
יַעֲלֹ֥ץ הַשָּׂדֶ֖ה וְכׇל־אֲשֶׁר־בּֽוֹ׃ אָ֥ז יְרַנְּנ֖וּ עֲצֵ֣י הַיָּ֑עַר
מִלִּפְנֵ֣י יְיָ֔ כִּי־בָ֖א לִשְׁפּ֥וֹט אֶת־הָאָֽרֶץ׃ הוֹד֤וּ לַֽייָ֙ כִּ֣י ט֔וֹב
כִּ֥י לְעוֹלָ֖ם חַסְדּֽוֹ׃ וְאִמְר֕וּ הוֹשִׁיעֵ֙נוּ֙ אֱלֹהֵ֣י יִשְׁעֵ֔נוּ
וְקַבְּצֵ֥נוּ וְהַצִּילֵ֖נוּ מִן־הַגּוֹיִ֑ם לְהֹדוֹת֙ לְשֵׁ֣ם קׇדְשֶׁ֔ךָ
לְהִשְׁתַּבֵּ֖חַ בִּתְהִלָּתֶֽךָ׃ בָּר֤וּךְ יְיָ֙ אֱלֹהֵ֣י יִשְׂרָאֵ֔ל
מִן־הָעוֹלָ֖ם וְעַ֣ד הָעֹלָ֑ם וַיֹּאמְר֤וּ כׇל־הָעָם֙ אָמֵ֔ן וְהַלֵּ֖ל לַייָ׃\\

רוֹמְמ֡וּ\source{תהילים צט} יְ֘יָ֤ אֱלֹהֵ֗ינוּ וְֽ֭הִשְׁתַּחֲווּ לַהֲדֹ֥ם רַגְלָ֗יו קָד֥וֹשׁ הֽוּא׃
רוֹמְמ֡וּ\source{תהילים צט} יְ֘יָ֤ אֱלֹהֵ֗ינוּ וְֽ֭הִשְׁתַּחֲווּ לְהַ֣ר קׇדְשׁ֑וֹ כִּי־קָ֝ד֗וֹשׁ יְיָ֥ אֱלֹהֵֽינוּ׃
\\
וְה֤וּא\source{תהילים עח} רַח֨וּם ׀ יְכַפֵּ֥ר עָוֺן֮ וְֽלֹא־יַֽ֫שְׁחִ֥ית וְ֭הִרְבָּה לְהָשִׁ֣יב אַפּ֑וֹ וְלֹא־יָ֝עִ֗יר כׇּל־חֲמָתֽוֹ׃
אַתָּ֤ה\source{תהילים מ} יְיָ֗ לֹֽא־תִכְלָ֣א רַחֲמֶ֣יךָ מִמֶּ֑נִּי חַסְדְּךָ֥ וַ֝אֲמִתְּךָ֗ תָּמִ֥יד יִצְּרֽוּנִי׃
זְכֹר־רַחֲמֶ֣יךָ\source{תהילים כה} יְיָ֭ וַחֲסָדֶ֑יךָ כִּ֖י מֵעוֹלָ֣ם הֵֽמָּה׃
תְּנ֥וּ\source{תהילים סח} עֹ֗ז לֵאלֹ֫הִ֥ים עַֽל־יִשְׂרָאֵ֥ל גַּאֲוָת֑וֹ וְ֝עֻזּ֗וֹ בַּשְּׁחָקִֽים׃ נ֤וֹרָ֥א אֱלֹהִ֗ים מִֽמִּקְדָּ֫שֶׁ֥יךָ אֵ֤ל יִשְׂרָאֵ֗ל ה֤וּא נֹתֵ֨ן ׀ עֹ֖ז וְתַעֲצֻמ֥וֹת לָעָ֗ם בָּר֥וּךְ אֱלֹהִֽים׃
אֵל־נְקָמ֥וֹת\source{תהילים צד} יְיָ֑ אֵ֖ל נְקָמ֣וֹת הוֹפִֽיעַ׃ הִ֭נָּשֵׂא שֹׁפֵ֣ט הָאָ֑רֶץ הָשֵׁ֥ב גְּ֝מ֗וּל עַל־גֵּאִֽים׃
לַֽייָ֥\source{תהילים ג} הַיְשׁוּעָ֑ה עַֽל־עַמְּךָ֖ בִרְכָתֶ֣ךָ סֶּֽלָה׃
יְיָ֣\source{תהילים מו} צְבָא֣וֹת עִמָּ֑נוּ מִשְׂגָּֽב־לָ֨נוּ אֱלֹהֵ֖י יַֽעֲקֹ֣ב סֶֽלָה׃
יְיָ֥\source{תהילים פד} צְבָא֑וֹת אַֽשְׁרֵ֥י אָ֝דָ֗ם בֹּטֵ֥חַ בָּֽךְ׃
יְיָ֥\source{תהילים כ} הוֹשִׁ֑יעָה הַ֝מֶּ֗לֶךְ יַעֲנֵ֥נוּ בְיוֹם־קׇרְאֵֽנוּ׃
\\
הוֹשִׁ֤יעָה\source{תהילים כח} ׀ אֶת־עַמֶּ֗ךָ וּבָרֵ֥ךְ אֶת־נַחֲלָתֶ֑ךָ וּֽרְעֵ֥ם וְ֝נַשְּׂאֵ֗ם עַד־הָעוֹלָֽם׃
נַ֭פְשֵׁנוּ\source{תהילים לג} חִכְּתָ֣ה לַֽייָ֑ עֶזְרֵ֖נוּ וּמָגִנֵּ֣נוּ הֽוּא׃ כִּי־ב֭וֹ יִשְׂמַ֣ח לִבֵּ֑נוּ כִּ֤י בְשֵׁ֖ם קׇדְשׁ֣וֹ בָטָֽחְנוּ׃ יְהִי־חַסְדְּךָ֣ יְיָ֣ עָלֵ֑ינוּ כַּ֝אֲשֶׁ֗ר יִחַ֥לְנוּ לָֽךְ׃
הַרְאֵ֣נוּ\source{תהילים פה} יְיָ֣ חַסְדֶּ֑ךָ וְ֝יֶשְׁעֲךָ֗ תִּתֶּן־לָֽנוּ׃
ק֭וּמָֽה\source{תהילים מד} עֶזְרָ֣תָה לָּ֑נוּ וּ֝פְדֵ֗נוּ לְמַ֣עַן חַסְדֶּֽךָ׃
אָֽנֹכִ֨י\source{תהילים פא} ׀ יְ֘יָ֤ אֱלֹהֶ֗יךָ הַֽ֭מַּעַלְךָ מֵאֶ֣רֶץ מִצְרָ֑יִם הַרְחֶב־פִּ֝֗יךָ וַאֲמַלְאֵֽהוּ׃
אַשְׁרֵ֣י\source{תהילים קמד} הָ֭עָם שֶׁכָּ֣כָה לּ֑וֹ אַֽשְׁרֵ֥י הָ֝עָ֗ם שֱׁייָ֥ אֱלֹהָֽיו׃
וַאֲנִ֤י\source{תהילים יג} ׀ בְּחַסְדְּךָ֣ בָטַחְתִּי֮ יָ֤גֵ֥ל לִבִּ֗י בִּֽישׁוּעָ֫תֶ֥ךָ אָשִׁ֥ירָה לַֽייָ֑ כִּ֖י גָמַ֣ל עָלָֽי׃

\negline

\ifboolexpr{togl {includeweekday} and (togl {includeshabbat} or togl {includefestival})}{ 
\begin{narrow}
\instruction{א״א מזמור לתודה בשבת, ביו״ט, בערב יום כפור, בערב פסח, בחול המועד פסח}
\todah
\end{narrow}
}{\ifboolexpr{togl {includeweekday} and not togl{includeshabbat} and not togl {includefestival}}{\todah}{}}

\ifboolexpr{(togl {includeshabbat} or togl {includefestival}) and (togl {includeweekday} or togl {includeChM})} {\instruction{בשבת וביו״ט ובהשענא רבא אומרים פסד״ז ארוך, בשאר ימים ממשיכים בעמ׳ \pageref{yehikvod}}
	\begin{narrow}	\longPDZ\end{narrow}} {
\ifboolexpr{(togl {includeshabbat} or togl {includefestival}) and not togl {includeweekday}} {\longPDZ} {}}


\label{yehikvod}
\firstword{יְהִ֤י כְב֣וֹד}\source{תהלים קד}
יְיָ֣ לְעוֹלָ֑ם יִשְׂמַ֖ח יְיָ֣ בְּמַעֲשָֽׂיו׃
\source{תהלים קיג}יְהִ֤י שֵׁ֣ם יְיָ֣ מְבֹרָ֑ךְ מֵ֝עַתָּ֗ה וְעַד־עוֹלָֽם׃
מִמִּזְרַח־שֶׁ֥מֶשׁ עַד־מְבוֹא֑וֹ מְ֝הֻלָּ֗ל שֵׁ֣ם יְיָ׃
רָ֖ם עַל־כׇּל־גּוֹיִ֥ם ׀ יְיָ֑ עַ֖ל הַשָּׁמַ֣יִם כְּבוֹדֽוֹ׃
\source{תהלים קלה} יְיָ֭ שִׁמְךָ֣ לְעוֹלָ֑ם יְ֝יָ֗ זִכְרְךָ֥ לְדֹר־וָדֹֽר׃
\source{תהלים קג} יְיָ֗ בַּ֭שָּׁמַיִם הֵכִ֣ין כִּסְא֑וֹ וּ֝מַלְכוּת֗וֹ בַּכֹּ֥ל מָשָֽׁלָה׃
\source{ד״ה א טז} יִשְׂמְח֤וּ הַשָּׁמַ֙יִם֙ וְתָגֵ֣ל הָאָ֔רֶץ וְיֹאמְר֥וּ בַגּוֹיִ֖ם יְיָ֥ מָלָֽךְ׃
\melekhmalakhyimlokh
\source{תהלים י} יְיָ֣ מֶ֭לֶךְ עוֹלָ֣ם וָעֶ֑ד אָבְד֥וּ ג֝וֹיִ֗ם מֵאַרְצֽוֹ׃
\source{תהלים לג} יְיָ֗ הֵפִ֥יר עֲצַת־גּוֹיִ֑ם הֵ֝נִ֗יא מַחְשְׁב֥וֹת עַמִּֽים׃
\source{משלי יט}רַבּ֣וֹת מַחֲשָׁב֣וֹת בְּלֶב־אִ֑ישׁ וַעֲצַ֥ת יְ֝יָ֗ הִ֣יא תָקֽוּם׃
\source{תהלים לג}עֲצַ֣ת יְיָ֭ לְעוֹלָ֣ם תַּעֲמֹ֑ד מַחְשְׁב֥וֹת לִ֝בּ֗וֹ לְדֹ֣ר וָדֹֽר׃
כִּ֤י ה֣וּא אָמַ֣ר וַיֶּ֑הִי הֽוּא־צִ֝וָּ֗ה וַֽיַּעֲמֹֽד׃
\source{תהלים קלב}כִּי־בָחַ֣ר יְיָ֣ בְּצִיּ֑וֹן אִ֝וָּ֗הּ לְמוֹשָׁ֥ב לֽוֹ׃
\source{תהלים קלה}כִּֽי־יַעֲקֹ֗ב בָּחַ֣ר ל֣וֹ יָ֑הּ יִ֝שְׂרָאֵ֗ל לִסְגֻלָּתֽוֹ׃
\source{תהלים צד}כִּ֤י ׀ לֹא־יִטֹּ֣שׁ יְיָ֣ עַמּ֑וֹ וְ֝נַחֲלָת֗וֹ לֹ֣א יַעֲזֹֽב׃
\source{תהלים עח}וְה֤וּא רַח֨וּם ׀ יְכַפֵּ֥ר עָוֺן֮ וְֽלֹא־יַֽ֫שְׁחִ֥ית וְ֭הִרְבָּה לְהָשִׁ֣יב אַפּ֑וֹ
וְלֹא־יָ֝עִ֗יר כׇּל־חֲמָתֽוֹ׃
\source{תהלים כ} יְיָ֥ הוֹשִׁ֑יעָה הַ֝מֶּ֗לֶךְ יַעֲנֵ֥נוּ בְיוֹם־קׇרְאֵֽנוּ׃

\ashrei
\enlargethispage{\baselineskip}

\firstword{הַֽלְלוּ־יָ֡הּ}\source{תהלים קמו}
הַֽלְלִ֥י נַ֝פְשִׁ֗י אֶת־יְיָ׃
אֲהַלְלָ֣ה יְיָ֣ בְּחַיָּ֑י אֲזַמְּרָ֖ה לֵאלֹהַ֣י בְּעוֹדִֽי׃
אַל־תִּבְטְח֥וּ בִנְדִיבִ֑ים בְּבֶן־אָדָ֓ם ׀ שֶׁ֤אֵ֖ין ל֥וֹ תְשׁוּעָֽה׃
תֵּצֵ֣א ר֭וּחוֹ יָשֻׁ֣ב לְאַדְמָת֑וֹ בַּיּ֥וֹם הַ֝ה֗וּא אָבְד֥וּ עֶשְׁתֹּֽנֹתָֽיו׃
אַשְׁרֵ֗י שֶׁ֤אֵ֣ל יַעֲקֹ֣ב בְּעֶזְר֑וֹ שִׂ֝בְר֗וֹ עַל־יְיָ֥ אֱלֹהָֽיו׃
עֹשֶׂ֤ה ׀ שָׁ֘מַ֤יִם וָאָ֗רֶץ אֶת־הַיָּ֥ם וְאֶת־כׇּל־אֲשֶׁר־בָּ֑ם הַשֹּׁמֵ֖ר אֱמֶ֣ת לְעוֹלָֽם׃
עֹשֶׂ֤ה מִשְׁפָּ֨ט ׀ לָעֲשׁוּקִ֗ים נֹתֵ֣ן לֶ֭חֶם לָרְעֵבִ֑ים יְ֝יָ֗ מַתִּ֥יר אֲסוּרִֽים׃
יְיָ֤ ׀ פֹּ֘קֵ֤חַ עִוְרִ֗ים יְיָ֭ זֹקֵ֣ף כְּפוּפִ֑ים יְ֝יָ֗ אֹהֵ֥ב צַדִּיקִֽים׃
יְיָ֤ ׀ שֹׁ֘מֵ֤ר אֶת־גֵּרִ֗ים יָת֣וֹם וְאַלְמָנָ֣ה יְעוֹדֵ֑ד וְדֶ֖רֶךְ רְשָׁעִ֣ים יְעַוֵּֽת׃
יִמְלֹ֤ךְ יְיָ֨ ׀ לְעוֹלָ֗ם\\ אֱלֹהַ֣יִךְ צִ֭יּוֹן לְדֹ֥ר וָדֹ֗ר הַֽלְלוּ־יָֽהּ׃



\firstword{הַ֥לְלוּ יָ֨הּ} ׀\source{תהלים קמז}
כִּי־ט֭וֹב זַמְּרָ֣ה אֱלֹהֵ֑ינוּ כִּי־נָ֝עִ֗ים נָאוָ֥ה תְהִלָּֽה׃
בּוֹנֵ֣ה יְרֽוּשָׁלַ֣‍ִם יְיָ֑ נִדְחֵ֖י יִשְׂרָאֵ֣ל יְכַנֵּֽס׃
הָ֭רֹפֵא לִשְׁב֣וּרֵי לֵ֑ב וּ֝מְחַבֵּ֗שׁ לְעַצְּבוֹתָֽם׃
מוֹנֶ֣ה מִ֭סְפָּר לַכּוֹכָבִ֑ים לְ֝כֻלָּ֗ם שֵׁמ֥וֹת יִקְרָֽא׃
גָּד֣וֹל אֲדוֹנֵ֣ינוּ וְרַב־כֹּ֑חַ לִ֝תְבוּנָת֗וֹ אֵ֣ין מִסְפָּֽר׃
מְעוֹדֵ֣ד עֲנָוִ֣ים יְיָ֑ מַשְׁפִּ֖יל רְשָׁעִ֣ים עֲדֵי־אָֽרֶץ׃
עֱנ֣וּ לַֽייָ֣ בְּתוֹדָ֑ה זַמְּר֖וּ לֵאלֹהֵ֣ינוּ בְכִנּֽוֹר׃
הַֽמְכַסֶּ֬ה שָׁמַ֨יִם ׀ בְּעָבִ֗ים הַמֵּכִ֣ין לָאָ֣רֶץ מָטָ֑ר הַמַּצְמִ֖יחַ הָרִ֣ים חָצִֽיר׃
נוֹתֵ֣ן לִבְהֵמָ֣ה לַחְמָ֑הּ לִבְנֵ֥י עֹ֝רֵ֗ב אֲשֶׁ֣ר יִקְרָֽאוּ׃
לֹ֤א בִגְבוּרַ֣ת הַסּ֣וּס יֶחְפָּ֑ץ לֹא־בְשׁוֹקֵ֖י הָאִ֣ישׁ יִרְצֶֽה׃
רוֹצֶ֣ה יְיָ֭ אֶת־יְרֵאָ֑יו אֶת־הַֽמְיַחֲלִ֥ים לְחַסְדּֽוֹ׃
שַׁבְּחִ֣י יְ֭רוּשָׁלַ‍ִם אֶת־יְיָ֑ הַֽלְלִ֖י אֱלֹהַ֣יִךְ צִיּֽוֹן׃
כִּֽי־חִ֭זַּק בְּרִיחֵ֣י שְׁעָרָ֑יִךְ בֵּרַ֖ךְ בָּנַ֣יִךְ בְּקִרְבֵּֽךְ׃
הַשָּׂם־גְּבוּלֵ֥ךְ שָׁל֑וֹם חֵ֥לֶב חִ֝טִּ֗ים יַשְׂבִּיעֵֽךְ׃
הַשֹּׁלֵ֣חַ אִמְרָת֣וֹ אָ֑רֶץ עַד־מְ֝הֵרָ֗ה יָר֥וּץ דְּבָרֽוֹ׃
הַנֹּתֵ֣ן שֶׁ֣לֶג כַּצָּ֑מֶר כְּ֝פ֗וֹר כָּאֵ֥פֶר יְפַזֵּֽר׃
מַשְׁלִ֣יךְ קַֽרְח֣וֹ כְפִתִּ֑ים לִפְנֵ֥י קָ֝רָת֗וֹ מִ֣י יַעֲמֹֽד׃
יִשְׁלַ֣ח דְּבָר֣וֹ וְיַמְסֵ֑ם יַשֵּׁ֥ב ר֝וּח֗וֹ יִזְּלוּ־מָֽיִם׃
מַגִּ֣יד דְּבָרָ֣ו לְיַעֲקֹ֑ב חֻקָּ֥יו וּ֝מִשְׁפָּטָ֗יו לְיִשְׂרָאֵֽל׃
לֹ֘א עָ֤שָׂה כֵ֨ן ׀ לְכׇל־גּ֗וֹי וּמִשְׁפָּטִ֥ים בַּל־יְדָע֗וּם הַֽלְלוּ־יָֽהּ׃


\firstword{הַ֥לְלוּ יָ֨הּ} ׀\source{תהלים קמח}
הַֽלְל֣וּ אֶת־יְיָ֭ מִן־הַשָּׁמַ֑יִם הַֽ֝לְל֗וּהוּ בַּמְּרוֹמִֽים׃
הַֽלְל֥וּהוּ כׇל־מַלְאָכָ֑יו הַ֝לְל֗וּהוּ כׇּל־צְבָאָֽו׃
הַֽ֭לְלוּהוּ שֶׁ֣מֶשׁ וְיָרֵ֑חַ הַֽ֝לְל֗וּהוּ כׇּל־כּ֥וֹכְבֵי אֽוֹר׃
הַֽ֭לְלוּהוּ שְׁמֵ֣י הַשָּׁמָ֑יִם וְ֝הַמַּ֗יִם אֲשֶׁ֤ר ׀ מֵעַ֬ל הַשָּׁמָֽיִם׃
יְֽ֭הַלְלוּ אֶת־שֵׁ֣ם יְיָ֑ כִּ֤י ה֖וּא צִוָּ֣ה וְנִבְרָֽאוּ׃
וַיַּעֲמִידֵ֣ם לָעַ֣ד לְעוֹלָ֑ם חׇק־נָ֝תַ֗ן וְלֹ֣א יַעֲבֽוֹר׃
הַֽלְל֣וּ אֶת־יְיָ֭ מִן־הָאָ֑רֶץ תַּ֝נִּינִ֗ים וְכׇל־תְּהֹמֽוֹת׃
אֵ֣שׁ וּ֭בָרָד שֶׁ֣לֶג וְקִיט֑וֹר ר֥וּחַ סְ֝עָרָ֗ה עֹשָׂ֥ה דְבָרֽוֹ׃
הֶהָרִ֥ים וְכׇל־גְּבָע֑וֹת עֵ֥ץ פְּ֝רִ֗י וְכׇל־אֲרָזִֽים׃
הַחַיָּ֥ה וְכׇל־בְּהֵמָ֑ה רֶ֗֝מֶשׂ וְצִפּ֥וֹר כָּנָֽף׃
מַלְכֵי־אֶ֭רֶץ וְכׇל־לְאֻמִּ֑ים שָׂ֝רִ֗ים וְכׇל־שֹׁ֥פְטֵי אָֽרֶץ׃
בַּחוּרִ֥ים וְגַם־בְּתוּל֑וֹת זְ֝קֵנִ֗ים עִם־נְעָרִֽים׃
יְהַלְל֤וּ ׀ אֶת־שֵׁ֬ם יְיָ֗ כִּֽי־נִשְׂגָּ֣ב שְׁמ֣וֹ לְבַדּ֑וֹ
ה֝וֹד֗וֹ עַל־אֶ֥רֶץ וְשָׁמָֽיִם׃ וַיָּ֤רֶם קֶ֨רֶן ׀ לְעַמּ֡וֹ תְּהִלָּ֤ה לְֽכׇל־חֲסִידָ֗יו
לִבְנֵ֣י יִ֭שְׂרָאֵל עַ֥ם קְרֹב֗וֹ הַֽלְלוּ־יָֽהּ׃\\
\firstword{הַ֥לְלוּ יָ֨הּ} ׀\source{תהלים קמט}
שִׁ֣ירוּ לַֽייָ֭ שִׁ֣יר חָדָ֑שׁ תְּ֝הִלָּת֗וֹ בִּקְהַ֥ל חֲסִידִֽים׃
יִשְׂמַ֣ח יִשְׂרָאֵ֣ל בְּעֹשָׂ֑יו בְּנֵֽי־צִ֝יּ֗וֹן יָגִ֥ילוּ בְמַלְכָּֽם׃
יְהַלְל֣וּ שְׁמ֣וֹ בְמָח֑וֹל בְּתֹ֥ף וְ֝כִנּ֗וֹר יְזַמְּרוּ־לֽוֹ׃
כִּֽי־רוֹצֶ֣ה יְיָ֣ בְּעַמּ֑וֹ יְפָאֵ֥ר עֲ֝נָוִ֗ים בִּישׁוּעָֽה׃
יַעְלְז֣וּ חֲסִידִ֣ים בְּכָב֑וֹד יְ֝רַנְּנ֗וּ עַל־מִשְׁכְּבוֹתָֽם׃
רוֹמְמ֣וֹת אֵ֭ל בִּגְרוֹנָ֑ם וְחֶ֖רֶב פִּיפִיּ֣וֹת בְּיָדָֽם׃
לַעֲשׂ֣וֹת נְ֭קָמָה בַּגּוֹיִ֑ם תּ֝וֹכֵח֗וֹת בַּלְאֻמִּֽים׃
לֶאְסֹ֣ר מַלְכֵיהֶ֣ם בְּזִקִּ֑ים וְ֝נִכְבְּדֵיהֶ֗ם בְּכַבְלֵ֥י בַרְזֶֽל׃
לַעֲשׂ֤וֹת בָּהֶ֨ם ׀ מִשְׁפָּ֬ט כָּת֗וּב הָדָ֣ר ה֭וּא לְכׇל־חֲסִידָ֗יו הַֽלְלוּ־יָֽהּ׃

\firstword{הַ֥לְלוּ יָ֨הּ} ׀\source{תהלים קנ}
הַֽלְלוּ־אֵ֥ל בְּקׇדְשׁ֑וֹ הַֽ֝לְל֗וּהוּ בִּרְקִ֥יעַ עֻזּֽוֹ׃
הַלְל֥וּהוּ בִגְבוּרֹתָ֑יו הַ֝לְל֗וּהוּ כְּרֹ֣ב גֻּדְלֽוֹ׃
הַ֭לְלוּהוּ בְּתֵ֣קַע שׁוֹפָ֑ר הַ֝לְל֗וּהוּ בְּנֵ֣בֶל וְכִנּֽוֹר׃
הַ֭לְלוּהוּ בְּתֹ֣ף וּמָח֑וֹל הַֽ֝לְל֗וּהוּ בְּמִנִּ֥ים וְעֻגָֽב׃
הַלְל֥וּהוּ בְצִלְצְלֵי־שָׁ֑מַע הַֽ֝לְל֗וּהוּ בְּֽצִלְצְלֵ֥י תְרוּעָֽה׃
כֹּ֣ל הַ֭נְּשָׁמָה תְּהַלֵּ֥ל יָ֗הּ הַֽלְלוּ־יָֽהּ׃
\scriptsize{כֹּ֣ל הַ֭נְּשָׁמָה תְּהַלֵּ֥ל יָ֗הּ הַֽלְלוּ־יָֽהּ׃ \\}
\normalsize{}

\negline

\firstword{בָּר֖וּךְ}\source{תהלים פט}
יְיָ֥ לְ֝עוֹלָ֗ם אָ֘מֵ֥ן ׀ וְאָמֵֽן׃ \hfill \break
\source{תהלים קלה}בָּ֘ר֤וּךְ יְיָ֨ ׀ מִצִּיּ֗וֹן שֹׁ֘כֵ֤ן יְֽרוּשָׁלָ֗‍ִם הַֽלְלוּ־יָֽהּ׃ \hfill \break
\source{תהלים עב}בָּר֤וּךְ ׀ יְיָ֣ אֱ֭לֹהִים אֱלֹהֵ֣י יִשְׂרָאֵ֑ל עֹשֵׂ֖ה נִפְלָא֣וֹת לְבַדּֽוֹ׃ וּבָר֤וּךְ ׀ שֵׁ֥ם כְּבוֹד֗וֹ לְע֫וֹלָ֥ם וְיִמָּלֵ֣א כְ֭בוֹדוֹ אֶת־כֹּ֥ל הָאָ֗רֶץ אָ֘מֵ֥ן ׀ וְאָמֵֽן׃





\firstword{וַיְבָ֤רֶךְ}\source{דה״א כט}
דָּוִיד֙ אֶת־יְיָ֔ לְעֵינֵ֖י כׇּל־הַקָּהָ֑ל וַיֹּ֣אמֶר דָּוִ֗יד בָּר֨וּךְ אַתָּ֤ה יְיָ֙ אֱלֹהֵי֙ יִשְׂרָאֵ֣ל אָבִ֔ינוּ מֵעוֹלָ֖ם וְעַד־עוֹלָֽם׃
לְךָ֣ יְ֠יָ֠ הַגְּדֻלָּ֨ה וְהַגְּבוּרָ֤ה וְהַתִּפְאֶ֙רֶת֙ וְהַנֵּ֣צַח וְהַה֔וֹד כִּי־כֹ֖ל בַּשָּׁמַ֣יִם וּבָאָ֑רֶץ לְךָ֤ יְיָ֙ הַמַּמְלָכָ֔ה וְהַמִּתְנַשֵּׂ֖א לְכֹ֥ל ׀ לְרֹֽאשׁ׃
וְהָעֹ֤שֶׁר וְהַכָּבוֹד֙ מִלְּפָנֶ֔יךָ וְאַתָּה֙ מוֹשֵׁ֣ל בַּכֹּ֔ל וּבְיָדְךָ֖ כֹּ֣חַ וּגְבוּרָ֑ה וּבְיָ֣דְךָ֔ לְגַדֵּ֥ל וּלְחַזֵּ֖ק לַכֹּֽל׃
וְעַתָּ֣ה אֱלֹהֵ֔ינוּ מוֹדִ֥ים אֲנַ֖חְנוּ לָ֑ךְ וּֽמְהַלְלִ֖ים לְשֵׁ֥ם תִּפְאַרְתֶּֽךָ׃


אַתָּה־ה֣וּא\source{נחמיה ט}
יְיָ לְבַדֶּ֒ךָ֒ אַתָּ֣ עָשִׂ֡יתָ אֶֽת־הַשָּׁמַ֩יִם֩ שְׁמֵ֨י הַשָּׁמַ֜יִם וְכׇל־צְבָאָ֗ם הָאָ֜רֶץ וְכׇל־אֲשֶׁ֤ר עָלֶ֙יהָ֙ הַיַּמִּים֙ וְכׇל־אֲשֶׁ֣ר בָּהֶ֔ם וְאַתָּ֖ה מְחַיֶּ֣ה אֶת־כֻּלָּ֑ם וּצְבָ֥א הַשָּׁמַ֖יִם לְךָ֥ מִשְׁתַּחֲוִֽים׃
אַתָּה־הוּא֙ יְיָ֣ הָאֱלֹהִ֔ים אֲשֶׁ֤ר בָּחַ֙רְתָּ֙ בְּאַבְרָ֔ם וְהוֹצֵאת֖וֹ מֵא֣וּר כַּשְׂדִּ֑ים וְשַׂ֥מְתָּ שְּׁמ֖וֹ אַבְרָהָֽם׃ וּמָצָ֣אתָ אֶת־לְבָבוֹ֮ נֶאֱמָ֣ן לְפָנֶ֒יךָ֒

וְכָר֨וֹת עִמּ֜וֹ הַבְּרִ֗ית לָתֵ֡ת אֶת־אֶ֩רֶץ֩ הַכְּנַעֲנִ֨י הַחִתִּ֜י הָאֱמֹרִ֧י וְהַפְּרִזִּ֛י וְהַיְבוּסִ֥י וְהַגִּרְגָּשִׁ֖י לָתֵ֣ת לְזַרְע֑וֹ וַתָּ֙קֶם֙ אֶת־דְּבָרֶ֔יךָ כִּ֥י צַדִּ֖יק אָֽתָּה׃ וַתֵּ֛רֶא אֶת־עֳנִ֥י אֲבֹתֵ֖ינוּ בְּמִצְרָ֑יִם וְאֶת־זַעֲקָתָ֥ם שָׁמַ֖עְתָּ עַל־יַם־סֽוּף׃ וַ֠תִּתֵּ֠ן אֹתֹ֨ת וּמֹֽפְתִ֜ים בְּפַרְעֹ֤ה וּבְכׇל־עֲבָדָיו֙ וּבְכׇל־עַ֣ם אַרְצ֔וֹ כִּ֣י יָדַ֔עְתָּ כִּ֥י הֵזִ֖ידוּ עֲלֵיהֶ֑ם וַתַּֽעַשׂ־לְךָ֥ שֵׁ֖ם כְּהַיּ֥וֹם הַזֶּֽה׃
וְהַיָּם֙ בָּקַ֣עְתָּ לִפְנֵיהֶ֔ם וַיַּֽעַבְר֥וּ בְתוֹךְ־הַיָּ֖ם בַּיַּבָּשָׁ֑ה וְֽאֶת־רֹ֨דְפֵיהֶ֜ם הִשְׁלַ֧כְתָּ בִמְצוֹלֹ֛ת כְּמוֹ־אֶ֖בֶן בְּמַ֥יִם עַזִּֽים׃

וַיּ֨וֹשַׁע\source{שמות יד}
יְיָ֜ בַּיּ֥וֹם הַה֛וּא אֶת־יִשְׂרָאֵ֖ל מִיַּ֣ד מִצְרָ֑יִם וַיַּ֤רְא יִשְׂרָאֵל֙ אֶת־מִצְרַ֔יִם מֵ֖ת עַל־שְׂפַ֥ת הַיָּֽם׃
וַיַּ֨רְא יִשְׂרָאֵ֜ל אֶת־הַיָּ֣ד הַגְּדֹלָ֗ה אֲשֶׁ֨ר עָשָׂ֤ה יְיָ֙ בְּמִצְרַ֔יִם וַיִּֽירְא֥וּ הָעָ֖ם אֶת־יְיָ֑ וַיַּֽאֲמִ֙ינוּ֙ בַּייָ֔ וּבְמֹשֶׁ֖ה עַבְדּֽוֹ׃

אָ֣ז\source{שמות טו} \hfill
יָשִֽׁיר־מֹשֶׁה֩ \hfill וּבְנֵ֨י \hfill יִשְׂרָאֵ֜ל \hfill אֶת־הַשִּׁירָ֤ה \hfill הַזֹּאת֙ \hfill לַֽייָ֔ \hfill וַיֹּאמְר֖וּ \\
לֵאמֹ֑ר \hfill אָשִׁ֤ירָה לַֽייָ֙ כִּֽי־גָאֹ֣ה גָּאָ֔ה \hfill ס֥וּס \\
וְרֹכְב֖וֹ רָמָ֥ה בַיָּֽם׃ \hfill עׇזִּ֤י וְזִמְרָת֙ יָ֔הּ וַֽיְהִי־לִ֖י \\
לִֽישׁוּעָ֑ה \hfill זֶ֤ה אֵלִי֙ וְאַנְוֵ֔הוּ \hfill אֱלֹהֵ֥י \\
אָבִ֖י וַאֲרֹמְמֶֽנְהוּ׃ \hfill יְיָ֖ אִ֣ישׁ מִלְחָמָ֑ה יְיָ֖ \\
שְׁמֽוֹ׃ \hfill מַרְכְּבֹ֥ת פַּרְעֹ֛ה וְחֵיל֖וֹ יָרָ֣ה בַיָּ֑ם \hfill וּמִבְחַ֥ר\\
שָֽׁלִשָׁ֖יו טֻבְּע֥וּ בְיַם־סֽוּף׃ \hfill תְּהֹמֹ֖ת יְכַסְיֻ֑מוּ יָרְד֥וּ בִמְצוֹלֹ֖ת כְּמוֹ־\\
אָֽבֶן׃ \hfill יְמִֽינְךָ֣ יְיָ֔ נֶאְדָּרִ֖י בַּכֹּ֑חַ \hfill יְמִֽינְךָ֥ \\
יְיָ֖ תִּרְעַ֥ץ אוֹיֵֽב׃ \hfill וּבְרֹ֥ב גְּאוֹנְךָ֖ תַּהֲרֹ֣ס \\
קָמֶ֑יךָ \hfill תְּשַׁלַּח֙ חֲרֹ֣נְךָ֔ יֹאכְלֵ֖מוֹ כַּקַּֽשׁ׃ \hfill וּבְר֤וּחַ \\
אַפֶּ֙יךָ֙ נֶ֣עֶרְמוּ מַ֔יִם \hfill נִצְּב֥וּ כְמוֹ־נֵ֖ד \\
נֹזְלִ֑ים \hfill קָֽפְא֥וּ תְהֹמֹ֖ת בְּלֶב־יָֽם׃ \hfill אָמַ֥ר \\
אוֹיֵ֛ב אֶרְדֹּ֥ף אַשִּׂ֖יג \hfill אֲחַלֵּ֣ק שָׁלָ֑ל תִּמְלָאֵ֣מוֹ \\
נַפְשִׁ֔י \hfill אָרִ֣יק חַרְבִּ֔י תּוֹרִישֵׁ֖מוֹ יָדִֽי׃ \hfill נָשַׁ֥פְתָּ \\
בְרוּחֲךָ֖ כִּסָּ֣מוֹ יָ֑ם \hfill צָֽלְלוּ֙ כַּֽעוֹפֶ֔רֶת בְּמַ֖יִם \\
אַדִּירִֽים׃ \hfill מִֽי־כָמֹ֤כָה בָּֽאֵלִם֙ יְיָ֔ \hfill מִ֥י \\
כָּמֹ֖כָה נֶאְדָּ֣ר בַּקֹּ֑דֶשׁ \hfill נוֹרָ֥א תְהִלֹּ֖ת עֹ֥שֵׂה \\
פֶֽלֶא׃ \hfill נָטִ֙יתָ֙ יְמִ֣ינְךָ֔ תִּבְלָעֵ֖מוֹ אָֽרֶץ׃ \hfill נָחִ֥יתָ \\
בְחַסְדְּךָ֖ עַם־ז֣וּ גָּאָ֑לְתָּ \hfill נֵהַ֥לְתָּ בְעׇזְּךָ֖ אֶל־נְוֵ֥ה \\
קׇדְשֶֽׁךָ׃ \hfill שָֽׁמְע֥וּ עַמִּ֖ים יִרְגָּז֑וּן \hfill חִ֣יל \\
אָחַ֔ז יֹשְׁבֵ֖י פְּלָֽשֶׁת׃ \hfill אָ֤ז נִבְהֲלוּ֙ אַלּוּפֵ֣י \\
אֱד֔וֹם \hfill אֵילֵ֣י מוֹאָ֔ב יֹֽאחֲזֵ֖מוֹ רָ֑עַד \hfill נָמֹ֕גוּ \\
כֹּ֖ל יֹשְׁבֵ֥י כְנָֽעַן׃ \hfill תִּפֹּ֨ל עֲלֵיהֶ֤ם אֵימָ֙תָה֙ \\
וָפַ֔חַד \hfill בִּגְדֹ֥ל זְרוֹעֲךָ֖ יִדְּמ֣וּ כָּאָ֑בֶן \hfill עַד־\\
יַעֲבֹ֤ר עַמְּךָ֙ יְיָ֔ \hfill עַֽד־יַעֲבֹ֖ר עַם־ז֥וּ \\
קָנִֽיתָ׃ \hfill תְּבִאֵ֗מוֹ וְתִטָּעֵ֙מוֹ֙ בְּהַ֣ר נַחֲלָֽתְךָ֔ \hfill מָכ֧וֹן \\
לְשִׁבְתְּךָ֛ פָּעַ֖לְתָּ יְיָ֑ \hfill מִקְּדָ֕שׁ אֲדֹנָ֖י כּוֹנְנ֥וּ \\
יָדֶֽיךָ׃ \hfill יְיָ֥ ׀ יִמְלֹ֖ךְ לְעֹלָ֥ם וָעֶֽד׃ \begin{footnotesize}יְיָ֥ ׀ יִמְלֹ֖ךְ לְעֹלָ֥ם וָעֶֽד׃\end{footnotesize}
\hfill \begin{small} כִּ֣י \\
	בָא֩ ס֨וּס פַּרְעֹ֜ה בְּרִכְבּ֤וֹ וּבְפָרָשָׁיו֙ בַּיָּ֔ם \hfill וַיָּ֧שֶׁב יְיָ֛ עֲלֵהֶ֖ם \\
	אֶת־מֵ֣י הַיָּ֑ם \hfill וּבְנֵ֧י יִשְׂרָאֵ֛ל הָלְכ֥וּ בַיַּבָּשָׁ֖ה בְּת֥וֹךְ\hfill הַיָּֽם׃\\
	וַתִּקַּח֩ מִרְיָ֨ם הַנְּבִיאָ֜ה אֲח֧וֹת אַהֲרֹ֛ן אֶת־הַתֹּ֖ף בְּיָדָ֑הּ וַתֵּצֶ֤אןָ כׇֽל־הַנָּשִׁים֙ אַחֲרֶ֔יהָ בְּתֻפִּ֖ים וּבִמְחֹלֹֽת׃ וַתַּ֥עַן לָהֶ֖ם מִרְיָ֑ם שִׁ֤ירוּ לַֽייָ֙ כִּֽי־גָאֹ֣ה גָּאָ֔ה ס֥וּס וְרֹכְב֖וֹ רָמָ֥ה בַיָּֽם׃\hfill\break 
\end{small}


כִּ֣י \source{תהלים כב}לַ֭ייָ֭ הַמְּלוּכָ֑ה וּ֝מֹשֵׁ֗ל בַּגּוֹיִֽם׃
וְעָל֤וּ \source{עובדיה א}מֽוֹשִׁעִים֙ בְּהַ֣ר צִיּ֔וֹן לִשְׁפֹּ֖ט אֶת־הַ֣ר עֵשָׂ֑ו וְהָיְתָ֥ה לַֽייָ֖ הַמְּלוּכָֽה׃
וְהָיָ֧ה \source{זכריה יד}יְיָ֛ לְמֶ֖לֶךְ עַל־כׇּל־הָאָ֑רֶץ בַּיּ֣וֹם הַה֗וּא יִהְיֶ֧ה יְיָ֛ אֶחָ֖ד וּשְׁמ֥וֹ אֶחָֽד׃

\ifboolexpr{togl {includeweekday} and not togl {includeshabbat} and not togl {includefestival}} {\yishtabach
		
		\mimaamakim
		
		\halfkaddish
		
		\enlargethispage{\baselineskip}
		
		\vspace{1.25\baselineskip}}{\ifboolexpr{togl {includeweekday} and (togl {includeshabbat} or togl {includefestival}}{
		
\instruction{בשבת וביו״ט ממשיכים בעמ׳ \pageref{nishmas}}\\

\yishtabach

\mimaamakim

\halfkaddish

\enlargethispage{\baselineskip}

\vspace{1.25\baselineskip}}}


%\chapter[פסוקי דזמרא]{\adforn{47} פסוקי דזמרא \adforn{19}}

\chanukat

\mournerskaddish

\firstword{בָּרוּךְ שֶׁאָמַר}
וְהָיָה הָעוֹלָם בָּרוּךְ הוּא׃
בָּרוּךְ עוֹשֶׂה בְרֵאשִׁית בָּרוּךְ אוֹמֵר וְעוֹשֶׂה׃
בָּרוּךְ גּוֹזֵר וּמְקַיֵּם בָּרוּךְ מְרַחֵם עַל הָאָֽרֶץ׃
בָּרוּךְ מְרַחֵם עַל הַבְּרִיּוֹת בָּרוּךְ מְשַׁלֵּם שָׂכָר טוֹב לִירֵאָיו׃
בָּרוּךְ חַי לָעַד וְקַיָּם לָנֶֽצַח בָּרוּךְ פּוֹדֶה וּמַצִּיל בָּרוּךְ שְׁמוֹ׃
בָּרוּךְ אַתָּה יְיָ אֱלֹהֵֽינוּ מֶֽלֶךְ הָעוֹלָם הָאֵל אָב הָרַחֲמָן הַמְהֻלָּל בְּפִי עַמּוֹ מְשֻׁבָּח וּמְפֹאָר בִּלְשׁוֹן חֲסִידָיו וַעֲבָדָיו וּבְשִׁירֵי דָוִד עַבְדֶּֽךָ נְהַלֶּלְךָ יְיָ אֱלֹהֵֽינוּ בִּשְׁבָחוֹת וּבִזְמִירוֹת׃ נְגַדֶּלְךָ וּנְשַׁבֵּחֲךָ וּנְפָאֶרְךָ וְנַמְלִיכְךָ וְנַזְכִּיר שִׁמְךָ מַלְכֵּֽנוּ אֱלֹהֵֽינוּ׃
יָחִיד חֵי הָעוֹלָמִים מֶֽלֶךְ מְשֻׁבָּח וּמְפֹאָר עֲדֵי עַד שְׁמוֹ הַגָּדוֹל׃ בָּרוּךְ אַתָּה יְיָ מֶֽלֶךְ מְהֻלָּל בַּתֻּשְׁבָּחוֹת׃

\firstword{הוֹד֤וּ}
לַֽייָ֙ קִרְא֣וּ בִשְׁמ֔וֹ\source{דה״א טז}
הוֹדִ֥יעוּ בָעַמִּ֖ים עֲלִילֹתָֽיו׃
שִׁ֤ירוּ לוֹ֙ זַמְּרוּ־ל֔וֹ שִׂ֖יחוּ בְּכׇל־נִפְלְאֹתָֽיו׃
הִֽתְהַלְלוּ֙ בְּשֵׁ֣ם קׇדְשׁ֔וֹ יִשְׂמַ֕ח לֵ֖ב מְבַקְשֵׁ֥י יְיָ׃
דִּרְשׁ֤וּ יְיָ֙ וְעֻזּ֔וֹ בַּקְּשׁ֥וּ פָנָ֖יו תָּמִֽיד׃
זִכְר֗וּ נִפְלְאֹתָיו֙ אֲשֶׁ֣ר עָשָׂ֔ה מֹפְתָ֖יו וּמִשְׁפְּטֵי־פִֽיהוּ׃
זֶ֚רַע יִשְׂרָאֵ֣ל עַבְדּ֔וֹ בְּנֵ֥י יַעֲקֹ֖ב בְּחִירָֽיו׃
ה֚וּא יְיָ֣ אֱלֹהֵ֔ינוּ בְּכׇל־הָאָ֖רֶץ מִשְׁפָּטָֽיו׃
זִכְר֤וּ לְעוֹלָם֙ בְּרִית֔וֹ דָּבָ֥ר צִוָּ֖ה לְאֶ֥לֶף דּֽוֹר׃
אֲשֶׁ֤ר כָּרַת֙ אֶת־אַבְרָהָ֔ם וּשְׁבוּעָת֖וֹ לְיִצְחָֽק׃
וַיַּעֲמִידֶ֤הָ לְיַֽעֲקֹב֙ לְחֹ֔ק לְיִשְׂרָאֵ֖ל בְּרִ֥ית עוֹלָֽם׃
לֵאמֹ֗ר לְךָ֙ אֶתֵּ֣ן אֶֽרֶץ־כְּנָ֔עַן חֶ֖בֶל נַחֲלַתְכֶֽם׃
בִּהְיֽוֹתְכֶם֙ מְתֵ֣י מִסְפָּ֔ר כִּמְעַ֖ט וְגָרִ֥ים בָּֽהּ׃
וַיִּֽתְהַלְּכוּ֙ מִגּ֣וֹי אֶל־גּ֔וֹי וּמִמַּמְלָכָ֖ה אֶל־עַ֥ם אַחֵֽר׃
לֹֽא־הִנִּ֤יחַ לְאִישׁ֙ לְעׇשְׁקָ֔ם וַיּ֥וֹכַח עֲלֵיהֶ֖ם מְלָכִֽים׃
אַֽל־תִּגְּעוּ֙ בִּמְשִׁיחָ֔י וּבִנְבִיאַ֖י אַל־תָּרֵֽעוּ׃
שִׁ֤ירוּ לַֽייָ֙ כׇּל־הָאָ֔רֶץ בַּשְּׂר֥וּ מִיּֽוֹם־אֶל־י֖וֹם יְשׁוּעָתֽוֹ׃
סַפְּר֤וּ בַגּוֹיִם֙ אֶת־כְּבוֹד֔וֹ בְּכׇל־הָעַמִּ֖ים נִפְלְאֹתָֽיו׃
כִּי֩ גָד֨וֹל יְיָ֤ וּמְהֻלָּל֙ מְאֹ֔ד וְנוֹרָ֥א ה֖וּא עַל־כׇּל־אֱלֹהִֽים׃
כִּ֠י כׇּל־אֱלֹהֵ֤י הָֽעַמִּים֙ אֱלִילִ֔ים...וַייָ֖ שָׁמַ֥יִם עָשָֽׂה׃

ה֤וֹד וְהָדָר֙ לְפָנָ֔יו עֹ֥ז וְחֶדְוָ֖ה בִּמְקֹמֽוֹ׃
הָב֤וּ לַֽייָ֙ מִשְׁפְּח֣וֹת עַמִּ֔ים הָב֥וּ לַייָ֖ כָּב֥וֹד וָעֹֽז׃
הָב֥וּ לַֽייָ֖ כְּב֣וֹד שְׁמ֑וֹ שְׂא֤וּ מִנְחָה֙ וּבֹ֣אוּ לְפָנָ֔יו
הִשְׁתַּחֲו֥וּ לַֽייָ֖ בְּהַדְרַת־קֹֽדֶשׁ׃ חִ֤ילוּ מִלְּפָנָיו֙ כׇּל־הָאָ֔רֶץ
אַף־תִּכּ֥וֹן תֵּבֵ֖ל בַּל־תִּמּֽוֹט׃ יִשְׂמְח֤וּ הַשָּׁמַ֙יִם֙ וְתָגֵ֣ל הָאָ֔רֶץ
וְיֹאמְר֥וּ בַגּוֹיִ֖ם יְיָ֥ מָלָֽךְ׃ יִרְעַ֤ם הַיָּם֙ וּמְלוֹא֔וֹ
יַעֲלֹ֥ץ הַשָּׂדֶ֖ה וְכׇל־אֲשֶׁר־בּֽוֹ׃ אָ֥ז יְרַנְּנ֖וּ עֲצֵ֣י הַיָּ֑עַר
מִלִּפְנֵ֣י יְיָ֔ כִּי־בָ֖א לִשְׁפּ֥וֹט אֶת־הָאָֽרֶץ׃ הוֹד֤וּ לַֽייָ֙ כִּ֣י ט֔וֹב
כִּ֥י לְעוֹלָ֖ם חַסְדּֽוֹ׃ וְאִמְר֕וּ הוֹשִׁיעֵ֙נוּ֙ אֱלֹהֵ֣י יִשְׁעֵ֔נוּ
וְקַבְּצֵ֥נוּ וְהַצִּילֵ֖נוּ מִן־הַגּוֹיִ֑ם לְהֹדוֹת֙ לְשֵׁ֣ם קׇדְשֶׁ֔ךָ
לְהִשְׁתַּבֵּ֖חַ בִּתְהִלָּתֶֽךָ׃ בָּר֤וּךְ יְיָ֙ אֱלֹהֵ֣י יִשְׂרָאֵ֔ל
מִן־הָעוֹלָ֖ם וְעַ֣ד הָעֹלָ֑ם וַיֹּאמְר֤וּ כׇל־הָעָם֙ אָמֵ֔ן וְהַלֵּ֖ל לַייָ׃\\

רוֹמְמ֡וּ\source{תהילים צט} יְ֘יָ֤ אֱלֹהֵ֗ינוּ וְֽ֭הִשְׁתַּחֲווּ לַהֲדֹ֥ם רַגְלָ֗יו קָד֥וֹשׁ הֽוּא׃
רוֹמְמ֡וּ\source{תהילים צט} יְ֘יָ֤ אֱלֹהֵ֗ינוּ וְֽ֭הִשְׁתַּחֲווּ לְהַ֣ר קׇדְשׁ֑וֹ כִּי־קָ֝ד֗וֹשׁ יְיָ֥ אֱלֹהֵֽינוּ׃
\\
וְה֤וּא\source{תהילים עח} רַח֨וּם ׀ יְכַפֵּ֥ר עָוֺן֮ וְֽלֹא־יַֽ֫שְׁחִ֥ית וְ֭הִרְבָּה לְהָשִׁ֣יב אַפּ֑וֹ וְלֹא־יָ֝עִ֗יר כׇּל־חֲמָתֽוֹ׃
אַתָּ֤ה\source{תהילים מ} יְיָ֗ לֹֽא־תִכְלָ֣א רַחֲמֶ֣יךָ מִמֶּ֑נִּי חַסְדְּךָ֥ וַ֝אֲמִתְּךָ֗ תָּמִ֥יד יִצְּרֽוּנִי׃
זְכֹר־רַחֲמֶ֣יךָ\source{תהילים כה} יְיָ֭ וַחֲסָדֶ֑יךָ כִּ֖י מֵעוֹלָ֣ם הֵֽמָּה׃
תְּנ֥וּ\source{תהילים סח} עֹ֗ז לֵאלֹ֫הִ֥ים עַֽל־יִשְׂרָאֵ֥ל גַּאֲוָת֑וֹ וְ֝עֻזּ֗וֹ בַּשְּׁחָקִֽים׃ נ֤וֹרָ֥א אֱלֹהִ֗ים מִֽמִּקְדָּ֫שֶׁ֥יךָ אֵ֤ל יִשְׂרָאֵ֗ל ה֤וּא נֹתֵ֨ן ׀ עֹ֖ז וְתַעֲצֻמ֥וֹת לָעָ֗ם בָּר֥וּךְ אֱלֹהִֽים׃
אֵל־נְקָמ֥וֹת\source{תהילים צד} יְיָ֑ אֵ֖ל נְקָמ֣וֹת הוֹפִֽיעַ׃ הִ֭נָּשֵׂא שֹׁפֵ֣ט הָאָ֑רֶץ הָשֵׁ֥ב גְּ֝מ֗וּל עַל־גֵּאִֽים׃
לַֽייָ֥\source{תהילים ג} הַיְשׁוּעָ֑ה עַֽל־עַמְּךָ֖ בִרְכָתֶ֣ךָ סֶּֽלָה׃
יְיָ֣\source{תהילים מו} צְבָא֣וֹת עִמָּ֑נוּ מִשְׂגָּֽב־לָ֨נוּ אֱלֹהֵ֖י יַֽעֲקֹ֣ב סֶֽלָה׃
יְיָ֥\source{תהילים פד} צְבָא֑וֹת אַֽשְׁרֵ֥י אָ֝דָ֗ם בֹּטֵ֥חַ בָּֽךְ׃
יְיָ֥\source{תהילים כ} הוֹשִׁ֑יעָה הַ֝מֶּ֗לֶךְ יַעֲנֵ֥נוּ בְיוֹם־קׇרְאֵֽנוּ׃
\\
הוֹשִׁ֤יעָה\source{תהילים כח} ׀ אֶת־עַמֶּ֗ךָ וּבָרֵ֥ךְ אֶת־נַחֲלָתֶ֑ךָ וּֽרְעֵ֥ם וְ֝נַשְּׂאֵ֗ם עַד־הָעוֹלָֽם׃
נַ֭פְשֵׁנוּ\source{תהילים לג} חִכְּתָ֣ה לַֽייָ֑ עֶזְרֵ֖נוּ וּמָגִנֵּ֣נוּ הֽוּא׃ כִּי־ב֭וֹ יִשְׂמַ֣ח לִבֵּ֑נוּ כִּ֤י בְשֵׁ֖ם קׇדְשׁ֣וֹ בָטָֽחְנוּ׃ יְהִי־חַסְדְּךָ֣ יְיָ֣ עָלֵ֑ינוּ כַּ֝אֲשֶׁ֗ר יִחַ֥לְנוּ לָֽךְ׃
הַרְאֵ֣נוּ\source{תהילים פה} יְיָ֣ חַסְדֶּ֑ךָ וְ֝יֶשְׁעֲךָ֗ תִּתֶּן־לָֽנוּ׃
ק֭וּמָֽה\source{תהילים מד} עֶזְרָ֣תָה לָּ֑נוּ וּ֝פְדֵ֗נוּ לְמַ֣עַן חַסְדֶּֽךָ׃
אָֽנֹכִ֨י\source{תהילים פא} ׀ יְ֘יָ֤ אֱלֹהֶ֗יךָ הַֽ֭מַּעַלְךָ מֵאֶ֣רֶץ מִצְרָ֑יִם הַרְחֶב־פִּ֝֗יךָ וַאֲמַלְאֵֽהוּ׃
אַשְׁרֵ֣י\source{תהילים קמד} הָ֭עָם שֶׁכָּ֣כָה לּ֑וֹ אַֽשְׁרֵ֥י הָ֝עָ֗ם שֱׁייָ֥ אֱלֹהָֽיו׃
וַאֲנִ֤י\source{תהילים יג} ׀ בְּחַסְדְּךָ֣ בָטַחְתִּי֮ יָ֤גֵ֥ל לִבִּ֗י בִּֽישׁוּעָ֫תֶ֥ךָ אָשִׁ֥ירָה לַֽייָ֑ כִּ֖י גָמַ֣ל עָלָֽי׃

\negline

\instruction{אין אומרים מזמור לתודה בערב יום כפור ובערב פסח}\\
מִזְמ֥וֹר\source{תהילים ק} לְתוֹדָ֑ה הָרִ֥יעוּ לַ֝ייָ֗ כׇּל־הָאָֽרֶץ׃ עִבְד֣וּ אֶת־יְיָ֣ בְּשִׂמְחָ֑ה בֹּ֥אוּ לְ֝פָנָ֗יו בִּרְנָנָֽה׃ דְּע֗וּ כִּֽי־יְיָ ה֤וּא אֱלֹ֫הִ֥ים הֽוּא־עָ֭שָׂנוּ (ולא) [וְל֣וֹ] אֲנַ֑חְנוּ עַ֝מּ֗וֹ וְצֹ֣אן מַרְעִיתֽוֹ׃ בֹּ֤אוּ שְׁעָרָ֨יו ׀ בְּתוֹדָ֗ה חֲצֵרֹתָ֥יו בִּתְהִלָּ֑ה הוֹדוּ־ל֝֗וֹ בָּרְכ֥וּ שְׁמֽוֹ׃ כִּי־ט֣וֹב יְיָ֭ לְעוֹלָ֣ם חַסְדּ֑וֹ וְעַד־דֹּ֥ר וָ֝דֹ֗ר אֱמוּנָתֽוֹ׃


\label{yehikvod}
\firstword{יְהִ֤י כְב֣וֹד}\source{תהלים קד}
יְיָ֣ לְעוֹלָ֑ם יִשְׂמַ֖ח יְיָ֣ בְּמַעֲשָֽׂיו׃
\source{תהלים קיג}יְהִ֤י שֵׁ֣ם יְיָ֣ מְבֹרָ֑ךְ מֵ֝עַתָּ֗ה וְעַד־עוֹלָֽם׃
מִמִּזְרַח־שֶׁ֥מֶשׁ עַד־מְבוֹא֑וֹ מְ֝הֻלָּ֗ל שֵׁ֣ם יְיָ׃
רָ֖ם עַל־כׇּל־גּוֹיִ֥ם ׀ יְיָ֑ עַ֖ל הַשָּׁמַ֣יִם כְּבוֹדֽוֹ׃
\source{תהלים קלה} יְיָ֭ שִׁמְךָ֣ לְעוֹלָ֑ם יְ֝יָ֗ זִכְרְךָ֥ לְדֹר־וָדֹֽר׃
\source{תהלים קג} יְיָ֗ בַּ֭שָּׁמַיִם הֵכִ֣ין כִּסְא֑וֹ וּ֝מַלְכוּת֗וֹ בַּכֹּ֥ל מָשָֽׁלָה׃
\source{ד״ה א טז} יִשְׂמְח֤וּ הַשָּׁמַ֙יִם֙ וְתָגֵ֣ל הָאָ֔רֶץ וְיֹאמְר֥וּ בַגּוֹיִ֖ם יְיָ֥ מָלָֽךְ׃
\melekhmalakhyimlokh
\source{תהלים י} יְיָ֣ מֶ֭לֶךְ עוֹלָ֣ם וָעֶ֑ד אָבְד֥וּ ג֝וֹיִ֗ם מֵאַרְצֽוֹ׃
\source{תהלים לג} יְיָ֗ הֵפִ֥יר עֲצַת־גּוֹיִ֑ם הֵ֝נִ֗יא מַחְשְׁב֥וֹת עַמִּֽים׃
\source{משלי יט}רַבּ֣וֹת מַחֲשָׁב֣וֹת בְּלֶב־אִ֑ישׁ וַעֲצַ֥ת יְ֝יָ֗ הִ֣יא תָקֽוּם׃
\source{תהלים לג}עֲצַ֣ת יְיָ֭ לְעוֹלָ֣ם תַּעֲמֹ֑ד מַחְשְׁב֥וֹת לִ֝בּ֗וֹ לְדֹ֣ר וָדֹֽר׃
כִּ֤י ה֣וּא אָמַ֣ר וַיֶּ֑הִי הֽוּא־צִ֝וָּ֗ה וַֽיַּעֲמֹֽד׃
\source{תהלים קלב}כִּי־בָחַ֣ר יְיָ֣ בְּצִיּ֑וֹן אִ֝וָּ֗הּ לְמוֹשָׁ֥ב לֽוֹ׃
\source{תהלים קלה}כִּֽי־יַעֲקֹ֗ב בָּחַ֣ר ל֣וֹ יָ֑הּ יִ֝שְׂרָאֵ֗ל לִסְגֻלָּתֽוֹ׃
\source{תהלים צד}כִּ֤י ׀ לֹא־יִטֹּ֣שׁ יְיָ֣ עַמּ֑וֹ וְ֝נַחֲלָת֗וֹ לֹ֣א יַעֲזֹֽב׃
\source{תהלים עח}וְה֤וּא רַח֨וּם ׀ יְכַפֵּ֥ר עָוֺן֮ וְֽלֹא־יַֽ֫שְׁחִ֥ית וְ֭הִרְבָּה לְהָשִׁ֣יב אַפּ֑וֹ
וְלֹא־יָ֝עִ֗יר כׇּל־חֲמָתֽוֹ׃
\source{תהלים כ} יְיָ֥ הוֹשִׁ֑יעָה הַ֝מֶּ֗לֶךְ יַעֲנֵ֥נוּ בְיוֹם־קׇרְאֵֽנוּ׃

\ashrei
\enlargethispage{\baselineskip}

\firstword{הַֽלְלוּ־יָ֡הּ}\source{תהלים קמו}
הַֽלְלִ֥י נַ֝פְשִׁ֗י אֶת־יְיָ׃
אֲהַלְלָ֣ה יְיָ֣ בְּחַיָּ֑י אֲזַמְּרָ֖ה לֵאלֹהַ֣י בְּעוֹדִֽי׃
אַל־תִּבְטְח֥וּ בִנְדִיבִ֑ים בְּבֶן־אָדָ֓ם ׀ שֶׁ֤אֵ֖ין ל֥וֹ תְשׁוּעָֽה׃
תֵּצֵ֣א ר֭וּחוֹ יָשֻׁ֣ב לְאַדְמָת֑וֹ בַּיּ֥וֹם הַ֝ה֗וּא אָבְד֥וּ עֶשְׁתֹּֽנֹתָֽיו׃
אַשְׁרֵ֗י שֶׁ֤אֵ֣ל יַעֲקֹ֣ב בְּעֶזְר֑וֹ שִׂ֝בְר֗וֹ עַל־יְיָ֥ אֱלֹהָֽיו׃
עֹשֶׂ֤ה ׀ שָׁ֘מַ֤יִם וָאָ֗רֶץ אֶת־הַיָּ֥ם וְאֶת־כׇּל־אֲשֶׁר־בָּ֑ם הַשֹּׁמֵ֖ר אֱמֶ֣ת לְעוֹלָֽם׃
עֹשֶׂ֤ה מִשְׁפָּ֨ט ׀ לָעֲשׁוּקִ֗ים נֹתֵ֣ן לֶ֭חֶם לָרְעֵבִ֑ים יְ֝יָ֗ מַתִּ֥יר אֲסוּרִֽים׃
יְיָ֤ ׀ פֹּ֘קֵ֤חַ עִוְרִ֗ים יְיָ֭ זֹקֵ֣ף כְּפוּפִ֑ים יְ֝יָ֗ אֹהֵ֥ב צַדִּיקִֽים׃
יְיָ֤ ׀ שֹׁ֘מֵ֤ר אֶת־גֵּרִ֗ים יָת֣וֹם וְאַלְמָנָ֣ה יְעוֹדֵ֑ד וְדֶ֖רֶךְ רְשָׁעִ֣ים יְעַוֵּֽת׃
יִמְלֹ֤ךְ יְיָ֨ ׀ לְעוֹלָ֗ם\\ אֱלֹהַ֣יִךְ צִ֭יּוֹן לְדֹ֥ר וָדֹ֗ר הַֽלְלוּ־יָֽהּ׃



\firstword{הַ֥לְלוּ יָ֨הּ} ׀\source{תהלים קמז}
כִּי־ט֭וֹב זַמְּרָ֣ה אֱלֹהֵ֑ינוּ כִּי־נָ֝עִ֗ים נָאוָ֥ה תְהִלָּֽה׃
בּוֹנֵ֣ה יְרֽוּשָׁלַ֣‍ִם יְיָ֑ נִדְחֵ֖י יִשְׂרָאֵ֣ל יְכַנֵּֽס׃
הָ֭רֹפֵא לִשְׁב֣וּרֵי לֵ֑ב וּ֝מְחַבֵּ֗שׁ לְעַצְּבוֹתָֽם׃
מוֹנֶ֣ה מִ֭סְפָּר לַכּוֹכָבִ֑ים לְ֝כֻלָּ֗ם שֵׁמ֥וֹת יִקְרָֽא׃
גָּד֣וֹל אֲדוֹנֵ֣ינוּ וְרַב־כֹּ֑חַ לִ֝תְבוּנָת֗וֹ אֵ֣ין מִסְפָּֽר׃
מְעוֹדֵ֣ד עֲנָוִ֣ים יְיָ֑ מַשְׁפִּ֖יל רְשָׁעִ֣ים עֲדֵי־אָֽרֶץ׃
עֱנ֣וּ לַֽייָ֣ בְּתוֹדָ֑ה זַמְּר֖וּ לֵאלֹהֵ֣ינוּ בְכִנּֽוֹר׃
הַֽמְכַסֶּ֬ה שָׁמַ֨יִם ׀ בְּעָבִ֗ים הַמֵּכִ֣ין לָאָ֣רֶץ מָטָ֑ר הַמַּצְמִ֖יחַ הָרִ֣ים חָצִֽיר׃
נוֹתֵ֣ן לִבְהֵמָ֣ה לַחְמָ֑הּ לִבְנֵ֥י עֹ֝רֵ֗ב אֲשֶׁ֣ר יִקְרָֽאוּ׃
לֹ֤א בִגְבוּרַ֣ת הַסּ֣וּס יֶחְפָּ֑ץ לֹא־בְשׁוֹקֵ֖י הָאִ֣ישׁ יִרְצֶֽה׃
רוֹצֶ֣ה יְיָ֭ אֶת־יְרֵאָ֑יו אֶת־הַֽמְיַחֲלִ֥ים לְחַסְדּֽוֹ׃
שַׁבְּחִ֣י יְ֭רוּשָׁלַ‍ִם אֶת־יְיָ֑ הַֽלְלִ֖י אֱלֹהַ֣יִךְ צִיּֽוֹן׃
כִּֽי־חִ֭זַּק בְּרִיחֵ֣י שְׁעָרָ֑יִךְ בֵּרַ֖ךְ בָּנַ֣יִךְ בְּקִרְבֵּֽךְ׃
הַשָּׂם־גְּבוּלֵ֥ךְ שָׁל֑וֹם חֵ֥לֶב חִ֝טִּ֗ים יַשְׂבִּיעֵֽךְ׃
הַשֹּׁלֵ֣חַ אִמְרָת֣וֹ אָ֑רֶץ עַד־מְ֝הֵרָ֗ה יָר֥וּץ דְּבָרֽוֹ׃
הַנֹּתֵ֣ן שֶׁ֣לֶג כַּצָּ֑מֶר כְּ֝פ֗וֹר כָּאֵ֥פֶר יְפַזֵּֽר׃
מַשְׁלִ֣יךְ קַֽרְח֣וֹ כְפִתִּ֑ים לִפְנֵ֥י קָ֝רָת֗וֹ מִ֣י יַעֲמֹֽד׃
יִשְׁלַ֣ח דְּבָר֣וֹ וְיַמְסֵ֑ם יַשֵּׁ֥ב ר֝וּח֗וֹ יִזְּלוּ־מָֽיִם׃
מַגִּ֣יד דְּבָרָ֣ו לְיַעֲקֹ֑ב חֻקָּ֥יו וּ֝מִשְׁפָּטָ֗יו לְיִשְׂרָאֵֽל׃
לֹ֘א עָ֤שָׂה כֵ֨ן ׀ לְכׇל־גּ֗וֹי וּמִשְׁפָּטִ֥ים בַּל־יְדָע֗וּם הַֽלְלוּ־יָֽהּ׃


\firstword{הַ֥לְלוּ יָ֨הּ} ׀\source{תהלים קמח}
הַֽלְל֣וּ אֶת־יְיָ֭ מִן־הַשָּׁמַ֑יִם הַֽ֝לְל֗וּהוּ בַּמְּרוֹמִֽים׃
הַֽלְל֥וּהוּ כׇל־מַלְאָכָ֑יו הַ֝לְל֗וּהוּ כׇּל־צְבָאָֽו׃
הַֽ֭לְלוּהוּ שֶׁ֣מֶשׁ וְיָרֵ֑חַ הַֽ֝לְל֗וּהוּ כׇּל־כּ֥וֹכְבֵי אֽוֹר׃
הַֽ֭לְלוּהוּ שְׁמֵ֣י הַשָּׁמָ֑יִם וְ֝הַמַּ֗יִם אֲשֶׁ֤ר ׀ מֵעַ֬ל הַשָּׁמָֽיִם׃
יְֽ֭הַלְלוּ אֶת־שֵׁ֣ם יְיָ֑ כִּ֤י ה֖וּא צִוָּ֣ה וְנִבְרָֽאוּ׃
וַיַּעֲמִידֵ֣ם לָעַ֣ד לְעוֹלָ֑ם חׇק־נָ֝תַ֗ן וְלֹ֣א יַעֲבֽוֹר׃
הַֽלְל֣וּ אֶת־יְיָ֭ מִן־הָאָ֑רֶץ תַּ֝נִּינִ֗ים וְכׇל־תְּהֹמֽוֹת׃
אֵ֣שׁ וּ֭בָרָד שֶׁ֣לֶג וְקִיט֑וֹר ר֥וּחַ סְ֝עָרָ֗ה עֹשָׂ֥ה דְבָרֽוֹ׃
הֶהָרִ֥ים וְכׇל־גְּבָע֑וֹת עֵ֥ץ פְּ֝רִ֗י וְכׇל־אֲרָזִֽים׃
הַחַיָּ֥ה וְכׇל־בְּהֵמָ֑ה רֶ֗֝מֶשׂ וְצִפּ֥וֹר כָּנָֽף׃
מַלְכֵי־אֶ֭רֶץ וְכׇל־לְאֻמִּ֑ים שָׂ֝רִ֗ים וְכׇל־שֹׁ֥פְטֵי אָֽרֶץ׃
בַּחוּרִ֥ים וְגַם־בְּתוּל֑וֹת זְ֝קֵנִ֗ים עִם־נְעָרִֽים׃
יְהַלְל֤וּ ׀ אֶת־שֵׁ֬ם יְיָ֗ כִּֽי־נִשְׂגָּ֣ב שְׁמ֣וֹ לְבַדּ֑וֹ
ה֝וֹד֗וֹ עַל־אֶ֥רֶץ וְשָׁמָֽיִם׃ וַיָּ֤רֶם קֶ֨רֶן ׀ לְעַמּ֡וֹ תְּהִלָּ֤ה לְֽכׇל־חֲסִידָ֗יו
לִבְנֵ֣י יִ֭שְׂרָאֵל עַ֥ם קְרֹב֗וֹ הַֽלְלוּ־יָֽהּ׃\\
\firstword{הַ֥לְלוּ יָ֨הּ} ׀\source{תהלים קמט}
שִׁ֣ירוּ לַֽייָ֭ שִׁ֣יר חָדָ֑שׁ תְּ֝הִלָּת֗וֹ בִּקְהַ֥ל חֲסִידִֽים׃
יִשְׂמַ֣ח יִשְׂרָאֵ֣ל בְּעֹשָׂ֑יו בְּנֵֽי־צִ֝יּ֗וֹן יָגִ֥ילוּ בְמַלְכָּֽם׃
יְהַלְל֣וּ שְׁמ֣וֹ בְמָח֑וֹל בְּתֹ֥ף וְ֝כִנּ֗וֹר יְזַמְּרוּ־לֽוֹ׃
כִּֽי־רוֹצֶ֣ה יְיָ֣ בְּעַמּ֑וֹ יְפָאֵ֥ר עֲ֝נָוִ֗ים בִּישׁוּעָֽה׃
יַעְלְז֣וּ חֲסִידִ֣ים בְּכָב֑וֹד יְ֝רַנְּנ֗וּ עַל־מִשְׁכְּבוֹתָֽם׃
רוֹמְמ֣וֹת אֵ֭ל בִּגְרוֹנָ֑ם וְחֶ֖רֶב פִּיפִיּ֣וֹת בְּיָדָֽם׃
לַעֲשׂ֣וֹת נְ֭קָמָה בַּגּוֹיִ֑ם תּ֝וֹכֵח֗וֹת בַּלְאֻמִּֽים׃
לֶאְסֹ֣ר מַלְכֵיהֶ֣ם בְּזִקִּ֑ים וְ֝נִכְבְּדֵיהֶ֗ם בְּכַבְלֵ֥י בַרְזֶֽל׃
לַעֲשׂ֤וֹת בָּהֶ֨ם ׀ מִשְׁפָּ֬ט כָּת֗וּב הָדָ֣ר ה֭וּא לְכׇל־חֲסִידָ֗יו הַֽלְלוּ־יָֽהּ׃

\firstword{הַ֥לְלוּ יָ֨הּ} ׀\source{תהלים קנ}
הַֽלְלוּ־אֵ֥ל בְּקׇדְשׁ֑וֹ הַֽ֝לְל֗וּהוּ בִּרְקִ֥יעַ עֻזּֽוֹ׃
הַלְל֥וּהוּ בִגְבוּרֹתָ֑יו הַ֝לְל֗וּהוּ כְּרֹ֣ב גֻּדְלֽוֹ׃
הַ֭לְלוּהוּ בְּתֵ֣קַע שׁוֹפָ֑ר הַ֝לְל֗וּהוּ בְּנֵ֣בֶל וְכִנּֽוֹר׃
הַ֭לְלוּהוּ בְּתֹ֣ף וּמָח֑וֹל הַֽ֝לְל֗וּהוּ בְּמִנִּ֥ים וְעֻגָֽב׃
הַלְל֥וּהוּ בְצִלְצְלֵי־שָׁ֑מַע הַֽ֝לְל֗וּהוּ בְּֽצִלְצְלֵ֥י תְרוּעָֽה׃
כֹּ֣ל הַ֭נְּשָׁמָה תְּהַלֵּ֥ל יָ֗הּ הַֽלְלוּ־יָֽהּ׃
\scriptsize{כֹּ֣ל הַ֭נְּשָׁמָה תְּהַלֵּ֥ל יָ֗הּ הַֽלְלוּ־יָֽהּ׃ \\}
\normalsize{}

\negline

\firstword{בָּר֖וּךְ}\source{תהלים פט}
יְיָ֥ לְ֝עוֹלָ֗ם אָ֘מֵ֥ן ׀ וְאָמֵֽן׃ \hfill \break
\source{תהלים קלה}בָּ֘ר֤וּךְ יְיָ֨ ׀ מִצִּיּ֗וֹן שֹׁ֘כֵ֤ן יְֽרוּשָׁלָ֗‍ִם הַֽלְלוּ־יָֽהּ׃ \hfill \break
\source{תהלים עב}בָּר֤וּךְ ׀ יְיָ֣ אֱ֭לֹהִים אֱלֹהֵ֣י יִשְׂרָאֵ֑ל עֹשֵׂ֖ה נִפְלָא֣וֹת לְבַדּֽוֹ׃ וּבָר֤וּךְ ׀ שֵׁ֥ם כְּבוֹד֗וֹ לְע֫וֹלָ֥ם וְיִמָּלֵ֣א כְ֭בוֹדוֹ אֶת־כֹּ֥ל הָאָ֗רֶץ אָ֘מֵ֥ן ׀ וְאָמֵֽן׃





\firstword{וַיְבָ֤רֶךְ}\source{דה״א כט}
דָּוִיד֙ אֶת־יְיָ֔ לְעֵינֵ֖י כׇּל־הַקָּהָ֑ל וַיֹּ֣אמֶר דָּוִ֗יד בָּר֨וּךְ אַתָּ֤ה יְיָ֙ אֱלֹהֵי֙ יִשְׂרָאֵ֣ל אָבִ֔ינוּ מֵעוֹלָ֖ם וְעַד־עוֹלָֽם׃
לְךָ֣ יְ֠יָ֠ הַגְּדֻלָּ֨ה וְהַגְּבוּרָ֤ה וְהַתִּפְאֶ֙רֶת֙ וְהַנֵּ֣צַח וְהַה֔וֹד כִּי־כֹ֖ל בַּשָּׁמַ֣יִם וּבָאָ֑רֶץ לְךָ֤ יְיָ֙ הַמַּמְלָכָ֔ה וְהַמִּתְנַשֵּׂ֖א לְכֹ֥ל ׀ לְרֹֽאשׁ׃
וְהָעֹ֤שֶׁר וְהַכָּבוֹד֙ מִלְּפָנֶ֔יךָ וְאַתָּה֙ מוֹשֵׁ֣ל בַּכֹּ֔ל וּבְיָדְךָ֖ כֹּ֣חַ וּגְבוּרָ֑ה וּבְיָ֣דְךָ֔ לְגַדֵּ֥ל וּלְחַזֵּ֖ק לַכֹּֽל׃
וְעַתָּ֣ה אֱלֹהֵ֔ינוּ מוֹדִ֥ים אֲנַ֖חְנוּ לָ֑ךְ וּֽמְהַלְלִ֖ים לְשֵׁ֥ם תִּפְאַרְתֶּֽךָ׃


אַתָּה־ה֣וּא\source{נחמיה ט}
יְיָ לְבַדֶּ֒ךָ֒ אַתָּ֣ עָשִׂ֡יתָ אֶֽת־הַשָּׁמַ֩יִם֩ שְׁמֵ֨י הַשָּׁמַ֜יִם וְכׇל־צְבָאָ֗ם הָאָ֜רֶץ וְכׇל־אֲשֶׁ֤ר עָלֶ֙יהָ֙ הַיַּמִּים֙ וְכׇל־אֲשֶׁ֣ר בָּהֶ֔ם וְאַתָּ֖ה מְחַיֶּ֣ה אֶת־כֻּלָּ֑ם וּצְבָ֥א הַשָּׁמַ֖יִם לְךָ֥ מִשְׁתַּחֲוִֽים׃
אַתָּה־הוּא֙ יְיָ֣ הָאֱלֹהִ֔ים אֲשֶׁ֤ר בָּחַ֙רְתָּ֙ בְּאַבְרָ֔ם וְהוֹצֵאת֖וֹ מֵא֣וּר כַּשְׂדִּ֑ים וְשַׂ֥מְתָּ שְּׁמ֖וֹ אַבְרָהָֽם׃ וּמָצָ֣אתָ אֶת־לְבָבוֹ֮ נֶאֱמָ֣ן לְפָנֶ֒יךָ֒

וְכָר֨וֹת עִמּ֜וֹ הַבְּרִ֗ית לָתֵ֡ת אֶת־אֶ֩רֶץ֩ הַכְּנַעֲנִ֨י הַחִתִּ֜י הָאֱמֹרִ֧י וְהַפְּרִזִּ֛י וְהַיְבוּסִ֥י וְהַגִּרְגָּשִׁ֖י לָתֵ֣ת לְזַרְע֑וֹ וַתָּ֙קֶם֙ אֶת־דְּבָרֶ֔יךָ כִּ֥י צַדִּ֖יק אָֽתָּה׃ וַתֵּ֛רֶא אֶת־עֳנִ֥י אֲבֹתֵ֖ינוּ בְּמִצְרָ֑יִם וְאֶת־זַעֲקָתָ֥ם שָׁמַ֖עְתָּ עַל־יַם־סֽוּף׃ וַ֠תִּתֵּ֠ן אֹתֹ֨ת וּמֹֽפְתִ֜ים בְּפַרְעֹ֤ה וּבְכׇל־עֲבָדָיו֙ וּבְכׇל־עַ֣ם אַרְצ֔וֹ כִּ֣י יָדַ֔עְתָּ כִּ֥י הֵזִ֖ידוּ עֲלֵיהֶ֑ם וַתַּֽעַשׂ־לְךָ֥ שֵׁ֖ם כְּהַיּ֥וֹם הַזֶּֽה׃
וְהַיָּם֙ בָּקַ֣עְתָּ לִפְנֵיהֶ֔ם וַיַּֽעַבְר֥וּ בְתוֹךְ־הַיָּ֖ם בַּיַּבָּשָׁ֑ה וְֽאֶת־רֹ֨דְפֵיהֶ֜ם הִשְׁלַ֧כְתָּ בִמְצוֹלֹ֛ת כְּמוֹ־אֶ֖בֶן בְּמַ֥יִם עַזִּֽים׃

וַיּ֨וֹשַׁע\source{שמות יד}
יְיָ֜ בַּיּ֥וֹם הַה֛וּא אֶת־יִשְׂרָאֵ֖ל מִיַּ֣ד מִצְרָ֑יִם וַיַּ֤רְא יִשְׂרָאֵל֙ אֶת־מִצְרַ֔יִם מֵ֖ת עַל־שְׂפַ֥ת הַיָּֽם׃
וַיַּ֨רְא יִשְׂרָאֵ֜ל אֶת־הַיָּ֣ד הַגְּדֹלָ֗ה אֲשֶׁ֨ר עָשָׂ֤ה יְיָ֙ בְּמִצְרַ֔יִם וַיִּֽירְא֥וּ הָעָ֖ם אֶת־יְיָ֑ וַיַּֽאֲמִ֙ינוּ֙ בַּייָ֔ וּבְמֹשֶׁ֖ה עַבְדּֽוֹ׃

אָ֣ז\source{שמות טו} \hfill
יָשִֽׁיר־מֹשֶׁה֩ \hfill וּבְנֵ֨י \hfill יִשְׂרָאֵ֜ל \hfill אֶת־הַשִּׁירָ֤ה \hfill הַזֹּאת֙ \hfill לַֽייָ֔ \hfill וַיֹּאמְר֖וּ \\
לֵאמֹ֑ר \hfill אָשִׁ֤ירָה לַֽייָ֙ כִּֽי־גָאֹ֣ה גָּאָ֔ה \hfill ס֥וּס \\
וְרֹכְב֖וֹ רָמָ֥ה בַיָּֽם׃ \hfill עׇזִּ֤י וְזִמְרָת֙ יָ֔הּ וַֽיְהִי־לִ֖י \\
לִֽישׁוּעָ֑ה \hfill זֶ֤ה אֵלִי֙ וְאַנְוֵ֔הוּ \hfill אֱלֹהֵ֥י \\
אָבִ֖י וַאֲרֹמְמֶֽנְהוּ׃ \hfill יְיָ֖ אִ֣ישׁ מִלְחָמָ֑ה יְיָ֖ \\
שְׁמֽוֹ׃ \hfill מַרְכְּבֹ֥ת פַּרְעֹ֛ה וְחֵיל֖וֹ יָרָ֣ה בַיָּ֑ם \hfill וּמִבְחַ֥ר\\
שָֽׁלִשָׁ֖יו טֻבְּע֥וּ בְיַם־סֽוּף׃ \hfill תְּהֹמֹ֖ת יְכַסְיֻ֑מוּ יָרְד֥וּ בִמְצוֹלֹ֖ת כְּמוֹ־\\
אָֽבֶן׃ \hfill יְמִֽינְךָ֣ יְיָ֔ נֶאְדָּרִ֖י בַּכֹּ֑חַ \hfill יְמִֽינְךָ֥ \\
יְיָ֖ תִּרְעַ֥ץ אוֹיֵֽב׃ \hfill וּבְרֹ֥ב גְּאוֹנְךָ֖ תַּהֲרֹ֣ס \\
קָמֶ֑יךָ \hfill תְּשַׁלַּח֙ חֲרֹ֣נְךָ֔ יֹאכְלֵ֖מוֹ כַּקַּֽשׁ׃ \hfill וּבְר֤וּחַ \\
אַפֶּ֙יךָ֙ נֶ֣עֶרְמוּ מַ֔יִם \hfill נִצְּב֥וּ כְמוֹ־נֵ֖ד \\
נֹזְלִ֑ים \hfill קָֽפְא֥וּ תְהֹמֹ֖ת בְּלֶב־יָֽם׃ \hfill אָמַ֥ר \\
אוֹיֵ֛ב אֶרְדֹּ֥ף אַשִּׂ֖יג \hfill אֲחַלֵּ֣ק שָׁלָ֑ל תִּמְלָאֵ֣מוֹ \\
נַפְשִׁ֔י \hfill אָרִ֣יק חַרְבִּ֔י תּוֹרִישֵׁ֖מוֹ יָדִֽי׃ \hfill נָשַׁ֥פְתָּ \\
בְרוּחֲךָ֖ כִּסָּ֣מוֹ יָ֑ם \hfill צָֽלְלוּ֙ כַּֽעוֹפֶ֔רֶת בְּמַ֖יִם \\
אַדִּירִֽים׃ \hfill מִֽי־כָמֹ֤כָה בָּֽאֵלִם֙ יְיָ֔ \hfill מִ֥י \\
כָּמֹ֖כָה נֶאְדָּ֣ר בַּקֹּ֑דֶשׁ \hfill נוֹרָ֥א תְהִלֹּ֖ת עֹ֥שֵׂה \\
פֶֽלֶא׃ \hfill נָטִ֙יתָ֙ יְמִ֣ינְךָ֔ תִּבְלָעֵ֖מוֹ אָֽרֶץ׃ \hfill נָחִ֥יתָ \\
בְחַסְדְּךָ֖ עַם־ז֣וּ גָּאָ֑לְתָּ \hfill נֵהַ֥לְתָּ בְעׇזְּךָ֖ אֶל־נְוֵ֥ה \\
קׇדְשֶֽׁךָ׃ \hfill שָֽׁמְע֥וּ עַמִּ֖ים יִרְגָּז֑וּן \hfill חִ֣יל \\
אָחַ֔ז יֹשְׁבֵ֖י פְּלָֽשֶׁת׃ \hfill אָ֤ז נִבְהֲלוּ֙ אַלּוּפֵ֣י \\
אֱד֔וֹם \hfill אֵילֵ֣י מוֹאָ֔ב יֹֽאחֲזֵ֖מוֹ רָ֑עַד \hfill נָמֹ֕גוּ \\
כֹּ֖ל יֹשְׁבֵ֥י כְנָֽעַן׃ \hfill תִּפֹּ֨ל עֲלֵיהֶ֤ם אֵימָ֙תָה֙ \\
וָפַ֔חַד \hfill בִּגְדֹ֥ל זְרוֹעֲךָ֖ יִדְּמ֣וּ כָּאָ֑בֶן \hfill עַד־\\
יַעֲבֹ֤ר עַמְּךָ֙ יְיָ֔ \hfill עַֽד־יַעֲבֹ֖ר עַם־ז֥וּ \\
קָנִֽיתָ׃ \hfill תְּבִאֵ֗מוֹ וְתִטָּעֵ֙מוֹ֙ בְּהַ֣ר נַחֲלָֽתְךָ֔ \hfill מָכ֧וֹן \\
לְשִׁבְתְּךָ֛ פָּעַ֖לְתָּ יְיָ֑ \hfill מִקְּדָ֕שׁ אֲדֹנָ֖י כּוֹנְנ֥וּ \\
יָדֶֽיךָ׃ \hfill יְיָ֥ ׀ יִמְלֹ֖ךְ לְעֹלָ֥ם וָעֶֽד׃ \begin{footnotesize}יְיָ֥ ׀ יִמְלֹ֖ךְ לְעֹלָ֥ם וָעֶֽד׃\end{footnotesize}
\hfill \begin{small} כִּ֣י \\
	בָא֩ ס֨וּס פַּרְעֹ֜ה בְּרִכְבּ֤וֹ וּבְפָרָשָׁיו֙ בַּיָּ֔ם \hfill וַיָּ֧שֶׁב יְיָ֛ עֲלֵהֶ֖ם \\
	אֶת־מֵ֣י הַיָּ֑ם \hfill וּבְנֵ֧י יִשְׂרָאֵ֛ל הָלְכ֥וּ בַיַּבָּשָׁ֖ה בְּת֥וֹךְ\hfill הַיָּֽם׃\\
	וַתִּקַּח֩ מִרְיָ֨ם הַנְּבִיאָ֜ה אֲח֧וֹת אַהֲרֹ֛ן אֶת־הַתֹּ֖ף בְּיָדָ֑הּ וַתֵּצֶ֤אןָ כׇֽל־הַנָּשִׁים֙ אַחֲרֶ֔יהָ בְּתֻפִּ֖ים וּבִמְחֹלֹֽת׃ וַתַּ֥עַן לָהֶ֖ם מִרְיָ֑ם שִׁ֤ירוּ לַֽייָ֙ כִּֽי־גָאֹ֣ה גָּאָ֔ה ס֥וּס וְרֹכְב֖וֹ רָמָ֥ה בַיָּֽם׃\hfill\break 
\end{small}


כִּ֣י \source{תהלים כב}לַ֭ייָ֭ הַמְּלוּכָ֑ה וּ֝מֹשֵׁ֗ל בַּגּוֹיִֽם׃
וְעָל֤וּ \source{עובדיה א}מֽוֹשִׁעִים֙ בְּהַ֣ר צִיּ֔וֹן לִשְׁפֹּ֖ט אֶת־הַ֣ר עֵשָׂ֑ו וְהָיְתָ֥ה לַֽייָ֖ הַמְּלוּכָֽה׃
וְהָיָ֧ה \source{זכריה יד}יְיָ֛ לְמֶ֖לֶךְ עַל־כׇּל־הָאָ֑רֶץ בַּיּ֣וֹם הַה֗וּא יִהְיֶ֧ה יְיָ֛ אֶחָ֖ד וּשְׁמ֥וֹ אֶחָֽד׃

\instruction{בשבת וביו״ט ממשיכים בעמ׳ \pageref{nishmas}}
\nextpage
\firstword{יִשְׁתַּבַּח}
שִׁמְךָ לָעַד מַלְכֵּֽנוּ הָאֵל הַמֶּֽלֶךְ הַגָּדוֹל וְהַקָּדוֹשׁ בַּשָׁמַֽיִם וּבָאָֽרֶץ כִּי לְךָ נָאֶה יְיָ אֱלֹהֵֽינוּ וֵאלֹהֵי אֲבוֹתֵֽינוּ שִׁיר וּשְׁבָחָה הַלֵּל וְזִמְרָה עֹז וּמֶמְשָׁלָה נֶֽצַח גְּדֻלָּה וּגְבוּרָה תְּהִלָּה וְתִפְאֶֽרֶת קְדֻשָּׁה וּמַלְכוּת בְּרָכוֹת וְהוֹדָאוֹת מֵעַתָּה וְעַד עוֹלָם׃ בָּרוּךְ אַתָּה יְיָ אֵל מֶֽלֶךְ גָּדוֹל בַּתֻּשְׁבָּחוֹת אֵל הַהוֹדָאוֹת אֲדוֹן הַנִּפְלָאוֹת הַבּוֹחֵר בְּשִׁירֵי זִמְרָה מֶֽלֶךְ אֵל חֵי הָעוֹלָמִים׃

\mimaamakim

\halfkaddish

\enlargethispage{\baselineskip}

`\vspace{1.25\baselineskip}
}

%\chapter[שחרית לחול Mornings Weekday ]{\adforn{47} Blessings its and Shem\ayin \adforn{19}\\קריאת שמע וברכותיה}
\chapter[שחרית לחול]{\adforn{47} קריאת שמע וברכותיה \adforn{19}}

\shacharitinstruction

\barachu

%\textbf{
בָּרוּךְ אַתָּה יְיָ אֱלֹהֵֽינוּ מֶֽלֶךְ הָעוֹלָם \middot יוֹצֵר אוֹר וּבוֹרֵא חֹֽשֶׁךְ עֹשֶׂה שָׁלוֹם וּבוֹרֵא אֶת־הַכֹּל׃
%}



\hameir

\yotzerhameoros

\ahavaraba

\label{morningshema}

\shema

\englishinst{Touch and kiss the tefillin each time they are mentioned.}
\veahavta

\vehaya

\englishinst{Kiss the tzitzi\thav\space each time they are mentioned.}
\vayomer

\englishinst{Join this paragraph with the end of the previous one with the word \hebineng{אמת}. Drop the tzitzi\thav\space at \hebineng{עולמים עלמי לעד}.}
\emesveyatziv

\ezrasavoseinu

\gaalyisroel\\
%\vspace{\baselineskip}

\AMamidainst

\section[תפילת העמידה]{\adforn{53} תפילת העמידה \adforn{25}}
\label{amidahshacharitchol}


\amidaopening{\ayt}{\englishinst{During the repetition of the Amidah, Kedusha is said here}}

\weekdaysakedusha
\sepline

\weekdaysabinah

\weekdaysateshuva

\weekdaysaselichah

\weekdaysageulah

\weekdaysaanneinu

\weekdaysarefuah

\weekdaysaberacha

\weekdaysashofar

\weekdaysamishpat

\weekdaysaminim

\weekdaysatzadikim

\ifboolexpr{togl {minchainshacharit}}{\yerushwithnachem}{\weekdaysayerushelayim}

\weekdaysamalchus

\ifboolexpr{togl {minchainshacharit}}{\weekdaysashemakoleinu{\footnote{
			\englishinst{At Min\d{h}a on public fasts during the silent Amidah, the following is added:}
			%\instruction{בת״צ במנחה היחיד מוסיף׃}
			\aneinubasetext}}}{\weekdaysashemakoleinu{}}

\retzeh

\yaalehveyavo

\enlargethispage{\baselineskip}

\zion

\modim

\alhanisim

\weekdaysahodos

%\bircaskohanim{ בחזרת הש״ץ בארץ ישראל אם יש כוהנים, הם מברכים את הקהל׃}{בחזרת הש״ץ בחו״ל או בא״י עם אין כהנים׃} \label{bk}
%\simshalom{\ayt}
\ifboolexpr{togl {minchainshacharit}}{
	\shatzbirkaskohanimenglish{During the repetition of the Amidah at Sha\d{h}arit and Min\d{h}a on public fasts:}
	
	\simshalomrav}{
	\shatzbirkaskohanimenglish{During the repetition of the Amidah:}
	
	\simshalom{\ayt}}

\tachanunim

\begin{footnotesize}
\sepline

\englishinst{If it is impossible to recite a complete Amidah, recite the first three blessings, the following text, and the last three blessings. This text cannot be used during the rainy season, or at the conclusion of Shabbat or Festivals.}
\havineinu
\instruction{רצה...}

\sepline

\personalfast

\sepline
\end{footnotesize}

\enlargethispage{\baselineskip}

%\instruction{אין אומרים תחנון בבית הכנסת ביום המילה, בבית אבל, כשיש חתן או כלה בבהכ״נ, ראש חודש, חדש ניסן, פסח שני, יום העצמאות, ל״ג בעומר, יום ירושלים, ר״ח סיון עד י״ב סיון, ט׳ באב, ט״ו באב, ערב ראש השנה, מערב יום כפור עד חדש מרחשון, חנוכה, פורים, שושן פורים, פורים קטן, ושושן פורים קטן, ושאר ימי שמחה}
%\instruction{בימים אלו אומרים קדיש עמ׳ \pageref{hatzi_kaddish}}\\
%\instruction{כשאומרים תחנון ביום ב׳ וביום ה׳ אומרים תחנון עמ׳ \pageref{tachanun mon thurs}},
%\instruction{ביום א׳ ג׳ ד׳ ו׳ נופלים על פניהם עמ׳ \pageref{nefilas_apayim}}\\
%\ifboolexpr{togl{includeChM}}{\instruction{בסוכות נוטלים הלולב עמ׳ \pageref{lulav}}\\}{}
%\instruction{ בחול המועד, חנוכה, וראש חודש אומרים הלל עמ׳\pageref{hallel}}\\
%\instruction{ בעשי״ת אומרים אבינו מלכנו עמ׳ \pageref{avinu malkeinu}}\\
%\instruction{ בת״צ אומרים סליחות ואח״כ אבינו מלכנו עמ׳ \pageref{avinu malkeinu}}
\longenginst{Ta\d{h}anun is not recited on the day of a bri\thav\space mila, in a house of mourning, or if a bride or groom is present. It is omitted on Rosh \d{H}odesh, the entire month of Nisan, Pesa\d{h} Sheini, Israeli Independence Day, Lag Ba\bigAyin omer, Jerusalem Reunification Day, from Rosh \d{H}odesh Sivan until the 12th of Sivan, the 9th of Av, the 15th of Av, the day before Rosh HaShana, from the day before Yom Kippur until the beginning of Mar\d{h}eshvan, \d{H}anukka, Purim, Shushan Purim, Purim Katan, Shushan Purim Katan, and other festive occasions.  On all those days, say half-kaddish on page \pageref{hatzi_kaddish}.

\ifboolexpr{togl {includefestival} or togl {includeChM}}{On Sukkot the Lulav is taken on page \pageref{lulav}, followed by Hallel.
}{}
During the Ten Penitential Days, Avinu Malkeinu is recited on page \pageref{avinu malkeinu}.  On minor fasts, most congregations recite Seli\d{h}ot now, followed by Avinu Malkeinu.

\ifboolexpr{togl {includeChM}}{On Intermediate Festival days, \d{H}anukka, and Rosh \d{H}odesh, recite Hallel on page \pageref{hallel}.}{On \d{H}anukka and Rosh \d{H}odesh, recite Hallel on page \pageref{hallel}.}

When Ta\d{h}anun is recited on Monday and Thursday, continue on page \pageref{tachanun mon thurs}.  On other days, continue on page \pageref{nefilas_apayim}.	
}

\ifboolexpr{not togl{includeshabbat} and not togl{includefestival}}{
%\let\clearpage\relax{
	\ifboolexpr{togl {includefestival} or togl {includeChM}}{
\section[נטילת הלולב]{\adforn{53} נטילת הלולב \adforn{25}}
\label{lulav}

בָּרוּךְ אַתָּה יְיָ אֱלֹהֵינוּ מֶלֶךְ הָעוֹלָם אֲשֶׁר קִדְּשָׁנוּ בְּמִצְוֹתָיו וְצִוָּנוּ עַל נְטִילַת לוּלָב׃

\englishinst{When first taking Lulav, add the following blessing:}
בָּרוּךְ אַתָּה יְיָ אֱלֹהֵינוּ מֶלֶךְ הָעוֹלָם שֶׁהֶחֱיָנוּ וְקִיְּמָנוּ וְהִגִּיעָנוּ לַזְמַן הַזֶּה׃
}{}

\newcommand{\diluginst}{\englishinst{}}
\ifboolexpr{(togl {includeweekday} or togl {includeshabbat}) and not togl {includefestival} and not togl {includeChM}}{
	\renewcommand{\diluginst}{\englishinst{This section is skipped on Rosh \d{H}odesh (except on \d{H}anukka).}}
	\chapter[הלל‎]{\adforn{53} הלל‎ \adforn{25}}
}{}
\ifboolexpr{(togl {includefestival} or togl {includeChM}) and (togl {includeweekday} or togl {includeshabbat})}{
	\renewcommand{\diluginst}{\englishinst{This section is skipped on Rosh \d{H}odesh (except on \d{H}anukka) and the intermediate and final days of Passover.}}
	
\chapter[הלל‎]{\adforn{53} הלל‎ \adforn{25}}
}{}
\ifboolexpr{(togl {includefestival} or togl {includeChM}) and not togl {includeweekday} and not togl {includeshabbat}}{
	\renewcommand{\diluginst}{\englishinst{This section is skipped on the intermediate and final days of Passover.}}
	
	\section[הלל‎]{\adforn{53} הלל‎ \adforn{25}}
}{}

\label{hallel}

%\instruction{החזן אומר הברכה בקול רם, הקהל אומר אמן ואחר כך חוזרים ומברכים:}\\
\firstword{בָּרוּךְ}
אַתָּה יְיָ אֱלֹהֵֽינוּ מֶֽלֶךְ הָעוֹלָם אֲשֶׁר קִדְּשָֽׁנוּ בְּמִצְוֹתָיו וְצִוָּֽנוּ לִקְרֹא אֶת־הַהַלֵּל׃

\firstword{הַ֥לְלוּ יָ֨הּ}\source{תהלים קיג}
׀ הַ֭לְלוּ עַבְדֵ֣י יְיָ֑ הַֽ֝לְל֗וּ אֶת־שֵׁ֥ם יְיָ׃
יְהִ֤י שֵׁ֣ם יְיָ֣ מְבֹרָ֑ךְ מֵ֝עַתָּ֗ה וְעַד־עוֹלָֽם׃
מִמִּזְרַח־שֶׁ֥מֶשׁ עַד־מְבוֹא֑וֹ מְ֝הֻלָּ֗ל שֵׁ֣ם יְיָ׃
רָ֖ם עַל־כׇּל־גּוֹיִ֥ם ׀ יְיָ֑ עַ֖ל הַשָּׁמַ֣יִם כְּבוֹדֽוֹ׃
מִ֭י כַּייָ֣ אֱלֹהֵ֑ינוּ הַֽמַּגְבִּיהִ֥י לָשָֽׁבֶת׃
הַֽמַּשְׁפִּילִ֥י לִרְא֑וֹת בַּשָּׁמַ֥יִם וּבָאָֽרֶץ׃
מְקִ֥ימִ֣י מֵעָפָ֣ר דָּ֑ל מֵ֝אַשְׁפֹּ֗ת יָרִ֥ים אֶבְיֽוֹן׃
לְהוֹשִׁיבִ֥י עִם־נְדִיבִ֑ים עִ֗֝ם נְדִיבֵ֥י עַמּֽוֹ׃
מֽוֹשִׁיבִ֨י ׀ עֲקֶ֬רֶת הַבַּ֗יִת אֵֽם־הַבָּנִ֥ים שְׂמֵחָ֗ה הַֽלְלוּ־יָֽהּ׃

\firstword{בְּצֵ֣את יִ֭שְׂרָאֵל}\source{תהלים קיד}
מִמִּצְרָ֑יִם בֵּ֥ית יַ֝עֲקֹ֗ב מֵעַ֥ם לֹעֵֽז׃
הָיְתָ֣ה יְהוּדָ֣ה לְקׇדְשׁ֑וֹ יִ֝שְׂרָאֵ֗ל מַמְשְׁלוֹתָֽיו׃
הַיָּ֣ם רָ֭אָה וַיָּנֹ֑ס הַ֝יַּרְדֵּ֗ן יִסֹּ֥ב לְאָחֽוֹר׃
הֶ֭הָרִים רָקְד֣וּ כְאֵילִ֑ים גְּ֝בָע֗וֹת כִּבְנֵי־צֹֽאן׃
מַה־לְּךָ֣ הַ֭יָּם כִּ֣י תָנ֑וּס הַ֝יַּרְדֵּ֗ן תִּסֹּ֥ב לְאָחֽוֹר׃
הֶ֭הָרִים תִּרְקְד֣וּ כְאֵילִ֑ים גְּ֝בָע֗וֹת כִּבְנֵי־צֹֽאן׃
מִלִּפְנֵ֣י אָ֭דוֹן ח֣וּלִי אָ֑רֶץ מִ֝לִּפְנֵ֗י אֱל֣וֹהַּ יַעֲקֹֽב׃
הַהֹפְכִ֣י הַצּ֣וּר אֲגַם־מָ֑יִם חַ֝לָּמִ֗ישׁ לְמַעְיְנוֹ־מָֽיִם׃

\diluginst
\begin{narrow}\vspace{-12pt}
	\firstword{לֹ֤א לָ֥נוּ}\source{תהלים קטו}
יְיָ֗ לֹ֫א לָ֥נוּ כִּֽי־לְ֭שִׁמְךָ תֵּ֣ן כָּב֑וֹד עַל־חַ֝סְדְּךָ֗ עַל־אֲמִתֶּֽךָ׃
לָ֭מָּה יֹאמְר֣וּ הַגּוֹיִ֑ם אַיֵּה־נָ֗֝א אֱלֹהֵיהֶֽם׃
וֵאלֹהֵ֥ינוּ בַשָּׁמָ֑יִם כֹּ֖ל אֲשֶׁר־חָפֵ֣ץ עָשָֽׂה׃
עֲֽ֭צַבֵּיהֶם כֶּ֣סֶף וְזָהָ֑ב מַ֝עֲשֵׂ֗ה יְדֵ֣י אָדָֽם׃
פֶּֽה־לָ֭הֶם וְלֹ֣א יְדַבֵּ֑רוּ עֵינַ֥יִם לָ֝הֶ֗ם וְלֹ֣א יִרְאֽוּ׃
אׇזְנַ֣יִם לָ֭הֶם וְלֹ֣א יִשְׁמָ֑עוּ אַ֥ף לָ֝הֶ֗ם וְלֹ֣א יְרִיחֽוּן׃
יְדֵיהֶ֤ם ׀ וְלֹ֬א יְמִישׁ֗וּן רַ֭גְלֵיהֶם וְלֹ֣א יְהַלֵּ֑כוּ לֹא־יֶ֝הְגּ֗וּ בִּגְרוֹנָֽם׃
כְּ֭מוֹהֶם יִהְי֣וּ עֹשֵׂיהֶ֑ם כֹּ֖ל אֲשֶׁר־בֹּטֵ֣חַ בָּהֶֽם׃
יִ֭שְׂרָאֵל בְּטַ֣ח בַּייָ֑ עֶזְרָ֖ם וּמָגִנָּ֣ם הֽוּא׃
בֵּ֣ית אַ֭הֲרֹן בִּטְח֣וּ בַייָ֑ עֶזְרָ֖ם וּמָגִנָּ֣ם הֽוּא׃
יִרְאֵ֣י יְיָ֭ בִּטְח֣וּ בַייָ֑ עֶזְרָ֖ם וּמָגִנָּ֣ם הֽוּא׃
\end{narrow}
\firstword{יְיָ זְכָרָ֢נוּ}
יְ֭בָרֵךְ אֶת־בֵּ֣ית יִשְׂרָאֵ֑ל יְ֝בָרֵ֗ךְ אֶת־בֵּ֥ית אַהֲרֹֽן׃
יְ֭בָרֵךְ יִרְאֵ֣י יְיָ֑ הַ֝קְּטַנִּ֗ים עִם־הַגְּדֹלִֽים׃
יֹסֵ֣ף יְיָ֣ עֲלֵיכֶ֑ם עֲ֝לֵיכֶ֗ם וְעַל־בְּנֵיכֶֽם׃
בְּרוּכִ֣ים אַ֭תֶּם לַייָ֑ עֹ֝שֵׂ֗ה שָׁמַ֥יִם וָאָֽרֶץ׃
הַשָּׁמַ֣יִם שָׁ֭מַיִם לַייָ֑ וְ֝הָאָ֗רֶץ נָתַ֥ן לִבְנֵי־אָדָֽם׃
לֹ֣א הַ֭מֵּתִים יְהַֽלְלוּ־יָ֑הּ וְ֝לֹ֗א כׇּל־יֹרְדֵ֥י דוּמָֽה׃
וַאֲנַ֤חְנוּ ׀ נְבָ֘רֵ֤ךְ יָ֗הּ מֵעַתָּ֥ה וְעַד־עוֹלָ֗ם הַֽלְלוּ־יָֽהּ׃

\diluginst
\begin{narrow}\vspace{-12pt}
	\firstword{אָ֭הַבְתִּי}\source{תהלים קטז}
כִּי־יִשְׁמַ֥ע ׀ יְיָ֑ אֶת־ק֝וֹלִ֗י תַּחֲנוּנָֽי׃
כִּי־הִטָּ֣ה אׇזְנ֣וֹ לִ֑י וּבְיָמַ֥י אֶקְרָֽא׃
אֲפָפ֤וּנִי ׀ חֶבְלֵי־מָ֗וֶת וּמְצָרֵ֣י שְׁא֣וֹל מְצָא֑וּנִי צָרָ֖ה וְיָג֣וֹן אֶמְצָֽא׃
וּבְשֵֽׁם־יְיָ֥ אֶקְרָ֑א אָנָּ֥ה יְ֝יָ֗ מַלְּטָ֥ה נַפְשִֽׁי׃
חַנּ֣וּן יְיָ֣ וְצַדִּ֑יק וֵ֖אלֹהֵ֣ינוּ מְרַחֵֽם׃
שֹׁמֵ֣ר פְּתָאיִ֣ם יְיָ֑ דַּ֝לֹּתִ֗י וְלִ֣י יְהוֹשִֽׁיעַ׃
שׁוּבִ֣י נַ֭פְשִׁי לִמְנוּחָ֑יְכִי כִּֽי־יְ֝יָ֗ גָּמַ֥ל עָלָֽיְכִי׃
כִּ֤י חִלַּ֥צְתָּ נַפְשִׁ֗י מִ֫מָּ֥וֶת אֶת־עֵינִ֥י מִן־דִּמְעָ֑ה אֶת־רַגְלִ֥י מִדֶּֽחִי׃
אֶ֭תְהַלֵּךְ לִפְנֵ֣י יְיָ֑ בְּ֝אַרְצ֗וֹת הַחַיִּֽים׃
הֶ֭אֱמַנְתִּי כִּ֣י אֲדַבֵּ֑ר אֲ֝נִ֗י עָנִ֥יתִי מְאֹֽד׃
אֲ֭נִי אָמַ֣רְתִּי בְחׇפְזִ֑י כׇּֽל־הָאָדָ֥ם כֹּזֵֽב׃
\end{narrow}

\firstword{מָה־אָשִׁ֥יב לַייָ֑}
כׇּֽל־תַּגְמוּל֥וֹהִי עָלָֽי׃
כּוֹס־יְשׁוּע֥וֹת אֶשָּׂ֑א וּבְשֵׁ֖ם יְיָ֣ אֶקְרָֽא׃
נְ֭דָרַי לַייָ֣ אֲשַׁלֵּ֑ם נֶגְדָה־נָּ֗֝א לְכׇל־עַמּֽוֹ׃
יָ֭קָר בְּעֵינֵ֣י יְיָ֑ הַ֝מָּ֗וְתָה לַחֲסִידָֽיו׃
אָנָּ֣ה יְיָ כִּֽי־אֲנִ֢י עַ֫בְדֶּ֥ךָ אֲנִי־עַ֭בְדְּךָ בֶּן־אֲמָתֶ֑ךָ פִּ֝תַּ֗חְתָּ לְמֽוֹסֵרָֽי׃
לְֽךָ־אֶ֭זְבַּח זֶ֣בַח תּוֹדָ֑ה וּבְשֵׁ֖ם יְיָ֣ אֶקְרָֽא׃
נְ֭דָרַי לַייָ֣ אֲשַׁלֵּ֑ם נֶגְדָה־נָּ֗֝א לְכׇל־עַמּֽוֹ׃
בְּחַצְר֤וֹת ׀ בֵּ֤ית יְיָ֗ בְּֽת֘וֹכֵ֤כִי יְֽרוּשָׁלָ֗‍ִם הַֽלְלוּ־יָֽהּ׃

\firstword{הַֽלְל֣וּ}\source{תהלים קיז}
אֶת־יְיָ֭ כׇּל־גּוֹיִ֑ם שַׁ֝בְּח֗וּהוּ כׇּל־הָאֻמִּֽים׃
כִּ֥י גָ֘בַ֤ר עָלֵ֨ינוּ ׀ חַסְדּ֗וֹ וֶאֱמֶת־יְיָ֥ לְעוֹלָ֗ם הַֽלְלוּ־יָֽהּ׃

\shatz \source{תהלים קיח}הוֹד֣וּ לַייָ֣ כִּי־ט֑וֹב כִּ֖י לְעוֹלָ֣ם חַסְדּֽוֹ׃ \hfill \break
\kahal \begin{small}הוֹד֣וּ לַייָ֣ כִּי־ט֑וֹב כִּ֖י לְעוֹלָ֣ם חַסְדּֽוֹ׃ \end{small}\\
\shatz יֹאמַר־נָ֥א יִשְׂרָאֵ֑ל כִּ֖י לְעוֹלָ֣ם חַסְדּֽוֹ׃\hfill \break
\kahal \begin{small}הוֹד֣וּ לַייָ֣ כִּי־ט֑וֹב כִּ֖י לְעוֹלָ֣ם חַסְדּֽוֹ׃ \end{small}\\
\shatz יֹאמְרוּ־נָ֥א בֵֽית־אַהֲרֹ֑ן כִּ֖י לְעוֹלָ֣ם חַסְדּֽוֹ׃ \hfill \break
\kahal \begin{small}הוֹד֣וּ לַייָ֣ כִּי־ט֑וֹב כִּ֖י לְעוֹלָ֣ם חַסְדּֽוֹ׃ \end{small}\\
\shatz יֹאמְרוּ־נָ֭א יִרְאֵ֣י יְיָ֑ כִּ֖י לְעוֹלָ֣ם חַסְדּֽוֹ׃\hfill \break
\kahal \begin{small}הוֹד֣וּ לַייָ֣ כִּי־ט֑וֹב כִּ֖י לְעוֹלָ֣ם חַסְדּֽוֹ׃ \end{small}\\
\firstword{מִֽן־הַ֭מֵּצַר}\source{תהלים קיח}
קָרָ֣אתִי יָּ֑הּ עָנָ֖נִי בַמֶּרְחָ֣ב יָֽהּ׃
יְיָ֣ לִ֭י לֹ֣א אִירָ֑א מַה־יַּעֲשֶׂ֖ה לִ֣י אָדָֽם׃
יְיָ֣ לִ֭י בְּעֹזְרָ֑י וַ֝אֲנִ֗י אֶרְאֶ֥ה בְשֹׂנְאָֽי׃
ט֗וֹב לַחֲס֥וֹת בַּייָ֑ מִ֝בְּטֹ֗חַ בָּאָדָֽם׃
ט֗וֹב לַחֲס֥וֹת בַּייָ֑ מִ֝בְּטֹ֗חַ בִּנְדִיבִֽים׃
כׇּל־גּוֹיִ֥ם סְבָב֑וּנִי בְּשֵׁ֥ם יְ֝יָ֗ כִּ֣י אֲמִילַֽם׃
סַבּ֥וּנִי גַם־סְבָב֑וּנִי בְּשֵׁ֥ם יְ֝יָ֗ כִּ֣י אֲמִילַֽם׃
סַבּ֤וּנִי כִדְבוֹרִ֗ים דֹּ֭עֲכוּ כְּאֵ֣שׁ קוֹצִ֑ים בְּשֵׁ֥ם יְ֝יָ֗ כִּ֣י אֲמִילַֽם׃
דַּחֹ֣ה דְחִיתַ֣נִי לִנְפֹּ֑ל וַ֖ייָ֣ עֲזָרָֽנִי׃
עָזִּ֣י וְזִמְרָ֣ת יָ֑הּ וַֽיְהִי־לִ֗֝י לִישׁוּעָֽה׃
ק֤וֹל ׀ רִנָּ֬ה וִישׁוּעָ֗ה בְּאׇהֳלֵ֥י צַדִּיקִ֑ים יְמִ֥ין יְ֝יָ֗ עֹ֣שָׂה חָֽיִל׃
יְמִ֣ין יְיָ֭ רוֹמֵמָ֑ה יְמִ֥ין יְ֝יָ֗ עֹ֣שָׂה חָֽיִל׃
לֹא־אָמ֥וּת כִּֽי־אֶחְיֶ֑ה וַ֝אֲסַפֵּ֗ר מַעֲשֵׂ֥י יָֽהּ׃
יַסֹּ֣ר יִסְּרַ֣נִּי יָּ֑הּ וְ֝לַמָּ֗וֶת לֹ֣א נְתָנָֽנִי׃
פִּתְחוּ־לִ֥י שַׁעֲרֵי־צֶ֑דֶק אָבֹא־בָ֗֝ם אוֹדֶ֥ה יָֽהּ׃
זֶה־הַשַּׁ֥עַר לַייָ֑ צַ֝דִּיקִ֗ים יָבֹ֥אוּ בֽוֹ׃\\
א֭וֹדְךָ כִּ֣י עֲנִיתָ֑נִי וַתְּהִי־לִ֗֝י לִישׁוּעָֽה׃ \\
\scriptsize{ א֭וֹדְךָ כִּ֣י עֲנִיתָ֑נִי וַתְּהִי־לִ֗֝י לִישׁוּעָֽה׃ \\}\normalsize{}
אֶ֭בֶן מָאֲס֣וּ הַבּוֹנִ֑ים הָ֝יְתָ֗ה לְרֹ֣אשׁ פִּנָּֽה׃ \\
\scriptsize{ אֶ֭בֶן מָאֲס֣וּ הַבּוֹנִ֑ים הָ֝יְתָ֗ה לְרֹ֣אשׁ פִּנָּֽה׃ \\}\normalsize{}
מֵאֵ֣ת יְיָ֭ הָ֣יְתָה זֹּ֑את הִ֖יא נִפְלָ֣את בְּעֵינֵֽינוּ׃ \\
\scriptsize{ מֵאֵ֣ת יְיָ֭ הָ֣יְתָה זֹּ֑את הִ֖יא נִפְלָ֣את בְּעֵינֵֽינוּ׃ \\}\normalsize{}
זֶה־הַ֭יּוֹם עָשָׂ֣ה יְיָ֑ נָגִ֖ילָה וְנִשְׂמְחָ֣ה בֽוֹ׃ \\
\scriptsize{ זֶה־הַ֭יּוֹם עָשָׂ֣ה יְיָ֑ נָגִ֖ילָה וְנִשְׂמְחָ֣ה בֽוֹ׃ } \normalsize{}


\instruction{ש״ץ ואח״כ הקהל׃}\\
אָנָּ֣א יְיָ֭ הוֹשִׁ֘יעָ֥ה נָּ֑א \hfill אָנָּ֣א יְיָ֭ הוֹשִׁ֘יעָ֥ה נָּ֑א\\
אָנָּ֥א יְ֝יָ֗ הַצְלִ֘יחָ֥ה נָּֽא׃ \hfill אָנָּ֥א יְ֝יָ֗ הַצְלִ֘יחָ֥ה נָּֽא׃\\
בָּר֣וּךְ הַ֭בָּא בְּשֵׁ֣ם יְיָ֑ בֵּ֝רַ֥כְנוּכֶ֗ם מִבֵּ֥ית יְיָ׃\\
\scriptsize{בָּר֣וּךְ הַ֭בָּא בְּשֵׁ֣ם יְיָ֑ בֵּ֝רַ֥כְנוּכֶ֗ם מִבֵּ֥ית יְיָ׃}\\
\normalsize{אֵ֤ל ׀ יְיָ וַיָּ֢אֶ֫ר לָ֥נוּ אִסְרוּ־חַ֥ג בַּעֲבֹתִ֑ים עַד־קַ֝רְנ֗וֹת הַמִּזְבֵּֽחַ׃}\\
\scriptsize{אֵ֤ל ׀ יְיָ וַיָּ֢אֶ֫ר לָ֥נוּ אִסְרוּ־חַ֥ג בַּעֲבֹתִ֑ים עַד־קַ֝רְנ֗וֹת הַמִּזְבֵּֽחַ׃}\\
\normalsize{אֵלִ֣י אַתָּ֣ה וְאוֹדֶ֑ךָּ אֱ֝לֹהַ֗י אֲרוֹמְמֶֽךָּ׃}\\
\scriptsize{אֵלִ֣י אַתָּ֣ה וְאוֹדֶ֑ךָּ אֱ֝לֹהַ֗י אֲרוֹמְמֶֽךָּ׃}\\
\normalsize{הוֹד֣וּ לַייָ֣ כִּי־ט֑וֹב כִּ֖י לְעוֹלָ֣ם חַסְדּֽוֹ׃}\\
\scriptsize{הוֹד֣וּ לַייָ֣ כִּי־ט֑וֹב כִּ֖י לְעוֹלָ֣ם חַסְדּֽוֹ׃} \\
\normalsize{}



\negline

\firstword{יְהַלְלֽוּךָ}
יְיָ אֱלֹהֵֽינוּ כׇּל־מַעֲשֶֽׂיךָ וַחֲסִידֶֽיךָ צַדִּיקִים עוֹשֵׂי רְצֹנֶֽךָ וְכׇל־עַמְּךָ בֵּית יִשְׂרָאֵל בְּרִנָּה יוֹדוּ וִיבָרְכוּ וִישַׁבְּחוּ וִיפָאֲרוּ וִירוֹמֲמוּ וְיַעֲרִֽיצוּ וְיַקְדִּֽישׁוּ וְיַמְלִֽיכוּ אֶת־שִׁמְךָ מַלְכֵּֽנוּ כִּי לְךָ טוֹב לְהוֹדוֹת וּלְשִׁמְךָ נָאֶה לְזַמֵּר כִּי מֵעוֹלָם וְעַד עוֹלָם אַתָּה אֵל׃ בָּרוּךְ אַתָּה יְיָ מֶֽלֶךְ מְהֻלָּל בַּתִּשְׁבָּחוֹת׃\\

%\ifboolexpr{togl {includeshabbat} and togl {includeweekday} and not togl {includeChM}}{\englishinst{On Shabbat, continue with Full Kaddish on page \pageref{shacharitShabbatYTtitkabel}. On weekday Rosh \d{H}odesh, continue with Full Kaddish on page \pageref{end of shacharis}, followed by the Psalm of the Day as relevant on page \pageref{shir_shel_yom}. On \d{H}anukka that is not Rosh \d{H}odesh, continue with Half Kaddish on page \pageref{hatzi_kaddish}.}}{}

\englishinst{On Shabbat, Festivals, and Rosh \d{H}odesh, recite Full Kaddish followed by the Psalm of the Day. Then Mourner's Kaddish is read, followed by the Torah service on page \pageref{shabYTtorah} for Shabbat and Festivals and page \pageref{weekday torah} for Intermediate Festival Days and Rosh \d{H}odesh.}

%}
}{}

%\section[אבינו מלכנו Malkeinu Avinu]{\adforn{53} Malkeinu Avinu \adforn{25}\\אבינו מלכנו}
\section[אבינו מלכנו]{\adforn{53} אבינו מלכנו \adforn{25}}
\label{avinu malkeinu}

%\instruction{פותחים הארון}
\englishinst{Avinu Malkeinu is recited standing. The ark is opened. On the Fast of Gedalia recite the text for the Ten Penitential Days.}
\avinumalkeinu

%\instruction{סגורים הארון}\\
\englishinst{The ark is closed.}  

%\section[תחנון Ta\d{h}anun]{\adforn{53} תחנון Ta\d{h}anun \adforn{25}}
\section[תחנון]{\adforn{53} תחנון \adforn{25}}
\label{tachanun mon thurs}

\englishinst{On Monday and Thursday, Ta\d{h}anun begins here. This portion of Ta\d{h}anun is recited standing.}
\firstword{וְה֤וּא רַח֨וּם}
׀ יְכַפֵּ֥ר\source{תהלים עח} עָוֺן֮ וְֽלֹא־יַֽ֫שְׁחִ֥ית וְ֭הִרְבָּה לְהָשִׁ֣יב אַפּ֑וֹ וְלֹא־יָ֝עִ֗יר כׇּל־חֲמָתֽוֹ׃
אַתָּה יְיָ לֹא תִכְלָא רַחֲמֶֽיךָ מִמֶּֽנּוּ חַסְדְּךָ וַאֲמִתְּךָ תָּמִיד יִצְּרֽוּנוּ׃
 הוֹשִׁיעֵ֨נוּ\source{תהילים קו} ׀ יְ֘יָ֤ אֱלֹהֵ֗ינוּ וְקַבְּצֵנוּ֮ מִֽן־הַגּ֫וֹיִ֥ם לְ֭הֹדוֹת לְשֵׁ֣ם קׇדְשֶׁ֑ךָ לְ֝הִשְׁתַּבֵּ֗חַ בִּתְהִלָּתֶֽךָ׃
 אִם־עֲוֺנ֥וֹת\source{תהילים קל} תִּשְׁמׇר־יָ֑הּ אֲ֝דֹנָ֗י מִ֣י יַעֲמֹֽד׃ כִּֽי־עִמְּךָ֥ הַסְּלִיחָ֑ה לְ֝מַ֗עַן תִּוָּרֵֽא׃
לֹא כַחֲטָאֵֽינוּ תַּעֲשֶׂה־לָּנוּ וְלֹא כַעֲוֹנוֹתֵֽינוּ תִּגְמוֹל עָלֵֽינוּ׃
 אִם־עֲוֺנֵ֙ינוּ֙\source{ירמיה יד} עָ֣נוּ בָ֔נוּ יְיָ֕ עֲשֵׂ֖ה לְמַ֣עַן שְׁמֶ֑ךָ זְכֹר־רַחֲמֶ֣יךָ\source{תהילים כה} יְיָ֭ וַחֲסָדֶ֑יךָ כִּ֖י מֵעוֹלָ֣ם הֵֽמָּה׃
יַעֲנֵֽנוּ יְיָ בְּיוֹם צָרָה יְשַׂגְּבֵֽנוּ שֵׁם אֱלֹהֵי יַעֲקֹב׃
 יְיָ֥\source{תהילים כ} הוֹשִׁ֑יעָה הַ֝מֶּ֗לֶךְ יַעֲנֵ֥נוּ בְיוֹם־קׇרְאֵֽנוּ׃
אָבִֽינוּ מַלְכֵּֽנוּ חׇנֵּֽנוּ וַעֲנֵֽנוּ כִּי אֵין בָּֽנוּ מַעֲשִׂים צְדָקָה עֲשֵׂה עִמָּֽנוּ לְמַֽעַן שְׁמֶֽךָ׃
אֲדוֹנֵֽינוּ אֱלֹהֵֽינוּ שְׁמַע קוֹל תַּחֲנוּנֵֽינוּ וּזְכׇר־לָֽנוּ אֶת־בְּרִית אֲבוֹתֵֽינוּ וְהוֹשִׁיעֵֽנוּ לְמַֽעַן שְׁמֶֽךָ׃
וְעַתָּ֣ה\source{דניאל ט} ׀ אֲדֹנָ֣י אֱלֹהֵ֗ינוּ אֲשֶׁר֩ הוֹצֵ֨אתָ אֶֽת־עַמְּךָ֜ מֵאֶ֤רֶץ מִצְרַ֙יִם֙ בְּיָ֣ד חֲזָקָ֔ה וַתַּֽעַשׂ־לְךָ֥ שֵׁ֖ם כַּיּ֣וֹם הַזֶּ֑ה חָטָ֖אנוּ רָשָֽׁעְנוּ׃ אֲדֹנָ֗י כְּכׇל־צִדְקֹתֶ֙ךָ֙ יָֽשׇׁב־נָ֤א אַפְּךָ֙ וַחֲמָ֣תְךָ֔ מֵעִֽירְךָ֥ יְרוּשָׁלַ֖͏ִם הַר־קׇדְשֶׁ֑ךָ כִּ֤י בַחֲטָאֵ֙ינוּ֙ וּבַעֲוֺנ֣וֹת אֲבֹתֵ֔ינוּ יְרוּשָׁלַ֧͏ִם וְעַמְּךָ֛ לְחֶרְפָּ֖ה לְכׇל־סְבִיבֹתֵֽינוּ׃ וְעַתָּ֣ה ׀ שְׁמַ֣ע אֱלֹהֵ֗ינוּ אֶל־תְּפִלַּ֤ת עַבְדְּךָ֙ וְאֶל־תַּ֣חֲנוּנָ֔יו וְהָאֵ֣ר פָּנֶ֔יךָ עַל־מִקְדָּשְׁךָ֖ הַשָּׁמֵ֑ם לְמַ֖עַן אֲדֹנָֽי׃\\

הַטֵּ֨ה\source{דניאל ט} אֱלֹהַ֥י ׀ אׇזְנְךָ֮ וּֽשְׁמָע֒ \qk{פְּקַ֣ח}{פקחה} עֵינֶ֗יךָ וּרְאֵה֙ שֹֽׁמְמֹתֵ֔ינוּ וְהָעִ֕יר אֲשֶׁר־נִקְרָ֥א שִׁמְךָ֖ עָלֶ֑יהָ כִּ֣י ׀ לֹ֣א עַל־צִדְקֹתֵ֗ינוּ אֲנַ֨חְנוּ מַפִּילִ֤ים תַּחֲנוּנֵ֙ינוּ֙ לְפָנֶ֔יךָ כִּ֖י עַל־רַחֲמֶ֥יךָ הָרַבִּֽים׃ אֲדֹנָ֤י ׀ שְׁמָ֙עָה֙ אֲדֹנָ֣י ׀ סְלָ֔חָה אֲדֹנָ֛י הַֽקְשִׁ֥יבָה וַעֲשֵׂ֖ה אַל־תְּאַחַ֑ר לְמַֽעַנְךָ֣ אֱלֹהַ֔י כִּֽי־שִׁמְךָ֣ נִקְרָ֔א עַל־עִירְךָ֖ וְעַל־עַמֶּֽךָ׃
אָבִֽינוּ אָב הָרַחֲמָן הַרְאֵֽנוּ אוֹת לְטוֹבָה וְקַבֵּץ נְפוּצוֹתֵֽינוּ מֵאַרְבַּע כַּנְפוֹת הָאָרֶץ יַכִּירוּ וְיֵדְעוּ כׇּל־הַגּוֹיִם כִּי אַתָּה יְיָ אֱלֹהֵֽינוּ׃
וְעַתָּ֥ה\source{ישעיה סד} יְיָ֖ אָבִ֣ינוּ אָ֑תָּה אֲנַ֤חְנוּ הַחֹ֙מֶר֙ וְאַתָּ֣ה יֹצְרֵ֔נוּ וּמַעֲשֵׂ֥ה יָדְךָ֖ כֻּלָּֽנוּ׃ הוֹשִׁיעֵֽנוּ לְמַֽעַן שְׁמֶֽךָ צוּרֵֽנוּ מַלְכֵּֽנוּ וְגוֹאֲלֵֽנוּ׃
ח֧וּסָה \source{יואל ב}יְיָ֣ עַל־עַמֶּ֗ךָ וְאַל־תִּתֵּ֨ן נַחֲלָתְךָ֤ לְחֶרְפָּה֙ לִמְשׇׁל־בָּ֣ם גּוֹיִ֔ם לָ֚מָּה יֹאמְר֣וּ בָעַמִּ֔ים אַיֵּ֖ה אֱלֹהֵיהֶֽם׃
יָדַֽעְנוּ כִּי חָטָֽאנוּ וְאֵין מִי יַעֲמֹד בַּעֲדֵֽנוּ שִׁמְךָ הַגָּדוֹל יַעֲמׇד־לָֽנוּ בְּעֵת צָרָה׃ יָדַֽעְנוּ כִּי אֵין בָּֽנוּ מַעֲשִׂים צְדָקָה עֲשֵׂה עִמָּֽנוּ לְמַֽעַן שְׁמֶֽךָ׃ כְּרַחֵם אָב עַל בָּנִים כֵּן תְּרַחֵם יְיָ עָלֵֽינוּ וְהוֹשִׁעֵֽנוּ לְמַֽעַן שְׁמֶֽךָ׃ חֲמוֹל עַל עַמֶּֽךָ רַחֵם עַל נַחֲלָתֶֽךָ חֽוּסָה נָּא כְּרוֹב רַחֲמֶֽיךָ׃ חׇנֵּֽנוּ וַעֲנֵֽנוּ כִּי לְךָ יְיָ הַצְּדָקָה עֹשֵׂה נִפְלָאוֹת בְּכׇל־עֵת׃\\

הַבֶּט־נָא רַחֶם־נָא עַל עַמְּךָ מְהֵרָה לְמַֽעַן שְׁמֶֽךָ׃ בְּרַחֲמֶֽיךָ הָרַבִּים יְיָ אֱלֹהֵֽינוּ חוּס וְרַחֵם וְהוֹשִֽׁיעָה צֹאן מַרְעִיתֶֽךָ וְאַל יִמְשׇׁל־בָּֽנוּ קֶֽצֶף כִּי לְךָ עֵינֵֽינוּ תְלוּיוֹת׃ הוֹשִׁיעֵֽנוּ לְמַֽעַן שְׁמֶֽךָ רַחֵם עָלֵֽינוּ לְמַֽעַן בְּרִיתֶֽךָ׃ הַבִּֽיטָה וַעֲנֵֽנוּ בְּעֵת צָרָה כִּי לְךָ יְיָ הַיְשׁוּעָה׃ בְּךָ תוֹחַלְתֵּֽנוּ אֱלֽוֹהַּ סְלִיחוֹת אָנָּא סְלַח־נָא אֵל טוֹב וְסַלָח כִּי אֵל מֶֽלֶךְ חַנּוּן וְרַחוּם אַֽתָּה׃\\

\firstword{אָנָּא} 
מֶֽלֶךְ חַנּוּן וְרַחוּם זְכוֹר וְהַבֵּט לִבְרִית בֵּין הַבְּתָרִים וְתֵרָאֶה לְפָנֶֽיךָ עֲקֵדַת יָחִיד לְמַֽעַן יִשְׂרָאֵל׃ אָבִֽינוּ מַלְכֵּֽנוּ חׇנֵּֽנוּ וְעֲנֵֽנוּ כִּי שִׁמְךָ הַגָּדוֹל נִקְרָא עָלֵֽינוּ׃ עֹשֵׂה נִפְלָאוֹת בְּכׇל־עֵת עֲשֵׂה עִמָּֽנוּ כְּחַסְדֶּֽךָ׃ חַנּוּן וְרַחוּם הַבִּֽיטָה וַעֲנֵֽנוּ בְּעֵת צָרָה כִּי לְךָ יְיָ הַיְשׁוּעָה׃ אָבִֽינוּ מַלְכֵּֽנוּ מַחֲסֵֽנוּ אַל תַּֽעַשׂ עִמָּֽנוּ כְּרֹֽעַ מַעֲלָלֵֽינוּ׃ זְכֹר רַחֲמֶֽיךָ יְיָ וְחֲסָדֶֽיךָ וּכְרֹב טוּבְךָ הוֹשִׁיעֵֽנוּ וַחֲמׇל־נָא עָלֵֽינוּ כִּי אֵין לָֽנוּ אֱלֽוֹהַּ אַחֵר מִבַּלְעָדֶיךָ צוּרֵֽנוּ׃ אַל תַּעַזְבֵֽנוּ יְיָ אֱלֹהֵֽינוּ וְאַל תִּרְחַק מִמֶּנּוּ כִּי נַפְשֵֽׁנוּ קְצָרָה מֵחֶֽרֶב וּמִשְּׁבִי וּמִדֶּֽבֶר וּמִמַּגֵּפָה וּמִכׇּל־צָרָה וְיָגוֹן׃ הַצִּילֵֽנוּ כִּי לְךָ קִוִּֽינוּ וְאַל תַּכְלִימֵֽנוּ יְיָ אֱלֹהֵֽינוּ׃ וְהָאֵר פָּנֶֽיךָ בָּֽנוּ וּזְכׇר־לָֽנוּ אֶת־בְּרִית אֲבוֹתֵֽינוּ וְהוֹשִׁיעֵֽנוּ לְמַֽעַן שְׁמֶֽךָ׃ רְאֵה בְצָרוֹתֵֽינוּ וּשְׁמַע קוֹל תְּפִלָּתֵֽנוּ כִּי אַתָּה שׁוֹמֵֽעַ תְּפִלַּת כׇּל־פֶּה׃\\
אֵל רַחוּם וְחַנּוּן רַחֵם עָלֵֽינוּ וְעַל כׇּל־מַעֲשֶֽׂיךָ כִּי אֵין כָּמֽוֹךָ׃ יְיָ אֱלֹהֵֽינוּ אָנָּא שָׂא־נָא פְשָׁעֵֽינוּ׃ אָבִֽינוּ מַלְכֵּֽנוּ צוּרֵֽנוּ וְגוֹאֲלֵֽנוּ אֵל חַי וְקַיָּם הַחֲסִין בַּכֹּֽחַ חָסִיד וָטוֹב עַל כׇּל־מַעֲשֶֽׂיךָ כִּי אַתָּה הוּא יְיָ אֱלֹהֵֽינוּ׃ אֵל אֶֽרֶךְ אַפַּֽיִם וּמָלֵא רַחֲמִים עֲשֵׂה עִמָּֽנוּ כְּרֹב רַחֲמֶֽיךָ וְהוֹשִׁיעֵֽנוּ לְמַֽעַן שְׁמֶֽךָ׃ שְׁמַע מַלְכֵּֽנוּ תְּפִלָּתֵֽנוּ וּמִיַד אוֹיְבֵֽינוּ הַצִּילֵֽנוּ׃ שְׁמַע מַלְכֵּֽנוּ תְּפִלָּתֵֽנוּ וּמִכׇּל־צָרָה וְיָגוֹן הַצִּילֵֽנוּ׃ אָבִֽינוּ מַלְכֵּֽנוּ אַֽתָּה \source{ירמיה יד}וְשִׁמְךָ֛ עָלֵ֥ינוּ נִקְרָ֖א אַל־תַּנִּחֵֽנוּ׃
אַל תַּעַזְבֵֽנוּ אָבִֽינוּ וְאַל תִּטְּשֵֽׁנוּ בּוֹרְאֵֽנוּ וְאַל תִּשְׁכָּחֵֽנוּ יוֹצְרֵֽנוּ כִּי אֵל מֶֽלֶךְ חַנּוּן וְרַחוּם אַֽתָּה׃\\

\firstword{אֵין כָּמֽוֹךָ}
חַנּוּן וְרַחוּם יְיָ אֱלֹהֵֽינוּ אֵין כָּמֽוֹךָ אֵל אֶֽרֶךְ אַפַּֽיִם וְרַב חֶֽסֶד וֶאֶמֶת׃ הוֹשִׁיעֵֽנוּ בְּרַחֲמֶֽיךָ הָרַבִּים מֵרַֽעַשׁ וּמֵרֹֽגֶז הַצִּילֵֽנוּ׃ זְכֹר לַעֲבָדֶֽיךָ לְאַבְרָהָם לְיִצְחָק וּלְיַעֲקֹב אַל תֵּֽפֶן אֶל קׇשְׁיֵֽנוּ וְאֶל רִשְׁעֵֽנוּ וְאֶל חַטָּאתֵֽנוּ׃
שׁ֚וּב \source{שמות לב}מֵחֲר֣וֹן אַפֶּ֔ךָ וְהִנָּחֵ֥ם עַל־הָרָעָ֖ה לְעַמֶּֽךָ׃
וְהָסֵר מִמֶּֽנּוּ מַכַּת הַמָּֽוֶת כִּי רַחוּם אַֽתָּה כִּי כֵן דַּרְכֶּֽךָ עֹֽשֶׂה חֶֽסֶד חִנָּם בְּכׇל־דוֹר וָדוֹר׃ חֽוּסָה יְיָ עַל עַמֶּֽךָ וְהַצִּילֵֽנוּ מִזַּעְמֶּֽךָ וְהָסֵר מִמֶּֽנּוּ מַכַּת הַמַּגֵּפָה וּגְזֵרָה קָשָׁה כִּי אַתָּה שׁוֹמֵר יִשְׂרָאֵל׃
לְךָ֤ \source{דניאל ט}אֲדֹנָי֙ הַצְּדָקָ֔ה וְלָ֛נוּ בֹּ֥שֶׁת הַפָּנִ֖ים
מַה נִּתְאוֹנֵן וּמַה נֹּאמַר מַה נְּדַבֵּר וּמַה נִּצְטַדָּק׃ נַחְפְּשָׂה דְרָכֵֽינוּ וְנַחְקֹֽרָה וְנָשֽׁוּבָה אֵלֶֽיךָ כִּי יְמִינְךָ פְשׁוּטָה לְקַבֵּל שָׁבִים׃
אָנָּ֣א \source{תהלים קיח}יְ֭יָ הוֹשִׁ֘יעָ֥ה נָּ֑א אָנָּ֥א יְ֝יָ֗ הַצְלִ֘יחָ֥ה נָּֽא׃
אָנָּא יְיָ עֲנֵֽנוּ בְּיוֹם קׇרְאֵֽנוּ׃ לְךָ יְיָ חִכִּֽינוּ לְךָ יְיָ קִוִּֽינוּ לְךָ יְיָ נְיַחֵל אַל תֶּחֱשֶׁה וּתְעַנֵּֽנוּ כִּי נָאֲמוּ גוֹיִם אָבְדָה תִקְוָתָם כׇּל־בֶּֽרֶךְ וְכׇל־קוֹמָה לְךָ לְבַד תִּשְׁתַּחֲוֶה׃\\
הַפּוֹתֵחַ יָד בִּתְשׁוּבָה לְקַבֵּל פּוֹשְׁעִים וְחַטָּאִים נִבְהֲלָה נַפְשֵֽׁנוּ מֵרֹב עִצְּבוֹנֵֽנוּ וְאַל תִּשְׁכָּחֵֽנוּ נֶֽצַח׃ קֽוּמָה וְהוֹשִׁיעֵֽנוּ כִּי חָסִֽינוּ בָךְ׃ אָבִֽינוּ מַלְכֵּֽנוּ אִם אֵין בָּֽנוּ צְדָקָה וּמַעֲשִׂים טוֹבִים זְכׇר־לָֽנוּ אֶת־בְּרִית אֲבוֹתֵֽינוּ וְעֵדוֹתֵֽינוּ בְּכׇל־יוֹם יְיָ אֶחָד׃ הַבִּֽיטָה בְעׇנְיֵֽנוּ כִּי רַֽבּוּ מַכְאוֹבֵֽינוּ וְצָרוֹת לְבָבֵֽנוּ׃ חֽוּסָה יְיָ עָלֵֽינוּ בְּאֶֽרֶץ שִׁבְיֵֽנוּ וְאַל תִּשְׁפּוֹךְ חֲרוֹנְךָ עָלֵֽינוּ כִּי אֲנַֽחְנוּ עַמְּךָ בְּנֵי בְרִיתֶֽךָ׃ אֵל הַבִּיטָה דַּל כְּבוֹדֵֽנוּ בַּגּוֹיִם וְשִׁקְּצֽוּנוּ כְּטֻמְאַת הַנִּדָּה׃ עַד מָתַי עֻזְּךָ לַשְּׁבִי וְתִפְאַרְתְּךָ בְּיַד צָר׃ עוֹרְרָה גְבוּרָתְךָ וְקִנְאָתְךָ עַל אוֹיְבֶֽיךָ הֵם יֵבֽוֹשׁוּ וְיֵחַֽתּוּ מִגְּבוּרָתָם וְאַל יִמְעֲטוּ לְפָנֶֽיךָ תְּלָאוֹתֵֽינוּ׃ מַהֵר יְקַדְּמֽוּנוּ רַחֲמֶֽיךָ בְּיוֹם צָרָתֵֽינוּ וְאִם לֹא לְמַעֲנֵֽנוּ לְמַעַנְךָ פְעַל וְאַל תַּשְׁחִית זֵֽכֶר שְׁאֵרִיתֵֽנוּ׃ וְחֹן אֹם הַמְיַחֲדִים שִׁמְךָ פַּעֲמַֽיִם בְּכׇל־יוֹם תָּמִיד בְּאַהֲבָה וְאוֹמְרִים׃
שְׁמַ֖ע \source{דברים ו}יִשְׂרָאֵ֑ל יְיָ֥ אֱלֹהֵ֖ינוּ יְיָ֥ ׀ אֶחָֽד׃

\subsection[נפילת אפים]{\adforn{18} נפילת אפים \adforn{17}}

\englishinst{On Sunday, Tuesday, Wednesday, and Friday, Ta\d{h}anun begins here.}\vspace{-\baselineskip}
\nefilasapayim \label{nefilas_apayim}

\begin{sometimes}
	
\englishinst{On Monday and Thursday:}
יְיָ אֱלֹהֵי יִשְׂרָאֵל \source{שמות לב} שׁ֚וּב מֵֽחֲר֣וֹן אַפֶּ֔ךָ וְהִנָּחֵ֥ם עַל־הָֽרָעָ֖ה לְעַמֶּֽךָ׃\\
הַבֵּט מִשָׁמַיִם וּרְאֵה כִּי הָיִינוּ לַעַג וָקֶֽלֶס בַּגּוֹיִם נֶחְשַׁבְנוּ כְּצֹאן לַטֶּֽבַח יוּבָל לַהֲרוֹג וּלְאַבֵּד וּלְמַכָּה וּלְחֶרְפָּה׃ וּבְכׇל־זֹאת שִׁמְךָ לֹא שָׁכָֽחְנוּ נָא אַל תִּשְׁכָּחֵֽנוּ׃\\
יְיָ אֱלֹהֵי יִשְׂרָאֵל שׁ֚וּב מֵֽחֲר֣וֹן אַפֶּ֔ךָ וְהִנָּחֵ֥ם עַל־הָֽרָעָ֖ה לְעַמֶּֽךָ׃\\
זָרִים אוֹמְרִים אֵין תּוֹחֶֽלֶת וְתִקְוָה חֹן אֹם לְשִׁמְךָ מְקַוָּה טָהוֹר יְשׁוּעָתֵֽנוּ קָרְבָה יָגַ֖עְנוּ וְלֹ֥א הֽוּנַֽח־לָֽנוּ רַחֲמֶֽיךָ יִכְבְּשׁוּ אֶת־כַּעַסְךָ מֵעָלֵֽינוּ׃ אָנָא שׁוּב מֵחֲרוֹנְךָ וְרַחֵם סְגֻלָּה אֲשֶׁר בָּחָֽרְתָּ\\
יְיָ אֱלֹהֵי יִשְׂרָאֵל שׁ֚וּב מֵֽחֲר֣וֹן אַפֶּ֔ךָ וְהִנָּחֵ֥ם עַל־הָֽרָעָ֖ה לְעַמֶּֽךָ׃\\
חוּסָה יְיָ עָלֵֽינוּ בְּרַחֲמֶֽיךָ וְאַל תִּתְּנֵֽנוּ בִּידֵי אַכְזָרִים׃
לָ֭מָּה יֹֽאמְר֣וּ הַגּוֹיִ֑ם אַיֵּה־נָ֝֗א אֱלֹֽהֵיהֶֽם׃
לְמַעַנְךָ עֲשֵׂה עִמָּנוּ חֶסֶד וְאַל תְּאַחַר׃
אָנָא שׁוּב מֵחֲרוֹנְךָ וְרַחֵם סְגֻלָּה אֲשֶׁר בָּחָֽֽרְתָּ\\
יְיָ אֱלֹהֵי יִשְׂרָאֵל שׁ֚וּב מֵֽחֲר֣וֹן אַפֶּ֔ךָ וְהִנָּחֵ֥ם עַל־הָֽרָעָ֖ה לְעַמֶּֽךָ׃\\
קוֹלֵנוּ תִשְׁמַע וְתָחֹן וְאַל תִּטְּשֵׁנוּ בְּיַד אוֹיְבֵינוּ לִמְחוֹת אֶת־שְׁמֵנוּ׃
זְכֹר אֲשֶׁר נִשְׁבַּֽעְתָּ לַאֲבוֹתֵֽינוּ כְּכוֹכְבֵי הַשָּׁמַֽיִם אַרְבֶּה אֶת־זַרְעֲכֶם וְעַתָּה נִשְׁאַרְנוּ מְעַט מֵהַרְבֵּה׃
וּבְכׇל־זֹאת שִׁמְךָ לֹא שָׁכָֽחְנוּ נָא אַל תִּשְׁכָּחֵֽנוּ׃\\
יְיָ אֱלֹהֵי יִשְׂרָאֵל שׁ֚וּב מֵֽחֲר֣וֹן אַפֶּ֔ךָ וְהִנָּחֵ֥ם עַל־הָֽרָעָ֖ה לְעַמֶּֽךָ׃\\
\source{תהלים עט}
עׇזְרֵ֤נוּ ׀ אֱלֹ֘הֵ֤י יִשְׁעֵ֗נוּ עַֽל־דְּבַ֥ר כְּבֽוֹד־שְׁמֶ֑ךָ וְהַצִּילֵ֥נוּ וְכַפֵּ֥ר עַל־חַ֝טֹּאתֵ֗ינוּ לְמַ֣עַן שְׁמֶֽךָ׃\\
יְיָ אֱלֹהֵי יִשְׂרָאֵל שׁ֚וּב מֵֽחֲר֣וֹן אַפֶּ֔ךָ וְהִנָּחֵ֥ם עַל־הָֽרָעָ֖ה לְעַמֶּֽךָ׃

\end{sometimes}

\shomeryisroel

\label{hatzi_kaddish}
\halfkaddish

\ifboolexpr{togl {includeChM}}{\englishinst{On Monday, Thursday, Rosh \d{H}odesh, \d{H}anukka, Purim, and Intermediate Festival Days the Torah is read.  On other days, continue with Ashrei on page \pageref{ashrei}.}}{
\englishinst{On Monday, Thursday, Rosh \d{H}odesh, \d{H}anukka, and Purim, the Torah is read.  On other days, continue with Ashrei on page \pageref{ashrei}.}
}

%\instruction{
%ביום ב׳ ה׳, ראש חודש, חנוכה, פורים, וחול המועד קוראים התורה.
%בשאר ימים ממשיכים באשרי עמ׳
%\pageref{ashrei}}

\section[סדר קריאת התורה]{\adforn{53} סדר קריאת התורה \adforn{25}}
%\section[סדר קריאת התורה Torah Service]{\adforn{53} Service Torah \adforn{25}\\ סדר קריאת התורה}
\label{weekday torah}

\ifboolexpr{togl {includeChM}}{\englishinst{The following is not said on Rosh \d{H}odesh, \d{H}anukka, or Intermediate Festival Days.  Most congregations say only one paragraph.}}{
\englishinst{The following is not said on Rosh \d{H}odesh or \d{H}anukka.  Most congregations say only one paragraph.}}
\instruction{אאאא״א בראש חודש, חנוכה, וחול המועד:}\\
\firstword{אֵל אֶֽרֶךְ}
אַפַּֽיִם וְרַב חֶֽסֶד וֶאֱמֶת אַל בְּאַפְּךָ תּוֹכִיחֵֽנוּ: 
ח֧וּסָה יְיָ֣ עַל־עַמֶּ֗ךָ וְהוֹשִׁיעֵֽנוּ מִכָּל־רָע:
 חָטָֽאנוּ לְךָ אָדוֹן סְלַח־נָא כְּרֹב רַחֲמֶֽיךָ אֵל:
 
\firstword{אֵל אֶֽרֶךְ}
אַפַּֽיִם וְרַב חֶֽסֶד וֶאֱמֶת אַל תַּסְתֵּר פָּנֶֽיךָ מִמֶּֽנּוּ׃
ח֧וּסָה יְיָ֣ עַל־יִשְׂרָאֵל עַמֶּֽךָ וְהַצּילֵֽנוּ מִכׇּל־רָע׃
חָטָֽאנוּ לְךָ אָדוֹן סְלַח־נָא כְּרֹב רַחֲמֶֽיךָ אֵל׃



\pesicha

%\brikhshmei

\gadlu

\avharachamim

\vesigale

\torahbarachu

\hagomel

\begin{small}
	
\misheberakhcholim{}

\misheberakhbaby

%\misheberakhbarmitzva


\end{small}



\sepline

\englishinst{After the Torah reading, Half Kaddish is recited:}
\halfkaddish


\hagbaha
יְהִי רָצוֹן מִלִּפְנֵי אָבִֽינוּ שֶׁבַּשָּׁמַֽיִם לְכוֹנֵן אֶת־בֵּית חַיֵּֽינוּ וּלְהָשִׁיב אֶת־שְׁכִינָתוֹ בְּתוֹכֵֽנוּ בִּמְהֵרָה בְיָמֵֽינוּ. וְנֹאמַר אָמֵן׃\\
יְהִי רָצוֹן מִלִּפְנֵי אָבִֽינוּ שֶׁבַּשָּׁמַֽיִם לְרַחֵם עָלֵֽינוּ וְעַל פְּלֵיטָתֵֽנוּ וְלִמְנֹֽעַ מַשְׁחִית וּמַגֵּפָה מֵעָלֵֽינוּ וּמֵעַל כׇּל עַמּוֹ בֵּית יִשְׂרָאֵל. וְנֹאמַר אָמֵן׃\\
יְהִי רָצוֹן מִלִּפְנֵי אָבִֽינוּ שֶׁבַּשָּׁמַֽיִם לְקַיֶּם־בָּֽנוּ חַכְמֵי יִשְׂרָאֵל.הֵם וּמִשְפְּחוֹתֵיהֶם וְתַלְמִידֵיהֶם וְתַלְמִידֵי תַלְמִידֵיהֶם בְּכׇל־מְקוֹמוֹת מֹושְׁבוֹתֵיהֶם וְנֹאמַר אָמֵן׃\\
יְהִי רָצוֹן מִלִּפְנֵי אָבִֽינוּ שֶׁבַּשָּׁמַֽיִם שֶׁנִּשְׁמַע וְנִתְבַּשֵּׂר בְּשׂוֹרוֹת טוֹבוֹת יְשׁוּעוֹת וְנֶחָמוֹת. וִיקַבֵּץ נִדָּחֵֽינוּ מֵאַרְבַּע כַּנְפוֹת הָאָֽרֶץ. וְנֹאמַר אָמֵן׃\\
אַחֵֽינוּ כׇּל בֵּית יִשְׂרָאֵל הַנְּתוּנִים בְּצָּרָה וּבְשִּׁבְיָה. הָעוֹמְדִים בֵּין בַּיָּם וּבֵין בַּיַּבָּשָׁה. הַמָּקוֹם יְרַחֵם עֲלֵיהֶם וְיוֹצִיאֵם מִצָּרָה לִרְוָחָה וּמֵאֲפֵלָה לְאוֹרָה וּמִשִּׁעְבּוּד לִגְאֻלָּה הַשְׁתָּא בַּעֲגָלָא וּבִזְמַן קָרִיב. וְנֹאמַר אָמֵן׃\\

\yehalelu

\englishinst{As the Torah is returned to the ark:}
\kafdalet

\etzchaim
\englishinst{The ark is closed.}

\section[סיום התפילה]{\adforn{53} סיום התפילה \adforn{25}}
%\section[סיום התפילה Prayers Concluding]{\adforn{53} סיום התפילה Prayers Concluding \adforn{25}}
\label{ashrei}
\ashrei

%\instruction{אין אומרים למנצח בראש חודש, חנוכה, י״ד וט״ו אדר א׳, פורים ושושן פורים, ערב פסח, חול המועד, אסרו חג, תשעה באב, ערב יום כפור, ובבית אבל}
\englishinst{The following Psalm is omitted on Rosh \d{H}odesh, \d{H}anukka, Purim and Shushan Purim, Purim Katan and Shushan Purim Katan, the day before Passover and Yom Kippur, Intermediate Festival Days, the day following festivals, the 9th of Av, or in a house of mourning.}
\firstword{לַמְנַצֵּ֗חַ מִזְמ֥וֹר לְדָוִֽד׃}\source{תהלים כ}
יַֽעַנְךָ֣ יְיָ֭ בְּי֣וֹם צָרָ֑ה יְ֝שַׂגֶּבְךָ֗ שֵׁ֤ם ׀ אֱלֹהֵ֬י יַעֲקֹֽב׃
יִשְׁלַֽח־עֶזְרְךָ֥ מִקֹּ֑דֶשׁ וּ֝מִצִּיּ֗וֹן יִסְעָדֶֽךָּ׃
יִזְכֹּ֥ר כׇּל־מִנְחֹתֶ֑ךָ וְעוֹלָתְךָ֖ יְדַשְּׁנֶ֣ה סֶֽלָה׃
יִֽתֶּן־לְךָ֥ כִלְבָבֶ֑ךָ וְֽכׇל־עֲצָתְךָ֥ יְמַלֵּֽא׃
נְרַנְּנָ֤ה ׀ בִּ֘ישׁ֤וּעָתֶ֗ךָ וּבְשֵֽׁם־אֱלֹהֵ֥ינוּ נִדְגֹּ֑ל יְמַלֵּ֥א יְ֝יָ֗ כׇּל־מִשְׁאֲלוֹתֶֽיךָ׃
עַתָּ֤ה יָדַ֗עְתִּי כִּ֤י הוֹשִׁ֥יעַ ׀ יְיָ֗ מְשִׁ֫יח֥וֹ יַ֭עֲנֵהוּ מִשְּׁמֵ֣י קׇדְשׁ֑וֹ בִּ֝גְבֻר֗וֹת יֵ֣שַׁע יְמִינֽוֹ׃
אֵ֣לֶּה בָ֭רֶכֶב וְאֵ֣לֶּה בַסּוּסִ֑ים וַאֲנַ֓חְנוּ ׀ בְּשֵׁם־יְיָ֖ אֱלֹהֵ֣ינוּ נַזְכִּֽיר׃
הֵ֭מָּה כָּרְע֣וּ וְנָפָ֑לוּ וַאֲנַ֥חְנוּ קַּ֝֗מְנוּ וַנִּתְעוֹדָֽד׃
יְיָ֥ הוֹשִׁ֑יעָה הַ֝מֶּ֗לֶךְ יַעֲנֵ֥נוּ בְיוֹם־קׇרְאֵֽנוּ׃

%\instruction{אין אומרים ואני זאת בריתי בבית אבל או בט״ב}
\englishinst{The line \hebineng{וכו׳ בריתי זאת ואני} is omitted in a house of mourning and on the 9th of Av.}
\uvaletzion

%ֺ\instruction{בראש חדש אומרים כאן חצי קדיש ומוסף עמ׳ \pageref{musaphrh}}\\
\englishinst{On Rosh \d{H}odesh, Half-Kaddish is recited followed by Musaf on page \pageref{musaphrh}.}
%\ifboolexpr{togl {includeChM}}{\instruction{בחול המועד אומרים כאן חצי קדיש ומוסף עמ׳ \pageref{musaphregel}}}{}\\
\ifboolexpr{togl {includeChM}}{\englishinst{On Intermediate Festival Days, Half-Kaddish is recited followed by Musaf on page \pageref{musaphregel}.}}{}\\


\label{end of shacharis}
\fullkaddish

\aleinu

\section[שיר של יום]{\adforn{53} שיר של יום \adforn{25}}
%\section[שיר של יום Psalms Daily]{\adforn{53} שיר של יום Psalms Daily \adforn{25}}
\label{shir_shel_yom}


\weekdayshir



%\instruction{באלול תוקעים בשופר, אבל לא בערה״ש:}
\englishinst{During the month of Elul the Shofar is blown here, except on the day before Rosh HaShana.}
תקיעה שברים תרועה תקיעה\\
\ledavid\\

\englishinst{On \d{H}anukka most congregations recite:}
\instruction{בחנכה׃}
\chanukat

\label{kaddish_yasom_shacharis}
\mournerskaddish

\begin{sometimes}

% \instruction{בבית אבל אומרים׃ }
\englishinst{In a house of mourning, on days Ta\d{h}anun is said:}
\source{תהלים מט}\firstword{לַמְנַצֵּ֬חַ ׀ לִבְנֵי־קֹ֬רַח מִזְמֽוֹר׃}
שִׁמְעוּ־זֹ֭את כׇּל־הָעַמִּ֑ים הַ֝אֲזִ֗ינוּ כׇּל־יֹ֥שְׁבֵי חָֽלֶד׃
גַּם־בְּנֵ֣י אָ֭דָם גַּם־בְּנֵי־אִ֑ישׁ יַ֗֝חַד עָשִׁ֥יר וְאֶבְיֽוֹן׃
פִּ֭י יְדַבֵּ֣ר חׇכְמ֑וֹת וְהָג֖וּת לִבִּ֣י תְבוּנֽוֹת׃
אַטֶּ֣ה לְמָשָׁ֣ל אׇזְנִ֑י אֶפְתַּ֥ח בְּ֝כִנּ֗וֹר חִידָתִֽי׃
לָ֣מָּה אִ֭ירָא בִּ֣ימֵי רָ֑ע עֲוֺ֖ן עֲקֵבַ֣י יְסוּבֵּֽנִי׃
הַבֹּטְחִ֥ים עַל־חֵילָ֑ם וּבְרֹ֥ב עׇ֝שְׁרָ֗ם יִתְהַלָּֽלוּ׃
אָ֗ח לֹא־פָדֹ֣ה יִפְדֶּ֣ה אִ֑ישׁ לֹא־יִתֵּ֖ן לֵאלֹהִ֣ים כׇּפְרֽוֹ׃
וְ֭יֵקַר פִּדְי֥וֹן נַפְשָׁ֗ם וְחָדַ֥ל לְעוֹלָֽם׃
וִיחִי־ע֥וֹד לָנֶ֑צַח לֹ֖א יִרְאֶ֣ה הַשָּֽׁחַת׃
כִּ֤י יִרְאֶ֨ה ׀ חֲכָ֘מִ֤ים יָמ֗וּתוּ יַ֤חַד כְּסִ֣יל וָבַ֣עַר יֹאבֵ֑דוּ וְעָזְב֖וּ לַאֲחֵרִ֣ים חֵילָֽם׃
קִרְבָּ֤ם בָּתֵּ֨ימוֹ ׀ לְֽעוֹלָ֗ם מִ֭שְׁכְּנֹתָם לְד֣וֹר וָדֹ֑ר קָרְא֥וּ בִ֝שְׁמוֹתָ֗ם עֲלֵ֣י אֲדָמֽוֹת׃
וְאָדָ֣ם בִּ֭יקָר בַּל־יָלִ֑ין נִמְשַׁ֖ל כַּבְּהֵמ֣וֹת נִדְמֽוּ׃
זֶ֣ה דַ֭רְכָּם כֵּ֣סֶל לָ֑מוֹ וְאַחֲרֵיהֶ֓ם ׀ בְּפִיהֶ֖ם יִרְצ֣וּ סֶֽלָה׃
כַּצֹּ֤אן ׀ לִ֥שְׁא֣וֹל שַׁתּוּ֮ מָ֤וֶת יִ֫רְעֵ֥ם וַיִּרְדּ֘וּ בָ֤ם יְשָׁרִ֨ים ׀ לַבֹּ֗קֶר ְ֭צוּרָם לְבַלּ֥וֹת שְׁא֗וֹל מִזְּבֻ֥ל לֽוֹ׃
אַךְ־אֱלֹהִ֗ים יִפְדֶּ֣ה נַ֭פְשִׁי מִֽיַּד־שְׁא֑וֹל כִּ֖י יִקָּחֵ֣נִי סֶֽלָה׃
אַל־תִּ֭ירָא כִּֽי־יַעֲשִׁ֣ר אִ֑ישׁ כִּי־יִ֝רְבֶּ֗ה כְּב֣וֹד בֵּיתֽוֹ׃
כִּ֤י לֹ֣א בְ֭מוֹתוֹ יִקַּ֣ח הַכֹּ֑ל לֹֽא־יֵרֵ֖ד אַחֲרָ֣יו כְּבוֹדֽוֹ׃
כִּֽי־נַ֭פְשׁוֹ בְּחַיָּ֣יו יְבָרֵ֑ךְ וְ֝יוֹדֻ֗ךָ כִּי־תֵיטִ֥יב לָֽךְ׃
תָּ֭בוֹא עַד־דּ֣וֹר אֲבוֹתָ֑יו עַד־נֵ֗֝צַח לֹ֣א יִרְאוּ־אֽוֹר׃
אָדָ֣ם בִּ֭יקָר וְלֹ֣א יָבִ֑ין נִמְשַׁ֖ל כַּבְּהֵמ֣וֹת נִדְמֽוּ׃


\sepline

%\instruction{בבית אבל בימים שאין אומרים תחנון׃}
\englishinst{In a house of mourning, on days Ta\d{h}anun is omitted:}
\source{תהלים טז}\firstword{מִכְתָּ֥ם לְדָוִ֑ד}
שׇֽׁמְרֵ֥נִי אֵ֝֗ל כִּֽי־חָסִ֥יתִי בָֽךְ׃
אָמַ֣רְתְּ לַֽייָ֭ אֲדֹנָ֣י אָ֑תָּה ט֝וֹבָתִ֗י בַּל־עָלֶֽיךָ׃
לִ֭קְדוֹשִׁים אֲשֶׁר־בָּאָ֣רֶץ הֵ֑מָּה וְ֝אַדִּירֵ֗י כׇּל־חֶפְצִי־בָֽם׃
יִרְבּ֥וּ עַצְּבוֹתָם֮ אַחֵ֢ר מָ֫הָ֥רוּ בַּל־אַסִּ֣יךְ נִסְכֵּיהֶ֣ם מִדָּ֑ם וּֽבַל־אֶשָּׂ֥א אֶת־שְׁ֝מוֹתָ֗ם עַל־שְׂפָתָֽי׃
יְיָ֗ מְנָת־חֶלְקִ֥י וְכוֹסִ֑י אַ֝תָּ֗ה תּוֹמִ֥יךְ גּוֹרָלִֽי׃
חֲבָלִ֣ים נָֽפְלוּ־לִ֭י בַּנְּעִמִ֑ים אַף־נַ֝חֲלָ֗ת שָֽׁפְרָ֥ה עָלָֽי׃
אֲבָרֵ֗ךְ אֶת־יְיָ֭ אֲשֶׁ֣ר יְעָצָ֑נִי אַף־לֵ֝יל֗וֹת יִסְּר֥וּנִי כִלְיוֹתָֽי׃
שִׁוִּ֬יתִי יְיָ֣ לְנֶגְדִּ֣י תָמִ֑יד כִּ֥י מִֽ֝ימִינִ֗י בַּל־אֶמּֽוֹט׃
לָכֵ֤ן ׀ שָׂמַ֣ח לִ֭בִּי וַיָּ֣גֶל כְּבוֹדִ֑י אַף־בְּ֝שָׂרִ֗י יִשְׁכֹּ֥ן לָבֶֽטַח׃
כִּ֤י ׀ לֹא־תַעֲזֹ֣ב נַפְשִׁ֣י לִשְׁא֑וֹל לֹֽא־תִתֵּ֥ן חֲ֝סִידְךָ֗ לִרְא֥וֹת שָֽׁחַת׃
תּֽוֹדִיעֵנִי֮ אֹ֤רַח חַ֫יִּ֥ים שֹׂ֣בַע שְׂ֭מָחוֹת אֶת־פָּנֶ֑יךָ נְעִמ֖וֹת בִּימִינְךָ֣ נֶֽצַח׃

\end{sometimes}


%\instruction{בצאתו מבית הכנסת:}\\
%\source{תהלים ה} יְיָ֤ נְחֵ֬נִי בְצִדְקָתֶ֗ךָ לְמַ֥עַן שׁוֹרְרָ֑י הַיְשַׁ֖ר לְפָנַ֣י דַּרְכֶּֽךָ׃

\adforn{43}\quad\adforn{4}\quad\adforn{42}\\

%\chapter[מוסף לראש חודש Months New for Musaf]{\adforn{47} Months New for Musaf \adforn{19}\\מוסף לראש חודש בחול}
\chapter[מוסף לראש חודש]{\adforn{47} מוסף לראש חודש בחול \adforn{19}}
\label{musaphrh}

\specialsaavos

\specialsameisim

\englishinst{During the repetition of the Amidah, Kedusha is said here}
\instruction{בחזרת הש״ץ אומרים קדושה כאן}

\firstword{אַתָּה}
קָדוֹשׁ וְשִׁמְךָ קָדוֹשׁ וּקְדוֹשִׁים בְּכׇל־יוֹם יְהַלְלוּךָ סֶּֽלָה׃ בָּרוּךְ אַתָּה יְיָ הָאֵל הַקָּדוֹשׁ׃\\

\kedusmusafchol{קדושה}{}

\firstword{רָאשֵׁי חֳדָשִׁים}
לְעַמְּךָ נָתַֽתָּ זְמַן כַּפָּרָה לְכׇל־תּוֹלְדוֹתָם בִּהְיוֹתָם מַקְרִיבִים לְפָנֶֽיךָ זִבְחֵי רָצוֹן וּשְׂעִירֵי חַטָּאת לְכַפֵּר בַּעֲדָם׃ זִכָּרוֹן לְכֻלָּם יִהְיוּ תְּשׁוּעַת נַפְשָׁם מִיַּד שׂוֹנֵא׃ מִזְבֵּֽחַ חָדָשׁ בְּצִיּוֹן תָּכִין וְעוֹלַת רֹאשׁ חֹֽדֶשׁ נַעֲלֶה עָלָיו וּשְׂעִירֵי עִזִּים נַעֲשֶׂה בְרָצוֹן׃ וּבַעֲבוֹדַת בֵּית הַמִּקְדָּשׁ נִשְׂמַח כֻּלָּֽנוּ וְשִׁירֵי דָוִד עַבְדֶּֽךָ נִּשְׁמָעִים בְּעִירֶֽךָ הָאֲמוּרִים לִפְנֵי מִזְבְּחֶֽךָ׃ אַהֲבַת עוֹלָם תָּבִיא לָהֶם וּבְרִית אָבוֹת לַבָּנִים תִּזְכּוֹר׃ וַהֲבִיאֵֽנוּ לְצִיּוֹן עִירְךָ בְּרִנָּה וְלִירוּשָׁלַ‍ִם בֵּית מִקְדָשְׁךָ בְּשִׂמְחַת עוֹלָם׃ וְשָׁם נַעֲשֶׂה לְפָנֶֽיךָ אֶת־קׇרְבְּנוֹת חוֹבוֹתֵֽינוּ תְּמִידִים כְּסִדְרָם וּמוּסָפִים כְּהִלְכָתָם׃ וְאֶת מוּסַף
יוֹם רֹאשׁ הַחֹֽדֶשׁ
הַזֶּה נַעֲשֶׂה וְנַקְרִיב לְפָנֶֽיךָ בְּאַהֲבָה כְּמִצְוַת רְצוֹנֶֽךָ כְּמוֹ שֶׁכָּתַֽבְתָּ עָלֵֽינוּ בְּתוֹרָתֶֽךָ עַל יְדֵי מֹשֶׁה עַבְדְּךָ מִפִּי כְבוֹדֶֽךָ כָּאָמוּר׃\\
\firstword{וּבְרָאשֵׁי֙ חׇדְשֵׁיכֶ֔ם}\source{במדבר כח}
תַּקְרִ֥יבוּ עֹלָ֖ה לַייָ֑ פָּרִ֨ים בְּנֵֽי־בָקָ֤ר שְׁנַ֙יִם֙ וְאַ֣יִל אֶחָ֔ד כְּבָשִׂ֧ים בְּנֵי־שָׁנָ֛ה שִׁבְעָ֖ה תְּמִימִֽם׃ וּמִנְחָתָם וְנִסְכֵּיהֶם כִּמְדֻבָּר שְׁלֹשָׁה עֶשְׂרֹנִים לַפָּר וּשְׁנֵי עֶשְׂרֹנִים לָאָֽיִל וְעִשָּׂרוֹן לַכֶּֽבֶשׂ וְיַֽיִן כְּנִסְכּוֹ וְשָׂעִיר לְכַפֵּר וּשְׁנֵי תְמִידִים כְּהִלְכָתָם׃

\firstword{אֱלֹהֵֽינוּ}
וֵאלֹהֵי אֲבוֹתֵֽינוּ חַדֵּשׁ עָלֵֽינוּ אֶת־הַחֹֽדֶשׁ הַזֶּה לְטוֹבָה וְלִבְרָכָה לְשָׂשׂוֹן וּלְשִׂמְחָה לִישׁוּעָה וּלְנֶחָמָה לְפַרְנָסָה וּלְכַלְכָּלָה לְחַיִּים וּלְשָׁלוֹם לִמְחִֽילַת חֵטְא וְלִסְלִיחַת עָוֹן [\instruction{בשנה העבור עד בכלל ר״ח אדר ב׳:}
וּלְכַפָּרַת פָּֽשַׁע]׃ כִּי בְעַמְּךָ יִשְׂרָאֵל בָּחַֽרְתָּ מִכׇּל־הָאֻמּוֹת וְחֻקֵּי רָאשֵׁי חֳדָשִׁים לָהֶם קָבָֽעְתָּ׃ בָּרוּךְ אַתָּה יְיָ מְקַדֵּשׁ יִשְׂרָאֵל וְרָאשֵׁי חֳדָשִׁים׃

\firstword{רְצֵה}
יְיָ אֱלֹהֵֽינוּ בְּעַמְּךָ יִשְׂרָאֵל וּבִתְפִלָּתָם וְהָשֵׁב הָעֲבוֹדָה לִדְבִיר בֵּיתֶֽךָ׃ וְאִשֵּׁי יִשְׂרָאֵל וּתְפִלָּתָם בְּאַהֲבָה תְקַבֵּל בְּרָצוֹן וּתְהִי לְרָצוֹן תָּמִיד עֲבוֹדַת יִשְׂרָאֵל עַמֶּֽךָ׃ וְתֶחֱזֶֽינָה עֵינֵֽינוּ בְּשׁוּבְךָ לְצִיּוֹן בְּרַחֲמִים׃ בָּרוּךְ אַתָּה יְיָ הַמַּחֲזִיר שְׁכִינָתוֹ לְצִיּוֹן׃

\modim

\enlargethispage{\baselineskip}

\begin{sometimes}

\instruction{בחנוכה:}
\textbf{עַל הַנִּסִּים}
וְעַל הַפֻּרְקָן וְעַל הַגְּבוּרוֹת וְעַל הַתְּשׁוּעוֹת וְעַל הַמִּלְחָמוֹת
שֶׁעָשִֽׂיתָ לַאֲבוֹתֵֽינוּ בַּיָּמִים הָהֵם בַּזְּמַן הַזֶּה׃
בִּימֵי מַתִּתְיָֽהוּ בֶּן יוֹחָנָן כֹּהֵן גָּדוֹל חַשְׁמֹנַי וּבָנָיו כְּשֶׁעָמְדָה מַלְכוּת יָוָן הָרְשָׁעָה עַל עַמְּךָ יִשְׂרָאֵל לְהַשְׁכִּיחָם תּוֹרָתֶֽךָ וּלְהַעֲבִירָם מֵחֻקֵּי רְצוֹנֶֽךָ׃ וְאַתָּה בְּרַחֲמֶֽיךָ הָרַבִּים עָמַֽדְתָּ לָהֶם בְּעֵת צָרָתָם רַֽבְתָּ אֶת־רִיבָם דַּֽנְתָּ אֶת־דִּינָם נָקַֽמְתָּ אֶת־נִקְמָתָם׃ מָסַֽרְתָּ גִּבּוֹרִים בְּיַד חַלָּשִׁים וְרַבִּים בְּיַד מְעַטִּים וּטְמֵאִים בְּיַד טְהוֹרִים וּרְשָׁעִים בְּיַד צַדִּיקִים וְזֵדִים בְּיַד עוֹסְקֵי תוֹרָתֶֽךָ׃ וּלְךָ עָשִֽׂיתָ שֵׁם גָּדוֹל וְקָדוֹשׁ בְּעוֹלָמֶֽךָ וּלְעַמְּךָ יִשְׂרָאֵל עָשִֽׂיתָ תְּשׁוּעָה גְדוֹלָה וּפֻרְקָן כְּהַיּוֹם הַזֶּה׃ וְאַֽחַר כַּךְ בָּֽאוּ בָנֶֽיךָ לִדְבִיר בֵּיתֶֽךָ וּפִנּוּ אֶת־הֵיכָלֶֽךָ וְטִהֲרוּ אֶת־מִקְדָּשֶֽׁךָ וְהִדְלִֽיקוּ נֵרוֹת בְּחַצְרוֹת קׇדְּֿשֶֽׁךָ וְקָבְעוּ שְׁמוֹנַת יְמֵי חֲנֻכָּה אֵֽלּוּ לְהוֹדוֹת לְהַלֵּל לְשִׁמְךָ הַגָּדוֹל׃

\end{sometimes}

\firstword{וְעַל כֻּלָּם}
יִתְבָּרַךְ וְיִתְרוֹמַם שִׁמְךָ מַלְכֵּֽנוּ תָּמִיד לְעוֹלָם וָעֶד׃
וְכׇל־הַחַיִּים יוֹדֽוּךָ סֶּֽלָה וִיהַלְלוּ אֶת־שִׁמְךָ בֶּאֱמֶת הָאֵל יְשׁוּעָתֵֽנוּ וְעֶזְרָתֵֽנוּ סֶֽלָה׃ בָּרוּךְ אַתָּה יְיָ הַטּוֹב שִׁמְךָ וּלְךָ נָאֶה לְהוֹדוֹת׃

\shatzbirkaskohanimenglish{During the repetition of the Amidah:}
\simshalomplain\space\vetov\space
בָּרוּךְ אַתָּה יְיָ הַמְבָרֵךְ אֶת־עַמּוֹ יִשְׂרָאֵל בַּשָּׁלוֹם׃

\tachanunim

\fullkaddish

\aleinu

\mournerskaddish

%\chapter[תפילות נוספות Prayers Other]{\adforn{53} Prayers Other \adforn{25}\\ תפילות נוספות }
\chapter[תפילות נוספות]{\adforn{53} תפילות נוספות \adforn{25}\\}

\section[עשרת הדברות]{עשרת הדברות}

\begin{footnotesize}
	אָֽנֹכִי֙ יְיָ֣ אֱלֹהֶ֔יךָ אֲשֶׁ֧ר הוֹצֵאתִ֛יךָ מֵאֶ֥רֶץ מִצְרַ֖יִם מִבֵּ֣ית עֲבָדִ֑ים לֹֽא־יִהְיֶ֥ה לְךָ֛ אֱלֹהִ֥ים אֲחֵרִ֖ים עַל־פָּנָֽי׃ לֹֽא־תַעֲשֶׂ֨ה לְךָ֥ פֶ֙סֶל֙ וְכׇל־תְּמוּנָ֔ה אֲשֶׁ֤ר בַּשָּׁמַ֙יִם֙ מִמַּ֔עַל וַֽאֲשֶׁ֥ר בָּאָ֖רֶץ מִתָּ֑חַת וַאֲשֶׁ֥ר בַּמַּ֖יִם מִתַּ֥חַת לָאָֽרֶץ׃ לֹֽא־תִשְׁתַּחֲוֶ֥ה לָהֶ֖ם וְלֹ֣א תׇעׇבְדֵ֑ם כִּ֣י אָֽנֹכִ֞י יְיָ֤ אֱלֹהֶ֙יךָ֙ אֵ֣ל קַנָּ֔א פֹּ֠קֵ֠ד עֲוֺ֨ן אָבֹ֧ת עַל־בָּנִ֛ים עַל־שִׁלֵּשִׁ֥ים וְעַל־רִבֵּעִ֖ים לְשֹׂנְאָֽי׃ וְעֹ֥שֶׂה חֶ֖סֶד לַאֲלָפִ֑ים לְאֹהֲבַ֖י וּלְשֹׁמְרֵ֥י מִצְוֺתָֽי׃	לֹ֥א תִשָּׂ֛א אֶת־שֵֽׁם־יְיָ֥ אֱלֹהֶ֖יךָ לַשָּׁ֑וְא כִּ֣י לֹ֤א יְנַקֶּה֙ יְיָ֔ אֵ֛ת אֲשֶׁר־יִשָּׂ֥א אֶת־שְׁמ֖וֹ לַשָּֽׁוְא׃\hfill\break
	זָכ֛וֹר אֶת־י֥וֹם הַשַּׁבָּ֖ת לְקַדְּשֽׁוֹ׃ שֵׁ֤שֶׁת יָמִים֙ תַּֽעֲבֹ֔ד וְעָשִׂ֖יתָ כׇּל־מְלַאכְתֶּֽךָ׃ וְיוֹם֙ הַשְּׁבִיעִ֔י שַׁבָּ֖ת לַייָ֣ אֱלֹהֶ֑יךָ לֹֽא־תַעֲשֶׂ֨ה כׇל־מְלָאכָ֜ה אַתָּ֣ה ׀ וּבִנְךָ֣ וּבִתֶּ֗ךָ עַבְדְּךָ֤ וַאֲמָֽתְךָ֙ וּבְהֶמְתֶּ֔ךָ וְגֵרְךָ֖ אֲשֶׁ֥ר בִּשְׁעָרֶֽיךָ׃ כִּ֣י שֵֽׁשֶׁת־יָמִים֩ עָשָׂ֨ה יְיָ֜ 
	אֶת־הַשָּׁמַ֣יִם וְאֶת־הָאָ֗רֶץ אֶת־הַיָּם֙ וְאֶת־כׇּל־אֲשֶׁר־בָּ֔ם וַיָּ֖נַח בַּיּ֣וֹם הַשְּׁבִיעִ֑י עַל־כֵּ֗ן בֵּרַ֧ךְ יְיָ֛  
	אֶת־י֥וֹם הַשַּׁבָּ֖ת וַֽיְקַדְּשֵֽׁהוּ׃	כַּבֵּ֥ד אֶת־אָבִ֖יךָ וְאֶת־אִמֶּ֑ךָ לְמַ֙עַן֙ יַאֲרִכ֣וּן יָמֶ֔יךָ עַ֚ל הָאֲדָמָ֔ה אֲשֶׁר־יְיָ֥  
	אֱלֹהֶ֖יךָ נֹתֵ֥ן לָֽךְ׃ לֹ֥א תִרְצָ֖ח לֹ֣א תִנְאָ֑ף לֹ֣א תִגְנֹ֔ב לֹֽא־תַעֲנֶ֥ה בְרֵעֲךָ֖ עֵ֥ד שָֽׁקֶר׃ לֹ֥א תַחְמֹ֖ד בֵּ֣ית רֵעֶ֑ךָ לֹֽא־תַחְמֹ֞ד אֵ֣שֶׁת רֵעֶ֗ךָ וְעַבְדּ֤וֹ וַאֲמָתוֹ֙ וְשׁוֹר֣וֹ וַחֲמֹר֔וֹ וְכֹ֖ל אֲשֶׁ֥ר לְרֵעֶֽךָ׃ 
\end{footnotesize}


\section[העקדה]{העקדה}

\begin{footnotesize}
וַיְהִ֗י\source{ברא׳ כב} אַחַר֙ הַדְּבָרִ֣ים הָאֵ֔לֶּה וְהָ֣אֱלֹהִ֔ים נִסָּ֖ה אֶת־אַבְרָהָ֑ם וַיֹּ֣אמֶר אֵלָ֔יו אַבְרָהָ֖ם וַיֹּ֥אמֶר הִנֵּֽנִי׃ וַיֹּ֡אמֶר קַח־נָ֠א אֶת־בִּנְךָ֨ אֶת־יְחִֽידְךָ֤ אֲשֶׁר־אָהַ֙בְתָּ֙ אֶת־יִצְחָ֔ק וְלֶ֨ךְ־לְךָ֔ אֶל־אֶ֖רֶץ הַמֹּרִיָּ֑ה וְהַעֲלֵ֤הוּ שָׁם֙ לְעֹלָ֔ה עַ֚ל אַחַ֣ד הֶֽהָרִ֔ים אֲשֶׁ֖ר אֹמַ֥ר אֵלֶֽיךָ׃ וַיַּשְׁכֵּ֨ם אַבְרָהָ֜ם בַּבֹּ֗קֶר וַֽיַּחֲבֹשׁ֙ אֶת־חֲמֹר֔וֹ וַיִּקַּ֞ח אֶת־שְׁנֵ֤י נְעָרָיו֙ אִתּ֔וֹ וְאֵ֖ת יִצְחָ֣ק בְּנ֑וֹ וַיְבַקַּע֙ עֲצֵ֣י עֹלָ֔ה וַיָּ֣קׇם וַיֵּ֔לֶךְ אֶל־הַמָּק֖וֹם אֲשֶׁר־אָֽמַר־ל֥וֹ הָאֱלֹהִֽים׃ בַּיּ֣וֹם הַשְּׁלִישִׁ֗י וַיִּשָּׂ֨א אַבְרָהָ֧ם אֶת־עֵינָ֛יו וַיַּ֥רְא אֶת־הַמָּק֖וֹם מֵרָחֹֽק׃ וַיֹּ֨אמֶר אַבְרָהָ֜ם אֶל־נְעָרָ֗יו שְׁבוּ־לָכֶ֥ם פֹּה֙ עִֽם־הַחֲמ֔וֹר וַאֲנִ֣י וְהַנַּ֔עַר נֵלְכָ֖ה עַד־כֹּ֑ה וְנִֽשְׁתַּחֲוֶ֖ה וְנָשׁ֥וּבָה אֲלֵיכֶֽם׃ וַיִּקַּ֨ח אַבְרָהָ֜ם אֶת־עֲצֵ֣י הָעֹלָ֗ה וַיָּ֙שֶׂם֙ עַל־יִצְחָ֣ק בְּנ֔וֹ וַיִּקַּ֣ח בְּיָד֔וֹ אֶת־הָאֵ֖שׁ וְאֶת־הַֽמַּאֲכֶ֑לֶת וַיֵּלְכ֥וּ שְׁנֵיהֶ֖ם יַחְדָּֽו׃ וַיֹּ֨אמֶר יִצְחָ֜ק אֶל־אַבְרָהָ֤ם אָבִיו֙ וַיֹּ֣אמֶר אָבִ֔י וַיֹּ֖אמֶר הִנֶּ֣נִּֽי בְנִ֑י וַיֹּ֗אמֶר הִנֵּ֤ה הָאֵשׁ֙ וְהָ֣עֵצִ֔ים וְאַיֵּ֥ה הַשֶּׂ֖ה לְעֹלָֽה׃ וַיֹּ֙אמֶר֙ אַבְרָהָ֔ם אֱלֹהִ֞ים יִרְאֶה־לּ֥וֹ הַשֶּׂ֛ה לְעֹלָ֖ה בְּנִ֑י וַיֵּלְכ֥וּ שְׁנֵיהֶ֖ם יַחְדָּֽו׃ וַיָּבֹ֗אוּ אֶֽל־הַמָּקוֹם֮ אֲשֶׁ֣ר אָֽמַר־ל֣וֹ הָאֱלֹהִים֒ וַיִּ֨בֶן שָׁ֤ם אַבְרָהָם֙ אֶת־הַמִּזְבֵּ֔חַ וַֽיַּעֲרֹ֖ךְ אֶת־הָעֵצִ֑ים וַֽיַּעֲקֹד֙ אֶת־יִצְחָ֣ק בְּנ֔וֹ וַיָּ֤שֶׂם אֹתוֹ֙ עַל־הַמִּזְבֵּ֔חַ מִמַּ֖עַל לָעֵצִֽים׃ וַיִּשְׁלַ֤ח אַבְרָהָם֙ אֶת־יָד֔וֹ וַיִּקַּ֖ח אֶת־הַֽמַּאֲכֶ֑לֶת לִשְׁחֹ֖ט אֶת־בְּנֽוֹ׃ וַיִּקְרָ֨א אֵלָ֜יו מַלְאַ֤ךְ יְיָ֙ מִן־הַשָּׁמַ֔יִם וַיֹּ֖אמֶר אַבְרָהָ֣ם ׀ אַבְרָהָ֑ם וַיֹּ֖אמֶר הִנֵּֽנִי׃ וַיֹּ֗אמֶר אַל־תִּשְׁלַ֤ח יָֽדְךָ֙ אֶל־הַנַּ֔עַר וְאַל־תַּ֥עַשׂ ל֖וֹ מְא֑וּמָה כִּ֣י ׀ עַתָּ֣ה יָדַ֗עְתִּי כִּֽי־יְרֵ֤א אֱלֹהִים֙ אַ֔תָּה וְלֹ֥א חָשַׂ֛כְתָּ אֶת־בִּנְךָ֥ אֶת־יְחִידְךָ֖ מִמֶּֽנִּי׃ וַיִּשָּׂ֨א אַבְרָהָ֜ם אֶת־עֵינָ֗יו וַיַּרְא֙ וְהִנֵּה־אַ֔יִל אַחַ֕ר נֶאֱחַ֥ז בַּסְּבַ֖ךְ בְּקַרְנָ֑יו וַיֵּ֤לֶךְ אַבְרָהָם֙ וַיִּקַּ֣ח אֶת־הָאַ֔יִל וַיַּעֲלֵ֥הוּ לְעֹלָ֖ה תַּ֥חַת בְּנֽוֹ׃ וַיִּקְרָ֧א אַבְרָהָ֛ם שֵֽׁם־הַמָּק֥וֹם הַה֖וּא יְיָ֣ ׀ יִרְאֶ֑ה אֲשֶׁר֙ יֵאָמֵ֣ר הַיּ֔וֹם בְּהַ֥ר יְיָ֖ יֵרָאֶֽה׃ וַיִּקְרָ֛א מַלְאַ֥ךְ יְיָ֖ אֶל־אַבְרָהָ֑ם שֵׁנִ֖ית מִן־הַשָּׁמָֽיִם׃ וַיֹּ֕אמֶר בִּ֥י נִשְׁבַּ֖עְתִּי נְאֻם־יְיָ֑ כִּ֗י יַ֚עַן אֲשֶׁ֤ר עָשִׂ֙יתָ֙ אֶת־הַדָּבָ֣ר הַזֶּ֔ה וְלֹ֥א חָשַׂ֖כְתָּ אֶת־בִּנְךָ֥ אֶת־יְחִידֶֽךָ׃ כִּֽי־בָרֵ֣ךְ אֲבָרֶכְךָ֗ וְהַרְבָּ֨ה אַרְבֶּ֤ה אֶֽת־זַרְעֲךָ֙ כְּכוֹכְבֵ֣י הַשָּׁמַ֔יִם וְכַח֕וֹל אֲשֶׁ֖ר עַל־שְׂפַ֣ת הַיָּ֑ם וְיִרַ֣שׁ זַרְעֲךָ֔ אֵ֖ת שַׁ֥עַר אֹיְבָֽיו׃ וְהִתְבָּרְכ֣וּ בְזַרְעֲךָ֔ כֹּ֖ל גּוֹיֵ֣י הָאָ֑רֶץ עֵ֕קֶב אֲשֶׁ֥ר שָׁמַ֖עְתָּ בְּקֹלִֽי׃ וַיָּ֤שׇׁב אַבְרָהָם֙ אֶל־נְעָרָ֔יו וַיָּקֻ֛מוּ וַיֵּלְכ֥וּ יַחְדָּ֖ו אֶל־בְּאֵ֣ר שָׁ֑בַע וַיֵּ֥שֶׁב אַבְרָהָ֖ם בִּבְאֵ֥ר שָֽׁבַע׃\hfill\break
\end{footnotesize}


\section[תפילת חנה]{תפילת חנה}

\begin{footnotesize}
וַתִּתְפַּלֵּ֤ל\source{ש״ב ב} חַנָּה֙ וַתֹּאמַ֔ר עָלַ֤ץ לִבִּי֙ בַּייָ֔ רָ֥מָה קַרְנִ֖י בַּייָ֑ רָ֤חַב פִּי֙ עַל־א֣וֹיְבַ֔י כִּ֥י שָׂמַ֖חְתִּי בִּישׁוּעָתֶֽךָ׃ אֵין־קָד֥וֹשׁ כַּייָ֖ כִּ֣י אֵ֣ין בִּלְתֶּ֑ךָ וְאֵ֥ין צ֖וּר כֵּאלֹהֵֽינוּ׃ אַל־תַּרְבּ֤וּ תְדַבְּרוּ֙ גְּבֹהָ֣ה גְבֹהָ֔ה יֵצֵ֥א עָתָ֖ק מִפִּיכֶ֑ם כִּ֣י אֵ֤ל דֵּעוֹת֙ יְיָ֔ \qk{וְל֥וֹ}{ולא} נִתְכְּנ֖וּ עֲלִלֽוֹת׃ קֶ֥שֶׁת גִּבֹּרִ֖ים חַתִּ֑ים וְנִכְשָׁלִ֖ים אָ֥זְרוּ חָֽיִל׃ שְׂבֵעִ֤ים בַּלֶּ֙חֶם֙ נִשְׂכָּ֔רוּ וּרְעֵבִ֖ים חָדֵ֑לּוּ עַד־עֲקָרָה֙ יָלְדָ֣ה שִׁבְעָ֔ה וְרַבַּ֥ת בָּנִ֖ים אֻמְלָֽלָה׃ יְיָ֖ מֵמִ֣ית וּמְחַיֶּ֑ה מוֹרִ֥יד שְׁא֖וֹל וַיָּֽעַל׃ יְיָ֖ מוֹרִ֣ישׁ וּמַעֲשִׁ֑יר מַשְׁפִּ֖יל אַף־מְרוֹמֵֽם׃ מֵקִ֨ים מֵעָפָ֜ר דָּ֗ל מֵֽאַשְׁפֹּת֙ יָרִ֣ים אֶבְי֔וֹן לְהוֹשִׁיב֙ עִם־נְדִיבִ֔ים וְכִסֵּ֥א כָב֖וֹד יַנְחִלֵ֑ם כִּ֤י לַֽייָ֙ מְצֻ֣קֵי אֶ֔רֶץ וַיָּ֥שֶׁת עֲלֵיהֶ֖ם תֵּבֵֽל׃ רַגְלֵ֤י חֲסִידָו֙ יִשְׁמֹ֔ר וּרְשָׁעִ֖ים בַּחֹ֣שֶׁךְ יִדָּ֑מּוּ כִּי־לֹ֥א בְכֹ֖חַ יִגְבַּר־אִֽישׁ׃ יְיָ֞ יֵחַ֣תּוּ מְרִיבָ֗ו עָלָו֙ בַּשָּׁמַ֣יִם יַרְעֵ֔ם יְיָ֖ יָדִ֣ין אַפְסֵי־אָ֑רֶץ וְיִתֶּן־עֹ֣ז לְמַלְכּ֔וֹ וְיָרֵ֖ם קֶ֥רֶן מְשִׁיחֽוֹ׃\hfill\break
\end{footnotesize}

\section[פרשת המן]{פרשת המן}

\begin{footnotesize}
	
וַיֹּ֤אמֶר\source{שמות טז} יְיָ֙ אֶל־מֹשֶׁ֔ה הִנְנִ֨י מַמְטִ֥יר לָכֶ֛ם לֶ֖חֶם מִן־הַשָּׁמָ֑יִם וְיָצָ֨א הָעָ֤ם וְלָֽקְטוּ֙ דְּבַר־י֣וֹם בְּיוֹמ֔וֹ לְמַ֧עַן אֲנַסֶּ֛נּוּ הֲיֵלֵ֥ךְ בְּתוֹרָתִ֖י אִם־לֹֽא׃ וְהָיָה֙ בַּיּ֣וֹם הַשִּׁשִּׁ֔י וְהֵכִ֖ינוּ אֵ֣ת אֲשֶׁר־יָבִ֑יאוּ וְהָיָ֣ה מִשְׁנֶ֔ה עַ֥ל אֲשֶֽׁר־יִלְקְט֖וּ י֥וֹם ׀ יֽוֹם׃ וַיֹּ֤אמֶר מֹשֶׁה֙ וְאַהֲרֹ֔ן אֶֽל־כׇּל־בְּנֵ֖י יִשְׂרָאֵ֑ל עֶ֕רֶב וִֽידַעְתֶּ֕ם כִּ֧י יְיָ֛ הוֹצִ֥יא אֶתְכֶ֖ם מֵאֶ֥רֶץ מִצְרָֽיִם׃ וּבֹ֗קֶר וּרְאִיתֶם֙ אֶת־כְּב֣וֹד יְיָ֔ בְּשׇׁמְע֥וֹ אֶת־תְּלֻנֹּתֵיכֶ֖ם עַל־יְיָ֑ וְנַ֣חְנוּ מָ֔ה כִּ֥י \qk{תַלִּ֖ינוּ}{תלונו} עָלֵֽינוּ׃ וַיֹּ֣אמֶר מֹשֶׁ֗ה בְּתֵ֣ת יְיָ֩ לָכֶ֨ם בָּעֶ֜רֶב בָּשָׂ֣ר לֶאֱכֹ֗ל וְלֶ֤חֶם בַּבֹּ֙קֶר֙ לִשְׂבֹּ֔עַ בִּשְׁמֹ֤עַ יְיָ֙ אֶת־תְּלֻנֹּ֣תֵיכֶ֔ם אֲשֶׁר־אַתֶּ֥ם מַלִּינִ֖ם עָלָ֑יו וְנַ֣חְנוּ מָ֔ה לֹא־עָלֵ֥ינוּ תְלֻנֹּתֵיכֶ֖ם כִּ֥י עַל־יְיָ׃ וַיֹּ֤אמֶר מֹשֶׁה֙ אֶֽל־אַהֲרֹ֔ן אֱמֹ֗ר אֶֽל־כׇּל־עֲדַת֙ בְּנֵ֣י יִשְׂרָאֵ֔ל קִרְב֖וּ לִפְנֵ֣י יְיָ֑ כִּ֣י שָׁמַ֔ע אֵ֖ת תְּלֻנֹּתֵיכֶֽם׃ וַיְהִ֗י כְּדַבֵּ֤ר אַהֲרֹן֙ אֶל־כׇּל־עֲדַ֣ת בְּנֵֽי־יִשְׂרָאֵ֔ל וַיִּפְנ֖וּ אֶל־הַמִּדְבָּ֑ר וְהִנֵּה֙ כְּב֣וֹד יְיָ֔ נִרְאָ֖ה בֶּעָנָֽן׃ \hfill\break
וַיְדַבֵּ֥ר יְיָ֖ אֶל־מֹשֶׁ֥ה לֵּאמֹֽר׃ שָׁמַ֗עְתִּי אֶת־תְּלוּנֹּת֮ בְּנֵ֣י יִשְׂרָאֵל֒ דַּבֵּ֨ר אֲלֵהֶ֜ם לֵאמֹ֗ר בֵּ֤ין הָֽעַרְבַּ֙יִם֙ תֹּאכְל֣וּ בָשָׂ֔ר וּבַבֹּ֖קֶר תִּשְׂבְּעוּ־לָ֑חֶם וִֽידַעְתֶּ֕ם כִּ֛י אֲנִ֥י יְיָ֖ אֱלֹהֵיכֶֽם׃ וַיְהִ֣י בָעֶ֔רֶב וַתַּ֣עַל הַשְּׂלָ֔ו וַתְּכַ֖ס אֶת־הַֽמַּחֲנֶ֑ה וּבַבֹּ֗קֶר הָֽיְתָה֙ שִׁכְבַ֣ת הַטַּ֔ל סָבִ֖יב לַֽמַּחֲנֶֽה׃ וַתַּ֖עַל שִׁכְבַ֣ת הַטָּ֑ל וְהִנֵּ֞ה עַל־פְּנֵ֤י הַמִּדְבָּר֙ דַּ֣ק מְחֻסְפָּ֔ס דַּ֥ק כַּכְּפֹ֖ר עַל־הָאָֽרֶץ׃ וַיִּרְא֣וּ בְנֵֽי־יִשְׂרָאֵ֗ל וַיֹּ֨אמְר֜וּ אִ֤ישׁ אֶל־אָחִיו֙ מָ֣ן ה֔וּא כִּ֛י לֹ֥א יָדְע֖וּ מַה־ה֑וּא וַיֹּ֤אמֶר מֹשֶׁה֙ אֲלֵהֶ֔ם ה֣וּא הַלֶּ֔חֶם אֲשֶׁ֨ר נָתַ֧ן יְיָ֛ לָכֶ֖ם לְאׇכְלָֽה׃ זֶ֤ה הַדָּבָר֙ אֲשֶׁ֣ר צִוָּ֣ה יְיָ֔ לִקְט֣וּ מִמֶּ֔נּוּ אִ֖ישׁ לְפִ֣י אׇכְל֑וֹ עֹ֣מֶר לַגֻּלְגֹּ֗לֶת מִסְפַּר֙ נַפְשֹׁ֣תֵיכֶ֔ם אִ֛ישׁ לַאֲשֶׁ֥ר בְּאׇהֳל֖וֹ תִּקָּֽחוּ׃ וַיַּעֲשׂוּ־כֵ֖ן בְּנֵ֣י יִשְׂרָאֵ֑ל וַֽיִּלְקְט֔וּ הַמַּרְבֶּ֖ה וְהַמַּמְעִֽיט׃ וַיָּמֹ֣דּוּ בָעֹ֔מֶר וְלֹ֤א הֶעְדִּיף֙ הַמַּרְבֶּ֔ה וְהַמַּמְעִ֖יט לֹ֣א הֶחְסִ֑יר אִ֥ישׁ לְפִֽי־אׇכְל֖וֹ לָקָֽטוּ׃ וַיֹּ֥אמֶר מֹשֶׁ֖ה אֲלֵהֶ֑ם אִ֕ישׁ אַל־יוֹתֵ֥ר מִמֶּ֖נּוּ עַד־בֹּֽקֶר׃ וְלֹא־שָׁמְע֣וּ אֶל־מֹשֶׁ֗ה וַיּוֹתִ֨רוּ אֲנָשִׁ֤ים מִמֶּ֙נּוּ֙ עַד־בֹּ֔קֶר וַיָּ֥רֻם תּוֹלָעִ֖ים וַיִּבְאַ֑שׁ וַיִּקְצֹ֥ף עֲלֵהֶ֖ם מֹשֶֽׁה׃ וַיִּלְקְט֤וּ אֹתוֹ֙ בַּבֹּ֣קֶר בַּבֹּ֔קֶר אִ֖ישׁ כְּפִ֣י אׇכְל֑וֹ וְחַ֥ם הַשֶּׁ֖מֶשׁ וְנָמָֽס׃ וַיְהִ֣י ׀ בַּיּ֣וֹם הַשִּׁשִּׁ֗י לָֽקְט֥וּ לֶ֙חֶם֙ מִשְׁנֶ֔ה שְׁנֵ֥י הָעֹ֖מֶר לָאֶחָ֑ד וַיָּבֹ֙אוּ֙ כׇּל־נְשִׂיאֵ֣י הָֽעֵדָ֔ה וַיַּגִּ֖ידוּ לְמֹשֶֽׁה׃ וַיֹּ֣אמֶר אֲלֵהֶ֗ם ה֚וּא אֲשֶׁ֣ר דִּבֶּ֣ר יְיָ֔ שַׁבָּת֧וֹן שַׁבַּת־קֹ֛דֶשׁ לַֽייָ֖ מָחָ֑ר אֵ֣ת אֲשֶׁר־תֹּאפ֞וּ אֵפ֗וּ וְאֵ֤ת אֲשֶֽׁר־תְּבַשְּׁלוּ֙ בַּשֵּׁ֔לוּ וְאֵת֙ כׇּל־הָ֣עֹדֵ֔ף הַנִּ֧יחוּ לָכֶ֛ם לְמִשְׁמֶ֖רֶת עַד־הַבֹּֽקֶר׃ וַיַּנִּ֤יחוּ אֹתוֹ֙ עַד־הַבֹּ֔קֶר כַּאֲשֶׁ֖ר צִוָּ֣ה מֹשֶׁ֑ה וְלֹ֣א הִבְאִ֔ישׁ וְרִמָּ֖ה לֹא־הָ֥יְתָה בּֽוֹ׃ וַיֹּ֤אמֶר מֹשֶׁה֙ אִכְלֻ֣הוּ הַיּ֔וֹם כִּֽי־שַׁבָּ֥ת הַיּ֖וֹם לַייָ֑ הַיּ֕וֹם לֹ֥א תִמְצָאֻ֖הוּ בַּשָּׂדֶֽה׃ שֵׁ֥שֶׁת יָמִ֖ים תִּלְקְטֻ֑הוּ וּבַיּ֧וֹם הַשְּׁבִיעִ֛י שַׁבָּ֖ת לֹ֥א יִֽהְיֶה־בּֽוֹ׃ וַֽיְהִי֙ בַּיּ֣וֹם הַשְּׁבִיעִ֔י יָצְא֥וּ מִן־הָעָ֖ם לִלְקֹ֑ט וְלֹ֖א מָצָֽאוּ׃  \hfill וַיֹּ֥אמֶר יְיָ֖ אֶל־מֹשֶׁ֑ה עַד־אָ֙נָה֙ מֵֽאַנְתֶּ֔ם לִשְׁמֹ֥ר מִצְוֺתַ֖י וְתוֹרֹתָֽי׃ רְא֗וּ כִּֽי־יְיָ נָתַ֣ן לָכֶ֣ם הַשַּׁבָּת֒ עַל־כֵּ֠ן ה֣וּא נֹתֵ֥ן לָכֶ֛ם בַּיּ֥וֹם הַשִּׁשִּׁ֖י לֶ֣חֶם יוֹמָ֑יִם שְׁב֣וּ ׀ אִ֣ישׁ תַּחְתָּ֗יו אַל־יֵ֥צֵא אִ֛ישׁ מִמְּקֹמ֖וֹ בַּיּ֥וֹם הַשְּׁבִיעִֽי׃ וַיִּשְׁבְּת֥וּ הָעָ֖ם בַּיּ֥וֹם הַשְּׁבִעִֽי׃ וַיִּקְרְא֧וּ בֵֽית־יִשְׂרָאֵ֛ל אֶת־שְׁמ֖וֹ מָ֑ן וְה֗וּא כְּזֶ֤רַע גַּד֙ לָבָ֔ן וְטַעְמ֖וֹ כְּצַפִּיחִ֥ת בִּדְבָֽשׁ׃ וַיֹּ֣אמֶר מֹשֶׁ֗ה זֶ֤ה הַדָּבָר֙ אֲשֶׁ֣ר צִוָּ֣ה יְיָ֔ מְלֹ֤א הָעֹ֙מֶר֙ מִמֶּ֔נּוּ לְמִשְׁמֶ֖רֶת לְדֹרֹתֵיכֶ֑ם לְמַ֣עַן ׀ יִרְא֣וּ אֶת־הַלֶּ֗חֶם אֲשֶׁ֨ר הֶאֱכַ֤לְתִּי אֶתְכֶם֙ בַּמִּדְבָּ֔ר בְּהוֹצִיאִ֥י אֶתְכֶ֖ם מֵאֶ֥רֶץ מִצְרָֽיִם׃ וַיֹּ֨אמֶר מֹשֶׁ֜ה אֶֽל־אַהֲרֹ֗ן קַ֚ח צִנְצֶ֣נֶת אַחַ֔ת וְתֶן־שָׁ֥מָּה מְלֹֽא־הָעֹ֖מֶר מָ֑ן וְהַנַּ֤ח אֹתוֹ֙ לִפְנֵ֣י יְיָ֔ לְמִשְׁמֶ֖רֶת לְדֹרֹתֵיכֶֽם׃ כַּאֲשֶׁ֛ר צִוָּ֥ה יְיָ֖ אֶל־מֹשֶׁ֑ה וַיַּנִּיחֵ֧הוּ אַהֲרֹ֛ן לִפְנֵ֥י הָעֵדֻ֖ת לְמִשְׁמָֽרֶת׃ וּבְנֵ֣י יִשְׂרָאֵ֗ל אָֽכְל֤וּ אֶת־הַמָּן֙ אַרְבָּעִ֣ים שָׁנָ֔ה עַד־בֹּאָ֖ם אֶל־אֶ֣רֶץ נוֹשָׁ֑בֶת אֶת־הַמָּן֙ אָֽכְל֔וּ עַד־בֹּאָ֕ם אֶל־קְצֵ֖ה אֶ֥רֶץ כְּנָֽעַן׃ וְהָעֹ֕מֶר עֲשִׂרִ֥ית הָאֵיפָ֖ה הֽוּא׃\hfill\break

עֶ֭זְרִי \source{תהילים קכא} מֵעִ֣ם יְיָ֑ עֹ֝שֵׂ֗ה שָׁמַ֥יִם וָאָֽרֶץ׃
הַשְׁלֵ֤ךְ \source{תהילים נה} עַל־יְיָ֨ ׀ יְהָבְךָ֮ וְה֢וּא יְכַ֫לְכְּלֶ֥ךָ לֹא־יִתֵּ֖ן לְעוֹלָ֥ם מ֗וֹט לַצַּדִּֽיק׃
שְׁמׇר־תָּ֭ם \source{תהילים לז} וּרְאֵ֣ה יָשָׁ֑ר כִּֽי־אַחֲרִ֖ית לְאִ֣ישׁ שָׁלֽוֹם: 
בְּטַ֣ח בַּייָ֭ וַעֲשֵׂה־ט֑וֹב שְׁכׇן־אֶ֝֗רֶץ וּרְעֵ֥ה אֱמוּנָֽה:
הִנֵּ֨ה \source{ישעיה יב} אֵ֧ל יְשׁוּעָתִ֛י אֶבְטַ֖ח וְלֹ֣א אֶפְחָ֑ד כִּֽי־עׇזִּ֤י וְזִמְרָת֙ יָ֣הּ יְיָ֔ וַֽיְהִי־לִ֖י לִֽישׁוּעָֽה׃
רִבּוֹנוֹ שֶׁל עוֹלָם בְּדִבְרֵי קָדְשְׁךָ כָּתוּב לֵאמר׃
...הַבּוֹטֵ֥חַ\source{תהילים לב} בַּייָ֑ חֶ֝֗סֶד יְסוֹבְבֶֽנּוּ׃
  וּכְתִיב׃ 
  וְאַתָּ֖ה\source{נחמיה ט} מְחַיֶּ֣ה אֶת־כֻּלָּ֑ם׃
יְיָ אֱלהִים אֱמֶת תֵּן לִי בְּרָכָה וְהַצְלָחָה בְּכׇל־מַעֲשֵׂה יָדַי. כִּי בָטַחְתִּי בְךָ שֶׁעַל יְדֵי מְלַאכְתִּי וּמַשָּׂא וּמַתָּן וַעֲסָקִים שֶׁלִּי תִּשְׁלַח־לִי בְּרָכָה שֶׁאוּכַל לְפַרְנֵס אֶת־עַצְמִי וּבְנֵי בֵיתִי בְּנַחַת וְלא בְצַֽעַר בְּהֶתֵּר וְלא בְאִסּוּר לְחַיִּים וּלְשָׁלוֹם וִיקוּיַם בִּי מִקְרָא שֶׁכָּתוּב׃
  הַשְׁלֵ֤ךְ\source{תהילים נה} עַל־יְיָ֨ ׀ יְהָבְךָ֮ וְה֢וּא יְכַ֫לְכְּלֶ֥ךָ׃
	
\end{footnotesize}

%\section[קדיש דרבנן]{\adforn{53} קדיש דרבנן \adforn{25}}
\section[קדיש דרבנן]
{קדיש דרבנן}
\label{kaddish derabonan}

\englishinst{After learning Torah with a minyan, it is customary to recite either the end of Gemara Berakhot or Avot 6:11 before the Rabbi's Kaddish.}
\sofberakhot

\firstword{רַבִּי חֲנַנְיָה בֶּן עֲקַשְׁיָא אוֹמֵר׃ }\source{אבות ו}
רָצָה הַקָּדוֹשׁ בָּרוּךְ הוּא לְזַכּוֹת אֶת־יִשְׂרָאֵל, לְפִיכָךְ הִרְבָּה לָהֶם תּוֹרָה וּמִצְוֹת, שֶׁנֶּאֱמַר׃
יְיָ \source{ישעיה מב}חָפֵ֖ץ לְמַ֣עַן צִדְק֑וֹ יַגְדִּ֥יל תּוֹרָ֖ה וְיַאְדִּֽיר׃

	\rabbiskaddish


\section[סיום הספר]{\adforn{53} סיום הספר \adforn{25}}

\englishinst{Upon completing a tractate of Gemara, or an order of Mishna, the person learning recites as follows:}
הֲדְרָן עֲלָךְ מַסֶּֽכֶת/סֵֽדֶר...וְהֲדְרָךְ עֲלָן \middot דַּעְתָּן עֲלָךְ מַסֶּֽכֶת/סֵֽדֶר...וְדַעְתָּךְ עֲלָן \middot לָא נִתְנְשֵׁי מִינָךְ מַסֶּֽכֶת/סֵֽדֶר...וְלֹא תִתְנְשֵׁי מִינָן לָא בְּעָלְמָא הָדֵין וְלֹא בְּעָלְמָא דְאַָתֵי׃

יְהִי רָצוֹן מִלְּפָנֶֽיךָ יְיָ אֱלֹהֵֽינוּ וֶאֱלֹהֵי אַבוֹתֵֽינוּ שֶׁתְּהֵא תוֹרָתְךָ אֻמָּנוּתֵֽנוּ בָּעוֹלָם הַזֶּה ותְהֵא עִמָּֽנוּ לָעוֹלָם הַבָּא. חֲנִינָא בַּר פָּפָּא, רָמִי בַּר פָּפָּא, נַחְמָן בַּר פָּפָּא, אַחָאי בַּר פָּפָּא, אַבָּא בַּר פָּפָּא, רַפֽרָם בַּר פָּפָּא, רָכִישׁ בַּר פָּפָּא, סוּרְחָב בַּר פָּפָּא, אַדָּא בַּר פָּפָּא, דָּרוּ בַּר פָּפָּא׃

הַעֲרֶב־נָא יְיָ אֱלֹהֵֽינוּ אֶת דִּבְרֵי תּוֹרָתְךָ בְּפִֽינוּ \middot  וּבְפִיפִיּוֹת עַמְּךָ בֵּית־יִשְׂרָאֵל \middot וְנִהְיֶה אֲנַחְנוּ כֻּלָּֽנוּ וְצֶאֱצָאֵֽינוּ וְצֶאֱצָאֵי עַמְּךָ בֵּית־יִשְׂרָאֵל כֻּלָּֽנוּ יוֹדְעֵי שְׁמֶךָ וְלוֹמְדֵי תּוֹרָתְךָ לִשְׁמָהּ׃\source{תהלים קיט}%
מֵֽ֭אֹיְבַי תְּחַכְּמֵ֣נִי מִצְוֺתֶ֑ךָ כִּ֖י לְעוֹלָ֣ם הִיא־לִֽי׃ יְהִי־לִבִּ֣י תָמִ֣ים בְּחֻקֶּ֑יךָ לְ֝מַ֗עַן לֹ֣א אֵבֽוֹשׁ׃ לְ֭עוֹלָם לֹא־אֶשְׁכַּ֣ח פִּקּוּדֶ֑יךָ כִּ֥י בָ֗֝ם חִיִּיתָֽנִי׃ בָּר֖וּךְ אַתָּ֥ה יְיָ֗ לַמְּדֵ֥נִי חֻקֶּֽיךָ׃

מוֹדִים אֲנַחְנוּ לְפָנֶֽיךָ יְיָ אֱלֹהֵֽינוּ וֶאֱלֹהֵי אַבוֹתֵֽינוּ שֶׁשַּׂמְתָּ חֶלְקֵֽנוּ מִיּוֹשְׁבֵי בֵּית הַמִּדְרָשׁ, וְלֹא שַׂמְתָּ חֶלְקֵֽנוּ מִיּוֹשְׁבֵי קְרָנוֹת \middot שֶׁאָנוּ מַשְׁכִּימִים וְהֵם מַשְׁכִּימִים אָנוּ מַשְׁכִּימִים לְדִבְרֵי תּוֹרָה וְהֵם מַשְׁכִּימִים לִדְבָרִים בְּטֵלִים \middot אָנוּ עֲמֵלִים וְהֵם עֲמֵלִים אָנוּ עֲמֵלִים וּמְקַבְּלִים שָׂכָר וְהֵם עֲמֵלִים וְאֵינָם מְקַבְּלִים שָׂכָר \middot אָנוּ רָצִים וְהֵם רָצִים אָנוּ רָצִים לְחַיֵּי הָעוֹלָם הַבָּא וְהֵם רָצִים לִבְאֵר שַׁחַת \middot שֶׁנֱאמַר:\source{תהלים נה}
וְאַתָּ֤ה אֱלֹהִ֨ים ׀ תּוֹרִדֵ֬ם ׀ לִבְאֵ֬ר שַׁ֗חַת אַנְשֵׁ֤י דָמִ֣ים וּ֭מִרְמָה לֹא־יֶחֱצ֣וּ יְמֵיהֶ֑ם וַ֝אֲנִ֗י אֶבְטַח־בָּֽךְ׃

יְהִי רָצוֹן מִלְּפָנֶֽיךָ יְיָ אֱלֹהֵי, כְּשֵׁם שֶׁעֲזַרְתַּֽנִי לְסַיֵים מַסֶּֽכֶת/סֵֽדֶר..., כֵּן תְּעַזְרֵֽנִי לְהַתְחִיל מְסֶכְתוֹת וּסְפָרִים אַחֵרים וּלְסַיֵימָם \middot לִלְמֹד וּלְלַמֵּד לִשְׁמֹר וְלַעֲשׂוֹת וּלְקַיֵּם אֶת כׇּל־דִּבְרֵי תַלְמוּד תּוֹרָתְךָ בְּאַהֲבָה \middot וּזְכוּת כֹֹּל הַתְנָאִים וְאָמוֹרָאִים וּתַּלְמִידֵי חֲכָמִים יַעֲמוֹד לִי וּלְזַרְעִי שֶׁלֹא תָּמוּש הַתּוֹרָה מִפִּי וּמִפִּי זַרְעִי עַד עוֹלָם׃ וַיִתְקַיֵים בִּי בְּהִתְהַלֶּכְךָ תַּנְחֶה אֹתָךְ בְּשׇׁכְבְּךָ תִּשְׁמֹר עָלֶיךָ וַהֲקִיצוֹתָ הִיא תְשִׂיחֶךָ \middot כִּי־בִי יִרְבּוּ יָמֶיךָ וְיוֹסִיפוּ לְּךָ שְׁנוֹת חַיִּים אֹרֶךְ יָמִים בִּימִינָהּ בִּשְׂמֹאולָהּ עֹשֶׁר וְכָבוֹד׃\source{תהלים כט}
יְיָ֗ עֹ֭ז לְעַמּ֣וֹ יִתֵּ֑ן יְיָ֓ ׀ יְבָרֵ֖ךְ אֶת־עַמּ֣וֹ בַשָּׁלֽוֹם׃

\englishinst{This is followed by Kaddish De'it\d{h}adta:}
\itchadtastart
\englishinst{Continue as in the Rabbi's Kaddish above.}

\clearpage
\let\clearpage\relax{\chapter[ערבית לחול]{\adforn{47} ערבית לחול \adforn{19}}

\textbf{וְה֤וּא רַח֨וּם}\source{תהלים עט}
׀ יְכַפֵּ֥ר עָוֺן֮ וְֽלֹא־יַֽ֫שְׁחִ֥ית וְ֭הִרְבָּה לְהָשִׁ֣יב אַפּ֑וֹ וְלֹא־יָ֝עִ֗יר כׇּל־חֲמָתֽוֹ׃
יְיָ֥\source{תהלים כ} הוֹשִׁ֑יעָה הַ֝מֶּ֗לֶךְ יַעֲנֵ֥נוּ בְיוֹם־קׇרְאֵֽנוּ׃


\barachu


\hamaarivaravim

\ahavasolam

\shema

\veahavta

\vehaya

\vayomer{}

\emesveemuna

\hashkiveinu{בָּרוּךְ אַתָּה יְיָ שׁוֹמֵר עַמּוֹ יִשְׂרָאֵל לָעַד׃}

\boruchhashemleolam

\halfkaddish

\section[תפילת העמידה]{\adforn{53} תפילת העמידה \adforn{25}}


\amidaopening{\ayt}{}

\firstword{אַתָּה חוֹנֵן}
לְאָדָם דַּֽעַת וּמְלַמֵּד לֶאֱנוֹשׁ בִּינָה.

\begin{sometimes}

\instruction{במוצאי שבת:}\\
אַתָּה חוֹנַנְתָּֽנוּ לְמַדַּע תּוֹרָתֶֽךָ וַתְּלַמְּדֵֽנוּ לַעֲשׂוֹת חֻקֵּי רְצוֹנֶֽךָ וַתַּבְדִּילֵֽנוּ יְיָ אֱלֹהֵֽינוּ בֵּין קֹֽדֶשׁ לְחוֹל בֵּין אוֹר לְחֹֽשֶׁךְ בֵּין יִשְׂרָאֵל לָעַמִּים בֵּין יוֹם הַשְּׁבִיעִי לְשֵֽׁשֶׁת יְמֵי הַמַּעֲשֶׂה׃ אָבִֽינוּ מַלְכֵּֽנוּ הָחֵל עָלֵֽינוּ הַיָּמִים הַבָּאִים לִקְרָאתֵֽנוּ לְשָׁלוֹם חֲשׂוּכִים מִכׇּל־חֵטְא וּמְנֻקִּים מִכׇּל־עָוֹן וּמְדֻבָּקִים בְּיִרְאָתֶֽךָ׃ וְ...

\end{sometimes}

חׇנֵּֽנוּ מֵאִתְּךָ בִּינָה דֵּעָה וְהַשְׂכֵּל׃ בָּרוּךְ אַתָּה יְיָ חוֹנֵן הַדָּֽעַת׃

\weekdaysateshuva

\weekdaysaselichah

\weekdaysageulah

\weekdaysarefuah

\weekdaysaberacha

\weekdaysashofar

\weekdaysamishpat

\weekdaysaminim

\weekdaysatzadikim

\weekdaysayerushelayim

\weekdaysamalchus

\weekdaysashemakoleinu

\retzeh

\yaalehveyavo

\zion

\maarivmodim

\alhanisim

\weekdaysahodos

\firstword{שָׁלוֹם}
רָב עַל יִשְׂרָאֵל עַמְּךָ תָּשִׂים לְעוֹלָם כִּי אַתָּה הוּא מֶֽלֶךְ אָדוֹן לְכׇל־הַשָּׁלוֹם׃ וְטוֹב בְּעֵינֶֽיךָ לְבָרֵךְ אֶת־עַמְּךָ יִשְׂרָאֵל בְּכׇל־עֵת וּבְכׇל־שָׁעָה בִּשְׁלוֹמֶךָ׃
\vspace{-0.4\baselineskip}
\columnratio{0.7}
\begin{paracol}{2}

\instruction{בעשי״ת:}
\begin{small}
בְּסֵֽפֶר חַיִּים בְּרָכָה וְשָׁלוֹם וּפַרְנָסָה טוֹבָה נִזָּכֵר וְנִכָּתֵב לְפָנֶֽיךָ אָֽנוּ וְכׇל־עַמְּךָ בֵּית יִשְׂרָאֵל לְחַיִּים וּלְשָׁלוֹם׃ בָּרוּךְ אַתָּה יְיָ עוֹשֵׂה הַשָּׁלוֹם׃

\end{small}
\switchcolumn
בָּרוּךְ אַתָּה יְיָ הַמְבָרֵךְ אֶת־עַמּוֹ יִשְׂרָאֵל בַּשָּׁלוֹם׃

\end{paracol}



\tachanunim

\vspace{\baselineskip}

\begin{sometimes}

\instruction{במוצ״ש אם אין בשבוע הבא יו״ט:}

\halfkaddish


\label{vihi noam}

\firstword{וִיהִ֤י ׀ נֹ֤עַם}\source{תהלים צ}
אֲדֹנָ֥י אֱלֹהֵ֗ינוּ עָ֫לֵ֥ינוּ וּמַעֲשֵׂ֣ה יָ֭דֵינוּ כּוֹנְנָ֥ה עָלֵ֑ינוּ וּֽמַעֲשֵׂ֥ה יָ֝דֵ֗ינוּ כּוֹנְנֵֽהוּ׃\\
%\tzadialeph

\label{v ata kadosh}
\kedushadesidra

\end{sometimes}

\fullkaddish

\vfill

\instruction{בחנוכה מדליקים את המנורה עמ׳ \pageref{chanukah}}\\
\instruction{בפורים קוראים את מגילת אסתר עמ׳ \pageref{purim}}\\

\aleinu

\ledavid

\mournerskaddish

\vfill


\instruction{סופרים כאן העומר עמ׳ \pageref{sefiras haomer}} \\
\instruction{מדליקים את המנורה לחנוכה כאן עמ׳ \pageref{chanukah}}
}
\clearpage
\let\clearpage\relax{\chapter[ברכת המזון]{\adforn{47} ברכת המזון \adforn{19}}

%\source{תהלים קלז}
%
%\columnratio{0.63}
%\begin{paracol}{2}
%\instruction{בימים שיש בהם תחנון:}\\
%\firstword{עַ֥ל נַהֲר֨וֹת}
% בָּבֶ֗ל שָׁ֣ם יָ֭שַׁבְנוּ גַּם־בָּכִ֑ינוּ בְּ֝זׇכְרֵ֗נוּ אֶת־צִיּֽוֹן׃ עַֽל־עֲרָבִ֥ים בְּתוֹכָ֑הּ תָּ֝לִ֗ינוּ כִּנֹּרוֹתֵֽינוּ׃ כִּ֤י שָׁ֨ם שְֽׁאֵל֪וּנוּ שׁוֹבֵ֡ינוּ דִּבְרֵי־שִׁ֭יר וְתוֹלָלֵ֣ינוּ שִׂמְחָ֑ה שִׁ֥ירוּ לָ֝֗נוּ מִשִּׁ֥יר צִיּֽוֹן׃ אֵ֗יךְ נָשִׁ֥יר אֶת־שִׁיר־יְיָ֑ עַ֝֗ל אַדְמַ֥ת נֵכָֽר׃ אִֽם־אֶשְׁכָּחֵ֥ךְ יְֽרוּשָׁלִָ֗ם תִּשְׁכַּ֥ח יְמִינִֽי׃ תִּדְבַּ֥ק־לְשׁוֹנִ֨י לְחִכִּי֮ אִם־לֹ֪א אֶ֫זְכְּרֵ֥כִי אִם־לֹ֣א אַ֭עֲלֶה אֶת־יְרוּשָׁלִַ֑ם עַ֝֗ל רֹ֣אשׁ שִׂמְחָתִֽי׃ זְכֹ֤ר יְיָ֨ לִבְנֵ֬י אֱד֗וֹם אֵת֮ י֤וֹם יְֽרוּשָׁ֫לִָ֥ם הָ֭אֹ֣מְרִים עָ֤רוּ | עָ֑רוּ עַ֝֗ד הַיְס֥וֹד בָּֽהּ׃ בַּת־בָּבֶ֗ל הַשְּׁד֫וּדָ֥ה אַשְׁרֵ֥י שֶׁיְשַׁלֶּם־לָ֑ךְ אֶת־גְּ֝מוּלֵ֗ךְ שֶׁגָּמַ֥לְתְּ לָֽנוּ׃ אַשְׁרֵ֤י שֶׁיֹּאחֵ֓ז וְנִפֵּ֬ץ אֶֽת־עֹ֝לָלַ֗יִךְ אֶל־הַסָּֽלַע׃
%
%\switchcolumn

\ifboolexpr{togl {includeweekday}}{\englishinst{The following is added on festive occasions:}}{}
\firstword{שִׁ֗יר הַֽמַּֽ֫עֲל֥וֹת}\source{תהלים קכו}
בְּשׁ֣וּב יְ֖יָ אֶת־שִׁיבַ֣ת צִיּ֑וֹן הָ֝יִ֗ינוּ כְּחֹלְמִֽים׃ אָ֤ז יִמָּלֵ֢א שְׂחֹ֡ק פִּינוּ֘ וּלְשׁוֹנֵ֢נוּ רִ֫נָּ֥ה אָ֭ז יֹֽאמְר֣וּ בַגּוֹיִ֑ם הִגְדִּ֥יל יְ֜יָ֗ לַֽעֲשׂ֥וֹת עִם־אֵֽלֶּה׃ הִגְדִּ֥יל יְ֖יָ לַֽעֲשׂ֣וֹת עִמָּ֑נוּ הָ֜יִ֗ינוּ שְׂמֵחִֽים׃ שׁוּבָ֣ה יְ֖יָ אֶת־שְׁבִיתֵ֑נוּ כַּֽאֲפִיקִ֥ים בַּנֶּֽגֶב׃ הַזֹּֽרְעִ֥ים בְּדִמְעָ֗ה בְּרִנָּ֥ה יִקְצֹֽרוּ׃ הָ֘ל֤וֹךְ יֵלֵ֨ךְ וּבָכֹה֘ נֹשֵׂ֢א מֶֽשֶׁךְ־הַ֫זָּ֥רַע בֹּֽא־יָבֹ֥א בְרִנָּ֗ה נֹשֵׂ֥א אֲלֻמֹּתָֽיו׃
%\end{paracol}

\englishinst{Three or more who ate together have one person invite the others to bless with a Zimmun.}
\begin{small}
\begin{tabular}{l p{.8\textwidth}}

\instruction{המזמן:} &
רַבּוֹתַי נְבָרֵךְ! \instruction{או} רַבּוֹתַי מיר וועלן בענטשן! \instruction{או} הַב לָן וְנִבְרִךְ!\\
\instruction{כולם:} &
יְהִ֤י שֵׁ֣ם יְיָ֣ מְבֹרָ֑ךְ מֵֽ֝עַתָּ֗ה וְעַד־עוֹלָֽם׃\\
\instruction{המזמן:} &
בִּרְשׁוּת ... נְבָרֵךְ (\instruction{בעשרה} אֱלֹהֵֽינוּ) שֶׁאָכַלְנוּ מִשֶּׁלּוֹ:\\
\instruction{כולם:} &
בָּרוּךְ (\instruction{בעשרה:} אֱלֹהֵֽינוּ) שֶׁאָכַֽלְנוּ מִשֶּׁלּוֹ וּבְטוּבוֹ חָיִֽינוּ:\\
(\instruction{מי שלא אכל:} &
בָּרוּךְ וּמְבֹרָךְ שְׁמוֹ תָּמִיד לְעוֹלָם וָעֶד׃)\\
\instruction{המזמן:} &
בָּרוּךְ (\instruction{בעשרה:} אֱלֹהֵֽינוּ) שֶׁאָכַֽלְנוּ מִשֶּׁלּוֹ וּבְטוּבוֹ חָיִֽינוּ:
\end{tabular}

בָּרוּךְ הוּא וּבָרוּךְ שְׁמוֹ׃\\
\end{small}

\firstword{בָּרוּךְ}
אַתָּה יְיָ אֱלֹהֵֽינוּ מֶֽלֶךְ הָעוֹלָם הַזָּן אֶת־הָעוֹלָם כֻּלּוֹ בְּטוּבוֹ בְּחֵן בְּחֶֽסֶד וּבְרַחֲמִים הוּא נֹתֵ֣ן \source{תהלים קלו}לֶ֭חֶם לְכׇל־בָּשָׂ֑ר כִּ֖י לְעוֹלָ֣ם חַסְדּֽוֹ׃ וּבְטוּבוֹ הַגָּדוֹל תָּמִיד לֹא חָסַר לָֽנוּ וְאַל יֶחְסַר לָֽנוּ מָזוֹן לְעוֹלָם וָעֶד׃ בַּעֲבוּר שְׁמוֹ הַגָּדוֹל כִּי הוּא זָן וּמְפַרְנֵס לַכֹּל וּמֵטִיב לַכֹּל וּמֵכִין מָזוֹן לְכׇל־בְּרִיּוֹתָיו אֲשֶׁר בָּרָא׃ בָּרוּךְ אַתָּה יְיָ הַזָּן אֶת־הַכֹּל׃



\firstword{נוֹדֶה}
לְךָ יְיָ אֱלֹהֵֽינוּ עַל שֶׁהִנְחַֽלְתָּ לַאֲבוֹתֵֽינוּ אֶֽרֶץ חֶמְדָה טוֹבָה וּרְחָבָה׃ וְעַל שֶׁהוֹצֵאתָֽנוּ יְיָ אֱלֹהֵֽינוּ מֵאֶֽרֶץ מִצְרַֽיִם וּפְדִיתָֽנוּ מִבֵּית עֲבָדִים וְעַל בְּרִיתְךָ שֶׁחָתַֽמְתָּ בִּבְשָׂרֵֽנוּ וְעַל תּוֹרָתְךָ שֶׁלִּמַּדְתָּֽנוּ וְעַל חֻקֶּֽיךָ שֶׁהוֹדַעְתָּֽנוּ וְעַל חַיִּים חֵן וָחֶֽסֶד שֶׁחוֹנַנְתָּֽנוּ וְעַל אֲכִילַת מָזוֹן שָׁאַתָּה זָן וּמְפַרְנֵס אוֹתָֽנוּ תָּמִיד בְּכׇל־יוֹם וּבְכׇל־עֵת וּבְכׇל־שָׁעָה׃


\alhanisim

\firstword{וְעַל הַכֹּל}
יְיָ אֱלֹהֵֽינוּ אֲנַֽחְנוּ מוֹדִים לָךְ וּמְבָרְכִים אוֹתָךְ יִתְבָּרַךְ שִׁמְךָ בְּפִי כׇל־חַי תָּמִיד לְעוֹלָם וָעֶד׃ כַּכָּתוּב׃ \source{דברים ח}%
וְאָכַלְתָּ֖ וְשָׂבָ֑עְתָּ וּבֵֽרַכְתָּ֙ אֶת־יְיָ֣ אֱלֹהֶ֔יךָ עַל־הָאָ֥רֶץ הַטֹּבָ֖ה אֲשֶׁ֥ר נָֽתַן־לָֽךְ׃
בָּרוּךְ אַתָּה יְיָ עַל הָאָֽרֶץ וְעַל הַמָּזוֹן׃



\firstword{רַחֵם}
יְיָ אֱלֹהֵֽינוּ עָלֵֽינוּ וְעַל יִשְׂרָאֵל עַמֶּךָ וְעַל יְרוּשָׁלַ‍ִם עִירֶֽךָ וְעַל צִיּוֹן מִשְׁכַּן כְּבוֹדֶֽךָ וְעַל מַלְכוּת בֵּית דָּוִד מְשִׁיחֶֽךָ וְעַל הַבַּֽיִת הַגָּדוֹל וְהַקָּדוֹשׁ שֶׁנִּקְרָא שִׁמְךָ עָלָיו׃ אֱלֹהֵֽינוּ אָבִֽינוּ רְעֵֽנוּ זוּנֵֽנוּ פַרְנְסֵֽנוּ וְכַלְכְּלֵֽנוּ וְהַרְוִיחֵֽנוּ וְהַרְוַח לָֽנוּ יְיָ אֱלֹהֵֽינוּ מְהֵרָה מִכׇּל־צָרוֹתֵֽינוּ׃ וְנָא אַל תַּצְרִיכֵֽנוּ יְיָ אֱלֹהֵֽינוּ לֹא לִידֵי מַתְּנַת בָּשָׂר וָדָם וְלֹא לִידֵי הַלְוָאָתָם. כִּי אִם לְיָדְךָ הַמְּלֵאָה הַפְּתוּחָה הַקְּדוֹשָׁה וְהָרְחָבָה שֶׁלֹּא נֵבוֹשׁ וְלֹא נִכָּלֵם לְעוֹלָם וָעֶד׃

%\enlargethispage{\baselineskip}
%
%\vspace{-.25\baselineskip}
\ifboolexpr{togl {includefestival} or (togl {includeshabbat} and togl {includeweekday})}{
\begin{sometimes}

\shabbos
רְצֵה וְהַחֲלִיצֵֽנוּ יְיָ אֱלֹהֵֽינוּ בְּמִצְוֹתֶֽיךָ וּבְמִצְוַת יוֹם הַשְּׁבִיעִי הַשַּׁבָּת הַגָּדוֹל וְהַקָּדוֹשׁ הַזֶּה כִּי יוֹם זֶה גָּדוֹל וְקָדוֹשׁ הוּא לְפָנֶֽיךָ לִשְׁבָּת בּוֹ וְלָנֽוּחַ בּוֹ בְּאַהֲבָה כְּמִצְוַת רְצוֹנֶךָ׃ בִּרְצוֹנְךָ הָנִֽיחַ לָֽנוּ יְיָ אֱלֹהֵֽינוּ שֶׁלֹא תְהֵי צָרָה וְיָגוֹן וַאֲנָחָה בְּיוֹם מְנוּחָתֵֽנוּ וְהַרְאֵֽנוּ יְיָ אֱלֹהֵֽינוּ בְּנֶחָמוֹת צִיּוֹן עִירֶֽךָ וּבְבִנְיַן יְרוּשָׁלַ‍ִם עִיר קׇדְשֶֽׁךָ כִּי אַתָּה הוּא בַּֽעַל הַיְשׁוּעוֹת וּבַֽעַל הַנֶּחָמוֹת׃


\sepline %These are really two "sometimes's". Sepline to separate them

\vspace{-.25\baselineskip}
\instruction{בראש חודש ומועדים:}\\
אֱלֹהֵֽינוּ וֵאלֹהֵי אֲבוֹתֵֽינוּ יַעֲלֶה וְיָבֹא וְיַגִּיעַ וְיֵרָאֶה וְיֵרָצֶה וְיִשָּׁמַע וְיִפָּקֵד וְיִזָּכֵר זִכְרוֹנֵֽנוּ וּפִקְדּוֹנֵֽנוּ וְזִכְרוֹן אֲבוֹתֵֽינוּ וְזִכְרוֹן מָשִׁיחַ בֶּן דָּוִד עַבְדֶּֽךָ וְזִכְרוֹן יְרוּשָׁלַ‍ִם עִיר קׇדְשֶֽׁךָ וְזִכְרוֹן כׇּל־עַמְּךָ בֵּית יִשְׂרָאֵל לְפָנֶיךָ לִפְלֵיטָה וּלְטוֹבָה וּלְחֵן וּלְחֶֽסֶד וּלְרַחֲמִים וּלְחַיִּים וּלְשָׁלוֹם בְּיוֹם\\
\begin{tabular}{c|c|c}
רֹאשׁ הַחֹֽדֶשׁ & חַג הַמַּצוֹת & חַג הַשָּׁבֻעוֹת\\ \hline
\end{tabular}\\
\begin{tabular}{c|c|c}
הַזִּכָּרוֹן & חַג הַסֻּכּוֹת & שְׁמִינִי חַג הָעֲצֶֽרֶת
\end{tabular}\\
הַזֶּה זׇכְרֵֽנּוּ יְיָ אֱלֹהֵֽינוּ בּוֹ לְטוֹבָה וּפׇקְדֵֽנוּ בוֹ לִבְרָכָה וְהוֹשִׁיעֵֽנוּ בוֹ לְחַיִּים וּבִדְבַר יְשׁוּעָה וְרַחֲמִים חוּס וְחׇׇׇׇנֵּנוּ וְרַחֵם עָלֵֽינוּ וְהוֹשִׁיעֵֽנוּ כִּי אֵלֶֽיךָ עֵינֵֽינוּ כִּי אֵל מֶֽלֶךְ חַנּוּן וְרַחוּם אַֽתָּה׃

\end{sometimes}}{
\ifboolexpr{togl {includeshabbat}}{
	רְצֵה וְהַחֲלִיצֵֽנוּ יְיָ אֱלֹהֵֽינוּ בְּמִצְוֹתֶֽיךָ וּבְמִצְוַת יוֹם הַשְּׁבִיעִי הַשַּׁבָּת הַגָּדוֹל וְהַקָּדוֹשׁ הַזֶּה כִּי יוֹם זֶה גָּדוֹל וְקָדוֹשׁ הוּא לְפָנֶֽיךָ לִשְׁבָּת בּוֹ וְלָנֽוּחַ בּוֹ בְּאַהֲבָה כְּמִצְוַת רְצוֹנֶךָ׃ בִּרְצוֹנְךָ הָנִֽיחַ לָֽנוּ יְיָ אֱלֹהֵֽינוּ שֶׁלֹא תְהֵי צָרָה וְיָגוֹן וַאֲנָחָה בְּיוֹם מְנוּחָתֵֽנוּ וְהַרְאֵֽנוּ יְיָ אֱלֹהֵֽינוּ בְּנֶחָמוֹת צִיּוֹן עִירֶֽךָ וּבְבִנְיַן יְרוּשָׁלַ‍ִם עִיר קׇדְשֶֽׁךָ כִּי אַתָּה הוּא בַּֽעַל הַיְשׁוּעוֹת וּבַֽעַל הַנֶּחָמוֹת׃
	
	\instruction{בר״ח׃}
	\yaalehveyavotemplate{רֹאשׁ הַחֹֽדֶשׁ}
}{}
}

\firstword{וּבְנֵה}
יְרוּשָׁלַ‍ִם עִיר הַקֹּֽדֶשׁ בִּמְהֵרָה בְּיָמֵֽינוּ׃ בָּרוּךְ אַתָּה יְיָ בֹּֽנֶה בְרַחֲמָיו יְרוּשָׁלַ‍ִם אָמֵן׃

%\begin{sometimes}
%
%\instruction{אם שכח רצה או יעלה ויבא:}\\
%בָּרוּךְ אַתָּה יְיָ אֱלֹהֵֽינוּ מֶֽלֶךְ הָעוֹלָם אֲשֶׁר נָתַן (שַׁבָּתוֹת לִמְנוּחָה לְעַמּוֹ יִשְׂרָאֵל בְּאַהֲבָה לְאוֹת וְלִבְרִית)
%(וְיָמִים טוֹבִים לְשָׂשׂוֹן וּלְשִׂמְחָה אֶת־יוֹם חַג ... הַזֶּה)(וְרָאשֵׁי חֳדָשִׁים לְזִכָּרוֹן \instruction{מסיים כאן בחול}):
%בָּרוּךְ אַתָּה יְיָ מְקַדֵּשׁ (הַשַּׁבָּת) ([וְ]יִשְׂרָאֵל וְהַזְּמַנִּים)(וְיִשְׂרָאֵל וְרָאשֵׁי חֳדָשִׁים׃)׃
%
%\end{sometimes}


\firstword{בָּרוּךְ}
אַתָּה יְיָ אֱלֹהֵֽינוּ מֶֽלֶךְ הָעוֹלָם הָאֵל אָבִֽינוּ מַלְכֵּֽנוּ אַדִּירֵֽנוּ בּוֹרְאֵֽנוּ גֹאֲלֵֽנוּ יוֹצְרֵֽנוּ קְדוֹשֵֽׁנוּ קְדוֹשׁ יַעֲקֹב רוֹעֵֽנוּ רוֹעֵה יִשְׂרָאֵל הַמֶּֽלֶךְ הַטּוֹב וְהַמֵּטִיב לַכֹּל שֶׁבְּכׇל־יוֹם וָיוֹם הוּא הֵטִיב הוּא מֵטִיב הוּא יֵיטִיב לָֽנוּ׃ הוּא גְמָלָֽנוּ הוּא גוֹמְלֵנוּ הוּא יִגְמְלֵנוּ לָעַד לְחֵן לְחֶֽסֶד וּלְרַחֲמִים וּלְרֶֽוַח הַצָּלָה וְהַצְלָחָה בְּרָכָה וִישׁוּעָה נֶחָמָה פַּרְנָסָה וְכַלְכָּלָה וְרַחֲמִים וְחַיִּים וְשָׁלוֹם וְכׇל־טוֹב וּמִכׇּל־טוֹב אַל יְחַסְּרֵֽנוּ׃

\firstword{הָרַחֲמָן}
הוּא יִמְלֹךְ עָלֵֽינוּ לְעוֹלָם וָעֶד׃
\firstword{הָרַחֲמָן}
הוּא יִתְבָּרַךְ בַּשָּׁמַֽיִם וּבָאָֽרֶץ׃
\firstword{הָרַחֲמָן}
הוּא יִשְׁתַּבַּח לְדוֹר דּוֹרִים וְיִתְפָּֽאַר בָּֽנוּ לָנֵֽצַח נְצָחִים
וְיִתְהַדַּר בָּֽנוּ לָעַד וּלְעוֹלְמֵי עוֹלָמִים׃
\firstword{הָרַחֲמָן}
הוּא יְפַרְנְסֵֽנוּ בְּכָבוֹד׃
\firstword{הָרַחֲמָן}
הוּא יִשְׁבּוֹר עֻלֵּֽנוּ מֵעַל צַוָּארֵֽנוּ וְהוּא יוֹלִיכֵֽנוּ קוֹמְמִיּוּת לְאַרְצֵֽנוּ׃
\firstword{הָרַחֲמָן}
הוּא יִשְׁלַח בְּרָכָה מְרֻבָּה בְּבַֽיִת זֶה וְעַל שֻׁלְחָן זֶה שֶׁאָכַֽלְנוּ עָלָיו׃
\firstword{הָרַחֲמָן}
הוּא יִשְׁלַח לָֽנוּ אֶת־אֵלִיָּֽהוּ הַנָּבִיא זָכוּר לַטּוֹב וִיבַשֵּׂר לָנוּ בְּשׂוֹרוֹת טוֹבוֹת יְשׁוּעוֹת וְנֶחָמוֹת׃


\begin{footnotesize}
\instruction{אורחים אומרים:}\\
יְהִי רָצוֹן שֶׁלֹא יֵבוֹשׁ בַּעַל הַבַּיִת בָּעוֹלָם הַזֶּה וְלֹא יִכָּלֵם לָעוֹלָם הַבָּא וְיִצְלַח מְאֹד בְּכׇל־נְכָסָיו וְיִהְיוּ נְכָסָיו מֻצְלָחִים וּקְרוֹבִים לָעִיר וְאַל יִשְׁלוֹט שָׂטָן לֹא בְּמַעֲשֵׂי יָדָיו וְלֹא בְּמַעֲשֵׂי יָדֵינוּ וְאַל יִזְדַקֵּר לֹא לְפָנָיו וְלֹא לְפָנֵינוּ שׁוּם דְבַר הִרְהוּר חֵטְא וַעֲבֵרָה וְעָוֹן מֵעַתָּה וְעַד עוֹלָם׃

\end{footnotesize}

\firstword{הָרַחֲמָן}
הוּא יְבָרֵךְ אֶת־[אָבִי מוֹרִי] בַּעַל הַבַּֽיִת הַזֶּה וְאֶת־[אִמִּי מוֹרָתִי] בַּעֲלַת הַבַּֽיִת הַזֶּה׃ אוֹתָם וְאֶת־בֵּיתָם וְאֶת־זַרְעָם וְאֶת־כׇּל־אַשֶׁר לָהֶם, אוֹתָנוּ וְאֶת־כׇּל־אַשֶׁר לָֽנוּ כְּמוֹ שֶׁנִּתְבָּרְכוּ אֲבוֹתֵֽינוּ אַבְרָהָם יִצְחָק וְיַעֲקֹב בַּכֹּל מִכֹּל כֹּל כֵּן יְבָרֵךְ אוֹתָֽנוּ כֻּלָּנוּ יַֽחַד בִּבְרָכָה שְׁלֵמָה וְנֹאמַר אָמֵן׃

\begin{sometimes}

\englishinst{At the meal following a Berit Mila:}\nopagebreak
\begin{center}
\textbf{הָרַחֲמָן}
הוּא אֲשֶׁר חָנַן אֶת־הַיֶּלֶד הַזֶּה לְאָבִיו וּלְאִמּוֹ הוּא יָגֵן עָלָיו מִמְּרוֹמוֹ וּבְשָׁלוֹם יָבֹא עַל־מְקוֹמוֹ וִיהִי אֱלֹהָיו עִמּוֹ וּכְאֶפְרַיִם וְכִמְנַשֶּׁה לְשׂוּמוֹ ְויִקָּרֵא בְיִשְׂרָאֵל שְׁמוֹ׃

\textbf{הָרַחֲמָן}
הוּא פָּקוֹד יִפְקְדֵהוּ בְּרַחֲמִים לַהֲבִינוֹ בְּדָת חִכּוּמִים וִיבַלֶּה בַטּוֹב יָמִים וּשְׁנוֹתָיו בַּנְּעִימִים יַעַבְדוּהוּ עַמִּים וְיִשְׁתַּחֲווּ לוֹ לְאֻמִּים׃

\textbf{הָרַחֲמָן}
הוּא רַבּוֹת שָׁנִים יְחַיֵּהוּ צֶדֶק לְרַגְלָיו יִקְרָאֵהוּ וְנֶחָמַת צִיּוֹן יַרְאֵהוּ וְאֶת־עַמּוֹ לְשָׁלוֹם יְבָרְכֵהוּ וִיעוֹרֵר חֲסָדָיו לְרַחֲמֵהוּ כִּי חָפֵץ חֶסֶד הוּא׃

\textbf{הָרַחֲמָן}
הוּא יְבָרֵךְ אֶת־הֶחָתָן הַזֶּה וּבַעַל בְּרִיתוֹ יִשְׂמַח אָבִיו וְתָגֵל יוֹלַדְתּוֹ וְיִתְבָּרַכוּ הַמְסֻבִּים בִּסְעוּדָתוֹ וְכֵן יִזְכּוּ שֶׁיִּשְׂמְחוּ בַּחֲתֻנָּתוֹ בְּנֵי בָנִים לְהַרְאוֹתוֹ וְיֵשׁ תִּקְוָה לְאַחֲרִיתוֹ׃

\textbf{הָרַחֲמָן}
הוּא מְהֵרָה יִזְכֹּר זֹאת מִצְוָתוֹ בָּהּ יִפְדֶה אֲיֻמָּתוֹ רַחֲמִים יְעוֹרֵר לַעֲדָתוֹ קְהָלָיו יְקַבֵּץ בְּחֶמְלָתוֹ בְּהַרְאֹתוֹ אֶת־עֹשֶׁר כְּבוֹד מַלְכוּתוֹ וְאֶת־יְקָר תִּפְאֶרֶת גְּדֻלָּתוֹ׃

\textbf{הָרַחֲמָן}
הוּא יְבָרֵךְ אֶת־הֶחָתָן הַזֶּה וּבַּעַל בְּרִיתוֹ וְאֶת אָבִיו וְאֶת אִמּוֹ וְאֶת רַבּוֹתֵינוּ וְאֶת־אַחֵינוּ הַיּוֹשְׁבִים פֹּה כְּמוֹ שֶׁנִתְבָּרְכוּ אֲבוֹתֵֽינוּ אַבְרָהָם יִצְחָק וְיַעֲקֹב בַּכֹּל מִכֹּל כֹּל כֵּן יְבָרֵךְ אוֹתָֽנוּ כֻּלָּנוּ יַֽחַד בִּבְרָכָה שְׁלֵמָה וְנֹאמַר אָמֵן׃
\end{center}
\end{sometimes}

\begin{center}
\firstword{בַּמָּרוֹם}
יְלַמְּדוּ עֲלֵיהֶם וְעָלֵֽינוּ זְכוּת שֶׁתְּהֵא לְמִשְׁמֶֽרֶת שָׁלוֹם׃ וְנִשָּׂא בְרָכָה מֵאֵת יְיָ וּצְדָקָה מֵאֱלֹהֵי יִשְׁעֵנוּ וְנִמְצָא חֵן וְשֵֽׂכֶל טוֹב בְּעֵינֵי אֱלֹהִים וְאָדָם׃
\end{center}

\begin{longtable}{l p{.8\textwidth}}

\shabbos &
הָרַחֲמָן הוּא יַנְחִילֵֽנוּ לְיּוֹם שֶׁכֻּלּוֹ שַׁבָּת וּמְנוּחָה לְחַיֵּי הָעוֹלָמִים׃ \\

\instruction{בראש חודש:} &
הָרַחֲמָן הוּא יְחַדֵּשׁ עָלֵֽינוּ אֶת־הַחֹֽדֶשׁ הַזֶּה לְטוֹבָה וְלִבְרָכָה׃ \\

\instruction{בשלש רגלים:} &
הָרַחֲמָן הוּא יַנְחִילֵֽנוּ לְיּוֹם שֶׁכֻּלּוֹ טוֹב׃ \\

\instruction{בראש השנה:} &
הָרַחֲמָן הוּא יְחַדֵּשׁ עָלֵֽינוּ אֶת־הַשָּׁנָה הַזֹּאת לְטוֹבָה וְלִבְרָכָה׃ \\

\instruction{בסכות:} &
הָרַחֲמָן הוּא יָקִים לָֽנוּ אֶת־סֻכַּ֥ת דָּוִ֖יד הַנֹּפֶ֑לֶת׃ \mdsource{עמוס ט}

\end{longtable}

\firstword{הָרַחֲמָן}
הוּא יְזַכֵּֽנוּ לִימוֹת הַמָּשִֽׁיחַ וּלְחַיֵּי עוֹלָם הַבָּא׃

\firstword{מַגְדִּיל֘}\source{תהלים יח}
(\instruction{בשבת, יו״ט, ור״ח׃ }
מִגְדּ֖וֹל)
יְשׁוּע֢וֹת מַ֫לְכּ֥וֹ וְעֹ֤שֶׂה חֶ֨סֶד ׀ לִמְשִׁיח֗וֹ לְדָוִ֥ד וּלְזַרְע֗וֹ עַד־עוֹלָֽם׃
עֹשֶׂה שָׁלוֹם בִּמְרוֹמָיו הוּא יַעֲשֶׂה שָׁלוֹם עָלֵֽינוּ וְעַל כׇּל־יִשְׂרָאֵל וְאִמְרוּ אָמֵן׃\\
יְר֣אוּ \source{תהלים לד}אֶת־יְיָ֣ קְדֹשָׁ֑יו כִּי־אֵ֥ין מַ֝חְס֗וֹר לִירֵאָֽיו׃ כְּ֭פִירִים רָשׁ֣וּ וְרָעֵ֑בוּ
וְדֹרְשֵׁ֥י יְ֝יָ֗ לֹא־יַחְסְר֥וּ כׇל־טֽוֹב׃
הוֹד֣וּ לַֽיְיָ֑ \source{תהלים קיח}כִּי־ט֑וֹב כִּ֖י לְעוֹלָ֣ם חַסְדּֽוֹ׃
פּוֹתֵ֥חַ \source{תהלים קמה}אֶת־יָדֶ֑ךָ וּמַשְׂבִּ֖יעַ לְכׇל־חַ֣י רָצֽוֹן׃
בָּר֣וּךְ \source{ירמיהו יז}הַגֶּ֔בֶר אֲשֶׁ֥ר יִבְטַ֖ח בַּייָ֑ וְהָיָ֥ה יְיָ֖ מִבְטַחֽוֹ׃
נַ֤עַר \source{תהלים לז}׀ הָיִ֗יתִי גַּם־זָ֫קַ֥נְתִּי וְֽלֹא־רָ֭אִיתִי צַדִּ֣יק נֶעֱזָ֑ב וְ֝זַרְע֗וֹ מְבַקֶּשׁ־לָֽחֶם׃
יְיָ֗ \source{תהלים כט}עֹ֭ז לְעַמּ֣וֹ יִתֵּ֑ן יְיָ֓ ׀ יְבָרֵ֖ךְ אֶת־עַמּ֣וֹ בַשָּׁלֽוֹם׃

\bigskip

\sepline

\bigskip

\instruction{המזמן:}
בָּרוּךְ אַתָּה יְיָ אֱלֹהֵֽינוּ מֶֽלֶךְ הָעוֹלָם בּוֹרֵא פְּרִי הַגָּֽפֶן׃

\vfill
\sepline

\ifboolexpr{togl {includeweekday}}{
\section[ברכת המזון בבית אבל]{\adforn{53} ברכת המזון בבית אבל \adforn{25}}

\instruction{זימון בבית־אבל}

\begin{small}
	\begin{tabular}{l p{.8\textwidth}}
		
		\instruction{המזמן:} &
		רַבּוֹתַי נְבָרֵךְ! \instruction{או} רַבּוֹתַי מיר וועלן בענטשן! \instruction{או} הַב לָן וְנִבְרִךְ!\\
		\instruction{כולם:} &
		יְהִ֤י שֵׁ֣ם יְיָ֣ מְבֹרָ֑ךְ מֵֽ֝עַתָּ֗ה וְעַד־עוֹלָֽם׃\\
		\instruction{המזמן:} &
		בִּרְשׁוּת ... נְבָרֵךְ (\instruction{בעשרה} אֱלֹהֵֽינוּ) מְנַחֵם אֲבֵלִים שֶׁאָכַלְנוּ מִשֶּׁלּוֹ:\\
		\instruction{כולם:} &
		בָּרוּךְ (\instruction{בעשרה:} אֱלֹהֵֽינוּ) מְנַחֵם אֲבֵלִים שֶׁאָכַֽלְנוּ מִשֶּׁלּוֹ וּבְטוּבוֹ חָיִֽינוּ:\\
		(\instruction{מי שלא אכל:} &
		בָּרוּךְ מְנַחֵם אֲבֵלִים וּמְבֹרָךְ שְׁמוֹ תָּמִיד לְעוֹלָם וָעֶד׃)\\
		\instruction{המזמן:} &
		בָּרוּךְ (\instruction{בעשרה:} אֱלֹהֵֽינוּ) מְנַחֵם אֲבֵלִים שֶׁאָכַֽלְנוּ מִשֶּׁלּוֹ וּבְטוּבוֹ חָיִֽינוּ:
	\end{tabular}

בָּרוּךְ הוּא וּבָרוּךְ שְׁמוֹ׃

\end{small}

נַחֵם יְיָ אֱלֹהֵינוּ אֶת אֲבֵלֵי יְרוּשָׁלַיִם. וְאֶת הָאֲבֵלִים הַמִּתְאַבְּלִים בָּאֵֽבֶל הַזֶּה. נְחַמֵּם מֵאֶבְלָם וְשֶׁמֵּחָם מִיגוֹנָם כָּאָמוּר׃\source{ישעיה סו} כְּאִ֕ישׁ אֲשֶׁ֥ר אִמּ֖וֹ תְּנַחֲמֶ֑נּוּ כֵּ֤ן אָֽנֹכִי֙ אֲנַ֣חֶמְכֶ֔ם וּבִירֽוּשָׁלַ֖͏ִם תְּנֻחָֽמוּ׃ בָּרוּךְ אַתָּה יְיָ מְנַחֵם צִיּוֹן בְּבִנְיַן יְרוּשָׁלַיִם׃

\firstword{בָּרוּךְ}
אַתָּה יְיָ אֱלֹהֵֽינוּ מֶֽלֶךְ הָעוֹלָם הָאֵל אָבִֽינוּ מַלְכֵּֽנוּ אַדִּירֵֽנוּ בּוֹרְאֵֽנוּ גֹאֲלֵֽנוּ יוֹצְרֵֽנוּ קְדוֹשֵֽׁנוּ קְדוֹשׁ יַעֲקֹב. הַמֶּלֶךְ הַחַי הַטּוֹב וְהַמֵּטִיב. אֵל אֱמֶת, דַּיַּן אֱמֶת, שׁוֹפֵט בְּצֶדֶק, לוֹקֵחַ נְפָשׁוֹת בַּמִּשְׁפָּט. שַׁלִּיט בְּעוֹלָמוֹ לַעֲשׂוֹת בּוֹ כִּרְצוֹנוֹ כִּי כׇל־דְּרָכָיו בַּמִּשְׁפָּט, וַאֲנַחְנוּ עַמּוֹ וַעֲבָדָיו, וּבַכֹּל אֲנַחְנוּ חַיָּבִים לְהוֹדוֹת לוֹ וּלְבָרְכוֹ. גּוֹדֵר פְּרָצוֹת יִשְׂרָאֵל הוּא יִגְדֹּר הַפִּרְצָה הַזֹּאת מֵעָלֵינוּ וּמֵעַל אֲבָל זֶה לְחַיִּים וּלְשָׁלוֹם. הוּא יִגְמְלֵנוּ לָעַד לְחֵן לְחֶֽסֶד וּלְרַחֲמִים וּלְרֶֽוַח הַצָּלָה וְהַצְלָחָה בְּרָכָה וִישׁוּעָה נֶחָמָה פַּרְנָסָה וְכַלְכָּלָה וְרַחֲמִים וְחַיִּים וְשָׁלוֹם וְכׇל־טוֹב וּמִכׇּל־טוֹב אַל יְחַסְּרֵֽנוּ׃}{}


\section[ברכה מעין שלש]{\adforn{53} ברכה מעין שלש \adforn{25}}

\englishinst{After eating foods made from the five grains (besides bread), grapes, figs, pomegranate, olives, dates, or drinking wine, recite the following blessing.}
\firstword{בָּרוּךְ}
אַתָּה יְיָ אֱלֹהֵֽינוּ מֶֽלֶךְ הָעוֹלָם עַל

\begin{tabular}{>{\centering\arraybackslash}m{.3\textwidth} | >{\centering\arraybackslash}m{.3\textwidth} | >{\centering\arraybackslash}m{.3\textwidth}}

הָעֵץ וְעַל פְּרִי הָעֵץ
&
הַמִּחְיָה וְעַל הַכַּלְכָּלָה
&
הַגֶּֽפֶן וְעַל פְּרִי הַגֶּֽפֶן \\

\end{tabular}

וְעַל תְּנוּבַת הַשָּׂדֶה וְעַל אֶֽרֶץ חֶמְדָּה טוֹבָה וּרְחָבָה
שֶׁרָצִֽיתָ וְהִנְחַֽלְתָּ לַאֲבוֹתֵֽינוּ לֶאֱכוֹל מִפִּרְיָהּ וְלִשְׂבּֽוֹעַ מִטּוּבָהּ׃
רַחֶם יְיָ אֱלֹהֵֽינוּ עַל יִשְׂרָאֵל עַמֶּֽךָ וְעַל יְרוּשָׁלַֽיִם עִירֶֽךָ וְעַל צִיּוֹן מִשְׁכַּן כְּבוֹדֶֽךָ וְעַל מִזְבַּחֲךָ וְעַל הֵיכָלֶֽךָ׃ וּבְנֵה יְרוּשָׁלַֽיִם עִיר הַקֹּדֶשׁ בִּמְהֵרָה בְּיָמֵֽינוּ וְהַעֲלֵֽנוּ לְתוֹכָהּ וְשַׂמְּחֵֽנוּ בְּבִנְיָנָהּ וְנֹאכַל מִפִּרְיָהּ וְנִשְׂבַּע מִטּוּבָהּ וּנְבָרֶכְךָ עָלֶיהָ בִּקְדֻשָּׁה וּבְטׇהֳרָה׃

\begin{small}

\begin{tabular}{l p{.7\textwidth}}
\instruction{שבת:}&
וּרְצֵה וְהַחֲלִיצֵֽנוּ בְּיוֹם הַשַּׁבָּת הַזֶּה׃ \\


\instruction{ראש חודש:}&
וְזׇכְרֵֽנוּ לְטוֹבָה
בְּיוֹם רֹאשׁ הַחֹֽדֶשׁ הַזֶּה׃ \\

\instruction{שלוש רגלים:}&
וְשַׂמְּחֵֽנוּ בְּיוֹם
חַג הַמַּצּוֹת \textbackslash \space הַשָּׁבֻעוֹת \textbackslash \space הַסֻּכּוֹת \textbackslash \space שְׁמִינִי חַג הָעֲצֶֽרֶת הַזֶּה׃\\


\instruction{ראש השנה:}&
וְזׇכְרֵֽנוּ לְטוֹבָה בְּיוֹם חַזִּכָּרוֹן הַזֶּה׃\\

\end{tabular}

\end{small}

כִּי אַתָּה טוֹב וּמֵטִיב לַכֹּל וְנוֹדֶה לְךָ עַל הָאָֽרֶץ,

\begin{tabular}{c|c|c}
וְעַל הַפֵּרוֹת & וְעַל הַמִּחְיָה & וְעַל פְּרִי הַגָּֽפֶן
\end{tabular}

בָּרוּךְ אַתָּה יְיָ עַל הָאָֽרֶץ

\begin{tabular}{c|c|c}
וְעַל הַפֵּרוֹת׃ & וְעַל הַמִּחְיָה׃ & וְעַל פְּרִי הַגָּֽפֶן׃
\end{tabular}

\englishinst{After all other foods:}
\firstword{בָּרוּךְ}
אַתָּה יְיָ אֱלֹהֵֽינוּ מֶֽלֶךְ הָעוֹלָם בּוֹרֵא נְפָשׁוֹת רַבּוֹת וְחֶסְרוֹנָן
עַל כׇּל־מַה שֶּׁבָּרָא לְהַחֲיוֹת בָּהֶם נֶֽפֶשׁ כׇּל־חָי׃ בָּרוּךְ חַי הָעוֹלָמִים׃\\

\chapter[ברכות]{\adforn{47} ברכות \adforn{19}}

\newcommand{\berakha}[2]{\englishinst{#1}
בָּרוּךְ אַתָּה יְיָ אֱלֺהֵֽינוּ מֶֽלֶךְ הָעוֹלָם #2׃}
\newcommand{\berakhamitzva}[2]{\berakha{#1}{אֲשֶׁר קִדְּשָֽׁנוּ בְּמִצְוֺתָיו וְצִוָּֽנוּ #2}}


\ssubsection{\adforn{18} ברכות על אכילה \adforn{17}}

\ifboolexpr{not togl {includeweekday}}{
\berakhamitzva{On washing hands before eating bread:}{עַל נְטִילַת יָדָיִם}
\berakha{Before eating bread:}{הַמּֽוֹצִיא לֶֽחֶם מִן הָאָֽרֶץ}
}{
\berakha{Before eating bread, wash hands and recite its blessing in the following section, then recite the following blessing before eating:}{הַמּֽוֹצִיא לֶֽחֶם מִן הָאָֽרֶץ}
}

\berakha{On food made from grains besides bread (e.g. crackers or cake):}{בּוֹרֵא מִינֵי מְזוֹנוֹת}

\berakha{Before drinking wine:}{בּוֹרֵא פְּרִי הַגָּֽפֶן}

\berakha{Before eating fruit that grows on trees:}{בּוֹרֵא פְּרִי הָעֵץ}

\berakha{Before eating fruits or vegetables:}{בּוֹרֵא פְּרִי הָאֲדָמָה}

\berakha{On all other foods:}{שֶׁהַכֹּל נִהְיֶה בִּדְבָרוֹ}

\ssubsection{\adforn{18} ברכות הריח \adforn{17}}

\berakha{On pleasant-smelling products from trees:}{בּוֹרֵא עֲצֵי בְשָׂמִים}

\berakha{On fragrant shrubs and grasses:}{בּוֹרֵא עִשְׂבֵי בְשָׂמִים}

\berakha{On fragrant fruits:}{הַנּוֹתֵן רֵֽיחַ טוֹב בַּפֵּרוֹת}

\berakha{On balsam oil:}{בּוֹרֵא שֶֽׁמֶן עָרֵב}

\berakha{On other pleasant scents:}{בּוֹרֵא מִינֵי בְשָׂמִים}

\ssubsection{\adforn{18} ברכות הראייה והשמיעה \adforn{17}}

\berakha{On seeing natural phenomena, such as lightning, mountains, or canyons:}{עֹשֶׂה מַעֲשֵׂה בְרֵאשִׁית}

\berakha{On hearing thunder or experiencing very strong winds:}{שֶׁכֹּחוֹ וּגְבוּרָתוֹ מָלֵא עוֹלָם}

\berakha{On seeing a rainbow:}{זוֹכֵר הַבְּרִית וְנֶאֱמָן בִּבְרִיתוֹ וְקַיָּם בְּמַאֲמָרוֹ}

\berakha{On seeing the ocean, for the first time in thirty days:}{שֶׁעָשָׂה אֶת הַיָּם הַגָּדוֹל}

\berakha{On seeing beautiful animals:}{שֶׁכָּֽכָה לּוֹ בְּעוֹלָמוֹ}

\berakha{On seeing trees blooming for the first time in spring:}{שֶׁלֹּא חִסַּר בְּעוֹלָמוֹ דָּבָר, וּבָרָא בוֹ בְּרִיּוֹת טוֹבוֹת וְאִילָנוֹת טוֹבִים לְהַנּוֹת בָּהֶם בְּנֵי אָדָם}

\berakha{On seeing unusual animals:}{מְשַׁנֶּה הַבְּרִיּוֹת}

\berakha{On seeing a monarch:}{שֶׁנָּתַן מִכְּבוֹדוֹ לַבָּשָׂר וָדָם}

\berakha{On seeing a Torah scholar:}{שֶׁחָלַק מֵחׇכְמָתוֹ לִירֵאָיו}

\berakha{On seeing scholars of other subjects:}{שֶׁנָּתַן מֵחׇכְמָתוֹ לְבָשָׂר וָדָם}

\berakha{On news that is good for multiple people:}{הַטוֹב וְהַמֵּטִיב}

\berakha{On good news for an individual:}{שֶׁהֶחֱיָֽנוּ וְקִיְּמָֽנוּ וְהִגִּיעָֽנוּ לַזְּמַן הַזֶּה}

\berakha{On hearing bad news:}{דַּיַּן הָאֱמֶת}

\berakha{On wearing a significant item of new clothing:}{מַלְבִּישׁ עַרֻמִּים}

\ifboolexpr{togl {includeweekday}}{
\ssubsection{\adforn{18} ברכות המצוה \adforn{17}}

\berakhamitzva{On washing hands before eating bread:}{עַל נְטִילַת יָדָיִם}

\berakhamitzva{On ritual immersion:}{עַל הַטְּבִילָה}

\berakhamitzva{On immersion of vessels:}{עַל טְבִילַת כֵּלִים}

\berakhamitzva{On fixing a mezuza to a doorpost:}{לִקְבּֽוֹעַ מְזוּזָה}

\berakhamitzva{On performing she\d{h}ita:}{עַל הַשְּׁחִיטָה}

\berakhamitzva{On covering the blood of she\d{h}ted fowl or deer:}{עַל כִּיסוּי הַדָּם}

\section[תפלת הדרך]{\adforn{47} תפלת הדרך \adforn{19}}

\englishinst{On embarking on a journey:}
יְהִי רָצוֹן מִלְּפָנֶֽיךָ יְיָ אֱלֹהֵֽינוּ וֵאלֹהֵי אֲבוֹתֵֽינוּ שֶׁתּוֹלִיכֵֽנוּ לְשָׁלוֹם וְתַצְעִידֵֽנוּ לְשָׁלוֹם וְתַדְרִיכֵֽנוּ לְשָׁלוֹם׃ וְתַגִּיעֵֽנוּ לִמְחוֹז חֶפְצֵֽנוּ לְחַיִּים וּלְשִׂמְחָה וּלְשָׁלוֹם וְתַצִּילֵֽנוּ מִכַּף כׇּל־אוֹיֵב וְאוֹרֵב בַּדֶּֽרֶךְ וְתִתְּנֵֽנוּ לְחֵן וּלְחֶֽסֶד וּלְרַחֲמִים בְּעֵינֶֽיךָ וּבְעֵינֵי כׇל־רוֹאֵֽנוּ וְתִשְׁמַע קוֹל תַּחֲנוּנֵֽינוּ כִּי אֵל שׁוֹמֵֽעַ תְּפִלָה וְתַחֲנוּן אַֽתָּה׃ בָּרוּךְ אַתָּה יְיָ שׁוֹמֵֽעַ תְּפִלָּה׃}{}
}
\clearpage
\let\clearpage\relax{\chapter[קריאת התורה]{\adforn{47} קריאת התורה \adforn{19}}
\label{torah}

\section{פרשיות למנחה בשבת, שני, וחמישי}


\begin{footnotesize}


\ssubsection{בראשית}\\
בְּרֵאשִׁ֖ית בָּרָ֣א אֱלֹהִ֑ים אֵ֥ת הַשָּׁמַ֖יִם וְאֵ֥ת הָאָֽרֶץ׃ וְהָאָ֗רֶץ הָיְתָ֥ה תֹ֙הוּ֙ וָבֹ֔הוּ וְחֹ֖שֶׁךְ עַל־פְּנֵ֣י תְה֑וֹם וְר֣וּחַ אֱלֹהִ֔ים מְרַחֶ֖פֶת עַל־פְּנֵ֥י הַמָּֽיִם׃ וַיֹּ֥אמֶר אֱלֹהִ֖ים יְהִ֣י א֑וֹר וַֽיְהִי־אֽוֹר׃ וַיַּ֧רְא אֱלֹהִ֛ים אֶת־הָא֖וֹר כִּי־ט֑וֹב וַיַּבְדֵּ֣ל אֱלֹהִ֔ים בֵּ֥ין הָא֖וֹר וּבֵ֥ין הַחֹֽשֶׁךְ׃ וַיִּקְרָ֨א אֱלֹהִ֤ים ׀ לָאוֹר֙ י֔וֹם וְלַחֹ֖שֶׁךְ קָ֣רָא לָ֑יְלָה וַֽיְהִי־עֶ֥רֶב וַֽיְהִי־בֹ֖קֶר י֥וֹם אֶחָֽד׃ 
\aliyah{לוי}
וַיֹּ֣אמֶר אֱלֹהִ֔ים יְהִ֥י רָקִ֖יעַ בְּת֣וֹךְ הַמָּ֑יִם וִיהִ֣י מַבְדִּ֔יל בֵּ֥ין מַ֖יִם לָמָֽיִם׃ וַיַּ֣עַשׂ אֱלֹהִים֮ אֶת־הָרָקִ֒יעַ֒ וַיַּבְדֵּ֗ל בֵּ֤ין הַמַּ֙יִם֙ אֲשֶׁר֙ מִתַּ֣חַת לָרָקִ֔יעַ וּבֵ֣ין הַמַּ֔יִם אֲשֶׁ֖ר מֵעַ֣ל לָרָקִ֑יעַ וַֽיְהִי־כֵֽן׃ וַיִּקְרָ֧א אֱלֹהִ֛ים לָֽרָקִ֖יעַ שָׁמָ֑יִם וַֽיְהִי־עֶ֥רֶב וַֽיְהִי־בֹ֖קֶר י֥וֹם שֵׁנִֽי׃ 
\aliyah{ישראל}
וַיֹּ֣אמֶר אֱלֹהִ֗ים יִקָּו֨וּ הַמַּ֜יִם מִתַּ֤חַת הַשָּׁמַ֙יִם֙ אֶל־מָק֣וֹם אֶחָ֔ד וְתֵרָאֶ֖ה הַיַּבָּשָׁ֑ה וַֽיְהִי־כֵֽן׃ וַיִּקְרָ֨א אֱלֹהִ֤ים ׀ לַיַּבָּשָׁה֙ אֶ֔רֶץ וּלְמִקְוֵ֥ה הַמַּ֖יִם קָרָ֣א יַמִּ֑ים וַיַּ֥רְא אֱלֹהִ֖ים כִּי־טֽוֹב׃ וַיֹּ֣אמֶר אֱלֹהִ֗ים תַּֽדְשֵׁ֤א הָאָ֙רֶץ֙ דֶּ֗שֶׁא עֵ֚שֶׂב מַזְרִ֣יעַ זֶ֔רַע עֵ֣ץ פְּרִ֞י עֹ֤שֶׂה פְּרִי֙ לְמִינ֔וֹ אֲשֶׁ֥ר זַרְעוֹ־ב֖וֹ עַל־הָאָ֑רֶץ וַֽיְהִי־כֵֽן׃ וַתּוֹצֵ֨א הָאָ֜רֶץ דֶּ֠שֶׁא עֵ֣שֶׂב מַזְרִ֤יעַ זֶ֙רַע֙ לְמִינֵ֔הוּ וְעֵ֧ץ עֹֽשֶׂה־פְּרִ֛י אֲשֶׁ֥ר זַרְעוֹ־ב֖וֹ לְמִינֵ֑הוּ וַיַּ֥רְא אֱלֹהִ֖ים כִּי־טֽוֹב׃ וַֽיְהִי־עֶ֥רֶב וַֽיְהִי־בֹ֖קֶר י֥וֹם שְׁלִישִֽׁי׃ 


\ssubsection{נח}\\
אֵ֚לֶּה תּוֹלְדֹ֣ת נֹ֔חַ נֹ֗חַ אִ֥ישׁ צַדִּ֛יק תָּמִ֥ים הָיָ֖ה בְּדֹֽרֹתָ֑יו אֶת־הָֽאֱלֹהִ֖ים הִֽתְהַלֶּךְ־נֹֽחַ׃ וַיּ֥וֹלֶד נֹ֖חַ שְׁלֹשָׁ֣ה בָנִ֑ים אֶת־שֵׁ֖ם אֶת־חָ֥ם וְאֶת־יָֽפֶת׃ וַתִּשָּׁחֵ֥ת הָאָ֖רֶץ לִפְנֵ֣י הָֽאֱלֹהִ֑ים וַתִּמָּלֵ֥א הָאָ֖רֶץ חָמָֽס׃ וַיַּ֧רְא אֱלֹהִ֛ים אֶת־הָאָ֖רֶץ וְהִנֵּ֣ה נִשְׁחָ֑תָה כִּֽי־הִשְׁחִ֧ית כׇּל־בָּשָׂ֛ר אֶת־דַּרְכּ֖וֹ עַל־הָאָֽרֶץ׃ {ס} וַיֹּ֨אמֶר אֱלֹהִ֜ים לְנֹ֗חַ קֵ֤ץ כׇּל־בָּשָׂר֙ בָּ֣א לְפָנַ֔י כִּֽי־מָלְאָ֥ה הָאָ֛רֶץ חָמָ֖ס מִפְּנֵיהֶ֑ם וְהִנְנִ֥י מַשְׁחִיתָ֖ם אֶת־הָאָֽרֶץ׃ עֲשֵׂ֤ה לְךָ֙ תֵּבַ֣ת עֲצֵי־גֹ֔פֶר קִנִּ֖ים תַּֽעֲשֶׂ֣ה אֶת־הַתֵּבָ֑ה וְכָֽפַרְתָּ֥ אֹתָ֛הּ מִבַּ֥יִת וּמִח֖וּץ בַּכֹּֽפֶר׃ וְזֶ֕ה אֲשֶׁ֥ר תַּֽעֲשֶׂ֖ה אֹתָ֑הּ שְׁלֹ֧שׁ מֵא֣וֹת אַמָּ֗ה אֹ֚רֶךְ הַתֵּבָ֔ה חֲמִשִּׁ֤ים אַמָּה֙ רׇחְבָּ֔הּ וּשְׁלֹשִׁ֥ים אַמָּ֖ה קוֹמָתָֽהּ׃ צֹ֣הַר ׀ תַּֽעֲשֶׂ֣ה לַתֵּבָ֗ה וְאֶל־אַמָּה֙ תְּכַלֶּ֣נָּה מִלְמַ֔עְלָה וּפֶ֥תַח הַתֵּבָ֖ה בְּצִדָּ֣הּ תָּשִׂ֑ים תַּחְתִּיִּ֛ם שְׁנִיִּ֥ם וּשְׁלִשִׁ֖ים תַּֽעֲשֶֽׂהָ׃
\aliyah{לוי}
וַאֲנִ֗י הִנְנִי֩ מֵבִ֨יא אֶת־הַמַּבּ֥וּל מַ֙יִם֙ עַל־הָאָ֔רֶץ לְשַׁחֵ֣ת כׇּל־בָּשָׂ֗ר אֲשֶׁר־בּוֹ֙ ר֣וּחַ חַיִּ֔ים מִתַּ֖חַת הַשָּׁמָ֑יִם כֹּ֥ל אֲשֶׁר־בָּאָ֖רֶץ יִגְוָֽע׃ וַהֲקִמֹתִ֥י אֶת־בְּרִיתִ֖י אִתָּ֑ךְ וּבָאתָ֙ אֶל־הַתֵּבָ֔ה אַתָּ֕ה וּבָנֶ֛יךָ וְאִשְׁתְּךָ֥ וּנְשֵֽׁי־בָנֶ֖יךָ אִתָּֽךְ׃ וּמִכׇּל־הָ֠חַ֠י מִֽכׇּל־בָּשָׂ֞ר שְׁנַ֧יִם מִכֹּ֛ל תָּבִ֥יא אֶל־הַתֵּבָ֖ה לְהַחֲיֹ֣ת אִתָּ֑ךְ זָכָ֥ר וּנְקֵבָ֖ה יִֽהְיֽוּ׃
\aliyah{ישראל}
מֵהָע֣וֹף לְמִינֵ֗הוּ וּמִן־הַבְּהֵמָה֙ לְמִינָ֔הּ מִכֹּ֛ל רֶ֥מֶשׂ הָֽאֲדָמָ֖ה לְמִינֵ֑הוּ שְׁנַ֧יִם מִכֹּ֛ל יָבֹ֥אוּ אֵלֶ֖יךָ לְהַֽחֲיֽוֹת׃ וְאַתָּ֣ה קַח־לְךָ֗ מִכׇּל־מַֽאֲכָל֙ אֲשֶׁ֣ר יֵֽאָכֵ֔ל וְאָסַפְתָּ֖ אֵלֶ֑יךָ וְהָיָ֥ה לְךָ֛ וְלָהֶ֖ם לְאׇכְלָֽה׃ וַיַּ֖עַשׂ נֹ֑חַ כְּ֠כֹ֠ל אֲשֶׁ֨ר צִוָּ֥ה אֹת֛וֹ אֱלֹהִ֖ים כֵּ֥ן עָשָֽׂה׃


\ssubsection{לך־לך}\\
וַיֹּ֤אמֶר יְיָ֙ אֶל־אַבְרָ֔ם לֶךְ־לְךָ֛ מֵאַרְצְךָ֥ וּמִמּֽוֹלַדְתְּךָ֖ וּמִבֵּ֣ית אָבִ֑יךָ אֶל־הָאָ֖רֶץ אֲשֶׁ֥ר אַרְאֶֽךָּ׃ וְאֶֽעֶשְׂךָ֙ לְג֣וֹי גָּד֔וֹל וַאֲבָ֣רֶכְךָ֔ וַאֲגַדְּלָ֖ה שְׁמֶ֑ךָ וֶהְיֵ֖ה בְּרָכָֽה׃ וַאֲבָֽרְכָה֙ מְבָ֣רְכֶ֔יךָ וּמְקַלֶּלְךָ֖ אָאֹ֑ר וְנִבְרְכ֣וּ בְךָ֔ כֹּ֖ל מִשְׁפְּחֹ֥ת הָאֲדָמָֽה׃
\aliyah{לוי}
וַיֵּ֣לֶךְ אַבְרָ֗ם כַּאֲשֶׁ֨ר דִּבֶּ֤ר אֵלָיו֙ יְיָ֔ וַיֵּ֥לֶךְ אִתּ֖וֹ ל֑וֹט וְאַבְרָ֗ם בֶּן־חָמֵ֤שׁ שָׁנִים֙ וְשִׁבְעִ֣ים שָׁנָ֔ה בְּצֵאת֖וֹ מֵחָרָֽן׃ וַיִּקַּ֣ח אַבְרָם֩ אֶת־שָׂרַ֨י אִשְׁתּ֜וֹ וְאֶת־ל֣וֹט בֶּן־אָחִ֗יו וְאֶת־כׇּל־רְכוּשָׁם֙ אֲשֶׁ֣ר רָכָ֔שׁוּ וְאֶת־הַנֶּ֖פֶשׁ אֲשֶׁר־עָשׂ֣וּ בְחָרָ֑ן וַיֵּצְא֗וּ לָלֶ֙כֶת֙ אַ֣רְצָה כְּנַ֔עַן וַיָּבֹ֖אוּ אַ֥רְצָה כְּנָֽעַן׃ וַיַּעֲבֹ֤ר אַבְרָם֙ בָּאָ֔רֶץ עַ֚ד מְק֣וֹם שְׁכֶ֔ם עַ֖ד אֵל֣וֹן מוֹרֶ֑ה וְהַֽכְּנַעֲנִ֖י אָ֥ז בָּאָֽרֶץ׃ וַיֵּרָ֤א יְיָ֙ אֶל־אַבְרָ֔ם וַיֹּ֕אמֶר לְזַ֨רְעֲךָ֔ אֶתֵּ֖ן אֶת־הָאָ֣רֶץ הַזֹּ֑את וַיִּ֤בֶן שָׁם֙ מִזְבֵּ֔חַ לַייָ֖ הַנִּרְאֶ֥ה אֵלָֽיו׃ וַיַּעְתֵּ֨ק מִשָּׁ֜ם הָהָ֗רָה מִקֶּ֛דֶם לְבֵֽית־אֵ֖ל וַיֵּ֣ט אׇהֳלֹ֑ה בֵּֽית־אֵ֤ל מִיָּם֙ וְהָעַ֣י מִקֶּ֔דֶם וַיִּֽבֶן־שָׁ֤ם מִזְבֵּ֙חַ֙ לַֽייָ֔ וַיִּקְרָ֖א בְּשֵׁ֥ם יְיָ׃ וַיִּסַּ֣ע אַבְרָ֔ם הָל֥וֹךְ וְנָס֖וֹעַ הַנֶּֽגְבָּה׃ 
\aliyah{ישראל}
וַיְהִ֥י רָעָ֖ב בָּאָ֑רֶץ וַיֵּ֨רֶד אַבְרָ֤ם מִצְרַ֙יְמָה֙ לָג֣וּר שָׁ֔ם כִּֽי־כָבֵ֥ד הָרָעָ֖ב בָּאָֽרֶץ׃ וַיְהִ֕י כַּאֲשֶׁ֥ר הִקְרִ֖יב לָב֣וֹא מִצְרָ֑יְמָה וַיֹּ֙אמֶר֙ אֶל־שָׂרַ֣י אִשְׁתּ֔וֹ הִנֵּה־נָ֣א יָדַ֔עְתִּי כִּ֛י אִשָּׁ֥ה יְפַת־מַרְאֶ֖ה אָֽתְּ׃ וְהָיָ֗ה כִּֽי־יִרְא֤וּ אֹתָךְ֙ הַמִּצְרִ֔ים וְאָמְר֖וּ אִשְׁתּ֣וֹ זֹ֑את וְהָרְג֥וּ אֹתִ֖י וְאֹתָ֥ךְ יְחַיּֽוּ׃ אִמְרִי־נָ֖א אֲחֹ֣תִי אָ֑תְּ לְמַ֙עַן֙ יִֽיטַב־לִ֣י בַעֲבוּרֵ֔ךְ וְחָיְתָ֥ה נַפְשִׁ֖י בִּגְלָלֵֽךְ׃


\ssubsection{וירא}\\
וַיֵּרָ֤א אֵלָיו֙ יְיָ֔ בְּאֵלֹנֵ֖י מַמְרֵ֑א וְה֛וּא יֹשֵׁ֥ב פֶּֽתַח־הָאֹ֖הֶל כְּחֹ֥ם הַיּֽוֹם׃ וַיִּשָּׂ֤א עֵינָיו֙ וַיַּ֔רְא וְהִנֵּה֙ שְׁלֹשָׁ֣ה אֲנָשִׁ֔ים נִצָּבִ֖ים עָלָ֑יו וַיַּ֗רְא וַיָּ֤רׇץ לִקְרָאתָם֙ מִפֶּ֣תַח הָאֹ֔הֶל וַיִּשְׁתַּ֖חוּ אָֽרְצָה׃ וַיֹּאמַ֑ר אֲדֹנָ֗י אִם־נָ֨א מָצָ֤אתִי חֵן֙ בְּעֵינֶ֔יךָ אַל־נָ֥א תַעֲבֹ֖ר מֵעַ֥ל עַבְדֶּֽךָ׃ יֻקַּֽח־נָ֣א מְעַט־מַ֔יִם וְרַחֲצ֖וּ רַגְלֵיכֶ֑ם וְהִֽשָּׁעֲנ֖וּ תַּ֥חַת הָעֵֽץ׃ וְאֶקְחָ֨ה פַת־לֶ֜חֶם וְסַעֲד֤וּ לִבְּכֶם֙ אַחַ֣ר תַּעֲבֹ֔רוּ כִּֽי־עַל־כֵּ֥ן עֲבַרְתֶּ֖ם עַֽל־עַבְדְּכֶ֑ם וַיֹּ֣אמְר֔וּ כֵּ֥ן תַּעֲשֶׂ֖ה כַּאֲשֶׁ֥ר דִּבַּֽרְתָּ׃
\aliyah{לוי}
וַיְמַהֵ֧ר אַבְרָהָ֛ם הָאֹ֖הֱלָה אֶל־שָׂרָ֑ה וַיֹּ֗אמֶר מַהֲרִ֞י שְׁלֹ֤שׁ סְאִים֙ קֶ֣מַח סֹ֔לֶת ל֖וּשִׁי וַעֲשִׂ֥י עֻגֽוֹת׃ וְאֶל־הַבָּקָ֖ר רָ֣ץ אַבְרָהָ֑ם וַיִּקַּ֨ח בֶּן־בָּקָ֜ר רַ֤ךְ וָטוֹב֙ וַיִּתֵּ֣ן אֶל־הַנַּ֔עַר וַיְמַהֵ֖ר לַעֲשׂ֥וֹת אֹתֽוֹ׃ וַיִּקַּ֨ח חֶמְאָ֜ה וְחָלָ֗ב וּבֶן־הַבָּקָר֙ אֲשֶׁ֣ר עָשָׂ֔ה וַיִּתֵּ֖ן לִפְנֵיהֶ֑ם וְהֽוּא־עֹמֵ֧ד עֲלֵיהֶ֛ם תַּ֥חַת הָעֵ֖ץ וַיֹּאכֵֽלוּ׃
\aliyah{ישראל}
וַיֹּאמְר֣וּ אֵׄלָ֔יׄוׄ אַיֵּ֖ה שָׂרָ֣ה אִשְׁתֶּ֑ךָ וַיֹּ֖אמֶר הִנֵּ֥ה בָאֹֽהֶל׃ וַיֹּ֗אמֶר שׁ֣וֹב אָשׁ֤וּב אֵלֶ֙יךָ֙ כָּעֵ֣ת חַיָּ֔ה וְהִנֵּה־בֵ֖ן לְשָׂרָ֣ה אִשְׁתֶּ֑ךָ וְשָׂרָ֥ה שֹׁמַ֛עַת פֶּ֥תַח הָאֹ֖הֶל וְה֥וּא אַחֲרָֽיו׃ וְאַבְרָהָ֤ם וְשָׂרָה֙ זְקֵנִ֔ים בָּאִ֖ים בַּיָּמִ֑ים חָדַל֙ לִהְי֣וֹת לְשָׂרָ֔ה אֹ֖רַח כַּנָּשִֽׁים׃ וַתִּצְחַ֥ק שָׂרָ֖ה בְּקִרְבָּ֣הּ לֵאמֹ֑ר אַחֲרֵ֤י בְלֹתִי֙ הָֽיְתָה־לִּ֣י עֶדְנָ֔ה וַֽאדֹנִ֖י זָקֵֽן׃ וַיֹּ֥אמֶר יְיָ֖ אֶל־אַבְרָהָ֑ם לָ֣מָּה זֶּה֩ צָחֲקָ֨ה שָׂרָ֜ה לֵאמֹ֗ר הַאַ֥ף אֻמְנָ֛ם אֵלֵ֖ד וַאֲנִ֥י זָקַֽנְתִּי׃ הֲיִפָּלֵ֥א מֵייָ֖ דָּבָ֑ר לַמּוֹעֵ֞ד אָשׁ֥וּב אֵלֶ֛יךָ כָּעֵ֥ת חַיָּ֖ה וּלְשָׂרָ֥ה בֵֽן׃


\ssubsection{חיי שרה}\\
וַיִּהְיוּ֙ חַיֵּ֣י שָׂרָ֔ה מֵאָ֥ה שָׁנָ֛ה וְעֶשְׂרִ֥ים שָׁנָ֖ה וְשֶׁ֣בַע שָׁנִ֑ים שְׁנֵ֖י חַיֵּ֥י שָׂרָֽה׃ וַתָּ֣מׇת שָׂרָ֗ה בְּקִרְיַ֥ת אַרְבַּ֛ע הִ֥וא חֶבְר֖וֹן בְּאֶ֣רֶץ כְּנָ֑עַן וַיָּבֹא֙ אַבְרָהָ֔ם לִסְפֹּ֥ד לְשָׂרָ֖ה וְלִבְכֹּתָֽהּ׃ וַיָּ֙קׇם֙ אַבְרָהָ֔ם מֵעַ֖ל פְּנֵ֣י מֵת֑וֹ וַיְדַבֵּ֥ר אֶל־בְּנֵי־חֵ֖ת לֵאמֹֽר׃ גֵּר־וְתוֹשָׁ֥ב אָנֹכִ֖י עִמָּכֶ֑ם תְּנ֨וּ לִ֤י אֲחֻזַּת־קֶ֙בֶר֙ עִמָּכֶ֔ם וְאֶקְבְּרָ֥ה מֵתִ֖י מִלְּפָנָֽי׃ וַיַּעֲנ֧וּ בְנֵי־חֵ֛ת אֶת־אַבְרָהָ֖ם לֵאמֹ֥ר לֽוֹ׃ שְׁמָעֵ֣נוּ ׀ אֲדֹנִ֗י נְשִׂ֨יא אֱלֹהִ֤ים אַתָּה֙ בְּתוֹכֵ֔נוּ בְּמִבְחַ֣ר קְבָרֵ֔ינוּ קְבֹ֖ר אֶת־מֵתֶ֑ךָ אִ֣ישׁ מִמֶּ֔נּוּ אֶת־קִבְר֛וֹ לֹֽא־יִכְלֶ֥ה מִמְּךָ֖ מִקְּבֹ֥ר מֵתֶֽךָ׃ וַיָּ֧קׇם אַבְרָהָ֛ם וַיִּשְׁתַּ֥חוּ לְעַם־הָאָ֖רֶץ לִבְנֵי־חֵֽת׃
\aliyah{לוי}
וַיְדַבֵּ֥ר אִתָּ֖ם לֵאמֹ֑ר אִם־יֵ֣שׁ אֶֽת־נַפְשְׁכֶ֗ם לִקְבֹּ֤ר אֶת־מֵתִי֙ מִלְּפָנַ֔י שְׁמָע֕וּנִי וּפִגְעוּ־לִ֖י בְּעֶפְר֥וֹן בֶּן־צֹֽחַר׃ וְיִתֶּן־לִ֗י אֶת־מְעָרַ֤ת הַמַּכְפֵּלָה֙ אֲשֶׁר־ל֔וֹ אֲשֶׁ֖ר בִּקְצֵ֣ה שָׂדֵ֑הוּ בְּכֶ֨סֶף מָלֵ֜א יִתְּנֶ֥נָּה לִּ֛י בְּתוֹכְכֶ֖ם לַאֲחֻזַּת־קָֽבֶר׃ וְעֶפְר֥וֹן יֹשֵׁ֖ב בְּת֣וֹךְ בְּנֵי־חֵ֑ת וַיַּ֩עַן֩ עֶפְר֨וֹן הַחִתִּ֤י אֶת־אַבְרָהָם֙ בְּאׇזְנֵ֣י בְנֵי־חֵ֔ת לְכֹ֛ל בָּאֵ֥י שַֽׁעַר־עִיר֖וֹ לֵאמֹֽר׃ לֹֽא־אֲדֹנִ֣י שְׁמָעֵ֔נִי הַשָּׂדֶה֙ נָתַ֣תִּי לָ֔ךְ וְהַמְּעָרָ֥ה אֲשֶׁר־בּ֖וֹ לְךָ֣ נְתַתִּ֑יהָ לְעֵינֵ֧י בְנֵי־עַמִּ֛י נְתַתִּ֥יהָ לָּ֖ךְ קְבֹ֥ר מֵתֶֽךָ׃ וַיִּשְׁתַּ֙חוּ֙ אַבְרָהָ֔ם לִפְנֵ֖י עַ֥ם הָאָֽרֶץ׃
\aliyah{ישראל}
וַיְדַבֵּ֨ר אֶל־עֶפְר֜וֹן בְּאׇזְנֵ֤י עַם־הָאָ֙רֶץ֙ לֵאמֹ֔ר אַ֛ךְ אִם־אַתָּ֥ה ל֖וּ שְׁמָעֵ֑נִי נָתַ֜תִּי כֶּ֤סֶף הַשָּׂדֶה֙ קַ֣ח מִמֶּ֔נִּי וְאֶקְבְּרָ֥ה אֶת־מֵתִ֖י שָֽׁמָּה׃ וַיַּ֧עַן עֶפְר֛וֹן אֶת־אַבְרָהָ֖ם לֵאמֹ֥ר לֽוֹ׃ אֲדֹנִ֣י שְׁמָעֵ֔נִי אֶ֩רֶץ֩ אַרְבַּ֨ע מֵאֹ֧ת שֶֽׁקֶל־כֶּ֛סֶף בֵּינִ֥י וּבֵֽינְךָ֖ מַה־הִ֑וא וְאֶת־מֵתְךָ֖ קְבֹֽר׃ וַיִּשְׁמַ֣ע אַבְרָהָם֮ אֶל־עֶפְרוֹן֒ וַיִּשְׁקֹ֤ל אַבְרָהָם֙ לְעֶפְרֹ֔ן אֶת־הַכֶּ֕סֶף אֲשֶׁ֥ר דִּבֶּ֖ר בְּאׇזְנֵ֣י בְנֵי־חֵ֑ת אַרְבַּ֤ע מֵאוֹת֙ שֶׁ֣קֶל כֶּ֔סֶף עֹבֵ֖ר לַסֹּחֵֽר׃


\ssubsection{תולדות}\\
וְאֵ֛לֶּה תּוֹלְדֹ֥ת יִצְחָ֖ק בֶּן־אַבְרָהָ֑ם אַבְרָהָ֖ם הוֹלִ֥יד אֶת־יִצְחָֽק׃ וַיְהִ֤י יִצְחָק֙ בֶּן־אַרְבָּעִ֣ים שָׁנָ֔ה בְּקַחְתּ֣וֹ אֶת־רִבְקָ֗ה בַּת־בְּתוּאֵל֙ הָֽאֲרַמִּ֔י מִפַּדַּ֖ן אֲרָ֑ם אֲח֛וֹת לָבָ֥ן הָאֲרַמִּ֖י ל֥וֹ לְאִשָּֽׁה׃ וַיֶּעְתַּ֨ר יִצְחָ֤ק לַֽייָ֙ לְנֹ֣כַח אִשְׁתּ֔וֹ כִּ֥י עֲקָרָ֖ה הִ֑וא וַיֵּעָ֤תֶר לוֹ֙ יְיָ֔ וַתַּ֖הַר רִבְקָ֥ה אִשְׁתּֽוֹ׃ וַיִּתְרֹֽצְצ֤וּ הַבָּנִים֙ בְּקִרְבָּ֔הּ וַתֹּ֣אמֶר אִם־כֵּ֔ן לָ֥מָּה זֶּ֖ה אָנֹ֑כִי וַתֵּ֖לֶךְ לִדְרֹ֥שׁ אֶת־יְיָ׃
\aliyah{לוי}
וַיֹּ֨אמֶר יְיָ֜ לָ֗הּ שְׁנֵ֤י (גיים) [גוֹיִם֙] בְּבִטְנֵ֔ךְ וּשְׁנֵ֣י לְאֻמִּ֔ים מִמֵּעַ֖יִךְ יִפָּרֵ֑דוּ וּלְאֹם֙ מִלְאֹ֣ם יֶֽאֱמָ֔ץ וְרַ֖ב יַעֲבֹ֥ד צָעִֽיר׃ וַיִּמְלְא֥וּ יָמֶ֖יהָ לָלֶ֑דֶת וְהִנֵּ֥ה תוֹמִ֖ם בְּבִטְנָֽהּ׃ וַיֵּצֵ֤א הָרִאשׁוֹן֙ אַדְמוֹנִ֔י כֻּלּ֖וֹ כְּאַדֶּ֣רֶת שֵׂעָ֑ר וַיִּקְרְא֥וּ שְׁמ֖וֹ עֵשָֽׂו׃ וְאַֽחֲרֵי־כֵ֞ן יָצָ֣א אָחִ֗יו וְיָד֤וֹ אֹחֶ֙זֶת֙ בַּעֲקֵ֣ב עֵשָׂ֔ו וַיִּקְרָ֥א שְׁמ֖וֹ יַעֲקֹ֑ב וְיִצְחָ֛ק בֶּן־שִׁשִּׁ֥ים שָׁנָ֖ה בְּלֶ֥דֶת אֹתָֽם׃
\aliyah{ישראל}
וַֽיִּגְדְּלוּ֙ הַנְּעָרִ֔ים וַיְהִ֣י עֵשָׂ֗ו אִ֛ישׁ יֹדֵ֥עַ צַ֖יִד אִ֣ישׁ שָׂדֶ֑ה וְיַעֲקֹב֙ אִ֣ישׁ תָּ֔ם יֹשֵׁ֖ב אֹהָלִֽים׃ וַיֶּאֱהַ֥ב יִצְחָ֛ק אֶת־עֵשָׂ֖ו כִּי־צַ֣יִד בְּפִ֑יו וְרִבְקָ֖ה אֹהֶ֥בֶת אֶֽת־יַעֲקֹֽב׃ וַיָּ֥זֶד יַעֲקֹ֖ב נָזִ֑יד וַיָּבֹ֥א עֵשָׂ֛ו מִן־הַשָּׂדֶ֖ה וְה֥וּא עָיֵֽף׃ וַיֹּ֨אמֶר עֵשָׂ֜ו אֶֽל־יַעֲקֹ֗ב הַלְעִיטֵ֤נִי נָא֙ מִן־הָאָדֹ֤ם הָאָדֹם֙ הַזֶּ֔ה כִּ֥י עָיֵ֖ף אָנֹ֑כִי עַל־כֵּ֥ן קָרָֽא־שְׁמ֖וֹ אֱדֽוֹם׃ וַיֹּ֖אמֶר יַעֲקֹ֑ב מִכְרָ֥ה כַיּ֛וֹם אֶת־בְּכֹרָֽתְךָ֖ לִֽי׃ וַיֹּ֣אמֶר עֵשָׂ֔ו הִנֵּ֛ה אָנֹכִ֥י הוֹלֵ֖ךְ לָמ֑וּת וְלָמָּה־זֶּ֥ה לִ֖י בְּכֹרָֽה׃ וַיֹּ֣אמֶר יַעֲקֹ֗ב הִשָּׁ֤בְעָה לִּי֙ כַּיּ֔וֹם וַיִּשָּׁבַ֖ע ל֑וֹ וַיִּמְכֹּ֥ר אֶת־בְּכֹרָת֖וֹ לְיַעֲקֹֽב׃ וְיַעֲקֹ֞ב נָתַ֣ן לְעֵשָׂ֗ו לֶ֚חֶם וּנְזִ֣יד עֲדָשִׁ֔ים וַיֹּ֣אכַל וַיֵּ֔שְׁתְּ וַיָּ֖קׇם וַיֵּלַ֑ךְ וַיִּ֥בֶז עֵשָׂ֖ו אֶת־הַבְּכֹרָֽה׃ {פ} וַיְהִ֤י רָעָב֙ בָּאָ֔רֶץ מִלְּבַד֙ הָרָעָ֣ב הָרִאשׁ֔וֹן אֲשֶׁ֥ר הָיָ֖ה בִּימֵ֣י אַבְרָהָ֑ם וַיֵּ֧לֶךְ יִצְחָ֛ק אֶל־אֲבִימֶ֥לֶךְ מֶֽלֶךְ־פְּלִשְׁתִּ֖ים גְּרָֽרָה׃ וַיֵּרָ֤א אֵלָיו֙ יְיָ֔ וַיֹּ֖אמֶר אַל־תֵּרֵ֣ד מִצְרָ֑יְמָה שְׁכֹ֣ן בָּאָ֔רֶץ אֲשֶׁ֖ר אֹמַ֥ר אֵלֶֽיךָ׃ גּ֚וּר בָּאָ֣רֶץ הַזֹּ֔את וְאֶֽהְיֶ֥ה עִמְּךָ֖ וַאֲבָרְכֶ֑ךָּ כִּֽי־לְךָ֣ וּֽלְזַרְעֲךָ֗ אֶתֵּן֙ אֶת־כׇּל־הָֽאֲרָצֹ֣ת הָאֵ֔ל וַהֲקִֽמֹתִי֙ אֶת־הַשְּׁבֻעָ֔ה אֲשֶׁ֥ר נִשְׁבַּ֖עְתִּי לְאַבְרָהָ֥ם אָבִֽיךָ׃ וְהִרְבֵּיתִ֤י אֶֽת־זַרְעֲךָ֙ כְּכוֹכְבֵ֣י הַשָּׁמַ֔יִם וְנָתַתִּ֣י לְזַרְעֲךָ֔ אֵ֥ת כׇּל־הָאֲרָצֹ֖ת הָאֵ֑ל וְהִתְבָּרְכ֣וּ בְזַרְעֲךָ֔ כֹּ֖ל גּוֹיֵ֥י הָאָֽרֶץ׃ עֵ֕קֶב אֲשֶׁר־שָׁמַ֥ע אַבְרָהָ֖ם בְּקֹלִ֑י וַיִּשְׁמֹר֙ מִשְׁמַרְתִּ֔י מִצְוֺתַ֖י חֻקּוֹתַ֥י וְתוֹרֹתָֽי׃


\ssubsection{ויצא}\\
וַיֵּצֵ֥א יַעֲקֹ֖ב מִבְּאֵ֣ר שָׁ֑בַע וַיֵּ֖לֶךְ חָרָֽנָה׃ וַיִּפְגַּ֨ע בַּמָּק֜וֹם וַיָּ֤לֶן שָׁם֙ כִּי־בָ֣א הַשֶּׁ֔מֶשׁ וַיִּקַּח֙ מֵאַבְנֵ֣י הַמָּק֔וֹם וַיָּ֖שֶׂם מְרַֽאֲשֹׁתָ֑יו וַיִּשְׁכַּ֖ב בַּמָּק֥וֹם הַהֽוּא׃ וַֽיַּחֲלֹ֗ם וְהִנֵּ֤ה סֻלָּם֙ מֻצָּ֣ב אַ֔רְצָה וְרֹאשׁ֖וֹ מַגִּ֣יעַ הַשָּׁמָ֑יְמָה וְהִנֵּה֙ מַלְאֲכֵ֣י אֱלֹהִ֔ים עֹלִ֥ים וְיֹרְדִ֖ים בּֽוֹ׃
\aliyah{לוי}
וְהִנֵּ֨ה יְיָ֜ נִצָּ֣ב עָלָיו֮ וַיֹּאמַר֒ אֲנִ֣י יְיָ֗ אֱלֹהֵי֙ אַבְרָהָ֣ם אָבִ֔יךָ וֵאלֹהֵ֖י יִצְחָ֑ק הָאָ֗רֶץ אֲשֶׁ֤ר אַתָּה֙ שֹׁכֵ֣ב עָלֶ֔יהָ לְךָ֥ אֶתְּנֶ֖נָּה וּלְזַרְעֶֽךָ׃ וְהָיָ֤ה זַרְעֲךָ֙ כַּעֲפַ֣ר הָאָ֔רֶץ וּפָרַצְתָּ֛ יָ֥מָּה וָקֵ֖דְמָה וְצָפֹ֣נָה וָנֶ֑גְבָּה וְנִבְרְכ֥וּ בְךָ֛ כׇּל־מִשְׁפְּחֹ֥ת הָאֲדָמָ֖ה וּבְזַרְעֶֽךָ׃ וְהִנֵּ֨ה אָנֹכִ֜י עִמָּ֗ךְ וּשְׁמַרְתִּ֙יךָ֙ בְּכֹ֣ל אֲשֶׁר־תֵּלֵ֔ךְ וַהֲשִׁ֣בֹתִ֔יךָ אֶל־הָאֲדָמָ֖ה הַזֹּ֑את כִּ֚י לֹ֣א אֶֽעֱזָבְךָ֔ עַ֚ד אֲשֶׁ֣ר אִם־עָשִׂ֔יתִי אֵ֥ת אֲשֶׁר־דִּבַּ֖רְתִּי לָֽךְ׃ וַיִּיקַ֣ץ יַעֲקֹב֮ מִשְּׁנָתוֹ֒ וַיֹּ֕אמֶר אָכֵן֙ יֵ֣שׁ יְיָ֔ בַּמָּק֖וֹם הַזֶּ֑ה וְאָנֹכִ֖י לֹ֥א יָדָֽעְתִּי׃ וַיִּירָא֙ וַיֹּאמַ֔ר מַה־נּוֹרָ֖א הַמָּק֣וֹם הַזֶּ֑ה אֵ֣ין זֶ֗ה כִּ֚י אִם־בֵּ֣ית אֱלֹהִ֔ים וְזֶ֖ה שַׁ֥עַר הַשָּׁמָֽיִם׃
\aliyah{ישראל}
וַיַּשְׁכֵּ֨ם יַעֲקֹ֜ב בַּבֹּ֗קֶר וַיִּקַּ֤ח אֶת־הָאֶ֙בֶן֙ אֲשֶׁר־שָׂ֣ם מְרַֽאֲשֹׁתָ֔יו וַיָּ֥שֶׂם אֹתָ֖הּ מַצֵּבָ֑ה וַיִּצֹ֥ק שֶׁ֖מֶן עַל־רֹאשָֽׁהּ׃ וַיִּקְרָ֛א אֶת־שֵֽׁם־הַמָּק֥וֹם הַה֖וּא בֵּֽית־אֵ֑ל וְאוּלָ֛ם ל֥וּז שֵׁם־הָעִ֖יר לָרִאשֹׁנָֽה׃ וַיִּדַּ֥ר יַעֲקֹ֖ב נֶ֣דֶר לֵאמֹ֑ר אִם־יִהְיֶ֨ה אֱלֹהִ֜ים עִמָּדִ֗י וּשְׁמָרַ֙נִי֙ בַּדֶּ֤רֶךְ הַזֶּה֙ אֲשֶׁ֣ר אָנֹכִ֣י הוֹלֵ֔ךְ וְנָֽתַן־לִ֥י לֶ֛חֶם לֶאֱכֹ֖ל וּבֶ֥גֶד לִלְבֹּֽשׁ׃ וְשַׁבְתִּ֥י בְשָׁל֖וֹם אֶל־בֵּ֣ית אָבִ֑י וְהָיָ֧ה יְיָ֛ לִ֖י לֵאלֹהִֽים׃ וְהָאֶ֣בֶן הַזֹּ֗את אֲשֶׁר־שַׂ֙מְתִּי֙ מַצֵּבָ֔ה יִהְיֶ֖ה בֵּ֣ית אֱלֹהִ֑ים וְכֹל֙ אֲשֶׁ֣ר תִּתֶּן־לִ֔י עַשֵּׂ֖ר אֲעַשְּׂרֶ֥נּוּ לָֽךְ׃


\ssubsection{וישלח}\\
וַיִּשְׁלַ֨ח יַעֲקֹ֤ב מַלְאָכִים֙ לְפָנָ֔יו אֶל־עֵשָׂ֖ו אָחִ֑יו אַ֥רְצָה שֵׂעִ֖יר שְׂדֵ֥ה אֱדֽוֹם׃ וַיְצַ֤ו אֹתָם֙ לֵאמֹ֔ר כֹּ֣ה תֹאמְר֔וּן לַֽאדֹנִ֖י לְעֵשָׂ֑ו כֹּ֤ה אָמַר֙ עַבְדְּךָ֣ יַעֲקֹ֔ב עִם־לָבָ֣ן גַּ֔רְתִּי וָאֵחַ֖ר עַד־עָֽתָּה׃ וַֽיְהִי־לִי֙ שׁ֣וֹר וַחֲמ֔וֹר צֹ֖אן וְעֶ֣בֶד וְשִׁפְחָ֑ה וָֽאֶשְׁלְחָה֙ לְהַגִּ֣יד לַֽאדֹנִ֔י לִמְצֹא־חֵ֖ן בְּעֵינֶֽיךָ׃
\aliyah{לוי}
וַיָּשֻׁ֙בוּ֙ הַמַּלְאָכִ֔ים אֶֽל־יַעֲקֹ֖ב לֵאמֹ֑ר בָּ֤אנוּ אֶל־אָחִ֙יךָ֙ אֶל־עֵשָׂ֔ו וְגַם֙ הֹלֵ֣ךְ לִקְרָֽאתְךָ֔ וְאַרְבַּע־מֵא֥וֹת אִ֖ישׁ עִמּֽוֹ׃ וַיִּירָ֧א יַעֲקֹ֛ב מְאֹ֖ד וַיֵּ֣צֶר ל֑וֹ וַיַּ֜חַץ אֶת־הָעָ֣ם אֲשֶׁר־אִתּ֗וֹ וְאֶת־הַצֹּ֧אן וְאֶת־הַבָּקָ֛ר וְהַגְּמַלִּ֖ים לִשְׁנֵ֥י מַחֲנֽוֹת׃ וַיֹּ֕אמֶר אִם־יָב֥וֹא עֵשָׂ֛ו אֶל־הַמַּחֲנֶ֥ה הָאַחַ֖ת וְהִכָּ֑הוּ וְהָיָ֛ה הַמַּחֲנֶ֥ה הַנִּשְׁאָ֖ר לִפְלֵיטָֽה׃
\aliyah{ישראל}
וַיֹּ֘אמֶר֮ יַעֲקֹב֒ אֱלֹהֵי֙ אָבִ֣י אַבְרָהָ֔ם וֵאלֹהֵ֖י אָבִ֣י יִצְחָ֑ק יְיָ֞ הָאֹמֵ֣ר אֵלַ֗י שׁ֧וּב לְאַרְצְךָ֛ וּלְמוֹלַדְתְּךָ֖ וְאֵיטִ֥יבָה עִמָּֽךְ׃ קָטֹ֜נְתִּי מִכֹּ֤ל הַחֲסָדִים֙ וּמִכׇּל־הָ֣אֱמֶ֔ת אֲשֶׁ֥ר עָשִׂ֖יתָ אֶת־עַבְדֶּ֑ךָ כִּ֣י בְמַקְלִ֗י עָבַ֙רְתִּי֙ אֶת־הַיַּרְדֵּ֣ן הַזֶּ֔ה וְעַתָּ֥ה הָיִ֖יתִי לִשְׁנֵ֥י מַחֲנֽוֹת׃ הַצִּילֵ֥נִי נָ֛א מִיַּ֥ד אָחִ֖י מִיַּ֣ד עֵשָׂ֑ו כִּֽי־יָרֵ֤א אָנֹכִי֙ אֹת֔וֹ פֶּן־יָב֣וֹא וְהִכַּ֔נִי אֵ֖ם עַל־בָּנִֽים׃ וְאַתָּ֣ה אָמַ֔רְתָּ הֵיטֵ֥ב אֵיטִ֖יב עִמָּ֑ךְ וְשַׂמְתִּ֤י אֶֽת־זַרְעֲךָ֙ כְּח֣וֹל הַיָּ֔ם אֲשֶׁ֥ר לֹא־יִסָּפֵ֖ר מֵרֹֽב׃


\ssubsection{וישב}\\
וַיֵּ֣שֶׁב יַעֲקֹ֔ב בְּאֶ֖רֶץ מְגוּרֵ֣י אָבִ֑יו בְּאֶ֖רֶץ כְּנָֽעַן׃ אֵ֣לֶּה ׀ תֹּלְד֣וֹת יַעֲקֹ֗ב יוֹסֵ֞ף בֶּן־שְׁבַֽע־עֶשְׂרֵ֤ה שָׁנָה֙ הָיָ֨ה רֹעֶ֤ה אֶת־אֶחָיו֙ בַּצֹּ֔אן וְה֣וּא נַ֗עַר אֶת־בְּנֵ֥י בִלְהָ֛ה וְאֶת־בְּנֵ֥י זִלְפָּ֖ה נְשֵׁ֣י אָבִ֑יו וַיָּבֵ֥א יוֹסֵ֛ף אֶת־דִּבָּתָ֥ם רָעָ֖ה אֶל־אֲבִיהֶֽם׃ וְיִשְׂרָאֵ֗ל אָהַ֤ב אֶת־יוֹסֵף֙ מִכׇּל־בָּנָ֔יו כִּֽי־בֶן־זְקֻנִ֥ים ה֖וּא ל֑וֹ וְעָ֥שָׂה ל֖וֹ כְּתֹ֥נֶת פַּסִּֽים׃
\aliyah{לוי}
וַיִּרְא֣וּ אֶחָ֗יו כִּֽי־אֹת֞וֹ אָהַ֤ב אֲבִיהֶם֙ מִכׇּל־אֶחָ֔יו וַֽיִּשְׂנְא֖וּ אֹת֑וֹ וְלֹ֥א יָכְל֖וּ דַּבְּר֥וֹ לְשָׁלֹֽם׃ וַיַּחֲלֹ֤ם יוֹסֵף֙ חֲל֔וֹם וַיַּגֵּ֖ד לְאֶחָ֑יו וַיּוֹסִ֥פוּ ע֖וֹד שְׂנֹ֥א אֹתֽוֹ׃ וַיֹּ֖אמֶר אֲלֵיהֶ֑ם שִׁמְעוּ־נָ֕א הַחֲל֥וֹם הַזֶּ֖ה אֲשֶׁ֥ר חָלָֽמְתִּי׃ וְ֠הִנֵּ֠ה אֲנַ֜חְנוּ מְאַלְּמִ֤ים אֲלֻמִּים֙ בְּת֣וֹךְ הַשָּׂדֶ֔ה וְהִנֵּ֛ה קָ֥מָה אֲלֻמָּתִ֖י וְגַם־נִצָּ֑בָה וְהִנֵּ֤ה תְסֻבֶּ֙ינָה֙ אֲלֻמֹּ֣תֵיכֶ֔ם וַתִּֽשְׁתַּחֲוֶ֖יןָ לַאֲלֻמָּתִֽי׃
\aliyah{ישראל}
וַיֹּ֤אמְרוּ לוֹ֙ אֶחָ֔יו הֲמָלֹ֤ךְ תִּמְלֹךְ֙ עָלֵ֔ינוּ אִם־מָשׁ֥וֹל תִּמְשֹׁ֖ל בָּ֑נוּ וַיּוֹסִ֤פוּ עוֹד֙ שְׂנֹ֣א אֹת֔וֹ עַל־חֲלֹמֹתָ֖יו וְעַל־דְּבָרָֽיו׃ וַיַּחֲלֹ֥ם עוֹד֙ חֲל֣וֹם אַחֵ֔ר וַיְסַפֵּ֥ר אֹת֖וֹ לְאֶחָ֑יו וַיֹּ֗אמֶר הִנֵּ֨ה חָלַ֤מְתִּֽי חֲלוֹם֙ ע֔וֹד וְהִנֵּ֧ה הַשֶּׁ֣מֶשׁ וְהַיָּרֵ֗חַ וְאַחַ֤ד עָשָׂר֙ כּֽוֹכָבִ֔ים מִֽשְׁתַּחֲוִ֖ים לִֽי׃ וַיְסַפֵּ֣ר אֶל־אָבִיו֮ וְאֶל־אֶחָיו֒ וַיִּגְעַר־בּ֣וֹ אָבִ֔יו וַיֹּ֣אמֶר ל֔וֹ מָ֛ה הַחֲל֥וֹם הַזֶּ֖ה אֲשֶׁ֣ר חָלָ֑מְתָּ הֲב֣וֹא נָב֗וֹא אֲנִי֙ וְאִמְּךָ֣ וְאַחֶ֔יךָ לְהִשְׁתַּחֲוֺ֥ת לְךָ֖ אָֽרְצָה׃ וַיְקַנְאוּ־ב֖וֹ אֶחָ֑יו וְאָבִ֖יו שָׁמַ֥ר אֶת־הַדָּבָֽר׃


\ssubsection{מקץ}\\
וַיְהִ֕י מִקֵּ֖ץ שְׁנָתַ֣יִם יָמִ֑ים וּפַרְעֹ֣ה חֹלֵ֔ם וְהִנֵּ֖ה עֹמֵ֥ד עַל־הַיְאֹֽר׃ וְהִנֵּ֣ה מִן־הַיְאֹ֗ר עֹלֹת֙ שֶׁ֣בַע פָּר֔וֹת יְפ֥וֹת מַרְאֶ֖ה וּבְרִיאֹ֣ת בָּשָׂ֑ר וַתִּרְעֶ֖ינָה בָּאָֽחוּ׃ וְהִנֵּ֞ה שֶׁ֧בַע פָּר֣וֹת אֲחֵר֗וֹת עֹל֤וֹת אַחֲרֵיהֶן֙ מִן־הַיְאֹ֔ר רָע֥וֹת מַרְאֶ֖ה וְדַקּ֣וֹת בָּשָׂ֑ר וַֽתַּעֲמֹ֛דְנָה אֵ֥צֶל הַפָּר֖וֹת עַל־שְׂפַ֥ת הַיְאֹֽר׃ וַתֹּאכַ֣לְנָה הַפָּר֗וֹת רָע֤וֹת הַמַּרְאֶה֙ וְדַקֹּ֣ת הַבָּשָׂ֔ר אֵ֚ת שֶׁ֣בַע הַפָּר֔וֹת יְפֹ֥ת הַמַּרְאֶ֖ה וְהַבְּרִיאֹ֑ת וַיִּיקַ֖ץ פַּרְעֹֽה׃
\aliyah{לוי}
וַיִּישָׁ֕ן וַֽיַּחֲלֹ֖ם שֵׁנִ֑ית וְהִנֵּ֣ה ׀ שֶׁ֣בַע שִׁבֳּלִ֗ים עֹל֛וֹת בְּקָנֶ֥ה אֶחָ֖ד בְּרִיא֥וֹת וְטֹבֽוֹת׃ וְהִנֵּה֙ שֶׁ֣בַע שִׁבֳּלִ֔ים דַּקּ֖וֹת וּשְׁדוּפֹ֣ת קָדִ֑ים צֹמְח֖וֹת אַחֲרֵיהֶֽן׃ וַתִּבְלַ֙עְנָה֙ הַשִּׁבֳּלִ֣ים הַדַּקּ֔וֹת אֵ֚ת שֶׁ֣בַע הַֽשִּׁבֳּלִ֔ים הַבְּרִיא֖וֹת וְהַמְּלֵא֑וֹת וַיִּיקַ֥ץ פַּרְעֹ֖ה וְהִנֵּ֥ה חֲלֽוֹם׃
\aliyah{ישראל}
וַיְהִ֤י בַבֹּ֙קֶר֙ וַתִּפָּ֣עֶם רוּח֔וֹ וַיִּשְׁלַ֗ח וַיִּקְרָ֛א אֶת־כׇּל־חַרְטֻמֵּ֥י מִצְרַ֖יִם וְאֶת־כׇּל־חֲכָמֶ֑יהָ וַיְסַפֵּ֨ר פַּרְעֹ֤ה לָהֶם֙ אֶת־חֲלֹמ֔וֹ וְאֵין־פּוֹתֵ֥ר אוֹתָ֖ם לְפַרְעֹֽה׃ וַיְדַבֵּר֙ שַׂ֣ר הַמַּשְׁקִ֔ים אֶת־פַּרְעֹ֖ה לֵאמֹ֑ר אֶת־חֲטָאַ֕י אֲנִ֖י מַזְכִּ֥יר הַיּֽוֹם׃ פַּרְעֹ֖ה קָצַ֣ף עַל־עֲבָדָ֑יו וַיִּתֵּ֨ן אֹתִ֜י בְּמִשְׁמַ֗ר בֵּ֚ית שַׂ֣ר הַטַּבָּחִ֔ים אֹתִ֕י וְאֵ֖ת שַׂ֥ר הָאֹפִֽים׃ וַנַּֽחַלְמָ֥ה חֲל֛וֹם בְּלַ֥יְלָה אֶחָ֖ד אֲנִ֣י וָה֑וּא אִ֛ישׁ כְּפִתְר֥וֹן חֲלֹמ֖וֹ חָלָֽמְנוּ׃ וְשָׁ֨ם אִתָּ֜נוּ נַ֣עַר עִבְרִ֗י עֶ֚בֶד לְשַׂ֣ר הַטַּבָּחִ֔ים וַ֨נְּסַפֶּר־ל֔וֹ וַיִּפְתׇּר־לָ֖נוּ אֶת־חֲלֹמֹתֵ֑ינוּ אִ֥ישׁ כַּחֲלֹמ֖וֹ פָּתָֽר׃ וַיְהִ֛י כַּאֲשֶׁ֥ר פָּֽתַר־לָ֖נוּ כֵּ֣ן הָיָ֑ה אֹתִ֛י הֵשִׁ֥יב עַל־כַּנִּ֖י וְאֹת֥וֹ תָלָֽה׃ וַיִּשְׁלַ֤ח פַּרְעֹה֙ וַיִּקְרָ֣א אֶת־יוֹסֵ֔ף וַיְרִיצֻ֖הוּ מִן־הַבּ֑וֹר וַיְגַלַּח֙ וַיְחַלֵּ֣ף שִׂמְלֹתָ֔יו וַיָּבֹ֖א אֶל־פַּרְעֹֽה׃


\ssubsection{ויגש}\\
וַיִּגַּ֨שׁ אֵלָ֜יו יְהוּדָ֗ה וַיֹּ֘אמֶר֮ בִּ֣י אֲדֹנִי֒ יְדַבֶּר־נָ֨א עַבְדְּךָ֤ דָבָר֙ בְּאׇזְנֵ֣י אֲדֹנִ֔י וְאַל־יִ֥חַר אַפְּךָ֖ בְּעַבְדֶּ֑ךָ כִּ֥י כָמ֖וֹךָ כְּפַרְעֹֽה׃ אֲדֹנִ֣י שָׁאַ֔ל אֶת־עֲבָדָ֖יו לֵאמֹ֑ר הֲיֵשׁ־לָכֶ֥ם אָ֖ב אוֹ־אָֽח׃ וַנֹּ֙אמֶר֙ אֶל־אֲדֹנִ֔י יֶשׁ־לָ֙נוּ֙ אָ֣ב זָקֵ֔ן וְיֶ֥לֶד זְקֻנִ֖ים קָטָ֑ן וְאָחִ֣יו מֵ֔ת וַיִּוָּתֵ֨ר ה֧וּא לְבַדּ֛וֹ לְאִמּ֖וֹ וְאָבִ֥יו אֲהֵבֽוֹ׃
\aliyah{לוי}
וַתֹּ֙אמֶר֙ אֶל־עֲבָדֶ֔יךָ הוֹרִדֻ֖הוּ אֵלָ֑י וְאָשִׂ֥ימָה עֵינִ֖י עָלָֽיו׃ וַנֹּ֙אמֶר֙ אֶל־אֲדֹנִ֔י לֹא־יוּכַ֥ל הַנַּ֖עַר לַעֲזֹ֣ב אֶת־אָבִ֑יו וְעָזַ֥ב אֶת־אָבִ֖יו וָמֵֽת׃ וַתֹּ֙אמֶר֙ אֶל־עֲבָדֶ֔יךָ אִם־לֹ֥א יֵרֵ֛ד אֲחִיכֶ֥ם הַקָּטֹ֖ן אִתְּכֶ֑ם לֹ֥א תֹסִפ֖וּן לִרְא֥וֹת פָּנָֽי׃ וַיְהִי֙ כִּ֣י עָלִ֔ינוּ אֶֽל־עַבְדְּךָ֖ אָבִ֑י וַנַּ֨גֶּד־ל֔וֹ אֵ֖ת דִּבְרֵ֥י אֲדֹנִֽי׃
\aliyah{ישראל}
וַיֹּ֖אמֶר אָבִ֑ינוּ שֻׁ֖בוּ שִׁבְרוּ־לָ֥נוּ מְעַט־אֹֽכֶל׃ וַנֹּ֕אמֶר לֹ֥א נוּכַ֖ל לָרֶ֑דֶת אִם־יֵשׁ֩ אָחִ֨ינוּ הַקָּטֹ֤ן אִתָּ֙נוּ֙ וְיָרַ֔דְנוּ כִּי־לֹ֣א נוּכַ֗ל לִרְאוֹת֙ פְּנֵ֣י הָאִ֔ישׁ וְאָחִ֥ינוּ הַקָּטֹ֖ן אֵינֶ֥נּוּ אִתָּֽנוּ׃ וַיֹּ֛אמֶר עַבְדְּךָ֥ אָבִ֖י אֵלֵ֑ינוּ אַתֶּ֣ם יְדַעְתֶּ֔ם כִּ֥י שְׁנַ֖יִם יָֽלְדָה־לִּ֥י אִשְׁתִּֽי׃ וַיֵּצֵ֤א הָֽאֶחָד֙ מֵֽאִתִּ֔י וָאֹמַ֕ר אַ֖ךְ טָרֹ֣ף טֹרָ֑ף וְלֹ֥א רְאִיתִ֖יו עַד־הֵֽנָּה׃ וּלְקַחְתֶּ֧ם גַּם־אֶת־זֶ֛ה מֵעִ֥ם פָּנַ֖י וְקָרָ֣הוּ אָס֑וֹן וְהֽוֹרַדְתֶּ֧ם אֶת־שֵׂיבָתִ֛י בְּרָעָ֖ה שְׁאֹֽלָה׃ וְעַתָּ֗ה כְּבֹאִי֙ אֶל־עַבְדְּךָ֣ אָבִ֔י וְהַנַּ֖עַר אֵינֶ֣נּוּ אִתָּ֑נוּ וְנַפְשׁ֖וֹ קְשׁוּרָ֥ה בְנַפְשֽׁוֹ׃


\ssubsection{ויחי}\\
וַיְחִ֤י יַעֲקֹב֙ בְּאֶ֣רֶץ מִצְרַ֔יִם שְׁבַ֥ע עֶשְׂרֵ֖ה שָׁנָ֑ה וַיְהִ֤י יְמֵֽי־יַעֲקֹב֙ שְׁנֵ֣י חַיָּ֔יו שֶׁ֣בַע שָׁנִ֔ים וְאַרְבָּעִ֥ים וּמְאַ֖ת שָׁנָֽה׃ וַיִּקְרְב֣וּ יְמֵֽי־יִשְׂרָאֵל֮ לָמוּת֒ וַיִּקְרָ֣א ׀ לִבְנ֣וֹ לְיוֹסֵ֗ף וַיֹּ֤אמֶר לוֹ֙ אִם־נָ֨א מָצָ֤אתִי חֵן֙ בְּעֵינֶ֔יךָ שִֽׂים־נָ֥א יָדְךָ֖ תַּ֣חַת יְרֵכִ֑י וְעָשִׂ֤יתָ עִמָּדִי֙ חֶ֣סֶד וֶאֱמֶ֔ת אַל־נָ֥א תִקְבְּרֵ֖נִי בְּמִצְרָֽיִם׃ וְשָֽׁכַבְתִּי֙ עִם־אֲבֹתַ֔י וּנְשָׂאתַ֙נִי֙ מִמִּצְרַ֔יִם וּקְבַרְתַּ֖נִי בִּקְבֻרָתָ֑ם וַיֹּאמַ֕ר אָנֹכִ֖י אֶֽעֱשֶׂ֥ה כִדְבָרֶֽךָ׃ וַיֹּ֗אמֶר הִשָּֽׁבְעָה֙ לִ֔י וַיִּשָּׁבַ֖ע ל֑וֹ וַיִּשְׁתַּ֥חוּ יִשְׂרָאֵ֖ל עַל־רֹ֥אשׁ הַמִּטָּֽה׃ 
\aliyah{לוי}
וַיְהִ֗י אַחֲרֵי֙ הַדְּבָרִ֣ים הָאֵ֔לֶּה וַיֹּ֣אמֶר לְיוֹסֵ֔ף הִנֵּ֥ה אָבִ֖יךָ חֹלֶ֑ה וַיִּקַּ֞ח אֶת־שְׁנֵ֤י בָנָיו֙ עִמּ֔וֹ אֶת־מְנַשֶּׁ֖ה וְאֶת־אֶפְרָֽיִם׃ וַיַּגֵּ֣ד לְיַעֲקֹ֔ב וַיֹּ֕אמֶר הִנֵּ֛ה בִּנְךָ֥ יוֹסֵ֖ף בָּ֣א אֵלֶ֑יךָ וַיִּתְחַזֵּק֙ יִשְׂרָאֵ֔ל וַיֵּ֖שֶׁב עַל־הַמִּטָּֽה׃ וַיֹּ֤אמֶר יַעֲקֹב֙ אֶל־יוֹסֵ֔ף אֵ֥ל שַׁדַּ֛י נִרְאָֽה־אֵלַ֥י בְּל֖וּז בְּאֶ֣רֶץ כְּנָ֑עַן וַיְבָ֖רֶךְ אֹתִֽי׃
\aliyah{ישראל}
וַיֹּ֣אמֶר אֵלַ֗י הִנְנִ֤י מַפְרְךָ֙ וְהִרְבִּיתִ֔ךָ וּנְתַתִּ֖יךָ לִקְהַ֣ל עַמִּ֑ים וְנָ֨תַתִּ֜י אֶת־הָאָ֧רֶץ הַזֹּ֛את לְזַרְעֲךָ֥ אַחֲרֶ֖יךָ אֲחֻזַּ֥ת עוֹלָֽם׃ וְעַתָּ֡ה שְׁנֵֽי־בָנֶ֩יךָ֩ הַנּוֹלָדִ֨ים לְךָ֜ בְּאֶ֣רֶץ מִצְרַ֗יִם עַד־בֹּאִ֥י אֵלֶ֛יךָ מִצְרַ֖יְמָה לִי־הֵ֑ם אֶפְרַ֙יִם֙ וּמְנַשֶּׁ֔ה כִּרְאוּבֵ֥ן וְשִׁמְע֖וֹן יִֽהְיוּ־לִֽי׃ וּמוֹלַדְתְּךָ֛ אֲשֶׁר־הוֹלַ֥דְתָּ אַחֲרֵיהֶ֖ם לְךָ֣ יִהְי֑וּ עַ֣ל שֵׁ֧ם אֲחֵיהֶ֛ם יִקָּרְא֖וּ בְּנַחֲלָתָֽם׃ וַאֲנִ֣י ׀ בְּבֹאִ֣י מִפַּדָּ֗ן מֵ֩תָה֩ עָלַ֨י רָחֵ֜ל בְּאֶ֤רֶץ כְּנַ֙עַן֙ בַּדֶּ֔רֶךְ בְּע֥וֹד כִּבְרַת־אֶ֖רֶץ לָבֹ֣א אֶפְרָ֑תָה וָאֶקְבְּרֶ֤הָ שָּׁם֙ בְּדֶ֣רֶךְ אֶפְרָ֔ת הִ֖וא בֵּ֥ית לָֽחֶם׃ וַיַּ֥רְא יִשְׂרָאֵ֖ל אֶת־בְּנֵ֣י יוֹסֵ֑ף וַיֹּ֖אמֶר מִי־אֵֽלֶּה׃ וַיֹּ֤אמֶר יוֹסֵף֙ אֶל־אָבִ֔יו בָּנַ֣י הֵ֔ם אֲשֶׁר־נָֽתַן־לִ֥י אֱלֹהִ֖ים בָּזֶ֑ה וַיֹּאמַ֕ר קָֽחֶם־נָ֥א אֵלַ֖י וַאֲבָרְכֵֽם׃


\ssubsection{שמות}\\
וְאֵ֗לֶּה שְׁמוֹת֙ בְּנֵ֣י יִשְׂרָאֵ֔ל הַבָּאִ֖ים מִצְרָ֑יְמָה אֵ֣ת יַעֲקֹ֔ב אִ֥ישׁ וּבֵית֖וֹ בָּֽאוּ׃ רְאוּבֵ֣ן שִׁמְע֔וֹן לֵוִ֖י וִיהוּדָֽה׃ יִשָּׂשכָ֥ר זְבוּלֻ֖ן וּבִנְיָמִֽן׃ דָּ֥ן וְנַפְתָּלִ֖י גָּ֥ד וְאָשֵֽׁר׃ וַֽיְהִ֗י כׇּל־נֶ֛פֶשׁ יֹצְאֵ֥י יֶֽרֶךְ־יַעֲקֹ֖ב שִׁבְעִ֣ים נָ֑פֶשׁ וְיוֹסֵ֖ף הָיָ֥ה בְמִצְרָֽיִם׃ וַיָּ֤מׇת יוֹסֵף֙ וְכׇל־אֶחָ֔יו וְכֹ֖ל הַדּ֥וֹר הַהֽוּא׃ וּבְנֵ֣י יִשְׂרָאֵ֗ל פָּר֧וּ וַֽיִּשְׁרְצ֛וּ וַיִּרְבּ֥וּ וַיַּֽעַצְמ֖וּ בִּמְאֹ֣ד מְאֹ֑ד וַתִּמָּלֵ֥א הָאָ֖רֶץ אֹתָֽם׃ 
\aliyah{לוי}
וַיָּ֥קׇם מֶֽלֶךְ־חָדָ֖שׁ עַל־מִצְרָ֑יִם אֲשֶׁ֥ר לֹֽא־יָדַ֖ע אֶת־יוֹסֵֽף׃ וַיֹּ֖אמֶר אֶל־עַמּ֑וֹ הִנֵּ֗ה עַ֚ם בְּנֵ֣י יִשְׂרָאֵ֔ל רַ֥ב וְעָצ֖וּם מִמֶּֽנּוּ׃ הָ֥בָה נִֽתְחַכְּמָ֖ה ל֑וֹ פֶּן־יִרְבֶּ֗ה וְהָיָ֞ה כִּֽי־תִקְרֶ֤אנָה מִלְחָמָה֙ וְנוֹסַ֤ף גַּם־הוּא֙ עַל־שֹׂ֣נְאֵ֔ינוּ וְנִלְחַם־בָּ֖נוּ וְעָלָ֥ה מִן־הָאָֽרֶץ׃ וַיָּשִׂ֤ימוּ עָלָיו֙ שָׂרֵ֣י מִסִּ֔ים לְמַ֥עַן עַנֹּת֖וֹ בְּסִבְלֹתָ֑ם וַיִּ֜בֶן עָרֵ֤י מִסְכְּנוֹת֙ לְפַרְעֹ֔ה אֶת־פִּתֹ֖ם וְאֶת־רַעַמְסֵֽס׃ וְכַאֲשֶׁר֙ יְעַנּ֣וּ אֹת֔וֹ כֵּ֥ן יִרְבֶּ֖ה וְכֵ֣ן יִפְרֹ֑ץ וַיָּקֻ֕צוּ מִפְּנֵ֖י בְּנֵ֥י יִשְׂרָאֵֽל׃
\aliyah{ישראל}
וַיַּעֲבִ֧דוּ מִצְרַ֛יִם אֶת־בְּנֵ֥י יִשְׂרָאֵ֖ל בְּפָֽרֶךְ׃ וַיְמָרְר֨וּ אֶת־חַיֵּיהֶ֜ם בַּעֲבֹדָ֣ה קָשָׁ֗ה בְּחֹ֙מֶר֙ וּבִלְבֵנִ֔ים וּבְכׇל־עֲבֹדָ֖ה בַּשָּׂדֶ֑ה אֵ֚ת כׇּל־עֲבֹ֣דָתָ֔ם אֲשֶׁר־עָבְד֥וּ בָהֶ֖ם בְּפָֽרֶךְ׃ וַיֹּ֙אמֶר֙ מֶ֣לֶךְ מִצְרַ֔יִם לַֽמְיַלְּדֹ֖ת הָֽעִבְרִיֹּ֑ת אֲשֶׁ֨ר שֵׁ֤ם הָֽאַחַת֙ שִׁפְרָ֔ה וְשֵׁ֥ם הַשֵּׁנִ֖ית פּוּעָֽה׃ וַיֹּ֗אמֶר בְּיַלֶּדְכֶן֙ אֶת־הָֽעִבְרִיּ֔וֹת וּרְאִיתֶ֖ן עַל־הָאׇבְנָ֑יִם אִם־בֵּ֥ן הוּא֙ וַהֲמִתֶּ֣ן אֹת֔וֹ וְאִם־בַּ֥ת הִ֖וא וָחָֽיָה׃ וַתִּירֶ֤אןָ הַֽמְיַלְּדֹת֙ אֶת־הָ֣אֱלֹהִ֔ים וְלֹ֣א עָשׂ֔וּ כַּאֲשֶׁ֛ר דִּבֶּ֥ר אֲלֵיהֶ֖ן מֶ֣לֶךְ מִצְרָ֑יִם וַתְּחַיֶּ֖יןָ אֶת־הַיְלָדִֽים׃


\ssubsection{וארא}\\
וַיְדַבֵּ֥ר אֱלֹהִ֖ים אֶל־מֹשֶׁ֑ה וַיֹּ֥אמֶר אֵלָ֖יו אֲנִ֥י יְיָ׃ וָאֵרָ֗א אֶל־אַבְרָהָ֛ם אֶל־יִצְחָ֥ק וְאֶֽל־יַעֲקֹ֖ב בְּאֵ֣ל שַׁדָּ֑י וּשְׁמִ֣י יְיָ֔ לֹ֥א נוֹדַ֖עְתִּי לָהֶֽם׃ וְגַ֨ם הֲקִמֹ֤תִי אֶת־בְּרִיתִי֙ אִתָּ֔ם לָתֵ֥ת לָהֶ֖ם אֶת־אֶ֣רֶץ כְּנָ֑עַן אֵ֛ת אֶ֥רֶץ מְגֻרֵיהֶ֖ם אֲשֶׁר־גָּ֥רוּ בָֽהּ׃ וְגַ֣ם ׀ אֲנִ֣י שָׁמַ֗עְתִּי אֶֽת־נַאֲקַת֙ בְּנֵ֣י יִשְׂרָאֵ֔ל אֲשֶׁ֥ר מִצְרַ֖יִם מַעֲבִדִ֣ים אֹתָ֑ם וָאֶזְכֹּ֖ר אֶת־בְּרִיתִֽי׃
\aliyah{לוי}
לָכֵ֞ן אֱמֹ֥ר לִבְנֵֽי־יִשְׂרָאֵל֮ אֲנִ֣י יְיָ֒ וְהוֹצֵאתִ֣י אֶתְכֶ֗ם מִתַּ֙חַת֙ סִבְלֹ֣ת מִצְרַ֔יִם וְהִצַּלְתִּ֥י אֶתְכֶ֖ם מֵעֲבֹדָתָ֑ם וְגָאַלְתִּ֤י אֶתְכֶם֙ בִּזְר֣וֹעַ נְטוּיָ֔ה וּבִשְׁפָטִ֖ים גְּדֹלִֽים׃ וְלָקַחְתִּ֨י אֶתְכֶ֥ם לִי֙ לְעָ֔ם וְהָיִ֥יתִי לָכֶ֖ם לֵֽאלֹהִ֑ים וִֽידַעְתֶּ֗ם כִּ֣י אֲנִ֤י יְיָ֙ אֱלֹ֣הֵיכֶ֔ם הַמּוֹצִ֣יא אֶתְכֶ֔ם מִתַּ֖חַת סִבְל֥וֹת מִצְרָֽיִם׃ וְהֵבֵאתִ֤י אֶתְכֶם֙ אֶל־הָאָ֔רֶץ אֲשֶׁ֤ר נָשָׂ֙אתִי֙ אֶת־יָדִ֔י לָתֵ֣ת אֹתָ֔הּ לְאַבְרָהָ֥ם לְיִצְחָ֖ק וּֽלְיַעֲקֹ֑ב וְנָתַתִּ֨י אֹתָ֥הּ לָכֶ֛ם מוֹרָשָׁ֖ה אֲנִ֥י יְיָ׃ וַיְדַבֵּ֥ר מֹשֶׁ֛ה כֵּ֖ן אֶל־בְּנֵ֣י יִשְׂרָאֵ֑ל וְלֹ֤א שָֽׁמְעוּ֙ אֶל־מֹשֶׁ֔ה מִקֹּ֣צֶר ר֔וּחַ וּמֵעֲבֹדָ֖ה קָשָֽׁה׃ 
\aliyah{ישראל}
וַיְדַבֵּ֥ר יְיָ֖ אֶל־מֹשֶׁ֥ה לֵּאמֹֽר׃ בֹּ֣א דַבֵּ֔ר אֶל־פַּרְעֹ֖ה מֶ֣לֶךְ מִצְרָ֑יִם וִֽישַׁלַּ֥ח אֶת־בְּנֵֽי־יִשְׂרָאֵ֖ל מֵאַרְצֽוֹ׃ וַיְדַבֵּ֣ר מֹשֶׁ֔ה לִפְנֵ֥י יְיָ֖ לֵאמֹ֑ר הֵ֤ן בְּנֵֽי־יִשְׂרָאֵל֙ לֹֽא־שָׁמְע֣וּ אֵלַ֔י וְאֵיךְ֙ יִשְׁמָעֵ֣נִי פַרְעֹ֔ה וַאֲנִ֖י עֲרַ֥ל שְׂפָתָֽיִם׃ {פ} וַיְדַבֵּ֣ר יְיָ אֶל־מֹשֶׁ֣ה וְאֶֽל־אַהֲרֹן֒ וַיְצַוֵּם֙ אֶל־בְּנֵ֣י יִשְׂרָאֵ֔ל וְאֶל־פַּרְעֹ֖ה מֶ֣לֶךְ מִצְרָ֑יִם לְהוֹצִ֥יא אֶת־בְּנֵֽי־יִשְׂרָאֵ֖ל מֵאֶ֥רֶץ מִצְרָֽיִם׃ 


\ssubsection{בא}\\
וַיֹּ֤אמֶר יְיָ֙ אֶל־מֹשֶׁ֔ה בֹּ֖א אֶל־פַּרְעֹ֑ה כִּֽי־אֲנִ֞י הִכְבַּ֤דְתִּי אֶת־לִבּוֹ֙ וְאֶת־לֵ֣ב עֲבָדָ֔יו לְמַ֗עַן שִׁתִ֛י אֹתֹתַ֥י אֵ֖לֶּה בְּקִרְבּֽוֹ׃ וּלְמַ֡עַן תְּסַפֵּר֩ בְּאׇזְנֵ֨י בִנְךָ֜ וּבֶן־בִּנְךָ֗ אֵ֣ת אֲשֶׁ֤ר הִתְעַלַּ֙לְתִּי֙ בְּמִצְרַ֔יִם וְאֶת־אֹתֹתַ֖י אֲשֶׁר־שַׂ֣מְתִּי בָ֑ם וִֽידַעְתֶּ֖ם כִּי־אֲנִ֥י יְיָ׃ וַיָּבֹ֨א מֹשֶׁ֣ה וְאַהֲרֹן֮ אֶל־פַּרְעֹה֒ וַיֹּאמְר֣וּ אֵלָ֗יו כֹּֽה־אָמַ֤ר יְיָ֙ אֱלֹהֵ֣י הָֽעִבְרִ֔ים עַד־מָתַ֣י מֵאַ֔נְתָּ לֵעָנֹ֖ת מִפָּנָ֑י שַׁלַּ֥ח עַמִּ֖י וְיַֽעַבְדֻֽנִי׃
\aliyah{לוי}
כִּ֛י אִם־מָאֵ֥ן אַתָּ֖ה לְשַׁלֵּ֣חַ אֶת־עַמִּ֑י הִנְנִ֨י מֵבִ֥יא מָחָ֛ר אַרְבֶּ֖ה בִּגְבֻלֶֽךָ׃ וְכִסָּה֙ אֶת־עֵ֣ין הָאָ֔רֶץ וְלֹ֥א יוּכַ֖ל לִרְאֹ֣ת אֶת־הָאָ֑רֶץ וְאָכַ֣ל ׀ אֶת־יֶ֣תֶר הַפְּלֵטָ֗ה הַנִּשְׁאֶ֤רֶת לָכֶם֙ מִן־הַבָּרָ֔ד וְאָכַל֙ אֶת־כׇּל־הָעֵ֔ץ הַצֹּמֵ֥חַ לָכֶ֖ם מִן־הַשָּׂדֶֽה׃ וּמָלְא֨וּ בָתֶּ֜יךָ וּבָתֵּ֣י כׇל־עֲבָדֶ֘יךָ֮ וּבָתֵּ֣י כׇל־מִצְרַ֒יִם֒ אֲשֶׁ֨ר לֹֽא־רָא֤וּ אֲבֹתֶ֙יךָ֙ וַאֲב֣וֹת אֲבֹתֶ֔יךָ מִיּ֗וֹם הֱיוֹתָם֙ עַל־הָ֣אֲדָמָ֔ה עַ֖ד הַיּ֣וֹם הַזֶּ֑ה וַיִּ֥פֶן וַיֵּצֵ֖א מֵעִ֥ם פַּרְעֹֽה׃
\aliyah{ישראל}
וַיֹּאמְרוּ֩ עַבְדֵ֨י פַרְעֹ֜ה אֵלָ֗יו עַד־מָתַי֙ יִהְיֶ֨ה זֶ֥ה לָ֙נוּ֙ לְמוֹקֵ֔שׁ שַׁלַּח֙ אֶת־הָ֣אֲנָשִׁ֔ים וְיַֽעַבְד֖וּ אֶת־יְיָ֣ אֱלֹהֵיהֶ֑ם הֲטֶ֣רֶם תֵּדַ֔ע כִּ֥י אָבְדָ֖ה מִצְרָֽיִם׃ וַיּוּשַׁ֞ב אֶת־מֹשֶׁ֤ה וְאֶֽת־אַהֲרֹן֙ אֶל־פַּרְעֹ֔ה וַיֹּ֣אמֶר אֲלֵהֶ֔ם לְכ֥וּ עִבְד֖וּ אֶת־יְיָ֣ אֱלֹהֵיכֶ֑ם מִ֥י וָמִ֖י הַהֹלְכִֽים׃ וַיֹּ֣אמֶר מֹשֶׁ֔ה בִּנְעָרֵ֥ינוּ וּבִזְקֵנֵ֖ינוּ נֵלֵ֑ךְ בְּבָנֵ֨ינוּ וּבִבְנוֹתֵ֜נוּ בְּצֹאנֵ֤נוּ וּבִבְקָרֵ֙נוּ֙ נֵלֵ֔ךְ כִּ֥י חַג־יְיָ֖ לָֽנוּ׃ וַיֹּ֣אמֶר אֲלֵהֶ֗ם יְהִ֨י כֵ֤ן יְיָ֙ עִמָּכֶ֔ם כַּאֲשֶׁ֛ר אֲשַׁלַּ֥ח אֶתְכֶ֖ם וְאֶֽת־טַפְּכֶ֑ם רְא֕וּ כִּ֥י רָעָ֖ה נֶ֥גֶד פְּנֵיכֶֽם׃ לֹ֣א כֵ֗ן לְכֽוּ־נָ֤א הַגְּבָרִים֙ וְעִבְד֣וּ אֶת־יְיָ֔ כִּ֥י אֹתָ֖הּ אַתֶּ֣ם מְבַקְשִׁ֑ים וַיְגָ֣רֶשׁ אֹתָ֔ם מֵאֵ֖ת פְּנֵ֥י פַרְעֹֽה׃ 


\ssubsection{בשלח}\\
וַיְהִ֗י בְּשַׁלַּ֣ח פַּרְעֹה֮ אֶת־הָעָם֒ וְלֹא־נָחָ֣ם אֱלֹהִ֗ים דֶּ֚רֶךְ אֶ֣רֶץ פְּלִשְׁתִּ֔ים כִּ֥י קָר֖וֹב ה֑וּא כִּ֣י ׀ אָמַ֣ר אֱלֹהִ֗ים פֶּֽן־יִנָּחֵ֥ם הָעָ֛ם בִּרְאֹתָ֥ם מִלְחָמָ֖ה וְשָׁ֥בוּ מִצְרָֽיְמָה׃ וַיַּסֵּ֨ב אֱלֹהִ֧ים ׀ אֶת־הָעָ֛ם דֶּ֥רֶךְ הַמִּדְבָּ֖ר יַם־ס֑וּף וַחֲמֻשִׁ֛ים עָל֥וּ בְנֵי־יִשְׂרָאֵ֖ל מֵאֶ֥רֶץ מִצְרָֽיִם׃ וַיִּקַּ֥ח מֹשֶׁ֛ה אֶת־עַצְמ֥וֹת יוֹסֵ֖ף עִמּ֑וֹ כִּי֩ הַשְׁבֵּ֨עַ הִשְׁבִּ֜יעַ אֶת־בְּנֵ֤י יִשְׂרָאֵל֙ לֵאמֹ֔ר פָּקֹ֨ד יִפְקֹ֤ד אֱלֹהִים֙ אֶתְכֶ֔ם וְהַעֲלִיתֶ֧ם אֶת־עַצְמֹתַ֛י מִזֶּ֖ה אִתְּכֶֽם׃ וַיִּסְע֖וּ מִסֻּכֹּ֑ת וַיַּחֲנ֣וּ בְאֵתָ֔ם בִּקְצֵ֖ה הַמִּדְבָּֽר׃ וַֽייָ֡ הֹלֵךְ֩ לִפְנֵיהֶ֨ם יוֹמָ֜ם בְּעַמּ֤וּד עָנָן֙ לַנְחֹתָ֣ם הַדֶּ֔רֶךְ וְלַ֛יְלָה בְּעַמּ֥וּד אֵ֖שׁ לְהָאִ֣יר לָהֶ֑ם לָלֶ֖כֶת יוֹמָ֥ם וָלָֽיְלָה׃ לֹֽא־יָמִ֞ישׁ עַמּ֤וּד הֶֽעָנָן֙ יוֹמָ֔ם וְעַמּ֥וּד הָאֵ֖שׁ לָ֑יְלָה לִפְנֵ֖י הָעָֽם׃ 
\aliyah{לוי}
וַיְדַבֵּ֥ר יְיָ֖ אֶל־מֹשֶׁ֥ה לֵּאמֹֽר׃ דַּבֵּר֮ אֶל־בְּנֵ֣י יִשְׂרָאֵל֒ וְיָשֻׁ֗בוּ וְיַחֲנוּ֙ לִפְנֵי֙ פִּ֣י הַחִירֹ֔ת בֵּ֥ין מִגְדֹּ֖ל וּבֵ֣ין הַיָּ֑ם לִפְנֵי֙ בַּ֣עַל צְפֹ֔ן נִכְח֥וֹ תַחֲנ֖וּ עַל־הַיָּֽם׃ וְאָמַ֤ר פַּרְעֹה֙ לִבְנֵ֣י יִשְׂרָאֵ֔ל נְבֻכִ֥ים הֵ֖ם בָּאָ֑רֶץ סָגַ֥ר עֲלֵיהֶ֖ם הַמִּדְבָּֽר׃ וְחִזַּקְתִּ֣י אֶת־לֵב־פַּרְעֹה֮ וְרָדַ֣ף אַחֲרֵיהֶם֒ וְאִכָּבְדָ֤ה בְּפַרְעֹה֙ וּבְכׇל־חֵיל֔וֹ וְיָדְע֥וּ מִצְרַ֖יִם כִּֽי־אֲנִ֣י יְיָ֑ וַיַּֽעֲשׂוּ־כֵֽן׃
\aliyah{ישראל}
וַיֻּגַּד֙ לְמֶ֣לֶךְ מִצְרַ֔יִם כִּ֥י בָרַ֖ח הָעָ֑ם וַ֠יֵּהָפֵ֠ךְ לְבַ֨ב פַּרְעֹ֤ה וַעֲבָדָיו֙ אֶל־הָעָ֔ם וַיֹּֽאמְרוּ֙ מַה־זֹּ֣את עָשִׂ֔ינוּ כִּֽי־שִׁלַּ֥חְנוּ אֶת־יִשְׂרָאֵ֖ל מֵעָבְדֵֽנוּ׃ וַיֶּאְסֹ֖ר אֶת־רִכְבּ֑וֹ וְאֶת־עַמּ֖וֹ לָקַ֥ח עִמּֽוֹ׃ וַיִּקַּ֗ח שֵׁשׁ־מֵא֥וֹת רֶ֙כֶב֙ בָּח֔וּר וְכֹ֖ל רֶ֣כֶב מִצְרָ֑יִם וְשָׁלִשִׁ֖ם עַל־כֻּלּֽוֹ׃ וַיְחַזֵּ֣ק יְיָ֗ אֶת־לֵ֤ב פַּרְעֹה֙ מֶ֣לֶךְ מִצְרַ֔יִם וַיִּרְדֹּ֕ף אַחֲרֵ֖י בְּנֵ֣י יִשְׂרָאֵ֑ל וּבְנֵ֣י יִשְׂרָאֵ֔ל יֹצְאִ֖ים בְּיָ֥ד רָמָֽה׃


\ssubsection{יתרו}\\
וַיִּשְׁמַ֞ע יִתְר֨וֹ כֹהֵ֤ן מִדְיָן֙ חֹתֵ֣ן מֹשֶׁ֔ה אֵת֩ כׇּל־אֲשֶׁ֨ר עָשָׂ֤ה אֱלֹהִים֙ לְמֹשֶׁ֔ה וּלְיִשְׂרָאֵ֖ל עַמּ֑וֹ כִּֽי־הוֹצִ֧יא יְיָ֛ אֶת־יִשְׂרָאֵ֖ל מִמִּצְרָֽיִם׃ וַיִּקַּ֗ח יִתְרוֹ֙ חֹתֵ֣ן מֹשֶׁ֔ה אֶת־צִפֹּרָ֖ה אֵ֣שֶׁת מֹשֶׁ֑ה אַחַ֖ר שִׁלּוּחֶֽיהָ׃ וְאֵ֖ת שְׁנֵ֣י בָנֶ֑יהָ אֲשֶׁ֨ר שֵׁ֤ם הָֽאֶחָד֙ גֵּֽרְשֹׁ֔ם כִּ֣י אָמַ֔ר גֵּ֣ר הָיִ֔יתִי בְּאֶ֖רֶץ נׇכְרִיָּֽה׃ וְשֵׁ֥ם הָאֶחָ֖ד אֱלִיעֶ֑זֶר כִּֽי־אֱלֹהֵ֤י אָבִי֙ בְּעֶזְרִ֔י וַיַּצִּלֵ֖נִי מֵחֶ֥רֶב פַּרְעֹֽה׃
\aliyah{לוי}
וַיָּבֹ֞א יִתְר֨וֹ חֹתֵ֥ן מֹשֶׁ֛ה וּבָנָ֥יו וְאִשְׁתּ֖וֹ אֶל־מֹשֶׁ֑ה אֶל־הַמִּדְבָּ֗ר אֲשֶׁר־ה֛וּא חֹנֶ֥ה שָׁ֖ם הַ֥ר הָאֱלֹהִֽים׃ וַיֹּ֙אמֶר֙ אֶל־מֹשֶׁ֔ה אֲנִ֛י חֹתֶנְךָ֥ יִתְר֖וֹ בָּ֣א אֵלֶ֑יךָ וְאִ֨שְׁתְּךָ֔ וּשְׁנֵ֥י בָנֶ֖יהָ עִמָּֽהּ׃ וַיֵּצֵ֨א מֹשֶׁ֜ה לִקְרַ֣את חֹֽתְנ֗וֹ וַיִּשְׁתַּ֙חוּ֙ וַיִּשַּׁק־ל֔וֹ וַיִּשְׁאֲל֥וּ אִישׁ־לְרֵעֵ֖הוּ לְשָׁל֑וֹם וַיָּבֹ֖אוּ הָאֹֽהֱלָה׃ וַיְסַפֵּ֤ר מֹשֶׁה֙ לְחֹ֣תְנ֔וֹ אֵת֩ כׇּל־אֲשֶׁ֨ר עָשָׂ֤ה יְיָ֙ לְפַרְעֹ֣ה וּלְמִצְרַ֔יִם עַ֖ל אוֹדֹ֣ת יִשְׂרָאֵ֑ל אֵ֤ת כׇּל־הַתְּלָאָה֙ אֲשֶׁ֣ר מְצָאָ֣תַם בַּדֶּ֔רֶךְ וַיַּצִּלֵ֖ם יְיָ׃
\aliyah{ישראל}
וַיִּ֣חַדְּ יִתְר֔וֹ עַ֚ל כׇּל־הַטּוֹבָ֔ה אֲשֶׁר־עָשָׂ֥ה יְיָ֖ לְיִשְׂרָאֵ֑ל אֲשֶׁ֥ר הִצִּיל֖וֹ מִיַּ֥ד מִצְרָֽיִם׃ וַיֹּ֘אמֶר֮ יִתְרוֹ֒ בָּר֣וּךְ יְיָ֔ אֲשֶׁ֨ר הִצִּ֥יל אֶתְכֶ֛ם מִיַּ֥ד מִצְרַ֖יִם וּמִיַּ֣ד פַּרְעֹ֑ה אֲשֶׁ֤ר הִצִּיל֙ אֶת־הָעָ֔ם מִתַּ֖חַת יַד־מִצְרָֽיִם׃ עַתָּ֣ה יָדַ֔עְתִּי כִּֽי־גָד֥וֹל יְיָ֖ מִכׇּל־הָאֱלֹהִ֑ים כִּ֣י בַדָּבָ֔ר אֲשֶׁ֥ר זָד֖וּ עֲלֵיהֶֽם׃ וַיִּקַּ֞ח יִתְר֨וֹ חֹתֵ֥ן מֹשֶׁ֛ה עֹלָ֥ה וּזְבָחִ֖ים לֵֽאלֹהִ֑ים וַיָּבֹ֨א אַהֲרֹ֜ן וְכֹ֣ל ׀ זִקְנֵ֣י יִשְׂרָאֵ֗ל לֶאֱכׇל־לֶ֛חֶם עִם־חֹתֵ֥ן מֹשֶׁ֖ה לִפְנֵ֥י הָאֱלֹהִֽים׃


\ssubsection{משפטים}\\
וְאֵ֙לֶּה֙ הַמִּשְׁפָּטִ֔ים אֲשֶׁ֥ר תָּשִׂ֖ים לִפְנֵיהֶֽם׃ כִּ֤י תִקְנֶה֙ עֶ֣בֶד עִבְרִ֔י שֵׁ֥שׁ שָׁנִ֖ים יַעֲבֹ֑ד וּבַ֨שְּׁבִעִ֔ת יֵצֵ֥א לַֽחׇפְשִׁ֖י חִנָּֽם׃ אִם־בְּגַפּ֥וֹ יָבֹ֖א בְּגַפּ֣וֹ יֵצֵ֑א אִם־בַּ֤עַל אִשָּׁה֙ ה֔וּא וְיָצְאָ֥ה אִשְׁתּ֖וֹ עִמּֽוֹ׃ אִם־אֲדֹנָיו֙ יִתֶּן־ל֣וֹ אִשָּׁ֔ה וְיָלְדָה־ל֥וֹ בָנִ֖ים א֣וֹ בָנ֑וֹת הָאִשָּׁ֣ה וִילָדֶ֗יהָ תִּהְיֶה֙ לַֽאדֹנֶ֔יהָ וְה֖וּא יֵצֵ֥א בְגַפּֽוֹ׃ וְאִם־אָמֹ֤ר יֹאמַר֙ הָעֶ֔בֶד אָהַ֙בְתִּי֙ אֶת־אֲדֹנִ֔י אֶת־אִשְׁתִּ֖י וְאֶת־בָּנָ֑י לֹ֥א אֵצֵ֖א חׇפְשִֽׁי׃ וְהִגִּישׁ֤וֹ אֲדֹנָיו֙ אֶל־הָ֣אֱלֹהִ֔ים וְהִגִּישׁוֹ֙ אֶל־הַדֶּ֔לֶת א֖וֹ אֶל־הַמְּזוּזָ֑ה וְרָצַ֨ע אֲדֹנָ֤יו אֶת־אׇזְנוֹ֙ בַּמַּרְצֵ֔עַ וַעֲבָד֖וֹ לְעֹלָֽם׃ 
\aliyah{לוי}
וְכִֽי־יִמְכֹּ֥ר אִ֛ישׁ אֶת־בִּתּ֖וֹ לְאָמָ֑ה לֹ֥א תֵצֵ֖א כְּצֵ֥את הָעֲבָדִֽים׃ אִם־רָעָ֞ה בְּעֵינֵ֧י אֲדֹנֶ֛יהָ אֲשֶׁר־[ל֥וֹ] (לא) יְעָדָ֖הּ וְהֶפְדָּ֑הּ לְעַ֥ם נׇכְרִ֛י לֹא־יִמְשֹׁ֥ל לְמׇכְרָ֖הּ בְּבִגְדוֹ־בָֽהּ׃ וְאִם־לִבְנ֖וֹ יִֽיעָדֶ֑נָּה כְּמִשְׁפַּ֥ט הַבָּנ֖וֹת יַעֲשֶׂה־לָּֽהּ׃ אִם־אַחֶ֖רֶת יִֽקַּֽח־ל֑וֹ שְׁאֵרָ֛הּ כְּסוּתָ֥הּ וְעֹנָתָ֖הּ לֹ֥א יִגְרָֽע׃ וְאִ֨ם־שְׁלׇשׁ־אֵ֔לֶּה לֹ֥א יַעֲשֶׂ֖ה לָ֑הּ וְיָצְאָ֥ה חִנָּ֖ם אֵ֥ין כָּֽסֶף׃ 
\aliyah{ישראל}
מַכֵּ֥ה אִ֛ישׁ וָמֵ֖ת מ֥וֹת יוּמָֽת׃ וַאֲשֶׁר֙ לֹ֣א צָדָ֔ה וְהָאֱלֹהִ֖ים אִנָּ֣ה לְיָד֑וֹ וְשַׂמְתִּ֤י לְךָ֙ מָק֔וֹם אֲשֶׁ֥ר יָנ֖וּס שָֽׁמָּה׃ {ס} וְכִֽי־יָזִ֥ד אִ֛ישׁ עַל־רֵעֵ֖הוּ לְהׇרְג֣וֹ בְעׇרְמָ֑ה מֵעִ֣ם מִזְבְּחִ֔י תִּקָּחֶ֖נּוּ לָמֽוּת׃ {ס} וּמַכֵּ֥ה אָבִ֛יו וְאִמּ֖וֹ מ֥וֹת יוּמָֽת׃ {ס} וְגֹנֵ֨ב אִ֧ישׁ וּמְכָר֛וֹ וְנִמְצָ֥א בְיָד֖וֹ מ֥וֹת יוּמָֽת׃ {ס} וּמְקַלֵּ֥ל אָבִ֛יו וְאִמּ֖וֹ מ֥וֹת יוּמָֽת׃ {ס} וְכִֽי־יְרִיבֻ֣ן אֲנָשִׁ֔ים וְהִכָּה־אִישׁ֙ אֶת־רֵעֵ֔הוּ בְּאֶ֖בֶן א֣וֹ בְאֶגְרֹ֑ף וְלֹ֥א יָמ֖וּת וְנָפַ֥ל לְמִשְׁכָּֽב׃ אִם־יָק֞וּם וְהִתְהַלֵּ֥ךְ בַּח֛וּץ עַל־מִשְׁעַנְתּ֖וֹ וְנִקָּ֣ה הַמַּכֶּ֑ה רַ֥ק שִׁבְתּ֛וֹ יִתֵּ֖ן וְרַפֹּ֥א יְרַפֵּֽא׃ 


\ssubsection{תרומה}\\
וַיְדַבֵּ֥ר יְיָ֖ אֶל־מֹשֶׁ֥ה לֵּאמֹֽר׃ דַּבֵּר֙ אֶל־בְּנֵ֣י יִשְׂרָאֵ֔ל וְיִקְחוּ־לִ֖י תְּרוּמָ֑ה מֵאֵ֤ת כׇּל־אִישׁ֙ אֲשֶׁ֣ר יִדְּבֶ֣נּוּ לִבּ֔וֹ תִּקְח֖וּ אֶת־תְּרוּמָתִֽי׃ וְזֹאת֙ הַתְּרוּמָ֔ה אֲשֶׁ֥ר תִּקְח֖וּ מֵאִתָּ֑ם זָהָ֥ב וָכֶ֖סֶף וּנְחֹֽשֶׁת׃ וּתְכֵ֧לֶת וְאַרְגָּמָ֛ן וְתוֹלַ֥עַת שָׁנִ֖י וְשֵׁ֥שׁ וְעִזִּֽים׃ וְעֹרֹ֨ת אֵילִ֧ם מְאׇדָּמִ֛ים וְעֹרֹ֥ת תְּחָשִׁ֖ים וַעֲצֵ֥י שִׁטִּֽים׃
\aliyah{לוי}
שֶׁ֖מֶן לַמָּאֹ֑ר בְּשָׂמִים֙ לְשֶׁ֣מֶן הַמִּשְׁחָ֔ה וְלִקְטֹ֖רֶת הַסַּמִּֽים׃ אַבְנֵי־שֹׁ֕הַם וְאַבְנֵ֖י מִלֻּאִ֑ים לָאֵפֹ֖ד וְלַחֹֽשֶׁן׃ וְעָ֥שׂוּ לִ֖י מִקְדָּ֑שׁ וְשָׁכַנְתִּ֖י בְּתוֹכָֽם׃ כְּכֹ֗ל אֲשֶׁ֤ר אֲנִי֙ מַרְאֶ֣ה אוֹתְךָ֔ אֵ֚ת תַּבְנִ֣ית הַמִּשְׁכָּ֔ן וְאֵ֖ת תַּבְנִ֣ית כׇּל־כֵּלָ֑יו וְכֵ֖ן תַּעֲשֽׂוּ׃ 
\aliyah{ישראל}
וְעָשׂ֥וּ אֲר֖וֹן עֲצֵ֣י שִׁטִּ֑ים אַמָּתַ֨יִם וָחֵ֜צִי אָרְכּ֗וֹ וְאַמָּ֤ה וָחֵ֙צִי֙ רׇחְבּ֔וֹ וְאַמָּ֥ה וָחֵ֖צִי קֹמָתֽוֹ׃ וְצִפִּיתָ֤ אֹתוֹ֙ זָהָ֣ב טָה֔וֹר מִבַּ֥יִת וּמִח֖וּץ תְּצַפֶּ֑נּוּ וְעָשִׂ֧יתָ עָלָ֛יו זֵ֥ר זָהָ֖ב סָבִֽיב׃ וְיָצַ֣קְתָּ לּ֗וֹ אַרְבַּע֙ טַבְּעֹ֣ת זָהָ֔ב וְנָ֣תַתָּ֔ה עַ֖ל אַרְבַּ֣ע פַּעֲמֹתָ֑יו וּשְׁתֵּ֣י טַבָּעֹ֗ת עַל־צַלְעוֹ֙ הָֽאֶחָ֔ת וּשְׁתֵּי֙ טַבָּעֹ֔ת עַל־צַלְע֖וֹ הַשֵּׁנִֽית׃ וְעָשִׂ֥יתָ בַדֵּ֖י עֲצֵ֣י שִׁטִּ֑ים וְצִפִּיתָ֥ אֹתָ֖ם זָהָֽב׃ וְהֵֽבֵאתָ֤ אֶת־הַבַּדִּים֙ בַּטַּבָּעֹ֔ת עַ֖ל צַלְעֹ֣ת הָאָרֹ֑ן לָשֵׂ֥את אֶת־הָאָרֹ֖ן בָּהֶֽם׃ בְּטַבְּעֹת֙ הָאָרֹ֔ן יִהְי֖וּ הַבַּדִּ֑ים לֹ֥א יָסֻ֖רוּ מִמֶּֽנּוּ׃ וְנָתַתָּ֖ אֶל־הָאָרֹ֑ן אֵ֚ת הָעֵדֻ֔ת אֲשֶׁ֥ר אֶתֵּ֖ן אֵלֶֽיךָ׃


\ssubsection{תצוה}\\
וְאַתָּ֞ה תְּצַוֶּ֣ה ׀ אֶת־בְּנֵ֣י יִשְׂרָאֵ֗ל וְיִקְח֨וּ אֵלֶ֜יךָ שֶׁ֣מֶן זַ֥יִת זָ֛ךְ כָּתִ֖ית לַמָּא֑וֹר לְהַעֲלֹ֥ת נֵ֖ר תָּמִֽיד׃ בְּאֹ֣הֶל מוֹעֵד֩ מִח֨וּץ לַפָּרֹ֜כֶת אֲשֶׁ֣ר עַל־הָעֵדֻ֗ת יַעֲרֹךְ֩ אֹת֨וֹ אַהֲרֹ֧ן וּבָנָ֛יו מֵעֶ֥רֶב עַד־בֹּ֖קֶר לִפְנֵ֣י יְיָ֑ חֻקַּ֤ת עוֹלָם֙ לְדֹ֣רֹתָ֔ם מֵאֵ֖ת בְּנֵ֥י יִשְׂרָאֵֽל׃ {ס} וְאַתָּ֡ה הַקְרֵ֣ב אֵלֶ֩יךָ֩ אֶת־אַהֲרֹ֨ן אָחִ֜יךָ וְאֶת־בָּנָ֣יו אִתּ֗וֹ מִתּ֛וֹךְ בְּנֵ֥י יִשְׂרָאֵ֖ל לְכַהֲנוֹ־לִ֑י אַהֲרֹ֕ן נָדָ֧ב וַאֲבִיה֛וּא אֶלְעָזָ֥ר וְאִיתָמָ֖ר בְּנֵ֥י אַהֲרֹֽן׃ וְעָשִׂ֥יתָ בִגְדֵי־קֹ֖דֶשׁ לְאַהֲרֹ֣ן אָחִ֑יךָ לְכָב֖וֹד וּלְתִפְאָֽרֶת׃ וְאַתָּ֗ה תְּדַבֵּר֙ אֶל־כׇּל־חַכְמֵי־לֵ֔ב אֲשֶׁ֥ר מִלֵּאתִ֖יו ר֣וּחַ חׇכְמָ֑ה וְעָשׂ֞וּ אֶת־בִּגְדֵ֧י אַהֲרֹ֛ן לְקַדְּשׁ֖וֹ לְכַהֲנוֹ־לִֽי׃ וְאֵ֨לֶּה הַבְּגָדִ֜ים אֲשֶׁ֣ר יַעֲשׂ֗וּ חֹ֤שֶׁן וְאֵפוֹד֙ וּמְעִ֔יל וּכְתֹ֥נֶת תַּשְׁבֵּ֖ץ מִצְנֶ֣פֶת וְאַבְנֵ֑ט וְעָשׂ֨וּ בִגְדֵי־קֹ֜דֶשׁ לְאַהֲרֹ֥ן אָחִ֛יךָ וּלְבָנָ֖יו לְכַהֲנוֹ־לִֽי׃ וְהֵם֙ יִקְח֣וּ אֶת־הַזָּהָ֔ב וְאֶת־הַתְּכֵ֖לֶת וְאֶת־הָֽאַרְגָּמָ֑ן וְאֶת־תּוֹלַ֥עַת הַשָּׁנִ֖י וְאֶת־הַשֵּֽׁשׁ׃ 
\aliyah{לוי}
וְעָשׂ֖וּ אֶת־הָאֵפֹ֑ד זָ֠הָ֠ב תְּכֵ֨לֶת וְאַרְגָּמָ֜ן תּוֹלַ֧עַת שָׁנִ֛י וְשֵׁ֥שׁ מׇשְׁזָ֖ר מַעֲשֵׂ֥ה חֹשֵֽׁב׃ שְׁתֵּ֧י כְתֵפֹ֣ת חֹֽבְרֹ֗ת יִֽהְיֶה־לּ֛וֹ אֶל־שְׁנֵ֥י קְצוֹתָ֖יו וְחֻבָּֽר׃ וְחֵ֤שֶׁב אֲפֻדָּתוֹ֙ אֲשֶׁ֣ר עָלָ֔יו כְּמַעֲשֵׂ֖הוּ מִמֶּ֣נּוּ יִהְיֶ֑ה זָהָ֗ב תְּכֵ֧לֶת וְאַרְגָּמָ֛ן וְתוֹלַ֥עַת שָׁנִ֖י וְשֵׁ֥שׁ מׇשְׁזָֽר׃ וְלָ֣קַחְתָּ֔ אֶת־שְׁתֵּ֖י אַבְנֵי־שֹׁ֑הַם וּפִתַּחְתָּ֣ עֲלֵיהֶ֔ם שְׁמ֖וֹת בְּנֵ֥י יִשְׂרָאֵֽל׃
\aliyah{ישראל}
שִׁשָּׁה֙ מִשְּׁמֹתָ֔ם עַ֖ל הָאֶ֣בֶן הָאֶחָ֑ת וְאֶת־שְׁמ֞וֹת הַשִּׁשָּׁ֧ה הַנּוֹתָרִ֛ים עַל־הָאֶ֥בֶן הַשֵּׁנִ֖ית כְּתוֹלְדֹתָֽם׃ מַעֲשֵׂ֣ה חָרַשׁ֮ אֶ֒בֶן֒ פִּתּוּחֵ֣י חֹתָ֗ם תְּפַתַּח֙ אֶת־שְׁתֵּ֣י הָאֲבָנִ֔ים עַל־שְׁמֹ֖ת בְּנֵ֣י יִשְׂרָאֵ֑ל מֻסַבֹּ֛ת מִשְׁבְּצ֥וֹת זָהָ֖ב תַּעֲשֶׂ֥ה אֹתָֽם׃ וְשַׂמְתָּ֞ אֶת־שְׁתֵּ֣י הָאֲבָנִ֗ים עַ֚ל כִּתְפֹ֣ת הָֽאֵפֹ֔ד אַבְנֵ֥י זִכָּרֹ֖ן לִבְנֵ֣י יִשְׂרָאֵ֑ל וְנָשָׂא֩ אַהֲרֹ֨ן אֶת־שְׁמוֹתָ֜ם לִפְנֵ֧י יְיָ֛ עַל־שְׁתֵּ֥י כְתֵפָ֖יו לְזִכָּרֹֽן׃ 


\ssubsection{כי תשא}\\
וַיְדַבֵּ֥ר יְיָ֖ אֶל־מֹשֶׁ֥ה לֵּאמֹֽר׃ כִּ֣י תִשָּׂ֞א אֶת־רֹ֥אשׁ בְּנֵֽי־יִשְׂרָאֵל֮ לִפְקֻדֵיהֶם֒ וְנָ֨תְנ֜וּ אִ֣ישׁ כֹּ֧פֶר נַפְשׁ֛וֹ לַייָ֖ בִּפְקֹ֣ד אֹתָ֑ם וְלֹא־יִהְיֶ֥ה בָהֶ֛ם נֶ֖גֶף בִּפְקֹ֥ד אֹתָֽם׃ זֶ֣ה ׀ יִתְּנ֗וּ כׇּל־הָעֹבֵר֙ עַל־הַפְּקֻדִ֔ים מַחֲצִ֥ית הַשֶּׁ֖קֶל בְּשֶׁ֣קֶל הַקֹּ֑דֶשׁ עֶשְׂרִ֤ים גֵּרָה֙ הַשֶּׁ֔קֶל מַחֲצִ֣ית הַשֶּׁ֔קֶל תְּרוּמָ֖ה לַֽייָ׃
\aliyah{לוי}
כֹּ֗ל הָעֹבֵר֙ עַל־הַפְּקֻדִ֔ים מִבֶּ֛ן עֶשְׂרִ֥ים שָׁנָ֖ה וָמָ֑עְלָה יִתֵּ֖ן תְּרוּמַ֥ת יְיָ׃ הֶֽעָשִׁ֣יר לֹֽא־יַרְבֶּ֗ה וְהַדַּל֙ לֹ֣א יַמְעִ֔יט מִֽמַּחֲצִ֖ית הַשָּׁ֑קֶל לָתֵת֙ אֶת־תְּרוּמַ֣ת יְיָ֔ לְכַפֵּ֖ר עַל־נַפְשֹׁתֵיכֶֽם׃ וְלָקַחְתָּ֞ אֶת־כֶּ֣סֶף הַכִּפֻּרִ֗ים מֵאֵת֙ בְּנֵ֣י יִשְׂרָאֵ֔ל וְנָתַתָּ֣ אֹת֔וֹ עַל־עֲבֹדַ֖ת אֹ֣הֶל מוֹעֵ֑ד וְהָיָה֩ לִבְנֵ֨י יִשְׂרָאֵ֤ל לְזִכָּרוֹן֙ לִפְנֵ֣י יְיָ֔ לְכַפֵּ֖ר עַל־נַפְשֹׁתֵיכֶֽם׃ 
\aliyah{ישראל}
וַיְדַבֵּ֥ר יְיָ֖ אֶל־מֹשֶׁ֥ה לֵּאמֹֽר׃ וְעָשִׂ֜יתָ כִּיּ֥וֹר נְחֹ֛שֶׁת וְכַנּ֥וֹ נְחֹ֖שֶׁת לְרׇחְצָ֑ה וְנָתַתָּ֣ אֹת֗וֹ בֵּֽין־אֹ֤הֶל מוֹעֵד֙ וּבֵ֣ין הַמִּזְבֵּ֔חַ וְנָתַתָּ֥ שָׁ֖מָּה מָֽיִם׃ וְרָחֲצ֛וּ אַהֲרֹ֥ן וּבָנָ֖יו מִמֶּ֑נּוּ אֶת־יְדֵיהֶ֖ם וְאֶת־רַגְלֵיהֶֽם׃ בְּבֹאָ֞ם אֶל־אֹ֧הֶל מוֹעֵ֛ד יִרְחֲצוּ־מַ֖יִם וְלֹ֣א יָמֻ֑תוּ א֣וֹ בְגִשְׁתָּ֤ם אֶל־הַמִּזְבֵּ֙חַ֙ לְשָׁרֵ֔ת לְהַקְטִ֥יר אִשֶּׁ֖ה לַֽייָ׃ וְרָחֲצ֛וּ יְדֵיהֶ֥ם וְרַגְלֵיהֶ֖ם וְלֹ֣א יָמֻ֑תוּ וְהָיְתָ֨ה לָהֶ֧ם חׇק־עוֹלָ֛ם ל֥וֹ וּלְזַרְע֖וֹ לְדֹרֹתָֽם׃ 


\ssubsection{ויקהל}\\
וַיַּקְהֵ֣ל מֹשֶׁ֗ה אֶֽת־כׇּל־עֲדַ֛ת בְּנֵ֥י יִשְׂרָאֵ֖ל וַיֹּ֣אמֶר אֲלֵהֶ֑ם אֵ֚לֶּה הַדְּבָרִ֔ים אֲשֶׁר־צִוָּ֥ה יְיָ֖ לַעֲשֹׂ֥ת אֹתָֽם׃ שֵׁ֣שֶׁת יָמִים֮ תֵּעָשֶׂ֣ה מְלָאכָה֒ וּבַיּ֣וֹם הַשְּׁבִיעִ֗י יִהְיֶ֨ה לָכֶ֥ם קֹ֛דֶשׁ שַׁבַּ֥ת שַׁבָּת֖וֹן לַייָ֑ כׇּל־הָעֹשֶׂ֥ה ב֛וֹ מְלָאכָ֖ה יוּמָֽת׃ לֹא־תְבַעֲר֣וּ אֵ֔שׁ בְּכֹ֖ל מֹשְׁבֹֽתֵיכֶ֑ם בְּי֖וֹם הַשַּׁבָּֽת׃ 
\aliyah{לוי}
וַיֹּ֣אמֶר מֹשֶׁ֔ה אֶל־כׇּל־עֲדַ֥ת בְּנֵֽי־יִשְׂרָאֵ֖ל לֵאמֹ֑ר זֶ֣ה הַדָּבָ֔ר אֲשֶׁר־צִוָּ֥ה יְיָ֖ לֵאמֹֽר׃ קְח֨וּ מֵֽאִתְּכֶ֤ם תְּרוּמָה֙ לַֽייָ֔ כֹּ֚ל נְדִ֣יב לִבּ֔וֹ יְבִיאֶ֕הָ אֵ֖ת תְּרוּמַ֣ת יְיָ֑ זָהָ֥ב וָכֶ֖סֶף וּנְחֹֽשֶׁת׃ וּתְכֵ֧לֶת וְאַרְגָּמָ֛ן וְתוֹלַ֥עַת שָׁנִ֖י וְשֵׁ֥שׁ וְעִזִּֽים׃ וְעֹרֹ֨ת אֵילִ֧ם מְאׇדָּמִ֛ים וְעֹרֹ֥ת תְּחָשִׁ֖ים וַעֲצֵ֥י שִׁטִּֽים׃ וְשֶׁ֖מֶן לַמָּא֑וֹר וּבְשָׂמִים֙ לְשֶׁ֣מֶן הַמִּשְׁחָ֔ה וְלִקְטֹ֖רֶת הַסַּמִּֽים׃ וְאַ֨בְנֵי־שֹׁ֔הַם וְאַבְנֵ֖י מִלֻּאִ֑ים לָאֵפ֖וֹד וְלַחֹֽשֶׁן׃ וְכׇל־חֲכַם־לֵ֖ב בָּכֶ֑ם יָבֹ֣אוּ וְיַעֲשׂ֔וּ אֵ֛ת כׇּל־אֲשֶׁ֥ר צִוָּ֖ה יְיָ׃
\aliyah{ישראל}
אֶ֨ת־הַמִּשְׁכָּ֔ן אֶֽת־אׇהֳל֖וֹ וְאֶת־מִכְסֵ֑הוּ אֶת־קְרָסָיו֙ וְאֶת־קְרָשָׁ֔יו אֶת־בְּרִיחָ֕ו אֶת־עַמֻּדָ֖יו וְאֶת־אֲדָנָֽיו׃ אֶת־הָאָרֹ֥ן וְאֶת־בַּדָּ֖יו אֶת־הַכַּפֹּ֑רֶת וְאֵ֖ת פָּרֹ֥כֶת הַמָּסָֽךְ׃ אֶת־הַשֻּׁלְחָ֥ן וְאֶת־בַּדָּ֖יו וְאֶת־כׇּל־כֵּלָ֑יו וְאֵ֖ת לֶ֥חֶם הַפָּנִֽים׃ וְאֶת־מְנֹרַ֧ת הַמָּא֛וֹר וְאֶת־כֵּלֶ֖יהָ וְאֶת־נֵרֹתֶ֑יהָ וְאֵ֖ת שֶׁ֥מֶן הַמָּאֽוֹר׃ וְאֶת־מִזְבַּ֤ח הַקְּטֹ֙רֶת֙ וְאֶת־בַּדָּ֔יו וְאֵת֙ שֶׁ֣מֶן הַמִּשְׁחָ֔ה וְאֵ֖ת קְטֹ֣רֶת הַסַּמִּ֑ים וְאֶת־מָסַ֥ךְ הַפֶּ֖תַח לְפֶ֥תַח הַמִּשְׁכָּֽן׃ אֵ֣ת ׀ מִזְבַּ֣ח הָעֹלָ֗ה וְאֶת־מִכְבַּ֤ר הַנְּחֹ֙שֶׁת֙ אֲשֶׁר־ל֔וֹ אֶת־בַּדָּ֖יו וְאֶת־כׇּל־כֵּלָ֑יו אֶת־הַכִּיֹּ֖ר וְאֶת־כַּנּֽוֹ׃ אֵ֚ת קַלְעֵ֣י הֶחָצֵ֔ר אֶת־עַמֻּדָ֖יו וְאֶת־אֲדָנֶ֑יהָ וְאֵ֕ת מָסַ֖ךְ שַׁ֥עַר הֶחָצֵֽר׃ אֶת־יִתְדֹ֧ת הַמִּשְׁכָּ֛ן וְאֶת־יִתְדֹ֥ת הֶחָצֵ֖ר וְאֶת־מֵיתְרֵיהֶֽם׃ אֶת־בִּגְדֵ֥י הַשְּׂרָ֖ד לְשָׁרֵ֣ת בַּקֹּ֑דֶשׁ אֶת־בִּגְדֵ֤י הַקֹּ֙דֶשׁ֙ לְאַהֲרֹ֣ן הַכֹּהֵ֔ן וְאֶת־בִּגְדֵ֥י בָנָ֖יו לְכַהֵֽן׃ וַיֵּ֥צְא֛וּ כׇּל־עֲדַ֥ת בְּנֵֽי־יִשְׂרָאֵ֖ל מִלִּפְנֵ֥י מֹשֶֽׁה׃


\ssubsection{פקודי}\\
אֵ֣לֶּה פְקוּדֵ֤י הַמִּשְׁכָּן֙ מִשְׁכַּ֣ן הָעֵדֻ֔ת אֲשֶׁ֥ר פֻּקַּ֖ד עַל־פִּ֣י מֹשֶׁ֑ה עֲבֹדַת֙ הַלְוִיִּ֔ם בְּיַד֙ אִֽיתָמָ֔ר בֶּֽן־אַהֲרֹ֖ן הַכֹּהֵֽן׃ וּבְצַלְאֵ֛ל בֶּן־אוּרִ֥י בֶן־ח֖וּר לְמַטֵּ֣ה יְהוּדָ֑ה עָשָׂ֕ה אֵ֛ת כׇּל־אֲשֶׁר־צִוָּ֥ה יְיָ֖ אֶת־מֹשֶֽׁה׃ וְאִתּ֗וֹ אׇהֳלִיאָ֞ב בֶּן־אֲחִיסָמָ֛ךְ לְמַטֵּה־דָ֖ן חָרָ֣שׁ וְחֹשֵׁ֑ב וְרֹקֵ֗ם בַּתְּכֵ֙לֶת֙ וּבָֽאַרְגָּמָ֔ן וּבְתוֹלַ֥עַת הַשָּׁנִ֖י וּבַשֵּֽׁשׁ׃ 
\aliyah{לוי}
כׇּל־הַזָּהָ֗ב הֶֽעָשׂוּי֙ לַמְּלָאכָ֔ה בְּכֹ֖ל מְלֶ֣אכֶת הַקֹּ֑דֶשׁ וַיְהִ֣י ׀ זְהַ֣ב הַתְּנוּפָ֗ה תֵּ֤שַׁע וְעֶשְׂרִים֙ כִּכָּ֔ר וּשְׁבַ֨ע מֵא֧וֹת וּשְׁלֹשִׁ֛ים שֶׁ֖קֶל בְּשֶׁ֥קֶל הַקֹּֽדֶשׁ׃ וְכֶ֛סֶף פְּקוּדֵ֥י הָעֵדָ֖ה מְאַ֣ת כִּכָּ֑ר וְאֶ֩לֶף֩ וּשְׁבַ֨ע מֵא֜וֹת וַחֲמִשָּׁ֧ה וְשִׁבְעִ֛ים שֶׁ֖קֶל בְּשֶׁ֥קֶל הַקֹּֽדֶשׁ׃ בֶּ֚קַע לַגֻּלְגֹּ֔לֶת מַחֲצִ֥ית הַשֶּׁ֖קֶל בְּשֶׁ֣קֶל הַקֹּ֑דֶשׁ לְכֹ֨ל הָעֹבֵ֜ר עַל־הַפְּקֻדִ֗ים מִבֶּ֨ן עֶשְׂרִ֤ים שָׁנָה֙ וָמַ֔עְלָה לְשֵׁשׁ־מֵא֥וֹת אֶ֙לֶף֙ וּשְׁלֹ֣שֶׁת אֲלָפִ֔ים וַחֲמֵ֥שׁ מֵא֖וֹת וַחֲמִשִּֽׁים׃ וַיְהִ֗י מְאַת֙ כִּכַּ֣ר הַכֶּ֔סֶף לָצֶ֗קֶת אֵ֚ת אַדְנֵ֣י הַקֹּ֔דֶשׁ וְאֵ֖ת אַדְנֵ֣י הַפָּרֹ֑כֶת מְאַ֧ת אֲדָנִ֛ים לִמְאַ֥ת הַכִּכָּ֖ר כִּכָּ֥ר לָאָֽדֶן׃
\aliyah{ישראל}
וְאֶת־הָאֶ֜לֶף וּשְׁבַ֤ע הַמֵּאוֹת֙ וַחֲמִשָּׁ֣ה וְשִׁבְעִ֔ים עָשָׂ֥ה וָוִ֖ים לָעַמּוּדִ֑ים וְצִפָּ֥ה רָאשֵׁיהֶ֖ם וְחִשַּׁ֥ק אֹתָֽם׃ וּנְחֹ֥שֶׁת הַתְּנוּפָ֖ה שִׁבְעִ֣ים כִּכָּ֑ר וְאַלְפַּ֥יִם וְאַרְבַּע־מֵא֖וֹת שָֽׁקֶל׃ וַיַּ֣עַשׂ בָּ֗הּ אֶת־אַדְנֵי֙ פֶּ֚תַח אֹ֣הֶל מוֹעֵ֔ד וְאֵת֙ מִזְבַּ֣ח הַנְּחֹ֔שֶׁת וְאֶת־מִכְבַּ֥ר הַנְּחֹ֖שֶׁת אֲשֶׁר־ל֑וֹ וְאֵ֖ת כׇּל־כְּלֵ֥י הַמִּזְבֵּֽחַ׃ וְאֶת־אַדְנֵ֤י הֶֽחָצֵר֙ סָבִ֔יב וְאֶת־אַדְנֵ֖י שַׁ֣עַר הֶחָצֵ֑ר וְאֵ֨ת כׇּל־יִתְדֹ֧ת הַמִּשְׁכָּ֛ן וְאֶת־כׇּל־יִתְדֹ֥ת הֶחָצֵ֖ר סָבִֽיב׃ וּמִן־הַתְּכֵ֤לֶת וְהָֽאַרְגָּמָן֙ וְתוֹלַ֣עַת הַשָּׁנִ֔י עָשׂ֥וּ בִגְדֵי־שְׂרָ֖ד לְשָׁרֵ֣ת בַּקֹּ֑דֶשׁ וַֽיַּעֲשׂ֞וּ אֶת־בִּגְדֵ֤י הַקֹּ֙דֶשׁ֙ אֲשֶׁ֣ר לְאַהֲרֹ֔ן כַּאֲשֶׁ֛ר צִוָּ֥ה יְיָ֖ אֶת־מֹשֶֽׁה׃ 


\ssubsection{ויקרא}\\
וַיִּקְרָ֖א אֶל־מֹשֶׁ֑ה וַיְדַבֵּ֤ר יְיָ֙ אֵלָ֔יו מֵאֹ֥הֶל מוֹעֵ֖ד לֵאמֹֽר׃ דַּבֵּ֞ר אֶל־בְּנֵ֤י יִשְׂרָאֵל֙ וְאָמַרְתָּ֣ אֲלֵהֶ֔ם אָדָ֗ם כִּֽי־יַקְרִ֥יב מִכֶּ֛ם קׇרְבָּ֖ן לַֽייָ֑ מִן־הַבְּהֵמָ֗ה מִן־הַבָּקָר֙ וּמִן־הַצֹּ֔אן תַּקְרִ֖יבוּ אֶת־קׇרְבַּנְכֶֽם׃ אִם־עֹלָ֤ה קׇרְבָּנוֹ֙ מִן־הַבָּקָ֔ר זָכָ֥ר תָּמִ֖ים יַקְרִיבֶ֑נּוּ אֶל־פֶּ֜תַח אֹ֤הֶל מוֹעֵד֙ יַקְרִ֣יב אֹת֔וֹ לִרְצֹנ֖וֹ לִפְנֵ֥י יְיָ׃ וְסָמַ֣ךְ יָד֔וֹ עַ֖ל רֹ֣אשׁ הָעֹלָ֑ה וְנִרְצָ֥ה ל֖וֹ לְכַפֵּ֥ר עָלָֽיו׃
\aliyah{לוי}
וְשָׁחַ֛ט אֶת־בֶּ֥ן הַבָּקָ֖ר לִפְנֵ֣י יְיָ֑ וְ֠הִקְרִ֠יבוּ בְּנֵ֨י אַהֲרֹ֤ן הַכֹּֽהֲנִים֙ אֶת־הַדָּ֔ם וְזָרְק֨וּ אֶת־הַדָּ֤ם עַל־הַמִּזְבֵּ֙חַ֙ סָבִ֔יב אֲשֶׁר־פֶּ֖תַח אֹ֥הֶל מוֹעֵֽד׃ וְהִפְשִׁ֖יט אֶת־הָעֹלָ֑ה וְנִתַּ֥ח אֹתָ֖הּ לִנְתָחֶֽיהָ׃ וְ֠נָתְנ֠וּ בְּנֵ֨י אַהֲרֹ֧ן הַכֹּהֵ֛ן אֵ֖שׁ עַל־הַמִּזְבֵּ֑חַ וְעָרְכ֥וּ עֵצִ֖ים עַל־הָאֵֽשׁ׃ וְעָרְכ֗וּ בְּנֵ֤י אַהֲרֹן֙ הַכֹּ֣הֲנִ֔ים אֵ֚ת הַנְּתָחִ֔ים אֶת־הָרֹ֖אשׁ וְאֶת־הַפָּ֑דֶר עַל־הָעֵצִים֙ אֲשֶׁ֣ר עַל־הָאֵ֔שׁ אֲשֶׁ֖ר עַל־הַמִּזְבֵּֽחַ׃ וְקִרְבּ֥וֹ וּכְרָעָ֖יו יִרְחַ֣ץ בַּמָּ֑יִם וְהִקְטִ֨יר הַכֹּהֵ֤ן אֶת־הַכֹּל֙ הַמִּזְבֵּ֔חָה עֹלָ֛ה אִשֵּׁ֥ה רֵֽיחַ־נִיח֖וֹחַ לַֽייָ׃ 
\aliyah{ישראל}
וְאִם־מִן־הַצֹּ֨אן קׇרְבָּנ֧וֹ מִן־הַכְּשָׂבִ֛ים א֥וֹ מִן־הָעִזִּ֖ים לְעֹלָ֑ה זָכָ֥ר תָּמִ֖ים יַקְרִיבֶֽנּוּ׃ וְשָׁחַ֨ט אֹת֜וֹ עַ֣ל יֶ֧רֶךְ הַמִּזְבֵּ֛חַ צָפֹ֖נָה לִפְנֵ֣י יְיָ֑ וְזָרְק֡וּ בְּנֵי֩ אַהֲרֹ֨ן הַכֹּהֲנִ֧ים אֶת־דָּמ֛וֹ עַל־הַמִּזְבֵּ֖חַ סָבִֽיב׃ וְנִתַּ֤ח אֹתוֹ֙ לִנְתָחָ֔יו וְאֶת־רֹאשׁ֖וֹ וְאֶת־פִּדְר֑וֹ וְעָרַ֤ךְ הַכֹּהֵן֙ אֹתָ֔ם עַל־הָֽעֵצִים֙ אֲשֶׁ֣ר עַל־הָאֵ֔שׁ אֲשֶׁ֖ר עַל־הַמִּזְבֵּֽחַ׃ וְהַקֶּ֥רֶב וְהַכְּרָעַ֖יִם יִרְחַ֣ץ בַּמָּ֑יִם וְהִקְרִ֨יב הַכֹּהֵ֤ן אֶת־הַכֹּל֙ וְהִקְטִ֣יר הַמִּזְבֵּ֔חָה עֹלָ֣ה ה֗וּא אִשֵּׁ֛ה רֵ֥יחַ נִיחֹ֖חַ לַייָ׃ 


\ssubsection{צו}\\
וַיְדַבֵּ֥ר יְיָ֖ אֶל־מֹשֶׁ֥ה לֵּאמֹֽר׃ צַ֤ו אֶֽת־אַהֲרֹן֙ וְאֶת־בָּנָ֣יו לֵאמֹ֔ר זֹ֥את תּוֹרַ֖ת הָעֹלָ֑ה הִ֣וא הָעֹלָ֡ה עַל֩ מוֹקְדָ֨ה עַל־הַמִּזְבֵּ֤חַ כׇּל־הַלַּ֙יְלָה֙ עַד־הַבֹּ֔קֶר וְאֵ֥שׁ הַמִּזְבֵּ֖חַ תּ֥וּקַד בּֽוֹ׃ וְלָבַ֨שׁ הַכֹּהֵ֜ן מִדּ֣וֹ בַ֗ד וּמִֽכְנְסֵי־בַד֮ יִלְבַּ֣שׁ עַל־בְּשָׂרוֹ֒ וְהֵרִ֣ים אֶת־הַדֶּ֗שֶׁן אֲשֶׁ֨ר תֹּאכַ֥ל הָאֵ֛שׁ אֶת־הָעֹלָ֖ה עַל־הַמִּזְבֵּ֑חַ וְשָׂמ֕וֹ אֵ֖צֶל הַמִּזְבֵּֽחַ׃
\aliyah{לוי}
וּפָשַׁט֙ אֶת־בְּגָדָ֔יו וְלָבַ֖שׁ בְּגָדִ֣ים אֲחֵרִ֑ים וְהוֹצִ֤יא אֶת־הַדֶּ֙שֶׁן֙ אֶל־מִח֣וּץ לַֽמַּחֲנֶ֔ה אֶל־מָק֖וֹם טָהֽוֹר׃ וְהָאֵ֨שׁ עַל־הַמִּזְבֵּ֤חַ תּֽוּקַד־בּוֹ֙ לֹ֣א תִכְבֶּ֔ה וּבִעֵ֨ר עָלֶ֧יהָ הַכֹּהֵ֛ן עֵצִ֖ים בַּבֹּ֣קֶר בַּבֹּ֑קֶר וְעָרַ֤ךְ עָלֶ֙יהָ֙ הָֽעֹלָ֔ה וְהִקְטִ֥יר עָלֶ֖יהָ חֶלְבֵ֥י הַשְּׁלָמִֽים׃ אֵ֗שׁ תָּמִ֛יד תּוּקַ֥ד עַל־הַמִּזְבֵּ֖חַ לֹ֥א תִכְבֶּֽה׃ 
\aliyah{ישראל}
וְזֹ֥את תּוֹרַ֖ת הַמִּנְחָ֑ה הַקְרֵ֨ב אֹתָ֤הּ בְּנֵֽי־אַהֲרֹן֙ לִפְנֵ֣י יְיָ֔ אֶל־פְּנֵ֖י הַמִּזְבֵּֽחַ׃ וְהֵרִ֨ים מִמֶּ֜נּוּ בְּקֻמְצ֗וֹ מִסֹּ֤לֶת הַמִּנְחָה֙ וּמִשַּׁמְנָ֔הּ וְאֵת֙ כׇּל־הַלְּבֹנָ֔ה אֲשֶׁ֖ר עַל־הַמִּנְחָ֑ה וְהִקְטִ֣יר הַמִּזְבֵּ֗חַ רֵ֧יחַ נִיחֹ֛חַ אַזְכָּרָתָ֖הּ לַייָ׃ וְהַנּוֹתֶ֣רֶת מִמֶּ֔נָּה יֹאכְל֖וּ אַהֲרֹ֣ן וּבָנָ֑יו מַצּ֤וֹת תֵּֽאָכֵל֙ בְּמָק֣וֹם קָדֹ֔שׁ בַּחֲצַ֥ר אֹֽהֶל־מוֹעֵ֖ד יֹאכְלֽוּהָ׃ לֹ֤א תֵאָפֶה֙ חָמֵ֔ץ חֶלְקָ֛ם נָתַ֥תִּי אֹתָ֖הּ מֵאִשָּׁ֑י קֹ֤דֶשׁ קׇֽדָשִׁים֙ הִ֔וא כַּחַטָּ֖את וְכָאָשָֽׁם׃ כׇּל־זָכָ֞ר בִּבְנֵ֤י אַהֲרֹן֙ יֹֽאכְלֶ֔נָּה חׇק־עוֹלָם֙ לְדֹרֹ֣תֵיכֶ֔ם מֵאִשֵּׁ֖י יְיָ֑ כֹּ֛ל אֲשֶׁר־יִגַּ֥ע בָּהֶ֖ם יִקְדָּֽשׁ׃ 


\ssubsection{שמיני}\\
וַיְהִי֙ בַּיּ֣וֹם הַשְּׁמִינִ֔י קָרָ֣א מֹשֶׁ֔ה לְאַהֲרֹ֖ן וּלְבָנָ֑יו וּלְזִקְנֵ֖י יִשְׂרָאֵֽל׃ וַיֹּ֣אמֶר אֶֽל־אַהֲרֹ֗ן קַח־לְ֠ךָ֠ עֵ֣גֶל בֶּן־בָּקָ֧ר לְחַטָּ֛את וְאַ֥יִל לְעֹלָ֖ה תְּמִימִ֑ם וְהַקְרֵ֖ב לִפְנֵ֥י יְיָ׃ וְאֶל־בְּנֵ֥י יִשְׂרָאֵ֖ל תְּדַבֵּ֣ר לֵאמֹ֑ר קְח֤וּ שְׂעִיר־עִזִּים֙ לְחַטָּ֔את וְעֵ֨גֶל וָכֶ֧בֶשׂ בְּנֵי־שָׁנָ֛ה תְּמִימִ֖ם לְעֹלָֽה׃ וְשׁ֨וֹר וָאַ֜יִל לִשְׁלָמִ֗ים לִזְבֹּ֙חַ֙ לִפְנֵ֣י יְיָ֔ וּמִנְחָ֖ה בְּלוּלָ֣ה בַשָּׁ֑מֶן כִּ֣י הַיּ֔וֹם יְיָ֖ נִרְאָ֥ה אֲלֵיכֶֽם׃ וַיִּקְח֗וּ אֵ֚ת אֲשֶׁ֣ר צִוָּ֣ה מֹשֶׁ֔ה אֶל־פְּנֵ֖י אֹ֣הֶל מוֹעֵ֑ד וַֽיִּקְרְבוּ֙ כׇּל־הָ֣עֵדָ֔ה וַיַּֽעַמְד֖וּ לִפְנֵ֥י יְיָ׃ וַיֹּ֣אמֶר מֹשֶׁ֔ה זֶ֧ה הַדָּבָ֛ר אֲשֶׁר־צִוָּ֥ה יְיָ֖ תַּעֲשׂ֑וּ וְיֵרָ֥א אֲלֵיכֶ֖ם כְּב֥וֹד יְיָ׃
\aliyah{לוי}
וַיֹּ֨אמֶר מֹשֶׁ֜ה אֶֽל־אַהֲרֹ֗ן קְרַ֤ב אֶל־הַמִּזְבֵּ֙חַ֙ וַעֲשֵׂ֞ה אֶת־חַטָּֽאתְךָ֙ וְאֶת־עֹ֣לָתֶ֔ךָ וְכַפֵּ֥ר בַּֽעַדְךָ֖ וּבְעַ֣ד הָעָ֑ם וַעֲשֵׂ֞ה אֶת־קׇרְבַּ֤ן הָעָם֙ וְכַפֵּ֣ר בַּֽעֲדָ֔ם כַּאֲשֶׁ֖ר צִוָּ֥ה יְיָ׃ וַיִּקְרַ֥ב אַהֲרֹ֖ן אֶל־הַמִּזְבֵּ֑חַ וַיִּשְׁחַ֛ט אֶת־עֵ֥גֶל הַחַטָּ֖את אֲשֶׁר־לֽוֹ׃ וַ֠יַּקְרִ֠בוּ בְּנֵ֨י אַהֲרֹ֣ן אֶת־הַדָּם֮ אֵלָיו֒ וַיִּטְבֹּ֤ל אֶצְבָּעוֹ֙ בַּדָּ֔ם וַיִּתֵּ֖ן עַל־קַרְנ֣וֹת הַמִּזְבֵּ֑חַ וְאֶת־הַדָּ֣ם יָצַ֔ק אֶל־יְס֖וֹד הַמִּזְבֵּֽחַ׃ וְאֶת־הַחֵ֨לֶב וְאֶת־הַכְּלָיֹ֜ת וְאֶת־הַיֹּתֶ֤רֶת מִן־הַכָּבֵד֙ מִן־הַ֣חַטָּ֔את הִקְטִ֖יר הַמִּזְבֵּ֑חָה כַּאֲשֶׁ֛ר צִוָּ֥ה יְיָ֖ אֶת־מֹשֶֽׁה׃
\aliyah{ישראל}
וְאֶת־הַבָּשָׂ֖ר וְאֶת־הָע֑וֹר שָׂרַ֣ף בָּאֵ֔שׁ מִח֖וּץ לַֽמַּחֲנֶֽה׃ וַיִּשְׁחַ֖ט אֶת־הָעֹלָ֑ה וַ֠יַּמְצִ֠אוּ בְּנֵ֨י אַהֲרֹ֤ן אֵלָיו֙ אֶת־הַדָּ֔ם וַיִּזְרְקֵ֥הוּ עַל־הַמִּזְבֵּ֖חַ סָבִֽיב׃ וְאֶת־הָעֹלָ֗ה הִמְצִ֧יאוּ אֵלָ֛יו לִנְתָחֶ֖יהָ וְאֶת־הָרֹ֑אשׁ וַיַּקְטֵ֖ר עַל־הַמִּזְבֵּֽחַ׃ וַיִּרְחַ֥ץ אֶת־הַקֶּ֖רֶב וְאֶת־הַכְּרָעָ֑יִם וַיַּקְטֵ֥ר עַל־הָעֹלָ֖ה הַמִּזְבֵּֽחָה׃ וַיַּקְרֵ֕ב אֵ֖ת קׇרְבַּ֣ן הָעָ֑ם וַיִּקַּ֞ח אֶת־שְׂעִ֤יר הַֽחַטָּאת֙ אֲשֶׁ֣ר לָעָ֔ם וַיִּשְׁחָטֵ֥הוּ וַֽיְחַטְּאֵ֖הוּ כָּרִאשֽׁוֹן׃ וַיַּקְרֵ֖ב אֶת־הָעֹלָ֑ה וַֽיַּעֲשֶׂ֖הָ כַּמִּשְׁפָּֽט׃


\ssubsection{תזריע}\\
וַיְדַבֵּ֥ר יְיָ֖ אֶל־מֹשֶׁ֥ה לֵּאמֹֽר׃ דַּבֵּ֞ר אֶל־בְּנֵ֤י יִשְׂרָאֵל֙ לֵאמֹ֔ר אִשָּׁה֙ כִּ֣י תַזְרִ֔יעַ וְיָלְדָ֖ה זָכָ֑ר וְטָֽמְאָה֙ שִׁבְעַ֣ת יָמִ֔ים כִּימֵ֛י נִדַּ֥ת דְּוֺתָ֖הּ תִּטְמָֽא׃ וּבַיּ֖וֹם הַשְּׁמִינִ֑י יִמּ֖וֹל בְּשַׂ֥ר עׇרְלָתֽוֹ׃ וּשְׁלֹשִׁ֥ים יוֹם֙ וּשְׁלֹ֣שֶׁת יָמִ֔ים תֵּשֵׁ֖ב בִּדְמֵ֣י טׇהֳרָ֑ה בְּכׇל־קֹ֣דֶשׁ לֹֽא־תִגָּ֗ע וְאֶל־הַמִּקְדָּשׁ֙ לֹ֣א תָבֹ֔א עַד־מְלֹ֖את יְמֵ֥י טׇהֳרָֽהּ׃
\aliyah{לוי}
וְאִם־נְקֵבָ֣ה תֵלֵ֔ד וְטָמְאָ֥ה שְׁבֻעַ֖יִם כְּנִדָּתָ֑הּ וְשִׁשִּׁ֥ים יוֹם֙ וְשֵׁ֣שֶׁת יָמִ֔ים תֵּשֵׁ֖ב עַל־דְּמֵ֥י טׇהֳרָֽה׃ וּבִמְלֹ֣את ׀ יְמֵ֣י טׇהֳרָ֗הּ לְבֵן֮ א֣וֹ לְבַת֒ תָּבִ֞יא כֶּ֤בֶשׂ בֶּן־שְׁנָתוֹ֙ לְעֹלָ֔ה וּבֶן־יוֹנָ֥ה אוֹ־תֹ֖ר לְחַטָּ֑את אֶל־פֶּ֥תַח אֹֽהֶל־מוֹעֵ֖ד אֶל־הַכֹּהֵֽן׃ וְהִקְרִיב֞וֹ לִפְנֵ֤י יְיָ֙ וְכִפֶּ֣ר עָלֶ֔יהָ וְטָהֲרָ֖ה מִמְּקֹ֣ר דָּמֶ֑יהָ זֹ֤את תּוֹרַת֙ הַיֹּלֶ֔דֶת לַזָּכָ֖ר א֥וֹ לַנְּקֵבָֽה׃ וְאִם־לֹ֨א תִמְצָ֣א יָדָהּ֮ דֵּ֣י שֶׂה֒ וְלָקְחָ֣ה שְׁתֵּֽי־תֹרִ֗ים א֤וֹ שְׁנֵי֙ בְּנֵ֣י יוֹנָ֔ה אֶחָ֥ד לְעֹלָ֖ה וְאֶחָ֣ד לְחַטָּ֑את וְכִפֶּ֥ר עָלֶ֛יהָ הַכֹּהֵ֖ן וְטָהֵֽרָה׃ 
\aliyah{ישראל}
וַיְדַבֵּ֣ר יְיָ֔ אֶל־מֹשֶׁ֥ה וְאֶֽל־אַהֲרֹ֖ן לֵאמֹֽר׃ אָדָ֗ם כִּֽי־יִהְיֶ֤ה בְעוֹר־בְּשָׂרוֹ֙ שְׂאֵ֤ת אֽוֹ־סַפַּ֙חַת֙ א֣וֹ בַהֶ֔רֶת וְהָיָ֥ה בְעוֹר־בְּשָׂר֖וֹ לְנֶ֣גַע צָרָ֑עַת וְהוּבָא֙ אֶל־אַהֲרֹ֣ן הַכֹּהֵ֔ן א֛וֹ אֶל־אַחַ֥ד מִבָּנָ֖יו הַכֹּהֲנִֽים׃ וְרָאָ֣ה הַכֹּהֵ֣ן אֶת־הַנֶּ֣גַע בְּעֽוֹר־הַ֠בָּשָׂ֠ר וְשֵׂעָ֨ר בַּנֶּ֜גַע הָפַ֣ךְ ׀ לָבָ֗ן וּמַרְאֵ֤ה הַנֶּ֙גַע֙ עָמֹק֙ מֵע֣וֹר בְּשָׂר֔וֹ נֶ֥גַע צָרַ֖עַת ה֑וּא וְרָאָ֥הוּ הַכֹּהֵ֖ן וְטִמֵּ֥א אֹתֽוֹ׃ וְאִם־בַּהֶ֩רֶת֩ לְבָנָ֨ה הִ֜וא בְּע֣וֹר בְּשָׂר֗וֹ וְעָמֹק֙ אֵין־מַרְאֶ֣הָ מִן־הָע֔וֹר וּשְׂעָרָ֖ה לֹא־הָפַ֣ךְ לָבָ֑ן וְהִסְגִּ֧יר הַכֹּהֵ֛ן אֶת־הַנֶּ֖גַע שִׁבְעַ֥ת יָמִֽים׃ וְרָאָ֣הוּ הַכֹּהֵן֮ בַּיּ֣וֹם הַשְּׁבִיעִי֒ וְהִנֵּ֤ה הַנֶּ֙גַע֙ עָמַ֣ד בְּעֵינָ֔יו לֹֽא־פָשָׂ֥ה הַנֶּ֖גַע בָּע֑וֹר וְהִסְגִּיר֧וֹ הַכֹּהֵ֛ן שִׁבְעַ֥ת יָמִ֖ים שֵׁנִֽית׃


\ssubsection{מצורע}\\
וַיְדַבֵּ֥ר יְיָ֖ אֶל־מֹשֶׁ֥ה לֵּאמֹֽר׃ זֹ֤את תִּֽהְיֶה֙ תּוֹרַ֣ת הַמְּצֹרָ֔ע בְּי֖וֹם טׇהֳרָת֑וֹ וְהוּבָ֖א אֶל־הַכֹּהֵֽן׃ וְיָצָא֙ הַכֹּהֵ֔ן אֶל־מִח֖וּץ לַֽמַּחֲנֶ֑ה וְרָאָה֙ הַכֹּהֵ֔ן וְהִנֵּ֛ה נִרְפָּ֥א נֶֽגַע־הַצָּרַ֖עַת מִן־הַצָּרֽוּעַ׃ וְצִוָּה֙ הַכֹּהֵ֔ן וְלָקַ֧ח לַמִּטַּהֵ֛ר שְׁתֵּֽי־צִפֳּרִ֥ים חַיּ֖וֹת טְהֹר֑וֹת וְעֵ֣ץ אֶ֔רֶז וּשְׁנִ֥י תוֹלַ֖עַת וְאֵזֹֽב׃ וְצִוָּה֙ הַכֹּהֵ֔ן וְשָׁחַ֖ט אֶת־הַצִּפּ֣וֹר הָאֶחָ֑ת אֶל־כְּלִי־חֶ֖רֶשׂ עַל־מַ֥יִם חַיִּֽים׃
\aliyah{לוי}
אֶת־הַצִּפֹּ֤ר הַֽחַיָּה֙ יִקַּ֣ח אֹתָ֔הּ וְאֶת־עֵ֥ץ הָאֶ֛רֶז וְאֶת־שְׁנִ֥י הַתּוֹלַ֖עַת וְאֶת־הָאֵזֹ֑ב וְטָבַ֨ל אוֹתָ֜ם וְאֵ֣ת ׀ הַצִּפֹּ֣ר הַֽחַיָּ֗ה בְּדַם֙ הַצִּפֹּ֣ר הַשְּׁחֻטָ֔ה עַ֖ל הַמַּ֥יִם הַֽחַיִּֽים׃ וְהִזָּ֗ה עַ֧ל הַמִּטַּהֵ֛ר מִן־הַצָּרַ֖עַת שֶׁ֣בַע פְּעָמִ֑ים וְטִ֣הֲר֔וֹ וְשִׁלַּ֛ח אֶת־הַצִּפֹּ֥ר הַֽחַיָּ֖ה עַל־פְּנֵ֥י הַשָּׂדֶֽה׃ וְכִבֶּס֩ הַמִּטַּהֵ֨ר אֶת־בְּגָדָ֜יו וְגִלַּ֣ח אֶת־כׇּל־שְׂעָר֗וֹ וְרָחַ֤ץ בַּמַּ֙יִם֙ וְטָהֵ֔ר וְאַחַ֖ר יָב֣וֹא אֶל־הַֽמַּחֲנֶ֑ה וְיָשַׁ֛ב מִח֥וּץ לְאׇהֳל֖וֹ שִׁבְעַ֥ת יָמִֽים׃ וְהָיָה֩ בַיּ֨וֹם הַשְּׁבִיעִ֜י יְגַלַּ֣ח אֶת־כׇּל־שְׂעָר֗וֹ אֶת־רֹאשׁ֤וֹ וְאֶת־זְקָנוֹ֙ וְאֵת֙ גַּבֹּ֣ת עֵינָ֔יו וְאֶת־כׇּל־שְׂעָר֖וֹ יְגַלֵּ֑חַ וְכִבֶּ֣ס אֶת־בְּגָדָ֗יו וְרָחַ֧ץ אֶת־בְּשָׂר֛וֹ בַּמַּ֖יִם וְטָהֵֽר׃
\aliyah{ישראל}
וּבַיּ֣וֹם הַשְּׁמִינִ֗י יִקַּ֤ח שְׁנֵֽי־כְבָשִׂים֙ תְּמִימִ֔ם וְכַבְשָׂ֥ה אַחַ֛ת בַּת־שְׁנָתָ֖הּ תְּמִימָ֑ה וּשְׁלֹשָׁ֣ה עֶשְׂרֹנִ֗ים סֹ֤לֶת מִנְחָה֙ בְּלוּלָ֣ה בַשֶּׁ֔מֶן וְלֹ֥ג אֶחָ֖ד שָֽׁמֶן׃ וְהֶעֱמִ֞יד הַכֹּהֵ֣ן הַֽמְטַהֵ֗ר אֵ֛ת הָאִ֥ישׁ הַמִּטַּהֵ֖ר וְאֹתָ֑ם לִפְנֵ֣י יְיָ֔ פֶּ֖תַח אֹ֥הֶל מוֹעֵֽד׃ וְלָקַ֨ח הַכֹּהֵ֜ן אֶת־הַכֶּ֣בֶשׂ הָאֶחָ֗ד וְהִקְרִ֥יב אֹת֛וֹ לְאָשָׁ֖ם וְאֶת־לֹ֣ג הַשָּׁ֑מֶן וְהֵנִ֥יף אֹתָ֛ם תְּנוּפָ֖ה לִפְנֵ֥י יְיָ׃


\ssubsection{אחרי מות}\\
וַיְדַבֵּ֤ר יְיָ֙ אֶל־מֹשֶׁ֔ה אַחֲרֵ֣י מ֔וֹת שְׁנֵ֖י בְּנֵ֣י אַהֲרֹ֑ן בְּקׇרְבָתָ֥ם לִפְנֵי־יְיָ֖ וַיָּמֻֽתוּ׃ וַיֹּ֨אמֶר יְיָ֜ אֶל־מֹשֶׁ֗ה דַּבֵּר֮ אֶל־אַהֲרֹ֣ן אָחִ֒יךָ֒ וְאַל־יָבֹ֤א בְכׇל־עֵת֙ אֶל־הַקֹּ֔דֶשׁ מִבֵּ֖ית לַפָּרֹ֑כֶת אֶל־פְּנֵ֨י הַכַּפֹּ֜רֶת אֲשֶׁ֤ר עַל־הָאָרֹן֙ וְלֹ֣א יָמ֔וּת כִּ֚י בֶּֽעָנָ֔ן אֵרָאֶ֖ה עַל־הַכַּפֹּֽרֶת׃ בְּזֹ֛את יָבֹ֥א אַהֲרֹ֖ן אֶל־הַקֹּ֑דֶשׁ בְּפַ֧ר בֶּן־בָּקָ֛ר לְחַטָּ֖את וְאַ֥יִל לְעֹלָֽה׃ כְּתֹֽנֶת־בַּ֨ד קֹ֜דֶשׁ יִלְבָּ֗שׁ וּמִֽכְנְסֵי־בַד֮ יִהְי֣וּ עַל־בְּשָׂרוֹ֒ וּבְאַבְנֵ֥ט בַּד֙ יַחְגֹּ֔ר וּבְמִצְנֶ֥פֶת בַּ֖ד יִצְנֹ֑ף בִּגְדֵי־קֹ֣דֶשׁ הֵ֔ם וְרָחַ֥ץ בַּמַּ֛יִם אֶת־בְּשָׂר֖וֹ וּלְבֵשָֽׁם׃ וּמֵאֵ֗ת עֲדַת֙ בְּנֵ֣י יִשְׂרָאֵ֔ל יִקַּ֛ח שְׁנֵֽי־שְׂעִירֵ֥י עִזִּ֖ים לְחַטָּ֑את וְאַ֥יִל אֶחָ֖ד לְעֹלָֽה׃ וְהִקְרִ֧יב אַהֲרֹ֛ן אֶת־פַּ֥ר הַחַטָּ֖את אֲשֶׁר־ל֑וֹ וְכִפֶּ֥ר בַּעֲד֖וֹ וּבְעַ֥ד בֵּיתֽוֹ׃
\aliyah{לוי}
וְלָקַ֖ח אֶת־שְׁנֵ֣י הַשְּׂעִירִ֑ם וְהֶעֱמִ֤יד אֹתָם֙ לִפְנֵ֣י יְיָ֔ פֶּ֖תַח אֹ֥הֶל מוֹעֵֽד׃ וְנָתַ֧ן אַהֲרֹ֛ן עַל־שְׁנֵ֥י הַשְּׂעִירִ֖ם גֹּרָל֑וֹת גּוֹרָ֤ל אֶחָד֙ לַייָ֔ וְגוֹרָ֥ל אֶחָ֖ד לַעֲזָאזֵֽל׃ וְהִקְרִ֤יב אַהֲרֹן֙ אֶת־הַשָּׂעִ֔יר אֲשֶׁ֨ר עָלָ֥ה עָלָ֛יו הַגּוֹרָ֖ל לַייָ֑ וְעָשָׂ֖הוּ חַטָּֽאת׃ וְהַשָּׂעִ֗יר אֲשֶׁר֩ עָלָ֨ה עָלָ֤יו הַגּוֹרָל֙ לַעֲזָאזֵ֔ל יׇֽעֳמַד־חַ֛י לִפְנֵ֥י יְיָ֖ לְכַפֵּ֣ר עָלָ֑יו לְשַׁלַּ֥ח אֹת֛וֹ לַעֲזָאזֵ֖ל הַמִּדְבָּֽרָה׃ וְהִקְרִ֨יב אַהֲרֹ֜ן אֶת־פַּ֤ר הַֽחַטָּאת֙ אֲשֶׁר־ל֔וֹ וְכִפֶּ֥ר בַּֽעֲד֖וֹ וּבְעַ֣ד בֵּית֑וֹ וְשָׁחַ֛ט אֶת־פַּ֥ר הַֽחַטָּ֖את אֲשֶׁר־לֽוֹ׃
\aliyah{ישראל}
וְלָקַ֣ח מְלֹֽא־הַ֠מַּחְתָּ֠ה גַּֽחֲלֵי־אֵ֞שׁ מֵעַ֤ל הַמִּזְבֵּ֙חַ֙ מִלִּפְנֵ֣י יְיָ֔ וּמְלֹ֣א חׇפְנָ֔יו קְטֹ֥רֶת סַמִּ֖ים דַּקָּ֑ה וְהֵבִ֖יא מִבֵּ֥ית לַפָּרֹֽכֶת׃ וְנָתַ֧ן אֶֽת־הַקְּטֹ֛רֶת עַל־הָאֵ֖שׁ לִפְנֵ֣י יְיָ֑ וְכִסָּ֣ה ׀ עֲנַ֣ן הַקְּטֹ֗רֶת אֶת־הַכַּפֹּ֛רֶת אֲשֶׁ֥ר עַל־הָעֵד֖וּת וְלֹ֥א יָמֽוּת׃ וְלָקַח֙ מִדַּ֣ם הַפָּ֔ר וְהִזָּ֧ה בְאֶצְבָּע֛וֹ עַל־פְּנֵ֥י הַכַּפֹּ֖רֶת קֵ֑דְמָה וְלִפְנֵ֣י הַכַּפֹּ֗רֶת יַזֶּ֧ה שֶֽׁבַע־פְּעָמִ֛ים מִן־הַדָּ֖ם בְּאֶצְבָּעֽוֹ׃ וְשָׁחַ֞ט אֶת־שְׂעִ֤יר הַֽחַטָּאת֙ אֲשֶׁ֣ר לָעָ֔ם וְהֵבִיא֙ אֶת־דָּמ֔וֹ אֶל־מִבֵּ֖ית לַפָּרֹ֑כֶת וְעָשָׂ֣ה אֶת־דָּמ֗וֹ כַּאֲשֶׁ֤ר עָשָׂה֙ לְדַ֣ם הַפָּ֔ר וְהִזָּ֥ה אֹת֛וֹ עַל־הַכַּפֹּ֖רֶת וְלִפְנֵ֥י הַכַּפֹּֽרֶת׃ וְכִפֶּ֣ר עַל־הַקֹּ֗דֶשׁ מִטֻּמְאֹת֙ בְּנֵ֣י יִשְׂרָאֵ֔ל וּמִפִּשְׁעֵיהֶ֖ם לְכׇל־חַטֹּאתָ֑ם וְכֵ֤ן יַעֲשֶׂה֙ לְאֹ֣הֶל מוֹעֵ֔ד הַשֹּׁכֵ֣ן אִתָּ֔ם בְּת֖וֹךְ טֻמְאֹתָֽם׃ וְכׇל־אָדָ֞ם לֹא־יִהְיֶ֣ה ׀ בְּאֹ֣הֶל מוֹעֵ֗ד בְּבֹא֛וֹ לְכַפֵּ֥ר בַּקֹּ֖דֶשׁ עַד־צֵאת֑וֹ וְכִפֶּ֤ר בַּעֲדוֹ֙ וּבְעַ֣ד בֵּית֔וֹ וּבְעַ֖ד כׇּל־קְהַ֥ל יִשְׂרָאֵֽל׃


\ssubsection{קדשים}\\
וַיְדַבֵּ֥ר יְיָ֖ אֶל־מֹשֶׁ֥ה לֵּאמֹֽר׃ דַּבֵּ֞ר אֶל־כׇּל־עֲדַ֧ת בְּנֵי־יִשְׂרָאֵ֛ל וְאָמַרְתָּ֥ אֲלֵהֶ֖ם קְדֹשִׁ֣ים תִּהְי֑וּ כִּ֣י קָד֔וֹשׁ אֲנִ֖י יְיָ֥ אֱלֹהֵיכֶֽם׃ אִ֣ישׁ אִמּ֤וֹ וְאָבִיו֙ תִּירָ֔אוּ וְאֶת־שַׁבְּתֹתַ֖י תִּשְׁמֹ֑רוּ אֲנִ֖י יְיָ֥ אֱלֹהֵיכֶֽם׃ אַל־תִּפְנוּ֙ אֶל־הָ֣אֱלִילִ֔ם וֵֽאלֹהֵי֙ מַסֵּכָ֔ה לֹ֥א תַעֲשׂ֖וּ לָכֶ֑ם אֲנִ֖י יְיָ֥ אֱלֹהֵיכֶֽם׃
\aliyah{לוי}
וְכִ֧י תִזְבְּח֛וּ זֶ֥בַח שְׁלָמִ֖ים לַייָ֑ לִֽרְצֹנְכֶ֖ם תִּזְבָּחֻֽהוּ׃ בְּי֧וֹם זִבְחֲכֶ֛ם יֵאָכֵ֖ל וּמִֽמׇּחֳרָ֑ת וְהַנּוֹתָר֙ עַד־י֣וֹם הַשְּׁלִישִׁ֔י בָּאֵ֖שׁ יִשָּׂרֵֽף׃ וְאִ֛ם הֵאָכֹ֥ל יֵאָכֵ֖ל בַּיּ֣וֹם הַשְּׁלִישִׁ֑י פִּגּ֥וּל ה֖וּא לֹ֥א יֵרָצֶֽה׃ וְאֹֽכְלָיו֙ עֲוֺנ֣וֹ יִשָּׂ֔א כִּֽי־אֶת־קֹ֥דֶשׁ יְיָ֖ חִלֵּ֑ל וְנִכְרְתָ֛ה הַנֶּ֥פֶשׁ הַהִ֖וא מֵעַמֶּֽיהָ׃ וּֽבְקֻצְרְכֶם֙ אֶת־קְצִ֣יר אַרְצְכֶ֔ם לֹ֧א תְכַלֶּ֛ה פְּאַ֥ת שָׂדְךָ֖ לִקְצֹ֑ר וְלֶ֥קֶט קְצִֽירְךָ֖ לֹ֥א תְלַקֵּֽט׃ וְכַרְמְךָ֙ לֹ֣א תְעוֹלֵ֔ל וּפֶ֥רֶט כַּרְמְךָ֖ לֹ֣א תְלַקֵּ֑ט לֶֽעָנִ֤י וְלַגֵּר֙ תַּעֲזֹ֣ב אֹתָ֔ם אֲנִ֖י יְיָ֥ אֱלֹהֵיכֶֽם׃
\aliyah{ישראל}
לֹ֖א תִּגְנֹ֑בוּ וְלֹא־תְכַחֲשׁ֥וּ וְלֹֽא־תְשַׁקְּר֖וּ אִ֥ישׁ בַּעֲמִיתֽוֹ׃ וְלֹֽא־תִשָּׁבְע֥וּ בִשְׁמִ֖י לַשָּׁ֑קֶר וְחִלַּלְתָּ֛ אֶת־שֵׁ֥ם אֱלֹהֶ֖יךָ אֲנִ֥י יְיָ׃ לֹֽא־תַעֲשֹׁ֥ק אֶת־רֵֽעֲךָ֖ וְלֹ֣א תִגְזֹ֑ל לֹֽא־תָלִ֞ין פְּעֻלַּ֥ת שָׂכִ֛יר אִתְּךָ֖ עַד־בֹּֽקֶר׃ לֹא־תְקַלֵּ֣ל חֵרֵ֔שׁ וְלִפְנֵ֣י עִוֵּ֔ר לֹ֥א תִתֵּ֖ן מִכְשֹׁ֑ל וְיָרֵ֥אתָ מֵּאֱלֹהֶ֖יךָ אֲנִ֥י יְיָ׃


\ssubsection{אמר}\\
וַיֹּ֤אמֶר יְיָ֙ אֶל־מֹשֶׁ֔ה אֱמֹ֥ר אֶל־הַכֹּהֲנִ֖ים בְּנֵ֣י אַהֲרֹ֑ן וְאָמַרְתָּ֣ אֲלֵהֶ֔ם לְנֶ֥פֶשׁ לֹֽא־יִטַּמָּ֖א בְּעַמָּֽיו׃ כִּ֚י אִם־לִשְׁאֵר֔וֹ הַקָּרֹ֖ב אֵלָ֑יו לְאִמּ֣וֹ וּלְאָבִ֔יו וְלִבְנ֥וֹ וּלְבִתּ֖וֹ וּלְאָחִֽיו׃ וְלַאֲחֹת֤וֹ הַבְּתוּלָה֙ הַקְּרוֹבָ֣ה אֵלָ֔יו אֲשֶׁ֥ר לֹֽא־הָיְתָ֖ה לְאִ֑ישׁ לָ֖הּ יִטַּמָּֽא׃ לֹ֥א יִטַּמָּ֖א בַּ֣עַל בְּעַמָּ֑יו לְהֵ֖חַלּֽוֹ׃ לֹֽא־[יִקְרְח֤וּ] (יקרחה) קׇרְחָה֙ בְּרֹאשָׁ֔ם וּפְאַ֥ת זְקָנָ֖ם לֹ֣א יְגַלֵּ֑חוּ וּבִ֨בְשָׂרָ֔ם לֹ֥א יִשְׂרְט֖וּ שָׂרָֽטֶת׃ קְדֹשִׁ֤ים יִהְיוּ֙ לֵאלֹ֣הֵיהֶ֔ם וְלֹ֣א יְחַלְּל֔וּ שֵׁ֖ם אֱלֹהֵיהֶ֑ם כִּי֩ אֶת־אִשֵּׁ֨י יְיָ֜ לֶ֧חֶם אֱלֹהֵיהֶ֛ם הֵ֥ם מַקְרִיבִ֖ם וְהָ֥יוּ קֹֽדֶשׁ׃
\aliyah{לוי}
אִשָּׁ֨ה זֹנָ֤ה וַחֲלָלָה֙ לֹ֣א יִקָּ֔חוּ וְאִשָּׁ֛ה גְּרוּשָׁ֥ה מֵאִישָׁ֖הּ לֹ֣א יִקָּ֑חוּ כִּֽי־קָדֹ֥שׁ ה֖וּא לֵאלֹהָֽיו׃ וְקִ֨דַּשְׁתּ֔וֹ כִּֽי־אֶת־לֶ֥חֶם אֱלֹהֶ֖יךָ ה֣וּא מַקְרִ֑יב קָדֹשׁ֙ יִֽהְיֶה־לָּ֔ךְ כִּ֣י קָד֔וֹשׁ אֲנִ֥י יְיָ֖ מְקַדִּשְׁכֶֽם׃ וּבַת֙ אִ֣ישׁ כֹּהֵ֔ן כִּ֥י תֵחֵ֖ל לִזְנ֑וֹת אֶת־אָבִ֙יהָ֙ הִ֣יא מְחַלֶּ֔לֶת בָּאֵ֖שׁ תִּשָּׂרֵֽף׃ {ס} וְהַכֹּהֵן֩ הַגָּד֨וֹל מֵאֶחָ֜יו אֲֽשֶׁר־יוּצַ֥ק עַל־רֹאשׁ֣וֹ ׀ שֶׁ֤מֶן הַמִּשְׁחָה֙ וּמִלֵּ֣א אֶת־יָד֔וֹ לִלְבֹּ֖שׁ אֶת־הַבְּגָדִ֑ים אֶת־רֹאשׁוֹ֙ לֹ֣א יִפְרָ֔ע וּבְגָדָ֖יו לֹ֥א יִפְרֹֽם׃ וְעַ֛ל כׇּל־נַפְשֹׁ֥ת מֵ֖ת לֹ֣א יָבֹ֑א לְאָבִ֥יו וּלְאִמּ֖וֹ לֹ֥א יִטַּמָּֽא׃ וּמִן־הַמִּקְדָּשׁ֙ לֹ֣א יֵצֵ֔א וְלֹ֣א יְחַלֵּ֔ל אֵ֖ת מִקְדַּ֣שׁ אֱלֹהָ֑יו כִּ֡י נֵ֠זֶר שֶׁ֣מֶן מִשְׁחַ֧ת אֱלֹהָ֛יו עָלָ֖יו אֲנִ֥י יְיָ׃
\aliyah{ישראל}
וְה֕וּא אִשָּׁ֥ה בִבְתוּלֶ֖יהָ יִקָּֽח׃ אַלְמָנָ֤ה וּגְרוּשָׁה֙ וַחֲלָלָ֣ה זֹנָ֔ה אֶת־אֵ֖לֶּה לֹ֣א יִקָּ֑ח כִּ֛י אִם־בְּתוּלָ֥ה מֵעַמָּ֖יו יִקַּ֥ח אִשָּֽׁה׃ וְלֹֽא־יְחַלֵּ֥ל זַרְע֖וֹ בְּעַמָּ֑יו כִּ֛י אֲנִ֥י יְיָ֖ מְקַדְּשֽׁוֹ׃ 


\ssubsection{בהר}\\
וַיְדַבֵּ֤ר יְיָ֙ אֶל־מֹשֶׁ֔ה בְּהַ֥ר סִינַ֖י לֵאמֹֽר׃ דַּבֵּ֞ר אֶל־בְּנֵ֤י יִשְׂרָאֵל֙ וְאָמַרְתָּ֣ אֲלֵהֶ֔ם כִּ֤י תָבֹ֙אוּ֙ אֶל־הָאָ֔רֶץ אֲשֶׁ֥ר אֲנִ֖י נֹתֵ֣ן לָכֶ֑ם וְשָׁבְתָ֣ה הָאָ֔רֶץ שַׁבָּ֖ת לַייָ׃ שֵׁ֤שׁ שָׁנִים֙ תִּזְרַ֣ע שָׂדֶ֔ךָ וְשֵׁ֥שׁ שָׁנִ֖ים תִּזְמֹ֣ר כַּרְמֶ֑ךָ וְאָסַפְתָּ֖ אֶת־תְּבוּאָתָֽהּ׃
\aliyah{לוי}
וּבַשָּׁנָ֣ה הַשְּׁבִיעִ֗ת שַׁבַּ֤ת שַׁבָּתוֹן֙ יִהְיֶ֣ה לָאָ֔רֶץ שַׁבָּ֖ת לַייָ֑ שָֽׂדְךָ֙ לֹ֣א תִזְרָ֔ע וְכַרְמְךָ֖ לֹ֥א תִזְמֹֽר׃ אֵ֣ת סְפִ֤יחַ קְצִֽירְךָ֙ לֹ֣א תִקְצ֔וֹר וְאֶת־עִנְּבֵ֥י נְזִירֶ֖ךָ לֹ֣א תִבְצֹ֑ר שְׁנַ֥ת שַׁבָּת֖וֹן יִהְיֶ֥ה לָאָֽרֶץ׃ וְ֠הָיְתָ֠ה שַׁבַּ֨ת הָאָ֤רֶץ לָכֶם֙ לְאׇכְלָ֔ה לְךָ֖ וּלְעַבְדְּךָ֣ וְלַאֲמָתֶ֑ךָ וְלִשְׂכִֽירְךָ֙ וּלְתוֹשָׁ֣בְךָ֔ הַגָּרִ֖ים עִמָּֽךְ׃ וְלִ֨בְהֶמְתְּךָ֔ וְלַֽחַיָּ֖ה אֲשֶׁ֣ר בְּאַרְצֶ֑ךָ תִּהְיֶ֥ה כׇל־תְּבוּאָתָ֖הּ לֶאֱכֹֽל׃ 
\aliyah{ישראל}
וְסָפַרְתָּ֣ לְךָ֗ שֶׁ֚בַע שַׁבְּתֹ֣ת שָׁנִ֔ים שֶׁ֥בַע שָׁנִ֖ים שֶׁ֣בַע פְּעָמִ֑ים וְהָי֣וּ לְךָ֗ יְמֵי֙ שֶׁ֚בַע שַׁבְּתֹ֣ת הַשָּׁנִ֔ים תֵּ֥שַׁע וְאַרְבָּעִ֖ים שָׁנָֽה׃ וְהַֽעֲבַרְתָּ֞ שׁוֹפַ֤ר תְּרוּעָה֙ בַּחֹ֣דֶשׁ הַשְּׁבִעִ֔י בֶּעָשׂ֖וֹר לַחֹ֑דֶשׁ בְּיוֹם֙ הַכִּפֻּרִ֔ים תַּעֲבִ֥ירוּ שׁוֹפָ֖ר בְּכׇל־אַרְצְכֶֽם׃ וְקִדַּשְׁתֶּ֗ם אֵ֣ת שְׁנַ֤ת הַחֲמִשִּׁים֙ שָׁנָ֔ה וּקְרָאתֶ֥ם דְּר֛וֹר בָּאָ֖רֶץ לְכׇל־יֹשְׁבֶ֑יהָ יוֹבֵ֥ל הִוא֙ תִּהְיֶ֣ה לָכֶ֔ם וְשַׁבְתֶּ֗ם אִ֚ישׁ אֶל־אֲחֻזָּת֔וֹ וְאִ֥ישׁ אֶל־מִשְׁפַּחְתּ֖וֹ תָּשֻֽׁבוּ׃ יוֹבֵ֣ל הִ֗וא שְׁנַ֛ת הַחֲמִשִּׁ֥ים שָׁנָ֖ה תִּהְיֶ֣ה לָכֶ֑ם לֹ֣א תִזְרָ֔עוּ וְלֹ֤א תִקְצְרוּ֙ אֶת־סְפִיחֶ֔יהָ וְלֹ֥א תִבְצְר֖וּ אֶת־נְזִרֶֽיהָ׃ כִּ֚י יוֹבֵ֣ל הִ֔וא קֹ֖דֶשׁ תִּהְיֶ֣ה לָכֶ֑ם מִ֨ן־הַשָּׂדֶ֔ה תֹּאכְל֖וּ אֶת־תְּבוּאָתָֽהּ׃ בִּשְׁנַ֥ת הַיּוֹבֵ֖ל הַזֹּ֑את תָּשֻׁ֕בוּ אִ֖ישׁ אֶל־אֲחֻזָּתֽוֹ׃




\ssubsection{בחקתי}\\
אִם־בְּחֻקֹּתַ֖י תֵּלֵ֑כוּ וְאֶת־מִצְוֺתַ֣י תִּשְׁמְר֔וּ וַעֲשִׂיתֶ֖ם אֹתָֽם׃ וְנָתַתִּ֥י גִשְׁמֵיכֶ֖ם בְּעִתָּ֑ם וְנָתְנָ֤ה הָאָ֙רֶץ֙ יְבוּלָ֔הּ וְעֵ֥ץ הַשָּׂדֶ֖ה יִתֵּ֥ן פִּרְיֽוֹ׃ וְהִשִּׂ֨יג לָכֶ֥ם דַּ֙יִשׁ֙ אֶת־בָּצִ֔יר וּבָצִ֖יר יַשִּׂ֣יג אֶת־זָ֑רַע וַאֲכַלְתֶּ֤ם לַחְמְכֶם֙ לָשֹׂ֔בַע וִֽישַׁבְתֶּ֥ם לָבֶ֖טַח בְּאַרְצְכֶֽם׃
\aliyah{לוי}
וְנָתַתִּ֤י שָׁלוֹם֙ בָּאָ֔רֶץ וּשְׁכַבְתֶּ֖ם וְאֵ֣ין מַחֲרִ֑יד וְהִשְׁבַּתִּ֞י חַיָּ֤ה רָעָה֙ מִן־הָאָ֔רֶץ וְחֶ֖רֶב לֹא־תַעֲבֹ֥ר בְּאַרְצְכֶֽם׃ וּרְדַפְתֶּ֖ם אֶת־אֹיְבֵיכֶ֑ם וְנָפְל֥וּ לִפְנֵיכֶ֖ם לֶחָֽרֶב׃ וְרָדְפ֨וּ מִכֶּ֤ם חֲמִשָּׁה֙ מֵאָ֔ה וּמֵאָ֥ה מִכֶּ֖ם רְבָבָ֣ה יִרְדֹּ֑פוּ וְנָפְל֧וּ אֹיְבֵיכֶ֛ם לִפְנֵיכֶ֖ם לֶחָֽרֶב׃ וּפָנִ֣יתִי אֲלֵיכֶ֔ם וְהִפְרֵיתִ֣י אֶתְכֶ֔ם וְהִרְבֵּיתִ֖י אֶתְכֶ֑ם וַהֲקִימֹתִ֥י אֶת־בְּרִיתִ֖י אִתְּכֶֽם׃
\aliyah{ישראל}
וַאֲכַלְתֶּ֥ם יָשָׁ֖ן נוֹשָׁ֑ן וְיָשָׁ֕ן מִפְּנֵ֥י חָדָ֖שׁ תּוֹצִֽיאוּ׃ וְנָתַתִּ֥י מִשְׁכָּנִ֖י בְּתוֹכְכֶ֑ם וְלֹֽא־תִגְעַ֥ל נַפְשִׁ֖י אֶתְכֶֽם׃ וְהִתְהַלַּכְתִּי֙ בְּת֣וֹכְכֶ֔ם וְהָיִ֥יתִי לָכֶ֖ם לֵֽאלֹהִ֑ים וְאַתֶּ֖ם תִּהְיוּ־לִ֥י לְעָֽם׃ אֲנִ֞י יְיָ֣ אֱלֹֽהֵיכֶ֗ם אֲשֶׁ֨ר הוֹצֵ֤אתִי אֶתְכֶם֙ מֵאֶ֣רֶץ מִצְרַ֔יִם מִֽהְיֹ֥ת לָהֶ֖ם עֲבָדִ֑ים וָאֶשְׁבֹּר֙ מֹטֹ֣ת עֻלְּכֶ֔ם וָאוֹלֵ֥ךְ אֶתְכֶ֖ם קֽוֹמְמִיּֽוּת׃ 




\ssubsection{במדבר}\\
וַיְדַבֵּ֨ר יְיָ֧ אֶל־מֹשֶׁ֛ה בְּמִדְבַּ֥ר סִינַ֖י בְּאֹ֣הֶל מוֹעֵ֑ד בְּאֶחָד֩ לַחֹ֨דֶשׁ הַשֵּׁנִ֜י בַּשָּׁנָ֣ה הַשֵּׁנִ֗ית לְצֵאתָ֛ם מֵאֶ֥רֶץ מִצְרַ֖יִם לֵאמֹֽר׃ שְׂא֗וּ אֶת־רֹאשׁ֙ כׇּל־עֲדַ֣ת בְּנֵֽי־יִשְׂרָאֵ֔ל לְמִשְׁפְּחֹתָ֖ם לְבֵ֣ית אֲבֹתָ֑ם בְּמִסְפַּ֣ר שֵׁמ֔וֹת כׇּל־זָכָ֖ר לְגֻלְגְּלֹתָֽם׃ מִבֶּ֨ן עֶשְׂרִ֤ים שָׁנָה֙ וָמַ֔עְלָה כׇּל־יֹצֵ֥א צָבָ֖א בְּיִשְׂרָאֵ֑ל תִּפְקְד֥וּ אֹתָ֛ם לְצִבְאֹתָ֖ם אַתָּ֥ה וְאַהֲרֹֽן׃ וְאִתְּכֶ֣ם יִהְי֔וּ אִ֥ישׁ אִ֖ישׁ לַמַּטֶּ֑ה אִ֛ישׁ רֹ֥אשׁ לְבֵית־אֲבֹתָ֖יו הֽוּא׃
\aliyah{לוי}
וְאֵ֙לֶּה֙ שְׁמ֣וֹת הָֽאֲנָשִׁ֔ים אֲשֶׁ֥ר יַֽעַמְד֖וּ אִתְּכֶ֑ם לִרְאוּבֵ֕ן אֱלִיצ֖וּר בֶּן־שְׁדֵיאֽוּר׃ לְשִׁמְע֕וֹן שְׁלֻמִיאֵ֖ל בֶּן־צוּרִֽישַׁדָּֽי׃ לִֽיהוּדָ֕ה נַחְשׁ֖וֹן בֶּן־עַמִּינָדָֽב׃ לְיִ֨שָּׂשכָ֔ר נְתַנְאֵ֖ל בֶּן־צוּעָֽר׃ לִזְבוּלֻ֕ן אֱלִיאָ֖ב בֶּן־חֵלֹֽן׃ לִבְנֵ֣י יוֹסֵ֔ף לְאֶפְרַ֕יִם אֱלִישָׁמָ֖ע בֶּן־עַמִּיה֑וּד לִמְנַשֶּׁ֕ה גַּמְלִיאֵ֖ל בֶּן־פְּדָהצֽוּר׃ לְבִ֨נְיָמִ֔ן אֲבִידָ֖ן בֶּן־גִּדְעֹנִֽי׃ לְדָ֕ן אֲחִיעֶ֖זֶר בֶּן־עַמִּֽישַׁדָּֽי׃ לְאָשֵׁ֕ר פַּגְעִיאֵ֖ל בֶּן־עׇכְרָֽן׃ לְגָ֕ד אֶלְיָסָ֖ף בֶּן־דְּעוּאֵֽל׃ לְנַ֨פְתָּלִ֔י אֲחִירַ֖ע בֶּן־עֵינָֽן׃ אֵ֚לֶּה (קריאי) [קְרוּאֵ֣י] הָעֵדָ֔ה נְשִׂיאֵ֖י מַטּ֣וֹת אֲבוֹתָ֑ם רָאשֵׁ֛י אַלְפֵ֥י יִשְׂרָאֵ֖ל הֵֽם׃
\aliyah{ישראל}
וַיִּקַּ֥ח מֹשֶׁ֖ה וְאַהֲרֹ֑ן אֵ֚ת הָאֲנָשִׁ֣ים הָאֵ֔לֶּה אֲשֶׁ֥ר נִקְּב֖וּ בְּשֵׁמֽוֹת׃ וְאֵ֨ת כׇּל־הָעֵדָ֜ה הִקְהִ֗ילוּ בְּאֶחָד֙ לַחֹ֣דֶשׁ הַשֵּׁנִ֔י וַיִּתְיַֽלְד֥וּ עַל־מִשְׁפְּחֹתָ֖ם לְבֵ֣ית אֲבֹתָ֑ם בְּמִסְפַּ֣ר שֵׁמ֗וֹת מִבֶּ֨ן עֶשְׂרִ֥ים שָׁנָ֛ה וָמַ֖עְלָה לְגֻלְגְּלֹתָֽם׃ כַּאֲשֶׁ֛ר צִוָּ֥ה יְיָ֖ אֶת־מֹשֶׁ֑ה וַֽיִּפְקְדֵ֖ם בְּמִדְבַּ֥ר סִינָֽי׃ 




\ssubsection{נשא}\\
וַיְדַבֵּ֥ר יְיָ֖ אֶל־מֹשֶׁ֥ה לֵּאמֹֽר׃ נָשֹׂ֗א אֶת־רֹ֛אשׁ בְּנֵ֥י גֵרְשׁ֖וֹן גַּם־הֵ֑ם לְבֵ֥ית אֲבֹתָ֖ם לְמִשְׁפְּחֹתָֽם׃ מִבֶּן֩ שְׁלֹשִׁ֨ים שָׁנָ֜ה וָמַ֗עְלָה עַ֛ד בֶּן־חֲמִשִּׁ֥ים שָׁנָ֖ה תִּפְקֹ֣ד אוֹתָ֑ם כׇּל־הַבָּא֙ לִצְבֹ֣א צָבָ֔א לַעֲבֹ֥ד עֲבֹדָ֖ה בְּאֹ֥הֶל מוֹעֵֽד׃ זֹ֣את עֲבֹדַ֔ת מִשְׁפְּחֹ֖ת הַגֵּרְשֻׁנִּ֑י לַעֲבֹ֖ד וּלְמַשָּֽׂא׃
\aliyah{לוי}
וְנָ֨שְׂא֜וּ אֶת־יְרִיעֹ֤ת הַמִּשְׁכָּן֙ וְאֶת־אֹ֣הֶל מוֹעֵ֔ד מִכְסֵ֕הוּ וּמִכְסֵ֛ה הַתַּ֥חַשׁ אֲשֶׁר־עָלָ֖יו מִלְמָ֑עְלָה וְאֶ֨ת־מָסַ֔ךְ פֶּ֖תַח אֹ֥הֶל מוֹעֵֽד׃ וְאֵת֩ קַלְעֵ֨י הֶֽחָצֵ֜ר וְאֶת־מָסַ֣ךְ ׀ פֶּ֣תַח ׀ שַׁ֣עַר הֶחָצֵ֗ר אֲשֶׁ֨ר עַל־הַמִּשְׁכָּ֤ן וְעַל־הַמִּזְבֵּ֙חַ֙ סָבִ֔יב וְאֵת֙ מֵֽיתְרֵיהֶ֔ם וְאֶֽת־כׇּל־כְּלֵ֖י עֲבֹדָתָ֑ם וְאֵ֨ת כׇּל־אֲשֶׁ֧ר יֵעָשֶׂ֛ה לָהֶ֖ם וְעָבָֽדוּ׃ עַל־פִּי֩ אַהֲרֹ֨ן וּבָנָ֜יו תִּהְיֶ֗ה כׇּל־עֲבֹדַת֙ בְּנֵ֣י הַגֵּרְשֻׁנִּ֔י לְכׇ֨ל־מַשָּׂאָ֔ם וּלְכֹ֖ל עֲבֹדָתָ֑ם וּפְקַדְתֶּ֤ם עֲלֵהֶם֙ בְּמִשְׁמֶ֔רֶת אֵ֖ת כׇּל־מַשָּׂאָֽם׃ זֹ֣את עֲבֹדַ֗ת מִשְׁפְּחֹ֛ת בְּנֵ֥י הַגֵּרְשֻׁנִּ֖י בְּאֹ֣הֶל מוֹעֵ֑ד וּמִ֨שְׁמַרְתָּ֔ם בְּיַד֙ אִֽיתָמָ֔ר בֶּֽן־אַהֲרֹ֖ן הַכֹּהֵֽן׃ 
\aliyah{ישראל}
בְּנֵ֖י מְרָרִ֑י לְמִשְׁפְּחֹתָ֥ם לְבֵית־אֲבֹתָ֖ם תִּפְקֹ֥ד אֹתָֽם׃ מִבֶּן֩ שְׁלֹשִׁ֨ים שָׁנָ֜ה וָמַ֗עְלָה וְעַ֛ד בֶּן־חֲמִשִּׁ֥ים שָׁנָ֖ה תִּפְקְדֵ֑ם כׇּל־הַבָּא֙ לַצָּבָ֔א לַעֲבֹ֕ד אֶת־עֲבֹדַ֖ת אֹ֥הֶל מוֹעֵֽד׃ וְזֹאת֙ מִשְׁמֶ֣רֶת מַשָּׂאָ֔ם לְכׇל־עֲבֹדָתָ֖ם בְּאֹ֣הֶל מוֹעֵ֑ד קַרְשֵׁי֙ הַמִּשְׁכָּ֔ן וּבְרִיחָ֖יו וְעַמּוּדָ֥יו וַאֲדָנָֽיו׃ וְעַמּוּדֵי֩ הֶחָצֵ֨ר סָבִ֜יב וְאַדְנֵיהֶ֗ם וִיתֵֽדֹתָם֙ וּמֵ֣יתְרֵיהֶ֔ם לְכׇ֨ל־כְּלֵיהֶ֔ם וּלְכֹ֖ל עֲבֹדָתָ֑ם וּבְשֵׁמֹ֣ת תִּפְקְד֔וּ אֶת־כְּלֵ֖י מִשְׁמֶ֥רֶת מַשָּׂאָֽם׃ זֹ֣את עֲבֹדַ֗ת מִשְׁפְּחֹת֙ בְּנֵ֣י מְרָרִ֔י לְכׇל־עֲבֹדָתָ֖ם בְּאֹ֣הֶל מוֹעֵ֑ד בְּיַד֙ אִֽיתָמָ֔ר בֶּֽן־אַהֲרֹ֖ן הַכֹּהֵֽן׃




\ssubsection{בהעלותך}\\
וַיְדַבֵּ֥ר יְיָ֖ אֶל־מֹשֶׁ֥ה לֵּאמֹֽר׃ דַּבֵּר֙ אֶֽל־אַהֲרֹ֔ן וְאָמַרְתָּ֖ אֵלָ֑יו בְּהַעֲלֹֽתְךָ֙ אֶת־הַנֵּרֹ֔ת אֶל־מוּל֙ פְּנֵ֣י הַמְּנוֹרָ֔ה יָאִ֖ירוּ שִׁבְעַ֥ת הַנֵּרֽוֹת׃ וַיַּ֤עַשׂ כֵּן֙ אַהֲרֹ֔ן אֶל־מוּל֙ פְּנֵ֣י הַמְּנוֹרָ֔ה הֶעֱלָ֖ה נֵרֹתֶ֑יהָ כַּֽאֲשֶׁ֛ר צִוָּ֥ה יְיָ֖ אֶת־מֹשֶֽׁה׃ וְזֶ֨ה מַעֲשֵׂ֤ה הַמְּנֹרָה֙ מִקְשָׁ֣ה זָהָ֔ב עַד־יְרֵכָ֥הּ עַד־פִּרְחָ֖הּ מִקְשָׁ֣ה הִ֑וא כַּמַּרְאֶ֗ה אֲשֶׁ֨ר הֶרְאָ֤ה יְיָ֙ אֶת־מֹשֶׁ֔ה כֵּ֥ן עָשָׂ֖ה אֶת־הַמְּנֹרָֽה׃ 
\aliyah{לוי}
וַיְדַבֵּ֥ר יְיָ֖ אֶל־מֹשֶׁ֥ה לֵּאמֹֽר׃ קַ֚ח אֶת־הַלְוִיִּ֔ם מִתּ֖וֹךְ בְּנֵ֣י יִשְׂרָאֵ֑ל וְטִהַרְתָּ֖ אֹתָֽם׃ וְכֹֽה־תַעֲשֶׂ֤ה לָהֶם֙ לְטַֽהֲרָ֔ם הַזֵּ֥ה עֲלֵיהֶ֖ם מֵ֣י חַטָּ֑את וְהֶעֱבִ֤ירוּ תַ֙עַר֙ עַל־כׇּל־בְּשָׂרָ֔ם וְכִבְּס֥וּ בִגְדֵיהֶ֖ם וְהִטֶּהָֽרוּ׃ וְלָֽקְחוּ֙ פַּ֣ר בֶּן־בָּקָ֔ר וּמִ֨נְחָת֔וֹ סֹ֖לֶת בְּלוּלָ֣ה בַשָּׁ֑מֶן וּפַר־שֵׁנִ֥י בֶן־בָּקָ֖ר תִּקַּ֥ח לְחַטָּֽאת׃ וְהִקְרַבְתָּ֙ אֶת־הַלְוִיִּ֔ם לִפְנֵ֖י אֹ֣הֶל מוֹעֵ֑ד וְהִ֨קְהַלְתָּ֔ אֶֽת־כׇּל־עֲדַ֖ת בְּנֵ֥י יִשְׂרָאֵֽל׃
\aliyah{ישראל}
וְהִקְרַבְתָּ֥ אֶת־הַלְוִיִּ֖ם לִפְנֵ֣י יְיָ֑ וְסָמְכ֧וּ בְנֵי־יִשְׂרָאֵ֛ל אֶת־יְדֵיהֶ֖ם עַל־הַלְוִיִּֽם׃ וְהֵנִיף֩ אַהֲרֹ֨ן אֶת־הַלְוִיִּ֤ם תְּנוּפָה֙ לִפְנֵ֣י יְיָ֔ מֵאֵ֖ת בְּנֵ֣י יִשְׂרָאֵ֑ל וְהָי֕וּ לַעֲבֹ֖ד אֶת־עֲבֹדַ֥ת יְיָ׃ וְהַלְוִיִּם֙ יִסְמְכ֣וּ אֶת־יְדֵיהֶ֔ם עַ֖ל רֹ֣אשׁ הַפָּרִ֑ים וַ֠עֲשֵׂ֠ה אֶת־הָאֶחָ֨ד חַטָּ֜את וְאֶת־הָאֶחָ֤ד עֹלָה֙ לַֽייָ֔ לְכַפֵּ֖ר עַל־הַלְוִיִּֽם׃ וְהַֽעֲמַדְתָּ֙ אֶת־הַלְוִיִּ֔ם לִפְנֵ֥י אַהֲרֹ֖ן וְלִפְנֵ֣י בָנָ֑יו וְהֵנַפְתָּ֥ אֹתָ֛ם תְּנוּפָ֖ה לַֽייָ׃ וְהִבְדַּלְתָּ֙ אֶת־הַלְוִיִּ֔ם מִתּ֖וֹךְ בְּנֵ֣י יִשְׂרָאֵ֑ל וְהָ֥יוּ לִ֖י הַלְוִיִּֽם׃


\ssubsection{שלח־לך}\\
וַיְדַבֵּ֥ר יְיָ֖ אֶל־מֹשֶׁ֥ה לֵּאמֹֽר׃ שְׁלַח־לְךָ֣ אֲנָשִׁ֗ים וְיָתֻ֙רוּ֙ אֶת־אֶ֣רֶץ כְּנַ֔עַן אֲשֶׁר־אֲנִ֥י נֹתֵ֖ן לִבְנֵ֣י יִשְׂרָאֵ֑ל אִ֣ישׁ אֶחָד֩ אִ֨ישׁ אֶחָ֜ד לְמַטֵּ֤ה אֲבֹתָיו֙ תִּשְׁלָ֔חוּ כֹּ֖ל נָשִׂ֥יא בָהֶֽם׃ וַיִּשְׁלַ֨ח אֹתָ֥ם מֹשֶׁ֛ה מִמִּדְבַּ֥ר פָּארָ֖ן עַל־פִּ֣י יְיָ֑ כֻּלָּ֣ם אֲנָשִׁ֔ים רָאשֵׁ֥י בְנֵֽי־יִשְׂרָאֵ֖ל הֵֽמָּה׃
\aliyah{לוי}
וְאֵ֖לֶּה שְׁמוֹתָ֑ם לְמַטֵּ֣ה רְאוּבֵ֔ן שַׁמּ֖וּעַ בֶּן־זַכּֽוּר׃ לְמַטֵּ֣ה שִׁמְע֔וֹן שָׁפָ֖ט בֶּן־חוֹרִֽי׃ לְמַטֵּ֣ה יְהוּדָ֔ה כָּלֵ֖ב בֶּן־יְפֻנֶּֽה׃ לְמַטֵּ֣ה יִשָּׂשכָ֔ר יִגְאָ֖ל בֶּן־יוֹסֵֽף׃ לְמַטֵּ֥ה אֶפְרָ֖יִם הוֹשֵׁ֥עַ בִּן־נֽוּן׃ לְמַטֵּ֣ה בִנְיָמִ֔ן פַּלְטִ֖י בֶּן־רָפֽוּא׃ לְמַטֵּ֣ה זְבוּלֻ֔ן גַּדִּיאֵ֖ל בֶּן־סוֹדִֽי׃ לְמַטֵּ֥ה יוֹסֵ֖ף לְמַטֵּ֣ה מְנַשֶּׁ֑ה גַּדִּ֖י בֶּן־סוּסִֽי׃ לְמַטֵּ֣ה דָ֔ן עַמִּיאֵ֖ל בֶּן־גְּמַלִּֽי׃ לְמַטֵּ֣ה אָשֵׁ֔ר סְת֖וּר בֶּן־מִיכָאֵֽל׃ לְמַטֵּ֣ה נַפְתָּלִ֔י נַחְבִּ֖י בֶּן־וׇפְסִֽי׃ לְמַטֵּ֣ה גָ֔ד גְּאוּאֵ֖ל בֶּן־מָכִֽי׃ אֵ֚לֶּה שְׁמ֣וֹת הָֽאֲנָשִׁ֔ים אֲשֶׁר־שָׁלַ֥ח מֹשֶׁ֖ה לָת֣וּר אֶת־הָאָ֑רֶץ וַיִּקְרָ֥א מֹשֶׁ֛ה לְהוֹשֵׁ֥עַ בִּן־נ֖וּן יְהוֹשֻֽׁעַ׃
\aliyah{ישראל}
וַיִּשְׁלַ֤ח אֹתָם֙ מֹשֶׁ֔ה לָת֖וּר אֶת־אֶ֣רֶץ כְּנָ֑עַן וַיֹּ֣אמֶר אֲלֵהֶ֗ם עֲל֥וּ זֶה֙ בַּנֶּ֔גֶב וַעֲלִיתֶ֖ם אֶת־הָהָֽר׃ וּרְאִיתֶ֥ם אֶת־הָאָ֖רֶץ מַה־הִ֑וא וְאֶת־הָעָם֙ הַיֹּשֵׁ֣ב עָלֶ֔יהָ הֶחָזָ֥ק הוּא֙ הֲרָפֶ֔ה הַמְעַ֥ט ה֖וּא אִם־רָֽב׃ וּמָ֣ה הָאָ֗רֶץ אֲשֶׁר־הוּא֙ יֹשֵׁ֣ב בָּ֔הּ הֲטוֹבָ֥ה הִ֖וא אִם־רָעָ֑ה וּמָ֣ה הֶֽעָרִ֗ים אֲשֶׁר־הוּא֙ יוֹשֵׁ֣ב בָּהֵ֔נָּה הַבְּמַֽחֲנִ֖ים אִ֥ם בְּמִבְצָרִֽים׃ וּמָ֣ה הָ֠אָ֠רֶץ הַשְּׁמֵנָ֨ה הִ֜וא אִם־רָזָ֗ה הֲיֵֽשׁ־בָּ֥הּ עֵץ֙ אִם־אַ֔יִן וְהִ֨תְחַזַּקְתֶּ֔ם וּלְקַחְתֶּ֖ם מִפְּרִ֣י הָאָ֑רֶץ וְהַ֨יָּמִ֔ים יְמֵ֖י בִּכּוּרֵ֥י עֲנָבִֽים׃


\ssubsection{קרח}\\
וַיִּקַּ֣ח קֹ֔רַח בֶּן־יִצְהָ֥ר בֶּן־קְהָ֖ת בֶּן־לֵוִ֑י וְדָתָ֨ן וַאֲבִירָ֜ם בְּנֵ֧י אֱלִיאָ֛ב וְא֥וֹן בֶּן־פֶּ֖לֶת בְּנֵ֥י רְאוּבֵֽן׃ וַיָּקֻ֙מוּ֙ לִפְנֵ֣י מֹשֶׁ֔ה וַאֲנָשִׁ֥ים מִבְּנֵֽי־יִשְׂרָאֵ֖ל חֲמִשִּׁ֣ים וּמָאתָ֑יִם נְשִׂיאֵ֥י עֵדָ֛ה קְרִאֵ֥י מוֹעֵ֖ד אַנְשֵׁי־שֵֽׁם׃ וַיִּֽקָּהֲל֞וּ עַל־מֹשֶׁ֣ה וְעַֽל־אַהֲרֹ֗ן וַיֹּאמְר֣וּ אֲלֵהֶם֮ רַב־לָכֶם֒ כִּ֤י כׇל־הָֽעֵדָה֙ כֻּלָּ֣ם קְדֹשִׁ֔ים וּבְתוֹכָ֖ם יְיָ֑ וּמַדּ֥וּעַ תִּֽתְנַשְּׂא֖וּ עַל־קְהַ֥ל יְיָ׃
\aliyah{לוי}
וַיִּשְׁמַ֣ע מֹשֶׁ֔ה וַיִּפֹּ֖ל עַל־פָּנָֽיו׃ וַיְדַבֵּ֨ר אֶל־קֹ֜רַח וְאֶֽל־כׇּל־עֲדָתוֹ֮ לֵאמֹר֒ בֹּ֠קֶר וְיֹדַ֨ע יְיָ֧ אֶת־אֲשֶׁר־ל֛וֹ וְאֶת־הַקָּד֖וֹשׁ וְהִקְרִ֣יב אֵלָ֑יו וְאֵ֛ת אֲשֶׁ֥ר יִבְחַר־בּ֖וֹ יַקְרִ֥יב אֵלָֽיו׃ זֹ֖את עֲשׂ֑וּ קְחוּ־לָכֶ֣ם מַחְתּ֔וֹת קֹ֖רַח וְכׇל־עֲדָתֽוֹ׃ וּתְנ֣וּ בָהֵ֣ן ׀ אֵ֡שׁ וְשִׂ֩ימוּ֩ עֲלֵיהֶ֨ן ׀ קְטֹ֜רֶת לִפְנֵ֤י יְיָ֙ מָחָ֔ר וְהָיָ֗ה הָאִ֛ישׁ אֲשֶׁר־יִבְחַ֥ר יְיָ֖ ה֣וּא הַקָּד֑וֹשׁ רַב־לָכֶ֖ם בְּנֵ֥י לֵוִֽי׃
\aliyah{ישראל}
וַיֹּ֥אמֶר מֹשֶׁ֖ה אֶל־קֹ֑רַח שִׁמְעוּ־נָ֖א בְּנֵ֥י לֵוִֽי׃ הַמְעַ֣ט מִכֶּ֗ם כִּֽי־הִבְדִּיל֩ אֱלֹהֵ֨י יִשְׂרָאֵ֤ל אֶתְכֶם֙ מֵעֲדַ֣ת יִשְׂרָאֵ֔ל לְהַקְרִ֥יב אֶתְכֶ֖ם אֵלָ֑יו לַעֲבֹ֗ד אֶת־עֲבֹדַת֙ מִשְׁכַּ֣ן יְיָ֔ וְלַעֲמֹ֛ד לִפְנֵ֥י הָעֵדָ֖ה לְשָׁרְתָֽם׃ וַיַּקְרֵב֙ אֹֽתְךָ֔ וְאֶת־כׇּל־אַחֶ֥יךָ בְנֵי־לֵוִ֖י אִתָּ֑ךְ וּבִקַּשְׁתֶּ֖ם גַּם־כְּהֻנָּֽה׃ לָכֵ֗ן אַתָּה֙ וְכׇל־עֲדָ֣תְךָ֔ הַנֹּעָדִ֖ים עַל־יְיָ֑ וְאַהֲרֹ֣ן מַה־ה֔וּא כִּ֥י (תלונו) [תַלִּ֖ינוּ] עָלָֽיו׃ וַיִּשְׁלַ֣ח מֹשֶׁ֔ה לִקְרֹ֛א לְדָתָ֥ן וְלַאֲבִירָ֖ם בְּנֵ֣י אֱלִיאָ֑ב וַיֹּאמְר֖וּ לֹ֥א נַעֲלֶֽה׃ הַמְעַ֗ט כִּ֤י הֶֽעֱלִיתָ֙נוּ֙ מֵאֶ֨רֶץ זָבַ֤ת חָלָב֙ וּדְבַ֔שׁ לַהֲמִיתֵ֖נוּ בַּמִּדְבָּ֑ר כִּֽי־תִשְׂתָּרֵ֥ר עָלֵ֖ינוּ גַּם־הִשְׂתָּרֵֽר׃


\ssubsection{חקת}\\
וַיְדַבֵּ֣ר יְיָ֔ אֶל־מֹשֶׁ֥ה וְאֶֽל־אַהֲרֹ֖ן לֵאמֹֽר׃ זֹ֚את חֻקַּ֣ת הַתּוֹרָ֔ה אֲשֶׁר־צִוָּ֥ה יְיָ֖ לֵאמֹ֑ר דַּבֵּ֣ר ׀ אֶל־בְּנֵ֣י יִשְׂרָאֵ֗ל וְיִקְח֣וּ אֵלֶ֩יךָ֩ פָרָ֨ה אֲדֻמָּ֜ה תְּמִימָ֗ה אֲשֶׁ֤ר אֵֽין־בָּהּ֙ מ֔וּם אֲשֶׁ֛ר לֹא־עָלָ֥ה עָלֶ֖יהָ עֹֽל׃ וּנְתַתֶּ֣ם אֹתָ֔הּ אֶל־אֶלְעָזָ֖ר הַכֹּהֵ֑ן וְהוֹצִ֤יא אֹתָהּ֙ אֶל־מִח֣וּץ לַֽמַּחֲנֶ֔ה וְשָׁחַ֥ט אֹתָ֖הּ לְפָנָֽיו׃ וְלָקַ֞ח אֶלְעָזָ֧ר הַכֹּהֵ֛ן מִדָּמָ֖הּ בְּאֶצְבָּע֑וֹ וְהִזָּ֞ה אֶל־נֹ֨כַח פְּנֵ֧י אֹֽהֶל־מוֹעֵ֛ד מִדָּמָ֖הּ שֶׁ֥בַע פְּעָמִֽים׃ וְשָׂרַ֥ף אֶת־הַפָּרָ֖ה לְעֵינָ֑יו אֶת־עֹרָ֤הּ וְאֶת־בְּשָׂרָהּ֙ וְאֶת־דָּמָ֔הּ עַל־פִּרְשָׁ֖הּ יִשְׂרֹֽף׃ וְלָקַ֣ח הַכֹּהֵ֗ן עֵ֥ץ אֶ֛רֶז וְאֵז֖וֹב וּשְׁנִ֣י תוֹלָ֑עַת וְהִשְׁלִ֕יךְ אֶל־תּ֖וֹךְ שְׂרֵפַ֥ת הַפָּרָֽה׃
\aliyah{לוי}
וְכִבֶּ֨ס בְּגָדָ֜יו הַכֹּהֵ֗ן וְרָחַ֤ץ בְּשָׂרוֹ֙ בַּמַּ֔יִם וְאַחַ֖ר יָבֹ֣א אֶל־הַֽמַּחֲנֶ֑ה וְטָמֵ֥א הַכֹּהֵ֖ן עַד־הָעָֽרֶב׃ וְהַשֹּׂרֵ֣ף אֹתָ֔הּ יְכַבֵּ֤ס בְּגָדָיו֙ בַּמַּ֔יִם וְרָחַ֥ץ בְּשָׂר֖וֹ בַּמָּ֑יִם וְטָמֵ֖א עַד־הָעָֽרֶב׃ וְאָסַ֣ף ׀ אִ֣ישׁ טָה֗וֹר אֵ֚ת אֵ֣פֶר הַפָּרָ֔ה וְהִנִּ֛יחַ מִח֥וּץ לַֽמַּחֲנֶ֖ה בְּמָק֣וֹם טָה֑וֹר וְ֠הָיְתָ֠ה לַעֲדַ֨ת בְּנֵֽי־יִשְׂרָאֵ֧ל לְמִשְׁמֶ֛רֶת לְמֵ֥י נִדָּ֖ה חַטָּ֥את הִֽוא׃
\aliyah{ישראל}
וְ֠כִבֶּ֠ס הָאֹסֵ֨ף אֶת־אֵ֤פֶר הַפָּרָה֙ אֶת־בְּגָדָ֔יו וְטָמֵ֖א עַד־הָעָ֑רֶב וְֽהָיְתָ֞ה לִבְנֵ֣י יִשְׂרָאֵ֗ל וְלַגֵּ֛ר הַגָּ֥ר בְּתוֹכָ֖ם לְחֻקַּ֥ת עוֹלָֽם׃ הַנֹּגֵ֥עַ בְּמֵ֖ת לְכׇל־נֶ֣פֶשׁ אָדָ֑ם וְטָמֵ֖א שִׁבְעַ֥ת יָמִֽים׃ ה֣וּא יִתְחַטָּא־ב֞וֹ בַּיּ֧וֹם הַשְּׁלִישִׁ֛י וּבַיּ֥וֹם הַשְּׁבִיעִ֖י יִטְהָ֑ר וְאִם־לֹ֨א יִתְחַטָּ֜א בַּיּ֧וֹם הַשְּׁלִישִׁ֛י וּבַיּ֥וֹם הַשְּׁבִיעִ֖י לֹ֥א יִטְהָֽר׃ כׇּֽל־הַנֹּגֵ֡עַ בְּמֵ֣ת בְּנֶ֩פֶשׁ֩ הָאָדָ֨ם אֲשֶׁר־יָמ֜וּת וְלֹ֣א יִתְחַטָּ֗א אֶת־מִשְׁכַּ֤ן יְיָ֙ טִמֵּ֔א וְנִכְרְתָ֛ה הַנֶּ֥פֶשׁ הַהִ֖וא מִיִּשְׂרָאֵ֑ל כִּי֩ מֵ֨י נִדָּ֜ה לֹא־זֹרַ֤ק עָלָיו֙ טָמֵ֣א יִהְיֶ֔ה ע֖וֹד טֻמְאָת֥וֹ בֽוֹ׃ זֹ֚את הַתּוֹרָ֔ה אָדָ֖ם כִּֽי־יָמ֣וּת בְּאֹ֑הֶל כׇּל־הַבָּ֤א אֶל־הָאֹ֙הֶל֙ וְכׇל־אֲשֶׁ֣ר בָּאֹ֔הֶל יִטְמָ֖א שִׁבְעַ֥ת יָמִֽים׃ וְכֹל֙ כְּלִ֣י פָת֔וּחַ אֲשֶׁ֛ר אֵין־צָמִ֥יד פָּתִ֖יל עָלָ֑יו טָמֵ֖א הֽוּא׃ וְכֹ֨ל אֲשֶׁר־יִגַּ֜ע עַל־פְּנֵ֣י הַשָּׂדֶ֗ה בַּֽחֲלַל־חֶ֙רֶב֙ א֣וֹ בְמֵ֔ת אֽוֹ־בְעֶ֥צֶם אָדָ֖ם א֣וֹ בְקָ֑בֶר יִטְמָ֖א שִׁבְעַ֥ת יָמִֽים׃ וְלָֽקְחוּ֙ לַטָּמֵ֔א מֵעֲפַ֖ר שְׂרֵפַ֣ת הַֽחַטָּ֑את וְנָתַ֥ן עָלָ֛יו מַ֥יִם חַיִּ֖ים אֶל־כֶּֽלִי׃


\ssubsection{בלק}\\
וַיַּ֥רְא בָּלָ֖ק בֶּן־צִפּ֑וֹר אֵ֛ת כׇּל־אֲשֶׁר־עָשָׂ֥ה יִשְׂרָאֵ֖ל לָֽאֱמֹרִֽי׃ וַיָּ֨גׇר מוֹאָ֜ב מִפְּנֵ֥י הָעָ֛ם מְאֹ֖ד כִּ֣י רַב־ה֑וּא וַיָּ֣קׇץ מוֹאָ֔ב מִפְּנֵ֖י בְּנֵ֥י יִשְׂרָאֵֽל׃ וַיֹּ֨אמֶר מוֹאָ֜ב אֶל־זִקְנֵ֣י מִדְיָ֗ן עַתָּ֞ה יְלַחֲכ֤וּ הַקָּהָל֙ אֶת־כׇּל־סְבִ֣יבֹתֵ֔ינוּ כִּלְחֹ֣ךְ הַשּׁ֔וֹר אֵ֖ת יֶ֣רֶק הַשָּׂדֶ֑ה וּבָלָ֧ק בֶּן־צִפּ֛וֹר מֶ֥לֶךְ לְמוֹאָ֖ב בָּעֵ֥ת הַהִֽוא׃
\aliyah{לוי}
וַיִּשְׁלַ֨ח מַלְאָכִ֜ים אֶל־בִּלְעָ֣ם בֶּן־בְּע֗וֹר פְּ֠ת֠וֹרָה אֲשֶׁ֧ר עַל־הַנָּהָ֛ר אֶ֥רֶץ בְּנֵי־עַמּ֖וֹ לִקְרֹא־ל֑וֹ לֵאמֹ֗ר הִ֠נֵּ֠ה עַ֣ם יָצָ֤א מִמִּצְרַ֙יִם֙ הִנֵּ֤ה כִסָּה֙ אֶת־עֵ֣ין הָאָ֔רֶץ וְה֥וּא יֹשֵׁ֖ב מִמֻּלִֽי׃ וְעַתָּה֩ לְכָה־נָּ֨א אָֽרָה־לִּ֜י אֶת־הָעָ֣ם הַזֶּ֗ה כִּֽי־עָצ֥וּם הוּא֙ מִמֶּ֔נִּי אוּלַ֤י אוּכַל֙ נַכֶּה־בּ֔וֹ וַאֲגָרְשֶׁ֖נּוּ מִן־הָאָ֑רֶץ כִּ֣י יָדַ֗עְתִּי אֵ֤ת אֲשֶׁר־תְּבָרֵךְ֙ מְבֹרָ֔ךְ וַאֲשֶׁ֥ר תָּאֹ֖ר יוּאָֽר׃ וַיֵּ֨לְכ֜וּ זִקְנֵ֤י מוֹאָב֙ וְזִקְנֵ֣י מִדְיָ֔ן וּקְסָמִ֖ים בְּיָדָ֑ם וַיָּבֹ֙אוּ֙ אֶל־בִּלְעָ֔ם וַיְדַבְּר֥וּ אֵלָ֖יו דִּבְרֵ֥י בָלָֽק׃
\aliyah{ישראל}
וַיֹּ֣אמֶר אֲלֵיהֶ֗ם לִ֤ינוּ פֹה֙ הַלַּ֔יְלָה וַהֲשִׁבֹתִ֤י אֶתְכֶם֙ דָּבָ֔ר כַּאֲשֶׁ֛ר יְדַבֵּ֥ר יְיָ֖ אֵלָ֑י וַיֵּשְׁב֥וּ שָׂרֵֽי־מוֹאָ֖ב עִם־בִּלְעָֽם׃ וַיָּבֹ֥א אֱלֹהִ֖ים אֶל־בִּלְעָ֑ם וַיֹּ֕אמֶר מִ֛י הָאֲנָשִׁ֥ים הָאֵ֖לֶּה עִמָּֽךְ׃ וַיֹּ֥אמֶר בִּלְעָ֖ם אֶל־הָאֱלֹהִ֑ים בָּלָ֧ק בֶּן־צִפֹּ֛ר מֶ֥לֶךְ מוֹאָ֖ב שָׁלַ֥ח אֵלָֽי׃ הִנֵּ֤ה הָעָם֙ הַיֹּצֵ֣א מִמִּצְרַ֔יִם וַיְכַ֖ס אֶת־עֵ֣ין הָאָ֑רֶץ עַתָּ֗ה לְכָ֤ה קָֽבָה־לִּי֙ אֹת֔וֹ אוּלַ֥י אוּכַ֛ל לְהִלָּ֥חֶם בּ֖וֹ וְגֵרַשְׁתִּֽיו׃ וַיֹּ֤אמֶר אֱלֹהִים֙ אֶל־בִּלְעָ֔ם לֹ֥א תֵלֵ֖ךְ עִמָּהֶ֑ם לֹ֤א תָאֹר֙ אֶת־הָעָ֔ם כִּ֥י בָר֖וּךְ הֽוּא׃


\ssubsection{פינחס}\\
וַיְדַבֵּ֥ר יְיָ֖ אֶל־מֹשֶׁ֥ה לֵּאמֹֽר׃ פִּֽינְחָ֨ס בֶּן־אֶלְעָזָ֜ר בֶּן־אַהֲרֹ֣ן הַכֹּהֵ֗ן הֵשִׁ֤יב אֶת־חֲמָתִי֙ מֵעַ֣ל בְּנֵֽי־יִשְׂרָאֵ֔ל בְּקַנְא֥וֹ אֶת־קִנְאָתִ֖י בְּתוֹכָ֑ם וְלֹא־כִלִּ֥יתִי אֶת־בְּנֵֽי־יִשְׂרָאֵ֖ל בְּקִנְאָתִֽי׃ לָכֵ֖ן אֱמֹ֑ר הִנְנִ֨י נֹתֵ֥ן ל֛וֹ אֶת־בְּרִיתִ֖י שָׁלֽוֹם׃*
\aliyah{לוי}
וְהָ֤יְתָה לּוֹ֙ וּלְזַרְע֣וֹ אַחֲרָ֔יו בְּרִ֖ית כְּהֻנַּ֣ת עוֹלָ֑ם תַּ֗חַת אֲשֶׁ֤ר קִנֵּא֙ לֵֽאלֹהָ֔יו וַיְכַפֵּ֖ר עַל־בְּנֵ֥י יִשְׂרָאֵֽל׃ וְשֵׁם֩ אִ֨ישׁ יִשְׂרָאֵ֜ל הַמֻּכֶּ֗ה אֲשֶׁ֤ר הֻכָּה֙ אֶת־הַמִּדְיָנִ֔ית זִמְרִ֖י בֶּן־סָל֑וּא נְשִׂ֥יא בֵֽית־אָ֖ב לַשִּׁמְעֹנִֽי׃ וְשֵׁ֨ם הָֽאִשָּׁ֧ה הַמֻּכָּ֛ה הַמִּדְיָנִ֖ית כׇּזְבִּ֣י בַת־צ֑וּר רֹ֣אשׁ אֻמּ֥וֹת בֵּֽית־אָ֛ב בְּמִדְיָ֖ן הֽוּא׃ 
\aliyah{ישראל}
וַיְדַבֵּ֥ר יְיָ֖ אֶל־מֹשֶׁ֥ה לֵּאמֹֽר׃ צָר֖וֹר אֶת־הַמִּדְיָנִ֑ים וְהִכִּיתֶ֖ם אוֹתָֽם׃ כִּ֣י צֹרְרִ֥ים הֵם֙ לָכֶ֔ם בְּנִכְלֵיהֶ֛ם אֲשֶׁר־נִכְּל֥וּ לָכֶ֖ם עַל־דְּבַר־פְּע֑וֹר וְעַל־דְּבַ֞ר כׇּזְבִּ֨י בַת־נְשִׂ֤יא מִדְיָן֙ אֲחֹתָ֔ם הַמֻּכָּ֥ה בְיוֹם־הַמַּגֵּפָ֖ה עַל־דְּבַר־פְּעֽוֹר׃ וַיְהִ֖י אַחֲרֵ֣י הַמַּגֵּפָ֑ה {פ} וַיֹּ֤אמֶר יְיָ֙ אֶל־מֹשֶׁ֔ה וְאֶ֧ל אֶלְעָזָ֛ר בֶּן־אַהֲרֹ֥ן הַכֹּהֵ֖ן לֵאמֹֽר׃ שְׂא֞וּ אֶת־רֹ֣אשׁ ׀ כׇּל־עֲדַ֣ת בְּנֵי־יִשְׂרָאֵ֗ל מִבֶּ֨ן עֶשְׂרִ֥ים שָׁנָ֛ה וָמַ֖עְלָה לְבֵ֣ית אֲבֹתָ֑ם כׇּל־יֹצֵ֥א צָבָ֖א בְּיִשְׂרָאֵֽל׃ וַיְדַבֵּ֨ר מֹשֶׁ֜ה וְאֶלְעָזָ֧ר הַכֹּהֵ֛ן אֹתָ֖ם בְּעַֽרְבֹ֣ת מוֹאָ֑ב עַל־יַרְדֵּ֥ן יְרֵח֖וֹ לֵאמֹֽר׃ מִבֶּ֛ן עֶשְׂרִ֥ים שָׁנָ֖ה וָמָ֑עְלָה כַּאֲשֶׁר֩ צִוָּ֨ה יְיָ֤ אֶת־מֹשֶׁה֙ וּבְנֵ֣י יִשְׂרָאֵ֔ל הַיֹּצְאִ֖ים מֵאֶ֥רֶץ מִצְרָֽיִם׃


\ssubsection{מטות}\\
וַיְדַבֵּ֤ר מֹשֶׁה֙ אֶל־רָאשֵׁ֣י הַמַּטּ֔וֹת לִבְנֵ֥י יִשְׂרָאֵ֖ל לֵאמֹ֑ר זֶ֣ה הַדָּבָ֔ר אֲשֶׁ֖ר צִוָּ֥ה יְיָ׃ אִישׁ֩ כִּֽי־יִדֹּ֨ר נֶ֜דֶר לַֽייָ֗ אֽוֹ־הִשָּׁ֤בַע שְׁבֻעָה֙ לֶאְסֹ֤ר אִסָּר֙ עַל־נַפְשׁ֔וֹ לֹ֥א יַחֵ֖ל דְּבָר֑וֹ כְּכׇל־הַיֹּצֵ֥א מִפִּ֖יו יַעֲשֶֽׂה׃ וְאִשָּׁ֕ה כִּֽי־תִדֹּ֥ר נֶ֖דֶר לַייָ֑ וְאָסְרָ֥ה אִסָּ֛ר בְּבֵ֥ית אָבִ֖יהָ בִּנְעֻרֶֽיהָ׃ וְשָׁמַ֨ע אָבִ֜יהָ אֶת־נִדְרָ֗הּ וֶֽאֱסָרָהּ֙ אֲשֶׁ֣ר אָֽסְרָ֣ה עַל־נַפְשָׁ֔הּ וְהֶחֱרִ֥ישׁ לָ֖הּ אָבִ֑יהָ וְקָ֙מוּ֙ כׇּל־נְדָרֶ֔יהָ וְכׇל־אִסָּ֛ר אֲשֶׁר־אָסְרָ֥ה עַל־נַפְשָׁ֖הּ יָקֽוּם׃ וְאִם־הֵנִ֨יא אָבִ֣יהָ אֹתָהּ֮ בְּי֣וֹם שׇׁמְעוֹ֒ כׇּל־נְדָרֶ֗יהָ וֶֽאֱסָרֶ֛יהָ אֲשֶׁר־אָסְרָ֥ה עַל־נַפְשָׁ֖הּ לֹ֣א יָק֑וּם וַֽייָ֙ יִֽסְלַח־לָ֔הּ כִּי־הֵנִ֥יא אָבִ֖יהָ אֹתָֽהּ׃ וְאִם־הָי֤וֹ תִֽהְיֶה֙ לְאִ֔ישׁ וּנְדָרֶ֖יהָ עָלֶ֑יהָ א֚וֹ מִבְטָ֣א שְׂפָתֶ֔יהָ אֲשֶׁ֥ר אָסְרָ֖ה עַל־נַפְשָֽׁהּ׃ וְשָׁמַ֥ע אִישָׁ֛הּ בְּי֥וֹם שׇׁמְע֖וֹ וְהֶחֱרִ֣ישׁ לָ֑הּ וְקָ֣מוּ נְדָרֶ֗יהָ וֶֽאֱסָרֶ֛הָ אֲשֶׁר־אָסְרָ֥ה עַל־נַפְשָׁ֖הּ יָקֻֽמוּ׃ וְ֠אִ֠ם בְּי֨וֹם שְׁמֹ֣עַ אִישָׁהּ֮ יָנִ֣יא אוֹתָהּ֒ וְהֵפֵ֗ר אֶת־נִדְרָהּ֙ אֲשֶׁ֣ר עָלֶ֔יהָ וְאֵת֙ מִבְטָ֣א שְׂפָתֶ֔יהָ אֲשֶׁ֥ר אָסְרָ֖ה עַל־נַפְשָׁ֑הּ וַייָ֖ יִֽסְלַֽח־לָֽהּ׃
\aliyah{לוי}
וְנֵ֥דֶר אַלְמָנָ֖ה וּגְרוּשָׁ֑ה כֹּ֛ל אֲשֶׁר־אָסְרָ֥ה עַל־נַפְשָׁ֖הּ יָק֥וּם עָלֶֽיהָ׃ וְאִם־בֵּ֥ית אִישָׁ֖הּ נָדָ֑רָה אֽוֹ־אָסְרָ֥ה אִסָּ֛ר עַל־נַפְשָׁ֖הּ בִּשְׁבֻעָֽה׃ וְשָׁמַ֤ע אִישָׁהּ֙ וְהֶחֱרִ֣שׁ לָ֔הּ לֹ֥א הֵנִ֖יא אֹתָ֑הּ וְקָ֙מוּ֙ כׇּל־נְדָרֶ֔יהָ וְכׇל־אִסָּ֛ר אֲשֶׁר־אָסְרָ֥ה עַל־נַפְשָׁ֖הּ יָקֽוּם׃ וְאִם־הָפֵר֩ יָפֵ֨ר אֹתָ֥ם ׀ אִישָׁהּ֮ בְּי֣וֹם שׇׁמְעוֹ֒ כׇּל־מוֹצָ֨א שְׂפָתֶ֧יהָ לִנְדָרֶ֛יהָ וּלְאִסַּ֥ר נַפְשָׁ֖הּ לֹ֣א יָק֑וּם אִישָׁ֣הּ הֲפֵרָ֔ם וַייָ֖ יִֽסְלַֽח־לָֽהּ׃
\aliyah{ישראל}
כׇּל־נֵ֛דֶר וְכׇל־שְׁבֻעַ֥ת אִסָּ֖ר לְעַנֹּ֣ת נָ֑פֶשׁ אִישָׁ֥הּ יְקִימֶ֖נּוּ וְאִישָׁ֥הּ יְפֵרֶֽנּוּ׃ וְאִם־הַחֲרֵשׁ֩ יַחֲרִ֨ישׁ לָ֥הּ אִישָׁהּ֮ מִיּ֣וֹם אֶל־יוֹם֒ וְהֵקִים֙ אֶת־כׇּל־נְדָרֶ֔יהָ א֥וֹ אֶת־כׇּל־אֱסָרֶ֖יהָ אֲשֶׁ֣ר עָלֶ֑יהָ הֵקִ֣ים אֹתָ֔ם כִּי־הֶחֱרִ֥שׁ לָ֖הּ בְּי֥וֹם שׇׁמְעֽוֹ׃ וְאִם־הָפֵ֥ר יָפֵ֛ר אֹתָ֖ם אַחֲרֵ֣י שׇׁמְע֑וֹ וְנָשָׂ֖א אֶת־עֲוֺנָֽהּ׃ אֵ֣לֶּה הַֽחֻקִּ֗ים אֲשֶׁ֨ר צִוָּ֤ה יְיָ֙ אֶת־מֹשֶׁ֔ה בֵּ֥ין אִ֖ישׁ לְאִשְׁתּ֑וֹ בֵּֽין־אָ֣ב לְבִתּ֔וֹ בִּנְעֻרֶ֖יהָ בֵּ֥ית אָבִֽיהָ׃ 


\ssubsection{מסעי}\\
אֵ֜לֶּה מַסְעֵ֣י בְנֵֽי־יִשְׂרָאֵ֗ל אֲשֶׁ֥ר יָצְא֛וּ מֵאֶ֥רֶץ מִצְרַ֖יִם לְצִבְאֹתָ֑ם בְּיַד־מֹשֶׁ֖ה וְאַהֲרֹֽן׃ וַיִּכְתֹּ֨ב מֹשֶׁ֜ה אֶת־מוֹצָאֵיהֶ֛ם לְמַסְעֵיהֶ֖ם עַל־פִּ֣י יְיָ֑ וְאֵ֥לֶּה מַסְעֵיהֶ֖ם לְמוֹצָאֵיהֶֽם׃ וַיִּסְע֤וּ מֵֽרַעְמְסֵס֙ בַּחֹ֣דֶשׁ הָֽרִאשׁ֔וֹן בַּחֲמִשָּׁ֥ה עָשָׂ֛ר י֖וֹם לַחֹ֣דֶשׁ הָרִאשׁ֑וֹן מִֽמׇּחֳרַ֣ת הַפֶּ֗סַח יָצְא֤וּ בְנֵֽי־יִשְׂרָאֵל֙ בְּיָ֣ד רָמָ֔ה לְעֵינֵ֖י כׇּל־מִצְרָֽיִם׃
\aliyah{לוי}
וּמִצְרַ֣יִם מְקַבְּרִ֗ים אֵת֩ אֲשֶׁ֨ר הִכָּ֧ה יְיָ֛ בָּהֶ֖ם כׇּל־בְּכ֑וֹר וּבֵאלֹ֣הֵיהֶ֔ם עָשָׂ֥ה יְיָ֖ שְׁפָטִֽים׃ וַיִּסְע֥וּ בְנֵֽי־יִשְׂרָאֵ֖ל מֵרַעְמְסֵ֑ס וַֽיַּחֲנ֖וּ בְּסֻכֹּֽת׃ וַיִּסְע֖וּ מִסֻּכֹּ֑ת וַיַּחֲנ֣וּ בְאֵתָ֔ם אֲשֶׁ֖ר בִּקְצֵ֥ה הַמִּדְבָּֽר׃
\aliyah{ישראל}
וַיִּסְעוּ֙ מֵֽאֵתָ֔ם וַיָּ֙שׇׁב֙ עַל־פִּ֣י הַחִירֹ֔ת אֲשֶׁ֥ר עַל־פְּנֵ֖י בַּ֣עַל צְפ֑וֹן וַֽיַּחֲנ֖וּ לִפְנֵ֥י מִגְדֹּֽל׃ וַיִּסְעוּ֙ מִפְּנֵ֣י הַֽחִירֹ֔ת וַיַּֽעַבְר֥וּ בְתוֹךְ־הַיָּ֖ם הַמִּדְבָּ֑רָה וַיֵּ֨לְכ֜וּ דֶּ֣רֶךְ שְׁלֹ֤שֶׁת יָמִים֙ בְּמִדְבַּ֣ר אֵתָ֔ם וַֽיַּחֲנ֖וּ בְּמָרָֽה׃ וַיִּסְעוּ֙ מִמָּרָ֔ה וַיָּבֹ֖אוּ אֵילִ֑מָה וּ֠בְאֵילִ֠ם שְׁתֵּ֣ים עֶשְׂרֵ֞ה עֵינֹ֥ת מַ֛יִם וְשִׁבְעִ֥ים תְּמָרִ֖ים וַיַּחֲנוּ־שָֽׁם׃ וַיִּסְע֖וּ מֵאֵילִ֑ם וַֽיַּחֲנ֖וּ עַל־יַם־סֽוּף׃


\ssubsection{דברים}\\
אֵ֣לֶּה הַדְּבָרִ֗ים אֲשֶׁ֨ר דִּבֶּ֤ר מֹשֶׁה֙ אֶל־כׇּל־יִשְׂרָאֵ֔ל בְּעֵ֖בֶר הַיַּרְדֵּ֑ן בַּמִּדְבָּ֡ר בָּֽעֲרָבָה֩ מ֨וֹל ס֜וּף בֵּֽין־פָּארָ֧ן וּבֵֽין־תֹּ֛פֶל וְלָבָ֥ן וַחֲצֵרֹ֖ת וְדִ֥י זָהָֽב׃ אַחַ֨ד עָשָׂ֥ר יוֹם֙ מֵֽחֹרֵ֔ב דֶּ֖רֶךְ הַר־שֵׂעִ֑יר עַ֖ד קָדֵ֥שׁ בַּרְנֵֽעַ׃ וַיְהִי֙ בְּאַרְבָּעִ֣ים שָׁנָ֔ה בְּעַשְׁתֵּֽי־עָשָׂ֥ר חֹ֖דֶשׁ בְּאֶחָ֣ד לַחֹ֑דֶשׁ דִּבֶּ֤ר מֹשֶׁה֙ אֶל־בְּנֵ֣י יִשְׂרָאֵ֔ל כְּ֠כֹ֠ל אֲשֶׁ֨ר צִוָּ֧ה יְיָ֛ אֹת֖וֹ אֲלֵהֶֽם׃
\aliyah{לוי}
אַחֲרֵ֣י הַכֹּת֗וֹ אֵ֚ת סִיחֹן֙ מֶ֣לֶךְ הָֽאֱמֹרִ֔י אֲשֶׁ֥ר יוֹשֵׁ֖ב בְּחֶשְׁבּ֑וֹן וְאֵ֗ת ע֚וֹג מֶ֣לֶךְ הַבָּשָׁ֔ן אֲשֶׁר־יוֹשֵׁ֥ב בְּעַשְׁתָּרֹ֖ת בְּאֶדְרֶֽעִי׃ בְּעֵ֥בֶר הַיַּרְדֵּ֖ן בְּאֶ֣רֶץ מוֹאָ֑ב הוֹאִ֣יל מֹשֶׁ֔ה בֵּאֵ֛ר אֶת־הַתּוֹרָ֥ה הַזֹּ֖את לֵאמֹֽר׃ יְיָ֧ אֱלֹהֵ֛ינוּ דִּבֶּ֥ר אֵלֵ֖ינוּ בְּחֹרֵ֣ב לֵאמֹ֑ר רַב־לָכֶ֥ם שֶׁ֖בֶת בָּהָ֥ר הַזֶּֽה׃ פְּנ֣וּ ׀ וּסְע֣וּ לָכֶ֗ם וּבֹ֨אוּ הַ֥ר הָֽאֱמֹרִי֮ וְאֶל־כׇּל־שְׁכֵנָיו֒ בָּעֲרָבָ֥ה בָהָ֛ר וּבַשְּׁפֵלָ֥ה וּבַנֶּ֖גֶב וּבְח֣וֹף הַיָּ֑ם אֶ֤רֶץ הַֽכְּנַעֲנִי֙ וְהַלְּבָנ֔וֹן עַד־הַנָּהָ֥ר הַגָּדֹ֖ל נְהַר־פְּרָֽת׃
\aliyah{ישראל}
רְאֵ֛ה נָתַ֥תִּי לִפְנֵיכֶ֖ם אֶת־הָאָ֑רֶץ בֹּ֚אוּ וּרְשׁ֣וּ אֶת־הָאָ֔רֶץ אֲשֶׁ֣ר נִשְׁבַּ֣ע יְ֠יָ֠ לַאֲבֹ֨תֵיכֶ֜ם לְאַבְרָהָ֨ם לְיִצְחָ֤ק וּֽלְיַעֲקֹב֙ לָתֵ֣ת לָהֶ֔ם וּלְזַרְעָ֖ם אַחֲרֵיהֶֽם׃ וָאֹמַ֣ר אֲלֵכֶ֔ם בָּעֵ֥ת הַהִ֖וא לֵאמֹ֑ר לֹא־אוּכַ֥ל לְבַדִּ֖י שְׂאֵ֥ת אֶתְכֶֽם׃ יְיָ֥ אֱלֹהֵיכֶ֖ם הִרְבָּ֣ה אֶתְכֶ֑ם וְהִנְּכֶ֣ם הַיּ֔וֹם כְּכוֹכְבֵ֥י הַשָּׁמַ֖יִם לָרֹֽב׃ יְיָ֞ אֱלֹהֵ֣י אֲבֽוֹתֵכֶ֗ם יֹסֵ֧ף עֲלֵיכֶ֛ם כָּכֶ֖ם אֶ֣לֶף פְּעָמִ֑ים וִיבָרֵ֣ךְ אֶתְכֶ֔ם כַּאֲשֶׁ֖ר דִּבֶּ֥ר לָכֶֽם׃


\ssubsection{ואתחנן}\\
וָאֶתְחַנַּ֖ן אֶל־יְיָ֑ בָּעֵ֥ת הַהִ֖וא לֵאמֹֽר׃ אֲדֹנָ֣י יְיָ֗ אַתָּ֤ה הַֽחִלּ֙וֹתָ֙ לְהַרְא֣וֹת אֶֽת־עַבְדְּךָ֔ אֶ֨ת־גׇּדְלְךָ֔ וְאֶת־יָדְךָ֖ הַחֲזָקָ֑ה אֲשֶׁ֤ר מִי־אֵל֙ בַּשָּׁמַ֣יִם וּבָאָ֔רֶץ אֲשֶׁר־יַעֲשֶׂ֥ה כְמַעֲשֶׂ֖יךָ וְכִגְבוּרֹתֶֽךָ׃ אֶעְבְּרָה־נָּ֗א וְאֶרְאֶה֙ אֶת־הָאָ֣רֶץ הַטּוֹבָ֔ה אֲשֶׁ֖ר בְּעֵ֣בֶר הַיַּרְדֵּ֑ן הָהָ֥ר הַטּ֛וֹב הַזֶּ֖ה וְהַלְּבָנֹֽן׃
\aliyah{לוי}
וְעַתָּ֣ה יִשְׂרָאֵ֗ל שְׁמַ֤ע אֶל־הַֽחֻקִּים֙ וְאֶל־הַמִּשְׁפָּטִ֔ים אֲשֶׁ֧ר אָֽנֹכִ֛י מְלַמֵּ֥ד אֶתְכֶ֖ם לַעֲשׂ֑וֹת לְמַ֣עַן תִּֽחְי֗וּ וּבָאתֶם֙ וִֽירִשְׁתֶּ֣ם אֶת־הָאָ֔רֶץ אֲשֶׁ֧ר יְיָ֛ אֱלֹהֵ֥י אֲבֹתֵיכֶ֖ם נֹתֵ֥ן לָכֶֽם׃ לֹ֣א תֹסִ֗פוּ עַל־הַדָּבָר֙ אֲשֶׁ֤ר אָנֹכִי֙ מְצַוֶּ֣ה אֶתְכֶ֔ם וְלֹ֥א תִגְרְע֖וּ מִמֶּ֑נּוּ לִשְׁמֹ֗ר אֶת־מִצְוֺת֙ יְיָ֣ אֱלֹֽהֵיכֶ֔ם אֲשֶׁ֥ר אָנֹכִ֖י מְצַוֶּ֥ה אֶתְכֶֽם׃ עֵֽינֵיכֶם֙ הָֽרֹא֔וֹת אֵ֛ת אֲשֶׁר־עָשָׂ֥ה יְיָ֖ בְּבַ֣עַל פְּע֑וֹר כִּ֣י כׇל־הָאִ֗ישׁ אֲשֶׁ֤ר הָלַךְ֙ אַחֲרֵ֣י בַֽעַל־פְּע֔וֹר הִשְׁמִיד֛וֹ יְיָ֥ אֱלֹהֶ֖יךָ מִקִּרְבֶּֽךָ׃ וְאַתֶּם֙ הַדְּבֵקִ֔ים בַּייָ֖ אֱלֹהֵיכֶ֑ם חַיִּ֥ים כֻּלְּכֶ֖ם הַיּֽוֹם׃
\aliyah{ישראל}
רְאֵ֣ה ׀ לִמַּ֣דְתִּי אֶתְכֶ֗ם חֻקִּים֙ וּמִשְׁפָּטִ֔ים כַּאֲשֶׁ֥ר צִוַּ֖נִי יְיָ֣ אֱלֹהָ֑י לַעֲשׂ֣וֹת כֵּ֔ן בְּקֶ֣רֶב הָאָ֔רֶץ אֲשֶׁ֥ר אַתֶּ֛ם בָּאִ֥ים שָׁ֖מָּה לְרִשְׁתָּֽהּ׃ וּשְׁמַרְתֶּם֮ וַעֲשִׂיתֶם֒ כִּ֣י הִ֤וא חׇכְמַתְכֶם֙ וּבִ֣ינַתְכֶ֔ם לְעֵינֵ֖י הָעַמִּ֑ים אֲשֶׁ֣ר יִשְׁמְע֗וּן אֵ֚ת כׇּל־הַחֻקִּ֣ים הָאֵ֔לֶּה וְאָמְר֗וּ רַ֚ק עַם־חָכָ֣ם וְנָב֔וֹן הַגּ֥וֹי הַגָּד֖וֹל הַזֶּֽה׃ כִּ֚י מִי־ג֣וֹי גָּד֔וֹל אֲשֶׁר־ל֥וֹ אֱלֹהִ֖ים קְרֹבִ֣ים אֵלָ֑יו כַּייָ֣ אֱלֹהֵ֔ינוּ בְּכׇל־קׇרְאֵ֖נוּ אֵלָֽיו׃ וּמִי֙ גּ֣וֹי גָּד֔וֹל אֲשֶׁר־ל֛וֹ חֻקִּ֥ים וּמִשְׁפָּטִ֖ים צַדִּיקִ֑ם כְּכֹל֙ הַתּוֹרָ֣ה הַזֹּ֔את אֲשֶׁ֧ר אָנֹכִ֛י נֹתֵ֥ן לִפְנֵיכֶ֖ם הַיּֽוֹם׃


\ssubsection{עקב}\\
וְהָיָ֣ה ׀ עֵ֣קֶב תִּשְׁמְע֗וּן אֵ֤ת הַמִּשְׁפָּטִים֙ הָאֵ֔לֶּה וּשְׁמַרְתֶּ֥ם וַעֲשִׂיתֶ֖ם אֹתָ֑ם וְשָׁמַר֩ יְיָ֨ אֱלֹהֶ֜יךָ לְךָ֗ אֶֽת־הַבְּרִית֙ וְאֶת־הַחֶ֔סֶד אֲשֶׁ֥ר נִשְׁבַּ֖ע לַאֲבֹתֶֽיךָ׃ וַאֲהֵ֣בְךָ֔ וּבֵרַכְךָ֖ וְהִרְבֶּ֑ךָ וּבֵרַ֣ךְ פְּרִֽי־בִטְנְךָ֣ וּפְרִֽי־אַ֠דְמָתֶ֠ךָ דְּגָ֨נְךָ֜ וְתִירֹֽשְׁךָ֣ וְיִצְהָרֶ֗ךָ שְׁגַר־אֲלָפֶ֙יךָ֙ וְעַשְׁתְּרֹ֣ת צֹאנֶ֔ךָ עַ֚ל הָֽאֲדָמָ֔ה אֲשֶׁר־נִשְׁבַּ֥ע לַאֲבֹתֶ֖יךָ לָ֥תֶת לָֽךְ׃ בָּר֥וּךְ תִּֽהְיֶ֖ה מִכׇּל־הָעַמִּ֑ים לֹא־יִהְיֶ֥ה בְךָ֛ עָקָ֥ר וַֽעֲקָרָ֖ה וּבִבְהֶמְתֶּֽךָ׃ וְהֵסִ֧יר יְיָ֛ מִמְּךָ֖ כׇּל־חֹ֑לִי וְכׇל־מַדְוֵי֩ מִצְרַ֨יִם הָרָעִ֜ים אֲשֶׁ֣ר יָדַ֗עְתָּ לֹ֤א יְשִׂימָם֙ בָּ֔ךְ וּנְתָנָ֖ם בְּכׇל־שֹׂנְאֶֽיךָ׃ וְאָכַלְתָּ֣ אֶת־כׇּל־הָֽעַמִּ֗ים אֲשֶׁ֨ר יְיָ֤ אֱלֹהֶ֙יךָ֙ נֹתֵ֣ן לָ֔ךְ לֹא־תָח֥וֹס עֵֽינְךָ֖ עֲלֵיהֶ֑ם וְלֹ֤א תַעֲבֹד֙ אֶת־אֱלֹ֣הֵיהֶ֔ם כִּֽי־מוֹקֵ֥שׁ ה֖וּא לָֽךְ׃ {ס} כִּ֤י תֹאמַר֙ בִּלְבָ֣בְךָ֔ רַבִּ֛ים הַגּוֹיִ֥ם הָאֵ֖לֶּה מִמֶּ֑נִּי אֵיכָ֥ה אוּכַ֖ל לְהוֹרִישָֽׁם׃ לֹ֥א תִירָ֖א מֵהֶ֑ם זָכֹ֣ר תִּזְכֹּ֗ר אֵ֤ת אֲשֶׁר־עָשָׂה֙ יְיָ֣ אֱלֹהֶ֔יךָ לְפַרְעֹ֖ה וּלְכׇל־מִצְרָֽיִם׃ הַמַּסֹּ֨ת הַגְּדֹלֹ֜ת אֲשֶׁר־רָא֣וּ עֵינֶ֗יךָ וְהָאֹתֹ֤ת וְהַמֹּֽפְתִים֙ וְהַיָּ֤ד הַחֲזָקָה֙ וְהַזְּרֹ֣עַ הַנְּטוּיָ֔ה אֲשֶׁ֥ר הוֹצִֽאֲךָ֖ יְיָ֣ אֱלֹהֶ֑יךָ כֵּֽן־יַעֲשֶׂ֞ה יְיָ֤ אֱלֹהֶ֙יךָ֙ לְכׇל־הָ֣עַמִּ֔ים אֲשֶׁר־אַתָּ֥ה יָרֵ֖א מִפְּנֵיהֶֽם׃ וְגַם֙ אֶת־הַצִּרְעָ֔ה יְשַׁלַּ֛ח יְיָ֥ אֱלֹהֶ֖יךָ בָּ֑ם עַד־אֲבֹ֗ד הַנִּשְׁאָרִ֛ים וְהַנִּסְתָּרִ֖ים מִפָּנֶֽיךָ׃ לֹ֥א תַעֲרֹ֖ץ מִפְּנֵיהֶ֑ם כִּֽי־יְיָ֤ אֱלֹהֶ֙יךָ֙ בְּקִרְבֶּ֔ךָ אֵ֥ל גָּד֖וֹל וְנוֹרָֽא׃
\aliyah{לוי}
וְנָשַׁל֩ יְיָ֨ אֱלֹהֶ֜יךָ אֶת־הַגּוֹיִ֥ם הָאֵ֛ל מִפָּנֶ֖יךָ מְעַ֣ט מְעָ֑ט לֹ֤א תוּכַל֙ כַּלֹּתָ֣ם מַהֵ֔ר פֶּן־תִּרְבֶּ֥ה עָלֶ֖יךָ חַיַּ֥ת הַשָּׂדֶֽה׃ וּנְתָנָ֛ם יְיָ֥ אֱלֹהֶ֖יךָ לְפָנֶ֑יךָ וְהָמָם֙ מְהוּמָ֣ה גְדֹלָ֔ה עַ֖ד הִשָּׁמְדָֽם׃ וְנָתַ֤ן מַלְכֵיהֶם֙ בְּיָדֶ֔ךָ וְהַאֲבַדְתָּ֣ אֶת־שְׁמָ֔ם מִתַּ֖חַת הַשָּׁמָ֑יִם לֹֽא־יִתְיַצֵּ֥ב אִישׁ֙ בְּפָנֶ֔יךָ עַ֥ד הִשְׁמִֽדְךָ֖ אֹתָֽם׃ פְּסִילֵ֥י אֱלֹהֵיהֶ֖ם תִּשְׂרְפ֣וּן בָּאֵ֑שׁ לֹֽא־תַחְמֹד֩ כֶּ֨סֶף וְזָהָ֤ב עֲלֵיהֶם֙ וְלָקַחְתָּ֣ לָ֔ךְ פֶּ֚ן תִּוָּקֵ֣שׁ בּ֔וֹ כִּ֧י תוֹעֲבַ֛ת יְיָ֥ אֱלֹהֶ֖יךָ הֽוּא׃ וְלֹא־תָבִ֤יא תֽוֹעֵבָה֙ אֶל־בֵּיתֶ֔ךָ וְהָיִ֥יתָ חֵ֖רֶם כָּמֹ֑הוּ שַׁקֵּ֧ץ ׀ תְּשַׁקְּצֶ֛נּוּ וְתַעֵ֥ב ׀ תְּֽתַעֲבֶ֖נּוּ כִּי־חֵ֥רֶם הֽוּא׃ {פ} כׇּל־הַמִּצְוָ֗ה אֲשֶׁ֨ר אָנֹכִ֧י מְצַוְּךָ֛ הַיּ֖וֹם תִּשְׁמְר֣וּן לַעֲשׂ֑וֹת לְמַ֨עַן תִּֽחְי֜וּן וּרְבִיתֶ֗ם וּבָאתֶם֙ וִֽירִשְׁתֶּ֣ם אֶת־הָאָ֔רֶץ אֲשֶׁר־נִשְׁבַּ֥ע יְיָ֖ לַאֲבֹתֵיכֶֽם׃ וְזָכַרְתָּ֣ אֶת־כׇּל־הַדֶּ֗רֶךְ אֲשֶׁ֨ר הוֹלִֽיכְךָ֜ יְיָ֧ אֱלֹהֶ֛יךָ זֶ֛ה אַרְבָּעִ֥ים שָׁנָ֖ה בַּמִּדְבָּ֑ר לְמַ֨עַן עַנֹּֽתְךָ֜ לְנַסֹּֽתְךָ֗ לָדַ֜עַת אֶת־אֲשֶׁ֧ר בִּֽלְבָבְךָ֛ הֲתִשְׁמֹ֥ר מִצְוֺתָ֖ו אִם־לֹֽא׃ וַֽיְעַנְּךָ֮ וַיַּרְעִבֶ֒ךָ֒ וַיַּאֲכִֽלְךָ֤ אֶת־הַמָּן֙ אֲשֶׁ֣ר לֹא־יָדַ֔עְתָּ וְלֹ֥א יָדְע֖וּן אֲבֹתֶ֑יךָ לְמַ֣עַן הוֹדִֽיעֲךָ֗ כִּ֠י לֹ֣א עַל־הַלֶּ֤חֶם לְבַדּוֹ֙ יִחְיֶ֣ה הָֽאָדָ֔ם כִּ֛י עַל־כׇּל־מוֹצָ֥א פִֽי־יְיָ֖ יִחְיֶ֥ה הָאָדָֽם׃
\aliyah{ישראל}
שִׂמְלָ֨תְךָ֜ לֹ֤א בָֽלְתָה֙ מֵֽעָלֶ֔יךָ וְרַגְלְךָ֖ לֹ֣א בָצֵ֑קָה זֶ֖ה אַרְבָּעִ֥ים שָׁנָֽה׃ וְיָדַעְתָּ֖ עִם־לְבָבֶ֑ךָ כִּ֗י כַּאֲשֶׁ֨ר יְיַסֵּ֥ר אִישׁ֙ אֶת־בְּנ֔וֹ יְיָ֥ אֱלֹהֶ֖יךָ מְיַסְּרֶֽךָּ׃ וְשָׁ֣מַרְתָּ֔ אֶת־מִצְוֺ֖ת יְיָ֣ אֱלֹהֶ֑יךָ לָלֶ֥כֶת בִּדְרָכָ֖יו וּלְיִרְאָ֥ה אֹתֽוֹ׃ כִּ֚י יְיָ֣ אֱלֹהֶ֔יךָ מְבִֽיאֲךָ֖ אֶל־אֶ֣רֶץ טוֹבָ֑ה אֶ֚רֶץ נַ֣חֲלֵי מָ֔יִם עֲיָנֹת֙ וּתְהֹמֹ֔ת יֹצְאִ֥ים בַּבִּקְעָ֖ה וּבָהָֽר׃ אֶ֤רֶץ חִטָּה֙ וּשְׂעֹרָ֔ה וְגֶ֥פֶן וּתְאֵנָ֖ה וְרִמּ֑וֹן אֶֽרֶץ־זֵ֥ית שֶׁ֖מֶן וּדְבָֽשׁ׃ אֶ֗רֶץ אֲשֶׁ֨ר לֹ֤א בְמִסְכֵּנֻת֙ תֹּֽאכַל־בָּ֣הּ לֶ֔חֶם לֹֽא־תֶחְסַ֥ר כֹּ֖ל בָּ֑הּ אֶ֚רֶץ אֲשֶׁ֣ר אֲבָנֶ֣יהָ בַרְזֶ֔ל וּמֵהֲרָרֶ֖יהָ תַּחְצֹ֥ב נְחֹֽשֶׁת׃ וְאָכַלְתָּ֖ וְשָׂבָ֑עְתָּ וּבֵֽרַכְתָּ֙ אֶת־יְיָ֣ אֱלֹהֶ֔יךָ עַל־הָאָ֥רֶץ הַטֹּבָ֖ה אֲשֶׁ֥ר נָֽתַן־לָֽךְ׃


\ssubsection{ראה}\\
רְאֵ֗ה אָנֹכִ֛י נֹתֵ֥ן לִפְנֵיכֶ֖ם הַיּ֑וֹם בְּרָכָ֖ה וּקְלָלָֽה׃ אֶֽת־הַבְּרָכָ֑ה אֲשֶׁ֣ר תִּשְׁמְע֗וּ אֶל־מִצְוֺת֙ יְיָ֣ אֱלֹֽהֵיכֶ֔ם אֲשֶׁ֧ר אָנֹכִ֛י מְצַוֶּ֥ה אֶתְכֶ֖ם הַיּֽוֹם׃ וְהַקְּלָלָ֗ה אִם־לֹ֤א תִשְׁמְעוּ֙ אֶל־מִצְוֺת֙ יְיָ֣ אֱלֹֽהֵיכֶ֔ם וְסַרְתֶּ֣ם מִן־הַדֶּ֔רֶךְ אֲשֶׁ֧ר אָנֹכִ֛י מְצַוֶּ֥ה אֶתְכֶ֖ם הַיּ֑וֹם לָלֶ֗כֶת אַחֲרֵ֛י אֱלֹהִ֥ים אֲחֵרִ֖ים אֲשֶׁ֥ר לֹֽא־יְדַעְתֶּֽם׃ {ס} וְהָיָ֗ה כִּ֤י יְבִֽיאֲךָ֙ יְיָ֣ אֱלֹהֶ֔יךָ אֶל־הָאָ֕רֶץ אֲשֶׁר־אַתָּ֥ה בָא־שָׁ֖מָּה לְרִשְׁתָּ֑הּ וְנָתַתָּ֤ה אֶת־הַבְּרָכָה֙ עַל־הַ֣ר גְּרִזִ֔ים וְאֶת־הַקְּלָלָ֖ה עַל־הַ֥ר עֵיבָֽל׃ הֲלֹא־הֵ֜מָּה בְּעֵ֣בֶר הַיַּרְדֵּ֗ן אַֽחֲרֵי֙ דֶּ֚רֶךְ מְב֣וֹא הַשֶּׁ֔מֶשׁ בְּאֶ֙רֶץ֙ הַֽכְּנַעֲנִ֔י הַיֹּשֵׁ֖ב בָּעֲרָבָ֑ה מ֚וּל הַגִּלְגָּ֔ל אֵ֖צֶל אֵלוֹנֵ֥י מֹרֶֽה׃ כִּ֤י אַתֶּם֙ עֹבְרִ֣ים אֶת־הַיַּרְדֵּ֔ן לָבֹא֙ לָרֶ֣שֶׁת אֶת־הָאָ֔רֶץ אֲשֶׁר־יְיָ֥ אֱלֹהֵיכֶ֖ם נֹתֵ֣ן לָכֶ֑ם וִֽירִשְׁתֶּ֥ם אֹתָ֖הּ וִֽישַׁבְתֶּם־בָּֽהּ׃
\aliyah{לוי}
וּשְׁמַרְתֶּ֣ם לַעֲשׂ֔וֹת אֵ֥ת כׇּל־הַֽחֻקִּ֖ים וְאֶת־הַמִּשְׁפָּטִ֑ים אֲשֶׁ֧ר אָנֹכִ֛י נֹתֵ֥ן לִפְנֵיכֶ֖ם הַיּֽוֹם׃ אֵ֠לֶּה הַֽחֻקִּ֣ים וְהַמִּשְׁפָּטִים֮ אֲשֶׁ֣ר תִּשְׁמְר֣וּן לַעֲשׂוֹת֒ בָּאָ֕רֶץ אֲשֶׁר֩ נָתַ֨ן יְיָ֜ אֱלֹהֵ֧י אֲבֹתֶ֛יךָ לְךָ֖ לְרִשְׁתָּ֑הּ כׇּ֨ל־הַיָּמִ֔ים אֲשֶׁר־אַתֶּ֥ם חַיִּ֖ים עַל־הָאֲדָמָֽה׃ אַבֵּ֣ד תְּ֠אַבְּד֠וּן אֶֽת־כׇּל־הַמְּקֹמ֞וֹת אֲשֶׁ֧ר עָֽבְדוּ־שָׁ֣ם הַגּוֹיִ֗ם אֲשֶׁ֥ר אַתֶּ֛ם יֹרְשִׁ֥ים אֹתָ֖ם אֶת־אֱלֹהֵיהֶ֑ם עַל־הֶהָרִ֤ים הָֽרָמִים֙ וְעַל־הַגְּבָע֔וֹת וְתַ֖חַת כׇּל־עֵ֥ץ רַעֲנָֽן׃ וְנִתַּצְתֶּ֣ם אֶת־מִזְבְּחֹתָ֗ם וְשִׁבַּרְתֶּם֙ אֶת־מַצֵּ֣בֹתָ֔ם וַאֲשֵֽׁרֵיהֶם֙ תִּשְׂרְפ֣וּן בָּאֵ֔שׁ וּפְסִילֵ֥י אֱלֹֽהֵיהֶ֖ם תְּגַדֵּע֑וּן וְאִבַּדְתֶּ֣ם אֶת־שְׁמָ֔ם מִן־הַמָּק֖וֹם הַהֽוּא׃ לֹֽא־תַעֲשׂ֣וּן כֵּ֔ן לַייָ֖ אֱלֹהֵיכֶֽם׃ כִּ֠י אִֽם־אֶל־הַמָּק֞וֹם אֲשֶׁר־יִבְחַ֨ר יְיָ֤ אֱלֹֽהֵיכֶם֙ מִכׇּל־שִׁבְטֵיכֶ֔ם לָשׂ֥וּם אֶת־שְׁמ֖וֹ שָׁ֑ם לְשִׁכְנ֥וֹ תִדְרְשׁ֖וּ וּבָ֥אתָ שָּֽׁמָּה׃
\aliyah{ישראל}
וַהֲבֵאתֶ֣ם שָׁ֗מָּה עֹלֹֽתֵיכֶם֙ וְזִבְחֵיכֶ֔ם וְאֵת֙ מַעְשְׂרֹ֣תֵיכֶ֔ם וְאֵ֖ת תְּרוּמַ֣ת יֶדְכֶ֑ם וְנִדְרֵיכֶם֙ וְנִדְבֹ֣תֵיכֶ֔ם וּבְכֹרֹ֥ת בְּקַרְכֶ֖ם וְצֹאנְכֶֽם׃ וַאֲכַלְתֶּם־שָׁ֗ם לִפְנֵי֙ יְיָ֣ אֱלֹֽהֵיכֶ֔ם וּשְׂמַחְתֶּ֗ם בְּכֹל֙ מִשְׁלַ֣ח יֶדְכֶ֔ם אַתֶּ֖ם וּבָתֵּיכֶ֑ם אֲשֶׁ֥ר בֵּֽרַכְךָ֖ יְיָ֥ אֱלֹהֶֽיךָ׃ לֹ֣א תַעֲשׂ֔וּן כְּ֠כֹ֠ל אֲשֶׁ֨ר אֲנַ֧חְנוּ עֹשִׂ֛ים פֹּ֖ה הַיּ֑וֹם אִ֖ישׁ כׇּל־הַיָּשָׁ֥ר בְּעֵינָֽיו׃ כִּ֥י לֹא־בָאתֶ֖ם עַד־עָ֑תָּה אֶל־הַמְּנוּחָה֙ וְאֶל־הַֽנַּחֲלָ֔ה אֲשֶׁר־יְיָ֥ אֱלֹהֶ֖יךָ נֹתֵ֥ן לָֽךְ׃ וַעֲבַרְתֶּם֮ אֶת־הַיַּרְדֵּן֒ וִֽישַׁבְתֶּ֣ם בָּאָ֔רֶץ אֲשֶׁר־יְיָ֥ אֱלֹהֵיכֶ֖ם מַנְחִ֣יל אֶתְכֶ֑ם וְהֵנִ֨יחַ לָכֶ֧ם מִכׇּל־אֹיְבֵיכֶ֛ם מִסָּבִ֖יב וִֽישַׁבְתֶּם־בֶּֽטַח׃


\ssubsection{שופטים}\\
שֹׁפְטִ֣ים וְשֹֽׁטְרִ֗ים תִּֽתֶּן־לְךָ֙ בְּכׇל־שְׁעָרֶ֔יךָ אֲשֶׁ֨ר יְיָ֧ אֱלֹהֶ֛יךָ נֹתֵ֥ן לְךָ֖ לִשְׁבָטֶ֑יךָ וְשָׁפְט֥וּ אֶת־הָעָ֖ם מִשְׁפַּט־צֶֽדֶק׃ לֹא־תַטֶּ֣ה מִשְׁפָּ֔ט לֹ֥א תַכִּ֖יר פָּנִ֑ים וְלֹא־תִקַּ֣ח שֹׁ֔חַד כִּ֣י הַשֹּׁ֗חַד יְעַוֵּר֙ עֵינֵ֣י חֲכָמִ֔ים וִֽיסַלֵּ֖ף דִּבְרֵ֥י צַדִּיקִֽם׃ צֶ֥דֶק צֶ֖דֶק תִּרְדֹּ֑ף לְמַ֤עַן תִּֽחְיֶה֙ וְיָרַשְׁתָּ֣ אֶת־הָאָ֔רֶץ אֲשֶׁר־יְיָ֥ אֱלֹהֶ֖יךָ נֹתֵ֥ן לָֽךְ׃ 
\aliyah{לוי}
לֹֽא־תִטַּ֥ע לְךָ֛ אֲשֵׁרָ֖ה כׇּל־עֵ֑ץ אֵ֗צֶל מִזְבַּ֛ח יְיָ֥ אֱלֹהֶ֖יךָ אֲשֶׁ֥ר תַּעֲשֶׂה־לָּֽךְ׃ וְלֹֽא־תָקִ֥ים לְךָ֖ מַצֵּבָ֑ה אֲשֶׁ֥ר שָׂנֵ֖א יְיָ֥ אֱלֹהֶֽיךָ׃ {ס} לֹא־תִזְבַּח֩ לַייָ֨ אֱלֹהֶ֜יךָ שׁ֣וֹר וָשֶׂ֗ה אֲשֶׁ֨ר יִהְיֶ֥ה בוֹ֙ מ֔וּם כֹּ֖ל דָּבָ֣ר רָ֑ע כִּ֧י תוֹעֲבַ֛ת יְיָ֥ אֱלֹהֶ֖יךָ הֽוּא׃ {ס} כִּֽי־יִמָּצֵ֤א בְקִרְבְּךָ֙ בְּאַחַ֣ד שְׁעָרֶ֔יךָ אֲשֶׁר־יְיָ֥ אֱלֹהֶ֖יךָ נֹתֵ֣ן לָ֑ךְ אִ֣ישׁ אוֹ־אִשָּׁ֗ה אֲשֶׁ֨ר יַעֲשֶׂ֧ה אֶת־הָרַ֛ע בְּעֵינֵ֥י יְיָ־אֱלֹהֶ֖יךָ לַעֲבֹ֥ר בְּרִיתֽוֹ׃ וַיֵּ֗לֶךְ וַֽיַּעֲבֹד֙ אֱלֹהִ֣ים אֲחֵרִ֔ים וַיִּשְׁתַּ֖חוּ לָהֶ֑ם וְלַשֶּׁ֣מֶשׁ ׀ א֣וֹ לַיָּרֵ֗חַ א֛וֹ לְכׇל־צְבָ֥א הַשָּׁמַ֖יִם אֲשֶׁ֥ר לֹא־צִוִּֽיתִי׃ וְהֻֽגַּד־לְךָ֖ וְשָׁמָ֑עְתָּ וְדָרַשְׁתָּ֣ הֵיטֵ֔ב וְהִנֵּ֤ה אֱמֶת֙ נָכ֣וֹן הַדָּבָ֔ר נֶעֶשְׂתָ֛ה הַתּוֹעֵבָ֥ה הַזֹּ֖את בְּיִשְׂרָאֵֽל׃ וְהֽוֹצֵאתָ֣ אֶת־הָאִ֣ישׁ הַה֡וּא אוֹ֩ אֶת־הָאִשָּׁ֨ה הַהִ֜וא אֲשֶׁ֣ר עָ֠שׂ֠וּ אֶת־הַדָּבָ֨ר הָרָ֤ע הַזֶּה֙ אֶל־שְׁעָרֶ֔יךָ אֶת־הָאִ֕ישׁ א֖וֹ אֶת־הָאִשָּׁ֑ה וּסְקַלְתָּ֥ם בָּאֲבָנִ֖ים וָמֵֽתוּ׃ עַל־פִּ֣י ׀ שְׁנַ֣יִם עֵדִ֗ים א֛וֹ שְׁלֹשָׁ֥ה עֵדִ֖ים יוּמַ֣ת הַמֵּ֑ת לֹ֣א יוּמַ֔ת עַל־פִּ֖י עֵ֥ד אֶחָֽד׃ יַ֣ד הָעֵדִ֞ים תִּֽהְיֶה־בּ֤וֹ בָרִאשֹׁנָה֙ לַהֲמִית֔וֹ וְיַ֥ד כׇּל־הָעָ֖ם בָּאַחֲרֹנָ֑ה וּבִֽעַרְתָּ֥ הָרָ֖ע מִקִּרְבֶּֽךָ׃ {פ} כִּ֣י יִפָּלֵא֩ מִמְּךָ֨ דָבָ֜ר לַמִּשְׁפָּ֗ט בֵּֽין־דָּ֨ם ׀ לְדָ֜ם בֵּֽין־דִּ֣ין לְדִ֗ין וּבֵ֥ין נֶ֙גַע֙ לָנֶ֔גַע דִּבְרֵ֥י רִיבֹ֖ת בִּשְׁעָרֶ֑יךָ וְקַמְתָּ֣ וְעָלִ֔יתָ אֶ֨ל־הַמָּק֔וֹם אֲשֶׁ֥ר יִבְחַ֛ר יְיָ֥ אֱלֹהֶ֖יךָ בּֽוֹ׃ וּבָאתָ֗ אֶל־הַכֹּהֲנִים֙ הַלְוִיִּ֔ם וְאֶ֨ל־הַשֹּׁפֵ֔ט אֲשֶׁ֥ר יִהְיֶ֖ה בַּיָּמִ֣ים הָהֵ֑ם וְדָרַשְׁתָּ֙ וְהִגִּ֣ידוּ לְךָ֔ אֵ֖ת דְּבַ֥ר הַמִּשְׁפָּֽט׃ וְעָשִׂ֗יתָ עַל־פִּ֤י הַדָּבָר֙ אֲשֶׁ֣ר יַגִּ֣ידֽוּ לְךָ֔ מִן־הַמָּק֣וֹם הַה֔וּא אֲשֶׁ֖ר יִבְחַ֣ר יְיָ֑ וְשָׁמַרְתָּ֣ לַעֲשׂ֔וֹת כְּכֹ֖ל אֲשֶׁ֥ר יוֹרֽוּךָ׃
\aliyah{ישראל}
עַל־פִּ֨י הַתּוֹרָ֜ה אֲשֶׁ֣ר יוֹר֗וּךָ וְעַל־הַמִּשְׁפָּ֛ט אֲשֶׁר־יֹאמְר֥וּ לְךָ֖ תַּעֲשֶׂ֑ה לֹ֣א תָס֗וּר מִן־הַדָּבָ֛ר אֲשֶׁר־יַגִּ֥ידֽוּ לְךָ֖ יָמִ֥ין וּשְׂמֹֽאל׃ וְהָאִ֞ישׁ אֲשֶׁר־יַעֲשֶׂ֣ה בְזָד֗וֹן לְבִלְתִּ֨י שְׁמֹ֤עַ אֶל־הַכֹּהֵן֙ הָעֹמֵ֞ד לְשָׁ֤רֶת שָׁם֙ אֶת־יְיָ֣ אֱלֹהֶ֔יךָ א֖וֹ אֶל־הַשֹּׁפֵ֑ט וּמֵת֙ הָאִ֣ישׁ הַה֔וּא וּבִֽעַרְתָּ֥ הָרָ֖ע מִיִּשְׂרָאֵֽל׃ וְכׇל־הָעָ֖ם יִשְׁמְע֣וּ וְיִרָ֑אוּ וְלֹ֥א יְזִיד֖וּן עֽוֹד׃ 


\ssubsection{כי־תצא}\\
כִּֽי־תֵצֵ֥א לַמִּלְחָמָ֖ה עַל־אֹיְבֶ֑יךָ וּנְתָנ֞וֹ יְיָ֧ אֱלֹהֶ֛יךָ בְּיָדֶ֖ךָ וְשָׁבִ֥יתָ שִׁבְיֽוֹ׃ וְרָאִ֙יתָ֙ בַּשִּׁבְיָ֔ה אֵ֖שֶׁת יְפַת־תֹּ֑אַר וְחָשַׁקְתָּ֣ בָ֔הּ וְלָקַחְתָּ֥ לְךָ֖ לְאִשָּֽׁה׃ וַהֲבֵאתָ֖הּ אֶל־תּ֣וֹךְ בֵּיתֶ֑ךָ וְגִלְּחָה֙ אֶת־רֹאשָׁ֔הּ וְעָשְׂתָ֖ה אֶת־צִפׇּרְנֶֽיהָ׃ וְהֵסִ֩ירָה֩ אֶת־שִׂמְלַ֨ת שִׁבְיָ֜הּ מֵעָלֶ֗יהָ וְיָֽשְׁבָה֙ בְּבֵיתֶ֔ךָ וּבָ֥כְתָ֛ה אֶת־אָבִ֥יהָ וְאֶת־אִמָּ֖הּ יֶ֣רַח יָמִ֑ים וְאַ֨חַר כֵּ֜ן תָּב֤וֹא אֵלֶ֙יהָ֙ וּבְעַלְתָּ֔הּ וְהָיְתָ֥ה לְךָ֖ לְאִשָּֽׁה׃ וְהָיָ֞ה אִם־לֹ֧א חָפַ֣צְתָּ בָּ֗הּ וְשִׁלַּחְתָּהּ֙ לְנַפְשָׁ֔הּ וּמָכֹ֥ר לֹא־תִמְכְּרֶ֖נָּה בַּכָּ֑סֶף לֹא־תִתְעַמֵּ֣ר בָּ֔הּ תַּ֖חַת אֲשֶׁ֥ר עִנִּיתָֽהּ׃ 
\aliyah{לוי}
כִּֽי־תִהְיֶ֨יןָ לְאִ֜ישׁ שְׁתֵּ֣י נָשִׁ֗ים הָאַחַ֤ת אֲהוּבָה֙ וְהָאַחַ֣ת שְׂנוּאָ֔ה וְיָֽלְדוּ־ל֣וֹ בָנִ֔ים הָאֲהוּבָ֖ה וְהַשְּׂנוּאָ֑ה וְהָיָ֛ה הַבֵּ֥ן הַבְּכֹ֖ר לַשְּׂנִיאָֽה׃ וְהָיָ֗ה בְּיוֹם֙ הַנְחִיל֣וֹ אֶת־בָּנָ֔יו אֵ֥ת אֲשֶׁר־יִהְיֶ֖ה ל֑וֹ לֹ֣א יוּכַ֗ל לְבַכֵּר֙ אֶת־בֶּן־הָ֣אֲהוּבָ֔ה עַל־פְּנֵ֥י בֶן־הַשְּׂנוּאָ֖ה הַבְּכֹֽר׃ כִּי֩ אֶת־הַבְּכֹ֨ר בֶּן־הַשְּׂנוּאָ֜ה יַכִּ֗יר לָ֤תֶת לוֹ֙ פִּ֣י שְׁנַ֔יִם בְּכֹ֥ל אֲשֶׁר־יִמָּצֵ֖א ל֑וֹ כִּי־הוּא֙ רֵאשִׁ֣ית אֹנ֔וֹ ל֖וֹ מִשְׁפַּ֥ט הַבְּכֹרָֽה׃ 
\aliyah{ישראל}
כִּֽי־יִהְיֶ֣ה לְאִ֗ישׁ בֵּ֚ן סוֹרֵ֣ר וּמוֹרֶ֔ה אֵינֶ֣נּוּ שֹׁמֵ֔עַ בְּק֥וֹל אָבִ֖יו וּבְק֣וֹל אִמּ֑וֹ וְיִסְּר֣וּ אֹת֔וֹ וְלֹ֥א יִשְׁמַ֖ע אֲלֵיהֶֽם׃ וְתָ֥פְשׂוּ ב֖וֹ אָבִ֣יו וְאִמּ֑וֹ וְהוֹצִ֧יאוּ אֹת֛וֹ אֶל־זִקְנֵ֥י עִיר֖וֹ וְאֶל־שַׁ֥עַר מְקֹמֽוֹ׃ וְאָמְר֞וּ אֶל־זִקְנֵ֣י עִיר֗וֹ בְּנֵ֤נוּ זֶה֙ סוֹרֵ֣ר וּמֹרֶ֔ה אֵינֶ֥נּוּ שֹׁמֵ֖עַ בְּקֹלֵ֑נוּ זוֹלֵ֖ל וְסֹבֵֽא׃ וּ֠רְגָמֻ֠הוּ כׇּל־אַנְשֵׁ֨י עִיר֤וֹ בָֽאֲבָנִים֙ וָמֵ֔ת וּבִֽעַרְתָּ֥ הָרָ֖ע מִקִּרְבֶּ֑ךָ וְכׇל־יִשְׂרָאֵ֖ל יִשְׁמְע֥וּ וְיִרָֽאוּ׃ 


\ssubsection{כי־תבוא}\\
וְהָיָה֙ כִּֽי־תָב֣וֹא אֶל־הָאָ֔רֶץ אֲשֶׁר֙ יְיָ֣ אֱלֹהֶ֔יךָ נֹתֵ֥ן לְךָ֖ נַחֲלָ֑ה וִֽירִשְׁתָּ֖הּ וְיָשַׁ֥בְתָּ בָּֽהּ׃ וְלָקַחְתָּ֞ מֵרֵאשִׁ֣ית ׀ כׇּל־פְּרִ֣י הָאֲדָמָ֗ה אֲשֶׁ֨ר תָּבִ֧יא מֵֽאַרְצְךָ֛ אֲשֶׁ֨ר יְיָ֧ אֱלֹהֶ֛יךָ נֹתֵ֥ן לָ֖ךְ וְשַׂמְתָּ֣ בַטֶּ֑נֶא וְהָֽלַכְתָּ֙ אֶל־הַמָּק֔וֹם אֲשֶׁ֤ר יִבְחַר֙ יְיָ֣ אֱלֹהֶ֔יךָ לְשַׁכֵּ֥ן שְׁמ֖וֹ שָֽׁם׃ וּבָאתָ֙ אֶל־הַכֹּהֵ֔ן אֲשֶׁ֥ר יִהְיֶ֖ה בַּיָּמִ֣ים הָהֵ֑ם וְאָמַרְתָּ֣ אֵלָ֗יו הִגַּ֤דְתִּי הַיּוֹם֙ לַייָ֣ אֱלֹהֶ֔יךָ כִּי־בָ֙אתִי֙ אֶל־הָאָ֔רֶץ אֲשֶׁ֨ר נִשְׁבַּ֧ע יְיָ֛ לַאֲבֹתֵ֖ינוּ לָ֥תֶת לָֽנוּ׃
\aliyah{לוי}
וְלָקַ֧ח הַכֹּהֵ֛ן הַטֶּ֖נֶא מִיָּדֶ֑ךָ וְהִ֨נִּיח֔וֹ לִפְנֵ֕י מִזְבַּ֖ח יְיָ֥ אֱלֹהֶֽיךָ׃ וְעָנִ֨יתָ וְאָמַרְתָּ֜ לִפְנֵ֣י ׀ יְיָ֣ אֱלֹהֶ֗יךָ אֲרַמִּי֙ אֹבֵ֣ד אָבִ֔י וַיֵּ֣רֶד מִצְרַ֔יְמָה וַיָּ֥גׇר שָׁ֖ם בִּמְתֵ֣י מְעָ֑ט וַֽיְהִי־שָׁ֕ם לְג֥וֹי גָּד֖וֹל עָצ֥וּם וָרָֽב׃ וַיָּרֵ֧עוּ אֹתָ֛נוּ הַמִּצְרִ֖ים וַיְעַנּ֑וּנוּ וַיִּתְּנ֥וּ עָלֵ֖ינוּ עֲבֹדָ֥ה קָשָֽׁה׃ וַנִּצְעַ֕ק אֶל־יְיָ֖ אֱלֹהֵ֣י אֲבֹתֵ֑ינוּ וַיִּשְׁמַ֤ע יְיָ֙ אֶת־קֹלֵ֔נוּ וַיַּ֧רְא אֶת־עׇנְיֵ֛נוּ וְאֶת־עֲמָלֵ֖נוּ וְאֶֽת־לַחֲצֵֽנוּ׃ וַיּוֹצִאֵ֤נוּ יְיָ֙ מִמִּצְרַ֔יִם בְּיָ֤ד חֲזָקָה֙ וּבִזְרֹ֣עַ נְטוּיָ֔ה וּבְמֹרָ֖א גָּדֹ֑ל וּבְאֹת֖וֹת וּבְמֹפְתִֽים׃ וַיְבִאֵ֖נוּ אֶל־הַמָּק֣וֹם הַזֶּ֑ה וַיִּתֶּן־לָ֙נוּ֙ אֶת־הָאָ֣רֶץ הַזֹּ֔את אֶ֛רֶץ זָבַ֥ת חָלָ֖ב וּדְבָֽשׁ׃ וְעַתָּ֗ה הִנֵּ֤ה הֵבֵ֙אתִי֙ אֶת־רֵאשִׁית֙ פְּרִ֣י הָאֲדָמָ֔ה אֲשֶׁר־נָתַ֥תָּה לִּ֖י יְיָ֑ וְהִנַּחְתּ֗וֹ לִפְנֵי֙ יְיָ֣ אֱלֹהֶ֔יךָ וְהִֽשְׁתַּחֲוִ֔יתָ לִפְנֵ֖י יְיָ֥ אֱלֹהֶֽיךָ׃ וְשָׂמַחְתָּ֣ בְכׇל־הַטּ֗וֹב אֲשֶׁ֧ר נָֽתַן־לְךָ֛ יְיָ֥ אֱלֹהֶ֖יךָ וּלְבֵיתֶ֑ךָ אַתָּה֙ וְהַלֵּוִ֔י וְהַגֵּ֖ר אֲשֶׁ֥ר בְּקִרְבֶּֽךָ׃ 
\aliyah{ישראל}
כִּ֣י תְכַלֶּ֞ה לַ֠עְשֵׂ֠ר אֶת־כׇּל־מַעְשַׂ֧ר תְּבוּאָתְךָ֛ בַּשָּׁנָ֥ה הַשְּׁלִישִׁ֖ת שְׁנַ֣ת הַֽמַּעֲשֵׂ֑ר וְנָתַתָּ֣ה לַלֵּוִ֗י לַגֵּר֙ לַיָּת֣וֹם וְלָֽאַלְמָנָ֔ה וְאָכְל֥וּ בִשְׁעָרֶ֖יךָ וְשָׂבֵֽעוּ׃ וְאָמַרְתָּ֡ לִפְנֵי֩ יְיָ֨ אֱלֹהֶ֜יךָ בִּעַ֧רְתִּי הַקֹּ֣דֶשׁ מִן־הַבַּ֗יִת וְגַ֨ם נְתַתִּ֤יו לַלֵּוִי֙ וְלַגֵּר֙ לַיָּת֣וֹם וְלָאַלְמָנָ֔ה כְּכׇל־מִצְוָתְךָ֖ אֲשֶׁ֣ר צִוִּיתָ֑נִי לֹֽא־עָבַ֥רְתִּי מִמִּצְוֺתֶ֖יךָ וְלֹ֥א שָׁכָֽחְתִּי׃ לֹא־אָכַ֨לְתִּי בְאֹנִ֜י מִמֶּ֗נּוּ וְלֹא־בִעַ֤רְתִּי מִמֶּ֙נּוּ֙ בְּטָמֵ֔א וְלֹא־נָתַ֥תִּי מִמֶּ֖נּוּ לְמֵ֑ת שָׁמַ֗עְתִּי בְּקוֹל֙ יְיָ֣ אֱלֹהָ֔י עָשִׂ֕יתִי כְּכֹ֖ל אֲשֶׁ֥ר צִוִּיתָֽנִי׃ הַשְׁקִ֩יפָה֩ מִמְּע֨וֹן קׇדְשְׁךָ֜ מִן־הַשָּׁמַ֗יִם וּבָרֵ֤ךְ אֶֽת־עַמְּךָ֙ אֶת־יִשְׂרָאֵ֔ל וְאֵת֙ הָאֲדָמָ֔ה אֲשֶׁ֥ר נָתַ֖תָּה לָ֑נוּ כַּאֲשֶׁ֤ר נִשְׁבַּ֙עְתָּ֙ לַאֲבֹתֵ֔ינוּ אֶ֛רֶץ זָבַ֥ת חָלָ֖ב וּדְבָֽשׁ׃ 


\ssubsection{נצבים}\\
אַתֶּ֨ם נִצָּבִ֤ים הַיּוֹם֙ כֻּלְּכֶ֔ם לִפְנֵ֖י יְיָ֣ אֱלֹהֵיכֶ֑ם רָאשֵׁיכֶ֣ם שִׁבְטֵיכֶ֗ם זִקְנֵיכֶם֙ וְשֹׁ֣טְרֵיכֶ֔ם כֹּ֖ל אִ֥ישׁ יִשְׂרָאֵֽל׃ טַפְּכֶ֣ם נְשֵׁיכֶ֔ם וְגֵ֣רְךָ֔ אֲשֶׁ֖ר בְּקֶ֣רֶב מַחֲנֶ֑יךָ מֵחֹטֵ֣ב עֵצֶ֔יךָ עַ֖ד שֹׁאֵ֥ב מֵימֶֽיךָ׃ לְעׇבְרְךָ֗ בִּבְרִ֛ית יְיָ֥ אֱלֹהֶ֖יךָ וּבְאָלָת֑וֹ אֲשֶׁר֙ יְיָ֣ אֱלֹהֶ֔יךָ כֹּרֵ֥ת עִמְּךָ֖ הַיּֽוֹם׃
\aliyah{לוי}
לְמַ֣עַן הָקִֽים־אֹתְךָ֩ הַיּ֨וֹם ׀ ל֜וֹ לְעָ֗ם וְה֤וּא יִֽהְיֶה־לְּךָ֙ לֵֽאלֹהִ֔ים כַּאֲשֶׁ֖ר דִּבֶּר־לָ֑ךְ וְכַאֲשֶׁ֤ר נִשְׁבַּע֙ לַאֲבֹתֶ֔יךָ לְאַבְרָהָ֥ם לְיִצְחָ֖ק וּֽלְיַעֲקֹֽב׃ וְלֹ֥א אִתְּכֶ֖ם לְבַדְּכֶ֑ם אָנֹכִ֗י כֹּרֵת֙ אֶת־הַבְּרִ֣ית הַזֹּ֔את וְאֶת־הָאָלָ֖ה הַזֹּֽאת׃ כִּי֩ אֶת־אֲשֶׁ֨ר יֶשְׁנ֜וֹ פֹּ֗ה עִמָּ֙נוּ֙ עֹמֵ֣ד הַיּ֔וֹם לִפְנֵ֖י יְיָ֣ אֱלֹהֵ֑ינוּ וְאֵ֨ת אֲשֶׁ֥ר אֵינֶ֛נּוּ פֹּ֖ה עִמָּ֥נוּ הַיּֽוֹם׃
\aliyah{ישראל}
כִּֽי־אַתֶּ֣ם יְדַעְתֶּ֔ם אֵ֥ת אֲשֶׁר־יָשַׁ֖בְנוּ בְּאֶ֣רֶץ מִצְרָ֑יִם וְאֵ֧ת אֲשֶׁר־עָבַ֛רְנוּ בְּקֶ֥רֶב הַגּוֹיִ֖ם אֲשֶׁ֥ר עֲבַרְתֶּֽם׃ וַתִּרְאוּ֙ אֶת־שִׁקּ֣וּצֵיהֶ֔ם וְאֵ֖ת גִּלֻּלֵיהֶ֑ם עֵ֣ץ וָאֶ֔בֶן כֶּ֥סֶף וְזָהָ֖ב אֲשֶׁ֥ר עִמָּהֶֽם׃ פֶּן־יֵ֣שׁ בָּ֠כֶ֠ם אִ֣ישׁ אוֹ־אִשָּׁ֞ה א֧וֹ מִשְׁפָּחָ֣ה אוֹ־שֵׁ֗בֶט אֲשֶׁר֩ לְבָב֨וֹ פֹנֶ֤ה הַיּוֹם֙ מֵעִם֙ יְיָ֣ אֱלֹהֵ֔ינוּ לָלֶ֣כֶת לַעֲבֹ֔ד אֶת־אֱלֹהֵ֖י הַגּוֹיִ֣ם הָהֵ֑ם פֶּן־יֵ֣שׁ בָּכֶ֗ם שֹׁ֛רֶשׁ פֹּרֶ֥ה רֹ֖אשׁ וְלַעֲנָֽה׃ וְהָיָ֡ה בְּשׇׁמְעוֹ֩ אֶת־דִּבְרֵ֨י הָאָלָ֜ה הַזֹּ֗את וְהִתְבָּרֵ֨ךְ בִּלְבָב֤וֹ לֵאמֹר֙ שָׁל֣וֹם יִֽהְיֶה־לִּ֔י כִּ֛י בִּשְׁרִר֥וּת לִבִּ֖י אֵלֵ֑ךְ לְמַ֛עַן סְפ֥וֹת הָרָוָ֖ה אֶת־הַצְּמֵאָֽה׃ לֹא־יֹאבֶ֣ה יְיָ סְלֹ֣חַֽ לוֹ֒ כִּ֣י אָ֠ז יֶעְשַׁ֨ן אַף־יְיָ֤ וְקִנְאָתוֹ֙ בָּאִ֣ישׁ הַה֔וּא וְרָ֤בְצָה בּוֹ֙ כׇּל־הָ֣אָלָ֔ה הַכְּתוּבָ֖ה בַּסֵּ֣פֶר הַזֶּ֑ה וּמָחָ֤ה יְיָ֙ אֶת־שְׁמ֔וֹ מִתַּ֖חַת הַשָּׁמָֽיִם׃ וְהִבְדִּיל֤וֹ יְיָ֙ לְרָעָ֔ה מִכֹּ֖ל שִׁבְטֵ֣י יִשְׂרָאֵ֑ל כְּכֹל֙ אָל֣וֹת הַבְּרִ֔ית הַכְּתוּבָ֕ה בְּסֵ֥פֶר הַתּוֹרָ֖ה הַזֶּֽה׃ וְאָמַ֞ר הַדּ֣וֹר הָאַחֲר֗וֹן בְּנֵיכֶם֙ אֲשֶׁ֤ר יָק֙וּמוּ֙ מֵאַ֣חֲרֵיכֶ֔ם וְהַ֨נׇּכְרִ֔י אֲשֶׁ֥ר יָבֹ֖א מֵאֶ֣רֶץ רְחוֹקָ֑ה וְ֠רָא֠וּ אֶת־מַכּ֞וֹת הָאָ֤רֶץ הַהִוא֙ וְאֶת־תַּ֣חֲלֻאֶ֔יהָ אֲשֶׁר־חִלָּ֥ה יְיָ֖ בָּֽהּ׃ גׇּפְרִ֣ית וָמֶ֘לַח֮ שְׂרֵפָ֣ה כׇל־אַרְצָהּ֒ לֹ֤א תִזָּרַע֙ וְלֹ֣א תַצְמִ֔חַ וְלֹא־יַעֲלֶ֥ה בָ֖הּ כׇּל־עֵ֑שֶׂב כְּֽמַהְפֵּכַ֞ת סְדֹ֤ם וַעֲמֹרָה֙ אַדְמָ֣ה וּצְבֹיִ֔ים אֲשֶׁר֙ הָפַ֣ךְ יְיָ֔ בְּאַפּ֖וֹ וּבַחֲמָתֽוֹ׃ וְאָֽמְרוּ֙ כׇּל־הַגּוֹיִ֔ם עַל־מֶ֨ה עָשָׂ֧ה יְיָ֛ כָּ֖כָה לָאָ֣רֶץ הַזֹּ֑את מֶ֥ה חֳרִ֛י הָאַ֥ף הַגָּד֖וֹל הַזֶּֽה׃ וְאָ֣מְר֔וּ עַ֚ל אֲשֶׁ֣ר עָזְב֔וּ אֶת־בְּרִ֥ית יְיָ֖ אֱלֹהֵ֣י אֲבֹתָ֑ם אֲשֶׁר֙ כָּרַ֣ת עִמָּ֔ם בְּהוֹצִיא֥וֹ אֹתָ֖ם מֵאֶ֥רֶץ מִצְרָֽיִם׃ וַיֵּלְכ֗וּ וַיַּֽעַבְדוּ֙ אֱלֹהִ֣ים אֲחֵרִ֔ים וַיִּֽשְׁתַּחֲו֖וּ לָהֶ֑ם אֱלֹהִים֙ אֲשֶׁ֣ר לֹֽא־יְדָע֔וּם וְלֹ֥א חָלַ֖ק לָהֶֽם׃ וַיִּחַר־אַ֥ף יְיָ֖ בָּאָ֣רֶץ הַהִ֑וא לְהָבִ֤יא עָלֶ֙יהָ֙ אֶת־כׇּל־הַקְּלָלָ֔ה הַכְּתוּבָ֖ה בַּסֵּ֥פֶר הַזֶּֽה׃ וַיִּתְּשֵׁ֤ם יְיָ֙ מֵעַ֣ל אַדְמָתָ֔ם בְּאַ֥ף וּבְחֵמָ֖ה וּבְקֶ֣צֶף גָּד֑וֹל וַיַּשְׁלִכֵ֛ם אֶל־אֶ֥רֶץ אַחֶ֖רֶת כַּיּ֥וֹם הַזֶּֽה׃ הַנִּ֨סְתָּרֹ֔ת לַייָ֖ אֱלֹהֵ֑ינוּ וְהַנִּגְלֹ֞ת לָ֤ׄנׄוּׄ וּׄלְׄבָׄנֵ֙ׄיׄנׄוּ֙ׄ עַׄד־עוֹלָ֔ם לַעֲשׂ֕וֹת אֶת־כׇּל־דִּבְרֵ֖י הַתּוֹרָ֥ה הַזֹּֽאת׃ 


\ssubsection{וילך}\\
וַיֵּ֖לֶךְ מֹשֶׁ֑ה וַיְדַבֵּ֛ר אֶת־הַדְּבָרִ֥ים הָאֵ֖לֶּה אֶל־כׇּל־יִשְׂרָאֵֽל׃ וַיֹּ֣אמֶר אֲלֵהֶ֗ם בֶּן־מֵאָה֩ וְעֶשְׂרִ֨ים שָׁנָ֤ה אָנֹכִי֙ הַיּ֔וֹם לֹא־אוּכַ֥ל ע֖וֹד לָצֵ֣את וְלָב֑וֹא וַֽייָ֙ אָמַ֣ר אֵלַ֔י לֹ֥א תַעֲבֹ֖ר אֶת־הַיַּרְדֵּ֥ן הַזֶּֽה׃ יְיָ֨ אֱלֹהֶ֜יךָ ה֣וּא ׀ עֹבֵ֣ר לְפָנֶ֗יךָ הֽוּא־יַשְׁמִ֞יד אֶת־הַגּוֹיִ֥ם הָאֵ֛לֶּה מִלְּפָנֶ֖יךָ וִירִשְׁתָּ֑ם יְהוֹשֻׁ֗עַ ה֚וּא עֹבֵ֣ר לְפָנֶ֔יךָ כַּאֲשֶׁ֖ר דִּבֶּ֥ר יְיָ׃
\aliyah{לוי}
וְעָשָׂ֤ה יְיָ֙ לָהֶ֔ם כַּאֲשֶׁ֣ר עָשָׂ֗ה לְסִיח֥וֹן וּלְע֛וֹג מַלְכֵ֥י הָאֱמֹרִ֖י וּלְאַרְצָ֑ם אֲשֶׁ֥ר הִשְׁמִ֖יד אֹתָֽם׃ וּנְתָנָ֥ם יְיָ֖ לִפְנֵיכֶ֑ם וַעֲשִׂיתֶ֣ם לָהֶ֔ם כְּכׇ֨ל־הַמִּצְוָ֔ה אֲשֶׁ֥ר צִוִּ֖יתִי אֶתְכֶֽם׃ חִזְק֣וּ וְאִמְצ֔וּ אַל־תִּֽירְא֥וּ וְאַל־תַּעַרְצ֖וּ מִפְּנֵיהֶ֑ם כִּ֣י ׀ יְיָ֣ אֱלֹהֶ֗יךָ ה֚וּא הַהֹלֵ֣ךְ עִמָּ֔ךְ לֹ֥א יַרְפְּךָ֖ וְלֹ֥א יַעַזְבֶֽךָּ׃ 
\aliyah{ישראל}
חִזְק֣וּ וְאִמְצ֔וּ אַל־תִּֽירְא֥וּ וְאַל־תַּעַרְצ֖וּ מִפְּנֵיהֶ֑ם כִּ֣י ׀ יְיָ֣ אֱלֹהֶ֗יךָ ה֚וּא הַהֹלֵ֣ךְ עִמָּ֔ךְ לֹ֥א יַרְפְּךָ֖ וְלֹ֥א יַעַזְבֶֽךָּ׃ {ס} וַיִּקְרָ֨א מֹשֶׁ֜ה לִיהוֹשֻׁ֗עַ וַיֹּ֨אמֶר אֵלָ֜יו לְעֵינֵ֣י כׇל־יִשְׂרָאֵל֮ חֲזַ֣ק וֶאֱמָץ֒ כִּ֣י אַתָּ֗ה תָּבוֹא֙ אֶת־הָעָ֣ם הַזֶּ֔ה אֶל־הָאָ֕רֶץ אֲשֶׁ֨ר נִשְׁבַּ֧ע יְיָ֛ לַאֲבֹתָ֖ם לָתֵ֣ת לָהֶ֑ם וְאַתָּ֖ה תַּנְחִילֶ֥נָּה אוֹתָֽם׃ וַייָ֞ ה֣וּא ׀ הַהֹלֵ֣ךְ לְפָנֶ֗יךָ ה֚וּא יִהְיֶ֣ה עִמָּ֔ךְ לֹ֥א יַרְפְּךָ֖ וְלֹ֣א יַעַזְבֶ֑ךָּ לֹ֥א תִירָ֖א וְלֹ֥א תֵחָֽת׃ וַיִּכְתֹּ֣ב מֹשֶׁה֮ אֶת־הַתּוֹרָ֣ה הַזֹּאת֒ וַֽיִּתְּנָ֗הּ אֶל־הַכֹּֽהֲנִים֙ בְּנֵ֣י לֵוִ֔י הַנֹּ֣שְׂאִ֔ים אֶת־אֲר֖וֹן בְּרִ֣ית יְיָ֑ וְאֶל־כׇּל־זִקְנֵ֖י יִשְׂרָאֵֽל׃ וַיְצַ֥ו מֹשֶׁ֖ה אוֹתָ֣ם לֵאמֹ֑ר מִקֵּ֣ץ ׀ שֶׁ֣בַע שָׁנִ֗ים בְּמֹעֵ֛ד שְׁנַ֥ת הַשְּׁמִטָּ֖ה בְּחַ֥ג הַסֻּכּֽוֹת׃ בְּב֣וֹא כׇל־יִשְׂרָאֵ֗ל לֵֽרָאוֹת֙ אֶת־פְּנֵי֙ יְיָ֣ אֱלֹהֶ֔יךָ בַּמָּק֖וֹם אֲשֶׁ֣ר יִבְחָ֑ר תִּקְרָ֞א אֶת־הַתּוֹרָ֥ה הַזֹּ֛את נֶ֥גֶד כׇּל־יִשְׂרָאֵ֖ל בְּאׇזְנֵיהֶֽם׃ הַקְהֵ֣ל אֶת־הָעָ֗ם הָֽאֲנָשִׁ֤ים וְהַנָּשִׁים֙ וְהַטַּ֔ף וְגֵרְךָ֖ אֲשֶׁ֣ר בִּשְׁעָרֶ֑יךָ לְמַ֨עַן יִשְׁמְע֜וּ וּלְמַ֣עַן יִלְמְד֗וּ וְיָֽרְאוּ֙ אֶת־יְיָ֣ אֱלֹהֵיכֶ֔ם וְשָֽׁמְר֣וּ לַעֲשׂ֔וֹת אֶת־כׇּל־דִּבְרֵ֖י הַתּוֹרָ֥ה הַזֹּֽאת׃ וּבְנֵיהֶ֞ם אֲשֶׁ֣ר לֹא־יָדְע֗וּ יִשְׁמְעוּ֙ וְלָ֣מְד֔וּ לְיִרְאָ֖ה אֶת־יְיָ֣ אֱלֹהֵיכֶ֑ם כׇּל־הַיָּמִ֗ים אֲשֶׁ֨ר אַתֶּ֤ם חַיִּים֙ עַל־הָ֣אֲדָמָ֔ה אֲשֶׁ֨ר אַתֶּ֜ם עֹבְרִ֧ים אֶת־הַיַּרְדֵּ֛ן שָׁ֖מָּה לְרִשְׁתָּֽהּ׃ 




\newcommand{\tsource}[1]{\begin{scriptsize} \textsf{(#1)} \end{scriptsize}}
\section[חנכה]{קריאה לחנכה}


\instruction{בכל ימי חנכה קוראים אותו היום ומחרתו בפרשת המלאים. כהן באותו יום עד "מלאה קטרת", לוי עד סוף היום, וישראל יום הבא. ביום אחרון קוראים עד התחלת פרשת בהעלתך}


\tsource{במדבר ז}\\
וַיְהִ֡י בְּיוֹם֩ כַּלּ֨וֹת מֹשֶׁ֜ה לְהָקִ֣ים אֶת־הַמִּשְׁכָּ֗ן וַיִּמְשַׁ֨ח אֹת֜וֹ וַיְקַדֵּ֤שׁ אֹתוֹ֙ וְאֶת־כׇּל־כֵּלָ֔יו וְאֶת־הַמִּזְבֵּ֖חַ וְאֶת־כׇּל־כֵּלָ֑יו וַיִּמְשָׁחֵ֖ם וַיְקַדֵּ֥שׁ אֹתָֽם׃ וַיַּקְרִ֙יבוּ֙ נְשִׂיאֵ֣י יִשְׂרָאֵ֔ל רָאשֵׁ֖י בֵּ֣ית אֲבֹתָ֑ם הֵ֚ם נְשִׂיאֵ֣י הַמַּטֹּ֔ת הֵ֥ם הָעֹמְדִ֖ים עַל־הַפְּקֻדִֽים׃ וַיָּבִ֨יאוּ אֶת־קׇרְבָּנָ֜ם לִפְנֵ֣י יְיָ֗ שֵׁשׁ־עֶגְלֹ֥ת צָב֙ וּשְׁנֵ֣י עָשָׂ֣ר בָּקָ֔ר עֲגָלָ֛ה עַל־שְׁנֵ֥י הַנְּשִׂאִ֖ים וְשׁ֣וֹר לְאֶחָ֑ד וַיַּקְרִ֥יבוּ אוֹתָ֖ם לִפְנֵ֥י הַמִּשְׁכָּֽן׃ וַיֹּ֥אמֶר יְיָ֖ אֶל־מֹשֶׁ֥ה לֵּאמֹֽר׃ קַ֚ח מֵֽאִתָּ֔ם וְהָי֕וּ לַעֲבֹ֕ד אֶת־עֲבֹדַ֖ת אֹ֣הֶל מוֹעֵ֑ד וְנָתַתָּ֤ה אוֹתָם֙ אֶל־הַלְוִיִּ֔ם אִ֖ישׁ כְּפִ֥י עֲבֹדָתֽוֹ׃ וַיִּקַּ֣ח מֹשֶׁ֔ה אֶת־הָעֲגָלֹ֖ת וְאֶת־הַבָּקָ֑ר וַיִּתֵּ֥ן אוֹתָ֖ם אֶל־הַלְוִיִּֽם׃ אֵ֣ת ׀ שְׁתֵּ֣י הָעֲגָל֗וֹת וְאֵת֙ אַרְבַּ֣עַת הַבָּקָ֔ר נָתַ֖ן לִבְנֵ֣י גֵרְשׁ֑וֹן כְּפִ֖י עֲבֹדָתָֽם׃ וְאֵ֣ת ׀ אַרְבַּ֣ע הָעֲגָלֹ֗ת וְאֵת֙ שְׁמֹנַ֣ת הַבָּקָ֔ר נָתַ֖ן לִבְנֵ֣י מְרָרִ֑י כְּפִי֙ עֲבֹ֣דָתָ֔ם בְּיַד֙ אִֽיתָמָ֔ר בֶּֽן־אַהֲרֹ֖ן הַכֹּהֵֽן׃ וְלִבְנֵ֥י קְהָ֖ת לֹ֣א נָתָ֑ן כִּֽי־עֲבֹדַ֤ת הַקֹּ֙דֶשׁ֙ עֲלֵהֶ֔ם בַּכָּתֵ֖ף יִשָּֽׂאוּ׃ וַיַּקְרִ֣יבוּ הַנְּשִׂאִ֗ים אֵ֚ת חֲנֻכַּ֣ת הַמִּזְבֵּ֔חַ בְּי֖וֹם הִמָּשַׁ֣ח אֹת֑וֹ וַיַּקְרִ֧יבוּ הַנְּשִׂיאִ֛ם אֶת־קׇרְבָּנָ֖ם לִפְנֵ֥י הַמִּזְבֵּֽחַ׃ וַיֹּ֥אמֶר יְיָ֖ אֶל־מֹשֶׁ֑ה נָשִׂ֨יא אֶחָ֜ד לַיּ֗וֹם נָשִׂ֤יא אֶחָד֙ לַיּ֔וֹם יַקְרִ֙יבוּ֙ אֶת־קׇרְבָּנָ֔ם לַחֲנֻכַּ֖ת הַמִּזְבֵּֽחַ׃ 
\aliyah{לוי ביום א׳}
וַיְהִ֗י הַמַּקְרִ֛יב בַּיּ֥וֹם הָרִאשׁ֖וֹן אֶת־קׇרְבָּנ֑וֹ נַחְשׁ֥וֹן בֶּן־עַמִּינָדָ֖ב לְמַטֵּ֥ה יְהוּדָֽה׃ וְקׇרְבָּנ֞וֹ קַֽעֲרַת־כֶּ֣סֶף אַחַ֗ת שְׁלֹשִׁ֣ים וּמֵאָה֮ מִשְׁקָלָהּ֒ מִזְרָ֤ק אֶחָד֙ כֶּ֔סֶף שִׁבְעִ֥ים שֶׁ֖קֶל בְּשֶׁ֣קֶל הַקֹּ֑דֶשׁ שְׁנֵיהֶ֣ם ׀ מְלֵאִ֗ים סֹ֛לֶת בְּלוּלָ֥ה בַשֶּׁ֖מֶן לְמִנְחָֽה׃ כַּ֥ף אַחַ֛ת עֲשָׂרָ֥ה זָהָ֖ב מְלֵאָ֥ה קְטֹֽרֶת׃
\aliyah{ישראל ביום א׳}
פַּ֣ר אֶחָ֞ד בֶּן־בָּקָ֗ר אַ֧יִל אֶחָ֛ד כֶּֽבֶשׂ־אֶחָ֥ד בֶּן־שְׁנָת֖וֹ לְעֹלָֽה׃ שְׂעִיר־עִזִּ֥ים אֶחָ֖ד לְחַטָּֽאת׃ וּלְזֶ֣בַח הַשְּׁלָמִים֮ בָּקָ֣ר שְׁנַ֒יִם֒ אֵילִ֤ם חֲמִשָּׁה֙ עַתּוּדִ֣ים חֲמִשָּׁ֔ה כְּבָשִׂ֥ים בְּנֵֽי־שָׁנָ֖ה חֲמִשָּׁ֑ה זֶ֛ה קׇרְבַּ֥ן נַחְשׁ֖וֹן בֶּן־עַמִּינָדָֽב׃ {פ} בַּיּוֹם֙ הַשֵּׁנִ֔י הִקְרִ֖יב נְתַנְאֵ֣ל בֶּן־צוּעָ֑ר נְשִׂ֖יא יִשָּׂשכָֽר׃ הִקְרִ֨ב אֶת־קׇרְבָּנ֜וֹ קַֽעֲרַת־כֶּ֣סֶף אַחַ֗ת שְׁלֹשִׁ֣ים וּמֵאָה֮ מִשְׁקָלָהּ֒ מִזְרָ֤ק אֶחָד֙ כֶּ֔סֶף שִׁבְעִ֥ים שֶׁ֖קֶל בְּשֶׁ֣קֶל הַקֹּ֑דֶשׁ שְׁנֵיהֶ֣ם ׀ מְלֵאִ֗ים סֹ֛לֶת בְּלוּלָ֥ה בַשֶּׁ֖מֶן לְמִנְחָֽה׃ כַּ֥ף אַחַ֛ת עֲשָׂרָ֥ה זָהָ֖ב מְלֵאָ֥ה קְטֹֽרֶת׃ פַּ֣ר אֶחָ֞ד בֶּן־בָּקָ֗ר אַ֧יִל אֶחָ֛ד כֶּֽבֶשׂ־אֶחָ֥ד בֶּן־שְׁנָת֖וֹ לְעֹלָֽה׃ שְׂעִיר־עִזִּ֥ים אֶחָ֖ד לְחַטָּֽאת׃ וּלְזֶ֣בַח הַשְּׁלָמִים֮ בָּקָ֣ר שְׁנַ֒יִם֒ אֵילִ֤ם חֲמִשָּׁה֙ עַתֻּדִ֣ים חֲמִשָּׁ֔ה כְּבָשִׂ֥ים בְּנֵי־שָׁנָ֖ה חֲמִשָּׁ֑ה זֶ֛ה קׇרְבַּ֥ן נְתַנְאֵ֖ל בֶּן־צוּעָֽר׃ {פ} בַּיּוֹם֙ הַשְּׁלִישִׁ֔י נָשִׂ֖יא לִבְנֵ֣י זְבוּלֻ֑ן אֱלִיאָ֖ב בֶּן־חֵלֹֽן׃ קׇרְבָּנ֞וֹ קַֽעֲרַת־כֶּ֣סֶף אַחַ֗ת שְׁלֹשִׁ֣ים וּמֵאָה֮ מִשְׁקָלָהּ֒ מִזְרָ֤ק אֶחָד֙ כֶּ֔סֶף שִׁבְעִ֥ים שֶׁ֖קֶל בְּשֶׁ֣קֶל הַקֹּ֑דֶשׁ שְׁנֵיהֶ֣ם ׀ מְלֵאִ֗ים סֹ֛לֶת בְּלוּלָ֥ה בַשֶּׁ֖מֶן לְמִנְחָֽה׃ כַּ֥ף אַחַ֛ת עֲשָׂרָ֥ה זָהָ֖ב מְלֵאָ֥ה קְטֹֽרֶת׃ פַּ֣ר אֶחָ֞ד בֶּן־בָּקָ֗ר אַ֧יִל אֶחָ֛ד כֶּֽבֶשׂ־אֶחָ֥ד בֶּן־שְׁנָת֖וֹ לְעֹלָֽה׃ שְׂעִיר־עִזִּ֥ים אֶחָ֖ד לְחַטָּֽאת׃ וּלְזֶ֣בַח הַשְּׁלָמִים֮ בָּקָ֣ר שְׁנַ֒יִם֒ אֵילִ֤ם חֲמִשָּׁה֙ עַתֻּדִ֣ים חֲמִשָּׁ֔ה כְּבָשִׂ֥ים בְּנֵי־שָׁנָ֖ה חֲמִשָּׁ֑ה זֶ֛ה קׇרְבַּ֥ן אֱלִיאָ֖ב בֶּן־חֵלֹֽן׃ {פ} בַּיּוֹם֙ הָרְבִיעִ֔י נָשִׂ֖יא לִבְנֵ֣י רְאוּבֵ֑ן אֱלִיצ֖וּר בֶּן־שְׁדֵיאֽוּר׃ קׇרְבָּנ֞וֹ קַֽעֲרַת־כֶּ֣סֶף אַחַ֗ת שְׁלֹשִׁ֣ים וּמֵאָה֮ מִשְׁקָלָהּ֒ מִזְרָ֤ק אֶחָד֙ כֶּ֔סֶף שִׁבְעִ֥ים שֶׁ֖קֶל בְּשֶׁ֣קֶל הַקֹּ֑דֶשׁ שְׁנֵיהֶ֣ם ׀ מְלֵאִ֗ים סֹ֛לֶת בְּלוּלָ֥ה בַשֶּׁ֖מֶן לְמִנְחָֽה׃ כַּ֥ף אַחַ֛ת עֲשָׂרָ֥ה זָהָ֖ב מְלֵאָ֥ה קְטֹֽרֶת׃ פַּ֣ר אֶחָ֞ד בֶּן־בָּקָ֗ר אַ֧יִל אֶחָ֛ד כֶּֽבֶשׂ־אֶחָ֥ד בֶּן־שְׁנָת֖וֹ לְעֹלָֽה׃ שְׂעִיר־עִזִּ֥ים אֶחָ֖ד לְחַטָּֽאת׃ וּלְזֶ֣בַח הַשְּׁלָמִים֮ בָּקָ֣ר שְׁנַ֒יִם֒ אֵילִ֤ם חֲמִשָּׁה֙ עַתֻּדִ֣ים חֲמִשָּׁ֔ה כְּבָשִׂ֥ים בְּנֵֽי־שָׁנָ֖ה חֲמִשָּׁ֑ה זֶ֛ה קׇרְבַּ֥ן אֱלִיצ֖וּר בֶּן־שְׁדֵיאֽוּר׃ {פ} בַּיּוֹם֙ הַחֲמִישִׁ֔י נָשִׂ֖יא לִבְנֵ֣י שִׁמְע֑וֹן שְׁלֻֽמִיאֵ֖ל בֶּן־צוּרִֽישַׁדָּֽי׃ קׇרְבָּנ֞וֹ קַֽעֲרַת־כֶּ֣סֶף אַחַ֗ת שְׁלֹשִׁ֣ים וּמֵאָה֮ מִשְׁקָלָהּ֒ מִזְרָ֤ק אֶחָד֙ כֶּ֔סֶף שִׁבְעִ֥ים שֶׁ֖קֶל בְּשֶׁ֣קֶל הַקֹּ֑דֶשׁ שְׁנֵיהֶ֣ם ׀ מְלֵאִ֗ים סֹ֛לֶת בְּלוּלָ֥ה בַשֶּׁ֖מֶן לְמִנְחָֽה׃ כַּ֥ף אַחַ֛ת עֲשָׂרָ֥ה זָהָ֖ב מְלֵאָ֥ה קְטֹֽרֶת׃ פַּ֣ר אֶחָ֞ד בֶּן־בָּקָ֗ר אַ֧יִל אֶחָ֛ד כֶּֽבֶשׂ־אֶחָ֥ד בֶּן־שְׁנָת֖וֹ לְעֹלָֽה׃ שְׂעִיר־עִזִּ֥ים אֶחָ֖ד לְחַטָּֽאת׃ וּלְזֶ֣בַח הַשְּׁלָמִים֮ בָּקָ֣ר שְׁנַ֒יִם֒ אֵילִ֤ם חֲמִשָּׁה֙ עַתֻּדִ֣ים חֲמִשָּׁ֔ה כְּבָשִׂ֥ים בְּנֵי־שָׁנָ֖ה חֲמִשָּׁ֑ה זֶ֛ה קׇרְבַּ֥ן שְׁלֻמִיאֵ֖ל בֶּן־צוּרִֽישַׁדָּֽי׃ {פ} בַּיּוֹם֙ הַשִּׁשִּׁ֔י נָשִׂ֖יא לִבְנֵ֣י גָ֑ד אֶלְיָסָ֖ף בֶּן־דְּעוּאֵֽל׃ קׇרְבָּנ֞וֹ קַֽעֲרַת־כֶּ֣סֶף אַחַ֗ת שְׁלֹשִׁ֣ים וּמֵאָה֮ מִשְׁקָלָהּ֒ מִזְרָ֤ק אֶחָד֙ כֶּ֔סֶף שִׁבְעִ֥ים שֶׁ֖קֶל בְּשֶׁ֣קֶל הַקֹּ֑דֶשׁ שְׁנֵיהֶ֣ם ׀ מְלֵאִ֗ים סֹ֛לֶת בְּלוּלָ֥ה בַשֶּׁ֖מֶן לְמִנְחָֽה׃ כַּ֥ף אַחַ֛ת עֲשָׂרָ֥ה זָהָ֖ב מְלֵאָ֥ה קְטֹֽרֶת׃ פַּ֣ר אֶחָ֞ד בֶּן־בָּקָ֗ר אַ֧יִל אֶחָ֛ד כֶּֽבֶשׂ־אֶחָ֥ד בֶּן־שְׁנָת֖וֹ לְעֹלָֽה׃ שְׂעִיר־עִזִּ֥ים אֶחָ֖ד לְחַטָּֽאת׃ וּלְזֶ֣בַח הַשְּׁלָמִים֮ בָּקָ֣ר שְׁנַ֒יִם֒ אֵילִ֤ם חֲמִשָּׁה֙ עַתֻּדִ֣ים חֲמִשָּׁ֔ה כְּבָשִׂ֥ים בְּנֵי־שָׁנָ֖ה חֲמִשָּׁ֑ה זֶ֛ה קׇרְבַּ֥ן אֶלְיָסָ֖ף בֶּן־דְּעוּאֵֽל׃ {פ} בַּיּוֹם֙ הַשְּׁבִיעִ֔י נָשִׂ֖יא לִבְנֵ֣י אֶפְרָ֑יִם אֱלִֽישָׁמָ֖ע בֶּן־עַמִּיהֽוּד׃ קׇרְבָּנ֞וֹ קַֽעֲרַת־כֶּ֣סֶף אַחַ֗ת שְׁלֹשִׁ֣ים וּמֵאָה֮ מִשְׁקָלָהּ֒ מִזְרָ֤ק אֶחָד֙ כֶּ֔סֶף שִׁבְעִ֥ים שֶׁ֖קֶל בְּשֶׁ֣קֶל הַקֹּ֑דֶשׁ שְׁנֵיהֶ֣ם ׀ מְלֵאִ֗ים סֹ֛לֶת בְּלוּלָ֥ה בַשֶּׁ֖מֶן לְמִנְחָֽה׃ כַּ֥ף אַחַ֛ת עֲשָׂרָ֥ה זָהָ֖ב מְלֵאָ֥ה קְטֹֽרֶת׃ פַּ֣ר אֶחָ֞ד בֶּן־בָּקָ֗ר אַ֧יִל אֶחָ֛ד כֶּֽבֶשׂ־אֶחָ֥ד בֶּן־שְׁנָת֖וֹ לְעֹלָֽה׃ שְׂעִיר־עִזִּ֥ים אֶחָ֖ד לְחַטָּֽאת׃ וּלְזֶ֣בַח הַשְּׁלָמִים֮ בָּקָ֣ר שְׁנַ֒יִם֒ אֵילִ֤ם חֲמִשָּׁה֙ עַתֻּדִ֣ים חֲמִשָּׁ֔ה כְּבָשִׂ֥ים בְּנֵֽי־שָׁנָ֖ה חֲמִשָּׁ֑ה זֶ֛ה קׇרְבַּ֥ן אֱלִישָׁמָ֖ע בֶּן־עַמִּיהֽוּד׃ {פ} בַּיּוֹם֙ הַשְּׁמִינִ֔י נָשִׂ֖יא לִבְנֵ֣י מְנַשֶּׁ֑ה גַּמְלִיאֵ֖ל בֶּן־פְּדָהצֽוּר׃ קׇרְבָּנ֞וֹ קַֽעֲרַת־כֶּ֣סֶף אַחַ֗ת שְׁלֹשִׁ֣ים וּמֵאָה֮ מִשְׁקָלָהּ֒ מִזְרָ֤ק אֶחָד֙ כֶּ֔סֶף שִׁבְעִ֥ים שֶׁ֖קֶל בְּשֶׁ֣קֶל הַקֹּ֑דֶשׁ שְׁנֵיהֶ֣ם ׀ מְלֵאִ֗ים סֹ֛לֶת בְּלוּלָ֥ה בַשֶּׁ֖מֶן לְמִנְחָֽה׃ כַּ֥ף אַחַ֛ת עֲשָׂרָ֥ה זָהָ֖ב מְלֵאָ֥ה קְטֹֽרֶת׃ פַּ֣ר אֶחָ֞ד בֶּן־בָּקָ֗ר אַ֧יִל אֶחָ֛ד כֶּֽבֶשׂ־אֶחָ֥ד בֶּן־שְׁנָת֖וֹ לְעֹלָֽה׃ שְׂעִיר־עִזִּ֥ים אֶחָ֖ד לְחַטָּֽאת׃ וּלְזֶ֣בַח הַשְּׁלָמִים֮ בָּקָ֣ר שְׁנַ֒יִם֒ אֵילִ֤ם חֲמִשָּׁה֙ עַתֻּדִ֣ים חֲמִשָּׁ֔ה כְּבָשִׂ֥ים בְּנֵי־שָׁנָ֖ה חֲמִשָּׁ֑ה זֶ֛ה קׇרְבַּ֥ן גַּמְלִיאֵ֖ל בֶּן־פְּדָהצֽוּר׃ {פ} בַּיּוֹם֙ הַתְּשִׁיעִ֔י נָשִׂ֖יא לִבְנֵ֣י בִנְיָמִ֑ן אֲבִידָ֖ן בֶּן־גִּדְעֹנִֽי׃ קׇרְבָּנ֞וֹ קַֽעֲרַת־כֶּ֣סֶף אַחַ֗ת שְׁלֹשִׁ֣ים וּמֵאָה֮ מִשְׁקָלָהּ֒ מִזְרָ֤ק אֶחָד֙ כֶּ֔סֶף שִׁבְעִ֥ים שֶׁ֖קֶל בְּשֶׁ֣קֶל הַקֹּ֑דֶשׁ שְׁנֵיהֶ֣ם ׀ מְלֵאִ֗ים סֹ֛לֶת בְּלוּלָ֥ה בַשֶּׁ֖מֶן לְמִנְחָֽה׃ כַּ֥ף אַחַ֛ת עֲשָׂרָ֥ה זָהָ֖ב מְלֵאָ֥ה קְטֹֽרֶת׃ פַּ֣ר אֶחָ֞ד בֶּן־בָּקָ֗ר אַ֧יִל אֶחָ֛ד כֶּֽבֶשׂ־אֶחָ֥ד בֶּן־שְׁנָת֖וֹ לְעֹלָֽה׃ שְׂעִיר־עִזִּ֥ים אֶחָ֖ד לְחַטָּֽאת׃ וּלְזֶ֣בַח הַשְּׁלָמִים֮ בָּקָ֣ר שְׁנַ֒יִם֒ אֵילִ֤ם חֲמִשָּׁה֙ עַתֻּדִ֣ים חֲמִשָּׁ֔ה כְּבָשִׂ֥ים בְּנֵי־שָׁנָ֖ה חֲמִשָּׁ֑ה זֶ֛ה קׇרְבַּ֥ן אֲבִידָ֖ן בֶּן־גִּדְעֹנִֽי׃ {פ} בַּיּוֹם֙ הָעֲשִׂירִ֔י נָשִׂ֖יא לִבְנֵ֣י דָ֑ן אֲחִיעֶ֖זֶר בֶּן־עַמִּישַׁדָּֽי׃ קׇרְבָּנ֞וֹ קַֽעֲרַת־כֶּ֣סֶף אַחַ֗ת שְׁלֹשִׁ֣ים וּמֵאָה֮ מִשְׁקָלָהּ֒ מִזְרָ֤ק אֶחָד֙ כֶּ֔סֶף שִׁבְעִ֥ים שֶׁ֖קֶל בְּשֶׁ֣קֶל הַקֹּ֑דֶשׁ שְׁנֵיהֶ֣ם ׀ מְלֵאִ֗ים סֹ֛לֶת בְּלוּלָ֥ה בַשֶּׁ֖מֶן לְמִנְחָֽה׃ כַּ֥ף אַחַ֛ת עֲשָׂרָ֥ה זָהָ֖ב מְלֵאָ֥ה קְטֹֽרֶת׃ פַּ֣ר אֶחָ֞ד בֶּן־בָּקָ֗ר אַ֧יִל אֶחָ֛ד כֶּֽבֶשׂ־אֶחָ֥ד בֶּן־שְׁנָת֖וֹ לְעֹלָֽה׃ שְׂעִיר־עִזִּ֥ים אֶחָ֖ד לְחַטָּֽאת׃ וּלְזֶ֣בַח הַשְּׁלָמִים֮ בָּקָ֣ר שְׁנַ֒יִם֒ אֵילִ֤ם חֲמִשָּׁה֙ עַתֻּדִ֣ים חֲמִשָּׁ֔ה כְּבָשִׂ֥ים בְּנֵֽי־שָׁנָ֖ה חֲמִשָּׁ֑ה זֶ֛ה קׇרְבַּ֥ן אֲחִיעֶ֖זֶר בֶּן־עַמִּישַׁדָּֽי׃ {פ} בְּיוֹם֙ עַשְׁתֵּ֣י עָשָׂ֣ר י֔וֹם נָשִׂ֖יא לִבְנֵ֣י אָשֵׁ֑ר פַּגְעִיאֵ֖ל בֶּן־עׇכְרָֽן׃ קׇרְבָּנ֞וֹ קַֽעֲרַת־כֶּ֣סֶף אַחַ֗ת שְׁלֹשִׁ֣ים וּמֵאָה֮ מִשְׁקָלָהּ֒ מִזְרָ֤ק אֶחָד֙ כֶּ֔סֶף שִׁבְעִ֥ים שֶׁ֖קֶל בְּשֶׁ֣קֶל הַקֹּ֑דֶשׁ שְׁנֵיהֶ֣ם ׀ מְלֵאִ֗ים סֹ֛לֶת בְּלוּלָ֥ה בַשֶּׁ֖מֶן לְמִנְחָֽה׃ כַּ֥ף אַחַ֛ת עֲשָׂרָ֥ה זָהָ֖ב מְלֵאָ֥ה קְטֹֽרֶת׃ פַּ֣ר אֶחָ֞ד בֶּן־בָּקָ֗ר אַ֧יִל אֶחָ֛ד כֶּֽבֶשׂ־אֶחָ֥ד בֶּן־שְׁנָת֖וֹ לְעֹלָֽה׃ שְׂעִיר־עִזִּ֥ים אֶחָ֖ד לְחַטָּֽאת׃ וּלְזֶ֣בַח הַשְּׁלָמִים֮ בָּקָ֣ר שְׁנַ֒יִם֒ אֵילִ֤ם חֲמִשָּׁה֙ עַתֻּדִ֣ים חֲמִשָּׁ֔ה כְּבָשִׂ֥ים בְּנֵֽי־שָׁנָ֖ה חֲמִשָּׁ֑ה זֶ֛ה קׇרְבַּ֥ן פַּגְעִיאֵ֖ל בֶּן־עׇכְרָֽן׃ {פ} בְּיוֹם֙ שְׁנֵ֣ים עָשָׂ֣ר י֔וֹם נָשִׂ֖יא לִבְנֵ֣י נַפְתָּלִ֑י אֲחִירַ֖ע בֶּן־עֵינָֽן׃ קׇרְבָּנ֞וֹ קַֽעֲרַת־כֶּ֣סֶף אַחַ֗ת שְׁלֹשִׁ֣ים וּמֵאָה֮ מִשְׁקָלָהּ֒ מִזְרָ֤ק אֶחָד֙ כֶּ֔סֶף שִׁבְעִ֥ים שֶׁ֖קֶל בְּשֶׁ֣קֶל הַקֹּ֑דֶשׁ שְׁנֵיהֶ֣ם ׀ מְלֵאִ֗ים סֹ֛לֶת בְּלוּלָ֥ה בַשֶּׁ֖מֶן לְמִנְחָֽה׃ כַּ֥ף אַחַ֛ת עֲשָׂרָ֥ה זָהָ֖ב מְלֵאָ֥ה קְטֹֽרֶת׃ פַּ֣ר אֶחָ֞ד בֶּן־בָּקָ֗ר אַ֧יִל אֶחָ֛ד כֶּֽבֶשׂ־אֶחָ֥ד בֶּן־שְׁנָת֖וֹ לְעֹלָֽה׃ שְׂעִיר־עִזִּ֥ים אֶחָ֖ד לְחַטָּֽאת׃ וּלְזֶ֣בַח הַשְּׁלָמִים֮ בָּקָ֣ר שְׁנַ֒יִם֒ אֵילִ֤ם חֲמִשָּׁה֙ עַתֻּדִ֣ים חֲמִשָּׁ֔ה כְּבָשִׂ֥ים בְּנֵֽי־שָׁנָ֖ה חֲמִשָּׁ֑ה זֶ֛ה קׇרְבַּ֥ן אֲחִירַ֖ע בֶּן־עֵינָֽן׃ {פ} זֹ֣את ׀ חֲנֻכַּ֣ת הַמִּזְבֵּ֗חַ בְּיוֹם֙ הִמָּשַׁ֣ח אֹת֔וֹ מֵאֵ֖ת נְשִׂיאֵ֣י יִשְׂרָאֵ֑ל קַעֲרֹ֨ת כֶּ֜סֶף שְׁתֵּ֣ים עֶשְׂרֵ֗ה מִֽזְרְקֵי־כֶ֙סֶף֙ שְׁנֵ֣ים עָשָׂ֔ר כַּפּ֥וֹת זָהָ֖ב שְׁתֵּ֥ים עֶשְׂרֵֽה׃ שְׁלֹשִׁ֣ים וּמֵאָ֗ה הַקְּעָרָ֤ה הָֽאַחַת֙ כֶּ֔סֶף וְשִׁבְעִ֖ים הַמִּזְרָ֣ק הָאֶחָ֑ד כֹּ֚ל כֶּ֣סֶף הַכֵּלִ֔ים אַלְפַּ֥יִם וְאַרְבַּע־מֵא֖וֹת בְּשֶׁ֥קֶל הַקֹּֽדֶשׁ׃ כַּפּ֨וֹת זָהָ֤ב שְׁתֵּים־עֶשְׂרֵה֙ מְלֵאֹ֣ת קְטֹ֔רֶת עֲשָׂרָ֧ה עֲשָׂרָ֛ה הַכַּ֖ף בְּשֶׁ֣קֶל הַקֹּ֑דֶשׁ כׇּל־זְהַ֥ב הַכַּפּ֖וֹת עֶשְׂרִ֥ים וּמֵאָֽה׃ כׇּל־הַבָּקָ֨ר לָעֹלָ֜ה שְׁנֵ֧ים עָשָׂ֣ר פָּרִ֗ים אֵילִ֤ם שְׁנֵים־עָשָׂר֙ כְּבָשִׂ֧ים בְּנֵֽי־שָׁנָ֛ה שְׁנֵ֥ים עָשָׂ֖ר וּמִנְחָתָ֑ם וּשְׂעִירֵ֥י עִזִּ֛ים שְׁנֵ֥ים עָשָׂ֖ר לְחַטָּֽאת׃ וְכֹ֞ל בְּקַ֣ר ׀ זֶ֣בַח הַשְּׁלָמִ֗ים עֶשְׂרִ֣ים וְאַרְבָּעָה֮ פָּרִים֒ אֵילִ֤ם שִׁשִּׁים֙ עַתֻּדִ֣ים שִׁשִּׁ֔ים כְּבָשִׂ֥ים בְּנֵי־שָׁנָ֖ה שִׁשִּׁ֑ים זֹ֚את חֲנֻכַּ֣ת הַמִּזְבֵּ֔חַ אַחֲרֵ֖י הִמָּשַׁ֥ח אֹתֽוֹ׃ וּבְבֹ֨א מֹשֶׁ֜ה אֶל־אֹ֣הֶל מוֹעֵד֮ לְדַבֵּ֣ר אִתּוֹ֒ וַיִּשְׁמַ֨ע אֶת־הַקּ֜וֹל מִדַּבֵּ֣ר אֵלָ֗יו מֵעַ֤ל הַכַּפֹּ֙רֶת֙ אֲשֶׁר֙ עַל־אֲרֹ֣ן הָעֵדֻ֔ת מִבֵּ֖ין שְׁנֵ֣י הַכְּרֻבִ֑ים וַיְדַבֵּ֖ר אֵלָֽיו׃ {פ} וַיְדַבֵּ֥ר יְיָ֖ אֶל־מֹשֶׁ֥ה לֵּאמֹֽר׃ דַּבֵּר֙ אֶֽל־אַהֲרֹ֔ן וְאָמַרְתָּ֖ אֵלָ֑יו בְּהַעֲלֹֽתְךָ֙ אֶת־הַנֵּרֹ֔ת אֶל־מוּל֙ פְּנֵ֣י הַמְּנוֹרָ֔ה יָאִ֖ירוּ שִׁבְעַ֥ת הַנֵּרֽוֹת׃ וַיַּ֤עַשׂ כֵּן֙ אַהֲרֹ֔ן אֶל־מוּל֙ פְּנֵ֣י הַמְּנוֹרָ֔ה הֶעֱלָ֖ה נֵרֹתֶ֑יהָ כַּֽאֲשֶׁ֛ר צִוָּ֥ה יְיָ֖ אֶת־מֹשֶֽׁה׃ וְזֶ֨ה מַעֲשֵׂ֤ה הַמְּנֹרָה֙ מִקְשָׁ֣ה זָהָ֔ב עַד־יְרֵכָ֥הּ עַד־פִּרְחָ֖הּ מִקְשָׁ֣ה הִ֑וא כַּמַּרְאֶ֗ה אֲשֶׁ֨ר הֶרְאָ֤ה יְיָ֙ אֶת־מֹשֶׁ֔ה כֵּ֥ן עָשָׂ֖ה אֶת־הַמְּנֹרָֽה׃ 


\section[פורים]{קריאה לפורים}


\tsource{שמות יז}\\
וַיָּבֹ֖א עֲמָלֵ֑ק וַיִּלָּ֥חֶם עִם־יִשְׂרָאֵ֖ל בִּרְפִידִֽם׃ וַיֹּ֨אמֶר מֹשֶׁ֤ה אֶל־יְהוֹשֻׁ֙עַ֙ בְּחַר־לָ֣נוּ אֲנָשִׁ֔ים וְצֵ֖א הִלָּחֵ֣ם בַּעֲמָלֵ֑ק מָחָ֗ר אָנֹכִ֤י נִצָּב֙ עַל־רֹ֣אשׁ הַגִּבְעָ֔ה וּמַטֵּ֥ה הָאֱלֹהִ֖ים בְּיָדִֽי׃ וַיַּ֣עַשׂ יְהוֹשֻׁ֗עַ כַּאֲשֶׁ֤ר אָֽמַר־לוֹ֙ מֹשֶׁ֔ה לְהִלָּחֵ֖ם בַּעֲמָלֵ֑ק וּמֹשֶׁה֙ אַהֲרֹ֣ן וְח֔וּר עָל֖וּ רֹ֥אשׁ הַגִּבְעָֽה׃
\aliyah{לוי}
וְהָיָ֗ה כַּאֲשֶׁ֨ר יָרִ֥ים מֹשֶׁ֛ה יָד֖וֹ וְגָבַ֣ר יִשְׂרָאֵ֑ל וְכַאֲשֶׁ֥ר יָנִ֛יחַ יָד֖וֹ וְגָבַ֥ר עֲמָלֵֽק׃ וִידֵ֤י מֹשֶׁה֙ כְּבֵדִ֔ים וַיִּקְחוּ־אֶ֛בֶן וַיָּשִׂ֥ימוּ תַחְתָּ֖יו וַיֵּ֣שֶׁב עָלֶ֑יהָ וְאַהֲרֹ֨ן וְח֜וּר תָּֽמְכ֣וּ בְיָדָ֗יו מִזֶּ֤ה אֶחָד֙ וּמִזֶּ֣ה אֶחָ֔ד וַיְהִ֥י יָדָ֛יו אֱמוּנָ֖ה עַד־בֹּ֥א הַשָּֽׁמֶשׁ׃ וַיַּחֲלֹ֧שׁ יְהוֹשֻׁ֛עַ אֶת־עֲמָלֵ֥ק וְאֶת־עַמּ֖וֹ לְפִי־חָֽרֶב׃ 
\aliyah{ישראל}
וַיֹּ֨אמֶר יְיָ֜ אֶל־מֹשֶׁ֗ה כְּתֹ֨ב זֹ֤את זִכָּרוֹן֙ בַּסֵּ֔פֶר וְשִׂ֖ים בְּאׇזְנֵ֣י יְהוֹשֻׁ֑עַ כִּֽי־מָחֹ֤ה אֶמְחֶה֙ אֶת־זֵ֣כֶר עֲמָלֵ֔ק מִתַּ֖חַת הַשָּׁמָֽיִם׃ וַיִּ֥בֶן מֹשֶׁ֖ה מִזְבֵּ֑חַ וַיִּקְרָ֥א שְׁמ֖וֹ יְיָ֥ ׀ נִסִּֽי׃ וַיֹּ֗אמֶר כִּֽי־יָד֙ עַל־כֵּ֣ס יָ֔הּ מִלְחָמָ֥ה לַייָ֖ בַּֽעֲמָלֵ֑ק מִדֹּ֖ר דֹּֽר׃ 


\section[תשעה באב]{קריאה לתשעה באב}


	\tsource{דברים ד}\\
כִּֽי־תוֹלִ֤יד בָּנִים֙ וּבְנֵ֣י בָנִ֔ים וְנוֹשַׁנְתֶּ֖ם בָּאָ֑רֶץ וְהִשְׁחַתֶּ֗ם וַעֲשִׂ֤יתֶם פֶּ֙סֶל֙ תְּמ֣וּנַת כֹּ֔ל וַעֲשִׂיתֶ֥ם הָרַ֛ע בְּעֵינֵ֥י יְיָ־אֱלֹהֶ֖יךָ לְהַכְעִיסֽוֹ׃ הַעִידֹ֩תִי֩ בָכֶ֨ם הַיּ֜וֹם אֶת־הַשָּׁמַ֣יִם וְאֶת־הָאָ֗רֶץ כִּֽי־אָבֹ֣ד תֹּאבֵדוּן֮ מַהֵר֒ מֵעַ֣ל הָאָ֔רֶץ אֲשֶׁ֨ר אַתֶּ֜ם עֹבְרִ֧ים אֶת־הַיַּרְדֵּ֛ן שָׁ֖מָּה לְרִשְׁתָּ֑הּ לֹֽא־תַאֲרִיכֻ֤ן יָמִים֙ עָלֶ֔יהָ כִּ֥י הִשָּׁמֵ֖ד תִּשָּׁמֵדֽוּן׃ וְהֵפִ֧יץ יְיָ֛ אֶתְכֶ֖ם בָּעַמִּ֑ים וְנִשְׁאַרְתֶּם֙ מְתֵ֣י מִסְפָּ֔ר בַּגּוֹיִ֕ם אֲשֶׁ֨ר יְנַהֵ֧ג יְיָ֛ אֶתְכֶ֖ם שָֽׁמָּה׃ וַעֲבַדְתֶּם־שָׁ֣ם אֱלֹהִ֔ים מַעֲשֵׂ֖ה יְדֵ֣י אָדָ֑ם עֵ֣ץ וָאֶ֔בֶן אֲשֶׁ֤ר לֹֽא־יִרְאוּן֙ וְלֹ֣א יִשְׁמְע֔וּן וְלֹ֥א יֹֽאכְל֖וּן וְלֹ֥א יְרִיחֻֽן׃ וּבִקַּשְׁתֶּ֥ם מִשָּׁ֛ם אֶת־יְיָ֥ אֱלֹהֶ֖יךָ וּמָצָ֑אתָ כִּ֣י תִדְרְשֶׁ֔נּוּ בְּכׇל־לְבָבְךָ֖ וּבְכׇל־נַפְשֶֽׁךָ׃
	\aliyah{לוי}
בַּצַּ֣ר לְךָ֔ וּמְצָא֕וּךָ כֹּ֖ל הַדְּבָרִ֣ים הָאֵ֑לֶּה בְּאַחֲרִית֙ הַיָּמִ֔ים וְשַׁבְתָּ֙ עַד־יְיָ֣ אֱלֹהֶ֔יךָ וְשָׁמַעְתָּ֖ בְּקֹלֽוֹ׃ כִּ֣י אֵ֤ל רַחוּם֙ יְיָ֣ אֱלֹהֶ֔יךָ לֹ֥א יַרְפְּךָ֖ וְלֹ֣א יַשְׁחִיתֶ֑ךָ וְלֹ֤א יִשְׁכַּח֙ אֶת־בְּרִ֣ית אֲבֹתֶ֔יךָ אֲשֶׁ֥ר נִשְׁבַּ֖ע לָהֶֽם׃ כִּ֣י שְׁאַל־נָא֩ לְיָמִ֨ים רִֽאשֹׁנִ֜ים אֲשֶׁר־הָי֣וּ לְפָנֶ֗יךָ לְמִן־הַיּוֹם֙ אֲשֶׁר֩ בָּרָ֨א אֱלֹהִ֤ים ׀ אָדָם֙ עַל־הָאָ֔רֶץ וּלְמִקְצֵ֥ה הַשָּׁמַ֖יִם וְעַד־קְצֵ֣ה הַשָּׁמָ֑יִם הֲנִֽהְיָ֗ה כַּדָּבָ֤ר הַגָּדוֹל֙ הַזֶּ֔ה א֖וֹ הֲנִשְׁמַ֥ע כָּמֹֽהוּ׃ הֲשָׁ֣מַֽע עָם֩ ק֨וֹל אֱלֹהִ֜ים מְדַבֵּ֧ר מִתּוֹךְ־הָאֵ֛שׁ כַּאֲשֶׁר־שָׁמַ֥עְתָּ אַתָּ֖ה וַיֶּֽחִי׃ א֣וֹ ׀ הֲנִסָּ֣ה אֱלֹהִ֗ים לָ֠ב֠וֹא לָקַ֨חַת ל֣וֹ גוֹי֮ מִקֶּ֣רֶב גּוֹי֒ בְּמַסֹּת֩ בְּאֹתֹ֨ת וּבְמוֹפְתִ֜ים וּבְמִלְחָמָ֗ה וּבְיָ֤ד חֲזָקָה֙ וּבִזְר֣וֹעַ נְטוּיָ֔ה וּבְמוֹרָאִ֖ים גְּדֹלִ֑ים כְּ֠כֹ֠ל אֲשֶׁר־עָשָׂ֨ה לָכֶ֜ם יְיָ֧ אֱלֹהֵיכֶ֛ם בְּמִצְרַ֖יִם לְעֵינֶֽיךָ׃ אַתָּה֙ הׇרְאֵ֣תָ לָדַ֔עַת כִּ֥י יְיָ֖ ה֣וּא הָאֱלֹהִ֑ים אֵ֥ין ע֖וֹד מִלְּבַדּֽוֹ׃
	\aliyah{ישראל}
מִן־הַשָּׁמַ֛יִם הִשְׁמִֽיעֲךָ֥ אֶת־קֹל֖וֹ לְיַסְּרֶ֑ךָּ וְעַל־הָאָ֗רֶץ הֶרְאֲךָ֙ אֶת־אִשּׁ֣וֹ הַגְּדוֹלָ֔ה וּדְבָרָ֥יו שָׁמַ֖עְתָּ מִתּ֥וֹךְ הָאֵֽשׁ׃ וְתַ֗חַת כִּ֤י אָהַב֙ אֶת־אֲבֹתֶ֔יךָ וַיִּבְחַ֥ר בְּזַרְע֖וֹ אַחֲרָ֑יו וַיּוֹצִֽאֲךָ֧ בְּפָנָ֛יו בְּכֹח֥וֹ הַגָּדֹ֖ל מִמִּצְרָֽיִם׃ לְהוֹרִ֗ישׁ גּוֹיִ֛ם גְּדֹלִ֧ים וַעֲצֻמִ֛ים מִמְּךָ֖ מִפָּנֶ֑יךָ לַהֲבִֽיאֲךָ֗ לָֽתֶת־לְךָ֧ אֶת־אַרְצָ֛ם נַחֲלָ֖ה כַּיּ֥וֹם הַזֶּֽה׃ וְיָדַעְתָּ֣ הַיּ֗וֹם וַהֲשֵׁבֹתָ֮ אֶל־לְבָבֶ֒ךָ֒ כִּ֤י יְיָ֙ ה֣וּא הָֽאֱלֹהִ֔ים בַּשָּׁמַ֣יִם מִמַּ֔עַל וְעַל־הָאָ֖רֶץ מִתָּ֑חַת אֵ֖ין עֽוֹד׃ וְשָׁמַרְתָּ֞ אֶת־חֻקָּ֣יו וְאֶת־מִצְוֺתָ֗יו אֲשֶׁ֨ר אָנֹכִ֤י מְצַוְּךָ֙ הַיּ֔וֹם אֲשֶׁר֙ יִיטַ֣ב לְךָ֔ וּלְבָנֶ֖יךָ אַחֲרֶ֑יךָ וּלְמַ֨עַן תַּאֲרִ֤יךְ יָמִים֙ עַל־הָ֣אֲדָמָ֔ה אֲשֶׁ֨ר יְיָ֧ אֱלֹהֶ֛יךָ נֹתֵ֥ן לְךָ֖ כׇּל־הַיָּמִֽים׃ 




\section*{הפטרה לתשעה באב}


	\tsource{ירמיה ח-ט} \\
אָסֹ֥ף אֲסִיפֵ֖ם נְאֻם־יְיָ֑ אֵין֩ עֲנָבִ֨ים בַּגֶּ֜פֶן וְאֵ֧ין תְּאֵנִ֣ים בַּתְּאֵנָ֗ה וְהֶעָלֶה֙ נָבֵ֔ל וָאֶתֵּ֥ן לָהֶ֖ם יַעַבְרֽוּם׃ עַל־מָה֙ אֲנַ֣חְנוּ יֹֽשְׁבִ֔ים הֵאָסְפ֗וּ וְנָב֛וֹא אֶל־עָרֵ֥י הַמִּבְצָ֖ר וְנִדְּמָה־שָּׁ֑ם כִּי֩ יְיָ֨ אֱלֹהֵ֤ינוּ הֲדִמָּ֙נוּ֙ וַיַּשְׁקֵ֣נוּ מֵי־רֹ֔אשׁ כִּ֥י חָטָ֖אנוּ לַייָ׃ קַוֵּ֥ה לְשָׁל֖וֹם וְאֵ֣ין ט֑וֹב לְעֵ֥ת מַרְפֵּ֖ה וְהִנֵּ֥ה בְעָתָֽה׃ מִדָּ֤ן נִשְׁמַע֙ נַחְרַ֣ת סוּסָ֔יו מִקּוֹל֙ מִצְהֲל֣וֹת אַבִּירָ֔יו רָעֲשָׁ֖ה כׇּל־הָאָ֑רֶץ וַיָּב֗וֹאוּ וַיֹּֽאכְלוּ֙ אֶ֣רֶץ וּמְלוֹאָ֔הּ עִ֖יר וְיֹ֥שְׁבֵי בָֽהּ׃ {פ} כִּי֩ הִנְנִ֨י מְשַׁלֵּ֜חַ בָּכֶ֗ם נְחָשִׁים֙ צִפְעֹנִ֔ים אֲשֶׁ֥ר אֵין־לָהֶ֖ם לָ֑חַשׁ וְנִשְּׁכ֥וּ אֶתְכֶ֖ם נְאֻם־יְיָ׃ {ס} מַבְלִ֥יגִיתִ֖י עֲלֵ֣י יָג֑וֹן עָלַ֖י לִבִּ֥י דַוָּֽי׃ הִנֵּה־ק֞וֹל שַֽׁוְעַ֣ת בַּת־עַמִּ֗י מֵאֶ֙רֶץ֙ מַרְחַקִּ֔ים הַֽייָ֙ אֵ֣ין בְּצִיּ֔וֹן אִם־מַלְכָּ֖הּ אֵ֣ין בָּ֑הּ מַדּ֗וּעַ הִכְעִס֛וּנִי בִּפְסִלֵיהֶ֖ם בְּהַבְלֵ֥י נֵכָֽר׃ עָבַ֥ר קָצִ֖יר כָּ֣לָה קָ֑יִץ וַאֲנַ֖חְנוּ ל֥וֹא נוֹשָֽׁעְנוּ׃ עַל־שֶׁ֥בֶר בַּת־עַמִּ֖י הׇשְׁבָּ֑רְתִּי קָדַ֕רְתִּי שַׁמָּ֖ה הֶחֱזִקָֽתְנִי׃ הַצֳּרִי֙ אֵ֣ין בְּגִלְעָ֔ד אִם־רֹפֵ֖א אֵ֣ין שָׁ֑ם כִּ֗י מַדּ֙וּעַ֙ לֹ֣א עָֽלְתָ֔ה אֲרֻכַ֖ת בַּת־עַמִּֽי׃ {ס} מִֽי־יִתֵּ֤ן רֹאשִׁי֙ מַ֔יִם וְעֵינִ֖י מְק֣וֹר דִּמְעָ֑ה וְאֶבְכֶּה֙ יוֹמָ֣ם וָלַ֔יְלָה אֵ֖ת חַֽלְלֵ֥י בַת־עַמִּֽי׃ {ס} מִֽי־יִתְּנֵ֣נִי בַמִּדְבָּ֗ר מְלוֹן֙ אֹֽרְחִ֔ים וְאֶֽעֶזְבָה֙ אֶת־עַמִּ֔י וְאֵלְכָ֖ה מֵֽאִתָּ֑ם כִּ֤י כֻלָּם֙ מְנָ֣אֲפִ֔ים עֲצֶ֖רֶת בֹּגְדִֽים׃ וַֽיַּדְרְכ֤וּ אֶת־לְשׁוֹנָם֙ קַשְׁתָּ֣ם שֶׁ֔קֶר וְלֹ֥א לֶאֱמוּנָ֖ה גָּבְר֣וּ בָאָ֑רֶץ כִּי֩ מֵרָעָ֨ה אֶל־רָעָ֧ה ׀ יָצָ֛אוּ וְאֹתִ֥י לֹא־יָדָ֖עוּ נְאֻם־יְיָ׃ אִ֤ישׁ מֵרֵעֵ֙הוּ֙ הִשָּׁמֵ֔רוּ וְעַל־כׇּל־אָ֖ח אַל־תִּבְטָ֑חוּ כִּ֤י כׇל־אָח֙ עָק֣וֹב יַעְקֹ֔ב וְכׇל־רֵ֖עַ רָכִ֥יל יַהֲלֹֽךְ׃ וְאִ֤ישׁ בְּרֵעֵ֙הוּ֙ יְהָתֵ֔לּוּ וֶאֱמֶ֖ת לֹ֣א יְדַבֵּ֑רוּ לִמְּד֧וּ לְשׁוֹנָ֛ם דַּבֶּר־שֶׁ֖קֶר הַעֲוֵ֥ה נִלְאֽוּ׃ שִׁבְתְּךָ֖ בְּת֣וֹךְ מִרְמָ֑ה בְּמִרְמָ֛ה מֵאֲנ֥וּ דַעַת־אוֹתִ֖י נְאֻם־יְיָ׃ {ס} לָכֵ֗ן כֹּ֤ה אָמַר֙ יְיָ֣ צְבָא֔וֹת הִנְנִ֥י צוֹרְפָ֖ם וּבְחַנְתִּ֑ים כִּי־אֵ֣יךְ אֶֽעֱשֶׂ֔ה מִפְּנֵ֖י בַּת־עַמִּֽי׃ חֵ֥ץ (שוחט) [שָׁח֛וּט] לְשׁוֹנָ֖ם מִרְמָ֣ה דִבֵּ֑ר בְּפִ֗יו שָׁל֤וֹם אֶת־רֵעֵ֙הוּ֙ יְדַבֵּ֔ר וּבְקִרְבּ֖וֹ יָשִׂ֥ים אׇרְבּֽוֹ׃ הַֽעַל־אֵ֥לֶּה לֹא־אֶפְקׇד־בָּ֖ם נְאֻם־יְיָ֑ אִ֚ם בְּג֣וֹי אֲשֶׁר־כָּזֶ֔ה לֹ֥א תִתְנַקֵּ֖ם נַפְשִֽׁי׃ {ס} עַל־הֶ֨הָרִ֜ים אֶשָּׂ֧א בְכִ֣י וָנֶ֗הִי וְעַל־נְא֤וֹת מִדְבָּר֙ קִינָ֔ה כִּ֤י נִצְּתוּ֙ מִבְּלִי־אִ֣ישׁ עֹבֵ֔ר וְלֹ֥א שָׁמְע֖וּ ק֣וֹל מִקְנֶ֑ה מֵע֤וֹף הַשָּׁמַ֙יִם֙ וְעַד־בְּהֵמָ֔ה נָדְד֖וּ הָלָֽכוּ׃ וְנָתַתִּ֧י אֶת־יְרוּשָׁלַ֛͏ִם לְגַלִּ֖ים מְע֣וֹן תַּנִּ֑ים וְאֶת־עָרֵ֧י יְהוּדָ֛ה אֶתֵּ֥ן שְׁמָמָ֖ה מִבְּלִ֖י יוֹשֵֽׁב׃ {ס} מִֽי־הָאִ֤ישׁ הֶחָכָם֙ וְיָבֵ֣ן אֶת־זֹ֔את וַאֲשֶׁ֨ר דִּבֶּ֧ר פִּֽי־יְיָ֛ אֵלָ֖יו וְיַגִּדָ֑הּ עַל־מָה֙ אָבְדָ֣ה הָאָ֔רֶץ נִצְּתָ֥ה כַמִּדְבָּ֖ר מִבְּלִ֖י עֹבֵֽר׃ {ס} וַיֹּ֣אמֶר יְיָ֔ עַל־עׇזְבָם֙ אֶת־תּ֣וֹרָתִ֔י אֲשֶׁ֥ר נָתַ֖תִּי לִפְנֵיהֶ֑ם וְלֹא־שָׁמְע֥וּ בְקוֹלִ֖י וְלֹא־הָ֥לְכוּ בָֽהּ׃ וַיֵּ֣לְכ֔וּ אַחֲרֵ֖י שְׁרִר֣וּת לִבָּ֑ם וְאַֽחֲרֵי֙ הַבְּעָלִ֔ים אֲשֶׁ֥ר לִמְּד֖וּם אֲבוֹתָֽם׃ {פ} לָכֵ֗ן כֹּה־אָמַ֞ר יְיָ֤ צְבָאוֹת֙ אֱלֹהֵ֣י יִשְׂרָאֵ֔ל הִנְנִ֧י מַאֲכִילָ֛ם אֶת־הָעָ֥ם הַזֶּ֖ה לַעֲנָ֑ה וְהִשְׁקִיתִ֖ים מֵי־רֹֽאשׁ׃ וַהֲפִֽצוֹתִים֙ בַּגּוֹיִ֔ם אֲשֶׁר֙ לֹ֣א יָדְע֔וּ הֵ֖מָּה וַאֲבוֹתָ֑ם וְשִׁלַּחְתִּ֤י אַֽחֲרֵיהֶם֙ אֶת־הַחֶ֔רֶב עַ֥ד כַּלּוֹתִ֖י אוֹתָֽם׃ {פ} כֹּ֤ה אָמַר֙ יְיָ֣ צְבָא֔וֹת הִתְבּ֥וֹנְנ֛וּ וְקִרְא֥וּ לַמְקוֹנְנ֖וֹת וּתְבוֹאֶ֑ינָה וְאֶל־הַחֲכָמ֥וֹת שִׁלְח֖וּ וְתָבֽוֹאנָה׃ וּתְמַהֵ֕רְנָה וְתִשֶּׂ֥נָה עָלֵ֖ינוּ נֶ֑הִי וְתֵרַ֤דְנָה עֵינֵ֙ינוּ֙ דִּמְעָ֔ה וְעַפְעַפֵּ֖ינוּ יִזְּלוּ־מָֽיִם׃ כִּ֣י ק֥וֹל נְהִ֛י נִשְׁמַ֥ע מִצִּיּ֖וֹן אֵ֣יךְ שֻׁדָּ֑דְנוּ בֹּ֤שְׁנֽוּ מְאֹד֙ כִּֽי־עָזַ֣בְנוּ אָ֔רֶץ כִּ֥י הִשְׁלִ֖יכוּ מִשְׁכְּנוֹתֵֽינוּ׃ {ס} כִּֽי־שְׁמַ֤עְנָה נָשִׁים֙ דְּבַר־יְיָ֔ וְתִקַּ֥ח אׇזְנְכֶ֖ם דְּבַר־פִּ֑יו וְלַמֵּ֤דְנָה בְנֽוֹתֵיכֶם֙ נֶ֔הִי וְאִשָּׁ֥ה רְעוּתָ֖הּ קִינָֽה׃ כִּי־עָ֤לָֽה מָ֙וֶת֙ בְּחַלּוֹנֵ֔ינוּ בָּ֖א בְּאַרְמְנוֹתֵ֑ינוּ לְהַכְרִ֤ית עוֹלָל֙ מִח֔וּץ בַּֽחוּרִ֖ים מֵרְחֹבֽוֹת׃ דַּבֵּ֗ר כֹּ֚ה נְאֻם־יְיָ֔ וְנָֽפְלָה֙ נִבְלַ֣ת הָאָדָ֔ם כְּדֹ֖מֶן עַל־פְּנֵ֣י הַשָּׂדֶ֑ה וּכְעָמִ֛יר מֵאַחֲרֵ֥י הַקֹּצֵ֖ר וְאֵ֥ין מְאַסֵּֽף׃ {ס} כֹּ֣ה ׀ אָמַ֣ר יְיָ֗ אַל־יִתְהַלֵּ֤ל חָכָם֙ בְּחׇכְמָת֔וֹ וְאַל־יִתְהַלֵּ֥ל הַגִּבּ֖וֹר בִּגְבֽוּרָת֑וֹ אַל־יִתְהַלֵּ֥ל עָשִׁ֖יר בְּעׇשְׁרֽוֹ׃ כִּ֣י אִם־בְּזֹ֞את יִתְהַלֵּ֣ל הַמִּתְהַלֵּ֗ל הַשְׂכֵּל֮ וְיָדֹ֣עַ אוֹתִי֒ כִּ֚י אֲנִ֣י יְיָ֔ עֹ֥שֶׂה חֶ֛סֶד מִשְׁפָּ֥ט וּצְדָקָ֖ה בָּאָ֑רֶץ כִּֽי־בְאֵ֥לֶּה חָפַ֖צְתִּי נְאֻם־יְיָ׃ 
	



\section[תענית ציבור]{קריאה לתענית ציבור}\label{torah taanis tzibbur}


	\tsource{שמות לב, לד}\\
וַיְחַ֣ל מֹשֶׁ֔ה אֶת־פְּנֵ֖י יְיָ֣ אֱלֹהָ֑יו וַיֹּ֗אמֶר לָמָ֤ה יְיָ֙ יֶחֱרֶ֤ה אַפְּךָ֙ בְּעַמֶּ֔ךָ אֲשֶׁ֤ר הוֹצֵ֙אתָ֙ מֵאֶ֣רֶץ מִצְרַ֔יִם בְּכֹ֥חַ גָּד֖וֹל וּבְיָ֥ד חֲזָקָֽה׃ לָ֩מָּה֩ יֹאמְר֨וּ מִצְרַ֜יִם לֵאמֹ֗ר בְּרָעָ֤ה הֽוֹצִיאָם֙ לַהֲרֹ֤ג אֹתָם֙ בֶּֽהָרִ֔ים וּ֨לְכַלֹּתָ֔ם מֵעַ֖ל פְּנֵ֣י הָֽאֲדָמָ֑ה שׁ֚וּב מֵחֲר֣וֹן אַפֶּ֔ךָ וְהִנָּחֵ֥ם עַל־הָרָעָ֖ה לְעַמֶּֽךָ׃ זְכֹ֡ר לְאַבְרָהָם֩ לְיִצְחָ֨ק וּלְיִשְׂרָאֵ֜ל עֲבָדֶ֗יךָ אֲשֶׁ֨ר נִשְׁבַּ֣עְתָּ לָהֶם֮ בָּךְ֒ וַתְּדַבֵּ֣ר אֲלֵהֶ֔ם אַרְבֶּה֙ אֶֽת־זַרְעֲכֶ֔ם כְּכוֹכְבֵ֖י הַשָּׁמָ֑יִם וְכׇל־הָאָ֨רֶץ הַזֹּ֜את אֲשֶׁ֣ר אָמַ֗רְתִּי אֶתֵּן֙ לְזַרְעֲכֶ֔ם וְנָחֲל֖וּ לְעֹלָֽם׃ וַיִּנָּ֖חֶם יְיָ֑ עַל־הָ֣רָעָ֔ה אֲשֶׁ֥ר דִּבֶּ֖ר לַעֲשׂ֥וֹת לְעַמּֽוֹ׃ 
\aliyah{לוי}
וַיֹּ֤אמֶר יְיָ֙ אֶל־מֹשֶׁ֔ה פְּסׇל־לְךָ֛ שְׁנֵֽי־לֻחֹ֥ת אֲבָנִ֖ים כָּרִאשֹׁנִ֑ים וְכָתַבְתִּי֙ עַל־הַלֻּחֹ֔ת אֶ֨ת־הַדְּבָרִ֔ים אֲשֶׁ֥ר הָי֛וּ עַל־הַלֻּחֹ֥ת הָרִאשֹׁנִ֖ים אֲשֶׁ֥ר שִׁבַּֽרְתָּ׃ וֶהְיֵ֥ה נָכ֖וֹן לַבֹּ֑קֶר וְעָלִ֤יתָ בַבֹּ֙קֶר֙ אֶל־הַ֣ר סִינַ֔י וְנִצַּבְתָּ֥ לִ֛י שָׁ֖ם עַל־רֹ֥אשׁ הָהָֽר׃ וְאִישׁ֙ לֹֽא־יַעֲלֶ֣ה עִמָּ֔ךְ וְגַם־אִ֥ישׁ אַל־יֵרָ֖א בְּכׇל־הָהָ֑ר גַּם־הַצֹּ֤אן וְהַבָּקָר֙ אַל־יִרְע֔וּ אֶל־מ֖וּל הָהָ֥ר הַהֽוּא׃
\aliyah{ישראל}
וַיִּפְסֹ֡ל שְׁנֵֽי־לֻחֹ֨ת אֲבָנִ֜ים כָּרִאשֹׁנִ֗ים וַיַּשְׁכֵּ֨ם מֹשֶׁ֤ה בַבֹּ֙קֶר֙ וַיַּ֙עַל֙ אֶל־הַ֣ר סִינַ֔י כַּאֲשֶׁ֛ר צִוָּ֥ה יְיָ֖ אֹת֑וֹ וַיִּקַּ֣ח בְּיָד֔וֹ שְׁנֵ֖י לֻחֹ֥ת אֲבָנִֽים׃ וַיֵּ֤רֶד יְיָ֙ בֶּֽעָנָ֔ן וַיִּתְיַצֵּ֥ב עִמּ֖וֹ שָׁ֑ם וַיִּקְרָ֥א בְשֵׁ֖ם יְיָ׃ וַיַּעֲבֹ֨ר יְיָ֥ ׀ עַל־פָּנָיו֮ וַיִּקְרָא֒ יְיָ֣ ׀ יְיָ֔ אֵ֥ל רַח֖וּם וְחַנּ֑וּן אֶ֥רֶךְ אַפַּ֖יִם וְרַב־חֶ֥סֶד וֶאֱמֶֽת׃ נֹצֵ֥ר חֶ֙סֶד֙ לָאֲלָפִ֔ים נֹשֵׂ֥א עָוֺ֛ן וָפֶ֖שַׁע וְחַטָּאָ֑ה וְנַקֵּה֙ לֹ֣א יְנַקֶּ֔ה פֹּקֵ֣ד ׀ עֲוֺ֣ן אָב֗וֹת עַל־בָּנִים֙ וְעַל־בְּנֵ֣י בָנִ֔ים עַל־שִׁלֵּשִׁ֖ים וְעַל־רִבֵּעִֽים׃ וַיְמַהֵ֖ר מֹשֶׁ֑ה וַיִּקֹּ֥ד אַ֖רְצָה וַיִּשְׁתָּֽחוּ׃ וַיֹּ֡אמֶר אִם־נָא֩ מָצָ֨אתִי חֵ֤ן בְּעֵינֶ֙יךָ֙ אֲדֹנָ֔י יֵֽלֶךְ־נָ֥א אֲדֹנָ֖י בְּקִרְבֵּ֑נוּ כִּ֤י עַם־קְשֵׁה־עֹ֙רֶף֙ ה֔וּא וְסָלַחְתָּ֛ לַעֲוֺנֵ֥נוּ וּלְחַטָּאתֵ֖נוּ וּנְחַלְתָּֽנוּ׃ וַיֹּ֗אמֶר הִנֵּ֣ה אָנֹכִי֮ כֹּרֵ֣ת בְּרִית֒ נֶ֤גֶד כׇּֽל־עַמְּךָ֙ אֶעֱשֶׂ֣ה נִפְלָאֹ֔ת אֲשֶׁ֛ר לֹֽא־נִבְרְא֥וּ בְכׇל־הָאָ֖רֶץ וּבְכׇל־הַגּוֹיִ֑ם וְרָאָ֣ה כׇל־הָ֠עָ֠ם אֲשֶׁר־אַתָּ֨ה בְקִרְבּ֜וֹ אֶת־מַעֲשֵׂ֤ה יְיָ֙ כִּֽי־נוֹרָ֣א ה֔וּא אֲשֶׁ֥ר אֲנִ֖י עֹשֶׂ֥ה עִמָּֽךְ׃




\section*{הפטרה לתענית ציבור}


	\tsource{ישעיה נה}\\
דִּרְשׁ֥וּ יְיָ֖ בְּהִמָּֽצְא֑וֹ קְרָאֻ֖הוּ בִּֽהְיוֹת֥וֹ קָרֽוֹב׃ יַעֲזֹ֤ב רָשָׁע֙ דַּרְכּ֔וֹ וְאִ֥ישׁ אָ֖וֶן מַחְשְׁבֹתָ֑יו וְיָשֹׁ֤ב אֶל־יְיָ֙ וִֽירַחֲמֵ֔הוּ וְאֶל־אֱלֹהֵ֖ינוּ כִּֽי־יַרְבֶּ֥ה לִסְלֽוֹחַ׃ כִּ֣י לֹ֤א מַחְשְׁבוֹתַי֙ מַחְשְׁב֣וֹתֵיכֶ֔ם וְלֹ֥א דַרְכֵיכֶ֖ם דְּרָכָ֑י נְאֻ֖ם יְיָ׃ כִּֽי־גָבְה֥וּ שָׁמַ֖יִם מֵאָ֑רֶץ כֵּ֣ן גָּבְה֤וּ דְרָכַי֙ מִדַּרְכֵיכֶ֔ם וּמַחְשְׁבֹתַ֖י מִמַּחְשְׁבֹֽתֵיכֶֽם׃ כִּ֡י כַּאֲשֶׁ֣ר יֵרֵד֩ הַגֶּ֨שֶׁם וְהַשֶּׁ֜לֶג מִן־הַשָּׁמַ֗יִם וְשָׁ֙מָּה֙ לֹ֣א יָשׁ֔וּב כִּ֚י אִם־הִרְוָ֣ה אֶת־הָאָ֔רֶץ וְהוֹלִידָ֖הּ וְהִצְמִיחָ֑הּ וְנָ֤תַן זֶ֙רַע֙ לַזֹּרֵ֔עַ וְלֶ֖חֶם לָאֹכֵֽל׃ כֵּ֣ן יִהְיֶ֤ה דְבָרִי֙ אֲשֶׁ֣ר יֵצֵ֣א מִפִּ֔י לֹא־יָשׁ֥וּב אֵלַ֖י רֵיקָ֑ם כִּ֤י אִם־עָשָׂה֙ אֶת־אֲשֶׁ֣ר חָפַ֔צְתִּי וְהִצְלִ֖יחַ אֲשֶׁ֥ר שְׁלַחְתִּֽיו׃ כִּֽי־בְשִׂמְחָ֣ה תֵצֵ֔אוּ וּבְשָׁל֖וֹם תּוּבָל֑וּן הֶהָרִ֣ים וְהַגְּבָע֗וֹת יִפְצְח֤וּ לִפְנֵיכֶם֙ רִנָּ֔ה וְכׇל־עֲצֵ֥י הַשָּׂדֶ֖ה יִמְחֲאוּ־כָֽף׃ תַּ֤חַת הַֽנַּעֲצוּץ֙ יַעֲלֶ֣ה בְר֔וֹשׁ (תחת) [וְתַ֥חַת] הַסִּרְפַּ֖ד יַעֲלֶ֣ה הֲדַ֑ס וְהָיָ֤ה לַֽייָ֙ לְשֵׁ֔ם לְא֥וֹת עוֹלָ֖ם לֹ֥א יִכָּרֵֽת׃ {פ} כֹּ֚ה אָמַ֣ר יְיָ֔ שִׁמְר֥וּ מִשְׁפָּ֖ט וַעֲשׂ֣וּ צְדָקָ֑ה כִּֽי־קְרוֹבָ֤ה יְשׁוּעָתִי֙ לָב֔וֹא וְצִדְקָתִ֖י לְהִגָּלֽוֹת׃ אַשְׁרֵ֤י אֱנוֹשׁ֙ יַֽעֲשֶׂה־זֹּ֔את וּבֶן־אָדָ֖ם יַחֲזִ֣יק בָּ֑הּ שֹׁמֵ֤ר שַׁבָּת֙ מֵֽחַלְּל֔וֹ וְשֹׁמֵ֥ר יָד֖וֹ מֵעֲשׂ֥וֹת כׇּל־רָֽע׃ {ס} וְאַל־יֹאמַ֣ר בֶּן־הַנֵּכָ֗ר הַנִּלְוָ֤ה אֶל־יְיָ֙ לֵאמֹ֔ר הַבְדֵּ֧ל יַבְדִּילַ֛נִי יְיָ֖ מֵעַ֣ל עַמּ֑וֹ וְאַל־יֹאמַר֙ הַסָּרִ֔יס הֵ֥ן אֲנִ֖י עֵ֥ץ יָבֵֽשׁ׃ {פ} כִּי־כֹ֣ה ׀ אָמַ֣ר יְיָ֗ לַסָּֽרִיסִים֙ אֲשֶׁ֤ר יִשְׁמְרוּ֙ אֶת־שַׁבְּתוֹתַ֔י וּבָחֲר֖וּ בַּאֲשֶׁ֣ר חָפָ֑צְתִּי וּמַחֲזִיקִ֖ים בִּבְרִיתִֽי׃ וְנָתַתִּ֨י לָהֶ֜ם בְּבֵיתִ֤י וּבְחֽוֹמֹתַי֙ יָ֣ד וָשֵׁ֔ם ט֖וֹב מִבָּנִ֣ים וּמִבָּנ֑וֹת שֵׁ֤ם עוֹלָם֙ אֶתֶּן־ל֔וֹ אֲשֶׁ֖ר לֹ֥א יִכָּרֵֽת׃ {ס} וּבְנֵ֣י הַנֵּכָ֗ר הַנִּלְוִ֤ים עַל־יְיָ֙ לְשָׁ֣רְת֔וֹ וּֽלְאַהֲבָה֙ אֶת־שֵׁ֣ם יְיָ֔ לִֽהְי֥וֹת ל֖וֹ לַעֲבָדִ֑ים כׇּל־שֹׁמֵ֤ר שַׁבָּת֙ מֵֽחַלְּל֔וֹ וּמַחֲזִיקִ֖ים בִּבְרִיתִֽי׃ וַהֲבִיאוֹתִ֞ים אֶל־הַ֣ר קׇדְשִׁ֗י וְשִׂמַּחְתִּים֙ בְּבֵ֣ית תְּפִלָּתִ֔י עוֹלֹתֵיהֶ֧ם וְזִבְחֵיהֶ֛ם לְרָצ֖וֹן עַֽל־מִזְבְּחִ֑י כִּ֣י בֵיתִ֔י בֵּית־תְּפִלָּ֥ה יִקָּרֵ֖א לְכׇל־הָעַמִּֽים׃ נְאֻם֙ אֲדֹנָ֣י יְיָ֔ מְקַבֵּ֖ץ נִדְחֵ֣י יִשְׂרָאֵ֑ל ע֛וֹד אֲקַבֵּ֥ץ עָלָ֖יו לְנִקְבָּצָֽיו׃




\section[ראש חדש]{קריאה לראש חדש}


	\tsource{במדבר כח}\\
וַיְדַבֵּ֥ר יְיָ֖ אֶל־מֹשֶׁ֥ה לֵּאמֹֽר׃ צַ֚ו אֶת־בְּנֵ֣י יִשְׂרָאֵ֔ל וְאָמַרְתָּ֖ אֲלֵהֶ֑ם אֶת־קׇרְבָּנִ֨י לַחְמִ֜י לְאִשַּׁ֗י רֵ֚יחַ נִֽיחֹחִ֔י תִּשְׁמְר֕וּ לְהַקְרִ֥יב לִ֖י בְּמוֹעֲדֽוֹ׃
	\aliyah{לוי}
וְאָמַרְתָּ֣ לָהֶ֔ם זֶ֚ה הָֽאִשֶּׁ֔ה אֲשֶׁ֥ר תַּקְרִ֖יבוּ לַייָ֑ כְּבָשִׂ֨ים בְּנֵֽי־שָׁנָ֧ה תְמִימִ֛ם שְׁנַ֥יִם לַיּ֖וֹם עֹלָ֥ה תָמִֽיד׃
	(\instruction{ע״כ כהן})
אֶת־הַכֶּ֥בֶשׂ אֶחָ֖ד תַּעֲשֶׂ֣ה בַבֹּ֑קֶר וְאֵת֙ הַכֶּ֣בֶשׂ הַשֵּׁנִ֔י תַּעֲשֶׂ֖ה בֵּ֥ין הָֽעַרְבָּֽיִם׃ וַעֲשִׂירִ֧ית הָאֵיפָ֛ה סֹ֖לֶת לְמִנְחָ֑ה בְּלוּלָ֛ה בְּשֶׁ֥מֶן כָּתִ֖ית רְבִיעִ֥ת הַהִֽין׃
	\aliyah{ישראל}
עֹלַ֖ת תָּמִ֑יד הָעֲשֻׂיָה֙ בְּהַ֣ר סִינַ֔י לְרֵ֣יחַ נִיחֹ֔חַ אִשֶּׁ֖ה לַֽייָ׃ וְנִסְכּוֹ֙ רְבִיעִ֣ת הַהִ֔ין לַכֶּ֖בֶשׂ הָאֶחָ֑ד בַּקֹּ֗דֶשׁ הַסֵּ֛ךְ נֶ֥סֶךְ שֵׁכָ֖ר לַייָ׃ וְאֵת֙ הַכֶּ֣בֶשׂ הַשֵּׁנִ֔י תַּעֲשֶׂ֖ה בֵּ֣ין הָֽעַרְבָּ֑יִם כְּמִנְחַ֨ת הַבֹּ֤קֶר וּכְנִסְכּוֹ֙ תַּעֲשֶׂ֔ה אִשֵּׁ֛ה רֵ֥יחַ נִיחֹ֖חַ לַייָ׃ {פ} וּבְיוֹם֙ הַשַּׁבָּ֔ת שְׁנֵֽי־כְבָשִׂ֥ים בְּנֵֽי־שָׁנָ֖ה תְּמִימִ֑ם וּשְׁנֵ֣י עֶשְׂרֹנִ֗ים סֹ֧לֶת מִנְחָ֛ה בְּלוּלָ֥ה בַשֶּׁ֖מֶן וְנִסְכּֽוֹ׃ עֹלַ֥ת שַׁבַּ֖ת בְּשַׁבַּתּ֑וֹ עַל־עֹלַ֥ת הַתָּמִ֖יד וְנִסְכָּֽהּ׃ 
	\aliyah{רביעי}
וּבְרָאשֵׁי֙ חׇדְשֵׁיכֶ֔ם תַּקְרִ֥יבוּ עֹלָ֖ה לַייָ֑ פָּרִ֨ים בְּנֵֽי־בָקָ֤ר שְׁנַ֙יִם֙ וְאַ֣יִל אֶחָ֔ד כְּבָשִׂ֧ים בְּנֵי־שָׁנָ֛ה שִׁבְעָ֖ה תְּמִימִֽם׃ וּשְׁלֹשָׁ֣ה עֶשְׂרֹנִ֗ים סֹ֤לֶת מִנְחָה֙ בְּלוּלָ֣ה בַשֶּׁ֔מֶן לַפָּ֖ר הָאֶחָ֑ד וּשְׁנֵ֣י עֶשְׂרֹנִ֗ים סֹ֤לֶת מִנְחָה֙ בְּלוּלָ֣ה בַשֶּׁ֔מֶן לָאַ֖יִל הָֽאֶחָֽד׃ וְעִשָּׂרֹ֣ן עִשָּׂר֗וֹן סֹ֤לֶת מִנְחָה֙ בְּלוּלָ֣ה בַשֶּׁ֔מֶן לַכֶּ֖בֶשׂ הָאֶחָ֑ד עֹלָה֙ רֵ֣יחַ נִיחֹ֔חַ אִשֶּׁ֖ה לַייָ׃ וְנִסְכֵּיהֶ֗ם חֲצִ֣י הַהִין֩ יִהְיֶ֨ה לַפָּ֜ר וּשְׁלִישִׁ֧ת הַהִ֣ין לָאַ֗יִל וּרְבִיעִ֥ת הַהִ֛ין לַכֶּ֖בֶשׂ יָ֑יִן זֹ֣את עֹלַ֥ת חֹ֙דֶשׁ֙ בְּחׇדְשׁ֔וֹ לְחׇדְשֵׁ֖י הַשָּׁנָֽה׃ וּשְׂעִ֨יר עִזִּ֥ים אֶחָ֛ד לְחַטָּ֖את לַייָ֑ עַל־עֹלַ֧ת הַתָּמִ֛יד יֵעָשֶׂ֖ה וְנִסְכּֽוֹ׃ 



\ifboolexpr{togl {includeChM}}{
\section[חה“מ פסח]{קריאה לחה“מ פסח}
	

	\ssubsection{
		יום א' דחול המועד
	}\\ \tsource{שמות יג}\\
וַיְדַבֵּ֥ר יְיָ֖ אֶל־מֹשֶׁ֥ה לֵּאמֹֽר׃ קַדֶּשׁ־לִ֨י כׇל־בְּכ֜וֹר פֶּ֤טֶר כׇּל־רֶ֙חֶם֙ בִּבְנֵ֣י יִשְׂרָאֵ֔ל בָּאָדָ֖ם וּבַבְּהֵמָ֑ה לִ֖י הֽוּא׃ וַיֹּ֨אמֶר מֹשֶׁ֜ה אֶל־הָעָ֗ם זָכ֞וֹר אֶת־הַיּ֤וֹם הַזֶּה֙ אֲשֶׁ֨ר יְצָאתֶ֤ם מִמִּצְרַ֙יִם֙ מִבֵּ֣ית עֲבָדִ֔ים כִּ֚י בְּחֹ֣זֶק יָ֔ד הוֹצִ֧יא יְיָ֛ אֶתְכֶ֖ם מִזֶּ֑ה וְלֹ֥א יֵאָכֵ֖ל חָמֵֽץ׃ הַיּ֖וֹם אַתֶּ֣ם יֹצְאִ֑ים בְּחֹ֖דֶשׁ הָאָבִֽיב׃
	\aliyah{לוי}
וְהָיָ֣ה כִֽי־יְבִיאֲךָ֣ יְיָ֡ אֶל־אֶ֣רֶץ הַֽ֠כְּנַעֲנִ֠י וְהַחִתִּ֨י וְהָאֱמֹרִ֜י וְהַחִוִּ֣י וְהַיְבוּסִ֗י אֲשֶׁ֨ר נִשְׁבַּ֤ע לַאֲבֹתֶ֙יךָ֙ לָ֣תֶת לָ֔ךְ אֶ֛רֶץ זָבַ֥ת חָלָ֖ב וּדְבָ֑שׁ וְעָבַדְתָּ֛ אֶת־הָעֲבֹדָ֥ה הַזֹּ֖את בַּחֹ֥דֶשׁ הַזֶּֽה׃ שִׁבְעַ֥ת יָמִ֖ים תֹּאכַ֣ל מַצֹּ֑ת וּבַיּוֹם֙ הַשְּׁבִיעִ֔י חַ֖ג לַייָ׃ מַצּוֹת֙ יֵֽאָכֵ֔ל אֵ֖ת שִׁבְעַ֣ת הַיָּמִ֑ים וְלֹֽא־יֵרָאֶ֨ה לְךָ֜ חָמֵ֗ץ וְלֹֽא־יֵרָאֶ֥ה לְךָ֛ שְׂאֹ֖ר בְּכׇל־גְּבֻלֶֽךָ׃ וְהִגַּדְתָּ֣ לְבִנְךָ֔ בַּיּ֥וֹם הַה֖וּא לֵאמֹ֑ר בַּעֲב֣וּר זֶ֗ה עָשָׂ֤ה יְיָ֙ לִ֔י בְּצֵאתִ֖י מִמִּצְרָֽיִם׃ וְהָיָה֩ לְךָ֨ לְא֜וֹת עַל־יָדְךָ֗ וּלְזִכָּרוֹן֙ בֵּ֣ין עֵינֶ֔יךָ לְמַ֗עַן תִּהְיֶ֛ה תּוֹרַ֥ת יְיָ֖ בְּפִ֑יךָ כִּ֚י בְּיָ֣ד חֲזָקָ֔ה הוֹצִֽאֲךָ֥ יְיָ֖ מִמִּצְרָֽיִם׃ וְשָׁמַרְתָּ֛ אֶת־הַחֻקָּ֥ה הַזֹּ֖את לְמוֹעֲדָ֑הּ מִיָּמִ֖ים יָמִֽימָה׃ 
	\aliyah{ישראל}
וְהָיָ֞ה כִּֽי־יְבִאֲךָ֤ יְיָ֙ אֶל־אֶ֣רֶץ הַֽכְּנַעֲנִ֔י כַּאֲשֶׁ֛ר נִשְׁבַּ֥ע לְךָ֖ וְלַֽאֲבֹתֶ֑יךָ וּנְתָנָ֖הּ לָֽךְ׃ וְהַעֲבַרְתָּ֥ כׇל־פֶּֽטֶר־רֶ֖חֶם לַֽייָ֑ וְכׇל־פֶּ֣טֶר ׀ שֶׁ֣גֶר בְּהֵמָ֗ה אֲשֶׁ֨ר יִהְיֶ֥ה לְךָ֛ הַזְּכָרִ֖ים לַייָ׃ וְכׇל־פֶּ֤טֶר חֲמֹר֙ תִּפְדֶּ֣ה בְשֶׂ֔ה וְאִם־לֹ֥א תִפְדֶּ֖ה וַעֲרַפְתּ֑וֹ וְכֹ֨ל בְּכ֥וֹר אָדָ֛ם בְּבָנֶ֖יךָ תִּפְדֶּֽה׃ וְהָיָ֞ה כִּֽי־יִשְׁאָלְךָ֥ בִנְךָ֛ מָחָ֖ר לֵאמֹ֣ר מַה־זֹּ֑את וְאָמַרְתָּ֣ אֵלָ֔יו בְּחֹ֣זֶק יָ֗ד הוֹצִיאָ֧נוּ יְיָ֛ מִמִּצְרַ֖יִם מִבֵּ֥ית עֲבָדִֽים׃ וַיְהִ֗י כִּֽי־הִקְשָׁ֣ה פַרְעֹה֮ לְשַׁלְּחֵ֒נוּ֒ וַיַּהֲרֹ֨ג יְיָ֤ כׇּל־בְּכוֹר֙ בְּאֶ֣רֶץ מִצְרַ֔יִם מִבְּכֹ֥ר אָדָ֖ם וְעַד־בְּכ֣וֹר בְּהֵמָ֑ה עַל־כֵּן֩ אֲנִ֨י זֹבֵ֜חַ לַֽייָ֗ כׇּל־פֶּ֤טֶר רֶ֙חֶם֙ הַזְּכָרִ֔ים וְכׇל־בְּכ֥וֹר בָּנַ֖י אֶפְדֶּֽה׃ וְהָיָ֤ה לְאוֹת֙ עַל־יָ֣דְכָ֔ה וּלְטוֹטָפֹ֖ת בֵּ֣ין עֵינֶ֑יךָ כִּ֚י בְּחֹ֣זֶק יָ֔ד הוֹצִיאָ֥נוּ יְיָ֖ מִמִּצְרָֽיִם׃ 
	

	

	\instruction{
		בכל יום  של חוה״מ קוראים רביעי בספר שני.\\
	}
	

	\ssubsection{יום ב' דחול המועד}\\
	\tsource{שמות כב}\\
אִם־כֶּ֣סֶף ׀ תַּלְוֶ֣ה אֶת־עַמִּ֗י אֶת־הֶֽעָנִי֙ עִמָּ֔ךְ לֹא־תִהְיֶ֥ה ל֖וֹ כְּנֹשֶׁ֑ה לֹֽא־תְשִׂימ֥וּן עָלָ֖יו נֶֽשֶׁךְ׃ אִם־חָבֹ֥ל תַּחְבֹּ֖ל שַׂלְמַ֣ת רֵעֶ֑ךָ עַד־בֹּ֥א הַשֶּׁ֖מֶשׁ תְּשִׁיבֶ֥נּוּ לֽוֹ׃ כִּ֣י הִ֤וא כְסוּתֹה֙ לְבַדָּ֔הּ הִ֥וא שִׂמְלָת֖וֹ לְעֹר֑וֹ בַּמֶּ֣ה יִשְׁכָּ֔ב וְהָיָה֙ כִּֽי־יִצְעַ֣ק אֵלַ֔י וְשָׁמַעְתִּ֖י כִּֽי־חַנּ֥וּן אָֽנִי׃ 
	\aliyah{לוי}
אֱלֹהִ֖ים לֹ֣א תְקַלֵּ֑ל וְנָשִׂ֥יא בְעַמְּךָ֖ לֹ֥א תָאֹֽר׃ מְלֵאָתְךָ֥ וְדִמְעֲךָ֖ לֹ֣א תְאַחֵ֑ר בְּכ֥וֹר בָּנֶ֖יךָ תִּתֶּן־לִֽי׃ כֵּֽן־תַּעֲשֶׂ֥ה לְשֹׁרְךָ֖ לְצֹאנֶ֑ךָ שִׁבְעַ֤ת יָמִים֙ יִהְיֶ֣ה עִם־אִמּ֔וֹ בַּיּ֥וֹם הַשְּׁמִינִ֖י תִּתְּנוֹ־לִֽי׃ וְאַנְשֵׁי־קֹ֖דֶשׁ תִּהְי֣וּן לִ֑י וּבָשָׂ֨ר בַּשָּׂדֶ֤ה טְרֵפָה֙ לֹ֣א תֹאכֵ֔לוּ לַכֶּ֖לֶב תַּשְׁלִכ֥וּן אֹתֽוֹ׃ {ס} לֹ֥א תִשָּׂ֖א שֵׁ֣מַע שָׁ֑וְא אַל־תָּ֤שֶׁת יָֽדְךָ֙ עִם־רָשָׁ֔ע לִהְיֹ֖ת עֵ֥ד חָמָֽס׃ לֹֽא־תִהְיֶ֥ה אַחֲרֵֽי־רַבִּ֖ים לְרָעֹ֑ת וְלֹא־תַעֲנֶ֣ה עַל־רִ֗ב לִנְטֹ֛ת אַחֲרֵ֥י רַבִּ֖ים לְהַטֹּֽת׃ וְדָ֕ל לֹ֥א תֶהְדַּ֖ר בְּרִיבֽוֹ׃ {ס} כִּ֣י תִפְגַּ֞ע שׁ֧וֹר אֹֽיִבְךָ֛ א֥וֹ חֲמֹר֖וֹ תֹּעֶ֑ה הָשֵׁ֥ב תְּשִׁיבֶ֖נּוּ לֽוֹ׃ {ס} כִּֽי־תִרְאֶ֞ה חֲמ֣וֹר שֹׂנַאֲךָ֗ רֹבֵץ֙ תַּ֣חַת מַשָּׂא֔וֹ וְחָדַלְתָּ֖ מֵעֲזֹ֣ב ל֑וֹ עָזֹ֥ב תַּעֲזֹ֖ב עִמּֽוֹ׃ 
	\aliyah{ישראל}
לֹ֥א תַטֶּ֛ה מִשְׁפַּ֥ט אֶבְיֹנְךָ֖ בְּרִיבֽוֹ׃ מִדְּבַר־שֶׁ֖קֶר תִּרְחָ֑ק וְנָקִ֤י וְצַדִּיק֙ אַֽל־תַּהֲרֹ֔ג כִּ֥י לֹא־אַצְדִּ֖יק רָשָֽׁע׃ וְשֹׁ֖חַד לֹ֣א תִקָּ֑ח כִּ֤י הַשֹּׁ֙חַד֙ יְעַוֵּ֣ר פִּקְחִ֔ים וִֽיסַלֵּ֖ף דִּבְרֵ֥י צַדִּיקִֽים׃ וְגֵ֖ר לֹ֣א תִלְחָ֑ץ וְאַתֶּ֗ם יְדַעְתֶּם֙ אֶת־נֶ֣פֶשׁ הַגֵּ֔ר כִּֽי־גֵרִ֥ים הֱיִיתֶ֖ם בְּאֶ֥רֶץ מִצְרָֽיִם׃ וְשֵׁ֥שׁ שָׁנִ֖ים תִּזְרַ֣ע אֶת־אַרְצֶ֑ךָ וְאָסַפְתָּ֖ אֶת־תְּבוּאָתָֽהּ׃ וְהַשְּׁבִיעִ֞ת תִּשְׁמְטֶ֣נָּה וּנְטַשְׁתָּ֗הּ וְאָֽכְלוּ֙ אֶבְיֹנֵ֣י עַמֶּ֔ךָ וְיִתְרָ֕ם תֹּאכַ֖ל חַיַּ֣ת הַשָּׂדֶ֑ה כֵּֽן־תַּעֲשֶׂ֥ה לְכַרְמְךָ֖ לְזֵיתֶֽךָ׃ שֵׁ֤שֶׁת יָמִים֙ תַּעֲשֶׂ֣ה מַעֲשֶׂ֔יךָ וּבַיּ֥וֹם הַשְּׁבִיעִ֖י תִּשְׁבֹּ֑ת לְמַ֣עַן יָנ֗וּחַ שֽׁוֹרְךָ֙ וַחֲמֹרֶ֔ךָ וְיִנָּפֵ֥שׁ בֶּן־אֲמָתְךָ֖ וְהַגֵּֽר׃ וּבְכֹ֛ל אֲשֶׁר־אָמַ֥רְתִּי אֲלֵיכֶ֖ם תִּשָּׁמֵ֑רוּ וְשֵׁ֨ם אֱלֹהִ֤ים אֲחֵרִים֙ לֹ֣א תַזְכִּ֔ירוּ לֹ֥א יִשָּׁמַ֖ע עַל־פִּֽיךָ׃ שָׁלֹ֣שׁ רְגָלִ֔ים תָּחֹ֥ג לִ֖י בַּשָּׁנָֽה׃ אֶת־חַ֣ג הַמַּצּוֹת֮ תִּשְׁמֹר֒ שִׁבְעַ֣ת יָמִים֩ תֹּאכַ֨ל מַצּ֜וֹת כַּֽאֲשֶׁ֣ר צִוִּיתִ֗ךָ לְמוֹעֵד֙ חֹ֣דֶשׁ הָֽאָבִ֔יב כִּי־ב֖וֹ יָצָ֣אתָ מִמִּצְרָ֑יִם וְלֹא־יֵרָא֥וּ פָנַ֖י רֵיקָֽם׃ וְחַ֤ג הַקָּצִיר֙ בִּכּוּרֵ֣י מַעֲשֶׂ֔יךָ אֲשֶׁ֥ר תִּזְרַ֖ע בַּשָּׂדֶ֑ה וְחַ֤ג הָֽאָסִף֙ בְּצֵ֣את הַשָּׁנָ֔ה בְּאׇסְפְּךָ֥ אֶֽת־מַעֲשֶׂ֖יךָ מִן־הַשָּׂדֶֽה׃ שָׁלֹ֥שׁ פְּעָמִ֖ים בַּשָּׁנָ֑ה יֵרָאֶה֙ כׇּל־זְכ֣וּרְךָ֔ אֶל־פְּנֵ֖י הָאָדֹ֥ן ׀ יְיָ׃ לֹֽא־תִזְבַּ֥ח עַל־חָמֵ֖ץ דַּם־זִבְחִ֑י וְלֹֽא־יָלִ֥ין חֵֽלֶב־חַגִּ֖י עַד־בֹּֽקֶר׃ רֵאשִׁ֗ית בִּכּוּרֵי֙ אַדְמָ֣תְךָ֔ תָּבִ֕יא בֵּ֖ית יְיָ֣ אֱלֹהֶ֑יךָ לֹֽא־תְבַשֵּׁ֥ל גְּדִ֖י בַּחֲלֵ֥ב אִמּֽוֹ׃ 
	

	\ssubsection{יום ג' דחול המועד}\\
	\tsource{שמות לד}\\
וַיֹּ֤אמֶר יְיָ֙ אֶל־מֹשֶׁ֔ה פְּסׇל־לְךָ֛ שְׁנֵֽי־לֻחֹ֥ת אֲבָנִ֖ים כָּרִאשֹׁנִ֑ים וְכָתַבְתִּי֙ עַל־הַלֻּחֹ֔ת אֶ֨ת־הַדְּבָרִ֔ים אֲשֶׁ֥ר הָי֛וּ עַל־הַלֻּחֹ֥ת הָרִאשֹׁנִ֖ים אֲשֶׁ֥ר שִׁבַּֽרְתָּ׃ וֶהְיֵ֥ה נָכ֖וֹן לַבֹּ֑קֶר וְעָלִ֤יתָ בַבֹּ֙קֶר֙ אֶל־הַ֣ר סִינַ֔י וְנִצַּבְתָּ֥ לִ֛י שָׁ֖ם עַל־רֹ֥אשׁ הָהָֽר׃ וְאִישׁ֙ לֹֽא־יַעֲלֶ֣ה עִמָּ֔ךְ וְגַם־אִ֥ישׁ אַל־יֵרָ֖א בְּכׇל־הָהָ֑ר גַּם־הַצֹּ֤אן וְהַבָּקָר֙ אַל־יִרְע֔וּ אֶל־מ֖וּל הָהָ֥ר הַהֽוּא׃ וַיִּפְסֹ֡ל שְׁנֵֽי־לֻחֹ֨ת אֲבָנִ֜ים כָּרִאשֹׁנִ֗ים וַיַּשְׁכֵּ֨ם מֹשֶׁ֤ה בַבֹּ֙קֶר֙ וַיַּ֙עַל֙ אֶל־הַ֣ר סִינַ֔י כַּאֲשֶׁ֛ר צִוָּ֥ה יְיָ֖ אֹת֑וֹ וַיִּקַּ֣ח בְּיָד֔וֹ שְׁנֵ֖י לֻחֹ֥ת אֲבָנִֽים׃ וַיֵּ֤רֶד יְיָ֙ בֶּֽעָנָ֔ן וַיִּתְיַצֵּ֥ב עִמּ֖וֹ שָׁ֑ם וַיִּקְרָ֥א בְשֵׁ֖ם יְיָ׃ וַיַּעֲבֹ֨ר יְיָ֥ ׀ עַל־פָּנָיו֮ וַיִּקְרָא֒ יְיָ֣ ׀ יְיָ֔ אֵ֥ל רַח֖וּם וְחַנּ֑וּן אֶ֥רֶךְ אַפַּ֖יִם וְרַב־חֶ֥סֶד וֶאֱמֶֽת׃ נֹצֵ֥ר חֶ֙סֶד֙ לָאֲלָפִ֔ים נֹשֵׂ֥א עָוֺ֛ן וָפֶ֖שַׁע וְחַטָּאָ֑ה וְנַקֵּה֙ לֹ֣א יְנַקֶּ֔ה פֹּקֵ֣ד ׀ עֲוֺ֣ן אָב֗וֹת עַל־בָּנִים֙ וְעַל־בְּנֵ֣י בָנִ֔ים עַל־שִׁלֵּשִׁ֖ים וְעַל־רִבֵּעִֽים׃ וַיְמַהֵ֖ר מֹשֶׁ֑ה וַיִּקֹּ֥ד אַ֖רְצָה וַיִּשְׁתָּֽחוּ׃ וַיֹּ֡אמֶר אִם־נָא֩ מָצָ֨אתִי חֵ֤ן בְּעֵינֶ֙יךָ֙ אֲדֹנָ֔י יֵֽלֶךְ־נָ֥א אֲדֹנָ֖י בְּקִרְבֵּ֑נוּ כִּ֤י עַם־קְשֵׁה־עֹ֙רֶף֙ ה֔וּא וְסָלַחְתָּ֛ לַעֲוֺנֵ֥נוּ וּלְחַטָּאתֵ֖נוּ וּנְחַלְתָּֽנוּ׃ וַיֹּ֗אמֶר הִנֵּ֣ה אָנֹכִי֮ כֹּרֵ֣ת בְּרִית֒ נֶ֤גֶד כׇּֽל־עַמְּךָ֙ אֶעֱשֶׂ֣ה נִפְלָאֹ֔ת אֲשֶׁ֛ר לֹֽא־נִבְרְא֥וּ בְכׇל־הָאָ֖רֶץ וּבְכׇל־הַגּוֹיִ֑ם וְרָאָ֣ה כׇל־הָ֠עָ֠ם אֲשֶׁר־אַתָּ֨ה בְקִרְבּ֜וֹ אֶת־מַעֲשֵׂ֤ה יְיָ֙ כִּֽי־נוֹרָ֣א ה֔וּא אֲשֶׁ֥ר אֲנִ֖י עֹשֶׂ֥ה עִמָּֽךְ׃
	\aliyah{לוי}
שְׁמׇ֨ר־לְךָ֔ אֵ֛ת אֲשֶׁ֥ר אָנֹכִ֖י מְצַוְּךָ֣ הַיּ֑וֹם הִנְנִ֧י גֹרֵ֣שׁ מִפָּנֶ֗יךָ אֶת־הָאֱמֹרִי֙ וְהַֽכְּנַעֲנִ֔י וְהַחִתִּי֙ וְהַפְּרִזִּ֔י וְהַחִוִּ֖י וְהַיְבוּסִֽי׃ הִשָּׁ֣מֶר לְךָ֗ פֶּן־תִּכְרֹ֤ת בְּרִית֙ לְיוֹשֵׁ֣ב הָאָ֔רֶץ אֲשֶׁ֥ר אַתָּ֖ה בָּ֣א עָלֶ֑יהָ פֶּן־יִהְיֶ֥ה לְמוֹקֵ֖שׁ בְּקִרְבֶּֽךָ׃ כִּ֤י אֶת־מִזְבְּחֹתָם֙ תִּתֹּצ֔וּן וְאֶת־מַצֵּבֹתָ֖ם תְּשַׁבֵּר֑וּן וְאֶת־אֲשֵׁרָ֖יו תִּכְרֹתֽוּן׃ כִּ֛י לֹ֥א תִֽשְׁתַּחֲוֶ֖ה לְאֵ֣ל אַחֵ֑ר כִּ֤י יְיָ֙ קַנָּ֣א שְׁמ֔וֹ אֵ֥ל קַנָּ֖א הֽוּא׃ פֶּן־תִּכְרֹ֥ת בְּרִ֖ית לְיוֹשֵׁ֣ב הָאָ֑רֶץ וְזָנ֣וּ ׀ אַחֲרֵ֣י אֱלֹֽהֵיהֶ֗ם וְזָבְחוּ֙ לֵאלֹ֣הֵיהֶ֔ם וְקָרָ֣א לְךָ֔ וְאָכַלְתָּ֖ מִזִּבְחֽוֹ׃ וְלָקַחְתָּ֥ מִבְּנֹתָ֖יו לְבָנֶ֑יךָ וְזָנ֣וּ בְנֹתָ֗יו אַחֲרֵי֙ אֱלֹ֣הֵיהֶ֔ן וְהִזְנוּ֙ אֶת־בָּנֶ֔יךָ אַחֲרֵ֖י אֱלֹהֵיהֶֽן׃ אֱלֹהֵ֥י מַסֵּכָ֖ה לֹ֥א תַעֲשֶׂה־לָּֽךְ׃
\aliyah{ישראל}
אֶת־חַ֣ג הַמַּצּוֹת֮ תִּשְׁמֹר֒ שִׁבְעַ֨ת יָמִ֜ים תֹּאכַ֤ל מַצּוֹת֙ אֲשֶׁ֣ר צִוִּיתִ֔ךָ לְמוֹעֵ֖ד חֹ֣דֶשׁ הָאָבִ֑יב כִּ֚י בְּחֹ֣דֶשׁ הָֽאָבִ֔יב יָצָ֖אתָ מִמִּצְרָֽיִם׃ כׇּל־פֶּ֥טֶר רֶ֖חֶם לִ֑י וְכׇֽל־מִקְנְךָ֙ תִּזָּכָ֔ר פֶּ֖טֶר שׁ֥וֹר וָשֶֽׂה׃ וּפֶ֤טֶר חֲמוֹר֙ תִּפְדֶּ֣ה בְשֶׂ֔ה וְאִם־לֹ֥א תִפְדֶּ֖ה וַעֲרַפְתּ֑וֹ כֹּ֣ל בְּכ֤וֹר בָּנֶ֙יךָ֙ תִּפְדֶּ֔ה וְלֹֽא־יֵרָא֥וּ פָנַ֖י רֵיקָֽם׃ שֵׁ֤שֶׁת יָמִים֙ תַּעֲבֹ֔ד וּבַיּ֥וֹם הַשְּׁבִיעִ֖י תִּשְׁבֹּ֑ת בֶּחָרִ֥ישׁ וּבַקָּצִ֖יר תִּשְׁבֹּֽת׃ וְחַ֤ג שָׁבֻעֹת֙ תַּעֲשֶׂ֣ה לְךָ֔ בִּכּוּרֵ֖י קְצִ֣יר חִטִּ֑ים וְחַג֙ הָֽאָסִ֔יף תְּקוּפַ֖ת הַשָּׁנָֽה׃ שָׁלֹ֥שׁ פְּעָמִ֖ים בַּשָּׁנָ֑ה יֵרָאֶה֙ כׇּל־זְכ֣וּרְךָ֔ אֶת־פְּנֵ֛י הָֽאָדֹ֥ן ׀ יְיָ֖ אֱלֹהֵ֥י יִשְׂרָאֵֽל׃ כִּֽי־אוֹרִ֤ישׁ גּוֹיִם֙ מִפָּנֶ֔יךָ וְהִרְחַבְתִּ֖י אֶת־גְּבֻלֶ֑ךָ וְלֹא־יַחְמֹ֥ד אִישׁ֙ אֶֽת־אַרְצְךָ֔ בַּעֲלֹֽתְךָ֗ לֵרָאוֹת֙ אֶת־פְּנֵי֙ יְיָ֣ אֱלֹהֶ֔יךָ שָׁלֹ֥שׁ פְּעָמִ֖ים בַּשָּׁנָֽה׃ לֹֽא־תִשְׁחַ֥ט עַל־חָמֵ֖ץ דַּם־זִבְחִ֑י וְלֹא־יָלִ֣ין לַבֹּ֔קֶר זֶ֖בַח חַ֥ג הַפָּֽסַח׃ רֵאשִׁ֗ית בִּכּוּרֵי֙ אַדְמָ֣תְךָ֔ תָּבִ֕יא בֵּ֖ית יְיָ֣ אֱלֹהֶ֑יךָ לֹא־תְבַשֵּׁ֥ל גְּדִ֖י בַּחֲלֵ֥ב אִמּֽוֹ׃ 


	\ssubsection{יום ד' דחול המועד}\\
	\tsource{במדבר ט}\\
וַיְדַבֵּ֣ר יְיָ֣ אֶל־מֹשֶׁ֣ה בְמִדְבַּר־סִ֠ינַ֠י בַּשָּׁנָ֨ה הַשֵּׁנִ֜ית לְצֵאתָ֨ם מֵאֶ֧רֶץ מִצְרַ֛יִם בַּחֹ֥דֶשׁ הָרִאשׁ֖וֹן לֵאמֹֽר׃ וְיַעֲשׂ֧וּ בְנֵי־יִשְׂרָאֵ֛ל אֶת־הַפָּ֖סַח בְּמוֹעֲדֽוֹ׃ בְּאַרְבָּעָ֣ה עָשָֽׂר־י֠וֹם בַּחֹ֨דֶשׁ הַזֶּ֜ה בֵּ֧ין הָֽעַרְבַּ֛יִם תַּעֲשׂ֥וּ אֹת֖וֹ בְּמֹעֲד֑וֹ כְּכׇל־חֻקֹּתָ֥יו וּכְכׇל־מִשְׁפָּטָ֖יו תַּעֲשׂ֥וּ אֹתֽוֹ׃ וַיְדַבֵּ֥ר מֹשֶׁ֛ה אֶל־בְּנֵ֥י יִשְׂרָאֵ֖ל לַעֲשֹׂ֥ת הַפָּֽסַח׃ וַיַּעֲשׂ֣וּ אֶת־הַפֶּ֡סַח בָּרִאשׁ֡וֹן בְּאַרְבָּעָה֩ עָשָׂ֨ר י֥וֹם לַחֹ֛דֶשׁ בֵּ֥ין הָעַרְבַּ֖יִם בְּמִדְבַּ֣ר סִינָ֑י כְּ֠כֹ֠ל אֲשֶׁ֨ר צִוָּ֤ה יְיָ֙ אֶת־מֹשֶׁ֔ה כֵּ֥ן עָשׂ֖וּ בְּנֵ֥י יִשְׂרָאֵֽל׃
	\aliyah{לוי}
וַיְהִ֣י אֲנָשִׁ֗ים אֲשֶׁ֨ר הָי֤וּ טְמֵאִים֙ לְנֶ֣פֶשׁ אָדָ֔ם וְלֹא־יָכְל֥וּ לַעֲשֹׂת־הַפֶּ֖סַח בַּיּ֣וֹם הַה֑וּא וַֽיִּקְרְב֞וּ לִפְנֵ֥י מֹשֶׁ֛ה וְלִפְנֵ֥י אַהֲרֹ֖ן בַּיּ֥וֹם הַהֽוּא׃ וַ֠יֹּאמְר֠וּ הָאֲנָשִׁ֤ים הָהֵ֙מָּה֙ אֵלָ֔יו אֲנַ֥חְנוּ טְמֵאִ֖ים לְנֶ֣פֶשׁ אָדָ֑ם לָ֣מָּה נִגָּרַ֗ע לְבִלְתִּ֨י הַקְרִ֜יב אֶת־קׇרְבַּ֤ן יְיָ֙ בְּמֹ֣עֲד֔וֹ בְּת֖וֹךְ בְּנֵ֥י יִשְׂרָאֵֽל׃ וַיֹּ֥אמֶר אֲלֵהֶ֖ם מֹשֶׁ֑ה עִמְד֣וּ וְאֶשְׁמְעָ֔ה מַה־יְצַוֶּ֥ה יְיָ֖ לָכֶֽם׃ 
	\aliyah{ישראל}
וַיְדַבֵּ֥ר יְיָ֖ אֶל־מֹשֶׁ֥ה לֵּאמֹֽר׃ דַּבֵּ֛ר אֶל־בְּנֵ֥י יִשְׂרָאֵ֖ל לֵאמֹ֑ר אִ֣ישׁ אִ֣ישׁ כִּי־יִהְיֶֽה־טָמֵ֣א ׀ לָנֶ֡פֶשׁ אוֹ֩ בְדֶ֨רֶךְ רְחֹקָ֜הׄ לָכֶ֗ם א֚וֹ לְדֹרֹ֣תֵיכֶ֔ם וְעָ֥שָׂה פֶ֖סַח לַייָ׃ בַּחֹ֨דֶשׁ הַשֵּׁנִ֜י בְּאַרְבָּעָ֨ה עָשָׂ֥ר י֛וֹם בֵּ֥ין הָעַרְבַּ֖יִם יַעֲשׂ֣וּ אֹת֑וֹ עַל־מַצּ֥וֹת וּמְרֹרִ֖ים יֹאכְלֻֽהוּ׃ לֹֽא־יַשְׁאִ֤ירוּ מִמֶּ֙נּוּ֙ עַד־בֹּ֔קֶר וְעֶ֖צֶם לֹ֣א יִשְׁבְּרוּ־ב֑וֹ כְּכׇל־חֻקַּ֥ת הַפֶּ֖סַח יַעֲשׂ֥וּ אֹתֽוֹ׃ וְהָאִישׁ֩ אֲשֶׁר־ה֨וּא טָה֜וֹר וּבְדֶ֣רֶךְ לֹא־הָיָ֗ה וְחָדַל֙ לַעֲשׂ֣וֹת הַפֶּ֔סַח וְנִכְרְתָ֛ה הַנֶּ֥פֶשׁ הַהִ֖וא מֵֽעַמֶּ֑יהָ כִּ֣י ׀ קׇרְבַּ֣ן יְיָ֗ לֹ֤א הִקְרִיב֙ בְּמֹ֣עֲד֔וֹ חֶטְא֥וֹ יִשָּׂ֖א הָאִ֥ישׁ הַהֽוּא׃ וְכִֽי־יָג֨וּר אִתְּכֶ֜ם גֵּ֗ר וְעָ֤שָֽׂה פֶ֙סַח֙ לַֽייָ֔ כְּחֻקַּ֥ת הַפֶּ֛סַח וּכְמִשְׁפָּט֖וֹ כֵּ֣ן יַעֲשֶׂ֑ה חֻקָּ֤ה אַחַת֙ יִהְיֶ֣ה לָכֶ֔ם וְלַגֵּ֖ר וּלְאֶזְרַ֥ח הָאָֽרֶץ׃ 
		
			\ssubsection{רביעי דחול המועד}\\
		\tsource{במדבר כח}\\
וְהִקְרַבְתֶּ֨ם אִשֶּׁ֤ה עֹלָה֙ לַֽייָ֔ פָּרִ֧ים בְּנֵי־בָקָ֛ר שְׁנַ֖יִם וְאַ֣יִל אֶחָ֑ד וְשִׁבְעָ֤ה כְבָשִׂים֙ בְּנֵ֣י שָׁנָ֔ה תְּמִימִ֖ם יִהְי֥וּ לָכֶֽם׃ וּמִ֨נְחָתָ֔ם סֹ֖לֶת בְּלוּלָ֣ה בַשָּׁ֑מֶן שְׁלֹשָׁ֨ה עֶשְׂרֹנִ֜ים לַפָּ֗ר וּשְׁנֵ֧י עֶשְׂרֹנִ֛ים לָאַ֖יִל תַּעֲשֽׂוּ׃ עִשָּׂר֤וֹן עִשָּׂרוֹן֙ תַּעֲשֶׂ֔ה לַכֶּ֖בֶשׂ הָאֶחָ֑ד לְשִׁבְעַ֖ת הַכְּבָשִֽׂים׃ וּשְׂעִ֥יר חַטָּ֖את אֶחָ֑ד לְכַפֵּ֖ר עֲלֵיכֶֽם׃ מִלְּבַד֙ עֹלַ֣ת הַבֹּ֔קֶר אֲשֶׁ֖ר לְעֹלַ֣ת הַתָּמִ֑יד תַּעֲשׂ֖וּ אֶת־אֵֽלֶּה׃ כָּאֵ֜לֶּה תַּעֲשׂ֤וּ לַיּוֹם֙ שִׁבְעַ֣ת יָמִ֔ים לֶ֛חֶם אִשֵּׁ֥ה רֵֽיחַ־נִיחֹ֖חַ לַייָ֑ עַל־עוֹלַ֧ת הַתָּמִ֛יד יֵעָשֶׂ֖ה וְנִסְכּֽוֹ׃ וּבַיּוֹם֙ הַשְּׁבִיעִ֔י מִקְרָא־קֹ֖דֶשׁ יִהְיֶ֣ה לָכֶ֑ם כׇּל־מְלֶ֥אכֶת עֲבֹדָ֖ה לֹ֥א תַעֲשֽׂוּ׃ 
	
\section[חוה“מ סכות]{קריאה לחוה“מ סכות}

\instruction{בכל יום של חוה״מ סכות הכהן קורא אותו היום בפרשת פנחס, ולוי מחרתו, שלישי מחרתיים, ורביעי חוזר וקורא אותו היום ומחרתו. בהו״ר כהן קורא  יום החמישי, לוי קורא יום הששי, שלישי קורא יום השביעי, ורביעי קורא יום הששי והשביעי.}
	

וּבַיּ֣וֹם הַשֵּׁנִ֗י פָּרִ֧ים בְּנֵי־בָקָ֛ר שְׁנֵ֥ים עָשָׂ֖ר אֵילִ֣ם שְׁנָ֑יִם כְּבָשִׂ֧ים בְּנֵי־שָׁנָ֛ה אַרְבָּעָ֥ה עָשָׂ֖ר תְּמִימִֽם׃ וּמִנְחָתָ֣ם וְנִסְכֵּיהֶ֡ם לַ֠פָּרִ֠ים לָאֵילִ֧ם וְלַכְּבָשִׂ֛ים בְּמִסְפָּרָ֖ם כַּמִּשְׁפָּֽט׃ וּשְׂעִיר־עִזִּ֥ים אֶחָ֖ד חַטָּ֑את מִלְּבַד֙ עֹלַ֣ת הַתָּמִ֔יד וּמִנְחָתָ֖הּ וְנִסְכֵּיהֶֽם׃ {ס} וּבַיּ֧וֹם הַשְּׁלִישִׁ֛י פָּרִ֥ים עַשְׁתֵּי־עָשָׂ֖ר אֵילִ֣ם שְׁנָ֑יִם כְּבָשִׂ֧ים בְּנֵי־שָׁנָ֛ה אַרְבָּעָ֥ה עָשָׂ֖ר תְּמִימִֽם׃ וּמִנְחָתָ֣ם וְנִסְכֵּיהֶ֡ם לַ֠פָּרִ֠ים לָאֵילִ֧ם וְלַכְּבָשִׂ֛ים בְּמִסְפָּרָ֖ם כַּמִּשְׁפָּֽט׃ וּשְׂעִ֥יר חַטָּ֖את אֶחָ֑ד מִלְּבַד֙ עֹלַ֣ת הַתָּמִ֔יד וּמִנְחָתָ֖הּ וְנִסְכָּֽהּ׃ {ס} וּבַיּ֧וֹם הָרְבִיעִ֛י פָּרִ֥ים עֲשָׂרָ֖ה אֵילִ֣ם שְׁנָ֑יִם כְּבָשִׂ֧ים בְּנֵֽי־שָׁנָ֛ה אַרְבָּעָ֥ה עָשָׂ֖ר תְּמִימִֽם׃ מִנְחָתָ֣ם וְנִסְכֵּיהֶ֡ם לַ֠פָּרִ֠ים לָאֵילִ֧ם וְלַכְּבָשִׂ֛ים בְּמִסְפָּרָ֖ם כַּמִּשְׁפָּֽט׃ וּשְׂעִיר־עִזִּ֥ים אֶחָ֖ד חַטָּ֑את מִלְּבַד֙ עֹלַ֣ת הַתָּמִ֔יד מִנְחָתָ֖הּ וְנִסְכָּֽהּ׃ {ס} וּבַיּ֧וֹם הַחֲמִישִׁ֛י פָּרִ֥ים תִּשְׁעָ֖ה אֵילִ֣ם שְׁנָ֑יִם כְּבָשִׂ֧ים בְּנֵֽי־שָׁנָ֛ה אַרְבָּעָ֥ה עָשָׂ֖ר תְּמִימִֽם׃ וּמִנְחָתָ֣ם וְנִסְכֵּיהֶ֡ם לַ֠פָּרִ֠ים לָאֵילִ֧ם וְלַכְּבָשִׂ֛ים בְּמִסְפָּרָ֖ם כַּמִּשְׁפָּֽט׃ וּשְׂעִ֥יר חַטָּ֖את אֶחָ֑ד מִלְּבַד֙ עֹלַ֣ת הַתָּמִ֔יד וּמִנְחָתָ֖הּ וְנִסְכָּֽהּ׃ {ס} וּבַיּ֧וֹם הַשִּׁשִּׁ֛י פָּרִ֥ים שְׁמֹנָ֖ה אֵילִ֣ם שְׁנָ֑יִם כְּבָשִׂ֧ים בְּנֵי־שָׁנָ֛ה אַרְבָּעָ֥ה עָשָׂ֖ר תְּמִימִֽם׃ וּמִנְחָתָ֣ם וְנִסְכֵּיהֶ֡ם לַ֠פָּרִ֠ים לָאֵילִ֧ם וְלַכְּבָשִׂ֛ים בְּמִסְפָּרָ֖ם כַּמִּשְׁפָּֽט׃ וּשְׂעִ֥יר חַטָּ֖את אֶחָ֑ד מִלְּבַד֙ עֹלַ֣ת הַתָּמִ֔יד מִנְחָתָ֖הּ וּנְסָכֶֽיהָ׃ {ס} וּבַיּ֧וֹם הַשְּׁבִיעִ֛י פָּרִ֥ים שִׁבְעָ֖ה אֵילִ֣ם שְׁנָ֑יִם כְּבָשִׂ֧ים בְּנֵי־שָׁנָ֛ה אַרְבָּעָ֥ה עָשָׂ֖ר תְּמִימִֽם׃ וּמִנְחָתָ֣ם וְנִסְכֵּהֶ֡ם לַ֠פָּרִ֠ים לָאֵילִ֧ם וְלַכְּבָשִׂ֛ים בְּמִסְפָּרָ֖ם כְּמִשְׁפָּטָֽם׃ וּשְׂעִ֥יר חַטָּ֖את אֶחָ֑ד מִלְּבַד֙ עֹלַ֣ת הַתָּמִ֔יד מִנְחָתָ֖הּ וְנִסְכָּֽהּ׃ }{}


\end{footnotesize}}

\end{document}