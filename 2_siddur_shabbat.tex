%% This work is licensed under a Creative Commons Attribution-ShareAlike 4.0 International License. %%


\documentclass[twoside, openany, parskip=half, 11pt]{book}

\usepackage[paperheight=8.5in,paperwidth=5.5in,top=.7in,bottom=.5in, inner=.875in, outer=.75in, marginparsep=.1in, headsep=16pt]{geometry}
\setlength{\marginparwidth}{.5in}

\usepackage{siddur}

\usepackage[backend=bibtex,style=verbose-trad2]{biblatex}
\bibliography{referenced_texts}

%\usepackage{newclude}

\begin{document}

\title{ \adforn{54} סדור \adforn{26}\\
לשון ישרים\\
\vspace{.5in}
לשבת
%\includegraphics[scale=.5]{wolf_shofar_bitmap.png}
}

\author{מסודר ע״י
\\
\textbf{ﭏיעזר בן זאב וואלף קזימיר}}
\date{נוסח אשכנז}

\maketitle

\begin{minipage}{\textwidth}
\begin{english}
\raggedright

©Nathan Kasimer, 2021 (5782). Shared under a Creative Commons Attribution 4.0 license.\\
\textcolor{blue}{https׃//creativecommons.org/licenses/by/4.0/}\\ \vspace{\baselineskip}


Shlomo font created by Shlomo Orbach and Ralph Hancock.\\ \textcolor{blue}{https׃//sites.google.com/site/orlaeinayim/download}\\ \vspace{\baselineskip}

Full \XeLaTeX \quad source and PDF can be found at:\\ \textcolor{blue}{link tbd}\\ \vspace{\baselineskip}


\end{english}
\end{minipage}

\begin{minipage}{\textwidth}

\begin{english}
\begin{center} %regular chapter (which I don't want in the TOC anyways) doesn't center.
\begin{LARGE}
Introduction
\end{LARGE}
\end{center}

This siddur is an attempt to make a siddur that reflects the liturgical practice I've adopted from various communities I've lived in--a mish-mash American pan-Eastern Ashkenaz tradition. This siddur is a work in progress. I hope to add a bit more instructions in the future, including instructions in English.  This siddur omits some customary texts which are now recited privately and can be found easily in other texts, such as Pirkei Avot.

The most important acknowledgment for this siddur is to Aaron Wolf, for his \textit{Siddur Olas Tamid}. This is a derivative of his work, with various changes to reflect the mostly Eastern Ashkenazi liturgical tradition I've received. His work, in turn, is largely derived from Rabbi Rallis Weisenthal's work \textit{Siddur Sefas Yisroel}. All biblical quotations are from Miqra Al Pi Masora, a remarkable project whose accurate and free text of Tanakh has helped me a great deal. I am also grateful to the Opensiddur community for supporting opensource liturgical ventures. The individuals behind these project have helped out the world of Jewish text distribution enormously. I would also like to thank Rabbanit Leah Sarna, for both highlighting the need for siddurim to include the text of Hatarat Nedarim for women, and for furnishing me with the text of it used in this siddur.\\

This work would not have been possible without the help of other individuals who assisted me with the technical aspects of creating this siddur. Noah Liebman, Benjamin Epstein, and Steven DuBois helped me learn the Python and \LaTeX{} skills necessary for this work. I would also like to thank my father, for bringing me to shul growing up, and for giving me an appreciation for the Jewish liturgical tradition. And my wife, for supporting my siddur-making interests. I hope this siddur is useful for others to daven from and repurpose.

\end{english}

\end{minipage}

\renewcommand{\contentsname}{}
\tableofcontents

%\clearpage
\mainmatter
\pagenumbering{arabic}

\setstretch{1.5}

\topskip0pt
\vspace*{\fill}

\thispagestyle{empty}
\begin{Large}
\begin{center}
\begin{tikzpicture}
\draw[-latex,white,postaction={decorate},decoration={text along path,
text={תמאב והארקי רשא לכל ויארק־לכל יי בורק},text align=center}]
(4,0) arc [start angle=180,end angle=0,radius=4];
\end{tikzpicture}
\end{center}
\end{Large}

\newcommand{\mizmorshabbat}{%
מִזְמ֥וֹר\source{תהילים צב} שִׁ֗יר לְי֣וֹם הַשַּׁבָּֽת׃ ט֗וֹב לְהֹד֥וֹת לַייָ֑ וּלְזַמֵּ֖ר לְשִׁמְךָ֣ עֶלְיֽוֹן׃ לְהַגִּ֣יד בַּבֹּ֣קֶר חַסְדֶּ֑ךָ וֶ֝אֱמ֥וּנָתְךָ֗ בַּלֵּילֽוֹת׃ עֲֽלֵי־עָ֭שׂוֹר וַעֲלֵי־נָ֑בֶל עֲלֵ֖י הִגָּי֣וֹן בְּכִנּֽוֹר׃ כִּ֤י שִׂמַּחְתַּ֣נִי יְיָ֣ בְּפׇעֳלֶ֑ךָ בְּֽמַעֲשֵׂ֖י יָדֶ֣יךָ אֲרַנֵּֽן׃ מַה־גָּדְל֣וּ מַעֲשֶׂ֣יךָ יְיָ֑ מְ֝אֹ֗ד עָמְק֥וּ מַחְשְׁבֹתֶֽיךָ׃ אִֽישׁ־בַּ֭עַר לֹ֣א יֵדָ֑ע וּ֝כְסִ֗יל לֹא־יָבִ֥ין אֶת־זֹֽאת׃ בִּפְרֹ֤חַ רְשָׁעִ֨ים ׀ כְּמ֥וֹ־עֵ֗שֶׂב וַ֭יָּצִיצוּ כׇּל־פֹּ֣עֲלֵי אָ֑וֶן לְהִשָּׁמְדָ֥ם עֲדֵי־עַֽד׃ וְאַתָּ֥ה מָר֗וֹם לְעֹלָ֥ם יְיָ׃ כִּ֤י הִנֵּ֪ה אֹיְבֶ֡יךָ ׀ יְיָ֗ כִּֽי־הִנֵּ֣ה אֹיְבֶ֣יךָ יֹאבֵ֑דוּ יִ֝תְפָּרְד֗וּ כׇּל־פֹּ֥עֲלֵי אָֽוֶן׃ וַתָּ֣רֶם כִּרְאֵ֣ים קַרְנִ֑י בַּ֝לֹּתִ֗י בְּשֶׁ֣מֶן רַעֲנָֽן׃ וַתַּבֵּ֥ט עֵינִ֗י בְּשׁ֫וּרָ֥י בַּקָּמִ֖ים עָלַ֥י מְרֵעִ֗ים תִּשְׁמַ֥עְנָה אׇזְנָֽי׃ צַ֭דִּיק כַּתָּמָ֣ר יִפְרָ֑ח כְּאֶ֖רֶז בַּלְּבָנ֣וֹן יִשְׂגֶּֽה׃ שְׁ֭תוּלִים בְּבֵ֣ית יְיָ֑ בְּחַצְר֖וֹת אֱלֹהֵ֣ינוּ יַפְרִֽיחוּ׃ ע֭וֹד יְנוּב֣וּן בְּשֵׂיבָ֑ה דְּשֵׁנִ֖ים וְֽרַעֲנַנִּ֣ים יִהְיֽוּ׃ לְ֭הַגִּיד כִּֽי־יָשָׁ֣ר יְיָ֑ צ֝וּרִ֗י וְֽלֹא־\qk{עַוְלָ֥תָה}{עלתה} בּֽוֹ׃}
\newcommand{\chanukat}{%
מִזְמ֡וֹר\source{תהילים ל} שִׁיר־חֲנֻכַּ֖ת הַבַּ֣יִת לְדָוִֽד׃ אֲרוֹמִמְךָ֣ יְיָ֭ כִּ֣י דִלִּיתָ֑נִי וְלֹֽא־שִׂמַּ֖חְתָּ אֹיְבַ֣י לִֽי׃ יְיָ֥ אֱלֹהָ֑י שִׁוַּ֥עְתִּי אֵ֝לֶ֗יךָ וַתִּרְפָּאֵֽנִי׃ יְיָ֗ הֶעֱלִ֣יתָ מִן־שְׁא֣וֹל נַפְשִׁ֑י חִ֝יִּיתַ֗נִי \qk{מִיׇּֽרְדִי־}{מיורדי}בֽוֹר׃ זַמְּר֣וּ לַייָ֣ חֲסִידָ֑יו וְ֝הוֹד֗וּ לְזֵ֣כֶר קׇדְשֽׁוֹ׃ כִּ֤י רֶ֨גַע ׀ בְּאַפּוֹ֮ חַיִּ֢ים בִּרְצ֫וֹנ֥וֹ בָּ֭עֶרֶב יָלִ֥ין בֶּ֗כִי וְלַבֹּ֥קֶר רִנָּֽה׃ וַ֭אֲנִי אָמַ֣רְתִּי בְשַׁלְוִ֑י בַּל־אֶמּ֥וֹט לְעוֹלָֽם׃ יְיָ֗ בִּרְצוֹנְךָ֮ הֶעֱמַ֢דְתָּה לְֽהַרְרִ֫י־עֹ֥ז הִסְתַּ֥רְתָּ פָנֶ֗יךָ הָיִ֥יתִי נִבְהָֽל׃ אֵלֶ֣יךָ יְיָ֣ אֶקְרָ֑א וְאֶל־אֲ֝דֹנָ֗י אֶתְחַנָּֽן׃ מַה־בֶּ֥צַע בְּדָמִי֮ בְּרִדְתִּ֢י אֶ֫ל־שָׁ֥חַת הֲיוֹדְךָ֥ עָפָ֑ר הֲיַגִּ֥יד אֲמִתֶּֽךָ׃ שְׁמַע־יְיָ֥ וְחׇנֵּ֑נִי יְ֝יָ֗ הֱֽיֵה־עֹזֵ֥ר לִֽי׃ הָפַ֣כְתָּ מִסְפְּדִי֮ לְמָח֢וֹל לִ֥י פִּתַּ֥חְתָּ שַׂקִּ֑י וַֽתְּאַזְּרֵ֥נִי שִׂמְחָֽה׃ לְמַ֤עַן ׀ יְזַמֶּרְךָ֣ כָ֭בוֹד וְלֹ֣א יִדֹּ֑ם יְיָ֥ אֱ֝לֹהַ֗י לְעוֹלָ֥ם אוֹדֶֽךָּ׃}
\newcommand{\lamenatzeachbinginot}{%
לַמְנַצֵּ֥חַ\source{תהילים סז} בִּנְגִינֹ֗ת מִזְמ֥וֹר שִֽׁיר׃ אֱֽלֹהִ֗ים יְחׇנֵּ֥נוּ וִיבָרְכֵ֑נוּ יָ֤אֵֽר פָּנָ֖יו אִתָּ֣נוּ סֶֽלָה׃ לָדַ֣עַת בָּאָ֣רֶץ דַּרְכֶּ֑ךָ בְּכׇל־גּ֝וֹיִ֗ם יְשׁוּעָתֶֽךָ׃ יוֹד֖וּךָ עַמִּ֥ים ׀ אֱלֹהִ֑ים י֝וֹד֗וּךָ עַמִּ֥ים כֻּלָּֽם׃ יִ֥שְׂמְח֥וּ וִירַנְּנ֗וּ לְאֻ֫מִּ֥ים כִּֽי־תִשְׁפֹּ֣ט עַמִּ֣ים מִישֹׁ֑ר וּלְאֻמִּ֓ים ׀ בָּאָ֖רֶץ תַּנְחֵ֣ם סֶֽלָה׃ יוֹד֖וּךָ עַמִּ֥ים ׀ אֱלֹהִ֑ים י֝וֹד֗וּךָ עַמִּ֥ים כֻּלָּֽם׃ אֶ֭רֶץ נָתְנָ֣ה יְבוּלָ֑הּ יְ֝בָרְכֵ֗נוּ אֱלֹהִ֥ים אֱלֹהֵֽינוּ׃ יְבָרְכֵ֥נוּ אֱלֹהִ֑ים וְיִֽירְא֥וּ א֝וֹת֗וֹ כׇּל־אַפְסֵי־אָֽרֶץ׃}

\newcommand{\kaddishamein}{(\nolinebreak\kahal\textbf{אָמֵן})\quad}

\newcommand{\kaddishtitle}[1]{\vspace{0.6\baselineskip}\instruction{#1}\nopagebreak\vspace{-0.6\baselineskip}\nopagebreak}

\newcommand{\kaddishstart}{

יִתְגַּדַּל וְיִתְקַדַּשׁ שְׁמֵיהּ רַבָּא \kaddishamein
בְּעָלְמָא דִּי בְרָא כִרְעוּתֵהּ וְיַמְלִיךְ מַלְכוּתֵהּ בְּחַיֵּיכוֹן וּבְיוֹמֵיכוֹן וּבְחַיֵּי דְכׇל־בֵּית יִשְׂרָאֵל בַּעֲגָלָא וּבִזְמַן קָרִיב׃ וְאִמְרוּ אָמֵן׃\\
\textbf{\kahal
אָמֵן׃ יְהֵא שְׁמֵיהּ רַבָּא מְבָרַךְ לְעָלַם וּלְעָלְמֵי עָלְמַיָּא׃}\\
יִתְבָּרַךְ וְיִשְׁתַּבַּח וְיִתְפָּאַר וְיִתְרוֹמַם וְיִתְנַשֵּׂא וְיִתְהַדַּר וְיִתְעַלֶּה וְיִתְהַלַּל שְׁמֵיהּ דְּקֻדְשָׁא בְּרִיךְ הוּא׃ *לְעֵֽלָּא מִן כׇּל־בִּרְכָתָא
(*\instruction{בעשי״ת:}
לְעֵֽלָּא לְעֵֽלָּא מִכׇּל־בִּרְכָתָא) וְשִׁירָתָא תֻּשְׁבְּחָתָא וְנֶחָמָתָא דַּאֲמִירָן בְּעָלְמָא וְאִמְרוּ אָמֵן׃
\kaddishamein
}

\newcommand{\osehshalom}{
עֹשֶׂה שָׁלוֹם בִּמְרוֹמָיו הוּא יַעֲשֶׂה שָׁלוֹם עָלֵֽינוּ וְעַל כׇּל־יִשְׂרָאֵל׃ וְאִמְרוּ אָמֵן׃
%עֹשֶׂה *שָׁלוֹם (*\instruction{בעשי״ת:} הַשָּׁלוֹם)
}

\newcommand{\halfkaddish}{
\kaddishtitle{חצי קדיש}
\footnotesize{\kaddishstart}
\normalsize
}

\newcommand{\fullkaddish}{\footnotesize{
\kaddishtitle{קדיש תתקבל}
\kaddishstart
% (\kahal
% \begin{footnotesize}
% קַבֵּל בְּרַחֲמִים וּבְרָצוֹן אֶת־תְּפִלָּתֵֽינוּ׃)\\
% \end{footnotesize}
תִּתְקַבַּל צְלוֹתְהוֹן וּבָעוּתְהוֹן דְּכׇל־יִשְׂרָאֵל קֳדָם אֲבוּהוֹן דִּי בִשְׁמַיָּא׃ וְאִמְרוּ אָמֵן׃
\kaddishamein
% (\kahal
% \begin{footnotesize}
%יְהִ֤י
% \source{תהלים קיג}%
% שֵׁ֣ם יְיָ֣ מְבֹרָ֑ךְ מֵֽ֝עַתָּ֗ה וְעַד־עוֹלָֽם׃)\\
% \end{footnotesize}
יְהֵא שְׁלָמָא רַבָּא מִן שְׁמַיָּא וְחַיִּים עָלֵֽינוּ וְעַל כׇּל־יִשְׂרָאֵל׃ וְאִמְרוּ אָמֵן׃
\kaddishamein
% (\kahal
% \begin{footnotesize}
% עֶ֭זְרִי
% \source{תהלים קכא}%
%מֵעִ֣ם יְיָ֑ עֹ֝שֵׂ֗ה שָׁמַ֥יִם וָאָֽרֶץ׃) \\
% \end{footnotesize}
\osehshalom
\kaddishamein
}\normalsize}

\newcommand{\mournerskaddish}{\footnotesize{
\kaddishtitle{קדיש יתום}
\kaddishstart
יְהֵא שְׁלָמָא רַבָּא מִן שְׁמַיָּא וְחַיִּים עָלֵֽינוּ וְעַל כׇּל־יִשְׂרָאֵל׃ וְאִמְרוּ אָמֵן׃
\kaddishamein
\osehshalom
\kaddishamein
\\}\normalsize
}

\newcommand{\alrabbanan}{
עַל יִשְׂרָאֵל וְעַל רַבָּנָן וְעַל תַּלְמִידֵיהוֹן וְעַל כׇּל־תַלְמִידֵי תַלְמִידֵיהוֹן וְעַל כׇּל־מָאן דְּעָסְקִין בְּאוֹרַיְתָא דִּי בְאַתְרָא הָדֵן וְדִי בְּכׇל־אֲתַר וַאֲתַר. יְהֵא לְהוֹן וּלְכוֹן שְׁלָמָא רַבָּא חִנָּא וְחִסְדָּא וְרַחֲמִין וְחַיִּין אֲרִיכִין וּמְזוֹנֵי רְוִיחֵי וּפֻרְקָנָא מִן־קֳדָם אֲבוּהוׂן דִּי בִשְׁמַיָּא׃ וְאִמְרוּ אָמֵן׃}

\newcommand{\rabbiskaddish}{\footnotesize{
\instruction{קדיש דרבנן}
\kaddishstart
\alrabbanan
\kaddishamein
יְהֵא שְׁלָמָא רַבָּא מִן שְׁמַיָּא וְחַיִּים עָלֵֽינוּ וְעַל כׇּל־יִשְׂרָאֵל׃ וְאִמְרוּ אָמֵן׃
\kaddishamein
\osehshalom
\kaddishamein
\\}\normalsize}

\newcommand{\itchadtastart}{
יִתְגַּדַּל וְיִתְקַדַּשׁ שְׁמֵיהּ רַבָּא 
\kaddishamein
בְּעָלְמָא דִּי הוּא עָתִיד לְאִתְחַדְתָּא \middot וּלְאַחֲיָאה מֵתַיָּא \middot וּלְאַסָּקָא יַתְּהוֹן לְחַיֵּי עָלְמָא \middot וּלְמִבְנָא קַרְתָּא דִּי יְרוּשְלֵם \middot וּלְשַׁכְלְלָא הֵיכָלֵהּ בְּגַוָּהּ \middot וּלְמֶעֱקַר פּוּלְחָנָא נוּכְרָאָה מִן אַרְעָה \middot וּלְאָתָבָא פּוּלְחָנָא דִּי שְׁמַיָּא לְאַתְרָהּ \middot וְיַמְלִיךְ קוּדְשָׁא בְּרִיךְ הוּא בּמַלְכוּתֵהּ וִיקָרֵהּ בְּחַיֵּיכוֹן וּבְיוֹמֵיכוֹן וּבְחַיֵּי דְכׇל־בֵּית יִשְׂרָאֵל בַּעֲגָלָא וּבִזְמַן קָרִיב׃ וְאִמְרוּ אָמֵן׃\\
}

\newcommand{\kaddishitchadeta}{
\itchadtastart
\textbf{\kahal
אָמֵן׃ יְהֵא שְׁמֵיהּ רַבָּא מְבָרַךְ לְעָלַם וּלְעָלְמֵי עָלְמַיָּא}\\
יִתְבָּרַךְ וְיִשְׁתַּבַּח וְיִתְפָּאַר וְיִתְרוֹמַם וְיִתְנַשֵּׂא וְיִתְהַדַּר וְיִתְעַלֶּה וְיִתְהַלַּל שְׁמֵיהּ דְּקֻדְשָׁא
\textbf{בְּרִיךְ הוּא}
׃ *לְעֵֽלָּא מִן כׇּל־בִּרְכָתָא
(*\instruction{בעשי״ת:}
לְעֵֽלָּא לְעֵֽלָּא מִכׇּל־בִּרְכָתָא) וְשִׁירָתָא תֻּשְׁבְּחָתָא וְנֶחָמָתָא דַּאֲמִירָן בְּעָלְמָא וְאִמְרוּ אָמֵן׃
\kaddishamein
\alrabbanan
\kaddishamein
יְהֵא שְׁלָמָא רַבָּא מִן שְׁמַיָּא וְחַיִּים עָלֵֽינוּ וְעַל כׇּל־יִשְׂרָאֵל׃ וְאִמְרוּ אָמֵן׃
\kaddishamein
\osehshalom
\kaddishamein}

\newcommand{\adonolam}{
	
	\firstword{אֲדוֹן עוֹלָם}
	אֲשֶׁר מָלַךְ \hfill בְּטֶֽרֶם כׇּל־יְצִיר נִבְרָא׃ \\
	לְעֵת נַעֲשָׂה בְחֶפְצוֹ כֹּל \hfill אֲזַי מֶֽלֶךְ שְׁמוֹ נִקְרָא׃\\
	וְאַֽחֲרֵי כִּכְלוֹת הַכֹּל \hfill לְבַדּוֹ יִמְלֹךְ נוֹרָא׃ \\
	וְהוּא הָיָה וְהוּא הוֶֹה \hfill וְהוּא יִהְיֶה בְּתִפְאָרָה׃ \\
	וְהוּא אֶחָד וְאֵין שֵׁנִי \hfill לְהַמְשִׁיל לוֹ לְהַחְבִּֽירָה׃ \\
	בְּלִי רֵאשִׁית בְּלִי תַכְלִית \hfill וְלוֹ הָעֹז וְהַמִּשְׂרָה׃ \\
	וְהוּא אֵלִי וְחַי גוֹאֲלִי \hfill וְצוּר חֶבְלִי בְּעֵת צָרָה׃ \\
	וְהוּא נִסִּי וּמָנוֹסִי \hfill מְנָת כּוֹסִי בְּיוֹם אֶקְרָא׃ \\
	בְּיָדוֹ אַפְקִיד רוּחִי \hfill בְּעֵת אִישַׁן וְאָעִֽירָה׃ \\
	וְעִם רוּחִי גְּוִיָּתִי \hfill אֲדוֹנָי לִי וְלֹא אִירָא׃\\
}

\newcommand{\tamid}{
	%יְהִי רָצוֹן מִלְּפָנֶֽיךָ יְיָ אֱלֹהֵינוּ וֵאלֹהֵי אֲבוֹתֵינוּ, שֶׁתְּרַחֵם עָלֵינוּ וְתִמְחוֹל לָנוּ עַל כׇּל־חַטֹּאתֵינוּ וּתְכַפֶּר לָנוּ עַל כׇּל־עֲוֹנוֹתֵינוּ וְתִמְחוֹל וְתִסְלַח לָנוּ עַל כׇּל־פְּשָׁעֵינוּ וְשֶׁיִבָּנֶה בֵּית הַמִּקְדָּשׁ בִּמְהֵרָה בְיָמֵינוּ וְנַקְרִיב לְפָנֶיךָ קׇרְבַּן הַתָּמִיד שֶׁיְּכַפֵּר בַּעֲדֵנוּ כְּמוֹ שֶׁכָּתַבְתָּ עָלֵינוּ בְּתוֹרָתֶךָ עַל יְדֵי משֶׁה עַבְדֶּךָ מִפִּי כְבוֹדֶךָ כָּאָמוּר׃\\
וַיְדַבֵּ֥ר\source{במדבר כח} יְיָ֖ אֶל־מֹשֶׁ֥ה לֵּאמֹֽר׃ צַ֚ו אֶת־בְּנֵ֣י יִשְׂרָאֵ֔ל וְאָמַרְתָּ֖ אֲלֵהֶ֑ם אֶת־קׇרְבָּנִ֨י לַחְמִ֜י לְאִשַּׁ֗י רֵ֚יחַ נִֽיחֹחִ֔י תִּשְׁמְר֕וּ לְהַקְרִ֥יב לִ֖י בְּמוֹעֲדֽוֹ׃ וְאָמַרְתָּ֣ לָהֶ֔ם זֶ֚ה הָֽאִשֶּׁ֔ה אֲשֶׁ֥ר תַּקְרִ֖יבוּ לַייָ֑ כְּבָשִׂ֨ים בְּנֵֽי־שָׁנָ֧ה תְמִימִ֛ם שְׁנַ֥יִם לַיּ֖וֹם עֹלָ֥ה תָמִֽיד׃ אֶת־הַכֶּ֥בֶשׂ אֶחָ֖ד תַּעֲשֶׂ֣ה בַבֹּ֑קֶר וְאֵת֙ הַכֶּ֣בֶשׂ הַשֵּׁנִ֔י תַּעֲשֶׂ֖ה בֵּ֥ין הָֽעַרְבָּֽיִם׃ וַעֲשִׂירִ֧ית הָאֵיפָ֛ה סֹ֖לֶת לְמִנְחָ֑ה בְּלוּלָ֛ה בְּשֶׁ֥מֶן כָּתִ֖ית רְבִיעִ֥ת הַהִֽין׃ עֹלַ֖ת תָּמִ֑יד הָעֲשֻׂיָה֙ בְּהַ֣ר סִינַ֔י לְרֵ֣יחַ נִיחֹ֔חַ אִשֶּׁ֖ה לַֽייָ׃ וְנִסְכּוֹ֙ רְבִיעִ֣ת הַהִ֔ין לַכֶּ֖בֶשׂ הָאֶחָ֑ד בַּקֹּ֗דֶשׁ הַסֵּ֛ךְ נֶ֥סֶךְ שֵׁכָ֖ר לַייָ׃ וְאֵת֙ הַכֶּ֣בֶשׂ הַשֵּׁנִ֔י תַּעֲשֶׂ֖ה בֵּ֣ין הָֽעַרְבָּ֑יִם כְּמִנְחַ֨ת הַבֹּ֤קֶר וּכְנִסְכּוֹ֙ תַּעֲשֶׂ֔ה אִשֵּׁ֛ה רֵ֥יחַ נִיחֹ֖חַ לַייָ׃	
וְשָׁחַ֨ט\source{ויקרא א} אֹת֜וֹ עַ֣ל יֶ֧רֶךְ הַמִּזְבֵּ֛חַ צָפֹ֖נָה לִפְנֵ֣י יְיָ֑ וְזָרְק֡וּ בְּנֵי֩ אַהֲרֹ֨ן הַכֹּהֲנִ֧ים אֶת־דָּמ֛וֹ עַל־הַמִּזְבֵּ֖חַ סָבִֽיב׃	%יְהִי רָצוֹן מִלְּפָנֶֽיךָ, יְיָ אֱלֹהֵֽינוּ וֵאלֹהֵי אֲבוֹתֵֽינוּ, שֶׁתְּהֵא אֲמִירָה זוֹ חֲשׁוּבָה וּמְקֻבֶּֽלֶת וּמְרֻצָּה לְפָנֶֽיךָ, כְּאִלּוּ הִקְרַֽבְנוּ קׇרְבַּן הַתָּמִיד בְּמוֹעֲדוֹ וּבִמְקוֹמוֹ וּכְהִלְכָתוֹ.
}

\newcommand{\ketoret}{
	\firstword{אַתָּה הוּא יְיָ אֱלֹהֵינוּ שֶׁהִקְטִירוּ}
	אֲבוֹתֵינוּ לְפָנֶיךָ אֶת־קְטֹרֶת הַסַּמִּים בִּזְמַן שֶׁבֵּית הַמִּקְדָּשׁ קַיָּם. כַּאֲשֶׁר צִוִּיתָ אוֹתָם עַל יְדֵי מֹשֶׁה נְבִיאֶךָ. כַּכָּתוּב בְּתוֹרָתֶךָ׃ \source{שמות ל} 
	וַיֹּ֩אמֶר֩ יְיָ֨ אֶל־מֹשֶׁ֜ה קַח־לְךָ֣ סַמִּ֗ים נָטָ֤ף ׀ וּשְׁחֵ֙לֶת֙ וְחֶלְבְּנָ֔ה סַמִּ֖ים וּלְבֹנָ֣ה זַכָּ֑ה בַּ֥ד בְּבַ֖ד יִהְיֶֽה׃ וְעָשִׂ֤יתָ אֹתָהּ֙ קְטֹ֔רֶת רֹ֖קַח מַעֲשֵׂ֣ה רוֹקֵ֑חַ מְמֻלָּ֖ח טָה֥וֹר קֹֽדֶשׁ׃ וְשָֽׁחַקְתָּ֣ מִמֶּ֘נָּה֮ הָדֵק֒ וְנָתַתָּ֨ה מִמֶּ֜נָּה לִפְנֵ֤י הָעֵדֻת֙ בְּאֹ֣הֶל מוֹעֵ֔ד אֲשֶׁ֛ר אִוָּעֵ֥ד לְךָ֖ שָׁ֑מָּה קֹ֥דֶשׁ קׇֽדָשִׁ֖ים תִּהְיֶ֥ה לָכֶֽם׃\\
	וְנֶאֱמַר׃ וְהִקְטִ֥יר עָלָ֛יו אַהֲרֹ֖ן קְטֹ֣רֶת סַמִּ֑ים בַּבֹּ֣קֶר בַּבֹּ֗קֶר בְּהֵיטִיב֛וֹ אֶת־הַנֵּרֹ֖ת יַקְטִירֶֽנָּה׃ וּבְהַעֲלֹ֨ת אַהֲרֹ֧ן אֶת־הַנֵּרֹ֛ת בֵּ֥ין הָעַרְבַּ֖יִם יַקְטִירֶ֑נָּה קְטֹ֧רֶת תָּמִ֛יד לִפְנֵ֥י יְיָ֖ לְדֹרֹתֵיכֶֽם׃
}

\newcommand{\melekhmalakhyimlokh}{יְיָ מֶֽלֶךְ \middot יְיָ מָלָךְ \middot יְיָ יִמְלֹךְ לְעוֹלָם וָעֶד׃}

\newcommand{\ashrei}{
	
	\firstword{אַ֭שְׁרֵי יוֹשְׁבֵ֣י בֵיתֶ֑ךָ ע֗֝וֹד יְֽהַלְל֥וּךָ סֶּֽלָה׃ }\source{תהלים פד}\\
	\textbf{אַשְׁרֵ֣י הָ֭עָם שֶׁכָּ֣כָה לּ֑וֹ אַֽשְׁרֵ֥י הָ֝עָ֗ם שֱׁייָ֥ אֱלֹהָֽיו׃ }\source{תהלים קמד}\\
	%used to have dots for the letters of the alphabet, but it makes it too messy with the trop marks
	תְּהִלָּ֗ה לְדָ֫וִ֥ד\source{תהלים קמה}
	אֲרוֹמִמְךָ֣ אֱלוֹהַ֣י הַמֶּ֑לֶךְ וַאֲבָרְכָ֥ה שִׁ֝מְךָ֗ לְעוֹלָ֥ם וָעֶֽד׃
	בְּכׇל־י֥וֹם אֲבָֽרְכֶ֑ךָּ וַאֲהַֽלְלָ֥ה שִׁ֝מְךָ֗ לְעוֹלָ֥ם וָעֶֽד׃
	גָּ֘ד֤וֹל יְיָ֣ וּמְהֻלָּ֣ל מְאֹ֑ד וְ֝לִגְדֻלָּת֗וֹ אֵ֣ין חֵֽקֶר׃
	דּ֣וֹר לְ֭דוֹר יְשַׁבַּ֣ח מַעֲשֶׂ֑יךָ וּגְב֖וּרֹתֶ֣יךָ יַגִּֽידוּ׃
	הֲ֭דַר כְּב֣וֹד הוֹדֶ֑ךָ וְדִבְרֵ֖י נִפְלְאֹתֶ֣יךָ אָשִֽׂיחָה׃
	וֶעֱז֣וּז נֽוֹרְאֹתֶ֣יךָ יֹאמֵ֑רוּ \qk{וּגְדֻלָּתְךָ֥}{וגדלותיך} אֲסַפְּרֶֽנָּה׃
	זֵ֣כֶר רַב־טוּבְךָ֣ יַבִּ֑יעוּ וְצִדְקָתְךָ֥ יְרַנֵּֽנוּ׃
	חַנּ֣וּן וְרַח֣וּם יְיָ֑ אֶ֥רֶךְ אַ֝פַּ֗יִם וּגְדׇל־חָֽסֶד׃
	טוֹב־יְיָ֥ לַכֹּ֑ל וְ֝רַחֲמָ֗יו עַל־כׇּל־מַעֲשָֽׂיו׃
	יוֹד֣וּךָ יְיָ֭ כׇּל־מַעֲשֶׂ֑יךָ וַ֝חֲסִידֶ֗יךָ יְבָרְכֽוּכָה׃
	כְּב֣וֹד מַלְכוּתְךָ֣ יֹאמֵ֑רוּ וּגְבוּרָתְךָ֥ יְדַבֵּֽרוּ׃
	לְהוֹדִ֤יעַ ׀ לִבְנֵ֣י הָ֭אָדָם גְּבוּרֹתָ֑יו וּ֝כְב֗וֹד הֲדַ֣ר מַלְכוּתֽוֹ׃
	מַֽלְכוּתְךָ֗ מַלְכ֥וּת כׇּל־עֹלָמִ֑ים וּ֝מֶֽמְשַׁלְתְּךָ֗ בְּכׇל־דּ֥וֹר וָדֹֽר׃
	סוֹמֵ֣ךְ יְיָ֭ לְכׇל־הַנֹּפְלִ֑ים וְ֝זוֹקֵ֗ף לְכׇל־הַכְּפוּפִֽים׃
	עֵֽינֵי־כֹ֭ל אֵלֶ֣יךָ יְשַׂבֵּ֑רוּ וְאַתָּ֤ה נֽוֹתֵן־לָהֶ֖ם אֶת־אׇכְלָ֣ם בְּעִתּֽוֹ׃
	פּוֹתֵ֥חַ אֶת־יָדֶ֑ךָ וּמַשְׂבִּ֖יעַ לְכׇל־חַ֣י רָצֽוֹן׃
	צַדִּ֣יק יְיָ֭ בְּכׇל־דְּרָכָ֑יו וְ֝חָסִ֗יד בְּכׇל־מַעֲשָֽׂיו׃
	קָר֣וֹב יְיָ֭ לְכׇל־קֹרְאָ֑יו לְכֹ֤ל אֲשֶׁ֖ר יִקְרָאֻ֣הוּ בֶֽאֱמֶֽת׃ 
	רְצוֹן־יְרֵאָ֥יו יַעֲשֶׂ֑ה וְֽאֶת־שַׁוְעָתָ֥ם יִ֝שְׁמַ֗ע וְיוֹשִׁיעֵֽם׃ 
	שׁוֹמֵ֣ר יְיָ֭ אֶת־כׇּל־אֹהֲבָ֑יו וְאֵ֖ת כׇּל־הָרְשָׁעִ֣ים יַשְׁמִֽיד׃ 
	תְּהִלַּ֥ת יְיָ֗ יְֽדַבֶּ֫ר פִּ֥י וִיבָרֵ֣ךְ כׇּל־בָּ֭שָׂר שֵׁ֥ם קׇדְשׁ֗וֹ לְעוֹלָ֥ם וָעֶֽד׃
	\source{תהלים קטו}וַאֲנַ֤חְנוּ ׀ נְבָ֘רֵ֤ךְ יָ֗הּ מֵעַתָּ֥ה וְעַד־עוֹלָ֗ם הַֽלְלוּ־יָֽהּ׃
	
}

\newcommand{\mimaamakim}{
	
	
	\begin{sometimes}\\
		
		\englishinst{During the Ten Penitential Days and Hoshana Rabba, the ark is opened for the following Psalm. The reader chants each verse, and the congregation repeats it.}
		%\instruction{בעשי״ת והושענא רבא, פותחים הארון. הש״ץ קורא בפסוק בפסוק והקהל חוזר׃}\\
		\firstword{שִׁ֥יר הַֽמַּעֲל֑וֹת מִמַּעֲמַקִּ֖ים}\source{תהלים קל}
		קְרָאתִ֣יךָ יְיָ׃\ \hfill\break
		אֲדֹנָי֮ שִׁמְעָ֢ה בְק֫וֹלִ֥י תִּהְיֶ֣ינָה אׇ֭זְנֶיךָ קַשֻּׁב֑וֹת לְ֝ק֗וֹל תַּחֲנוּנָֽי׃\hfill \break
		אִם־עֲוֺנ֥וֹת תִּשְׁמׇר־יָ֑הּ אֲ֝דֹנָ֗י מִ֣י יַעֲמֹֽד׃\hfill \break
		כִּֽי־עִמְּךָ֥ הַסְּלִיחָ֑ה לְ֝מַ֗עַן תִּוָּרֵֽא׃\hfill \break
		קִוִּ֣יתִי יְיָ֭ קִוְּתָ֣ה נַפְשִׁ֑י וְֽלִדְבָר֥וֹ הוֹחָֽלְתִּי׃\hfill \break
		נַפְשִׁ֥י לַאדֹנָ֑י מִשֹּׁמְרִ֥ים לַ֝בֹּ֗קֶר שֹׁמְרִ֥ים לַבֹּֽקֶר׃\hfill \break
		יַחֵ֥ל יִשְׂרָאֵ֗ל אֶל־יְ֫יָ֥ כִּֽי־עִם־יְיָ֥ הַחֶ֑סֶד וְהַרְבֵּ֖ה עִמּ֣וֹ פְדֽוּת׃\hfill \break
		וְ֭הוּא יִפְדֶּ֣ה אֶת־יִשְׂרָאֵ֑ל מִ֝כֹּ֗ל עֲוֺנֹתָֽיו׃ \instruction{סוגרים הארון}\hfill \break
		
	\end{sometimes}
	
}

\newcommand{\tzadialeph}{
יֹ֭שֵׁב\source{תהילים צא} בְּסֵ֣תֶר עֶלְי֑וֹן בְּצֵ֥ל שַׁ֝דַּ֗י יִתְלוֹנָֽן׃ אֹמַ֗ר לַ֭ייָ מַחְסִ֣י וּמְצוּדָתִ֑י אֱ֝לֹהַ֗י אֶבְטַח־בּֽוֹ׃ כִּ֤י ה֣וּא יַ֭צִּילְךָ מִפַּ֥ח יָק֗וּשׁ מִדֶּ֥בֶר הַוּֽוֹת׃ בְּאֶבְרָת֨וֹ ׀ יָ֣סֶךְ לָ֭ךְ וְתַחַת־כְּנָפָ֣יו תֶּחְסֶ֑ה צִנָּ֖ה וְסֹחֵרָ֣ה אֲמִתּֽוֹ׃ לֹֽא־תִ֭ירָא מִפַּ֣חַד לָ֑יְלָה מֵ֝חֵ֗ץ יָע֥וּף יוֹמָֽם׃ מִ֭דֶּבֶר בָּאֹ֣פֶל יַהֲלֹ֑ךְ מִ֝קֶּ֗טֶב יָשׁ֥וּד צׇהֳרָֽיִם׃ יִפֹּ֤ל מִצִּדְּךָ֨ ׀ אֶ֗לֶף וּרְבָבָ֥ה מִימִינֶ֑ךָ אֵ֝לֶ֗יךָ לֹ֣א יִגָּֽשׁ׃ רַ֭ק בְּעֵינֶ֣יךָ תַבִּ֑יט וְשִׁלֻּמַ֖ת רְשָׁעִ֣ים תִּרְאֶֽה׃ כִּֽי־אַתָּ֣ה יְיָ֣ מַחְסִ֑י עֶ֝לְי֗וֹן שַׂ֣מְתָּ מְעוֹנֶֽךָ׃ לֹא־תְאֻנֶּ֣ה אֵלֶ֣יךָ רָעָ֑ה וְ֝נֶ֗גַע לֹא־יִקְרַ֥ב בְּאׇהֳלֶֽךָ׃ כִּ֣י מַ֭לְאָכָיו יְצַוֶּה־לָּ֑ךְ לִ֝שְׁמׇרְךָ֗ בְּכׇל־דְּרָכֶֽיךָ׃ עַל־כַּפַּ֥יִם יִשָּׂא֑וּנְךָ פֶּן־תִּגֹּ֖ף בָּאֶ֣בֶן רַגְלֶֽךָ׃ עַל־שַׁ֣חַל וָפֶ֣תֶן תִּדְרֹ֑ךְ תִּרְמֹ֖ס כְּפִ֣יר וְתַנִּֽין׃ כִּ֤י בִ֣י חָ֭שַׁק וַאֲפַלְּטֵ֑הוּ אֲ֝שַׂגְּבֵ֗הוּ כִּֽי־יָדַ֥ע שְׁמִֽי׃ יִקְרָאֵ֨נִי ׀ וְֽאֶעֱנֵ֗הוּ עִמּֽוֹ־אָנֹכִ֥י בְצָרָ֑ה אֲ֝חַלְּצֵ֗הוּ וַאֲכַבְּדֵֽהוּ׃ אֹ֣רֶךְ יָ֭מִים אַשְׂבִּיעֵ֑הוּ וְ֝אַרְאֵ֗הוּ בִּישׁוּעָתִֽי׃	\scriptsize{אֹ֣רֶךְ יָ֭מִים אַשְׂבִּיעֵ֑הוּ וְ֝אַרְאֵ֗הוּ בִּישׁוּעָתִֽי׃}
	\normalsize{}
}

\newcommand{\nishmat}{
\firstword{נִשְׁמַת}
כׇּל־חַי תְּבָרֵךְ אֶת־שִׁמְךָ יְיָ אֱלֹהֵֽינוּ \middot וְרֽוּחַ כׇּל־בָּשָׂר תְּפָאֵר וּתְרוֹמֵם זִכְרְךָ מַלְכֵּֽנוּ תָּמִיד \middot מִן הָעוֹלָם וְעַד הָעוֹלָם אַתָּה אֵל וּמִבַּלְעָדֶֽיךָ אֵין לָֽנּוּ מֶֽלֶךְ גּוֹאֵל וּמוֹשִֽׁיעַ פּוֹדֶה וּמַצִיל וּמְפַרְנֵס וּמְרַחֵם בְּכׇל־עֵת צָרָה וְצוּקָה \middot אֵין לָֽנוּ מֶֽלֶךְ אֶלָּא אַֽתָּה׃ אֱלֹהֵי הָרִאשׁוֹנִים וְהָאַחֲרוֹנִים \middot אֱלֽוֹהַּ כׇּל־בְּרִיּוֹת \middot אֲדוֹן כׇּל־תּוֹלָדוֹת הַמְהֻלָּל בְּרֹב תִּשְׁבָּחוֹת \middot הַמְנַהֵג עוֹלָמוֹ בְּחֶֽסֶד וּבְרִיּוֹתָיו בְּרַחֲמִים׃ וַייָ לֹא יָנוּם וְלֹא יִישָׁן \middot הַמְעוֹרֵר יְשֵׁנִים וְהַמֵּקִיץ רְדוּמִים וְהַמֵּשִֽׂיחַ אִלְּמִים וְהַמַּתִּיר אֲסוּרִים וְהַסּוֹמֵךְ נוֹפְלִים וְהַזּוֹקֵף כְּפוּפִים \middot לְךָ לְבַדְּךָ אֲנַֽחְנוּ מוֹדִים׃ 
אִֽלּוּ פִֽינוּ מָלֵא שִׁירָה כַּיָּם וּלְשׁוֹנֵֽנוּ רִנָּה כַּהֲמוֹן גַּלָּיו וְשִׂפְתוֹתֵֽינוּ שֶֽׁבַח כְּמֶרְחֲבֵי רָקִֽיעַ וְעֵינֵֽינוּ מְאִירוֹת כַּשֶּֽׁמֶשׁ וְכַיָּרֵֽחַ וְיָדֵֽינוּ פְּרוּשׂוֹת כְּנִשְׁרֵי שָּׁמָֽיִם וְרַגְלֵֽינוּ קַלּוֹת כָּאַיָּלוֹת \middot אֵין אָֽנוּ מַסְפִּיקִים לְהוֹדוֹת לְךָ יְיָ אֱלֹהֵֽינוּ וֵאלֹהֵי אֲבוֹתֵֽינוּ וּלְבָרֵךְ אֶת־שְׁמֶֽךָ \middot עַל אַֽחַת מֵאָֽלֶף אֶֽלֶף אַלְפֵי אֲלָפִים וְרִבֵּי רְבָבוֹת פְּעָמִים הַטּוֹבוֹת שֶׁעָשִֽׂיתָ עִם אֲבוֹתֵֽינוּ וְעִמָּֽנוּ׃ 
מִמִּצְרַֽיִם גְּאַלְתָּֽנוּ יְיָ אֱלֹהֵֽינוּ וּמִבֵּית עֲבָדִים פְּדִיתָֽנוּ \middot בָּרָעָב זַנְתָּֽנוּ וּבְשָׂבָע כִּלְכַּלְתָּֽנוּ מֵחֶֽרֶב הִצַּלְתָּֽנּוּ וּמִדֶּֽבֶר מִלַּטְתָּֽנוּ וּמֵחֳלָיִם רָעִים וְנֶאֱמָנִים דִּלִּיתָֽנוּ \middot עַד הֵֽנָּה עֲזָרֽוּנוּ רַחֲמֶֽיךָ וְלֹא עֲזָבֽוּנוּ חֲסָדֶֽיךָ \middot וְאַל תִּטְּשֵׁנוּ יְיָ אֱלֹהֵֽינוּ לָנֶֽצַח׃ עַל כֵּן אֵבָרִים שֶׁפִּלַּגְתָּ בָּֽנוּ וְרֽוּחַ וּנְשָׁמָה שֶׁנָּפַֽחְתָּ בְּאַפֵּֽינוּ וְלָשׁוֹן אֲשֶׁר שַֽׂמְתָּ בְּפִֽינוּ \middot הֵן הֵם יוֹדוּ וִיבָרְכוּ וִישַׁבְּחוּ וִיפָאֲרוּ וִירוֹמֲמוּ וְיַעֲרִֽיצוּ וְיַקְדִּֽישׁוּ וְיַמְלִֽיכוּ אֶת־שִׁמְךָ מַלְכֵּֽנוּ׃ כִּי כׇל־פֶּה לְךָ יוֹדֶה וְכׇל־לָשׁוֹן לְךָ תִּשָּׁבַע וְכׇל־בֶּֽרֶךְ לְךָ תִּכְרַע וְכׇל־קוֹמָה לְפָנֶֽיךָ תִשְׁתַּחֲוֶה וְכׇל־לְבָבוֹת יִירָאֽוּךָ וְכׇל־קֶֽרֶב וּכְלָיוֹת יְזַמְּרוּ לִשְׁמֶֽךָ \middot כַּדָּבָר שֶׁכָּתוּב׃
כׇּ‍֥ל \source{תהלים לה}עַצְמוֹתַ֨י ׀ תֹּאמַרְנָה֮ יְיָ֗ מִ֥י כָ֫מ֥וֹךָ מַצִּ֣יל עָ֭נִי מֵחָזָ֣ק מִמֶּ֑נּוּ וְעָנִ֥י וְ֝אֶבְי֗וֹן מִגֹּֽזְלֽוֹ׃
מִי יִדְמֶה־לָּךְ וּמִי יִשְׁוֶה־לָּךְ וּמִי יַעֲרׇךְ־לָּךְ׃ הָאֵל הַגָּדוֹל הַגִּבּוֹר וְהַנּוֹרָא אֵל עֶלְיוֹן קֹנֵה שָׁמַֽיִם וָאָֽרֶץ׃
נְהַלֶּלְךָ וּנְשַׁבֵּחֲךָ וּנְפָאֶרְךָ וּנְבָרֵךְ אֶת־שֵׁם קׇדְשֶׁךָ כָּאָמוּר׃
לְדָוִ֨ד \source{תהלים קג} ׀ בָּרְכִ֣י נַ֭פְשִׁי אֶת־יְיָ֑ וְכׇל־קְ֝רָבַ֗י אֶת־שֵׁ֥ם קׇדְשֽׁוֹ׃
}

\newcommand{\hael}{
\firstword{הָאֵל}
בְּתַעֲצֻמוֹת עֻזֶּֽךָ \middot \\
\firstword{הַגָּדוֹל}
בִּכְבוֹד שְׁמֶֽךָ \middot \\
\firstword{הַגִּבּוֹר}
לָנֶֽצַח וְהַנּוֹרָא בְּנוֹרְאוֹתֶֽיךָ \middot\\
\firstword{הַמֶּֽלֶךְ}
הַיּוֹשֵׁב עַל כִּסֵּא רָם וְנִשָּׂא׃
}

\newcommand{\shochenad}{
\firstword{שׁוֹכֵן}
עַד מָרוֹם וְקָדוֹשׁ שְׁמוֹ \middot וְכָתוּב׃ רַֽנֲנ֣וּ \source{תהלים לג}צַ֭דִּיקִים בַּֽיְיָ֑ לַ֝יְשָׁרִ֗ים נָאוָ֥ה תְהִלָּֽה׃ בְּפִי יְשָׁרִים תִּתְהַלָּל \middot וּבְדִבְרֵי צַדִּיקִים תִּתְבָּרַךְ \middot וּבִלְשׁוֹן חֲסִידִים תִּתְרוֹמָם \middot וּבְקֶֽרֶב קְדוֹשִׁים תִּתְקַדָּשׁ׃

\firstword{וּבְמַקְהֲלוֹת}
רִבְבוֹת עַמְּךָ בֵּית יִשְׂרָאֵל בְּרִנָּה יִתְפָּאַר שִׁמְךָ מַלְכֵּֽנוּ בְּכׇל־דּוֹר וָדוֹר \middot שֶׁכֵּן חוֹבַת כׇּל־הַיְצוּרִים לְפָנֶֽיךָ יְיָ אֱלֹהֵֽינוּ וֵאלֹהֵי אֲבוֹתֵֽינוּ לְהוֹדוֹת לְהַלֵּל לְשַׁבֵּֽחַ לְפָאֵר לְרוֹמֵם לְהַדֵּר לְבָרֵךְ לְעַלֵּה וּלְקַלֵּס עַל כׇּל־דִּבְרֵי שִׁירוֹת וְתֻשְׁבְּחוֹת דָּוִד בֶּן־יִשַׁי עַבְדְּךָ מְשִׁיחֶֽךָ׃
}

\newcommand{\barachu}{
	%\begin{wrapfigure}[5]{I}{0.6\textwidth}
	\begin{minipage}{0.8\textwidth}
		\leftskip=0pt plus-.5fil
		\rightskip=0pt plus.5fil
		\parfillskip=0pt plus1fil
		\begin{large}
			
			\shatz
			\begin{Large}\textbf{בָּרְכוּ אֶת־יְיָ הַמְבֹרָךְ׃}\end{Large}
		\end{large}
		
		\vspace{12pt}
		
		\shatzvkahal
		בָּרוּךְ יְיָ הַמְבֹרָךְ לְעוֹלָם וָעֶד׃
	\end{minipage}
	ֺֺ%\end{wrapfigure}
	
	%\begin{footnotesize}
	%יִתְבָּרַךְ וְיִשְׁתַּבַּח וְיִתְפָּאַר וְיִתְרוֹמַם וְיִתְנַשֵּׂא שְׁמוֹ שֶׁל מֶֽלֶךְ מַלְכֵי הַמְּלָכִים הַקָּדוֹשׁ בָּרוּךְ הוּא׃ שֶׁהוּא רִאשׁוֹן וְהוּא אַחֲרוֹן וּמִבַּלְעָדָיו אֵין אֱלֹהִים׃ \source{תהלים סח}סֹ֡לּוּ לָרֹכֵ֣ב בָּ֭עֲרָבוֹת בְּיָ֥הּ שְׁמ֗וֹ וְעִלְז֥וּ לְפָנָֽיו׃ וּשְׁמוֹ מְרוֹמָם עַל־כׇּל־בְּרָכָה וּתְהִלָּה׃ בָּרוּךְ שֵׁם כְּבוֺד מַלְכוּתוֺ לְעוֺלָם וָעֶד׃ \source{תהלים קיג}יְהִ֤י שֵׁ֣ם יְיָ֣ מְבֹרָ֑ךְ מֵ֝עַתָּ֗ה וְעַד־עוֹלָֽם׃
	%
	%\end{footnotesize}
}

\newcommand{\hameir}{
	\firstword{הַמֵּאִיר}
	לָאָֽרֶץ וְלַדָּרִים עָלֶֽיהָ בְּרַחֲמִים \middot וּבְטוּבוֹ מְחַדֵּשׁ בְּכׇל־יוֹם תָּמִיד מַעֲשֵׂה בְרֵאשִׁית׃ 
	\ifboolexpr{togl {includeshabbat} and togl {includefestival}}{\mdsource{תהלים קד}}{\source{תהלים קד}}
	מָה־רַבּ֬וּ מַעֲשֶׂ֨יךָ ׀ יְיָ֗ כֻּ֭לָּם בְּחׇכְמָ֣ה עָשִׂ֑יתָ מָלְאָ֥ה הָ֝אָ֗רֶץ קִנְיָנֶֽךָ׃ הַמֶּֽלֶךְ הַמְרוֹמָם לְבַדּוֹ מֵאָז \middot הַמְשֻׁבָּח וְהַמְפֹאָר וְהַמִּתְנַשֵּׂא מִימוֹת עוֹלָם׃ אֱלֹהֵי עוֹלָם בְּרַחֲמֶיךָ הָרַבִּים רַחֵם עָלֵינוּ \middot אֲדוֹן עֻזֵּֽנוּ צוּר מִשְׂגַּבֵּנוּ מָגֵן יִשְׁעֵֽנוּ מִשְׂגָּב בַּעֲדֵֽנוּ׃ אֵ֗ל בָּ֗רוּךְ גְּ֗דוֹל דֵּ֗עָה \middot הֵ֗כִין וּ֗פָעַל זׇ֗הֳרֵי חַ֗מָּה \middot ט֗וֹב יָ֗צַר כָּ֗בוֹד לִ֗שְׁמוֹ \middot מְ֗אוֹרוֹת נָ֗תַן סְ֗בִיבוֹת עֻ֗זּוֹ \middot פִּ֗נּוֹת צְ֗בָאָיו קְ֗דוֹשִׁים ר֗וֹמֲמֵי שַׁ֗דַּי \middot תָּ֗מִיד מְסַפְּרִים כְּבוֹד־אֵל וּקְדֻשָׁתוֹ׃ תִּתְבָּרַךְ יְיָ אֱלֹהֵֽינוּ עַל־שֶׁבַח מַעֲשֵׂי יָדֶֽיךָ \middot וְעַל־מְאֽוֹרֵי־אוֹר שֶׁעָשִֽׂיתָ יְפָאֲרֽוּךָ סֶּֽלָה׃
}

\newcommand{\kadoshbase}{\textbf{קָד֧וֹשׁ ׀ קָד֛וֹשׁ קָד֖וֹשׁ יְיָ֣ צְבָא֑וֹת מְלֹ֥א כׇל־הָאָ֖רֶץ כְּבוֹדֽוֹ׃}}
\newcommand{\barukhbase}{\textbf{בָּר֥וּךְ כְּבוֹד־יְיָ֖ מִמְּקוֹמֽוֹ׃}}
\newcommand{\yimlokhbase}{\textbf{יִמְלֹ֤ךְ יְיָ֨ ׀ לְעוֹלָ֗ם אֱלֹהַ֣יִךְ צִ֭יּוֹן לְדֹ֥ר וָ֝דֹ֗ר הַֽלְלוּיָֽהּ׃}}

\newcommand{\kadoshkadoshkadosh}{\kadoshbase\mdsource{ישעיה ו}} %\source{ישעיה ו}
\newcommand{\barukhhashem}{\barukhbase\mdsource{יחזקאל ג}}%\source{יחזקאל ג}
\newcommand{\yimloch}{\yimlokhbase\mdsource{תהלים קמו}}%\source{תהלים קמו}

\newcommand{\kadoshkadoshkadoshsource}{\kadoshbase\source{ישעיה ו}} %\source{ישעיה ו}
\newcommand{\barukhhashemsource}{\barukhbase\source{יחזקאל ג}}%\source{יחזקאל ג}
\newcommand{\yimlochsource}{\yimlokhbase\source{תהלים קמו}}%\source{תהלים קמו}

\newcommand{\hashemyimloch}{\textbf{יְיָ֥ ׀ יִמְלֹ֖ךְ לְעֹלָ֥ם וָעֶֽד׃} \source{שמות טו}}

\newcommand{\yotzerhameoros}{
	תִּתְבָּרַךְ צוּרֵֽנוּ מַלְכֵּֽנוּ וְגוֹאֲלֵֽנוּ בּוֹרֵא קְדוֹשִׁים \middot יִשְׁתַּבַּח שִׁמְךָ לָעַד מַלְכֵּֽנוּ יוֹצֵר מְשָׁרְתִים \middot וַאֲשֶׁר מְשָׁרְתָיו כֻּלָּם עוֹמְדִים בְּרוּם עוֹלָם \middot וּמַשְׁמִיעִים בְּיִרְאָה יַֽחַד בְּקוֹל \middot דִּבְרֵי אֱלֹהִים חַיִּים וּמֶֽלֶךְ עוֹלָם׃ כֻּלָּם אֲהוּבִים כֻּלָּם בְּרוּרִים כֻּלָּם גִּבּוֹרִים וְכֻלָּם עֹשִׂים בְּאֵימָה וּבְיִרְאָה רְצוֹן קוׂנָם \middot וְכֻלָּם פּוֹתְחִים אֶת־פִּיהֶם בִּקְדֻשָּׁה וּבְטׇהֳרָה בְּשִׁירָה וּבְזִמְרָה וּמְבָרְכִים וּמְשַׁבְּחִים וּמְפָאֲרִים וּמַעֲרִיצִים וּמַקְדִּישִׁים וּמַמְלִיכִים
	
	\kahal אֶת־שֵׁם הָאֵל הַמֶּֽלֶךְ הַגָּדוֹל הַגִּבּוֹר וְהַנּוֹרָא קָדוֹשׁ הוּא׃
	
	וְכֻלָּם מְקַבְּלִים עֲלֵיהֶם עֹל מַלְכוּת שָׁמַֽיִם זֶה מִזֶּה וְנוֹתְנִים רְשׁוּת זֶה לָזֶה \middot לְהַקְדִּישׁ לְיוֹצְרָם בְּנַֽחַת רֽוּחַ בְּשָׂפָה בְרוּרָה וּבִנְעִימָה \middot קְדֻשָּׁה כֻּלָּם כְּאֶחָד עוֹנִים וְאוֹמְרִים בְּיִרְאָה׃
	
	\kahal\kadoshkadoshkadoshsource
	
	וְהָאוֹפַנִּים וְחַיּוֹת הַקֹּֽדֶשׁ בְּרַֽעַשׁ גָּדוֹל מִתְנַשְּׂאִים לְעֻמַּת שְׂרָפִים לְעֻמָּתָם מְשַׁבְּחִים וְאוֹמְרִים׃
	
	\kahal\barukhhashemsource
	
	\firstword{לְאֵל}
	בָּרוּךְ נְעִימוֹת יִתֵּֽנוּ \middot לְמֶֽלֶךְ אֵל חַי וְקַיָּם זְמִירוֹת יֹאמֵֽרוּ וְתֻשְׁבָּחוֹת יַשְׁמִֽיעוּ \middot כִּי הוּא לְבַדּוֹ פּוֹעֵל גְּבוּרוֹת עֹשֶׂה חֲדָשׁוֹת בַּֽעַל מִלְחָמוֹת זוֹרֵֽעַ צְדָקוֹת מַצְמִֽיחַ יְשׁוּעוֹת בּוֹרֵא רְפוּאוֹת נוֹרָא תְהִלּוֹת אֲדוֹן הַנִּפְלָאוֹת \middot הַמְחַדֵּשׁ בְּטוּבוֹ בְּכׇל־יוֹם תָּמִיד מַעֲשֵׂה בְרֵאשִׁית׃ כָּאָמוּר׃ \source{תהלים קלו}לְ֭עֹשֵׂה אוֹרִ֣ים גְּדֹלִ֑ים כִּ֖י לְעוֹלָ֣ם חַסְדּֽוֹ׃ אוֹר חָדָשׁ עַל־צִיּוֹן תָּאִיר וְנִזְכֶּה כֻלָּֽנוּ מְהֵרָה לְאוֹרוֹ׃ בָּרוּךְ אַתָּה יְיָ יוֹצֵר הַמְּאוֹרוֹת׃
}

\newcommand{\ahavaraba}{
	\englishinst{At \hebineng{קנפות ארבע} gather the front two tzitzi\thav.}
	\firstword{אַהֲבָה רַבָּה}
	אֲהַבְתָּֽנוּ יְיָ אֱלֹהֵֽינוּ חֶמְלָה גְּדוֹלָה וִיתֵירָה חָמַֽלְתָּ עָלֵֽינוּ׃ אָבִֽינוּ מַלְכֵּֽנוּ בַּעֲבוּר אֲבוֹתֵֽינוּ שֶׁבָּטְחוּ בְךָ וַתְּלַמְּדֵם חֻקֵּי חַיִּים כֵּן תְּחׇנֵּֽנוּ וּתְלַמְּדֵֽנוּ׃ אָבִֽינוּ הָאָב הָרַחֲמָן הַמְרַחֵם רַחֵם עָלֵֽינוּ וְתֵן בְּלִבֵּֽנוּ לְהָבִין וּלְהַשְׂכִּיל לִשְׁמֹֽעַ לִלְמֹד וּלְלַמֵּד לִשְׁמֹר וְלַעֲשׂוֹת וּלְקַיֵּם אֶת־כׇּל־דִּבְרֵי תַלְמוּד תּוֹרָתֶֽךָ בְּאַהֲבָה׃ וְהָאֵר עֵינֵֽינוּ בְּתוֹרָתֶֽךָ וְדַבֵּק לִבֵּֽנוּ בְּמִצְוֹתֶֽיךָ וְיַחֵד לְבָבֵֽנוּ לְאַהֲבָה וּלְיִרְאָה אֶת־שְׁמֶֽךָ וְלֹא נֵבוֹשׁ לְעוֹלָם וָעֶד׃ כִּי בְשֵׁם קׇדְשְׁךָ הַגָּדוֹל וְהַנּוֹרָא בָּטָֽחְנוּ נָגִֽילָה וְנִשְׂמְחָה בִּישׁוּעָתֶֽךָ׃ וַהֲבִיאֵֽנוּ לְשָׁלוֹם מֵאַרְבַּע כַּנְפוֹת הָאָֽרֶץ וְתוֹלִיכֵֽנוּ קוֹמְמִיּוּת לְאַרְצֵֽנוּ׃ כִּי אֵל פּוֹעֵל יְשׁוּעוֹת אַֽתָּה וּבָֽנוּ בָחַֽרְתָּ מִכׇּל־עַם וְלָשׁוֹן וְקֵרַבְתָּֽנוּ לְשִׁמְךָ הַגָּדוֹל סֶֽלָה בֶּאֱמֶת \middot לְהוֹדוֹת לְךָ וּלְיַחֶדְךָ בְּאַהֲבָה׃ בָּרוּךְ אַתָּה יְיָ הַבּוֹחֵר בְּעַמּוֹ יִשְׂרָאֵל בְּאַהֲבָה׃
}

\newcommand{\shema}{
	
	\englishinst{Recite the line below when praying privately:}
	(\instruction{יחיד אומר׃}
	 אֵל מֶֽלֶךְ נֶאֱמָן
	 )
	\\
	\begin{Large}
		\textbf{
			שְׁמַ֖ע יִשְׂרָאֵ֑ל יְיָ֥ אֱלֹהֵ֖ינוּ יְיָ֥ ׀ אֶחָֽד׃} \source{דברים ו}\\
	\end{Large}
	\begin{large}
		\instruction{בלחש:} \textbf{בָּרוּךְ שֵׁם כְּבוֹד מַלְכוּתוֹ לְעוֹלָם וָעֶד:}
	\end{large}
}

\newcommand{\veahavta}{
	\firstword{וְאָ֣הַבְתָּ֔}\source{דברים ו}
	אֵ֖ת יְיָ֣ אֱלֹהֶ֑יךָ בְּכׇל־לְבָבְךָ֥ וּבְכׇל־נַפְשְׁךָ֖ וּבְכׇל־מְאֹדֶֽךָ׃
	וְהָי֞וּ הַדְּבָרִ֣ים הָאֵ֗לֶּה אֲשֶׁ֨ר אָנֹכִ֧י מְצַוְּךָ֛ הַיּ֖וֹם עַל־לְבָבֶֽךָ׃
	וְשִׁנַּנְתָּ֣ם לְבָנֶ֔יךָ וְדִבַּרְתָּ֖ בָּ֑ם בְּשִׁבְתְּךָ֤ בְּבֵיתֶ֙ךָ֙ וּבְלֶכְתְּךָ֣ בַדֶּ֔רֶךְ וּֽבְשׇׁכְבְּךָ֖ וּבְקוּמֶֽךָ׃
	וּקְשַׁרְתָּ֥ם לְא֖וֹת עַל־יָדֶ֑ךָ וְהָי֥וּ לְטֹטָפֹ֖ת בֵּ֥ין עֵינֶֽיךָ׃
	וּכְתַבְתָּ֛ם עַל־מְזֻז֥וֹת בֵּיתֶ֖ךָ וּבִשְׁעָרֶֽיךָ׃
}

\newcommand{\vehaya}{
	\firstword{וְהָיָ֗ה}\source{דברים יא}
	אִם־שָׁמֹ֤עַ תִּשְׁמְעוּ֙ אֶל־מִצְוֺתַ֔י אֲשֶׁ֧ר אָנֹכִ֛י מְצַוֶּ֥ה אֶתְכֶ֖ם הַיּ֑וֹם לְאַהֲבָ֞ה אֶת־יְיָ֤ אֱלֹֽהֵיכֶם֙ וּלְעׇבְד֔וֹ בְּכׇל־לְבַבְכֶ֖ם וּבְכׇל־נַפְשְׁכֶֽם׃
	וְנָתַתִּ֧י מְטַֽר־אַרְצְכֶ֛ם בְּעִתּ֖וֹ יוֹרֶ֣ה וּמַלְק֑וֹשׁ וְאָסַפְתָּ֣ דְגָנֶ֔ךָ וְתִֽירֹשְׁךָ֖ וְיִצְהָרֶֽךָ׃
	וְנָתַתִּ֛י עֵ֥שֶׂב בְּשָׂדְךָ֖ לִבְהֶמְתֶּ֑ךָ וְאָכַלְתָּ֖ וְשָׂבָֽעְתָּ׃
	הִשָּֽׁמְר֣וּ לָכֶ֔ם פֶּ֥ן יִפְתֶּ֖ה לְבַבְכֶ֑ם וְסַרְתֶּ֗ם וַעֲבַדְתֶּם֙ אֱלֹהִ֣ים אֲחֵרִ֔ים וְהִשְׁתַּחֲוִיתֶ֖ם לָהֶֽם׃
	וְחָרָ֨ה אַף־יְיָ֜ בָּכֶ֗ם וְעָצַ֤ר אֶת־הַשָּׁמַ֙יִם֙ וְלֹֽא־יִהְיֶ֣ה מָטָ֔ר וְהָ֣אֲדָמָ֔ה לֹ֥א תִתֵּ֖ן אֶת־יְבוּלָ֑הּ וַאֲבַדְתֶּ֣ם מְהֵרָ֗ה מֵעַל֙ הָאָ֣רֶץ הַטֹּבָ֔ה אֲשֶׁ֥ר יְיָ֖ נֹתֵ֥ן לָכֶֽם׃
	וְשַׂמְתֶּם֙ אֶת־דְּבָרַ֣י אֵ֔לֶּה עַל־לְבַבְכֶ֖ם וְעַֽל־נַפְשְׁכֶ֑ם וּקְשַׁרְתֶּ֨ם אֹתָ֤ם לְאוֹת֙ עַל־יֶדְכֶ֔ם וְהָי֥וּ לְטוֹטָפֹ֖ת בֵּ֥ין עֵינֵיכֶֽם׃
	וְלִמַּדְתֶּ֥ם אֹתָ֛ם אֶת־בְּנֵיכֶ֖ם לְדַבֵּ֣ר בָּ֑ם בְּשִׁבְתְּךָ֤ בְּבֵיתֶ֙ךָ֙ וּבְלֶכְתְּךָ֣ בַדֶּ֔רֶךְ וּֽבְשׇׁכְבְּךָ֖ וּבְקוּמֶֽךָ׃
	וּכְתַבְתָּ֛ם עַל־מְזוּז֥וֹת בֵּיתֶ֖ךָ וּבִשְׁעָרֶֽיךָ׃
	לְמַ֨עַן יִרְבּ֤וּ יְמֵיכֶם֙ וִימֵ֣י בְנֵיכֶ֔ם עַ֚ל הָֽאֲדָמָ֔ה אֲשֶׁ֨ר נִשְׁבַּ֧ע יְיָ֛ לַאֲבֹתֵיכֶ֖ם לָתֵ֣ת לָהֶ֑ם כִּימֵ֥י הַשָּׁמַ֖יִם עַל־הָאָֽרֶץ׃
}

\newcommand{\vayomer}{
	\firstword{וַיֹּ֥אמֶר}\source{במדבר טו}
	יְיָ֖ אֶל־מֹשֶׁ֥ה לֵּאמֹֽר׃
	דַּבֵּ֞ר אֶל־בְּנֵ֤י יִשְׂרָאֵל֙ וְאָמַרְתָּ֣ אֲלֵהֶ֔ם וְעָשׂ֨וּ לָהֶ֥ם צִיצִ֛ת עַל־כַּנְפֵ֥י בִגְדֵיהֶ֖ם לְדֹרֹתָ֑ם וְנָ֥תְנ֛וּ עַל־צִיצִ֥ת הַכָּנָ֖ף פְּתִ֥יל תְּכֵֽלֶת׃
	וְהָיָ֣ה לָכֶם֮ לְצִיצִת֒ וּרְאִיתֶ֣ם אֹת֗וֹ וּזְכַרְתֶּם֙ אֶת־כׇּל־מִצְוֺ֣ת יְיָ֔ וַעֲשִׂיתֶ֖ם אֹתָ֑ם וְלֹֽא־תָת֜וּרוּ אַחֲרֵ֤י לְבַבְכֶם֙ וְאַחֲרֵ֣י עֵֽינֵיכֶ֔ם אֲשֶׁר־אַתֶּ֥ם זֹנִ֖ים אַחֲרֵיהֶֽם׃
	לְמַ֣עַן תִּזְכְּר֔וּ וַעֲשִׂיתֶ֖ם אֶת־כׇּל־מִצְוֺתָ֑י וִהְיִיתֶ֥ם קְדֹשִׁ֖ים לֵאלֹֽהֵיכֶֽם׃
	אֲנִ֞י יְיָ֣ אֱלֹֽהֵיכֶ֗ם אֲשֶׁ֨ר הוֹצֵ֤אתִי אֶתְכֶם֙ מֵאֶ֣רֶץ מִצְרַ֔יִם לִהְי֥וֹת לָכֶ֖ם לֵאלֹהִ֑ים אֲנִ֖י יְיָ֥ אֱלֹהֵיכֶֽם׃
}

\newcommand{\emesveyatziv}{
	\firstword{אֱמֶת}
	וְיַצִּיב וְנָכוֹן וְקַיָּם וְיָשָׁר וְנֶאֱמָן וְאָהוּב וְחָבִיב וְנֶחְמָד וְנָעִים וְנוֹרָא וְאַדִּיר וּמְתֻקָּן וּמְקֻבָּל וְטוֹב וְיָפֶה הַדָּבָר הַזֶּה עָלֵֽינוּ לְעוֹלָם וָעֶד׃ אֱמֶת אֱלֹהֵי עוֹלָם מַלְכֵּֽנוּ צוּר יַעֲקֹב מָגֵן יִשְׁעֵֽנוּ׃ לְדוֹר וָדוֹר הוּא קַיָּם וּשְׁמוֹ קַיָּם וְכִסְאוֹ נָכוֹן וּמַלְכוּתוֹ וֶאֱמוּנָתוֹ לָעַד קַיָּֽמֶת׃ וּדְבָרָיו חָיִים וְקַיָּמִים נֶאֱמָנִים וְנֶחֱמָדִים לָעַד וּלְעוֹלְמֵי עוֹלָמִים׃ עַל־אֲבוֹתֵֽינוּ וְעָלֵֽינוּ עַל־בָּנֵֽינוּ וְעַל־דּוֹרוֹתֵֽינוּ וְעַל כׇּל־דּוֹרוֹת זֶֽרַע יִשְׂרָאֵל עֲבָדֶֽיךָ׃
	
	\firstword{עַל־הָרִאשׁוֹנִים}
	וְעַל־הָאַחֲרוֹנִים דָּבָר טוֹב וְקַיָּם לְעוֹלָם וָעֶד \middot אֱמֶת וֶאֱמוּנָה חֹק וְלֹא יַעֲבוֹר׃ אֱמֶת שֶׁאַתָּה הוּא יְיָ אֱלֹהֵֽינוּ וֵאלֹהֵי אֲבוֹתֵֽינוּ מַלְכֵּֽנוּ מֶֽלֶךְ אֲבוֹתֵֽינוּ גּוֹאֲלֵֽנוּ גּוֹאֵל אֲבוֹתֵֽינוּ \middot יוֹצְרֵֽנוּ צוּר יְשׁוּעָתֵֽנוּ פּוֹדֵֽנוּ וּמַצִּילֵֽנוּ מֵעוֹלָם שְׁמֶֽךָ \middot אֵין אֱלֹהִים זוּלָתֶֽךָ׃
}


\newcommand{\ezrasavoseinu}{
	\firstword{עֶזְרַת}
	אֲבוֹתֵֽינוּ אַתָּה הוּא מֵעוֹלָם מָגֵן וּמוֹשִֽׁיעַ לִבְנֵיהֶם אַחֲרֵיהֶם בְּכׇל־דּוֹר וָדוֹר׃ בְּרוּם עוֹלָם מוֹשָׁבֶֽךָ וּמִשְׁפָּטֶֽיךָ וְצִדְקָתְךָ עַד־אַפְסֵי אָֽרֶץ׃ אַשְׁרֵי אִישׁ שֶׁיִּשְׁמַע לְמִצְוֹתֶֽיךָ וְתוֹרָתְךָ וּדְבָרְךָ יָשִׂים עַל־לִבּוֹ׃ אֱמֶת אַתָּה הוּא אָדוֹן לְעַמֶּֽךָ וּמֶֽלֶךְ גִּבּוֹר לָרִיב רִיבָם׃ אֱמֶת אַתָּה הוּא רִאשׁוֹן וְאַתָּה הוּא אַחֲרוֹן וּמִבַּלְעָדֶֽיךָ אֵין לָֽנוּ מֶֽלֶךְ גּוֹאֵל וּמוֹשִֽׁיעַ׃ מִמִּצְרַֽיִם גְּאַלְתָּֽנוּ יְיָ אֱלֹהֵֽינוּ וּמִבֵּית עֲבָדִים פְּדִיתָֽנוּ \middot כׇּל־בְּכוֹרֵיהֶם הָרַֽגְתָּ וּבְכוֹרְךָ גָּאַֽלְתָּ וְיַם־סוּף בָּקַֽעְתָּ וְזֵדִים טִבַּֽעְתָּ וִידִידִים הֶעֱבַֽרְתָּ
	\source{תהלים קו}וַיְכַסּוּ־מַ֥יִם צָרֵיהֶ֑ם אֶחָ֥ד מֵ֝הֶ֗ם לֹ֣א נוֹתָֽר׃
	עַל־זֹאת שִׁבְּחוּ אֲהוּבִים וְרוֹמְמוּ אֵל וְנָתְנוּ יְדִידִים זְמִירוֹת שִׁירוֹת וְתֻשְׁבָּחוֹת בְּרָכוֹת וְהוֹדָאוֹת לְמֶלֶךְ אֵל חַי וְקַיָּם. רָם וְנִשָּׂא גָּדוֹל וְנוֹרָא מַשְׁפִּיל גֵּאִים וּמַגְבִּֽיהַּ שְׁפָלִים מוֹצִיא אֲסִירִים וּפוֹדֶה עֲנָוִים וְעוֹזֵר דַּלִּים וְעוֹנֶה לְעַמּוֹ בְּעֵת שַׁוְּעָם אֵלָיו׃
	\englishinst{Stand here and take three steps back.}
	 תְּהִלּוֹת לְאֵל עֶלְיוֹן בָּרוּךְ הוּא וּמְבֹרָךְ \middot מֹשֶׁה וּבְנֵי יִשְׂרָאֵל לְךָ עָנוּ שִׁירָה בְּשִׂמְחָה רַבָּה וְאָמְרוּ כֻלָּם׃
}

\newcommand{\gaalyisroel}{
	\kahal \source{שמות טו}\textbf{מִֽי־כָמֹ֤כָה בָּֽאֵלִם֙ יְיָ֔ מִ֥י כָּמֹ֖כָה נֶאְדָּ֣ר בַּקֹּ֑דֶשׁ נוֹרָ֥א תְהִלֹּ֖ת עֹ֥שֵׂה פֶֽלֶא׃}
	
	שִׁירָה חֲדָשָׁה שִׁבְּחוּ גְאוּלִים לְשִׁמְךָ עַל־שְׂפַת הַיָּם יַֽחַד כֻּלָּם הוֹדוּ וְהִמְלִֽיכוּ וְאָמְרוּ׃
	
	\kahal \hashemyimloch\\
	\englishinst{Recite the blessing of \hebineng{ישראל גאל} together with the reader.}
	צוּר יִשְׂרָאֵל קֽוּמָה בְּעֶזְרַת יִשְׂרָאֵל וּפְדֵה כִנְאֻמֶֽךָ יְהוּדָה וְיִשְׂרָאֵל׃ גֹּאֲלֵ֕נוּ\source{ישעיהו מז} יְיָ֥ צְבָא֖וֹת שְׁמ֑וֹ קְד֖וֹשׁ יִשְׂרָאֵֽל׃ בָּרוּךְ אַתָּה יְיָ גָּאַל יִשְׂרָאֵל׃
}

%\newcommand{\ayt}{ בעשי״ת׃}
\newcommand{\ayt}{ בעשי״ת}

\newcommand{\amidaopening}[2]{
	\englishinst{Take three steps forward while reciting the following verse.  Bend the knees and bow for the words \hebineng{ברוך} and \hebineng{אתה} respectively at the beginning and end of the first blessing.}
	אֲ֭דֹנָי\source{תהלים נא} שְׂפָתַ֣י תִּפְתָּ֑ח וּ֝פִ֗י יַגִּ֥יד תְּהִלָּתֶֽךָ׃\\
	\firstword{בָּרוּךְ}
	אַתָּה יְיָ אֱלֹהֵֽינוּ וֵאלֹהֵי אֲבוֹתֵֽינוּ \middot אֱלֹהֵי אַבְרָהָם אֱלֹהֵי יִצְחָק וֵאלֹהֵי יַעֲקֹב \middot הָאֵל הַגָּדוֹל הַגִּבּוֹר וְהַנּוֹרָא אֵל עֶלְיוֹן גּוֹמֵל חֲסָדִים טוֹבִים וְקוֹנֵה הַכֹּל \middot
	וְזוֹכֵר חַסְדֵי אָבוֹת וּמֵבִיא גוֹאֵל לִבְנֵי בְנֵיהֶם לְמַֽעַן שְׁמוֹ בְּאַהֲבָה׃
	\vspace{-12pt}
\begin{footnotesize}\begin{Center}
	(\instruction{#1} 
זׇכְרֵֽנוּ לְחַיִּים מֶֽלֶךְ חָפֵץ בַּחַיִּים \middot\\
וְכׇתְבֵֽנוּ בְּסֵפֶר הַחַיִּים לְמַעַנְךָ אֱלֹהִים חַיִּים׃)
\end{Center}\end{footnotesize}
\vspace{-24pt}
\begin{Center}
\firstword{מֶֽלֶךְ}
עוֹזֵר וּמוֹשִֽׁיעַ וּמָגֵן׃ בָּרוּךְ אַתָּה יְיָ מָגֵן אַבְרָהָם׃
	
\firstword{אַתָּה}
גִּבּוֹר לְעוֹלָם אֲדֹנָי מְחַיֵּה מֵתִים אַתָּה רַב לְהוֹשִֽׁיעַ \middot
\end{Center}
\vspace{-16pt}
\begin{Center}
	\englishinst{From Shemini Atzeret till Pesach:}
	%\instruction{ממוסף של שמיני עצרת עד מוסף יום א׳ של פסח אומרים׃}\\
מַשִּׁיב הָרוּחַ וּמורִיד הַגָּֽשֶׁם \middot
\end{Center}
\begin{justify}
\firstword{מְכַלְכֵּל}
חַיִּים בְּחֶֽסֶד מְחַיֵּה מֵתִים בְּרַחֲמִים רַבִּים סוֹמֵךְ נוֹפְלִים וְרוֹפֵא חוֹלִים וּמַתִּיר אֲסוּרִים וּמְקַיֵּם אֱמוּנָתוֹ לִישֵׁנֵי עָפָר \middot מִי כָמֽוֹךָ בַּֽעַל גְּבוּרוֹת וּמִי דּֽוֹמֶה לָּךְ מֶֽלֶךְ מֵמִית וּמְחַיֶּה וּמַצְמִֽיחַ יְשׁוּעָה׃\end{justify}
	\begin{small}(\instruction{#1}
		מִי כָמֽוֹךָ אַב הָרַחֲמִים זוֹכֵר יְצוּרָיו לְחַיִּים בְּרַחֲמִים׃)\end{small}\\
\firstword{וְנֶאֱמָן}
	אַתָּה לְהַחֲיוֹת מֵתִים׃ בָּרוּךְ אַתָּה יְיָ מְחַיֵּה הַמֵּתִים׃
\begin{Center} 
#2
\firstword{אַתָּה}
	קָדוֹשׁ וְשִׁמְךָ קָדוֹשׁ וּקְדוֹשִׁים בְּכׇל־יוֹם יְהַלְלוּךָ סֶּֽלָה׃ בָּרוּךְ אַתָּה יְיָ *הָאֵל
(*\instruction{#1}
	הַמֶּֽלֶךְ)
	הַקָּדוֹשׁ׃\end{Center}
}

\newcommand{\weekdaysakedusha}{
\begin{Center}\ssubsection{\adforn{48} קדושה \adforn{22}}\end{Center}
	
	\begin{small}
		%\setlength{\LTpost}{0pt}
		\begin{longtable}{l p{.90\textwidth}}
			
			\shatz &
			נְקַדֵּשׁ אֶת־שִׁמְךָ בָּעוֹלָם כְּשֵׁם שֶׁמַּקְדִּישִׁים אוֹתוֹ בִּשְׁמֵי מָרוֹם כַּכָּתוּב עַל יַד נְבִיאֶךָ קָרָ֨א זֶ֤ה אֶל־זֶה֙ וְאָמַ֔ר׃\\
			
			\shatzvkahal &
			\kadoshkadoshkadosh\\
			
			\shatz &
			לְעֻמָּתָם בָּרוּךְ יֹאמֵרוּ׃\\
			
			\shatzvkahal &
			\barukhhashem\\
			
			\shatz &
			וּבְדִבְרֵי קׇדְשְׁךָ כָּתוּב לֵאמֹר׃ \\
			
			\shatzvkahal &
			\yimloch\\
			
			\shatz &
			לְדוֹר וָדוֹר נַגִּיד גׇּדְלֶךָ וּלְנֵצַח נְצָחִים קְדֻשָּׁתְךָ נַקְדִּישׁ \middot וְשִׁבְחֲךָ אֱלֹהֵֽינוּ מִפִּינוּ לֹא יָמוּשׁ לְעוֹלָם וָעֶד \middot כִּי אֵל מֶלֶךְ גָּדוֹל וְקָדוֹשׁ אַֽתָּה: בָּרוּךְ אַתָּה יְיָ *הָאֵל (*\instruction{בעשי״ת:} הַמֶּֽלֶךְ) הַקָּדוֹשׁ:
		\end{longtable}	\end{small}\vspace{-\baselineskip}
}


\newcommand{\weekdaysakiddushhashem}{
	\firstword{אַתָּה}
	קָדוֹשׁ וְשִׁמְךָ קָדוֹשׁ וּקְדוֹשִׁים בְּכׇל־יוֹם יְהַלְלוּךָ סֶּֽלָה׃ בָּרוּךְ אַתָּה יְיָ *הָאֵל
	(*\instruction{בעשי״ת:}
	הַמֶּֽלֶךְ)
	הַקָּדוֹשׁ׃
}


\newcommand{\weekdaysabinah}{
	\firstword{אַתָּה חוֹנֵן}
	לְאָדָם דַּֽעַת וּמְלַמֵּד לֶאֱנוֹשׁ בִּינָה \middot חׇנֵּֽנוּ מֵאִתְּךָ בִּינָה דֵּעָה וְהַשְׂכֵּל׃ בָּרוּךְ אַתָּה יְיָ חוֹנֵן הַדָּֽעַת׃
}

\newcommand{\weekdaysateshuva}{
	\firstword{הֲשִׁיבֵֽנוּ}
	אָבִֽינוּ לְתוֹרָתֶֽךָ וְקָרְבֵֽנוּ מַלְכֵּֽנוּ לַעֲבוֹדָתֶֽךָ וְהַחֲזִירֵֽנוּ בִתְשׁוּבָה שְׁלֵמָה לְפָנֶֽיךָ׃ בָּרוּךְ אַתָּה יְיָ הָרוֹצֶה בִּתְשׁוּבָה׃
}

\newcommand{\weekdaysaselichah}{
	\englishinst{Strike the chest with the right fist at the words \hebineng{חטאנו} and \hebineng{פשענו}.}
	\firstword{סְלַח}
	לָֽנוּ אָבִֽינוּ כִּי חָטָֽאנוּ מְחַל לָֽנוּ מַלְכֵּֽנוּ כִּי פָשָֽׁעְנוּ \middot כִּי מוֹחֵל וְסוֹלֵֽחַ אַֽתָּה׃ בָּרוּךְ אַתָּה יְיָ חַנּוּן הַמַּרְבֶּה לִסְלֽוֹחַ׃
}

\newcommand{\weekdaysageulah}{
	\firstword{רְאֵה}
	בְעׇנְיֵֽנוּ וְרִיבָה רִיבֵֽנוּ וּגְאָלֵֽנוּ מְהֵרָה לְמַֽעַן שְׁמֶֽךָ כִּי גוֹאֵל חָזָק אַֽתָּה׃ בָּרוּךְ אַתָּה יְיָ גּוֹאֵל יִשְׂרָאֵל׃
}

\newcommand{\aneinubasetext}{
	עֲנֵֽנוּ יְיָ עֲנֵֽנוּ בְּיוֹם צוֹם תַּעֲנִיתֵֽנוּ כִּי בְצָרָה גְדוֹלָה אֲנָֽחְנוּ \middot אַל תֵּֽפֶן אֶל רִשְׁעֵֽנוּ וְאַל תַּסְתֵּר פָּנֶֽיךָ מִמֶּֽנּוּ וְאַל תִּתְעַלַּם מִתְּחִנָּתֵֽנוּ׃ הֱיֵה־נָא קָרוֹב לְשַׁוְעָתֵֽנוּ \middot יְהִי־נָא חַסְדְּךָ לְנַחֲמֵֽנוּ טֶֽרֶם נִקְרָא אֵלֶֽיךָ עֲנֵֽנוּ כַּדָּבָר שֶׁנֶּאֱמַר׃ \mdsource{ישעיה סה}וְהָיָ֥ה טֶֽרֶם־יִקְרָ֖אוּ וַֽאֲנִ֣י אֶעֱנֶ֑ה ע֛וֹד הֵ֥ם מְדַבְּרִ֖ים וַֽאֲנִ֥י אֶשְׁמָֽע׃ כִּי אַתָּה יְיָ הָעוֹנֶה בְּעֵת צָרָה פּוֹדֶה וּמַצִּיל בְּכׇל־עֵת צָרָה וְצוּקָה׃}

\newcommand{\weekdaysaanneinu}{
	\begin{sometimes}
		
		\begin{small}
			\englishinst{On a public fast during the repetition of the Amidah, the leader includes the following:}
			%\instruction{בת״צ הש״ץ אומר עננו:}
			\aneinubasetext
			 בָּרוּךְ אַתָּה יְיָ הָעוֹנֶה בְּעֵת צָרָה׃
			
		\end{small}
	\end{sometimes}
}


\newcommand{\weekdaysarefuah}{
	\firstword{רְפָאֵֽנוּ}
	יְיָ וְנֵרָפֵא הוֹשִׁיעֵֽנוּ וְנִוָּשֵֽׁעָה כִּי תְהִלָּתֵֽנוּ אַֽתָּה \middot וְהַעֲלֵה רְפוּאָה שְׁלֵמָה לְכׇל־מַכּוֹתֵֽינוּ
	\footnote{
		\englishinst{In private prayer, individuals can add a prayer for a specific sick person:}
		יהִי רָצוֹן מִלְּפָנֶֽיךָ יְיָ אֱלֹהֵֽינוּ שֶׁתִּשְׁלַח מְהֵרָה רְפוּאָה שְׁלֵמָה מִן הַשָּׁמַֽיִם רְפוּאַת הַנֶּֽפֶשׁ וּרְפוּאַת הַגּוּף לְחוֹלֶה/לְחוֹלָה/לְחוֹלִים (\instruction{פב״פ}) בְּתוֹךְ שְׁאָר חוֹלֵי יִשְׂרָאֵל:\\
	}
	כִּי אֵל מֶֽלֶךְ רוֹפֵא נֶאֱמָן וְרַחֲמָן אַֽתָּה׃ בָּרוּךְ אַתָּה יְיָ רוֹפֵא חוֹלֵי עַמּוֹ יִשְׂרָאֵל׃
}

\newcommand{\weekdaysaberacha}{
	\firstword{בָּרֵךְ}
	עָלֵֽינוּ יְיָ אֱלֹהֵֽינוּ אֶת־הַשָּׁנָה הַזֹּאת וְאֶת־כׇּל־מִינֵי תְבוּאָתָהּ לְטוֹבָה \middot וְתֵן (\instruction{בקיץ:}
	\textbf{בְּרָכָה})
	(\instruction{בחורף:}
	\textbf{טַל וּמָטָר לִבְרָכָה})
	עַל פְּנֵי הָאֲדָמָה וְשַׂבְּעֵֽנוּ מִטּוּבָהּ וּבָרֵךְ שְׁנוֹתֵֽנוּ כַּשָּׁנִים הַטּוֹבוֹת: בָּרוּךְ אַתָּה יְיָ מְבָרֵךְ הַשָּׁנִים:
}

%\footnote{\instruction{נ״א}: מִטּוּבֶֽךָ}
\newcommand{\weekdaysashofar}{
	\firstword{תְּקַע}
	בְּשׁוֹפָר גָּדוֹל לְחֵרוּתֵֽנוּ וְשָׂא נֵס לְקַבֵּץ גָּלֻיּוֹתֵֽינוּ \middot וְקַבְּצֵֽנוּ יַֽחַד מֵאַרְבַּע כַּנְפוֹת הָאָֽרֶץ׃ בָּרוּךְ אַתָּה יְיָ מְקַבֵּץ נִדְחֵי עַמּוֹ יִשְׂרָאֵל׃
}

\newcommand{\weekdaysamishpat}{
	\firstword{הָשִֽׁיבָה}
	שׁוֹפְטֵֽינוּ כְּבָרִאשׁוֹנָה וְיוֹעֲצֵֽינוּ כְּבַתְּחִלָּה וְהָסֵר מִמֶּֽנּוּ יָגוֹן וַאֲנָחָה \middot וּמְלוֹךְ עָלֵֽינוּ אַתָּה יְיָ לְבַדְּךָ בְּחֶֽסֶד וּבְרַחֲמִים וְצַדְּקֵֽנוּ בַּמִּשְׁפָּט׃ בָּרוּךְ אַתָּה יְיָ *מֶֽלֶךְ אוֹהֵב צְדָקָה וּמִשְׁפָּט
	(*\instruction{בעשי״ת:}
	הַמֶּֽלֶךְ הַמִּשְׁפָּט)׃
}


%וְלַמְשֻׁמָּדִים אַל תְּהִי תִקְוָה וְכׇל־הַמִּינִים כְּרֶֽגַע יֹאבֵֽדוּ וְכׇל־אוֹיְבֵי עַמְּךָ מְהֵרָה יִכָּרֵֽתוּ \middot וּמַלְכוּת זָדוֹן מְהֵרָה תְעַקֵּר וּתְשַׁבֵּר וּתְמַגֵּר וְתַכְנִֽיעַ כׇּל־אוֹיְבֵינוּ בִּמְהֵרָה בְיָמֵֽינוּ׃


\newcommand{\weekdaysaminim}{
	\firstword{וְלַמַּלְשִׁינִים}
	אַל תְּהִי תִקְוָה וְכׇל־הָרִשְׁעָה כְּרֶגַע תֺּאבֵד וְכׇל־אוֹיְבֶֽיךָ מְהֵרָה יִכָּרֵתוּ \middot וְהַזֵדִים מְהֵרָה תְעַקֵּר וּתְשַׁבֵּר וּתְמַגֵּר וְתַכְנִיעַ בִּמְהֵרָה בְיָמֵינוּ׃ בָּרוּךְ אַתָּה יְיָ שׁוֹבֵר אוֹיְבִים וּמַכְנִיעַ זֵדִים׃
}



\newcommand{\weekdaysatzadikim}{
	\firstword{עַל הַצַּדִּיקִים}
	וְעַל הַחֲסִידִים וְעַל זִקְנֵי עַמְּךָ בֵּית יִשְׂרָאֵל וְעַל פְּלֵיטַת סוֹפְרֵיהֶם וְעַל גֵּרֵי הַצֶּֽדֶק וְעָלֵֽינוּ \middot יֶהֱמוּ־נָא רַחֲמֶֽיךָ יְיָ אֱלֹהֵֽינוּ \middot וְתֵן שָׂכָר טוֹב לְכֹל הַבּוֹטְחִים בְּשִׁמְךָ בֶּאֱמֶת וְשִׂים חֶלְקֵֽנוּ עִמָּהֶם \middot וּלְעוֹלָם לֹא נֵבוֹשׁ כִּי בְךָ בָּטָֽחְנוּ׃ בָּרוּךְ אַתָּה יְיָ מִשְׁעָן וּמִבְטָח לַצַּדִּיקִים׃
}

\newcommand{\weekdaysayerushelayim}{
	\firstword{וְלִיְרוּשָׁלַ‍ִם}
	עִירְךָ בְּרַחֲמִים תָּשׁוּב וְתִשְׁכּוֹן בְּתוֹכָהּ כַּאֲשֶׁר דִּבַּֽרְתָּ \middot וּבְנֵה אוֹתָהּ בְּקָרוֹב בְּיָמֵֽינוּ בִּנְיַן עוֹלָם וְכִסֵּא דָוִד מְהֵרָה לְתוֹכָהּ תָּכִין׃
	בָּרוּךְ אַתָּה יְיָ בּוֹנֵה יְרוּשָׁלַ‍ִם׃
}

\newcommand{\yerushwithnachem}{\firstword{וְלִירוּשָׁלַ‍ִם}
	עִירְךָ בְּרַחֲמִים תָּשׁוּב וְתִשְׁכּוֹן בְּתוֹכָהּ כַּאֲשֶׁר דִּבַּֽרְתָּ \middot וּבְנֵה אוֹתָהּ בְּקָרוֹב בְּיָמֵֽינוּ בִּנְיַן עוֹלָם וְכִסֵּא דָוִד מְהֵרָה לְתוֹכָהּ תָּכִין׃
	\footnote{
		\instruction{בתשעה באב במנחה׃ }
		\textbf{נַחֵם}
		יְיָ אֱלֹהֵֽינוּ אֶת־אֲבֵלֵי צִיּוֺן וְאֶת־אֲבֵלֵי יְרוּשָׁלַֽ֔֗͏ִם וְאֶת־הָעִיר הָאֲבֵלָה וְהֶחֳרֵבָה וְהַבְּזוּיָה וְהַשׁוֺמֵמָה \middot הָאֲבֵלָה מִבְּלִי בָנֱיהָ וְהֶחֳרֵבָה מִמְּעוֺנוֺתֶֽיהָ וְהַבְּזוּיָה מִכְּבוֺדָהּ וְהַשׁוֺמֵמָה מֵאֵין יוֺשֵׁב \middot וְהִיא יוֺשֶֽׁבֶת וְרֹאשָׁה חָפוּי בְּאִשָׁה עֲקַרָה שֶׁלֹּא יָלֳדָה \middot וַיְבַלְְּעֽוּהָ לִגְיוֺנוֺת וַיְּירָשׁוּהָ עוֺבְדֵי פְסִילִים \middot וַיָטִֽילוּ אֶת־עַמְּךָ יִשְׂרָאֵל לֶחָרֱֽב וַיַּהַרְגוּ בְזָדוֺן חֲסִידֵי עֶלְיוֺן \middot עַל־כֵּן צִיּוֺן בְּמַר תִּבְכֶּה וִירוּשָׁלַֽ͏ִם תִּתֵּן קוֺלָהּ \middot לִבִּי לִבִּי עַל חַלְלֵיהֶם מֵעַי מֵעַי עַל חַלְלֵיהֶם \middot כִּי אַתָּה יְיָ בָּאֵשׁ הִצַּתָּהּ וּבָאֵשׁ אַתָּה עָתִיד לִבְנוֺתָה, כָּאָמוּר׃
		\mdsource{זכריה ב}%
		וַֽאֲנִ֤י אֶֽהְיֶה־לָּהּ֙ נְאֻם־יְיָ֔ ח֥וֹמַת אֵ֖שׁ סָבִ֑יב וּלְכָב֖וֹד אֶֽהְיֶ֥ה בְתוֹכָֽהּ׃
		בָּרוּךְ אַתָּה יְיָ מְנַחֵם צִיוֺן וּבֹנֵה יְרוּשָׁלַֽ͏ִם׃ \instruction{את צמח וכו׳}\\
	}
	בָּרוּךְ אַתָּה יְיָ בּוֹנֵה יְרוּשָׁלַ‍ִם׃}

\newcommand{\weekdaysamalchus}{
	\firstword{אֶת צֶֽמַח}
	דָּוִד עַבְדְּךָ מְהֵרָה תַצְמִֽיחַ וְקַרְנוֹ תָּרוּם בִּישׁוּעָתֶֽךָ \middot כִּי לִישׁוּעָתְךָ קִוִּֽינוּ כׇּל־הַיּוֹם׃ בָּרוּךְ אַתָּה יְיָ מַצְמִֽיחַ קֶֽרֶן יְשׁוּעָה׃
}

\newcommand{\weekdaysashemakoleinu}[1]{
	\firstword{שְׁמַע קוֹלֵֽנוּ}
	יְיָ אֱלֹהֵֽינוּ חוּס וְרַחֵם עָלֵֽינוּ וְקַבֵּל בְּרַחֲמִים וּבְרָצוֹן אֶת־תְּפִלָּתֵֽנוּ \middot כִּי אֵל שׁוֹמֵעַ תְּפִלּוֹת וְתַחֲנוּנִים אַֽתָּה וּמִלְּפָנֶֽיךָ מַלְכֵּֽנוּ רֵיקָם אַל תְּשִׁיבֵֽנוּ #1
	כִּי אַתָּה שׁוֹמֵֽעַ תְּפִלַּת עַמְּךָ יִשְׂרָאֵל בְּרַחֲמִים׃ בָּרוּךְ אַתָּה יְיָ שׁוֹמֵֽעַ תְּפִלָּה׃
}

\newcommand{\retzeh}{
	\firstword{רְצֵה}
	יְיָ אֱלֹהֵֽינוּ בְּעַמְּךָ יִשְׂרָאֵל וּבִתְפִלָּתָם וְהָשֵׁב הָעֲבוֹדָה לִדְבִיר בֵּיתֶֽךָ \middot וְאִשֵּׁי יִשְׂרָאֵל וּתְפִלָּתָם בְּאַהֲבָה תְקַבֵּל בְּרָצוֹן וּתְהִי לְרָצוֹן תָּמִיד עֲבוֹדַת יִשְׂרָאֵל עַמֶּֽךָ \middot
}

\newcommand{\yaalehveyavo}{
	
	\begin{sometimes}
		
		\instruction{בראש חודש וחול המועד:}\\
		\yaalehveyavotemplate{\\\begin{tabular}{c|c|c}
				רֹאשׁ הַחֹֽדֶשׁ & חַג הַמַּצוֹת & חַג הַסֻּכּוֹת
		\end{tabular}\\}
	
	\end{sometimes}
	
}

\newcommand{\yaalehveyavotemplate}[1]{
\firstword{אֱלֹהֵֽינוּ}
וֵאלֹהֵי אֲבוֹתֵֽינוּ \middot יַעֲלֶה וְיָבֹא וְיַגִּיעַ וְיֵרָאֶה וְיֵרָצֶה וְיִשָּׁמַע וְיִפָּקֵד וְיִזָּכֵר זִכְרוֹנֵֽנוּ וּפִקְדּוֹנֵֽנוּ וְזִכְרוֹן אֲבוֹתֵֽינוּ \middot וְזִכְרוֹן מָשִׁיחַ בֶּן דָּוִד עַבְדֶּֽךָ וְזִכְרוֹן יְרוּשָׁלַ‍ִם עִיר קׇדְשֶֽׁךָ וְזִכְרוֹן כׇּל־עַמְּךָ בֵּית יִשְׂרָאֵל לְפָנֶיךָ \middot לִפְלֵיטָה וּלְטוֹבָה וּלְחֵן וּלְחֶֽסֶד וּלְרַחֲמִים וּלְחַיִּים וּלְשָׁלוֹם בְּיוֹם #1 הַזֶּה \middot זׇכְרֵֽנּוּ יְיָ אֱלֹהֵֽינוּ בּוֹ לְטוֹבָה וּפׇקְדֵֽנוּ בוֹ לִבְרָכָה וְהוֹשִׁיעֵֽנוּ בוֹ לְחַיִּים \middot וּבִדְבַר יְשׁוּעָה וְרַחֲמִים חוּס וְחׇׇׇׇנֵּנוּ וְרַחֵם עָלֵֽינוּ וְהוֹשִׁיעֵֽנוּ כִּי אֵלֶֽיךָ עֵינֵֽינוּ כִּי אֵל מֶֽלֶךְ חַנּוּן וְרַחוּם אַֽתָּה׃
}


\newcommand{\zion}{
	\firstword{וְתֶחֱזֶֽינָה}
	עֵינֵֽינוּ בְּשׁוּבְךָ לְצִיּוֹן בְּרַחֲמִים׃ בָּרוּךְ אַתָּה יְיָ הַמַּחֲזִיר שְׁכִינָתוֹ לְצִיּוֹן׃
}

\newcommand{\modim}{
	\columnratio{0.43}
	\setlength{\columnsep}{2em}
	
	\smallskip
	\englishinst{Bow for the first five words of the following paragraph.  Recite the paragraph on the right during private prayer.  During the repetition of the Amidah, the congregation recites the paragraph on the left while the reader says the paragraph on the right.}
	\begin{paracol}{2}
		
		\begin{small}
			מוֹדִים אֲנַֽחְנוּ לָךְ שָׁאַתָּה הוּא יְיָ אֱלֹהֵֽינוּ וֵאלֹהֵי אֲבוֹתֵֽינוּ \middot אֱלֹהֵי כׇל־בָּשָׂר יוֹצְרֵֽנוּ יוֹצֵר בְּרֵאשִׁית \middot בְּרָכוֹת וְהוֹדָאוֹת לְשִׁמְךָ הַגָּדוֹל וְהַקָּדוֹשׁ עַל שֶׁהֶחֱיִיתָֽנוּ וְקִיַּמְתָּֽנוּ \middot כֵּן תְּחַיֵּֽנוּ וּתְקַיְּמֵֽנוּ וְתֶאֱסוֹף גָּלֻיּוֹתֵֽינוּ לְחַצְרֹת קׇדְשֶֽׁךָ לִשְׁמֹר חֻקֶּֽיךָ וְלַעֲשׂוֹת רְצֹנֶֽךָ וּלְעׇבְדְּךָ בְּלֵבָב שָׁלֵם \middot עַל שֶׁאֲנַֽחְנוּ מוֹדִים לָךְ \middot בָּרוּךְ אֵל הַהוֹדָאוֹת׃
			
		\end{small}
		
		\switchcolumn
		
		\firstword{מוֹדִים}
		אֲנַֽחְנוּ לָךְ שָׁאַתָּה הוּא יְיָ אֱלֹהֵֽינוּ וֵאלֹהֵי אֲבוֹתֵֽינוּ לְעוֹלָם וָעֶד \middot צוּר חַיֵּֽינוּ מָגֵן יִשְׁעֵֽנוּ אַתָּה הוּא לְדוֹר וָדוֹר \middot נוֹדֶה לְךָ וּנְסַפֵּר תְּהִלָּתֶֽךָ עַל חַיֵּֽינוּ הַמְּסוּרִים בְּיָדֶֽךָ וְעַל נִשְׁמוֹתֵֽינוּ הַפְּקוּדוֹת לָךְ \middot וְעַל נִסֶּֽיךָ שֶׁבְּכׇל־יוֹם עִמָּֽנוּ וְעַל נִפְלְאוֹתֶֽיךָ וְטוֹבוֹתֶֽיךָ שֶׁבְּכׇל־עֵת עֶֽרֶב וָבֹֽקֶר וְצׇהֳרָֽיִם \middot הַטּוֹב כִּי לֹא כָלוּ רַחֲמֶֽיךָ וְהַמְרַחֵם כִּי לֹא תַֽמּוּ חֲסָדֶֽיךָ מֵעוֹלָם קִוִֽינוּ לָךְ׃
		
	\end{paracol}
}

\newcommand{\bimeimatityahu}{
		בִּימֵי מַתִּתְיָֽהוּ בֶּן יוֹחָנָן כֹּהֵן גָּדוֹל חַשְׁמֹנַי וּבָנָיו \middot כְּשֶׁעָמְדָה מַלְכוּת יָוָן הָרְשָׁעָה עַל עַמְּךָ יִשְׂרָאֵל לְהַשְׁכִּיחָם תּוֹרָתֶֽךָ וּלְהַעֲבִירָם מֵחֻקֵּי רְצוֹנֶֽךָ׃ וְאַתָּה בְּרַחֲמֶֽיךָ הָרַבִּים עָמַֽדְתָּ לָהֶם בְּעֵת צָרָתָם רַֽבְתָּ אֶת־רִיבָם דַּֽנְתָּ אֶת־דִּינָם נָקַֽמְתָּ אֶת־נִקְמָתָם \middot מָסַֽרְתָּ גִּבּוֹרִים בְּיַד חַלָּשִׁים וְרַבִּים בְּיַד מְעַטִּים וּטְמֵאִים בְּיַד טְהוֹרִים וּרְשָׁעִים בְּיַד צַדִּיקִים וְזֵדִים בְּיַד עוֹסְקֵי תוֹרָתֶֽךָ׃ וּלְךָ עָשִֽׂיתָ שֵׁם גָּדוֹל וְקָדוֹשׁ בְּעוֹלָמֶֽךָ וּלְעַמְּךָ יִשְׂרָאֵל עָשִֽׂיתָ תְּשׁוּעָה גְדוֹלָה וּפֻרְקָן כְּהַיּוֹם הַזֶּה׃ וְאַֽחַר כַּךְ בָּֽאוּ בָנֶֽיךָ לִדְבִיר בֵּיתֶֽךָ וּפִנּוּ אֶת־הֵיכָלֶֽךָ וְטִהֲרוּ אֶת־מִקְדָּשֶֽׁךָ \middot וְהִדְלִֽיקוּ נֵרוֹת בְּחַצְרוֹת קׇדְּשֶֽׁךָ וְקָבְעוּ שְׁמוֹנַת יְמֵי חֲנֻכָּה אֵֽלּוּ לְהוֹדוֹת לְהַלֵּל לְשִׁמְךָ הַגָּדוֹל׃
}

\newcommand{\alhanisim}{
	
	\begin{sometimes}
		
		%\vspace{-7mm}
		\columnratio{0.62}
		\begin{paracol}{2}[
			\englishinst{On \d{H}anukka and Purim:}
			\firstword{עַל הַנִּסִּים}
			וְעַל הַפֻּרְקָן וְעַל הַגְּבוּרוֹת וְעַל הַתְּשׁוּעוֹת וְעַל הַמִּלְחָמוֹת שֶׁעָשִֽׂיתָ לַאֲבוֹתֵֽינוּ בַּיָּמִים הָהֵם בַּזְּמַן הַזֶּה׃
			]
			\instruction{בחנוכה:} \bimeimatityahu
		
			\switchcolumn
			\instruction{בפורים:}
			בִּימֵי מׇרְדְּכַי וְִאֶסְתֵּר בְּשׁוּשַׁן הַבִּירָה כְּשֶׁעָמַד עֲלֵיהֶם הָמָן הָרָשָׁע בִּקֵּשׁ
			\mdsource{אסתר ג}
			לְהַשְׁמִ֡יד לַֽהֲרֹ֣ג וּלְאַבֵּ֣ד אֶת־כׇּל־הַ֠יְּהוּדִים מִנַּ֨עַר וְעַד־זָקֵ֨ן טַ֤ף וְנָשִׁים֙ בְּי֣וֹם אֶחָ֔ד בִּשְׁלוֹשָׁ֥ה עָשָׂ֛ר לְחֹ֥דֶשׁ שְׁנֵים־עָשָׂ֖ר הוּא־חֹ֣דֶשׁ אֲדָ֑ר וּשְׁלָלָ֖ם לָבֽוֹז׃ וְאַתָּה בְּרַחֲמֶֽיךָ הָרַבִּים הֵפַֽרְתָּ אֶת־עֲצָתוֹ וְקִלְקַלְתָּ אֶת־מַחֲשַׁבְתּוֹ וַהֲשֵׁבֽוֹתָ גְּמוּלוֹ בְּרֹאשׁוֹ וְתָלוּ אוֹתוֹ וְאֶת־בָּנָיו עַל הָעֵץ׃
		\end{paracol}
	\end{sometimes}
}



\newcommand{\weekdaysahodos}{
	\firstword{וְעַל כֻּלָּם}
	יִתְבָּרַךְ וְיִתְרוֹמַם שִׁמְךָ מַלְכֵּֽנוּ תָּמִיד לְעוֹלָם וָעֶד \middot
	
	\begin{small}
		
		\instruction{בעשי״ת:}
		וּכְתוֹב לְחַיִּים טוֹבִים בְּנֵי בְרִיתֶֽךָ \middot
		
	\end{small}
	\englishinst{Bend the knees and bow when saying \hebineng{ברוך} and \hebineng{אתה} respectively for the blessing in the following paragraph}
	\firstword{וְכׇל־הַחַיִּים}
	יוֹדֽוּךָ סֶּֽלָה וִיהַלְלוּ אֶת־שִׁמְךָ בֶּאֱמֶת \middot הָאֵל יְשׁוּעָתֵֽנוּ וְעֶזְרָתֵֽנוּ סֶֽלָה׃ בָּרוּךְ אַתָּה יְיָ הַטּוֹב שִׁמְךָ וּלְךָ נָאֶה לְהוֹדוֹת׃
	
	
	
	%\columnratio{0.52}
	%\begin{paracol}{2}
	%\instruction{כהנים לעצמם בלחש:}\\
	%יְהִי רָצוׂן מִלְּפָנֶֽךָ יְיָ אֱלֹהֵֽינוּ וֵאלֹהֵי אֲבוׂתֵֽינוּ שֶׁתְּהִי הַבְּרָכָה הַזֹּאת שֶׁצִּוִּיתָנוּ לְבָרֵךְ אֶת־עַמְּךָ יִשׂרָאֵל בְּרָכָה שְׁלֵמָה. וְלֹא יִהְיֶה בָּה מִכְשׁוֹל וְעָוׂן מֵעַתָּה וְעַד עוׂלָם׃
	
	%\switchcolumn
}




\newcommand{\bircaskohanimnl}[2]{
	\begin{small}
		
		\vspace{-.5\baselineskip}\rule[-0.5ex]{2in}{1pt}
		
		\instruction{#1}
		
		%\columnratio{0.52}
		%\begin{paracol}{2}
		%\instruction{כהנים לעצמם בלחש:}\\
		%יְהִי רָצוׂן מִלְּפָנֶֽךָ יְיָ אֱלֹהֵֽינוּ וֵאלֹהֵי אֲבוׂתֵֽינוּ שֶׁתְּהִי הַבְּרָכָה הַזֹּאת שֶׁצִּוִּיתָנוּ לְבָרֵךְ אֶת־עַמְּךָ יִשׂרָאֵל בְּרָכָה שְׁלֵמָה. וְלֹא יִהְיֶה בָּה מִכְשׁוֹל וְעָוׂן מֵעַתָּה וְעַד עוׂלָם׃
		
		%\switchcolumn
		
		\shatz \firstword{אֱלֹהֵֽינוּ}
		וֵאלֹהֵי אֲבוֹתֵֽינוּ בָּרְכֵֽנוּ בַּבְּרָכָה הַמְשֻׁלֶּֽשֶׁת בַּתּוֹרָה \middot הַכְּתוּבָה עַל יְדֵי מֹשֶׁה עַבְדֶּֽךָ \middot הָאֲמוּרָה מִפִּי אַהֲרֹן וּבָנָיו כֹּהֲנִים
		
		\instruction{ש״ץ וקהל:}
		\textbf{עַם קְדוֹשֶֽׁךָ כָּאָמוּר׃}
		%\end{paracol}
		
		\instruction{כהנים:}
		בָּרוּךְ אַתָּה יְיָ אֱלֹהֵֽינוּ מֶֽלֶךְ הָעוֹלָם \middot אֲשֶׁר קִדְּשָֽׁנוּ בִּקְדֻשָּׁתוֹ שֶׁל אַהֲרֹן וְצִוָּֽנוּ לְבָרֵךְ אֶת־עַמּוֹ יִשְׂרָאֵל בְּאַהֲבָה׃

		\begin{large}
			
			\textbf{
				יְבָֽרֶכְךָ֥\source{במדבר ו} יְיָ֖ וְיִשְׁמְרֶֽךָ׃\\
				יָאֵ֨ר יְיָ֧ פָּנָ֛יו אֵלֶ֖יךָ וִֽיחֻנֶּֽךָּ׃\\
				יִשָּׂ֨א יְיָ֤ פָּנָיו֙ אֵלֶ֔יךָ וְיָשֵׂ֥ם לְךָ֖ שָׁלֽוֹם׃
			}
			
		\end{large}
		
		\columnratio{0.6}
		\begin{paracol}{2}
			\instruction{כהנים:}
			רִבּוֹן הָעוֹלָם עָשִֽׂינוּ מַה שֶּׁגָּזַֽרְתָּ עָלֵֽינוּ אַף אַתָּה עֲשֵׂה עִמָּֽנוּ כַּאֲשֶׁר הִבְטַחְתָּֽנוּ׃ הַשְׁקִ֩יפָה֩ מִמְּע֨וֹן קׇדְשְׁךָ֜ מִן־הַשָּׁמַ֗יִם וּבָרֵ֤ךְ אֶֽת־עַמְּךָ֙ אֶת־יִשְׂרָאֵ֔ל וְאֵת֙ הָֽאֲדָמָ֔ה אֲשֶׁ֥ר נָתַ֖תָּה לָ֑נוּ כַּֽאֲשֶׁ֤ר נִשְׁבַּ֨עְתָּ֙ לַֽאֲבֹתֵ֔ינוּ אֶ֛רֶץ זָבַ֥ת חָלָ֖ב וּדְבָֽשׁ׃
			
			\switchcolumn
			
			\kahal
			אַדִּיר בַּמָּרוֹם שׁוֹכֵן בִּגְבוּרָה אַתָּה שָׁלוֹם וְשִׁמְךָ שָׁלוֹם׃ יְהִי רָצוֹן שֶׁתָּשִׂים עָלֵֽינוּ וְעַל כׇּל־עַמְּךָ בֵּית יִשְׂרָאֵל חַיִּים וּבְרָכָה לְמִשְׁמֶֽרֶת שָׁלוֹם׃
		\end{paracol}
		
		\sepline
		
		\instruction{#2}\\
		אֱלֹהֵֽינוּ וֵאלֹהֵי אֲבוֹתֵֽינוּ בָּרְכֵֽנוּ בַּבְּרָכָה הַמְשֻׁלֶּֽשֶׁת בַּתּוֹרָה \middot
		הַכְּתוּבָה עַל יְדֵי מֹשֶׁה עַבְדֶּֽךָ \middot הָאֲמוּרָה מִפִּי אַהֲרֹן וּבָנָיו כֹּהֲנִים עַם קְדוֹשֶֽׁךָ כָּאָמוּר׃
		
		יְבָֽרֶכְךָ֥\source{במדבר ו} יְיָ֖ וְיִשְׁמְרֶֽךָ׃ \hfill \kahal כֵּן יְהִי רָצוׂן \\
		יָאֵ֨ר יְיָ֧ פָּנָ֛יו אֵלֶ֖יךָ וִֽיחֻנֶּֽךָּ׃ \hfill \kahal כֵּן יְהִי רָצוׂן \\
		יִשָּׂ֨א יְיָ֤ פָּנָיו֙ אֵלֶ֔יךָ וְיָשֵׂ֥ם לְךָ֖ שָׁלֽוֹם׃ \hfill \kahal כי״ר
		
	\end{small}
}

\newcommand{\tachanunim}{
	\firstword{אֱלֹהַי}
	נְצֹר לְשׁוֹנִי מֵרָע וּשְׂפָתַי מִדַּבֵּר מִרְמָה וְלִמְקַלְלַי נַפְשִׁי תִדּוֹם וְנַפְשִׁי כֶּעָפָר לַכֹּל תִּהְיֶה׃ פְּתַח לִבִּי בְּתוֹרָתֶֽךָ וּבְמִצְוֹתֶֽיךָ תִּרְדּוֹף נַפְשִׁי \middot וְכֹל הַחוֹשְׁבִים עָלַי רָעָה מְהֵרָה הָפֵר עֲצָתָם וְקַלְקֵל מַחֲשַׁבְתָם׃ עֲשֵׂה לְמַֽעַן שְׁמֶֽךָ עֲשֵׂה לְמַֽעַן יְמִינֶֽךָ עֲשֵׂה לְמַֽעַן קְדֻשָּׁתֶֽךָ עֲשֵׂה לְמַֽעַן תּוֹרָתֶֽךָ׃ לְ֭מַעַן \source{תהלים ס}יֵחָֽלְצ֥וּן יְדִידֶ֑יךָ הֽוֹשִׁ֖יעָה יְמִֽינְךָ֣ וַֽעֲנֵֽנִי׃ \source{תהלים יט}יִֽהְי֥וּ לְרָצ֨וֹן אִמְרֵי־פִ֡י וְהֶגְי֣וֹן לִבִּ֣י לְפָנֶ֑יךָ יְ֜יָ֗ צוּרִ֥י וְגֹֽאֲלִֽי׃
	\osehshalom
	
	
	\begin{small}
		
		יְהִי רָצוֹן מִלְּפָנֶֽיךָ יְיָ אֱלֹהֵֽינוּ וִֵאלֹהֵי אֲבוֹתֵֽינוּ שֶׁיִבָּנֶה בֵּית הַמִּקְדָּשׁ בִּמְהֵרָה בְיָמֵֽינוּ וְתֵן חֶלְקֵֽנוּ בְּתוֹרָתֶֽךָ׃ וְשָׁם נַעֲבׇדְךָ בְּיִרְאָה כִּימֵי עוֹלָם וּכְשָׁנִים קַדְמֹנִיּוֹת׃
		וְעָֽרְבָה֙ \source{מלאכי ג}לַֽיְיָ֔ מִנְחַ֥ת יְהוּדָ֖ה וִירוּשָׁלָ֑םִ כִּימֵ֣י עוֹלָ֔ם וּכְשָׁנִ֖ים קַדְמֹֽנִיּֽוֹת׃
		
		
	\end{small}
}

\newcommand{\shatzbirkaskohanim}[1]{
	
	\begin{narrow}
		
		\instruction{#1}
		אֱלֹהֵֽינוּ וֵאלֹהֵי אֲבוֹתֵֽינוּ בָּרְכֵֽנוּ בַּבְּרָכָה הַמְשֻׁלֶּֽשֶׁת בַּתּוֹרָה
		הַכְּתוּבָה עַל יְדֵי מֹשֶׁה עַבְדֶּֽךָ הָאֲמוּרָה מִפִּי אַהֲרֹן וּבָנָיו כֹּהֲנִים עַם קְדוֹשֶֽׁךָ כָּאָמוּר׃
		
		יְבָֽרֶכְךָ֥\source{במדבר ו} יְיָ֖ וְיִשְׁמְרֶֽךָ׃ \hfill \kahal כֵּן יְהִי רָצוׂן \\
		יָאֵ֨ר יְיָ֧ ׀ פָּנָ֛יו אֵלֶ֖יךָ וִֽיחֻנֶּֽךָּ׃ \hfill \kahal כֵּן יְהִי רָצוׂן \\
		יִשָּׂ֨א יְיָ֤ ׀ פָּנָיו֙ אֵלֶ֔יךָ וְיָשֵׂ֥ם לְךָ֖ שָׁלֽוֹם׃ \hfill \kahal כי״ר
	\end{narrow}
}

\newcommand{\mishnahtamid}{
\firstword{הַשִּׁיר שֶׁהַלְוִיִּם}\source{תמיד פ״ז}
הָיוּ אוֹמְרִים בְּבֵית הַמִּקְדָּשׁ׃
בַּיּוֹם הַרִאשׁוֹן הָיוּ אוֹמְרִים \source{תהלים כד}%
לְדָוִ֗ד מִ֫זְמ֥וֹר לַֽייָ֭ הָאָ֣רֶץ וּמְלוֹאָ֑הּ תֵּ֝בֵ֗ל וְיֹ֣שְׁבֵי בָֽהּ׃
בַּשֵּׁנִי הָיוּ אוֹמְרִים \source{תהלים מח}
גָּ֘ד֤וֹל יְיָ֣ וּמְהֻלָּ֣ל מְאֹ֑ד בְּעִ֥יר אֱ֝לֹהֵ֗ינוּ הַר־קׇדְשֽׁוֹ׃
בַּשְּׁלִישִׁי הָיוּ אוֹמְרִים \source{תהלים פב}
אֱֽלֹהִ֗ים נִצָּ֥ב בַּעֲדַת־אֵ֑ל בְּקֶ֖רֶב אֱלֹהִ֣ים יִשְׁפֹּֽט׃
בָּרְבִיעִי הָיוּ אוֹמְרִים \source{תהלים צד}
אֵל־נְקָמ֥וֹת יְיָ֑ אֵ֖ל נְקָמ֣וֹת הוֹפִֽיעַ׃
בַּחֲמִישִׁי הָיוּ אוֹמְרִים \source{תהלים פא}
הַ֭רְנִינוּ לֵאלֹהִ֣ים עוּזֵּ֑נוּ הָ֝רִ֗יעוּ לֵאלֹהֵ֥י יַעֲקֹֽב׃
בַּשִּׁשִּׁי הָיוּ אוֹמְרִים \source{תהלים צג}
יְיָ֣ מָלָךְ֮ גֵּא֢וּת לָ֫בֵ֥שׁ לָבֵ֣שׁ יְיָ֭ עֹ֣ז הִתְאַזָּ֑ר אַף־תִּכּ֥וֹן תֵּ֝בֵ֗ל בַּל־תִּמּֽוֹט׃
בַּשַׁבָּת הָיוּ אוֹמְרִים \source{תהלים צב}
מִזְמ֥וֹר שִׁ֗יר לְי֣וֹם הַשַּׁבָּֽת׃
מִזְמוֹר שִׁיר לֶעָתִיד לָבוֹא לְיוֹם שֶׁכֻּלּוֹ שַׁבָּת וּמְנוּחָה לְחַיֵּי הָעוֹלָמִים׃
}

\newcommand{\einkeloheinu}{
 	אֵין כֵּאלֹהֵֽינוּ\hfill אֵין כַּאדוֹנֵֽנוּ \hfill אֵין כְּמַלְכֵּֽנוּ \hfill אֵין כְּמוֹשִׁיעֵֽנוּ׃\\
 	מִי כֵאלֹהֵֽינוּ \hfill מִי כַאדוֹנֵֽנוּ \hfill מִי כְמַלְכֵּֽנוּ \hfill מִי כְמוֹשִׁיעֵֽנוּ׃\\
 	נוֹדֶה לֵאלֹהֵֽינוּ \hfill נוֹדֶה לַאדוֹנֵֽנוּ \hfill נוֹדֶה לְמַלְכֵּֽנוּ \hfill נוֹדֶה לְמוֹשִׁיעֵֽנוּ׃\\
 	בָּרוּךְ אֱלֹהֵֽינוּ \hfill בָּרוּךְ אֲדוֹנֵֽנוּ \hfill בָּרוּךְ מַלְכֵּֽנוּ \hfill בָּרוּךְ מוֹשִׁיעֵֽנוּ׃\\
 	\hfill 
 	אַתָּה הוּא אֱלֹהֵֽינוּ\hfill אַתָּה הוּא אֲדוֹנֵֽנוּ\hfill\\\hfill אַתָּה הוּא מַלְכֵּֽנוּ\hfill אַתָּה הוּא מוֹשִׁיעֵֽנוּ׃\hfill 
 	אַתָּה הוּא שֶׁהִקְטִֽירוּ אֲבוֹתֵֽינוּ לְפָנֶֽיךָ אֶת־קְטֹֽרֶת הַסַּמִּים׃
}

\newcommand{\conclusionshabYT}{
\einkeloheinu
	
\pitumhaketoret

\mishnahtamid

\sofberakhot

\rabbiskaddish

\aleinu
}

\newcommand{\sofberakhot}{
	\firstword{אָמַר רַבִּי אֱלְעָזָר}\source{ברכות סד}
	אָמַר רַבִּי חֲנִינָא׃ תַּלְמִידֵי חֲכָמִים מַרְבִּים שָׁלוֹם בָּעוֹלָם שֶׁנֶּאֱמַר׃ וְכׇל־בָּנַ֖יִךְ \source{ישעיה נד}לִמּוּדֵ֣י יְיָ֑ וְרַ֖ב שְׁל֥וֹם בָּנָֽיִךְ׃ אַל תִּקְרָא בָּנַֽיִךְ אֶלָּא בּוֹנַֽיִךְ׃ שָׁל֣וֹם רָ֭ב \source{תהלים קיט}לְאֹהֲבֵ֣י תוֹרָתֶ֑ךָ וְאֵֽין־לָ֥מוֹ מִכְשֽׁוֹל׃ יְהִי־שָׁל֥וֹם \source{תהלים קכב}בְּחֵילֵ֑ךְ שַׁ֝לְוָ֗ה בְּאַרְמְנוֹתָֽיִךְ׃ לְ֭מַעַן אַחַ֣י וְרֵעָ֑י אֲדַבְּרָה־נָּ֖א שָׁל֣וֹם בָּֽךְ׃ לְ֭מַעַן בֵּית־יְיָ֣ אֱלֹהֵ֑ינוּ אֲבַקְשָׁ֖ה ט֣וֹב לָֽךְ׃ יְיָ֗ \source{תהלים כט} עֹ֭ז לְעַמּ֣וֹ יִתֵּ֑ן יְיָ֓ ׀ יְבָרֵ֖ךְ אֶת־עַמּ֣וֹ בַשָּׁלֽוֹם׃}

\newcommand{\barukhbayom}{}

\newcommand{\yerueinnu}{
	
	\firstword{יִרְאוּ}
	עֵינֵֽינוּ וְיִשְׂמַח לִבֵּֽנוּ וְתָגֵל נַפְשֵֽׁנוּ בִּישׁוּעָתְךָ בֶּאֱמֶת בֶּאֱמֹר לְצִיּוֹן מָלַךְ אֱלֹהָֽיִךְ׃
	\melekhmalakhyimlokh
	כִּי הַמַּלְכוּת שֶׁלְּךָ הִיא וּלְעֽוֹלְמֵי עַד תִּמְלֹךְ בְּכָבוֹד כִּי אֵין לָֽנוּ מֶֽלֶךְ אֶלָּא אַֽתָּה׃
}

\newcommand{\boruchhashemleolam}{
	\firstword{בָּר֖וּךְ} \source{תהלים פט}יְיָ֥ לְ֝עוֹלָ֗ם אָ֘מֵ֥ן ׀ וְאָמֵֽן׃
	בָּ֘ר֤וּךְ \source{תהלים קלה}יְיָ֨ ׀ מִצִּיּ֗וֹן שֹׁ֘כֵ֤ן יְֽרוּשָׁלָ֗‍ִם הַֽלְלוּ־יָֽהּ׃
	בָּר֤וּךְ \source{תהלים עב}׀ יְיָ֣ אֱ֭לֹהִים אֱלֹהֵ֣י יִשְׂרָאֵ֑ל עֹשֵׂ֖ה נִפְלָא֣וֹת לְבַדּֽוֹ׃ וּבָר֤וּךְ ׀ שֵׁ֥ם כְּבוֹד֗וֹ לְע֫וֹלָ֥ם וְיִמָּלֵ֣א כְ֭בוֹדוֹ אֶת־כֹּ֥ל הָאָ֗רֶץ אָ֘מֵ֥ן ׀ וְאָמֵֽן׃
	יְהִ֤י \source{תהלים קד}כְב֣וֹד יְיָ֣ לְעוֹלָ֑ם יִשְׂמַ֖ח יְיָ֣ בְּמַעֲשָֽׂיו׃
	יְהִ֤י \source{תהלים קיג}שֵׁ֣ם יְיָ֣ מְבֹרָ֑ךְ מֵ֝עַתָּ֗ה וְעַד־עוֹלָֽם׃
	כִּ֠י \source{שמ״א יב}לֹֽא־יִטֹּ֤שׁ יְיָ֙ אֶת־עַמּ֔וֹ בַּעֲב֖וּר שְׁמ֣וֹ הַגָּד֑וֹל כִּ֚י הוֹאִ֣יל יְיָ֔ לַעֲשׂ֥וֹת אֶתְכֶ֛ם ל֖וֹ לְעָֽם׃
	וַיַּרְא֙ \source{מ״א יח}כׇּל־הָעָ֔ם וַֽיִּפְּל֖וּ עַל־פְּנֵיהֶ֑ם וַיֹּ֣אמְר֔וּ יְיָ֙ ה֣וּא הָאֱלֹהִ֔ים יְיָ֖ ה֥וּא הָאֱלֹהִֽים׃
	\source{זכריה יד}וְהָיָ֧ה יְיָ֛ לְמֶ֖לֶךְ עַל־כׇּל־הָאָ֑רֶץ בַּיּ֣וֹם הַה֗וּא יִהְיֶ֧ה יְיָ֛ אֶחָ֖ד וּשְׁמ֥וֹ אֶחָֽד׃
	\source{תהלים לג}יְהִי־חַסְדְּךָ֣ יְיָ֣ עָלֵ֑ינוּ כַּ֝אֲשֶׁ֗ר יִחַ֥לְנוּ לָֽךְ׃
	\source{תהלים קו}הוֹשִׁיעֵ֨נוּ ׀ יְ֘יָ֤ אֱלֹהֵ֗ינוּ וְקַבְּצֵנוּ֮ מִֽן־הַגּ֫וֹיִ֥ם לְ֭הֹדוֹת לְשֵׁ֣ם קׇדְשֶׁ֑ךָ לְ֝הִשְׁתַּבֵּ֗חַ בִּתְהִלָּתֶֽךָ׃
	כׇּל־גּוֹיִ֤ם \source{תהלים פו}׀ אֲשֶׁ֥ר עָשִׂ֗יתָ יָב֤וֹאוּ ׀ וְיִשְׁתַּֽחֲו֣וּ לְפָנֶ֣יךָ אֲדֹנָ֑י וִ֖יכַבְּד֣וּ לִשְׁמֶֽךָ׃ כִּֽי־גָד֣וֹל אַ֭תָּה וְעֹשֵׂ֣ה נִפְלָא֑וֹת אַתָּ֖ה אֱלֹהִ֣ים לְבַדֶּֽךָ׃
	וַאֲנַ֤חְנוּ \source{תהלים עט}עַמְּךָ֨ ׀ וְצֹ֥אן מַרְעִיתֶךָ֮ נ֤וֹדֶ֥ה לְּךָ֗ לְע֫וֹלָ֥ם לְד֥וֹר וָדֹ֑ר נְ֝סַפֵּ֗ר תְּהִלָּתֶֽךָ׃
	בָּרוּךְ יְיָ בַּיּוֹם. בָּרוּךְ יְיָ בַּלָּֽיְלָה׃ בָּרוּךְ יְיָ בְּשׇׁכְבֵֽנוּ. בָּרוּךְ יְיָ בְּקוּמֵֽנוּ׃ כִּי בְיָדְךָ נַפְשׁוֹת הַחַיִּים וְהַמֵּתִים׃
	אֲשֶׁ֣ר \source{איוב יב}בְּ֭יָדוֹ נֶ֣פֶשׁ כׇּל־חָ֑י וְ֝ר֗וּחַ כׇּל־בְּשַׂר־אִֽישׁ׃
	בְּיָדְךָ֮ \source{תהלים לא}אַפְקִ֢יד ר֫וּחִ֥י פָּדִ֖יתָ אוֹתִ֥י יְיָ֗ אֵ֣ל אֱמֶֽת׃
	אֱלֹהֵֽינוּ שֶׁבַּשָּׁמַֽיִם יַחֵד שִׁמְךָ וְקַיֵּם מַלְכוּתְךָ תָּמִיד וּמְלֹךְ עָלֵֽינוּ לְעוֹלָם וָעֶד׃
	\yerueinnu
	בָּרוּךְ אַתָּה יְיָ הַמֶּֽלֶךְ בִּכְבוֹדוֹ תָּמִיד יִמְלוֹךְ עָלֵֽינוּ לְעוֹלָם וָעֶד וְעַל כׇּל־מַעֲשָׂיו׃
}

\newcommand{\aleinu}{
	\firstword{עָלֵֽינוּ}
	לְשַׁבֵּחַ לַאֲדוֹן הַכֹּל \middot לָתֵת גְּדֻלָּה לְיוֹצֵר בְּרֵאשִׁית׃ שֶׁלֹּא עָשָׂנוּ כְּגוֹיֵי הָאֲרָצוֹת \middot וְלֹא שָׂמָנוּ כְּמִשְׁפְּחוֹת הָאֲדָמָה׃ שֶׁלֹּא שָׂם חֶלְקֵנוּ כָּהֶם \middot וְגוֹרָלֵנוּ כְּכׇל־הֲמוֹנָם׃ [שֶׁהֵם מִשְׁתַּחֲוִים לְהֶבֶל וָרִיק \middot וּמִתְפַּלֲּלִים אֶל אֵל לֹא יוֹשִׁיעַ׃] וַאֲנַחְנוּ כּוֹרְעִים וּמִשְׁתַּחֲוִים וּמוֹדִים לִפְנֵי מֶלֶךְ מַלְכֵי הַמְּלָכִים הַקָּדוֹשׁ בָּרוּךְ הוּא׃ שֶׁהוּא נוֹטֶה שָׁמַיִם וְיֹסֵד אָרֶץ \middot וּמוֹשַׁב יְקָרוֹ בַּשָּׁמַיִם מִמַּעַל \middot וּשְׁכִינַת עֻזּוֹ בְּגׇבְהֵי מְרוֹמִים׃ הוּא אֱלֹהֵינוּ אֵין עוֹד \middot אֱמֶת מַלְכֵּנוּ אֶפֶס זוּלָתוֹ׃ כַּכָּתוּב בְּתּוֹרָתוֹ׃ וְיָדַעְתָּ֣
	\source{דברים ד}
	הַיּ֗וֹם וַהֲשֵׁבֹתָ֮ אֶל־לְבָבֶ֒ךָ֒ כִּ֤י יְיָ֙ ה֣וּא הָֽאֱלֹהִ֔ים בַּשָּׁמַ֣יִם מִמַּ֔עַל וְעַל־הָאָ֖רֶץ מִתָּ֑חַת אֵ֖ין עֽוֹד׃\\
	עַל כֵּן נְקַוֶּה לְךָ יְיָ אֱלֹהֵינוּ לִרְאוֹת מְהֵרָה בְּתִפְאֶרֶת עֻזֶּךָ \middot לְהַעֲבִיר גִּלּוּלִים מִן הָאָרֶץ וְהָאֱלִילִים כָּרוֹת יִכָּרֵתוּן \middot לְתַקֵּן עוֹלָם בְּמַלְכוּת שַׁדַּי, וְכׇל־בְּנֵי בָשָׂר יִקְרְאוּ בִשְׁמֶךָ \middot לְהַפְנוֹת אֵלֶיךָ כָּל־רִשְׁעֵי אָרֶץ \middot יַכִּירוּ וְיֵדְעוּ כָּל־יוֹשְׁבֵי תֵבֵל \middot כִּי לְךָ תִכְרַע כָּל־בֶּרֶךְ תִּשָּׁבַע כָּל־לָשׁוֹן׃ לְפָנֶיךָ יְיָ אֱלֹהֵינוּ יִכְרְעוּ וְיִפֹּלוּ וְלִכְבוֹד שִׁמְךָ יְקָר יִתֵּנוּ \middot וִיקַבְּלוּ כֻלָּם אֶת־עֹל מַלְכוּתֶךָ וְתִמְלֹךְ עֲלֵיהֶם מְהֵרָה לְעוֹלָם וָעֶד׃ כִּי הַמַּלְכוּת שֶׁלְּךָ הִיא וּלְעוֹלְמֵי עַד תִּמְלֹךְ בְּכָבוֹד׃ כַּכָּתוּב בְּתוֹרָתֶךָ׃\source{שמות טו} יְיָ֥ ׀ יִמְלֹ֖ךְ לְעֹלָ֥ם וָעֶֽד׃ וְנֶאֱמַר׃\source{זכריה יד} וְהָיָ֧ה יְיָ֛ לְמֶ֖לֶךְ עַל־כׇּל־הָאָ֑רֶץ בַּיּ֣וֹם הַה֗וּא יִהְיֶ֧ה יְיָ֛ אֶחָ֖ד וּשְׁמ֥וֹ אֶחָֽד׃
	%\firstword{אַל־תִּ֭ירָא}\source{משלי ג}
	%מִפַּ֣חַד פִּתְאֹ֑ם וּמִשֹּׁאַ֥ת רְ֝שָׁעִ֗ים כִּ֣י תָבֹֽא׃\source{ישעיה ח}
	%עֻ֥צוּ עֵצָ֖ה וְתֻפָ֑ר דַּבְּר֤וּ דָבָר֙ וְלֹ֣א יָק֔וּם כִּ֥י עִמָּ֖נוּ אֵֽל׃\source{ישעיה מו}
	%וְעַד־זִקְנָה֙ אֲנִ֣י ה֔וּא וְעַד־שֵׂיבָ֖ה אֲנִ֣י אֶסְבֹּ֑ל אֲנִ֤י עָשִׂ֙יתִי֙ וַאֲנִ֣י אֶשָּׂ֔א וַאֲנִ֥י אֶסְבֹּ֖ל וַאֲמַלֵּֽט׃
	
}

\newcommand{\shabboshodos}{
	\firstword{וְעַל כֻּלָּם}
	יִתְבָּרַךְ וְיִתְרוֹמַם שִׁמְךָ מַלְכֵּֽנוּ תָּמִיד לְעוֹלָם וָעֶד׃
	
	\instruction{בשבת שובה:}
	וּכְתוֹב לְחַיִּים טוֹבִים בְּנֵי בְרִיתֶֽךָ׃
	
	\firstword{וְכׇל־הַחַיִּים}
	יוֹדֽוּךָ סֶּֽלָה וִיהַלְלוּ אֶת־שִׁמְךָ בֶּאֱמֶת הָאֵל יְשׁוּעָתֵֽנוּ וְעֶזְרָתֵֽנוּ סֶֽלָה׃ בָּרוּךְ אַתָּה יְיָ הַטּוֹב שִׁמְךָ וּלְךָ נָאֶה לְהוֹדוֹת׃
}

\newcommand{\hamaarivaravim}{
	\firstword{בָּרוּךְ}
	אַתָּה יְיָ אֱלֹהֵֽינוּ מֶֽלֶךְ הָעוֹלָם אֲשֶׁר בִּדְבָרוֹ מַעֲרִיב עֲרָבִים בְּחׇכְמָה פּוֹתֵֽחַ שְׁעָרִים וּבִתְבוּנָה מְשַׁנֶּה עִתִּים וּמַחֲלִיף אֶת־הַזְּמַנִּים וּמְסַדֵּר אֶת־הַכּוֹכָבִים בְּמִשְׁמְרוֹתֵֽיהֶם בָּרָקִֽיעַ כִּרְצוֹנוֹ׃ בּוֹרֵא יוֹם וָלָֽיְלָה גּוֹלֵל אוֹר מִפְּנֵי חֹֽשֶׁךְ וְחֹֽשֶׁךְ מִפְּנֵי אוֹר׃ וּמַעֲבִיר יוֹם וּמֵֽבִיא לָֽיְלָה וּמַבְדִּיל בֵּין יוֹם וּבֵין לָֽיְלָה יְיָ צְבָאוֹת שְׁמוֹ׃ אֵל חַי וְקַיָּם תָּמִיד יִמְלוֹךְ עָלֵֽינוּ לְעוֹלָם וָעֶד׃ בָּרוּךְ אַתָּה יְיָ הַמַּעֲרִיב עֲרָבִים׃
}

\newcommand{\ahavasolam}{
	\firstword{אַהֲבַת}
	עוֹלָם בֵּית יִשְׂרָאֵל עַמְּךָ אָהַבְתָּ \middot תּוֹרָה וּמִצְוֹת חֻקִּים וּמִשְׁפָּטִים אוֹתָֽנוּ לִמַֽדְתָּ׃ עַל כֵּן יְיָ אֱלֹהֵֽינוּ בְּשׇׁכְבֵּֽנוּ וּבְקוּמֵֽנוּ נָשִֽׂיחַ בְּחֻקֶּיךָ וְנִשְׂמַח בְּדִבְרֵי תוֹרָתֶֽךָ וּבְמִצְוֹתֶֽיךָ לְעוֹלָם וָעֶד׃ כִּי הֵם חַיֵּֽינוּ וְאֹֽרֶךְ יָמֵֽינוּ וּבָהֶם נֶהְגֶּה יוֹמָם וָלָֽיְלָה \middot וְאַהֲבָתְךָ אַל תָּסִיר מִמֶּֽנּוּ לְעוֹלָמִים׃ בָּרוּךְ אַתָּה יְיָ אוֹהֵב עַמּוֹ יִשְׂרָאֵל׃
}

\newcommand{\emesveemuna}{
	%\instruction{הש״ץ אומר אמת בקול רם:}\\
	\firstword{אֱמֶת}
	וֶאֱמוּנָה כׇּל־זֹאת וְקַיָּם עָלֵֽינוּ כִּי הוּא יְיָ אֱלֹהֵֽינוּ וְאֵין זוּלָתוֹ וַאֲנַֽחְנוּ יִשְׂרָאֵל עַמּוֹ׃ הַפּוֹדֵֽנוּ מִיַּד מְלָכִים מַלְכֵּֽנוּ הַגּוֹאֲלֵֽנוּ מִכַּף כׇּל־הֶעָרִיצִים \middot הָאֵל הַנִּפְרָע לָֽנוּ מִצָּרֵֽנוּ וְהַמְשַׁלֵּם גְּמוּל לְכׇל־אוֹיְבֵי נַפְשֵֽׁנוּ׃ \source{איוב ט}עֹשֶׂ֣ה גְ֭דֹלוֹת עַד־אֵ֣ין חֵ֑קֶר וְנִפְלָא֗וֹת עַד־אֵ֥ין מִסְפָּֽר׃ \source{תהלים סו}הַשָּׂ֣ם נַ֭פְשֵׁנוּ בַּחַיִּ֑ים וְלֹֽא־נָתַ֖ן לַמּ֣וֹט רַגְלֵֽנוּ׃ הַמַּדְרִיכֵֽנוּ עַל בָּמוֹת אוֹיְבֵֽינוּ וַיָּֽרֶם קַרְנֵֽנוּ עַל כׇּל־שׂנְאֵֽינוּ׃ הָעֹֽשֶׂה לָּֽנוּ נִסִּים וּנְקָמָה בְּפַרְעֹה אוֹתֹת וּמוֹפְתִים בְּאַדְמַת בְּנֵי חָם \middot הַמַּכֶּה בְעֶבְרָתוֹ כׇּל־בְּכוֹרֵי מִצְרָֽיִם וַיּוֹצֵא אֶת־עַמּוֹ יִשְׂרָאֵל מִתּוֹכָם לְחֵרוּת עוֹלָם׃ הַמַּעֲבִיר בָּנָיו בֵּין גִּזְרֵי יַם סוּף אֶת־רוֹדְפֵיהֶם וְאֶת־שׂוֹנְאֵיהֶם בִּתְהוֹמוֹת טִבַּע׃ וְרָאוּ בָנָיו גְּבוּרָתוֹ שִׁבְּחוּ וְהוֹדוּ לִשְׁמוֹ׃ וּמַלְכוּתוֹ בְּרָצוֹן קִבְּלוּ עַלֵיהֶם \middot מֹשֶׁה וּבְנֵי יִשְׂרָאֵל לְךָ עָנוּ שִׁירָה בְּשִׂמְחָה רַבָּה וְאָמְרוּ כֻלָּם׃
	
	
	\kahal\source{שמות טו}\textbf{%
		מִֽי־כָמֹ֤כָה בָּֽאֵלִם֙ יְיָ֔ מִ֥י כָּמֹ֖כָה נֶאְדָּ֣ר בַּקֹּ֑דֶשׁ נוֹרָ֥א תְהִלֹּ֖ת עֹ֥שֵׂה פֶֽלֶא׃
	}
	
	
	מַלְכוּתְךָ רָאוּ בָנֶֽיךָ בּוֹקֵֽעַ יָם לִפְנֵי משֶׁה זֶ֤ה אֵלִי֙ עָנוּ וְאָמְרוּ׃
	
	\kahal \hashemyimloch
	
	
	וְנֶאֱמַר׃ \source{ירמיה לא}כִּֽי־פָדָ֥ה יְיָ֖ אֶֽת־יַעֲקֹ֑ב וּגְאָל֕וֹ מִיַּ֖ד חָזָ֥ק מִמֶּֽנּוּ׃ בָּרוּךְ אַתָּה יְיָ גָּאַל יִשְׂרָאֵל׃
}

\newcommand{\maarivmodim}{
	\firstword{מוֹדִים}
	אֲנַֽחְנוּ לָךְ שָׁאַתָּה הוּא יְיָ אֱלֹהֵֽינוּ וֵאלֹהֵי אֲבוֹתֵֽינוּ לְעוֹלָם וָעֶד צוּר חַיֵּֽינוּ מָגֵן יִשְׁעֵֽנוּ אַתָּה הוּא לְדוֹר וָדוֹר׃ נוֹדֶה לְךָ וּנְסַפֵּר תְּהִלָּתֶֽךָ עַל חַיֵּֽינוּ הַמְּסוּרִים בְּיָדֶֽךָ וְעַל נִשְׁמוֹתֵֽינוּ הַפְּקוּדוֹת לָךְ וְעַל נִסֶּֽיךָ שֶׁבְּכׇל־יוֹם עִמָּֽנוּ וְעַל נִפְלְאוֹתֶֽיךָ וְטוֹבוֹתֶֽיךָ שֶׁבְּכׇל־עֵת עֶֽרֶב וָבֹֽקֶר וְצׇהֳרָֽיִם׃ הַטּוֹב כִּי לֹא כָלוּ רַחֲמֶֽיךָ וְהַמְרַחֵם כִּי לֹא תַֽמּוּ חֲסָדֶֽיךָ מֵעוֹלָם קִוִֽינוּ לָךְ׃
}


\newcommand{\hashkiveinu}[1]{
	\firstword{הַשְׁכִּיבֵֽנוּ}
	יְיָ אֱלֹהֵֽינוּ לְשָׁלוֹם \middot וְהַעֲמִידֵֽנוּ מַלְכֵּֽנוּ לְחַיִּים \middot וּפְרוֹשׂ עָלֵֽינוּ סֻכַּת שְׁלוֹמֶֽךָ \middot וְתַקְּנֵֽנוּ בְּעֵצָה טוֹבָה מִלְּפָנֶֽיךָ וְהוֹשִׁיעֵֽנוּ לְמַֽעַן שְׁמֶֽךָ׃ וְהָגֵן בַּעֲדֵֽנוּ וְהָסֵר מֵעָלֵֽינוּ אוֹיֵב דֶּֽבֶר וְחֶֽרֶב וְרָעָב וְיָגוֹן \middot וְהָסֵר שָׂטָן מִלְּפָנֵֽינוּ וּמֵאַחֲרֵֽנוּ וּבְצֵל כְּנָפֶֽיךָ תַּסְתִּירֵֽנוּ׃ כִּי אֵל שׁוֹמְרֵֽנוּ וּמַצִּילֵֽנוּ אַֽתָּה \middot כִּי אֵל מֶֽלֶךְ חַנּוּן וְרַחוּם אַֽתָּה׃ וּשְׁמוֹר צֵאתֵֽנוּ וּבוֹאֵֽנוּ לְחַיִּים וּלְשָׁלוֹם מֵעַתָּה וְעַד עוֹלָם׃ #1
}

\newcommand{\avinumalkeinu}{
	
	אָבִֽינוּ מַלְכֵּֽנוּ חָטָאנוּ לְפָנֶיךָ׃\hfill \break
	אָבִֽינוּ מַלְכֵּֽנוּ אֵין לָנוּ מֶֽלֶךְ אֶלָּא אַֽתָּה׃ \hfill \break
	אָבִֽינוּ מַלְכֵּֽנוּ עֲשֵׂה עִמָֽנוּ לְמַעַן שְׁמֶךָ׃\hfill \break
	אָבִֽינוּ מַלְכֵּֽנוּ (\instruction{בת״צ:} בָּרֵךְ)(\instruction{בעשי״ת:} חַדֵּשׁ) עָלֵינוּ שָׁנָה טוֹבָה:\hfill \break
	אָבִֽינוּ מַלְכֵּֽנוּ בַּטֵל מֵעָלֵינוּ כׇּל־גְּזֵּרוֹת קָשׁוֹת׃\hfill \break
	אָבִֽינוּ מַלְכֵּֽנוּ בַּטֵל מַחְשְׁבוֹת שׂוֹנְאֵֽינוּ׃\hfill \break
	אָבִֽינוּ מַלְכֵּֽנוּ הָפֵר עֲצַת אוֹיְּבֵֽינוּ׃\hfill \break
	אָבִֽינוּ מַלְכֵּֽנוּ כַּלֵה כׇּל־צָר וּמַשְׂטִין מֵעָלֵֽינוּ׃\hfill \break
	אָבִֽינוּ מַלְכֵּֽנוּ סְתוֹם פִּיּוֹת מַשְׂטִינֵֽנוּ וּמְקַטְרְגֵֽינוּ׃\hfill \break
	אָבִֽינוּ מַלְכֵּֽנוּ כַּלֵּה דֶּבֶר וְחֶרֶב וְרָעָב וּשְׁבִי וּמַשְׁחִית וּמַגֵּפָה מִבְּנֵי בְּרִיתֶֽךָ׃\hfill \break
	אָבִֽינוּ מַלְכֵּֽנוּ מְנַע מַגֵּפָה מִנַּחֲלָתֶֽךָ׃\hfill \break
	אָבִֽינוּ מַלְכֵּֽנוּ סְלַח וּמְחַל לְכׇל־עֲוֹנוֹתֵֽינוּ׃\hfill \break
	אָבִֽינוּ מַלְכֵּֽנוּ מְחֵה וְהַעֲבֵר פְּשָׁעֵֽינוּ וְחַטֹּאתֵֽינוּ מִנֶּגֶד עֵינֶֽיךָ׃\hfill \break
	אָבִֽינוּ מַלְכֵּֽנוּ מְחוֹק בְּרַחֲמֶיךָ הָרַבִּים כׇּל־שִׁטְרֵי חוֹבוֹתֵֽינוּ׃\hfill \break
	אָבִֽינוּ מַלְכֵּֽנוּ הַחֲזִירֵֽנוּ בִּתְשׁוּבָה שְׁלֵמָה לְפָנֶיךָ׃\hfill \break
	אָבִֽינוּ מַלְכֵּֽנוּ שְׁלַח רְפוּאָה שְׁלֵמָה לְחוֹלֵי עַמֶּךָ׃\hfill \break
	אָבִֽינוּ מַלְכֵּֽנוּ קְרַע רֽוֹעַ גְּזַר דִּינֵֽנוּ׃\hfill \break
	אָבִֽינוּ מַלְכֵּֽנוּ זׇכְרֵֽנוּ בְּזִכְרוׂן טוֹב לְפָנֶיךָ׃\hfill \break
	\begin{longtable}{>{\centering\arraybackslash}m{.48\textwidth} | >{\centering\arraybackslash}m{.48\textwidth}}
		
		\instruction{בעשי״ת:} & \instruction{בת״צ:}\\
		אָבִֽינוּ מַלְכֵּֽנוּ כׇּתְבֵֽנוּ בְּסֵפֶר חַיִּים טוֹבִים׃ & אָבִֽינוּ מַלְכֵּֽנוּ זׇכְרֵֽנוּ בְּסֵפֶר חַיִּים טוֹבִים׃\\
		אָבִֽינוּ מַלְכֵּֽנוּ כׇּתְבֵֽנוּ בְּסֵפֶר זָכִיּוֹת׃ & אָבִֽינוּ מַלְכֵּֽנוּ זׇכְרֵֽנוּ בְּסֵפֶר זָכִיּוֹת׃\\
		אָבִֽינוּ מַלְכֵּֽנוּ כׇּתְבֵֽנוּ בְּסֵפֶר פַּרְנָסָה וְכַלְכָּלָה׃ & אָבִֽינוּ מַלְכֵּֽנוּ זׇכְרֵֽנוּ בְּסֵפֶר פַּרְנָסָה וְכַלְכָּלָה׃\\
		אָבִֽינוּ מַלְכֵּֽנוּ כׇּתְבֵֽנוּ בְּסֵפֶר גְּאֻלָה וִישׁוּעָה׃ & אָבִֽינוּ מַלְכֵּֽנוּ זׇכְרֵֽנוּ בְּסֵפֶר גְּאֻלָה וִישׁוּעָה׃\\
		אָבִֽינוּ מַלְכֵּֽנוּ כׇּתְבֵֽנוּ בְּסֵפֶר מְחִילָה וּסְלִיחָה׃ & אָבִֽינוּ מַלְכֵּֽנוּ זׇכְרֵֽנוּ בְּסֵפֶר מְחִילָה וּסְלִיחָה׃\\
		%אָבִֽינוּ מַלְכֵּֽנוּ חַדֵּשׁ עָלֵינוּ שָׁנָה טוֹבָה׃ & אָבִֽינוּ מַלְכֵּֽנוּ בָּרֵךְ עָלֵינוּ שָׁנָה טוֹבָה׃
	\end{longtable}
	
	אָבִֽינוּ מַלְכֵּֽנוּ הַצְמַח לָֽנוּ יְשׁוּעָה בְּקָרוֹב׃\hfill \break
	אָבִֽינוּ מַלְכֵּֽנוּ הָרֵם קֶֽרֶן יִשְׂרָאֵל עַמֶּךָ׃\hfill \break
	אָבִֽינוּ מַלְכֵּֽנוּ הָרֵם קֶרֶן מְשִׁיחֶֽךָ׃\hfill \break
	אָבִֽינוּ מַלְכֵּֽנוּ מַלֵּא יָדֵֽינוּ מִבִּרְכוֹתֶֽיךָ׃\hfill \break
	אָבִֽינוּ מַלְכֵּֽנוּ מַלֵּא אֲסָמֵֽינוּ שָׂבָע׃\hfill \break
	אָבִֽינוּ מַלְכֵּֽנוּ שְׁמַע קּוֹלֵֽנוּ חוּס וְרַחֵם עָלֵֽינוּ׃\hfill \break
	אָבִֽינוּ מַלְכֵּֽנוּ קַבֵּל בְּרַחֲמִים וּבְרָצוֹן אֶת־תְּפִלָּתֵֽינוּ׃\hfill \break
	אָבִֽינוּ מַלְכֵּֽנוּ פְּתַח שַׁעֲרֵי שָׁמַֽיִם לִתְפִלָּתֵֽנוּ׃\hfill \break
	אָבִֽינוּ מַלְכֵּֽנוּ זְכוֹר כִּי עָפָר אֲנָֽחְנוּ׃\hfill \break
	אָבִֽינוּ מַלְכֵּֽנוּ נָא אַל תְּשִׁיבֵֽנּוּ רֵיקָם מִלְּפָנֶיךָ׃\hfill \break
	אָבִֽינוּ מַלְכֵּֽנוּ תְּהֵא הַשָּׁעָה הַזֹּאת שְׁעַת רַחֲמִים וְעֵת רָצוֹן מִלְּפָנֶֽיךָ׃\hfill \break
	אָבִֽינוּ מַלְכֵּֽנוּ חֲמוֹל עָלֵֽינוּ וְעַל עוֹלָלֵֽינוּ וְטַפֵּֽנוּ׃\hfill \break
	אָבִֽינוּ מַלְכֵּֽנוּ עֲשֵׂה לְמַעַן הֲרוּגִים עַל־שֵׁם קׇדְשֶׁךָ׃\hfill \break
	אָבִֽינוּ מַלְכֵּֽנוּ עֲשֵׂה לְמַעַן טְבוּחִים עַל־יִחוּדֶֽךָ׃\hfill \break
	אָבִֽינוּ מַלְכֵּֽנוּ עֲשֵׂה לְמַעַן בָּאֵי בָאֵשׁ וּבַמַּיִם עַל־קִּדוּשׁ שְׁמֶךָ׃\hfill \break
	אָבִֽינוּ מַלְכֵּֽנוּ נְקוֹם לְעֵינֵֽינוּ נִקְמַת דַּם עֲבָדֶיךָ הַשָׁפוּךְ׃\hfill \break
	אָבִֽינוּ מַלְכֵּֽנוּ עֲשֵׂה לְמַעַנְךָ אִם־לֹא־לְמַעֲנֵֽנוּ׃\hfill \break
	אָבִֽינוּ מַלְכֵּֽנוּ עֲשֵׂה לְמַעַנְךָ וְהוֹשִׁיעֵֽנוּ׃\hfill \break
	אָבִֽינוּ מַלְכֵּֽנוּ עֲשֵׂה לְמַעַן רַחֲמֶיךָ הָרַבִּים׃\hfill \break
	אָבִֽינוּ מַלְכֵּֽנוּ עֲשֵׂה לְמַעַן שִׁמְךָ הַגָּדוֹל הַגִּבּוֹר וְהַנוֹרָא שֶׁנִקְרָא עָלֵינוּ׃\hfill \break
	אָבִֽינוּ מַלְכֵּֽנוּ חׇנֵּנוּ וַעֲנֵנוּ כִּי אֵין בָּנוּ מַעֲשִׂים עֲשֵׂה עִמָּנוּ צְדָקָה וָחֶסֶד וְהוֹשִׁיעֵנוּ׃
}

\newcommand{\pesicha}{
	\englishinst{The ark is opened.}
	%\instruction{פותחים הארון}\\
	\firstword{וַיְהִ֛י}\source{במדבר י}
	בִּנְסֹ֥עַ הָאָרֹ֖ן וַיֹּ֣אמֶר מֹשֶׁ֑ה קוּמָ֣ה ׀ יְיָ֗ וְיָפֻ֙צוּ֙ אֹֽיְבֶ֔יךָ וְיָנֻ֥סוּ מְשַׂנְאֶ֖יךָ מִפָּנֶֽיךָ׃
	כִּ֤י \source{ישעיה ב}מִצִּיּוֹן֙ תֵּצֵ֣א תוֹרָ֔ה וּדְבַר־יְיָ֖ מִירוּשָׁלָֽ‍ִם׃
	בָּרוּךְ שֶׁנָּתַן תּוֹרָה לְעַמּוֹ יִשְׂרָאֵל בִּקְדֻשָּׁתוֹ׃
}


\newcommand{\brikhshmei}{\begin{small}
		בְּרִיךְ שְׁמֵהּ דְּמָרֵא עָלְמָא בְּרִיךְ כִּתְרָךְ וְאַתְרָךְ \middot יְהֵא רְעוּתָךְ עִם עַמָּךְ יִשְׂרָאֵל לְעָלַם וּפוּרְקַן יְמִינָךְ אַחֲזֵי לְעַמָּךְ בְּבֵית מַקְדְּשָׁךְ \middot וּלְאַמְטוּיֵי לָנָא מִטּוּב נְהוֹרָךְ וּלְקַבֵּל צְלוֹתָנָא בְּרַחֲמִין׃ יְהֵא רַעֲוָא קֳדָמָךְ דְּתוֹרִיךְ לָן חַיִּין בְּטִיבוּתָא \middot וְלֶהֱוֵי אֲנָא פְקִידָא בְּגוֹ צַדִּיקַיָּא לְמִרְחַם עָלַי וּלְמִנְטַר יָתִי וְיַת כׇּל־דִּי לִי וְדִי לְעַמָּךְ יִשְׂרָאֵל׃ אַנְתְּ הוּא זָן לְכֹלָּא וּמְפַרְנֵס לְכֹלָּא אַנְתְּ הוּא שַׁלִּיט עַל כֹּלָּא אַנְתְּ הוּא דְשַׁלִּיט עַל מַלְכַיָּא וּמַלְכוּתָא דִּילָךְ הִיא׃ אֲנָא עַבְדָּא דְקֻדְשָׁא בְּרִיךְ הוּא דְּסָגִידְנָא קַמֵּהּ וּמִקַּמֵּי דִּיקַר אוֹרַיְתֵהּ בְּכׇל־עִדָּן וְעִדָּן \middot לָא עַל אֱנָשׁ רָחִיצְנָא וְלָא עַל בַּר אֱלָהִין סָמִיכְנָא \middot אֶלָּא בֶּאֱלָהָא דִשְׁמַיָּא דְּהוּא אֱלָהָא קְשׁוֹט וְאוֹרַיְתֵהּ קְשׁוֹט וּנְבִיאוֹהִי קְשׁוֹט וּמַסְגֵּא לְמֶעְבַּד טַבְוָן וּקְשׁוֹט׃ בֵּהּ אֲנָא רְחִיץ וְלִשְׁמֵהּ קַדִּישָׁא יַקִּירָא אֲנָא אֵמַר תֻּשְׁבְּחָן \middot יְהֵא רַעֲוָא קֳדָמָךְ דְּתִפְתַּח לִבָּאִי בְּאוֹרַיְתָא וְתַשְׁלִים מִשְׁאֲלִין דְּלִבָּאִי וְלִבָּא דְכׇל־עַמָּךְ יִשְׂרָאֵל \middot לְטַב וּלְחַיִּין וְלִשְלָם. אָמֵן׃
	\end{small}
}

\newcommand{\gadlu}{
	\shatz \begin{large}\textbf{ֽגַּדְּל֣וּ לַייָ֣ אִתִּ֑י וּנְרוֹמְמָ֖ה שְׁמ֣וֹ יַחְדָּֽו׃} \source{תהלים לד}\end{large}
	
	
	\instruction{כולם׃}
	\firstword{לְךָ֣ יְ֠יָ֠}\source{דה״א כט}
	הַגְּדֻלָּ֨ה וְהַגְּבוּרָ֤ה וְהַתִּפְאֶ֙רֶת֙ וְהַנֵּ֣צַח וְהַה֔וֹד כִּי־כֹ֖ל בַּשָּׁמַ֣יִם וּבָאָ֑רֶץ לְךָ֤ יְיָ֙ הַמַּמְלָכָ֔ה וְהַמִּתְנַשֵּׂ֖א לְכֹ֥ל ׀ לְרֹֽאשׁ׃
	רוֹמְמ֡וּ\source{תהלים צט} יְ֘יָ֤ אֱלֹהֵ֗ינוּ וְֽ֭הִשְׁתַּחֲווּ לַהֲדֹ֥ם רַגְלָ֗יו קָד֥וֹשׁ הֽוּא׃
	רוֹמְמ֡וּ יְ֘יָ֤ אֱלֹהֵ֗ינוּ וְֽ֭הִשְׁתַּחֲווּ לְהַ֣ר קׇדְשׁ֑וֹ כִּי־קָ֝ד֗וֹשׁ יְיָ֥ אֱלֹהֵֽינוּ׃
}
\newcommand{\avharachamim}{
	\firstword{אַב הָרַחֲמִים}
	הוּא יְרַחֵם עַם עֲמוּסִים וְיִזְכּוֹר בְּרִית אֵיתָנִים וְיַצִּיל נַפְשׁוֹתֵֽינוּ מִן הַשָּׁעוֹת הָרָעוֹת וְיִגְעַר בְּיֵֽצֶר הָרַע מִן הַנְּשׂוּאִים וְיָחוֹן עָלֵֽינוּ לִפְלֵיטַת עוֹלָמִים וִימַלֵּא מִשְׁאֲלוֹתֵֽינוּ בְּמִדָּה טוֹבָה יְשׁוּעָה וְרַחֲמִים׃
}


\newcommand{\vesigale}{
	\englishinst{As the Torah is placed on the reading desk:}
	\instruction{גבאי:}
	\firstword{וְתִגָּלֶה}
	וְתֵרָאֶה מַלְכוּתוֹ עָלֵֽינוּ בִּזְמַן קָרוֹב וְיָחֹן פְּלֵטָתֵֽנוּ וּפְלֵטַת עַמּוֹ בֵּית יִשְׂרָאֵל לְחֵן וּלְחֶֽסֶד וּלְרַחֲמִים וּלְרָצוֹן׃ וְנֹאמַר אָמֵן׃
	הַכֹּל הָבוּ גוֹדֶל לֵאלֹהֵֽינוּ וּתְנוּ כָבוֹד לַתּוֹרָה׃ כֹּהֵן קְרָב יַעֲמוֹד \ תַּעֲמוֹד \instruction{(פב״פ)} הַכֹּהֵן.
	(\instruction{עם אין כהן׃ }
		אֵין כַּאן כֹּהֵן, יַעֲמוֹד \ תַּעֲמוֹד 
		\instruction{(פב״פ)}
		בִּמְקוׂם כֹּהֵן)
 בָּרוּךְ שֶׁנָּתַן תּוֹרָה לְעַמּוֹ יִשְׂרָאֵל בִּקְדֻשָּׁתוֹ׃
	\instruction{קהל ואח״כ הגבאי׃}
	\textbf{וְאַתֶּם֙ הַדְּבֵקִ֔ים בַּייָ֖ אֱלֹהֵיכֶ֑ם חַיִּ֥ים כֻּלְּכֶ֖ם הַיּֽוֹם׃} \source{דברים ד}
}



\newcommand{\torahbarachu}{
	\instruction{עולה:}
	\begin{large}
		\firstword{בָּרְכוּ אֶת־יְיָ הַמְבֹרָךְ׃}\\
		\instruction{עולה וקהל:}\firstword{בָּרוּךְ יְיָ הַמְבֹרָךְ לְעוֹלָם וָעֶד:}\\
	\end{large}
	\instruction{עולה:}
	בָּרוּךְ אַתָּה יְיָ אֱלֹהֵֽינוּ מֶֽלֶךְ הָעוֹלָם אֲשֶׁר בָּֽחַר בָּֽנוּ מִכׇּל־הָעַמִּים
	וְנָֽתַן לָֽנוּ אֶת־תּוֹרָתוֹ׃ בָּרוּךְ אַתָּה יְיָ נוֹתֵן הַתּוֹרָה׃
	
	\instruction{אחר הקריאה:}\\
	בָּרוּךְ אַתָּה יְיָ אֱלֹהֵֽינוּ מֶֽלֶךְ הָעוֹלָם אֲשֶׁר נָֽתַן לָֽנוּ תּוֹרַת אֱמֶת
	וְחַיֵּי עוֹלָם נָטַע בְּתוֹכֵֽנוּ׃ בָּרוּךְ אַתָּה יְיָ נוֹתֵן הַתּוֹרָה׃
}

\newcommand{\hagomel}{
	\begin{sometimes}
		
		\instruction{ברכת הגומל:}\\
		בָּרוּךְ אַתָּה יְיָ אֱלֹהֵֽינוּ מֶֽלֶךְ הָעוֹלָם הַגּוֹמֵל לְחַיָּבִים טוֹבוֹת שֶׁגְּמָלַֽנִי כׇּל־טוֹב׃\\
		\kahal
		מִי שֶׁגְּמׇלְךָ/שֶׁגְּמָלֵךְ כׇּל־טוֹב הוּא יִגְמׇלְךָ/שֶׁגְּמָלֵךְ כׇּל־טוֹב סֶֽלָה׃
		
\end{sometimes}}

\newcommand{\misheberakholim}[1]{\instruction{מי שברך לעולֶה:}\\
	\firstword{מִי שֶׁבֵּרַךְ}
	אֲבוֹתֵֽינוּ אַבְרָהָם יִצְחָק וְיַעֲקֹב שָׂרָה רִבְקָה רָחֵל וְלֵאָה, הוּא יְבָרֵךְ אֶת
	\instruction{(פב״פ)}
	בַּעֲבוּר שֶׁעָלָה לִכְבוֹד הַמָּקוֹם וְלִכְבוֹד הַתּוֹרָה #1 בִּשְׂכַר זֶה הַקָּדוֹשׁ בָּרוּךְ הוּא יִשְׁמְרֵֽהוּ וְיַצִּילֵֽהוּ מִכׇּל־צָרָה וְצוּקָה וְיִשְׁלַח בְּרָכָה וְהַצְלָחָה בְּכׇל־מַעֲשֵׂה יָדָיו עִם כׇּל־יִשְׂרָאֵל אֶחָיו׃ וְנֹאמַר אָמֵן׃
	
	\instruction{מי שברך לעולָה:}\\
	\firstword{מִי שֶׁבֵּרַךְ}
	אֲבוֹתֵֽינוּ אַבְרָהָם יִצְחָק וְיַעֲקֹב שָׂרָה רִבְקָה רָחֵל וְלֵאָה, הוּא יְבָרֵךְ אֶת
	\instruction{(פב״פ)}
	בַּעֲבוּר שֶׁעָלְתָה לִכְבוֹד הַמָּקוֹם וְלִכְבוֹד הַתּוֹרָה #1 בִּשְׂכַר זֶה הַקָּדוֹשׁ בָּרוּךְ הוּא יִשְׁמְרֶֽהָ וְיַצִּילֶֽהָ מִכׇּל־צָרָה וְצוּקָה, וּמִכׇּל־נֶֽגַע וּמַחֲלָה, וְיִשְׁלַח בְּרָכָה וְהַצְלָחָה בְּכׇל־מַעֲשֵׂה יָדֶֽיהָ עִם כׇּל־יִשְׂרָאֵל אֲחֶֽיהָ׃ וְנֹאמַר אָמֵן׃
	
	\instruction{מי שברך לעולים:}\\
	\firstword{מִי שֶׁבֵּרַךְ}
	אֲבוֹתֵֽינוּ אַבְרָהָם יִצְחָק וְיַעֲקֹב שָׂרָה רִבְקָה רָחֵל וְלֵאָה, הוּא יְבָרֵךְ אֶת
	בַּעֲבוּר שֶׁעָלָה לִכְבוֹד הַמָּקוֹם וְלִכְבוֹד הַתּוֹרָה #1
	בִּשְׂכַר זֶה הַקָּדוֹשׁ בָּרוּךְ הוּא יִשְׁמְרֵֽהוּ וְיַצִּילֵֽהוּ מִכׇּל־צָרָה וְצוּקָה וְיִשְׁלַח בְּרָכָה וְהַצְלָחָה בְּכׇל־מַעֲשֵׂה יָדָיו עִם כׇּל־יִשְׂרָאֵל אֶחָיו׃ וְנֹאמַר אָמֵן׃
}

\vspace{-\baselineskip}
\newcommand{\misheberakhcholimmulti}[1]{
	
	\firstword{מִי שֶׁבֵּרַךְ}
	אֲבוֹתֵינוּ אַבְרָהָם יִצְחָק וְיַעֲקֹב, שָׂרָה רִבְקָה רָחֵל וְלֵאָה, הוּא יְבָרֵךְ וִירַפֵּא
	
	\setcolumnwidth{1.4in,1.4in,1.4in}
	\begin{paracol}{3}
		\instruction{לחולֶה}\\
		אֶת־הַחוֹלֶה \instruction{פלוני בן פלונית} עֲבוּר שֶׁאָנוּ מִתְפַּלְלִים בַּעֲבוּרוֹ. בִּשְׂכַר זֶה, הַקָּדוֹשׁ בָּרוּךְ הוּא יִמָּלֵא רַחֲמִים עָלָיו לְהַחֲלִימוֹ וּלְרַפֹּאתוֹ, לְהַחֲזִיקוֹ וּלְהַחֲיוֹתוֹ, וְיִשְׁלַח לוֹ מְהֵרָה רְפוּאָה שְׁלֵמָה, רְפוּאַת הַנֶּֽפֶשׁ וּרְפוּאַת הַגּוּף
		\switchcolumn
		\instruction{לחולָה}\\
		אֶת־הַחוֹלָה \instruction{פלונית בת פלונית} בַּעֲבוּר שֶׁאָנוּ מִתְפַּלְלִים בַּעֲבוּרָהּ. בִּשְׂכַר זֶה, הַקָּדוֹשׁ בָּרוּךְ הוּא יִמָּלֵא רַחֲמִים עָלֶיהָ לְהַחֲלִימָהּ וּלְרַפֹּאתָהּ, לְהַחֲזִיקָהּ וּלְהַחֲיוֹתָהּ, וְיִשְׁלַח לָהּ מְהֵרָה רְפוּאָה שְׁלֵמָה, רְפוּאַת הַנֶּֽפֶשׁ וּרְפוּאַת הַגּוּף
		\switchcolumn
		\instruction{לחולים}\\
		אֶת־הַחוׂלִים...בַּעֲבוּר שֶׁאָנוּ מִתְפַּלְלִים בַּעֲבוּרָם. בִּשְׂכַר זֶה, הַקָּדוֹשׁ בָּרוּךְ הוּא יִמָּלֵא רַחֲמִים עָלֵיהֶם, לְהַחֲלִימָם וּלְרַפְּאֹתָם וּלְהַחֲזִיקָם וּלְהַחֲיוֹתָם, וְיִשְׁלַח לָהֶם מְהֵרָה רְפוּאָה שְׁלֵמָה מִן הַשָּׁמַיִם לְכׇל־אֵבָרֵֽיהֶם וְגִידֵֽיהֶם,
		
	\end{paracol}
	בְּתוֹךְ שְׁאָר חוֹלֵי יִשְׂרָאֵל, רְפוּאַת הַנֶּֽפֶשׁ וּרְפוּאַת הַגּוּף #1 וּרְפוּאָה קְרוֹבָה לָבוֹא, הַשְׁתָּא בַּעֲגָלָא וּבִזְמַן קָרִיב. וְנֹאמַר: אָמֵן׃
}

\newcommand{\misheberakhcholim}[1]{
\firstword{מִי שֶׁבֵּרַךְ}
אֲבוֹתֵינוּ אַבְרָהָם יִצְחָק וְיַעֲקֹב שָׂרָה רִבְקָה רָחֵל וְלֵאָה \middot הוּא יְבָרֵךְ וִירַפֵּאאֶת־הַחוׂלִים...בַּעֲבוּר שֶׁאָנוּ מִתְפַּלְלִים בַּעֲבוּרָם׃ בִּשְׂכַר זֶה, הַקָּדוֹשׁ בָּרוּךְ הוּא יִמָּלֵא רַחֲמִים עָלֵיהֶם \middot לְהַחֲלִימָם וּלְרַפְּאֹתָם וּלְהַחֲזִיקָם וּלְהַחֲיוֹתָם \middot וְיִשְׁלַח לָהֶם מְהֵרָה רְפוּאָה שְׁלֵמָה מִן הַשָּׁמַיִם לְכׇל־אֵבָרֵֽיהֶם וְגִידֵֽיהֶם בְּתוֹךְ שְׁאָר חוֹלֵי יִשְׂרָאֵל רְפוּאַת הַנֶּֽפֶשׁ וּרְפוּאַת הַגּוּף \middot #1 וּרְפוּאָה קְרוֹבָה לָבוֹא \middot הַשְׁתָּא בַּעֲגָלָא וּבִזְמַן קָרִיב. וְנֹאמַר: אָמֵן׃
}

\newcommand{\misheberakhbaby}{\instruction{מי שבירך ליולדת בן:}\\
	\firstword{מִי שֶׁבֵּרַךְ}
	אֲבוֹתֵֽינוּ אַבְרָהָם יִצְחָק וְיַעֲקֹב, שָׂרָה רִבְקָה רָחֵל וְלֵאָה, הוּא יְבָרֵךְ אֶת־הָאִשָׁה הַיוֹלֶֽדֶת
	\instruction{(פלונית בת פלונית)}
	עִם בְּנָהּ הַנוֹלָד בְּמַזָל טוֹב בַּעֲבוּר שֶׁבַּעֲלָהּ נָדַר...בַּעֲדָם. בִּשְׂכַר זֶה הַקָדוֹשׁ בָּרוּךְ הוּא יְהִי בְּעֶזְרָם וְיִשְׁמְרֵם וִיזַכֶּה אֶת־הָאֵם לְגַדֵל אֶת־בְּנָהּ בַּטוֹב וּבַנְעִימִים וּלְהַדְרִיכוֹ בְּאֹֽרַח מִישׁוֹר לַתּוֹרָה לְחֻפָּה וּלְמַעֲשִׂים טוֹבִים. וְנֹאמַר אָמֵן׃
	
	\instruction{מי שבירך ליולדת בת:}\\
	\firstword{מִי שֶׁבֵּרַךְ}
	אֲבוֹתֵֽינוּ אַבְרָהָם יִצְחָק וְיַעֲקֹב הוּא יְבָרֵךְ אֶת־הָאִשָׁה הַיוֹלֶֽדֶת
	\instruction{(פלונית בת פלונית)}
	עִם בִּתָּהּ הַנוֹלֶֽדֶת בְּמַזָל טוֹב (וְיִקָרֵא שְׁמָהּ בְּיִשְׂרָאֵל...) בַּעֲבוּר שֶׁבַּעֲלָהּ נָדַר...בַּעֲדָן בִּשְׂכַר זֶה הַקָדוֹשׁ בָּרוּךְ הוּא יְהִי בְּעֶזְרָן וְיִשְׁמְרֵן וִיזַכֶּה אֶת־הָאֵם לְגַדֵל אֶת־בִּתָּהּ בַּטוֹב ובַנְעִימִים וּלְהַדְרִיכָה בְּאֹֽרַח מִישׁוֹר לְמִצְוֹת לְחֻפָּה וּלְמַעֲשִׂים טוֹבִים. וְנֹאמַר אָמֵן׃
}

\newcommand{\misheberakhbarmitzva}{\instruction{מי שבירך לבר מצוה:}\\
	\firstword{מִי שֶׁבֵּרַךְ}
	אֲבוֹתֵֽינוּ אַבְרָהָם יִצְחָק וְיַעֲקֹב הוּא יְבָרֵךְ אֶת
	\instruction{(פב״פ)}
	שֶׁהִגִֽיעוּ יָמָיו לִהְיוֹת בַּר מִצְוָה וְעָלָה הַיוֹם לַתּוֹרָה בַּפַּֽעַם הָרִאשׁוֹנָה לָתֵת שֶֽׁבַח וְהוֹדָאָה לְהַשֵׁם יִתְבָּרַךְ עַל כׇּל־הַטוֹבָה אֲשֶׁר עָשָׂה לוֹ (וְנָדַר...) בִּשְׂכַר זֶה הַקָדוֹשׁ בָּרוּךְ הוּא יִשְׁמְרֵֽהוּ וִיחַיֵֽהוּ וִיכוֹנֵן אֶת־לִבּוֹ לִהְיוֹת שָׁלֵם עִם יְיָ וְלָלֶֽכֶת בִּדְרָכָיו וְלִשְׁמֹר מִצְוֹתָיו כׇּל־הַיָמִים וְנֹאמַר אָמֵן׃}
\newcommand{\hagbaha}{
	\instruction{הגבה:}
	\firstword{וְזֹ֖את הַתּוֹרָ֑ה}\source{דברים ד}
	אֲשֶׁר־שָׂ֣ם מֹשֶׁ֔ה לִפְנֵ֖י בְּנֵ֥י יִשְׂרָאֵֽל׃
	עַל־פִּ֥י \source{במדבר ט}יְיָ֖ בְּיַד־מֹשֶֽׁה׃
	עֵץ־חַיִּ֣ים \source{משלי ג} הִ֭יא לַמַּחֲזִיקִ֣ים בָּ֑הּ וְֽתֹמְכֶ֥יהָ מְאֻשָּֽׁר׃
	דְּרָכֶ֥יהָ דַרְכֵי־נֹ֑עַם וְֽכׇל־נְתִ֖יבוֹתֶ֣יהָ שָׁלֽוֹם׃
	אֹ֣רֶךְ יָ֭מִים בִּֽימִינָ֑הּ בִּ֝שְׂמֹאולָ֗הּ עֹ֣שֶׁר וְכָבֽוֹד׃
	יְיָ֥ \source{ישעיה מב}חָפֵ֖ץ לְמַ֣עַן צִדְק֑וֹ יַגְדִּ֥יל תּוֹרָ֖ה וְיַאְדִּֽיר׃
	
}


\newcommand{\yehalelu}{
	\shatz
	\begin{large}
		\textbf{יְהַלְל֤וּ ׀ אֶת־שֵׁ֬ם יְיָ֗ כִּֽי־נִשְׂגָּ֣ב שְׁמ֣וֹ לְבַדּ֑וֹ}\source{תהלים קמח}
	\end{large}
	
	\kahal
	ה֝וֹד֗וֹ עַל־אֶ֥רֶץ וְשָׁמָֽיִם׃ וַיָּ֤רֶם קֶ֨רֶן ׀ לְעַמּ֡וֹ תְּהִלָּ֤ה לְֽכׇל־חֲסִידָ֗יו לִבְנֵ֣י יִ֭שְׂרָאֵל עַ֥ם קְרֹב֗וֹ הַֽלְלוּ־יָֽהּ׃
}


\newcommand{\kafdalet}{
	
	\firstword{לְדָוִ֗ד מִ֫זְמ֥וֹר}\source{תהלים כד}
	לַֽייָ֭ הָאָ֣רֶץ וּמְלוֹאָ֑הּ תֵּ֝בֵ֗ל וְיֹ֣שְׁבֵי בָֽהּ׃
	כִּי־ה֭וּא עַל־יַמִּ֣ים יְסָדָ֑הּ וְעַל־נְ֝הָר֗וֹת יְכוֹנְנֶֽהָ׃
	מִֽי־יַעֲלֶ֥ה בְהַר־יְיָ֑ וּמִי־יָ֝קוּם בִּמְק֥וֹם קׇדְשֽׁוֹ׃
	נְקִ֥י כַפַּ֗יִם וּֽבַר־לֵ֫בָ֥ב אֲשֶׁ֤ר ׀ לֹא־נָשָׂ֣א לַשָּׁ֣וְא נַפְשִׁ֑י וְלֹ֖א נִשְׁבַּ֣ע לְמִרְמָֽה׃
	יִשָּׂ֣א בְ֭רָכָה מֵאֵ֣ת יְיָ֑ וּ֝צְדָקָ֗ה מֵאֱלֹהֵ֥י יִשְׁעֽוֹ׃
	זֶ֭ה דּ֣וֹר דֹּרְשָׁ֑ו מְבַקְשֵׁ֨י פָנֶ֖יךָ יַעֲקֹ֣ב סֶֽלָה׃
	שְׂא֤וּ שְׁעָרִ֨ים ׀ רָֽאשֵׁיכֶ֗ם וְֽ֭הִנָּשְׂאוּ פִּתְחֵ֣י עוֹלָ֑ם וְ֝יָב֗וֹא מֶ֣לֶךְ הַכָּבֽוֹד׃
	מִ֥י זֶה֮ מֶ֤לֶךְ הַכָּ֫ב֥וֹד יְיָ֭ עִזּ֣וּז וְגִבּ֑וֹר יְ֝יָ֗ גִּבּ֥וֹר מִלְחָמָֽה׃
	שְׂא֤וּ שְׁעָרִ֨ים ׀ רָֽאשֵׁיכֶ֗ם וּ֭שְׂאוּ פִּתְחֵ֣י עוֹלָ֑ם וְ֝יָבֹ֗א מֶ֣לֶךְ הַכָּבֽוֹד׃
	מִ֤י ה֣וּא זֶה֮ מֶ֤לֶךְ הַכָּ֫ב֥וֹד יְיָ֥ צְבָא֑וֹת ה֤וּא מֶ֖לֶךְ הַכָּב֣וֹד סֶֽלָה׃
}

\newcommand{\etzchaim}{
	\firstword{וּבְנֻחֹ֖ה יֹאמַ֑ר}\source{במדבר י}
	שׁוּבָ֣ה יְיָ֔ רִֽבְב֖וֹת אַלְפֵ֥י יִשְׂרָאֵֽל׃
	קוּמָ֣ה \source{תהלים קלב}יְ֖יָ לִמְנוּחָתֶ֑ךָ אַ֝תָּ֗ה וַאֲר֥וֹן עֻזֶּֽךָ׃
	כֹּהֲנֶ֥יךָ יִלְבְּשׁוּ־צֶ֑דֶק וַחֲסִידֶ֥יךָ יְרַנֵּֽנוּ׃
	בַּֽ֭עֲבוּר דָּוִ֣ד עַבְדֶּ֑ךָ אַל־תָּ֝שֵׁ֗ב פְּנֵ֣י מְשִׁיחֶֽךָ׃
	כִּ֤י \source{משלי ד}לֶ֣קַח ט֭וֹב נָתַ֣תִּי לָכֶ֑ם תּ֝וֹרָתִ֗י אַֽל־תַּעֲזֹֽבוּ׃
	עֵץ־חַיִּ֣ים \source{משלי ג}הִ֭יא לַמַּחֲזִיקִ֣ים בָּ֑הּ וְֽתֹמְכֶ֥יהָ מְאֻשָּֽׁר׃
	דְּרָכֶ֥יהָ דַרְכֵי־נֹ֑עַם וְֽכׇל־נְתִ֖יבוֹתֶ֣יהָ שָׁלֽוֹם׃
	הֲשִׁיבֵ֨נוּ \source{איכה ה}יְיָ֤ ׀ אֵלֶ֙יךָ֙ וְֽנָשׁ֔וּבָ חַדֵּ֥שׁ יָמֵ֖ינוּ כְּקֶֽדֶם׃
}

\newcommand{\kedushadesidra}{
	וְאַתָּ֥ה
	\source{תהלים כב}%
	קָד֑וֹשׁ י֝וֹשֵׁ֗ב תְּהִלּ֥וֹת יִשְׂרָאֵֽל׃
	\source{ישעיה ו}%
	וְקָרָ֨א זֶ֤ה אֶל־זֶה֙ וְאָמַ֔ר\\
	\kahal
	\kadoshbase\\
	וּמְקַבְּלִין דֵּין מִן דֵּין וְאָמְרִין׃ קַדִּישׁ בִּשְׁמֵי מְרוֹמָא עִלָּאָה בֵּית שְׁכִינְתֵּהּ קַדִּישׁ עַל אַרְעָא עוֹבַד גְּבוּרְתֵּהּ קַדִּישׁ לְעָלַם וּלְעָלְמֵי עָלְמַיָּא יְיָ צְבָאוֹת מַלְיָא כׇל־אַרְעָא זִיו יְקָרֵהּ׃
	
	וַתִּשָּׂאֵ֣נִי ר֔וּחַ וָאֶשְׁמַ֣ע אַחֲרַ֔י ק֖וֹל רַ֣עַשׁ גָּד֑וֹל\\
	\kahal
	\barukhhashemsource\\
	וּנְטָלַֽתְנִי רוּחָא וּשְׁמָעֵת בָּתְרַי קָל זִֽיעַ סַגִּיא דִּמְשַׁבְּחִין וְאָמְרִין׃ בְּרִיךְ יְקָרָא דַיְיָ מֵאֲתַר בֵּית שְׁכִינְתֵּהּ׃\\
	\kahal
	\hashemyimloch\\
	יְיָ מַלְכוּתֵהּ [קָאֵם] לְעָלַם וּלְעַלְמֵי עָלְמַיָּא׃ יְיָ֗
	\source{דה״א כט}%
	אֱ֠לֹהֵ֠י אַבְרָהָ֞ם יִצְחָ֤ק וְיִשְׂרָאֵל֙ אֲבֹתֵ֔ינוּ שׇׁמְרָה־זֹּ֣את לְעוֹלָ֔ם לְיֵ֥צֶר מַחְשְׁב֖וֹת לְבַ֣ב עַמֶּ֑ךָ וְהָכֵ֥ן לְבָבָ֖ם אֵלֶֽיךָ׃ וְה֤וּא
	\source{תהלים עח}%
	רַח֨וּם ׀ יְכַפֵּ֥ר עָוֺן֮ וְֽלֹא־יַֽ֫שְׁחִ֥ית וְ֭הִרְבָּה לְהָשִׁ֣יב אַפּ֑וֹ וְלֹא־יָ֝עִ֗יר כׇּל־חֲמָתֽוֹ׃ כִּֽי־אַתָּ֣ה
	\source{תהלים פו}%
	אֲ֭דֹנָי ט֣וֹב וְסַלָּ֑ח וְרַב־חֶ֗֝סֶד לְכׇל־קֹֽרְאֶֽיךָ׃ צִדְקָתְךָ֣
	\source{תהלים קיט}%
	צֶ֣דֶק לְעוֹלָ֑ם וְֽתוֹרָתְךָ֥ אֱמֶֽת׃ תִּתֵּ֤ן
	\source{מיכה ז}%
	אֱמֶת֙ לְיַֽעֲקֹ֔ב חֶ֖סֶד לְאַבְרָהָ֑ם אֲשֶׁר־נִשְׁבַּ֥עְתָּ לַאֲבֹתֵ֖ינוּ מִ֥ימֵי קֶֽדֶם׃ בָּ֤ר֣וּךְ
	\source{תהלים סח}%
	אֲדֹנָי֮ י֤וֹם ׀ י֥֫וֹם יַעֲמׇס־לָ֗נוּ הָ֘אֵ֤ל יְֽשׁוּעָתֵ֬נוּ סֶֽלָה׃ יְיָ֣
	\source{תהלים מו}%
	צְבָא֣וֹת עִמָּ֑נוּ מִשְׂגָּֽב־לָ֨נוּ אֱלֹהֵ֖י יַֽעֲקֹ֣ב סֶֽלָה׃ יְיָ֥
	\source{תהלים פד}%
	צְבָא֑וֹת אַֽשְׁרֵ֥י אָ֝דָ֗ם בֹּטֵ֥חַ בָּֽךְ׃ יְיָ֥
	\source{תהלים כ}%
	הוֹשִׁ֑יעָה הַ֝מֶּ֗לֶךְ יַעֲנֵ֥נוּ בְיוֹם־קׇרְאֵֽנוּ׃ \\
	בָּרוּךְ הוּא אֱלֹהֵֽינוּ שֶׁבְּרָאָֽנוּ לִכְבוֹדוֹ וְהִבְדִּילָֽנוּ מִן הַתּוֹעִים וְנָֽתַן לָֽנוּ תּוֹרַת אֱמֶת וְחַיֵּי עוֹלָם נָטַע בְּתוֹכֵֽנוּ הוּא יִפְתַּח לִבֵּֽנוּ בְּתוֹרָתוֹ וְיָשֵׂם בְּלִבֵּֽנוּ אַהֲבָתוֹ וְיִרְאָתוֹ וְלַעֲשׂוֹת רְצוֹנוֹ וּלְעׇבְדוֹ בְלֵבָב שָׁלֵם לְמַֽעַן לֹא נִיגַע לָרִיק וְלֹא נֵלֵד לַבֶּהָלָה׃\\
	יְהִי רָצוֹן מִלְּפָנֶֽיךָ יְיָ אֱלֹהֵֽינוּ וֵאלֹהֵי אֲבוֹתֵֽינוּ שֶׁנִּשְׁמוֹר חֻקֶּֽיךָ בָּעוֹלָם הַזֶּה וְנִזְכֶּה וְנִחְיֶה וְנִרְאֶה וְנִירַשׁ טוֹבָה וּבְרָכָה לִשְׁנֵי יְמוֹת הַמָּשִֽׁיחַ וּלְחַיֵּי הָעוֹלָם הַבָּא׃ לְמַ֤עַן\source{תהלים ל} יְזַמֶּרְךָ֣ כָ֭בוֹד וְלֹ֣א יִדֹּ֑ם יְיָ֥ אֱ֝לֹהַ֗י לְעוֹלָ֣ם אוֹדֶֽךָּ׃ בָּר֣וּךְ\source{ירמיה יז} הַגֶּ֔בֶר אֲשֶׁ֥ר יִבְטַ֖ח בַּייָ֑ וְהָיָ֥ה יְיָ֖ מִבְטַחֽוֹ׃ בִּטְח֥וּ
	\source{ישעיה כו}%
	בַֽייָ֖ עֲדֵי־עַ֑ד כִּ֚י בְּיָ֣הּ יְיָ֔ צ֖וּר עוֹלָמִֽים׃ וְיִבְטְח֣וּ
	\source{תהלים ט}%
	בְ֭ךָ יוֹדְעֵ֣י שְׁמֶ֑ךָ כִּ֤י לֹֽא־עָזַ֖בְתָּ דֹרְשֶׁ֣יךָ יְיָ׃ יְיָ֥
	\source{ישעיה מב}%
	חָפֵ֖ץ לְמַ֣עַן צִדְק֑וֹ יַגְדִּ֥יל תּוֹרָ֖ה וְיַאְדִּֽיר׃
}


\newcommand{\uvaletzion}{
	\firstword{וּבָ֤א לְצִיּוֹן֙}\source{ישעיה נט}
	גּוֹאֵ֔ל וּלְשָׁבֵ֥י פֶ֖שַׁע בְּיַֽעֲקֹ֑ב נְאֻ֖ם יְיָ׃ וַאֲנִ֗י זֹ֣את בְּרִיתִ֤י אוֹתָם֙ אָמַ֣ר יְיָ֔ רוּחִי֙ אֲשֶׁ֣ר עָלֶ֔יךָ וּדְבָרַ֖י אֲשֶׁר־שַׂ֣מְתִּי בְּפִ֑יךָ לֹֽא־יָמ֡וּשׁוּ מִפִּ֩יךָ֩ וּמִפִּ֨י זַרְעֲךָ֜ וּמִפִּ֨י זֶ֤רַע זַרְעֲךָ֙ אָמַ֣ר יְיָ֔ מֵעַתָּ֖ה וְעַד־עוֹלָֽם׃
	\kedushadesidra
}

\newcommand{\shirshelyomintro}[1]{\begin{small}הַיּוֹם יוֹם #1 שֶׁבּוֹ הָיוּ הַלְוִיִּם אוֹמְרִים בְּבֵית־הַמִּקְדָּשׁ׃ \end{small}\vspace{-0.7\baselineskip}}

\newcommand{\weekdayshir}{
	\shirshelyomintro{רִאשׁוֹן בַּשַּׁבָּת}
	\begin{narrow}\kafdalet\end{narrow}
	
	\shirshelyomintro{שֵׁנִי בַּשַּׁבָּת}\begin{narrow}
		\source{תהלים מח}%
		שִׁ֥יר מִ֝זְמ֗וֹר לִבְנֵי־קֹֽרַח׃
		גָּ֘ד֤וֹל יְיָ֣ וּמְהֻלָּ֣ל מְאֹ֑ד בְּעִ֥יר אֱ֝לֹהֵ֗ינוּ הַר־קׇדְשֽׁוֹ׃
		יְפֵ֥ה נוֹף֮ מְשׂ֢וֹשׂ כׇּל־הָ֫אָ֥רֶץ הַר־צִ֭יּוֹן יַרְכְּתֵ֣י צָפ֑וֹן קִ֝רְיַ֗ת מֶ֣לֶךְ רָֽב׃
		אֱלֹהִ֥ים בְּאַרְמְנוֹתֶ֗יהָ נוֹדַ֥ע לְמִשְׂגָּֽב׃
		כִּֽי־הִנֵּ֣ה הַ֭מְּלָכִים נ֥וֹעֲד֑וּ עָבְר֥וּ יַחְדָּֽו׃
		הֵ֣מָּה רָ֭אוּ כֵּ֣ן תָּמָ֑הוּ נִבְהֲל֥וּ נֶחְפָּֽזוּ׃
		רְ֭עָדָה אֲחָזָ֣תַם שָׁ֑ם חִ֗֝יל כַּיּוֹלֵדָֽה׃
		בְּר֥וּחַ קָדִ֑ים תְּ֝שַׁבֵּ֗ר אֳנִיּ֥וֹת תַּרְשִֽׁישׁ׃
		כַּאֲשֶׁ֤ר שָׁמַ֨עְנוּ ׀ כֵּ֤ן רָאִ֗ינוּ בְּעִיר־יְיָ֣ צְ֭בָאוֹת בְּעִ֣יר אֱלֹהֵ֑ינוּ אֱלֹ֘הִ֤ים יְכוֹנְנֶ֖הָ עַד־עוֹלָ֣ם סֶֽלָה׃
		דִּמִּ֣ינוּ אֱלֹהִ֣ים חַסְדֶּ֑ךָ בְּ֝קֶ֗רֶב הֵיכָלֶֽךָ׃
		כְּשִׁמְךָ֤ אֱלֹהִ֗ים כֵּ֣ן תְּ֭הִלָּתְךָ עַל־קַצְוֵי־אֶ֑רֶץ צֶ֗֝דֶק מָלְאָ֥ה יְמִינֶֽךָ׃
		יִשְׂמַ֤ח ׀ הַר־צִיּ֗וֹן תָּ֭גֵלְנָה בְּנ֣וֹת יְהוּדָ֑ה לְ֝מַ֗עַן מִשְׁפָּטֶֽיךָ׃
		סֹ֣בּוּ צִ֭יּוֹן וְהַקִּיפ֑וּהָ סִ֝פְר֗וּ מִגְדָּלֶֽיהָ׃
		שִׁ֤יתוּ לִבְּכֶ֨ם ׀ לְֽחֵילָ֗ה פַּסְּג֥וּ אַרְמְנוֹתֶ֑יהָ לְמַ֥עַן תְּ֝סַפְּר֗וּ לְד֣וֹר אַֽחֲרֽוֹן׃
		כִּ֤י זֶ֨ה ׀ אֱלֹהִ֣ים אֱ֭לֹהֵינוּ עוֹלָ֣ם וָעֶ֑ד ה֖וּא יְנַהֲגֵ֣נוּ עַל־מֽוּת׃
	\end{narrow}
	
	\enlargethispage{\baselineskip}
	\shirshelyomintro{שְׁלִישִׁי בַּשַּׁבָּת}
	\begin{narrow}
		\source{תהלים פב}%
		מִזְמ֗וֹר לְאָ֫סָ֥ף אֱֽלֹהִ֗ים נִצָּ֥ב בַּעֲדַת־אֵ֑ל בְּקֶ֖רֶב אֱלֹהִ֣ים יִשְׁפֹּֽט׃
		עַד־מָתַ֥י תִּשְׁפְּטוּ־עָ֑וֶל וּפְנֵ֥י רְ֝שָׁעִ֗ים תִּשְׂאוּ־סֶֽלָה׃
		שִׁפְטוּ־דַ֥ל וְיָת֑וֹם עָנִ֖י וָרָ֣שׁ הַצְדִּֽיקוּ׃
		פַּלְּטוּ־דַ֥ל וְאֶבְי֑וֹן מִיַּ֖ד רְשָׁעִ֣ים הַצִּֽילוּ׃
		לֹ֤א יָדְע֨וּ ׀ וְלֹ֥א יָבִ֗ינוּ בַּחֲשֵׁכָ֥ה יִתְהַלָּ֑כוּ יִ֝מּ֗וֹטוּ כׇּל־מ֥וֹסְדֵי אָֽרֶץ׃
		אֲֽנִי־אָ֭מַרְתִּי אֱלֹהִ֣ים אַתֶּ֑ם וּבְנֵ֖י עֶלְי֣וֹן כֻּלְּכֶֽם׃
		אָ֭כֵן כְּאָדָ֣ם תְּמוּת֑וּן וּכְאַחַ֖ד הַשָּׂרִ֣ים תִּפֹּֽלוּ׃
		קוּמָ֣ה אֱ֭לֹהִים שׇׁפְטָ֣ה הָאָ֑רֶץ כִּֽי־אַתָּ֥ה תִ֝נְחַ֗ל בְּכׇל־הַגּוֹיִֽם׃
		
	\end{narrow}
	
	\shirshelyomintro{רְבִיעִי בַּשַּׁבָּת}
	\begin{narrow}
		\source{תהלים צד}%
		אֵל־נְקָמ֥וֹת יְיָ֑ אֵ֖ל נְקָמ֣וֹת הוֹפִֽיעַ׃
		הִ֭נָּשֵׂא שֹׁפֵ֣ט הָאָ֑רֶץ הָשֵׁ֥ב גְּ֝מ֗וּל עַל־גֵּאִֽים׃
		עַד־מָתַ֖י רְשָׁעִ֥ים ׀ יְיָ֑ עַד־מָ֝תַ֗י רְשָׁעִ֥ים יַעֲלֹֽזוּ׃
		יַבִּ֣יעוּ יְדַבְּר֣וּ עָתָ֑ק יִ֝תְאַמְּר֗וּ כׇּל־פֹּ֥עֲלֵי אָֽוֶן׃
		עַמְּךָ֣ יְיָ֣ יְדַכְּא֑וּ וְֽנַחֲלָתְךָ֥ יְעַנּֽוּ׃
		אַלְמָנָ֣ה וְגֵ֣ר יַהֲרֹ֑גוּ וִ֖יתוֹמִ֣ים יְרַצֵּֽחוּ׃
		וַ֭יֹּ֣אמְרוּ לֹ֣א יִרְאֶה־יָּ֑הּ וְלֹא־יָ֝בִ֗ין אֱלֹהֵ֥י יַעֲקֹֽב׃
		בִּ֭ינוּ בֹּעֲרִ֣ים בָּעָ֑ם וּ֝כְסִילִ֗ים מָתַ֥י תַּשְׂכִּֽילוּ׃
		הֲנֹ֣טַֽע אֹ֭זֶן הֲלֹ֣א יִשְׁמָ֑ע אִֽם־יֹ֥צֵֽר עַ֗֝יִן הֲלֹ֣א יַבִּֽיט׃
		הֲיֹסֵ֣ר גּ֭וֹיִם הֲלֹ֣א יוֹכִ֑יחַ הַֽמְלַמֵּ֖ד אָדָ֣ם דָּֽעַת׃
		יְיָ֗ יֹ֭דֵעַ מַחְשְׁב֣וֹת אָדָ֑ם כִּי־הֵ֥מָּה הָֽבֶל׃
		אַשְׁרֵ֤י ׀ הַגֶּ֣בֶר אֲשֶׁר־תְּיַסְּרֶ֣נּוּ יָּ֑הּ וּֽמִתּוֹרָתְךָ֥ תְלַמְּדֶֽנּוּ׃
		לְהַשְׁקִ֣יט ל֭וֹ מִ֣ימֵי רָ֑ע עַ֤ד יִכָּרֶ֖ה לָרָשָׁ֣ע שָֽׁחַת׃
		כִּ֤י ׀ לֹא־יִטֹּ֣שׁ יְיָ֣ עַמּ֑וֹ וְ֝נַחֲלָת֗וֹ לֹ֣א יַעֲזֹֽב׃
		כִּֽי־עַד־צֶ֭דֶק יָשׁ֣וּב מִשְׁפָּ֑ט וְ֝אַחֲרָ֗יו כׇּל־יִשְׁרֵי־לֵֽב׃
		מִֽי־יָק֣וּם לִ֭י עִם־מְרֵעִ֑ים מִי־יִתְיַצֵּ֥ב לִ֗֝י עִם־פֹּ֥עֲלֵי אָֽוֶן׃
		לוּלֵ֣י יְיָ֭ עֶזְרָ֣תָה לִּ֑י כִּמְעַ֓ט ׀ שָׁכְנָ֖ה דוּמָ֣ה נַפְשִֽׁי׃
		אִם־אָ֭מַרְתִּי מָ֣טָה רַגְלִ֑י חַסְדְּךָ֥ יְ֝יָ֗ יִסְעָדֵֽנִי׃
		בְּרֹ֣ב שַׂרְעַפַּ֣י בְּקִרְבִּ֑י תַּ֝נְחוּמֶ֗יךָ יְֽשַׁעַשְׁע֥וּ נַפְשִֽׁי׃
		הַֽ֭יְחׇבְרְךָ כִּסֵּ֣א הַוּ֑וֹת יֹצֵ֖ר עָמָ֣ל עֲלֵי־חֹֽק׃
		יָ֭גוֹדּוּ עַל־נֶ֣פֶשׁ צַדִּ֑יק וְדָ֖ם נָקִ֣י יַרְשִֽׁיעוּ׃
		וַיְהִ֬י יְיָ֣ לִ֣י לְמִשְׂגָּ֑ב וֵ֝אלֹהַ֗י לְצ֣וּר מַחְסִֽי׃
		וַיָּ֤שֶׁב עֲלֵיהֶ֨ם ׀ אֶת־אוֹנָ֗ם וּבְרָעָתָ֥ם יַצְמִיתֵ֑ם יַ֝צְמִיתֵ֗ם יְיָ֥ אֱלֹהֵֽינוּ׃\\
		\source{תהלים צה}%
		לְ֭כוּ נְרַנְּנָ֣ה לַייָ֑ נָ֝רִ֗יעָה לְצ֣וּר יִשְׁעֵֽנוּ׃
		נְקַדְּמָ֣ה פָנָ֣יו בְּתוֹדָ֑ה בִּ֝זְמִר֗וֹת נָרִ֥יעַֽ לֽוֹ׃
		כִּ֤י אֵ֣ל גָּד֣וֹל יְיָ֑ וּמֶ֥לֶךְ גָּ֝ד֗וֹל עַל־כׇּל־אֱלֹהִֽים׃
	\end{narrow}
	
	\shirshelyomintro{חַמִישִׁי בַּשַּׁבָּת}
	\begin{narrow}
		\source{תהלים פא}%
		לַמְנַצֵּ֬חַ ׀ עַֽל־הַגִּתִּ֬ית לְאָסָֽף׃
		הַ֭רְנִינוּ לֵאלֹהִ֣ים עוּזֵּ֑נוּ הָ֝רִ֗יעוּ לֵאלֹהֵ֥י יַעֲקֹֽב׃
		שְֽׂאוּ־זִ֭מְרָה וּתְנוּ־תֹ֑ף כִּנּ֖וֹר נָעִ֣ים עִם־נָֽבֶל׃
		תִּקְע֣וּ בַחֹ֣דֶשׁ שׁוֹפָ֑ר בַּ֝כֵּ֗סֶה לְי֣וֹם חַגֵּֽנוּ׃
		כִּ֤י חֹ֣ק לְיִשְׂרָאֵ֣ל ה֑וּא מִ֝שְׁפָּ֗ט לֵאלֹהֵ֥י יַעֲקֹֽב׃
		עֵ֤דוּת ׀ בִּיה֘וֹסֵ֤ף שָׂמ֗וֹ בְּ֭צֵאתוֹ עַל־אֶ֣רֶץ מִצְרָ֑יִם שְׂפַ֖ת לֹא־יָדַ֣עְתִּי אֶשְׁמָֽע׃
		הֲסִיר֣וֹתִי מִסֵּ֣בֶל שִׁכְמ֑וֹ כַּ֝פָּ֗יו מִדּ֥וּד תַּעֲבֹֽרְנָה׃
		בַּצָּרָ֥ה קָרָ֗אתָ וָאֲחַ֫לְּצֶ֥ךָּ אֶ֭עֶנְךָ בְּסֵ֣תֶר רַ֑עַם אֶבְחׇנְךָ֨ עַל־מֵ֖י מְרִיבָ֣ה סֶֽלָה׃
		שְׁמַ֣ע עַ֭מִּי וְאָעִ֣ידָה בָּ֑ךְ יִ֝שְׂרָאֵ֗ל אִם־תִּֽשְׁמַֽע־לִֽי׃
		לֹא־יִהְיֶ֣ה בְ֭ךָ אֵ֣ל זָ֑ר וְלֹ֥א תִ֝שְׁתַּחֲוֶ֗ה לְאֵ֣ל נֵכָֽר׃
		אָֽנֹכִ֨י ׀ יְ֘יָ֤ אֱלֹהֶ֗יךָ הַֽ֭מַּעַלְךָ מֵאֶ֣רֶץ מִצְרָ֑יִם הַרְחֶב־פִּ֗֝יךָ וַאֲמַלְאֵֽהוּ׃
		וְלֹֽא־שָׁמַ֣ע עַמִּ֣י לְקוֹלִ֑י וְ֝יִשְׂרָאֵ֗ל לֹא־אָ֥בָה לִֽי׃
		וָ֭אֲשַׁלְּחֵהוּ בִּשְׁרִיר֣וּת לִבָּ֑ם יֵ֝לְכ֗וּ בְּֽמוֹעֲצ֖וֹתֵיהֶֽם׃
		ל֗וּ עַ֭מִּי שֹׁמֵ֣עַֽ לִ֑י יִ֝שְׂרָאֵ֗ל בִּדְרָכַ֥י יְהַלֵּֽכוּ׃
		כִּ֭מְעַט אוֹיְבֵיהֶ֣ם אַכְנִ֑יעַ וְעַ֥ל צָ֝רֵיהֶ֗ם אָשִׁ֥יב יָדִֽי׃
		מְשַׂנְאֵ֣י יְיָ֭ יְכַחֲשׁוּ־ל֑וֹ וִיהִ֖י עִתָּ֣ם לְעוֹלָֽם׃
		וַֽ֭יַּאֲכִילֵהוּ מֵחֵ֣לֶב חִטָּ֑ה וּ֝מִצּ֗וּר דְּבַ֣שׁ אַשְׂבִּיעֶֽךָ׃
	\end{narrow}
	
	\shirshelyomintro{שִׁשִּׁי בַּשַּׁבָּת}
	\begin{narrow}
		\source{תהלים צג}%
		יְיָ֣ מָלָךְ֮ גֵּא֢וּת לָ֫בֵ֥שׁ לָבֵ֣שׁ יְיָ֭ עֹ֣ז הִתְאַזָּ֑ר אַף־תִּכּ֥וֹן תֵּ֝בֵ֗ל בַּל־תִּמּֽוֹט׃
		נָכ֣וֹן כִּסְאֲךָ֣ מֵאָ֑ז מֵעוֹלָ֣ם אָֽתָּה׃
		נָשְׂא֤וּ נְהָר֨וֹת ׀ יְיָ֗ נָשְׂא֣וּ נְהָר֣וֹת קוֹלָ֑ם יִשְׂא֖וּ נְהָר֣וֹת דׇּכְיָֽם׃
		מִקֹּל֨וֹת ׀ מַ֤יִם רַבִּ֗ים אַדִּירִ֣ים מִשְׁבְּרֵי־יָ֑ם אַדִּ֖יר בַּמָּר֣וֹם יְיָ׃
		עֵֽדֹתֶ֨יךָ ׀ נֶאֶמְנ֬וּ מְאֹ֗ד לְבֵיתְךָ֥ נַאֲוָה־קֹ֑דֶשׁ יְ֝יָ֗ לְאֹ֣רֶךְ יָמִֽים׃
	\end{narrow}
}

\newcommand{\RChBarekhi}{
	\instruction{בראש חדש׃}\space
	\firstword{בָּרְכִ֥י נַפְשִׁ֗י } \source{תהלים קד}
	אֶת־יְ֫יָ֥ יְיָ֣ אֱ֭לֹהַי גָּדַ֣לְתָּ מְּאֹ֑ד ה֖וֹד וְהָדָ֣ר לָבָֽשְׁתָּ׃
	עֹֽטֶה־א֭וֹר כַּשַּׂלְמָ֑ה נוֹטֶ֥ה שָׁ֝מַ֗יִם כַּיְרִיעָֽה׃
	הַ֥מְקָרֶ֥ה בַמַּ֗יִם עֲֽלִיּ֫וֹתָ֥יו הַשָּׂם־עָבִ֥ים רְכוּב֑וֹ הַֽ֝מְהַלֵּ֗ךְ עַל־כַּנְפֵי־רֽוּחַ׃
	עֹשֶׂ֣ה מַלְאָכָ֣יו רוּח֑וֹת מְ֝שָׁרְתָ֗יו אֵ֣שׁ לֹהֵֽט׃
	יָֽסַד־אֶ֭רֶץ עַל־מְכוֹנֶ֑יהָ בַּל־תִּ֝מּ֗וֹט עוֹלָ֥ם וָעֶֽד׃
	תְּ֭הוֹם כַּלְּב֣וּשׁ כִּסִּית֑וֹ עַל־הָ֝רִ֗ים יַ֖עַמְדוּ מָֽיִם׃
	מִן־גַּעֲרָ֣תְךָ֣ יְנוּס֑וּן מִן־ק֥וֹל רַֽ֝עַמְךָ֗ יֵחָפֵזֽוּן׃
	יַעֲל֣וּ הָ֭רִים יֵרְד֣וּ בְקָע֑וֹת אֶל־מְ֝ק֗וֹם זֶ֤ה ׀ יָסַ֬דְתָּ לָהֶֽם׃
	גְּֽבוּל־שַׂ֭מְתָּ בַּל־יַעֲבֹר֑וּן בַּל־יְ֝שֻׁב֗וּן לְכַסּ֥וֹת הָאָֽרֶץ׃
	הַֽמְשַׁלֵּ֣חַ מַ֭עְיָנִים בַּנְּחָלִ֑ים בֵּ֥ין הָ֝רִ֗ים יְהַלֵּכֽוּן׃
	יַ֭שְׁקוּ כׇּל־חַיְת֣וֹ שָׂדָ֑י יִשְׁבְּר֖וּ פְרָאִ֣ים צְמָאָֽם׃
	עֲ֭לֵיהֶם עוֹף־הַשָּׁמַ֣יִם יִשְׁכּ֑וֹן מִבֵּ֥ין עֳ֝פָאיִ֗ם יִתְּנוּ־קֽוֹל׃
	מַשְׁקֶ֣ה הָ֭רִים מֵעֲלִיּוֹתָ֑יו מִפְּרִ֥י מַ֝עֲשֶׂ֗יךָ תִּשְׂבַּ֥ע הָאָֽרֶץ׃
	מַצְמִ֤יחַ חָצִ֨יר ׀ לַבְּהֵמָ֗ה וְ֭עֵשֶׂב לַעֲבֹדַ֣ת הָאָדָ֑ם לְה֥וֹצִיא לֶ֗֝חֶם מִן־הָאָֽרֶץ׃
	וְיַ֤יִן ׀ יְשַׂמַּ֬ח לְֽבַב־אֱנ֗וֹשׁ לְהַצְהִ֣יל פָּנִ֣ים מִשָּׁ֑מֶן וְ֝לֶ֗חֶם לְֽבַב־אֱנ֥וֹשׁ יִסְעָֽד׃
	יִ֭שְׂבְּעוּ עֲצֵ֣י יְיָ֑ אַֽרְזֵ֥י לְ֝בָנ֗וֹן אֲשֶׁ֣ר נָטָֽע׃
	אֲשֶׁר־שָׁ֭ם צִפֳּרִ֣ים יְקַנֵּ֑נוּ חֲ֝סִידָ֗ה בְּרוֹשִׁ֥ים בֵּיתָֽהּ׃
	הָרִ֣ים הַ֭גְּבֹהִים לַיְּעֵלִ֑ים סְ֝לָעִ֗ים מַחְסֶ֥ה לַֽשְׁפַנִּֽים׃
	עָשָׂ֣ה יָ֭רֵחַ לְמוֹעֲדִ֑ים שֶׁ֝֗מֶשׁ יָדַ֥ע מְבוֹאֽוֹ׃
	תָּֽשֶׁת־חֹ֭שֶׁךְ וִ֣יהִי לָ֑יְלָה בּוֹ־תִ֝רְמֹ֗שׂ כׇּל־חַיְתוֹ־יָֽעַר׃
	הַ֭כְּפִירִים שֹׁאֲגִ֣ים לַטָּ֑רֶף וּלְבַקֵּ֖שׁ מֵאֵ֣ל אׇכְלָֽם׃
	תִּזְרַ֣ח הַ֭שֶּׁמֶשׁ יֵאָסֵפ֑וּן וְאֶל־מְ֝עוֹנֹתָ֗ם יִרְבָּצֽוּן׃
	יֵצֵ֣א אָדָ֣ם לְפׇעֳל֑וֹ וְֽלַעֲבֹ֖דָת֣וֹ עֲדֵי־עָֽרֶב׃
	מָה־רַבּ֬וּ מַעֲשֶׂ֨יךָ ׀ יְיָ֗ כֻּ֭לָּם בְּחׇכְמָ֣ה עָשִׂ֑יתָ מָלְאָ֥ה הָ֝אָ֗רֶץ קִנְיָנֶֽךָ׃
	זֶ֤ה ׀ הַיָּ֥ם גָּדוֹל֮ וּרְחַ֢ב יָ֫דָ֥יִם שָֽׁם־רֶ֭מֶשׂ וְאֵ֣ין מִסְפָּ֑ר חַיּ֥וֹת קְ֝טַנּ֗וֹת עִם־גְּדֹלֽוֹת׃
	שָׁ֭ם אֳנִיּ֣וֹת יְהַלֵּכ֑וּן לִ֝וְיָתָ֗ן זֶֽה־יָצַ֥רְתָּ לְשַֽׂחֶק־בּֽוֹ׃
	כֻּ֭לָּם אֵלֶ֣יךָ יְשַׂבֵּר֑וּן לָתֵ֖ת אׇכְלָ֣ם בְּעִתּֽוֹ׃
	תִּתֵּ֣ן לָ֭הֶם יִלְקֹט֑וּן תִּפְתַּ֥ח יָ֝דְךָ֗ יִשְׂבְּע֥וּן טֽוֹב׃
	תַּסְתִּ֥יר פָּנֶיךָ֮ יִֽבָּהֵ֫ל֥וּן תֹּסֵ֣ף ר֭וּחָם יִגְוָע֑וּן וְֽאֶל־עֲפָרָ֥ם יְשׁוּבֽוּן׃
	תְּשַׁלַּ֣ח ר֭וּחֲךָ יִבָּרֵא֑וּן וּ֝תְחַדֵּ֗שׁ פְּנֵ֣י אֲדָמָֽה׃
	יְהִ֤י כְב֣וֹד יְיָ֣ לְעוֹלָ֑ם יִשְׂמַ֖ח יְיָ֣ בְּמַעֲשָֽׂיו׃
	הַמַּבִּ֣יט לָ֭אָרֶץ וַתִּרְעָ֑ד יִגַּ֖ע בֶּהָרִ֣ים וְֽיֶעֱשָֽׁנוּ׃
	אָשִׁ֣ירָה לַייָ֣ בְּחַיָּ֑י אֲזַמְּרָ֖ה לֵאלֹהַ֣י בְּעוֹדִֽי׃
	יֶעֱרַ֣ב עָלָ֣יו שִׂיחִ֑י אָ֝נֹכִ֗י אֶשְׂמַ֥ח בַּייָ׃
	יִתַּ֤מּוּ חַטָּאִ֨ים ׀ מִן־הָאָ֡רֶץ וּרְשָׁעִ֤ים ׀ ע֤וֹד אֵינָ֗ם בָּרְכִ֣י נַ֭פְשִׁי אֶת־יְיָ֗ הַֽלְלוּ־יָֽהּ׃
}

\newcommand{\ledavid}{
	\englishinst{The following Psalm is said from Rosh \d{H}odesh Elul until Hoshana Rabba (some say only until Yom Kippur):}
	\firstword{לְדָוִ֨ד ׀ יְיָ֤ ׀ אוֹרִ֣י וְ֭יִשְׁעִי}\source{תהלים בז}
	מִמִּ֣י אִירָ֑א יְיָ֥ מָעוֹז־חַ֝יַּ֗י מִמִּ֥י אֶפְחָֽד׃
	בִּקְרֹ֤ב עָלַ֨י ׀ מְרֵעִים֮ לֶאֱכֹ֢ל אֶת־בְּשָׂ֫רִ֥י צָרַ֣י וְאֹיְבַ֣י לִ֑י הֵ֖מָּה כָשְׁל֣וּ וְנָפָֽלוּ׃
	אִם־תַּחֲנֶ֬ה עָלַ֨י ׀ מַחֲנֶה֮ לֹא־יִירָ֢א לִ֫בִּ֥י אִם־תָּק֣וּם עָ֭לַי מִלְחָמָ֑ה בְּ֝זֹ֗את אֲנִ֣י בוֹטֵֽחַ׃
	אַחַ֤ת ׀ שָׁאַ֣לְתִּי מֵֽאֵת־יְיָ אוֹתָ֢הּ אֲבַ֫קֵּ֥שׁ שִׁבְתִּ֣י בְּבֵית־יְיָ֭ כׇּל־יְמֵ֣י חַיַּ֑י לַחֲז֥וֹת בְּנֹעַם־יְ֝יָ֗ וּלְבַקֵּ֥ר בְּהֵֽיכָלֽוֹ׃
	כִּ֤י יִצְפְּנֵ֨נִי ׀ בְּסֻכֹּה֮ בְּי֢וֹם רָ֫עָ֥ה יַ֭סְתִּרֵנִי בְּסֵ֣תֶר אׇהֳל֑וֹ בְּ֝צ֗וּר יְרוֹמְמֵֽנִי׃
	וְעַתָּ֨ה יָר֪וּם רֹאשִׁ֡י עַ֤ל אֹיְבַ֬י סְֽבִיבוֹתַ֗י וְאֶזְבְּחָ֣ה בְ֭אׇהֳלוֹ זִבְחֵ֣י תְרוּעָ֑ה אָשִׁ֥ירָה וַ֝אֲזַמְּרָ֗ה לַֽייָ׃
	שְׁמַע־יְיָ֖ קוֹלִ֥י אֶקְרָ֗א וְחׇנֵּ֥נִי וַֽעֲנֵֽנִי׃
	לְךָ֤ ׀ אָמַ֣ר לִ֭בִּי בַּקְּשׁ֣וּ פָנָ֑י אֶת־פָּנֶ֖יךָ יְיָ֣ אֲבַקֵּֽשׁ׃
	אַל־תַּסְתֵּ֬ר פָּנֶ֨יךָ ׀ מִמֶּנִּי֮ אַ֥ל תַּט־בְּאַ֗ף עַ֫בְדֶּ֥ךָ עֶזְרָתִ֥י הָיִ֑יתָ אַֽל־תִּטְּשֵׁ֥נִי וְאַל־תַּ֝עַזְבֵ֗נִי אֱלֹהֵ֥י יִשְׁעִֽי׃
	כִּֽי־אָבִ֣י וְאִמִּ֣י עֲזָב֑וּנִי וַֽייָ֣ יַאַסְפֵֽנִי׃
	ה֤וֹרֵ֥נִי יְיָ֗ דַּ֫רְכֶּ֥ךָ וּ֭נְחֵנִי בְּאֹ֣רַח מִישׁ֑וֹר לְ֝מַ֗עַן שֽׁוֹרְרָֽי׃
	אַֽל־תִּ֭תְּנֵנִי בְּנֶ֣פֶשׁ צָרָ֑י כִּ֥י קָמוּ־בִ֥י עֵדֵי־שֶׁ֝֗קֶר וִיפֵ֥חַ חָמָֽס׃
	לׅׄוּׅׄלֵׅׄ֗אׅׄ הֶ֭אֱמַנְתִּי לִרְא֥וֹת בְּֽטוּב־יְיָ֗ בְּאֶ֣רֶץ חַיִּֽים׃
	קַוֵּ֗ה אֶל־יְ֫יָ֥ חֲ֭זַק וְיַאֲמֵ֣ץ לִבֶּ֑ךָ וְ֝קַוֵּ֗ה אֶל־יְיָ׃
}

\newcommand{\specialsaavos}{
	\begin{small}
		אֲ֭דֹנָי שְׂפָתַ֣י תִּפְתָּ֑ח וּ֝פִ֗י יַגִּ֥יד תְּהִלָּתֶֽךָ׃
		\source{תהלים נא}\\
	\end{small}
	\firstword{בָּרוּךְ}
	אַתָּה יְיָ אֱלֹהֵֽינוּ וֵאלֹהֵי אֲבוֹתֵֽינוּ אֱלֹהֵי אַבְרָהָם אֱלֹהֵי יִצְחָק וֵאלֹהֵי יַעֲקֹב הָאֵל הַגָּדוֹל הַגִּבּוֹר וְהַנּוֹרָא אֵל עֶלְיוֹן גּוֹמֵל חֲסָדִים טוֹבִים וְקוֹנֵה הַכֹּל וְזוֹכֵר חַסְדֵי אָבוֹת וּמֵבִיא גוֹאֵל לִבְנֵי בְנֵיהֶם לְמַֽעַן שְׁמוֹ בְּאַהֲבָה׃ מֶֽלֶךְ עוֹזֵר וּמוֹשִֽׁיעַ וּמָגֵן׃ בָּרוּךְ אַתָּה יְיָ מָגֵן אַבְרָהָם׃
}

\newcommand{\specialsameisim}{
	\firstword{אַתָּה}
	גִּבּוֹר לְעוֹלָם אֲדֹנָי מְחַיֵּה מֵתִים אַתָּה רַב לְהוֹשִֽׁיעַ׃
	
	\englishinst{From Shemini Atzeret till Pesach:}
	%\instruction{ממוסף של שמיני עצרת עד מוסף יום א׳ של פסח אומרים:}\\
	מַשִּׁיב הָרֽוּחַ וּמוֹרִיד הַגָּֽשֶׁם:
	%\footnote{\instruction{נ״א}: הַגֶּֽשֶׁם}
	
	\firstword{מְכַלְכֵּל}
	חַיִּים בְּחֶֽסֶד מְחַיֵּה מֵתִים בְּרַחֲמִים רַבִּים סוֹמֵךְ נוֹפְלִים וְרוֹפֵא חוֹלִים וּמַתִּיר אֲסוּרִים וּמְקַיֵּם אֱמוּנָתוֹ לִישֵׁנֵי עָפָר׃ מִי כָמֽוֹךָ בַּֽעַל גְּבוּרוֹת וּמִי דּֽוֹמֶה לָּךְ מֶֽלֶךְ מֵמִית וּמְחַיֶּה וּמַצְמִֽיחַ יְשׁוּעָה׃ וְנֶאֱמָן אַתָּה לְהַחֲיוֹת מֵתִים׃ בָּרוּךְ אַתָּה יְיָ מְחַיֵּה הַמֵּתִים׃
}

\newcommand{\kedusmusafchol}[2]{
	\ssubsection{\adforn{48} #1 \adforn{22}}
	
	\begin{small}
		%\setlength{\LTpost}{0pt}
		\begin{tabular}{l p{.85\textwidth}}
			
			\shatz &
			נְקַדֵּשׁ אֶת־שִׁמְךָ בָּעוֹלָם כְּשֵׁם שֶׁמַּקְדִּישִׁים אוֹתוֹ בִּשְׁמֵי מָרוֹם כַּכָּתוּב עַל יַד נְבִיאֶךָ קָרָ֨א זֶ֤ה אֶל־זֶה֙ וְאָמַ֔ר׃\\
			
			\shatzvkahal &
			\kadoshkadoshkadosh \\
			
			\shatz &
			לְעֻמָּתָם בָּרוּךְ יֹאמֵרוּ׃\\
			
			\shatzvkahal &
			\barukhhashem\\
			
			\shatz &
			וּבְדִבְרֵי קׇדְשְׁךָ כָּתוּב לֵאמֹר׃ \\
			
			\shatzvkahal &
			\yimloch\\
			
			\shatz &
			לְדוֹר וָדוֹר נַגִּיד גׇּדְלֶךָ וּלְנֵצַח נְצָחִים קְדֻשָּׁתְךָ נַקְדִּישׁ וְשִׁבְחֲךָ אֱלֹהֵֽינוּ מִפִּינוּ לֹא יָמוּשׁ לְעוֹלָם וָעֶד כִּי אֵל מֶלֶךְ גָּדוֹל וְקָדוֹשׁ אַֽתָּה׃ בָּרוּךְ אַתָּה יְיָ הָאֵל הַקָּדוֹשׁ׃ #2\\
		\end{tabular}
		
\end{small}}

\newcommand{\longpesicha}{
אֵין־כָּמ֖וֹךָ\source{תהלים פו} בָאֱלֹהִ֥ים ׀ אֲדֹנָ֗י וְאֵ֣ין כְּֽמַעֲשֶֽׂיךָ׃
מַֽלְכוּתְךָ֗ \source{תהלים קמה}מַלְכ֥וּת כׇּל־עֹלָמִ֑ים וּ֝מֶֽמְשַׁלְתְּךָ֗ בְּכׇל־דּ֥וֹר וָדֹֽר׃
\melekhmalakhyimlokh 
יְיָ֗ \source{תהלים כט}עֹ֭ז לְעַמּ֣וֹ יִתֵּ֑ן יְיָ֓ ׀ יְבָרֵ֖ךְ אֶת־עַמּ֣וֹ בַשָּׁלֽוֹם׃\\
אַב הָרַחֲמִים\source{תהלים נא} הֵיטִ֣יבָה בִ֭רְצוֹנְךָ אֶת־צִיּ֑וֹן תִּ֝בְנֶ֗ה חוֹמ֥וֹת יְרוּשָׁלָֽ‍ִם׃
כִּי בְךָ לְבַד בָּטָֽחְנוּ מֶֽלֶךְ אֵל רָם וְנִשָּׂא אֲדוֹן עוֹלָמִים׃

\pesicha
}

\newcommand{\yekumpurkans}{
\firstword{יְקוּם פֻּרְקָן}
מִן שְׁמַיָּא חִנָּא וְחִסְדָּא וְרַחֲמֵי וְחַיֵּי אֲרִיכֵי וּמְזוֹנֵי רְוִיחֵי וְסִיַּעְתָּא דִשְּׁמַיָּא וּבַרְיוּת גּוּפָא וּנְהוֹרָא מַעַלְיָא׃ זַרְעָא חַיָּא וְקַיָּמָא זַרְעָא דִּי לֹא יִפְסוּק וְדִי לָא יִבְטוּל מִפִּתְגָּמֵי אוֹרַיְתָא׃ לְמָרָנָן וְרַבָּנָן חֲבוּרָתָא קַדִּישָׁתָא דִּי בְאַרְעָא דְיִשְׂרָאֵל וְדִי בְּבָבֶל לְרֵישֵׁי כַלֵּי וּלְרֵישֵׁי גַלְוָתָא וּלְרֵישֵׁי מְתִיבָתָא וּלְדַיָּנֵי דִי בָבָא׃ לְכׇל־תַּלְמִידֵיהוֹן וּלְכׇל־תַּלְמִידֵי תַלְמִידֵיהוֹן וּלְכׇל־מָן דְּעָסְקִין בְּאוֹרַיְתָא׃ מַלְכָּא דְעָלְמָא יְבָרֵךְ יַתְהוֹן יַפִּישׁ חַיֵּיהוֹן וְיַסְגֵּא יוֹמֵיהוֹן וְיִתֵּן אָרְכָה לִשְׁנֵיהוֹן וְיִתְפָּרְקוּן וְיִשְׁתֵּזְבוּן מִן כׇּל־עָקָא וּמִן כׇּל־מַרְעִין בִּישִׁין מָרָן דִּי בִשְׁמַיָּא יְהֵא בְּסַעְדְּהוֹן כׇּל־זְמַן וְעִדָּן׃ וְנֹאמַר אָמֵן׃



\firstword{יְקוּם פֻּרְקָן}
מִן שְׁמַיָּא חִנָּא וְחִסְדָּא וְרַחֲמֵי וְחַיֵּי אֲרִיכֵי וּמְזוֹנֵי רְוִיחֵי וְסִיַּעְתָּא דִּשְׁמַיָּא וּבַרְיוּת גּוּפָא וּנְהוֹרָא מַעַלְיָא׃ זַרְעָא חַיָּא וְקַיָּמָא זַרְעָא דִּי לָא יִפְסוּק וְדִי לָא יִבְטוּל מִפִּתְגָּמֵי אוֹרַיְתָא׃ לְכׇל־קְהָלָא קַדִּישָׁא הָדֵין רַבְרְבַיָּא עִם זְעֵרַיָּא טַפְלָא וּנְשַׁיָּא׃ מַלְכָּא דְעָלְמָא יְבָרֵךְ יָתְכוֹן יַפִּישׁ חַיֵּיכוֹן וְיַסְגֵּא יוֹמֵיכוֹן וְיִתֵּן אָרְכָה לִשְׁנֵיכוֹן וְתִתְפָּרְקוּן וְתִשְׁתֵּזְבוּן מִן כׇּל־עָקָא וּמִן כׇּל־מַרְעִין בִּישִׁין מָרָן דִּי בִשְׁמַיָּא יְהֵא בְּסַעְדְּכוֹן כׇּל־זְמַן וְעִדָּן׃ וְנֹאמַר אָמֵן׃

\firstword{מִי שֶׁבֵּירַךְ}
אֲבוֹתֵֽינוּ אַבְרָהָם יִצְחָק וְיַעֲקֹב הוּא יְבָרֵךְ אֶת־כׇּל־הַקָּהָל הַקָּדוֹשׁ הַזֶּה עִם כׇּל־קְהִילּוֹת הַקּוֹדֶשׁ הֵם וּמִשְׁפְּחוֹתֵיהֶם וְכׇל־אֲשֶׁר לָהֶם׃ וּמִי שֶׁמְּיַחֲדִים בָּתֵּי־כְנֵסִיּוֹת לִתְפִלָּה וּמִי שֶׁבָּאִים בְּתוֹכָם לְהִתְפַּלֵּל וּמִי שֶׁנּוֹתְנִים נֵר לַמָּאוֹר וְיַֽיִן לְקִדּוּשׁ וּלְהַבְדָּלָה וּפַת לְאוֹרְחִים וּצְדָקָה לַעֲנִיִּים וְכׇל־מִי שֶׁעוֹסְקִים בְּצׇרְכֵי צִבּוּר בֶּאֱמוּנָה הַקָּדוֹשׁ בָּרוּךְ הוּא יְשַׁלֵם שְׂכָרָם וְיָסִיר מֵהֶם כׇּל־מַחֲלָה וְיִרְפָּא לְכׇל־גּוּפָם וְיִסְלַח לְכׇל־עֲוֹנָם וְיִשְׁלַח בְּרָכָה וְהַצְלָחָה בְּכׇל־מַעֲשֵׂה יְדֵיהֶם עִם כׇּל־יִשְׂרָאֵל אֲחֵיהֶם וְנֹאמַר אָמֵן׃
}

\newcommand{\shomeryisroel}{
	\englishinst{Sit upright for this paragraph.}
	\textbf{שׁוֹמֵר יִשְׂרָאֵל}
	שְׁמוֹר שְׁאֵרִית יִשְׂרָאֵל וְאַל יֹאבַד יִשְׂרָאֵל הָאוֹמְרִים שְׁמַע יִשְׂרָאֵל׃
	שׁוֹמֵר גּוֹי אֶחָד שְׁמוֹר שְׁאֵרִית עַם אֶחָד וְאַל יֹאבַד גּוֹי אֶחָד
	הַמְיַחֲדִים שִׁמְךָ יְיָ אֱלֹהֵֽינוּ יְיָ אֶחָד׃
	שׁוֹמֵר גּוֹי קָדוֹשׁ שְׁמוֹר שְׁאֵרִית עַם קָדוֹשׁ
	וְאַל יֹאבַד גּוֹי קָדוֹשׁ הַמְשַׁלְּשִׁים בְּשָׁלוֹשׁ קְדֻשּׁוֹת לְקָדוֹשׁ׃
	מִתְרַצֶּה בְּרַחֲמִים וּמִתְפַּיֵּס בְּתַחֲנוּנִים הִתְרַצֶּה וְהִתְפַּיֵּס לְדוֹר עָנִי כִּי אֵין עוֹזֵר׃
	אָבִינוּ מַלְכֵּנוּ חׇנֵּנוּ וַעֲנֵנוּ כִּי אֵין בָּנוּ מַעֲשִׂים עֲשֵׂה עִמָּנוּ צְדָקָה וָחֶסֶד וְהוֹשִׁיעֵנוּ׃\\
	\englishinst{Stand for the recitation of the following paragraph.}
	\firstword{וַאֲנַ֗חְנוּ}\source{דה״ב כ}
	לֹ֤א נֵדַע֙ מַֽה־נַּעֲשֶׂ֔ה כִּ֥י עָלֶ֖יךָ עֵינֵֽינוּ׃
	זְכֹר־רַחֲמֶ֣יךָ
	\source{תהלים כה}%
	יְיָ֭ וַחֲסָדֶ֑יךָ כִּ֖י מֵעוֹלָ֣ם הֵֽמָּה׃
	יְהִי־חַסְדְּךָ֣ \source{תהלים לג}יְיָ֣ עָלֵ֑ינוּ כַּֽ֝אֲשֶׁ֗ר יִחַ֥לְנוּ לָֽךְ׃
	אַֽל־תִּזְכׇּר־לָנוּ֮ \source{תהלים עט}עֲוֺנֹ֢ת רִאשֹׁ֫נִ֥ים מַ֭הֵר יְקַדְּמ֣וּנוּ רַחֲמֶ֑יךָ כִּ֖י דַלּ֣וֹנוּ מְאֹֽד׃
	חׇנֵּ֣נוּ \source{תהלים קכג}יְיָ֣ חׇנֵּ֑נוּ כִּי־רַ֗֝ב שָׂבַ֥עְנוּ בֽוּז׃
	בְּרֹ֖גֶז \source{חבקוק ג}רַחֵ֥ם תִּזְכּֽוֹר׃
	כִּי־ה֭וּא \source{תהלים קג} יָדַ֣ע יִצְרֵ֑נוּ זָ֝כ֗וּר כִּי־עָפָ֥ר אֲנָֽחְנוּ׃
	עׇזְרֵ֤נוּ \source{תהלים עט}
	׀ אֱלֹ֘הֵ֤י יִשְׁעֵ֗נוּ עַֽל־דְּבַ֥ר כְּבֽוֹד־שְׁמֶ֑ךָ וְהַצִּילֵ֥נוּ וְכַפֵּ֥ר עַל־חַ֝טֹּאתֵ֗ינוּ לְמַ֣עַן שְׁמֶֽךָ׃
}

\newcommand{\nefilasapayim}{
	
	\englishinst{In a place with a Sefer Torah, the following is recited leaning the head on the left forearm, unless tefillin are worn on that arm in which case it is recited leaning on the right arm.}
	וַיֹּ֧אמֶר דָּוִ֛ד אֶל־גָּ֖ד\source{שמ״ב כד} צַר־לִ֣י מְאֹ֑ד נִפְּלָה־נָּ֤א בְיַד־יְיָ֙ כִּֽי־רַבִּ֣ים רַֽחֲמָ֔ו וּבְיַד־אָדָ֖ם אַל־אֶפֹּֽלָה׃\\
	\firstword{רַחוּם וְחַנּוּן,}
	חָטָֽאתִי לְפָנֶֽיךָ יְיָ מָלֵא רַחֲמִים רַחֵם עָלַי וְקַבֵּל תַּחֲנוּנָי׃
	יְיָ֗ \source{תהלים ו}אַל־בְּאַפְּךָ֥ תוֹכִיחֵ֑נִי וְֽאַל־בַּחֲמָתְךָ֥ תְיַסְּרֵֽנִי׃
	חׇנֵּ֥נִי יְיָ כִּ֤י אֻמְלַ֫ל אָ֥נִי רְפָאֵ֥נִי יְיָ֑ כִּ֖י נִבְהֲל֣וּ עֲצָמָֽי׃
	וְ֭נַפְשִׁי נִבְהֲלָ֣ה מְאֹ֑ד וְאַתָּ֥ה יְ֝יָ֗ עַד־מָתָֽי׃
	שׁוּבָ֣ה יְיָ֭ חַלְּצָ֣ה נַפְשִׁ֑י ה֝וֹשִׁיעֵ֗נִי לְמַ֣עַן חַסְדֶּֽךָ׃
	כִּ֤י אֵ֣ין בַּמָּ֣וֶת זִכְרֶ֑ךָ בִּ֝שְׁא֗וֹל מִ֣י יֽוֹדֶה־לָּֽךְ׃
	יָגַ֤עְתִּי ׀ בְּֽאַנְחָתִ֗י אַשְׂחֶ֣ה בְכׇל־לַ֭יְלָה מִטָּתִ֑י בְּ֝דִמְעָתִ֗י עַרְשִׂ֥י אַמְסֶֽה׃
	עָשְׁשָׁ֣ה מִכַּ֣עַס עֵינִ֑י עָ֝תְקָ֗ה בְּכׇל־צוֹרְרָֽי׃
	ס֣וּרוּ מִ֭מֶּנִּי כׇּל־פֹּ֣עֲלֵי אָ֑וֶן כִּֽי־שָׁמַ֥ע יְ֝יָ֗ ק֣וֹל בִּכְיִֽי׃
	שָׁמַ֣ע יְיָ֭ תְּחִנָּתִ֑י יְ֝יָ֗ תְּֽפִלָּתִ֥י יִקָּֽח׃
	יֵבֹ֤שׁוּ ׀ וְיִבָּהֲל֣וּ מְ֭אֹד כׇּל־אֹיְבָ֑י יָ֝שֻׁ֗בוּ יֵבֹ֥שׁוּ רָֽגַע׃
}

\newcommand{\yishtabach}{\firstword{יִשְׁתַּבַּח}
שִׁמְךָ לָעַד מַלְכֵּֽנוּ הָאֵל הַמֶּֽלֶךְ הַגָּדוֹל וְהַקָּדוֹשׁ בַּשָׁמַֽיִם וּבָאָֽרֶץ \middot כִּי לְךָ נָאֶה יְיָ אֱלֹהֵֽינוּ וֵאלֹהֵי אֲבוֹתֵֽינוּ שִׁיר וּשְׁבָחָה הַלֵּל וְזִמְרָה עֹז וּמֶמְשָׁלָה נֶֽצַח גְּדֻלָּה וּגְבוּרָה תְּהִלָּה וְתִפְאֶֽרֶת קְדֻשָּׁה וּמַלְכוּת בְּרָכוֹת וְהוֹדָאוֹת מֵעַתָּה וְעַד עוֹלָם׃ בָּרוּךְ אַתָּה יְיָ אֵל מֶֽלֶךְ גָּדוֹל בַּתֻּשְׁבָּחוֹת אֵל הַהוֹדָאוֹת אֲדוֹן הַנִּפְלָאוֹת הַבּוֹחֵר בְּשִׁירֵי זִמְרָה מֶֽלֶךְ אֵל חֵי הָעוֹלָמִים׃}

\newcommand{\shabmusafpesukim}{\firstword{וּבְיוֹם֙ הַשַּׁבָּ֔ת }\source{במדבר כח}
	שְׁנֵֽי־כְבָשִׂ֥ים בְּנֵֽי־שָׁנָ֖ה תְּמִימִ֑ם וּשְׁנֵ֣י עֶשְׂרֹנִ֗ים סֹ֧לֶת מִנְחָ֛ה בְּלוּלָ֥ה בַשֶּׁ֖מֶן וְנִסְכּֽוֹ׃
	עֹלַ֥ת שַׁבַּ֖ת בְּשַׁבַּתּ֑וֹ עַל־עֹלַ֥ת הַתָּמִ֖יד וְנִסְכָּֽהּ׃}

\newcommand{\shabbosshuva}{בשבת שובה׃}


\newcommand{\shabboskiddushhashem}{
	\firstword{אַתָּה קָדוֹשׁ}
	וְשִׁמְךָ קָדוֹשׁ וּקְדוֹשִׁים בְּכׇל־יוֹם יְהַלְלוּךָ סֶּֽלָה׃ בָּרוּךְ אַתָּה יְיָ *הָאֵל
	(*\instruction{בשבת שובה:}
	הַמֶּֽלֶךְ)
	הַקָּדוֹשׁ׃
}


\newcommand{\shabboskiddushhayom}[1]{{
		\firstword{אֱלֹהֵינוּ}
		וֵאלֹהֵי אֲבוֹתֵינוּ רְצֵה בִמְנוּחָתֵנוּ קַדְּשֵׁנוּ בְּמִצְוֹתֶיךָ וְתֵן חֶלְקֵנוּ בְּתוֹרָתֶךָ \middot שַׂבְּעֵנוּ מִטּוּבֶךָ וְשַׂמְּחֵנוּ בִּישׁוּעָתֶךָ וְטַהֵר לִבֵּנוּ לְעׇבְדְּךָ בֶּאֱמֶת׃ וְהַנְחִילֵנוּ יְיָ אֱלֹהֵינוּ בְּאַהֲבָה וּבְרָצוֹן שַׁבַּת קׇדְשֶׁךָ \middot וְיָנוּחוּ בָהּ#1 יִשְׂרָאֵל מְקַדְּשֵׁי שְׁמֶךָ׃
		בָּרוּךְ אַתָּה יְיָ מְקַדֵּשׁ הַשַּׁבָּת׃
}}

\newcommand{\YTShabboshavdalah}{
	
	\begin{sometimes}
		
		\instruction{במוצאי שבת:}
		וַתּוֹדִיעֵֽנוּ יְיָ אֱלֹהֵֽינוּ אֶת־מִשְׁפְּטֵי צִדְקֶֽךָ וַתְּלַמְּדֵֽנוּ לַעֲשׂוֹת חֻקֵּי רְצוֹנֶֽךָ וַתִּתֶּן־לָֽנוּ יְיָ אֱלֹהֵֽינוּ מִשְׁפָּטִים יְשָׁרִים וְתוֹרוֹת אֱמֶת חֻקִּים וּמִצְוֹת טוֹבִים׃ וַתַּנְחִילֵֽנוּ זְמַנֵּי שָׂשׂוֹן וּמֽוֹעֲדֵי קֹֽדֶשׁ וְחַגֵּי נְדָבָה׃ וַתּוֹרִישֵֽׁנוּ קְדֻשַּׁת שַׁבָּת וּכְבוֹד מוֹעֵד וַחֲגִיגַת הָרֶֽגֶל׃ וַתַּבְדִּילֵֽנוּ יְיָ אֱלֹהֵֽינוּ בֵּין קֹֽדֶשׁ לְחוֹל בֵּין אוֹר לְחֹֽשֶׁךְ בֵּין יִשְׂרָאֵל לָעַמִּים בֵּין יוֹם הַשְּׁבִיעִי לְשֵֽׁשֶׁת יְמֵי הַמַּעֲשֶׂה׃ בֵּין קְדֻשַּׁת שַׁבָּת לִקְדֻשַּׁת יוֹם טוֹב הִבְדַּֽלְתָּ וְאֶת־יוֹם הַשְּׁבִיעִי מִשֵּֽׁשֶׁת יְמֵי הַמַּעֲשֶׂה קִדַּֽשְׁתָּ הִבְדַּֽלְתָּ וְקִדַּֽשְׁתָּ אֶת־עַמְּךָ יִשְׂרָאֵל בִּקְדֻשָּׁתֶֽךָ׃
		
	\end{sometimes}
	
}

\newcommand{\atavechartanu}{\firstword{אַתָּה בְחַרְתָּֽנוּ}
	מִכׇּל־הָעַמִּים אָהַֽבְתָּ אוֹתָֽנוּ וְרָצִֽיתָ בָּֽנוּ וְרוֹמַמְתָּֽנוּ מִכׇּל־הַלְּשׁוֹנוֹת וְקִדַּשְׁתָּֽנוּ בְּמִצְוֹתֶֽיךָ וְקֵרַבְתָּֽנוּ מַלְכֵּֽנוּ לַעֲבוֹדָתֶֽךָ וְשִׁמְךָ הַגָּדוֹל וְהַקָּדוֹשׁ עָלֵֽינוּ קָרָֽאתָ׃}

\newcommand{\ytkiddushhayom}[1]{{
		\atavechartanu
		
		\enlargethispage{\baselineskip}
		
		#1
		
		\firstword{וַתִּתֶּן}
		לָֽנוּ יְיָ אֱלֹהֵֽינוּ בְּאַהֲבָה
		\shabaddition{שַׁבָּתוֹת לִמְנוּחָה וּ}		]
		מוֹעֲדִים
		לְשִׂמְחָה חַגִּים וּזְמַנִּים לְשָׂשׂוֹן אֶת־יוֹם
		\shabaddition{הַשַּׁבָּת הַזֶּה וְאֶת־יוֹם}
		
		
		\begin{tabular}{>{\centering\arraybackslash}m{.2\textwidth} | >{\centering\arraybackslash}m{.2\textwidth} | >{\centering\arraybackslash}m{.2\textwidth} | >{\centering\arraybackslash}m{.24\textwidth}}
			
			\instruction{לפסח} & \instruction{לשבעות} & \instruction{לסכות} & \instruction{לשמיני עצרת} \\
			
			חַג הַמַּצּוֹת הַזֶּה זְמַן חֵרוּתֵֽנוּ & חַג הַשָּׁבֻעוֹת הַזֶּה זְמַן מַתַּן תּוֹרָתֵֽנוּ & חַג הַסֻּכּוֹת הַזֶּה זְמַן שִׂמְחָתֵֽנוּ & שְׁמִינִי חַג הָעֲצֶֽרֶת הַזֶּה זְמַן שִׂמְחָתֵֽנוּ
		\end{tabular}
		
		\shabaddition{בְּאַהֲבָה}
		מִקְרָא קֹֽדֶשׁ זֵֽכֶר לִיצִיאַת מִצְרָֽיִם׃
		
		\yaalehveyavotemplate{
		\begin{tabular}{>{\centering\arraybackslash}m{.2\textwidth} | >{\centering\arraybackslash}m{.2\textwidth} | >{\centering\arraybackslash}m{.2\textwidth} | >{\centering\arraybackslash}m{.24\textwidth}}
			
			\instruction{לפסח} & \instruction{לשבעות} & \instruction{לסכות} & \instruction{לשמיני עצרת} \\
			
			חַג הַמַּצּוֹת הַזֶּה & חַג הַשָּׁבֻעוֹת הַזֶּה & חַג הַסֻּכּוֹת הַזֶּה & שְׁמִינִי חַג הָעֲצֶֽרֶת הַזֶּה
		\end{tabular}}
		
		\firstword{וְהַשִּׂיאֵֽנוּ}
		יְיָ אֱלֹהֵֽינוּ אֶת־בִּרְכַּת מוֹעֲדֶֽיךָ לְחַיִּים וּלְשָׁלוֹם לְשִׂמְחָה וּלְשָׂשׂוֹן כַּאֲשֶׁר רָצִֽיתָ וְאָמַֽרְתָּ לְבָרְכֵֽנוּ׃ [\shabbos%
		אֱלֹהֵֽינוּ וֵאלֹהֵי אֲבוֹתֵֽינוּ רְצֵה בִמְנוּחָתֵֽנוּ] קַדְּשֵֽׁנוּ בְּמִצְוֹתֶֽיךָ וְתֵן חֶלְקֵֽנוּ בְּתוֹרָתֶֽךָ שַׂבְּעֵֽנוּ מִטּוּבֶֽךָ וְשַׂמְּחֵֽנוּ בִּישׁוּעָתֶֽךָ וְטַהֵר לִבֵּֽנוּ לְעׇבְדְּךָ בֶּאֱמֶת וְהַנְחִילֵֽנוּ יְיָ אֱלֹהֵֽינוּ \shabaddition{בְּאַהֲבָה וּבְרָצוֹן} בְּשִׂמְחָה וּבְשָׂשׂוֹן
		\shabaddition{שַׁבַּת וּ}
		מוֹעֲדֵי קׇדְשֶֽׁךָ וְיִשְׂמְחוּ בְךָ יִשְׂרָאֵל מְקַדְּשֵׁי שְׁמֶךָ׃ בָּרוּךְ אַתָּה יְיָ מְקַדֵּשׁ
		\shabaddition{הַשַּׁבָּת וְ}
		 יִשְׂרָאֵל וְהַזְּמַנִּים׃
}}

\newcommand{\shabboschanukah}{
	\begin{sometimes}
		
		\instruction{בחנוכה:}
		עַל הַנִּסִּים וְעַל הַפֻּרְקָן וְעַל הַגְּבוּרוֹת וְעַל הַתְּשׁוּעוֹת וְעַל הַמִּלְחָמוֹת
		שֶׁעָשִֽׂיתָ לַאֲבוֹתֵֽינוּ בַּיָּמִים הָהֵם בַּזְּמַן הַזֶּה׃
		\bimeimatityahu
		
	\end{sometimes}
}

\newcommand{\veshameru}{
	\source{שמות לא}\firstword{וְשָׁמְר֥וּ}
	בְנֵֽי־יִשְׂרָאֵ֖ל אֶת־הַשַּׁבָּ֑ת לַעֲשׂ֧וֹת אֶת־הַשַּׁבָּ֛ת לְדֹרֹתָ֖ם בְּרִ֥ית עוֹלָֽם׃ בֵּינִ֗י וּבֵין֙ בְּנֵ֣י יִשְׂרָאֵ֔ל א֥וֹת הִ֖וא לְעֹלָ֑ם כִּי־שֵׁ֣שֶׁת יָמִ֗ים עָשָׂ֤ה יְיָ֙ אֶת־הַשָּׁמַ֣יִם וְאֶת־הָאָ֔רֶץ וּבַיּוֹם֙ הַשְּׁבִיעִ֔י שָׁבַ֖ת וַיִּנָּפַֽשׁ׃}

\newcommand{\personalfast}{
		
\englishinst{If a person wants to accept a voluntary fast, they say the following text at Min\d{h}a the preceding day:}
		רִבּוֹן הָעוֹלָמִים הֲרֵי אֲנִי לְפָנֶיךָ בְּתַעֲנִית נְדָבָה לְמָחָר׃\\
\englishinst{If they plan to fast a partial day, add:}
		עַד חֲצִי הַיּוֹם: \instruction{או:} עַד אַחֲרֵי תְּפִלַת מִנְחָה:\\
		יְהִי רָצוֹן מִלְּפָנֶֽיךָ יְיָ אֱלֹהַי וֵאלֹהֵי אֲבוֹתַי שֶׁתְּקַבְּלֵֽנִי בְּאַהֲבָה וּבְרָצוֹן וְתָבֹא לְפָנֶיךָ תְּפִלָתִי
		וְתַעֲנֶה עֲתִירָתִי בְּרַחֲמֶֽיךָ הָרַבִּים: כִּי אַתָּה שׁוֹמֵֽעַ תְּפִלַת כׇּל־פֶּה: \instruction{יהיו לרצון ...}
			
\englishinst{At min\d{h}a on the day of a personal fast say:}
		רִבּוֹן הָעוֹלָמִים גָּלוּי וְיָדֽוּעַ לְפָנֶיךָ בִּזְמַן שֶׁבֵּית הַמִּקְדָּשׁ קַיָּם אָדָם חוֹטֵא מַקְרִיב קׇרְבָּן וְאֵין מַקְרִיבִין מִמֶּֽנּוּ אֶלָּא חֶלְבּוֹ וְדָמוֹ וְאַתָּה בְּרַחֲמֶֽיךָ הָרַבִּים מְכַפֵּר \middot וְעַכְשָׁיו יָשַֽׁבְתִּי בְּתַעֲנִית וְנִתְמַעֵט חֶלְבִּי וְדָמִי׃ יְהִי רָצוֹן מִלְּפָנֶֽיךָ שֶׁיְּהִי חֶלְבִּי וְדָמִי שֶׁנִּתְמַעַט הַיּוֹם כְּאִילּוּ הִקְרַבְתִּיו לְפָנֶֽיךָ עַל גַּבֵּי הַמִּזְבֵּֽחַ וְתִרְצֵֽנִי׃}

\newcommand{\shalomravbase}{\firstword{שָׁלוֹם}
	רָב עַל יִשְׂרָאֵל עַמְּךָ תָּשִׂים לְעוֹלָם \middot כִּי אַתָּה הוּא מֶֽלֶךְ אָדוֹן לְכׇל־הַשָּׁלוֹם׃}

\newcommand{\simshalombase}{\firstword{שִׂים שָׁלוֹם}
	טוֹבָה וּבְרָכָה חֵן וָחֶֽסֶד וְרַחֲמִים עָלֵֽינוּ וְעַל כׇּל־יִשְׂרָאֵל עַמֶּֽךָ \middot בָּרְכֵֽנוּ אָבִֽינוּ כֻּלָּֽנוּ כְּאֶחָד בְּאוֹר פָּנֶֽיךָ \middot כִּי בְאוֹר פָּנֶֽיךָ נָתַֽתָּ לָֽנוּ יְיָ אֱלֹהֵֽינוּ תּוֹרַת חַיִּים וְאַהֲבַת חֶֽסֶד וּצְדָקָה וּבְרָכָה וְרַחֲמִים וְחַיִּים וְשָׁלוֹם \middot}

\newcommand{\vetov}{וְטוֹב בְּעֵינֶֽיךָ לְבָרֵךְ אֶת־עַמְּךָ יִשְׂרָאֵל בְּכׇל־עֵת וּבְכׇל־שָׁעָה בִּשְׁלוֹמֶֽךָ׃}

\newcommand{\shalomendingAYT}[1]{
\columnratio{0.7}
\begin{paracol}{2}
	
	\instruction{#1}
	\begin{small}
		בְּסֵֽפֶר חַיִּים בְּרָכָה וְשָׁלוֹם וּפַרְנָסָה טוֹבָה \middot נִזָּכֵר וְנִכָּתֵב לְפָנֶֽיךָ אָֽנוּ וְכׇל־עַמְּךָ בֵּית יִשְׂרָאֵל לְחַיִּים וּלְשָׁלוֹם׃ בָּרוּךְ אַתָּה יְיָ עוֹשֵׂה הַשָּׁלוֹם׃
		
	\end{small}
	\switchcolumn
	בָּרוּךְ אַתָּה יְיָ הַמְבָרֵךְ אֶת־עַמּוֹ יִשְׂרָאֵל בַּשָּׁלוֹם׃
\end{paracol}
}

\newcommand{\simshalom}[1]{
	\simshalombase 
\vetov
	\shalomendingAYT{בעשי״ת׃}
}

\newcommand{\shabbossimshalom}{
	\simshalombase
\vetov
	
	\shalomendingAYT{בשבת שובה׃}
}

\newcommand{\shabbosshalomrav}{
	\shalomravbase
\vetov

\shalomendingAYT{בשבת שובה׃}
}

\newcommand{\simshalomrav}{
\columnratio{0.3}
\begin{paracol}{2}
	\instruction{במנחה׃}\\
	\shalomravbase
	\switchcolumn
	\instruction{בשחרית ובמנחה בת״צ׃}\\
	\simshalombase
	
\end{paracol}

\vetov

\shalomendingAYT{בעשי״ת׃}
}

\newcommand{\simshalomplain}{\simshalombase}

\newcommand{\savri}{\instruction{סַבְרִי מָרָנָן וְרְבָּנָן וְרַבּוֹתַי}\\}

\newcommand{\pitumhaketoret}{
	\firstword{פִּטּוּם הַקְּטֹֽרֶת׃}\source{מסכת כריתות}
	(א) הַצֳּרִי (ב) וְהַצִּפֹּֽרֶן (ג) וְהַחֶלְבְּנָה (ד) וְהַלְּבוֹנָה מִשְׁקַל שִׁבְעִים שִׁבְעִים מָנֶה (ה) מֹר (ו) וּקְצִיעָה (ז) שִׁבֹּֽלֶת נֵרְדְּ (ח) וְכַרְכֹּם מִשְׁקַל שִׁשָּׁה עָשָׂר שִׁשָּׁה עָשָׂר מָנֶה (ט) הַקֹּשְׁטְ שְׁנֵים עָשָׂר (י) וְקִלּוּפָה שְׁלֹשָׁה (יא) וְקִנָּמוֹן תִּשְׁעָה׃ בֹּרִית כַּרְשִׁינָה תִּשְׁעָה קַבִּין יֵין קַפְרִיסִין סְאִין תְּלָתָא וְקַבִּין תְּלָתָא וְאִם אֵין לוֹ יֵין קַפְרִיסִין מֵבִיא חֲמַר חִוַּרְיָן עַתִּיק מֶֽלַח סְדוֹמִית רֹבַע [הַקָּב] מַעֲלֶה עָשָׁן כׇּל־שֶׁהוּא׃ רַבִּי נָתָן אוֹמֵר׃ אַף כִּפַּת הַיַּרְדֵּן כׇּל־שֶׁהוּא וְאִם נָתַן בָּהּ דְּבַשׁ פְּסָלָהּ׃ וְאִם חִסַּר אַחַת מִכׇּל־סַמָּנֶֽיהָ חַיַּב מִיתָה׃ רַבָּן שִׁמְעוֹן בֶּן גַּמְלִיאֵל אוֹמֵר׃ הַצֳּרִי אֵינוֹ אֶלָּא שְׂרָף הַנּוֹטֵף מֵעֲצֵי הַקְּטָף׃ בֹּרִית כַּרְשִׁינָה שֶׁשָּׁפִין בָּהּ אֶת־הַצִּפֹּֽרֶן כְּדֵי שֶׁתְּהֵא נָאָה׃ יֵין קַפְרִיסִין שֶׁשּׁוֹרִין בּוֹ אֶת־הַצִּפֹּֽרֶן כְּדֵי שֶׁתְּהֵא עַזָּה וַהֲלֹא מֵי רַגְלַֽיִם יָפִין לָהּ אֶלָּא שֶׁאֵין מַכְנִיסִין מֵי רַגְלַֽיִם בָּעֲזָרָה מִפְּנֵי הַכָּבוֹד׃
}

\newcommand{\AVHHN}{
	\begin{large}
	\textbf{אֲנִי וָהוֹ הוֹשִֽׁיעָה נָּא׃}
\end{large}

\begin{small}
	כְּהוֹשַֽׁעְתָּ אֵלִים בְּלוּד עִמָּךְ\hfill\break\hfill בְּצֵאתְךָ לְיֵֽשַׁע עַמָּךְ \hfill כֵּן הוֹשַׁע נָא׃ \\
	כְּהוֹשַֽׁעְתָּ גּוֹי וֵאלֹהִים\hfill\break\hfill דְּרוּשִׁים לְיֵֽשַׁע אֱלֹהִים \hfill כֵּן הוֹשַׁע נָא׃ \\
	כְּהוֹשַֽׁעְתָּ הֲמוֹן צְבָאוֹת\hfill\break\hfill וְעִמָּם מַלְאֲכֵי צְבָאוֹת \hfill כֵּן הוֹשַׁע נָא׃ \\
	כְּהוֹשַֽׁעְתָּ זַכִּים מִבֵּית עֲבָדִים\hfill\break\hfill חַנּוּן בְּיָדָם מַעֲבִידִים \hfill כֵּן הוֹשַׁע נָא׃ \\
	כְּהוֹשַֽׁעְתָּ טְבוּעִים בְּצוּל גְּזָרִים\hfill\break\hfill יְקָרְךָ עִמָּם מַעֲבִירִים \hfill כֵּן הוֹשַׁע נָא׃ \\
	כְּהוֹשַֽׁעְתָּ כַּנָּה מְשׁוֹרֶֽרֶת וַיּֽוֹשַׁע\hfill\break\hfill לְגוֹחָהּ מְצֻיֶּנֶת וַיִוָּֽשַׁע \hfill כֵּן הוֹשַׁע נָא׃ \\
	כְּהוֹשַֽׁעְתָּ מַאֲמַר וְהוֹצֵאתִי אֶתְכֶם\hfill\break\hfill נָקוּב וְהוּצֵאתִי אִתְּכֶם \hfill כֵּן הוֹשַׁע נָא׃\\
	כְּהוֹשַֽׁעְתָּ סוֹבְבֵי מִזְבֵּֽחַ\hfill\break\hfill עוֹמְסֵי עֲרָבָה לְהַקִּיף מִזְבֵּֽחַ \hfill כֵּן הוֹשַׁע נָא׃ \\
	כְּהוֹשַֽׁעְתָּ פִּלְאֵי אָרוֹן כְּהֻפְשַׁע\hfill\break\hfill צִעֵר פְּלֶֽשֶׁת בַּחֲרוֹן אַף וְנוֹשַׁע \hfill כֵּן הוֹשַׁע נָא׃\\
	כְּהוֹשַֽׁעְתָּ קְהִלּוֹת בָּבֶֽלָה שִׁלַּֽחְתָּ\hfill\break\hfill רַחוּם לְמַעֲנָם שֻׁלַּחְתָּ \hfill כֵּן הוֹשַׁע נָא׃\\
	כְּהוֹשַֽׁעְתָּ שְׁבוּת שִׁבְטֵי יַעֲקֹב\hfill\break\hfill תָּשׁוּב וְתָשִׁיב שְׁבוּת אׇהֳלֵי יַעֲקֹב \hfill וְהוֹשִׁיעָה נָּא׃\\
	כְּהוֹשַֽׁעְתָּ שׁ֗וֹמְרֵי מִ֗צְווֹת וְ֗חוֹכֵי יְשׁוּעוֹת\hfill\break\hfill אֵ֗ל֗ לְמוֹשָׁעוֹת \hfill וְהוֹשִׁיעָה נָּא׃
	
\end{small}

\begin{large}
	\textbf{אֲנִי וָהוֹ הוֹשִֽׁיעָה נָּא׃}
\end{large}
}

\newcommand{\havineinu}{
הֲבִינֵֽנוּ יְיָ אֱלֹהֵֽינוּ לָדַֽעַת דְּרָכֶיךָ. וּמוֹל אֶת־לְבָבֵֽנוּ לְיִרְאָתֶֽךָ. וְתִסְלַח לָֽנוּ לִהְיוֹת גְּאוּלִים. וְרַחֲקֵנוּ מִמַּכְאוֹב. וְדַשְּׁנֵֽנוּ בִּנְאוֹת אַרְצֶֽךָ. וּנְפוּצוֹתֵֽינוּ מֵאַרְבַּע כַּנְפוֹת הָאָֽרֶץ תְּקַבֵּץ. וְהַתּוֹעִים עַל דַּעְתְּךָ יִשָׁפֵֽטוּ. וְעַל הַרְשָׁעִים תָּנִיף יָדֶֽךָ. וְיִשְׂמְחוּ צַדִיקִים בְּבִנְיַן עִירֶֽךָ. וּבְתִקּוּן הֵיכָלֶֽךָ. וּבִצְמִֽיחַת קֶֽרֶן לְדָוִד עַבְדֶּֽךָ. וּבְעֲרִֽיכַת נֵר לְבֶן יִשַׁי מְשִׁיחֶֽךָ. טֶֽרֶם נִקְרָא אַתָּה תַעֲנֶה׃ בָּרוּךְ אַתָּה יְיָ שׁוֹמֵֽעַ תְּפִלָּה׃}

\newcommand{\shacharitinstruction}{\longenginst{The following begins the formal morning prayer. Barekhu is recited only in public prayer. The reader bends their knees and bows while saying the word \hebineng{ברכו} and straightens for the divine name, and the congregation does the same while saying \hebineng{ברוך}.}}

\newcommand{\maarivinst}{}

\newcommand{\AMamidainst}{}

\vspace*{\fill}

\setstretch{1.5}

\centerlast

\renewcommand{\thefootnote}{\roman{footnote}} % makes footnote lower-case Roman Numeral
\setlength{\parskip}{0.75em}

\newcommand{\halfline}{\vspace{0.5\baselineskip}}

\newtoggle{includeshabbat}
\newtoggle{includefestival}
\newtoggle{includeChM}
\newtoggle{includeweekday}
\newtoggle{includeRCh}
\newtoggle{includeAYT}
\newtoggle{minchainshacharit}
\toggletrue{includeshabbat}
\togglefalse{includeweekday}
\toggletrue{includeRCh}
\toggletrue{includeAYT}
\togglefalse{includefestival}
\togglefalse{includeChM}
\togglefalse{minchainshacharit}

\begingroup
\let\clearpage\relax
\chapter[מנחה לחול]{\adforn{47} מנחה לחול \adforn{19}}
\vspace{0.25in}
\ashrei

\halfkaddish

\instruction{בתענית קוראים התורה עמ׳ \pageref{weekday torah}},

\section[תפילת העמידה]{\adforn{53} תפילת העמידה \adforn{25}}


\amidaopening{\ayt}{}

\weekdaysakedusha \vspace{0.5\baselineskip}

\sepline

\weekdaysabinah

\weekdaysateshuva

\weekdaysaselichah

\weekdaysageulah

\weekdaysaanneinu

\weekdaysarefuah

\weekdaysaberacha

\weekdaysashofar

\weekdaysamishpat

\weekdaysaminim

\weekdaysatzadikim

\yerushwithnachem

\weekdaysamalchus

\firstword{שְׁמַע קוֹלֵֽנוּ}
יְיָ אֱלֹהֵֽינוּ חוּס וְרַחֵם עָלֵֽינוּ וְקַבֵּל בְּרַחֲמִים וּבְרָצוֹן אֶת־תְּפִלָּתֵֽנוּ כִּי אֵל שׁוֹמֵעַ תְּפִלּוֹת וְתַחֲנוּנִים אַֽתָּה וּמִלְּפָנֶֽיךָ מַלְכֵּֽנוּ רֵיקָם אַל תְּשִׁיבֵֽנוּ
\footnote{
\instruction{בתענית ציבור היחיד אומר כאן עננו, אלא מסיימים בכִּי אַתָּה שׁוֹמֵֽעַ...\\
}
עֲנֵֽנוּ יְיָ עֲנֵֽנוּ בְּיוֹם צוֹם תַּעֲנִיתֵֽנוּ כִּי בְצָרָה גְדוֹלָה אֲנָֽחְנוּ אַל תֵּֽפֶן אֶל רִשְׁעֵֽנוּ וְאַל תַּסְתֵּר פָּנֶֽיךָ מִמֶּֽנּוּ וְאַל תִּתְעַלַּם מִתְּחִנָּתֵֽנוּ׃ הֱיֵה־נָא קָרוֹב לְשַׁוְעָתֵֽנוּ יְהִי־נָא חַסְדְּךָ לְנַחֲמֵֽנוּ טֶֽרֶם נִקְרָא אֵלֶֽיךָ עֲנֵֽנוּ כַּדָּבָר שֶׁנֶּאֱמַר׃
\mdsource{ישעיה סה}%
וְהָיָ֥ה טֶֽרֶם־יִקְרָ֖אוּ וַאֲנִ֣י אֶעֱנֶ֑ה ע֛וֹד הֵ֥ם מְדַבְּרִ֖ים וַאֲנִ֥י אֶשְׁמָֽע׃ כִּי אַתָּה יְיָ הָעוֹנֶה בְּעֵת צָרָה פּוֹדֶה וּמַצִּיל בְּכׇל־עֵת צָרָה וְצוּקָה׃
}
כִּי אַתָּה שׁוֹמֵֽעַ תְּפִלַּת עַמְּךָ יִשְׂרָאֵל בְּרַחֲמִים׃ בָּרוּךְ אַתָּה יְיָ שׁוֹמֵֽעַ תְּפִלָּה׃

\retzeh

\yaalehveyavo

\zion

\modim

\alhanisim

\weekdaysahodos

\instruction{בחזרת הש״ץ בתענית ציבור:}
אֱלֹהֵֽינוּ וֵאלֹהֵי אֲבוֹתֵֽינוּ בָּרְכֵֽנוּ בַּבְּרָכָה הַמְשֻׁלֶּֽשֶׁת בַּתּוֹרָה
הַכְּתוּבָה עַל יְדֵי מֹשֶׁה עַבְדֶּֽךָ הָאֲמוּרָה מִפִּי אַהֲרֹן וּבָנָיו כֹּהֲנִים עַם קְדוֹשֶֽׁךָ כָּאָמוּר׃

יְבָֽרֶכְךָ֥ יְיָ֖ וְיִשְׁמְרֶֽךָ׃ \hfill \kahal כֵּן יְהִי רָצוׂן \\
יָאֵ֨ר יְיָ֧ פָּנָ֛יו אֵלֶ֖יךָ וִֽיחֻנֶּֽךָּ׃ \hfill \kahal כֵּן יְהִי רָצוׂן \\
יִשָּׂ֨א יְיָ֤ פָּנָיו֙ אֵלֶ֔יךָ וְיָשֵׂ֥ם לְךָ֖ שָׁלֽוֹם׃ \hfill \kahal כֵּן יְהִי רָצוׂן

\rule[-0.5ex]{3in}{1pt}

\columnratio{0.73}
\begin{paracol}{2}
\instruction{בתענית ציבור:}\\
\firstword{שִׂים שָׁלוֹם}
טוֹבָה וּבְרָכָה חֵן וָחֶֽסֶד וְרַחֲמִים עָלֵֽינוּ וְעַל כׇּל־יִשְׂרָאֵל עַמֶּֽךָ׃ בָּרְכֵֽנוּ אָבִֽינוּ כֻּלָּֽנוּ כְּאֶחָד בְּאוֹר פָּנֶֽיךָ כִּי בְאוֹר פָּנֶֽיךָ נָתַֽתָּ לָֽנוּ יְיָ אֱלֹהֵֽינוּ תּוֹרַת חַיִּים וְאַהֲבַת חֶֽסֶד וּצְדָקָה וּבְרָכָה וְרַחֲמִים וְחַיִּים וְשָׁלוֹם׃
\switchcolumn
\firstword{שָׁלוֹם}
רָב עַל יִשְׂרָאֵל עַמְּךָ תָּשִׂים לְעוֹלָם כִּי אַתָּה הוּא מֶֽלֶךְ אָדוֹן לְכׇל־הַשָּׁלוֹם׃
\end{paracol}
וְטוֹב בְּעֵינֶֽיךָ לְבָרֵךְ אֶת־עַמְּךָ יִשְׂרָאֵל בְּכׇל־עֵת וּבְכׇל־שָׁעָה בִּשְׁלוֹמֶֽךָ׃


\columnratio{0.7}
\begin{paracol}{2}
\begin{small}
\instruction{בעשי״ת:}
בְּסֵֽפֶר חַיִּים בְּרָכָה וְשָׁלוֹם וּפַרְנָסָה טוֹבָה נִזָּכֵר וְנִכָּתֵב לְפָנֶֽיךָ אָֽנוּ וְכׇל־עַמְּךָ בֵּית יִשְׂרָאֵל לְחַיִּים וּלְשָׁלוֹם׃ בָּרוּךְ אַתָּה יְיָ עוֹשֵׂה הַשָּׁלוֹם׃

\end{small}
\switchcolumn
בָּרוּךְ אַתָּה יְיָ הַמְבָרֵךְ אֶת־עַמּוֹ יִשְׂרָאֵל בַּשָּׁלוֹם׃

\end{paracol}

\firstword{אֱלֹהַי}
נְצֹר לְשׁוֹנִי מֵרָע וּשְׂפָתַי מִדַּבֵּר מִרְמָה וְלִמְקַלְלַי נַפְשִׁי תִדּוֹם וְנַפְשִׁי כֶּעָפָר לַכֹּל תִּהְיֶה׃ פְּתַח לִבִּי בְּתוֹרָתֶֽךָ וּבְמִצְוֹתֶֽיךָ תִּרְדּוֹף נַפְשִׁי׃ וְכֹל הַחוֹשְׁבִים עָלַי רָעָה מְהֵרָה הָפֵר עֲצָתָם וְקַלְקֵל מַחֲשַׁבְתָם׃ עֲשֵׂה לְמַֽעַן שְׁמֶֽךָ עֲשֵׂה לְמַֽעַן יְמִינֶֽךָ עֲשֵׂה לְמַֽעַן קְדֻשָּׁתֶֽךָ עֲשֵׂה לְמַֽעַן תּוֹרָתֶֽךָ׃ לְ֭מַעַן \source{תהלים ס}יֵחָלְצ֣וּן יְדִידֶ֑יךָ
הוֹשִׁ֖יעָה יְמִינְךָ֣ וַעֲנֵֽנִי׃

\personalfast

יִֽהְי֥וּ לְרָצ֨וֹן אִמְרֵי־פִ֡י \source{תהלים יט}וְהֶגְי֣וֹן לִבִּ֣י לְפָנֶ֑יךָ יְ֜יָ֗ צוּרִ֥י וְגֹֽאֲלִֽי׃ עֹשֶׂה שָׁלוֹם בִּמְרוֹמָיו הוּא יַעֲשֶׂה שָׁלוֹם עָלֵֽינוּ וְעַל כׇּל־יִשְׂרָאֵל וְאִמְרוּ אָמֵן׃


\begin{small}

יְהִי רָצוֹן מִלְּפָנֶֽיךָ יְיָ אֱלֹהֵֽינוּ וִֵאלֹהֵי אֲבוֹתֵֽינוּ שֶׁיִבָּנֶה בֵּית הַמִּקְדָּשׁ בִּמְהֵרָה בְיָמֵֽינוּ וְתֵן חֶלְקֵֽנוּ בְּתוֹרָתֶֽךָ׃ וְשָׁם נַעֲבׇדְךָ בְּיִרְאָה כִּימֵי עוֹלָם וּכְשָׁנִים קַדְמֹנִיּוֹת׃
וְעָֽרְבָה֙ \source{מלאכי ג}לַֽיְיָ֔ מִנְחַ֥ת יְהוּדָ֖ה וִירוּשָׁלָ֑םִ כִּימֵ֣י עוֹלָ֔ם וּכְשָׁנִ֖ים קַדְמֹֽנִיּֽוֹת׃


\end{small}



\instruction{בימים שאין בהם תחנון ממשיכים עם עלינו עמ׳ \pageref{mincha aleinu}}\\
\instruction{בעשי״ת (לא בערב שבת וערב יוה״כ) ובת״צ אומרים אבינו מלכנו}

\section[אבינו מלכנו]{\adforn{53} אבינו מלכנו \adforn{25}}

\instruction{פותחים הארון}

\avinumalkeinu

\vfill
\instruction{סגורים הארון}\\

\section[תחנון]{\adforn{53} תחנון \adforn{25}}

\instruction{אין אומרים תחנון בימים ובערב ימים אלו׃ שבת, יו״ט, ר״ח, פסח שני, יום העצמאות, ל״ג בעומר, יום ירושלים, ט׳ באב, חנכה, פורים, שושן פורים, פורים קטן, ושושן פורים קטן. גם א״א תחנון כשיש חתן או כלה בבהכ״נ, בבית אבל, בחודש ניסן, ר״ח סיון עד י״ב סיון, ערב יום כפור עד אחרי ר״ח מרחשון, ושאר ימי שמחה}

\nefilasapayim

\shomeryisroel

\fullkaddish

\label{mincha aleinu}

\aleinu
\mournerskaddish
\vspace{0.25in}
\chapter[ערובין והדלקת נרות]{\adforn{47} ערובין והדלקת נרות \adforn{19}}
\vspace{0.15in}
\ifboolexpr{togl {includefestival}}{
\englishinst{To make an Eruv, hold the food on which the Eruv is made and recite:}
\firstword{בָּרוּךְ}
אַתָּה יְיָ אֱלֹהֵינוּ מֶלֶךְ הָעוֹלָם אֲשֶׁר קִדְּשָׁנוּ בְּמִצְוֹתָיו וְצִוָּנוּ עַל מִצְוַת עֵרוּב׃

\englishinst{To make an Eruv permitting preparing for Shabbat during a Festival, when Friday is a festival:}
\instruction{ערוב תבשילין׃ }\firstword{בְּדֵן עֵרוּבָא}
 יְהֵא שְׁרֵא לַֽנָא לְמֵיפֵא וּלְבַשּּׁלָא וּלְאַטְמָנָא וּלְאַדְלָקָא שְׁרָגָא וּלְמֶעְבַּד כׇּל־צָרְכָּנָא מִיּוֹמָא טָבָא לְשַׁבְּתָא (לָֽנוּ וּלְכׇל־הַדָּרִים בָּעִיר הַזּׂאת)׃
 
 \engliturgy{By means of this \textit{Eruv} it shall be permitted for us to bake, cook, warm, light flame, and prepare for all our needs from the festival to the Sabbath (for us and all those who live in this city).}
 ‏}{}

\englishinst{To make an Eruv to reset the starting point for the Te\d{h}um:}
\instruction{ערוב תחומין׃ }\firstword{בַּהֲדֵין עֵירוּבָא}
  יְהֵא שְׁרֵא לָֽנָא לֵילֵךְ מִמָקוֹם זֶה אַלְפַּֽיִם אַמָּה לְכׇל־רֽוּחַ׃
  
\engliturgy{By means of this \textit{Eruv} it shall be permitted for us to walk from this place two thousand cubits in all directions.}

\englishinst{To make an Eruv to combining a shared courtyard:}
\instruction{ערוב חצרות׃ }\firstword{בַּהֲדֵין עֵירוּבָא}
 יְהֵא שְׁרֵא לָֽנָא לְאַפּוּקֵי וּלְעַיוּלֵי מִבָּתִּים לֶחָצֵר, וּמֵחָצֵר לְבָתִּים, וּמִבַּֽיִת לְבַֽיִת, לָֽנוּ וּלְכׇל־יִשְׂרָאֵל הַדָּרִים בֶּחָצֵר הַזֶּה׃

\engliturgy{By means of this \textit{Eruv} it shall be permitted for us to take items out and bring them in from houses to the courtyard, from the courtyard to houses, and from house to house, for us and all Jews who live in this courtyard.}

\ifboolexpr{togl {includeshabbat}}{\ifboolexpr{togl {includefestival}}{\shabbos}{}\firstword{בָּרוּךְ}
אַתָּה, יְיָ אֱלֹהֵֽינוּ מֶֽלֶךְ הָעוֹלָם אֲשֶׁר קִדְשָֽׁנוּ בְּמִצְוֹתָיו וְצִוְּֽנוּ לְהַדְלִיק נֵר שֶׁל שַׁבָּת׃\\}{}

\ifboolexpr{togl {includefestival}}{\instruction{בערב יום טוב׃}\\
\firstword{בָּרוּךְ}
אַתָּה, יְיָ אֱלֹהֵֽינוּ מֶֽלֶךְ הָעוֹלָם אֲשֶׁר קִדְשָֽׁנוּ בְּמִצְוֹתָיו וְצִוְּֽנוּ לְהַדְלִיק נֵר שֶׁל
\shabaddition{\\שַׁבָּת וְ}יוֹם טוֹב׃

\instruction{בערב יום כפור׃}\\
\firstword{בָּרוּךְ}
אַתָּה, יְיָ אֱלֹהֵֽינוּ מֶֽלֶךְ הָעוֹלָם אֲשֶׁר קִדְשָֽׁנוּ בְּמִצְוֹתָיו וְצִוְּֽנוּ לְהַדְלִיק נֵר שֶׁל
\shabaddition{\\שַׁבָּת וְ}יוֹם הַכִּפּוּרִים׃

%\instruction{בערב יום טוב (חוץ משביעי של פסח), ראש השנה, וים כפור׃}\\
\englishinst{When lighting candles for all holidays except the last days of Pesa\d{h}:}
\firstword{בָּרוּךְ}
אַתָּה יְיָ אֱלֹהֵינוּ מֶלֶךְ הָעוֹלָם שֶׁהֶחֱיָנוּ וְקִיְּמָנוּ וְהִגִּיעָנוּ לַזְמַן הַזֶּה׃\\}{}


\chapter[קבלת שבת]{\adforn{47} קבלת שבת \adforn{19}}
\label{kabalas_shabbos}

\ifboolexpr{togl {includeshabbat}}{
	\englishinst{Some read the Song of Songs on Friday afternoon. If time does not permit that, the following verses from Song of Songs may be recited:}
	%\instruction{יש נוהגים לקרא שיר השירים בע״ש. ואם אין זמן לקרא כל הספר, יש לקרא פסוקים אלו׃}
יִשָּׁקֵ֙נִי֙ \source{שיר השירים ב}מִנְּשִׁיק֣וֹת פִּ֔יהוּ כִּֽי־טוֹבִ֥ים דֹּדֶ֖יךָ מִיָּֽיִן׃
ע֤וּרִי\source{שיר השירים ד} צָפוֹן֙ וּב֣וֹאִי תֵימָ֔ן הָפִ֥יחִי גַנִּ֖י יִזְּל֣וּ בְשָׂמָ֑יו יָבֹ֤א דוֹדִי֙ לְגַנּ֔וֹ וְיֹאכַ֖ל פְּרִ֥י מְגָדָֽיו׃ ק֣וֹל \source{שיר השירים ב}דּוֹדִ֔י הִנֵּה־זֶ֖ה בָּ֑א מְדַלֵּג֙ עַל־הֶ֣הָרִ֔ים מְקַפֵּ֖ץ עַל־הַגְּבָעֽוֹת׃ בָּ֣אתִי \source{שיר השירים ה}לְגַנִּי֮ אֲחֹתִ֣י כַלָּה֒ אָרִ֤יתִי מוֹרִי֙ עִם־בְּשָׂמִ֔י אָכַ֤לְתִּי יַעְרִי֙ עִם־דִּבְשִׁ֔י שָׁתִ֥יתִי יֵינִ֖י עִם־חֲלָבִ֑י אִכְל֣וּ רֵעִ֔ים שְׁת֥וּ וְשִׁכְר֖וּ דּוֹדִֽים׃

\englishinst{Some sing the following before Kabbalat Shabbat:}
\firstword{יְדִיד נֶֽפֶשׁ}
אָב הָרַחֲמָן מְשֹׁךְ עַבְדָּךְ אֶל רְצוֹנָךְ׃\\
יָרוּץ עַבְדָּךְ כְּמוֹ אַיָּל יִשְׁתַּחֲוֶה מוּל הֲדָרָךְ׃\\
כִּי יֶעְרַב־לוֹ יְדִידוּתָךְ מִנֹּֽפֶת צוּף וְכׇל־טָעַם׃

הָדוּר נָאֶה זִיו הָעוֹלָם נַפְשִׁי חוֹלַת אַהֲבָתָךְ׃\\
אָנָּא אֵל נָא רְפָא־נָא לָהּ בְּהַרְאוֹת לָהּ נֹֽעַם זִיוָךְ׃\\
אָז תִּתְחַזֵּק וְתִתְרַפֵּא וְהָיְתָה לָּךְ שִׁפְחַת עוֹלָם׃

וָתִיק יֶהְמוּ־נָא רַחְמֶיךָ וְחוּס־נָא עַל בֵּן אוֹהֲבָךְ׃\\
כִּי זֶה כַּמָּה נִכְסֹף נִכְסַף לִרְאוֹת בְּתִפְאֶֽרֶת עֻזָּךְ׃\\
אָנָּא אֵלִי מַחְמַד לִבִּי חֽוּשָׁה נָּא וְאַל תִּתְעַלָּם׃

הִגָּלֵה־נָא וּפְרֹס חָבִיב עָלַי אֶת־סֻכַּת שְׁלוֹמָךְ׃\\
תָּאִיר אֶרֶץ מִכְּבוֹדָךְ נָגִֽילָה וְנִשְׂמְחָה בָךְ׃\\
מַהֵר אָהוּב כִּי בָא מוֹעֵד וְחׇנֵּנִי כִּימֵי עוֹלָם׃

\firstword{לְ֭כוּ נְרַנְּנָ֣ה}\source{תהלים צה}
לַייָ֑ נָ֝רִ֗יעָה לְצ֣וּר יִשְׁעֵֽנוּ׃
נְקַדְּמָ֣ה פָנָ֣יו בְּתוֹדָ֑ה בִּ֝זְמִר֗וֹת נָרִ֥יעַֽ לֽוֹ׃
כִּ֤י אֵ֣ל גָּד֣וֹל יְיָ֑ וּמֶ֥לֶךְ גָּ֝ד֗וֹל עַל־כׇּל־אֱלֹהִֽים׃
אֲשֶׁ֣ר בְּ֭יָדוֹ מֶחְקְרֵי־אָ֑רֶץ וְתוֹעֲפֹ֖ת הָרִ֣ים לֽוֹ׃
אֲשֶׁר־ל֣וֹ הַ֭יָּם וְה֣וּא עָשָׂ֑הוּ וְ֝יַבֶּ֗שֶׁת יָדָ֥יו יָצָֽרוּ׃
בֹּ֭אוּ נִשְׁתַּחֲוֶ֣ה וְנִכְרָ֑עָה נִ֝בְרְכָ֗ה לִֽפְנֵי־יְיָ֥ עֹשֵֽׂנוּ׃
כִּ֘י ה֤וּא אֱלֹהֵ֗ינוּ וַאֲנַ֤חְנוּ עַ֣ם מַ֭רְעִיתוֹ וְצֹ֣אן יָד֑וֹ הַ֝יּ֗וֹם אִֽם־בְּקֹל֥וֹ תִשְׁמָֽעוּ׃
אַל־תַּקְשׁ֣וּ לְ֭בַבְכֶם כִּמְרִיבָ֑ה כְּי֥וֹם מַ֝סָּ֗ה בַּמִּדְבָּֽר׃
אֲשֶׁ֣ר נִ֭סּוּנִי אֲבֽוֹתֵיכֶ֑ם בְּ֝חָנ֗וּנִי גַּם־רָא֥וּ פׇעֳלִֽי׃
אַרְבָּ֘עִ֤ים שָׁנָ֨ה ׀ אָ֘ק֤וּט בְּד֗וֹר וָאֹמַ֗ר עַ֤ם תֹּעֵ֣י לֵבָ֣ב הֵ֑ם וְ֝הֵ֗ם לֹא־יָדְע֥וּ דְרָכָֽי׃
אֲשֶׁר־נִשְׁבַּ֥עְתִּי בְאַפִּ֑י אִם־יְ֝בֹא֗וּן אֶל־מְנוּחָתִֽי׃


\firstword{שִׁ֣ירוּ לַ֭יְיָ}\source{תהלים צו}
שִׁ֣יר חָדָ֑שׁ שִׁ֥ירוּ לַ֝ייָ֗ כׇּל־הָאָֽרֶץ׃
שִׁ֣ירוּ לַ֭ייָ בָּרְכ֣וּ שְׁמ֑וֹ בַּשְּׂר֥וּ מִיּֽוֹם־לְ֝י֗וֹם יְשׁוּעָתֽוֹ׃
סַפְּר֣וּ בַגּוֹיִ֣ם כְּבוֹד֑וֹ בְּכׇל־הָ֝עַמִּ֗ים נִפְלְאוֹתָֽיו׃
כִּ֥י גָ֘ד֤וֹל יְיָ֣ וּמְהֻלָּ֣ל מְאֹ֑ד נוֹרָ֥א ה֗֝וּא עַל־כׇּל־אֱלֹהִֽים׃
כִּ֤י ׀ כׇּל־אֱלֹהֵ֣י הָעַמִּ֣ים אֱלִילִ֑ים וַ֝ייָ֗ שָׁמַ֥יִם עָשָֽׂה׃
הוֹד־וְהָדָ֥ר לְפָנָ֑יו עֹ֥ז וְ֝תִפְאֶ֗רֶת בְּמִקְדָּשֽׁוֹ׃
הָב֣וּ לַ֭ייָ מִשְׁפְּח֣וֹת עַמִּ֑ים הָב֥וּ לַ֝ייָ֗ כָּב֥וֹד וָעֹֽז׃
הָב֣וּ לַ֭ייָ כְּב֣וֹד שְׁמ֑וֹ שְׂאֽוּ־מִ֝נְחָ֗ה וּבֹ֥אוּ לְחַצְרוֹתָֽיו׃
הִשְׁתַּחֲו֣וּ לַ֭ייָ בְּהַדְרַת־קֹ֑דֶשׁ חִ֥ילוּ מִ֝פָּנָ֗יו כׇּל־הָאָֽרֶץ׃
אִמְר֤וּ בַגּוֹיִ֨ם ׀ יְ֘יָ֤ מָלָ֗ךְ אַף־תִּכּ֣וֹן תֵּ֭בֵל בַּל־תִּמּ֑וֹט יָדִ֥ין עַ֝מִּ֗ים בְּמֵישָׁרִֽים׃
יִשְׂמְח֣וּ הַ֭שָּׁמַיִם וְתָגֵ֣ל הָאָ֑רֶץ יִֽרְעַ֥ם הַ֝יָּ֗ם וּמְלֹאֽוֹ׃
יַעֲלֹ֣ז שָׂ֭דַי וְכׇל־אֲשֶׁר־בּ֑וֹ אָ֥ז יְ֝רַנְּנ֗וּ כׇּל־עֲצֵי־יָֽעַר׃
לִפְנֵ֤י יְיָ֨ ׀ כִּ֬י בָ֗א כִּ֥י בָא֮ לִשְׁפֹּ֢ט הָ֫אָ֥רֶץ יִשְׁפֹּֽט־תֵּבֵ֥ל בְּצֶ֑דֶק וְ֝עַמִּ֗ים בֶּאֱמוּנָתֽוֹ׃

\firstword{יְיָ֣ מָלָךְ}\source{תהלים צז}
תָּגֵ֣ל הָאָ֑רֶץ יִ֝שְׂמְח֗וּ אִיִּ֥ים רַבִּֽים׃
עָנָ֣ן וַעֲרָפֶ֣ל סְבִיבָ֑יו צֶ֥דֶק וּ֝מִשְׁפָּ֗ט מְכ֣וֹן כִּסְאֽוֹ׃
אֵ֭שׁ לְפָנָ֣יו תֵּלֵ֑ךְ וּתְלַהֵ֖ט סָבִ֣יב צָרָֽיו׃
הֵאִ֣ירוּ בְרָקָ֣יו תֵּבֵ֑ל רָאֲתָ֖ה וַתָּחֵ֣ל הָאָֽרֶץ׃
הָרִ֗ים כַּדּוֹנַ֗ג נָ֭מַסּוּ מִלִּפְנֵ֣י יְיָ֑ מִ֝לִּפְנֵ֗י אֲד֣וֹן כׇּל־הָאָֽרֶץ׃
הִגִּ֣ידוּ הַשָּׁמַ֣יִם צִדְק֑וֹ וְרָא֖וּ כׇל־הָעַמִּ֣ים כְּבוֹדֽוֹ׃
יֵבֹ֤שׁוּ ׀ כׇּל־עֹ֬בְדֵי פֶ֗סֶל הַמִּֽתְהַלְלִ֥ים בָּאֱלִילִ֑ים הִשְׁתַּחֲווּ־ל֗֝וֹ כׇּל־אֱלֹהִֽים׃
שָׁמְעָ֬ה וַתִּשְׂמַ֨ח ׀ צִיּ֗וֹן וַ֭תָּגֵלְנָה בְּנ֣וֹת יְהוּדָ֑ה לְמַ֖עַן מִשְׁפָּטֶ֣יךָ יְיָ׃
כִּֽי־אַתָּ֤ה יְיָ֗ עֶלְי֥וֹן עַל־כׇּל־הָאָ֑רֶץ מְאֹ֥ד נַ֝עֲלֵ֗יתָ עַל־כׇּל־אֱלֹהִֽים׃
אֹֽהֲבֵ֥י יְיָ֗ שִׂנְא֫וּ רָ֥ע שֹׁ֭מֵר נַפְשׁ֣וֹת חֲסִידָ֑יו מִיַּ֥ד רְ֝שָׁעִ֗ים יַצִּילֵֽם׃
א֭וֹר זָרֻ֣עַ לַצַּדִּ֑יק וּֽלְיִשְׁרֵי־לֵ֥ב שִׂמְחָֽה׃
שִׂמְח֣וּ צַ֭דִּיקִים בַּייָ֑ וְ֝הוֹד֗וּ לְזֵ֣כֶר קׇדְשֽׁוֹ׃

\firstword{מִזְמ֡וֹר שִׁ֤ירוּ}\source{תהלים צח}
לַֽיְיָ֙ שִׁ֥ייָ֨ ׀ שִׁ֣יר חָ֭דָשׁ כִּֽי־נִפְלָא֣וֹת עָשָׂ֑ה הוֹשִׁיעָה־לּ֥וֹ יְ֝מִינ֗וֹ וּזְר֥וֹעַ קׇדְשֽׁוֹ׃
הוֹדִ֣יעַ יְיָ֭ יְשׁוּעָת֑וֹ לְעֵינֵ֥י הַ֝גּוֹיִ֗ם גִּלָּ֥ה צִדְקָתֽוֹ׃
זָ֘כַ֤ר חַסְדּ֨וֹ ׀ וֶ֥אֱֽמוּנָתוֹ֮ לְבֵ֢ית יִשְׂרָ֫אֵ֥ל רָא֥וּ כׇל־אַפְסֵי־אָ֑רֶץ אֵ֗֝ת יְשׁוּעַ֥ת אֱלֹהֵֽינוּ׃
הָרִ֣יעוּ לַ֭ייָ כׇּל־הָאָ֑רֶץ פִּצְח֖וּ וְרַנְּנ֣וּ וְזַמֵּֽרוּ׃
זַמְּר֣וּ לַייָ֣ בְּכִנּ֑וֹר בְּ֝כִנּ֗וֹר וְק֣וֹל זִמְרָֽה׃
בַּ֭חֲצֹ֣צְרוֹת וְק֣וֹל שׁוֹפָ֑ר הָ֝רִ֗יעוּ לִפְנֵ֤י ׀ הַמֶּ֬לֶךְ יְיָ׃
יִרְעַ֣ם הַ֭יָּם וּמְלֹא֑וֹ תֵּ֝בֵ֗ל וְיֹ֣שְׁבֵי בָֽהּ׃
נְהָר֥וֹת יִמְחֲאוּ־כָ֑ף יַ֗֝חַד הָרִ֥ים יְרַנֵּֽנוּ׃
לִ֥פְֽנֵי יְיָ֗ כִּ֥י בָא֮ לִשְׁפֹּ֢ט הָ֫אָ֥רֶץ יִשְׁפֹּֽט־תֵּבֵ֥ל בְּצֶ֑דֶק וְ֝עַמִּ֗ים בְּמֵישָׁרִֽים׃

\firstword{יְיָ֣ מָ֭לָךְ}\source{תהלים צט}
יִרְגְּז֣וּ עַמִּ֑ים יֹשֵׁ֥ב כְּ֝רוּבִ֗ים תָּנ֥וּט הָאָֽרֶץ׃
יְיָ֭ בְּצִיּ֣וֹן גָּד֑וֹל וְרָ֥ם ה֗֝וּא עַל־כׇּל־הָעַמִּֽים׃
יוֹד֣וּ שִׁ֭מְךָ גָּד֥וֹל וְנוֹרָ֗א קָד֥וֹשׁ הֽוּא׃
וְעֹ֥ז מֶלֶךְ֮ מִשְׁפָּ֢ט אָ֫הֵ֥ב אַ֭תָּה כּוֹנַ֣נְתָּ מֵישָׁרִ֑ים מִשְׁפָּ֥ט וּ֝צְדָקָ֗ה בְּיַעֲקֹ֤ב ׀ אַתָּ֬ה עָשִֽׂיתָ׃
רוֹמְמ֡וּ יְ֘יָ֤ אֱלֹהֵ֗ינוּ וְֽ֭הִשְׁתַּחֲווּ לַהֲדֹ֥ם רַגְלָ֗יו קָד֥וֹשׁ הֽוּא׃
מֹ֘שֶׁ֤ה וְאַֽהֲרֹ֨ן ׀ בְּֽכֹהֲנָ֗יו וּ֭שְׁמוּאֵל בְּקֹרְאֵ֣י שְׁמ֑וֹ קֹרִ֥אים אֶל־יְ֝יָ֗ וְה֣וּא יַעֲנֵֽם׃
בְּעַמּ֣וּד עָ֭נָן יְדַבֵּ֣ר אֲלֵיהֶ֑ם שָׁמְר֥וּ עֵ֝דֹתָ֗יו וְחֹ֣ק נָֽתַן־לָֽמוֹ׃
יְיָ֣ אֱלֹהֵינוּ֮ אַתָּ֢ה עֲנִ֫יתָ֥ם אֵ֣ל נֹ֭שֵׂא הָיִ֣יתָ לָהֶ֑ם וְ֝נֹקֵ֗ם עַל־עֲלִילוֹתָֽם׃
רוֹמְמ֡וּ יְ֘יָ֤ אֱלֹהֵ֗ינוּ וְֽ֭הִשְׁתַּחֲווּ לְהַ֣ר קׇדְשׁ֑וֹ כִּי־קָ֝ד֗וֹשׁ יְיָ֥ אֱלֹהֵֽינוּ׃

\englishinst{The following Psalm is said standing.}
\firstword{מִזְמ֗וֹר לְדָ֫וִ֥ד}\source{תהלים כט}
הָב֣וּ לַ֭ייָ בְּנֵ֣י אֵלִ֑ים הָב֥וּ לַ֝ייָ֗ כָּב֥וֹד וָעֹֽז׃
הָב֣וּ לַ֭ייָ כְּב֣וֹד שְׁמ֑וֹ הִשְׁתַּחֲו֥וּ לַ֝ייָ֗ בְּהַדְרַת־קֹֽדֶשׁ׃
ק֥וֹל יְיָ֗ עַל־הַ֫מָּ֥יִם אֵֽל־הַכָּב֥וֹד הִרְעִ֑ים יְ֝יָ֗ עַל־מַ֥יִם רַבִּֽים׃
קוֹל־יְיָ֥ בַּכֹּ֑חַ ק֥וֹל יְ֝יָ֗ בֶּהָדָֽר׃
ק֣וֹל יְיָ֭ שֹׁבֵ֣ר אֲרָזִ֑ים וַיְשַׁבֵּ֥ר יְ֝יָ֗ אֶת־אַרְזֵ֥י הַלְּבָנֽוֹן׃
וַיַּרְקִידֵ֥ם כְּמוֹ־עֵ֑גֶל לְבָנ֥וֹן וְ֝שִׂרְיֹ֗ן כְּמ֣וֹ בֶן־רְאֵמִֽים׃
קוֹל־יְיָ֥ חֹצֵ֗ב לַהֲב֥וֹת אֵֽשׁ׃
ק֣וֹל יְיָ֭ יָחִ֣יל מִדְבָּ֑ר יָחִ֥יל יְ֝יָ֗ מִדְבַּ֥ר קָדֵֽשׁ׃
ק֤וֹל יְיָ֨ ׀ יְחוֹלֵ֣ל אַיָּלוֹת֮ וַֽיֶּחֱשֹׂ֢ף יְעָ֫ר֥וֹת וּבְהֵיכָל֑וֹ כֻּ֝לּ֗וֹ אֹמֵ֥ר כָּבֽוֹד׃
יְיָ֭ לַמַּבּ֣וּל יָשָׁ֑ב וַיֵּ֥שֶׁב יְ֝יָ֗ מֶ֣לֶךְ לְעוֹלָֽם׃
יְיָ֗ עֹ֭ז לְעַמּ֣וֹ יִתֵּ֑ן יְיָ֓ ׀ יְבָרֵ֖ךְ אֶת־עַמּ֣וֹ בַשָּׁלֽוֹם׃


\section[לכה דודי]{\adforn{53} לכה דודי \adforn{25}}

\newcommand{\lechadodi}{\textbf{לְכָה דוֹדִי לִקְרַאת כַּלָּה פְּנֵי שַׁבָּת נְקַבְּלָה׃}\\}

%\leftskip=0pt plus-.2fil
%\rightskip=0pt plus.2fil
%\parfillskip=0pt plus1fil
%\vspace{-0.5\baselineskip}
\newcommand{\lechadodiverse}[4]{
#1 \hfill #2\\
#3 \hfill #4\\
}

\lechadodi
%\lechadodiverse{\acrostic{שָׁ}מוֹר וְזָכוֹר בְּדִבּוּר אֶחָד}{הִשְׁמִיעָֽנוּ אֵל הַמְיֻחָד}{יְיָ אֶחָד וּשְׁמוֹ אֶחָד}{לְשֵׁם וּלְתִפְאֶֽרֶת וְלִתְהִלָּה׃}
%\lechadodi
\lechadodiverse{ \acrostic{שָׁ}מוֹר וְזָכוֹר בְּדִבּוּר אֶחָד}{הִשְׁמִיעָֽנוּ אֵל הַמְיֻחָד}{יְיָ אֶחָד וּשְׁמוֹ אֶחָד}{לְשֵׁם וּלְתִפְאֶֽרֶת וְלִתְהִלָּה׃}
\lechadodi
\lechadodiverse{\acrostic{לִ}קְרַאת שַׁבָּת לְכוּ וְנֵלְכָה}{כִּי הִיא מְקוֹר הַבְּרָכָה}{מֵרֹאשׁ מִקֶּֽדֶם נְסוּכָה}{סוֹף מַעֲשֶׂה בְּמַחֲשָׁבָה תְּחִלָּה׃}
\lechadodi
\lechadodiverse{\acrostic{מִ}קְדַּשׁ מֶֽלֶךְ עִיר מְלוּכָה}{קֽוּמִי צְאִי מִתּוֹךְ הַהֲפֵכָה}{רַב לָךְ שֶֽׁבֶת בְּעֵֽמֶק הַבָּכָא}{וְהוּא יַחֲמוֹל עָלַֽיִךְ חֶמְלָה׃}
\lechadodi
\lechadodiverse{\acrostic{הִ}תְנַעֲרִי מֵעָפָר קֽוּמִי}{לִבְשִׁי בִּגְדֵי תִפְאַרְתֵּךְ עַמִּי}{עַל יַד בֶּן יִשַׁי בֵּית הַלַּחְמִי}{קׇרְבָה אֶל נַפְשִׁי גְאָלָהּ׃}
\lechadodi
\lechadodiverse{\acrostic{הִ}תְעוֹרְרִי הִתְעוֹרְרִי}{כִּי בָא אוֹרֵךְ קֽוּמִי אֽוֹרִי}{עֽוּרִי עֽוּרִי שִׁיר דַבֵּֽרִי}{כְּבוֹד יְיָ עָלַֽיִךְ נִגְלָה׃}
\lechadodi
\lechadodiverse{\acrostic{לֹ}א תֵבֽוֹשִׁי וְלֹא תִכָּלְמִי}{מַה תִּשְׁתּוֹחֲחִי וּמַה תֶּהֱמִי}{בָּךְ יֶחֱסוּ עֲנִיֵּי עַמִּי}{וְנִבְנְתָה עִיר עַל תִּלָּהּ׃}
\lechadodi
\lechadodiverse{\acrostic{וְ}הָיוּ לִמְשִׁסָּה שֹׁאסָֽיִךְ}{וְרָחֲקוּ כׇּל־מְבַלְּעָֽיִךְ}{יָשִׂישׂ עָלַֽיִךְ אֱלֹהָֽיִךְ}{כִּמְשׂוֹשׂ חָתָן עַל כַּלָּה׃}
\lechadodi
\lechadodiverse{\acrostic{יָ}מִין וּשְׂמֹאל תִּפְרֽוֹצִי}{וְאֶת יְיָ תַּעֲרִֽיצִי}{עַל יַד אִישׁ בֶּן פַּרְצִי}{וְנִשְׂמְחָה וְנָגִֽילָה׃}
\lechadodi
\englishinst{Stand, and face the synagogue entrance for this verse:}
\lechadodiverse{בּֽוֹאִי בְשָׁלוֹם עֲטֶרֶת בַּעְלָהּ}{גַּם בְּשִׂמְחָה וּבְצׇהֳלָה}{תּוֹךְ אֱמוּנֵי עַם סְגֻּלָּה}{בּֽוֹאִי כַלָּה בּֽוֹאִי כַלָּה׃}
\lechadodi \vspace{-0.5\baselineskip}

\begin{sometimes}

\englishinst{Mourners do not attend Kabbalat Shabbat until this point.  As they enter, the congregation says to the mourners:}
%\instruction{הציבור לאבלים:}\\
הַמָּקוֹם יְנַחֵם אֶתְכֶם בְּתוֹךְ שְׁאָר אֲבֵלֵי צִיּוֹן וִירוּשָׁלָֽיִם׃
\end{sometimes}

\ifboolexpr{togl {includefestival}}{\englishinst{On Festivals and Intermediate Festival Shabbat, Kabbalat Shabbat begins here.}}{}
}{\englishinst{When a Festival falls on Friday evening, the following Psalms are said:}}
\ifboolexpr{togl {includefestival}}{\englishinst{}}{}
\mizmorshabbat

\firstword{יְיָ֣ מָלָךְ֘ גֵּא֢וּת לָ֫בֵ֥שׁ}\source{תהלים צג}
לָבֵ֣שׁ יְיָ֭ עֹ֣ז הִתְאַזָּ֑ר אַף־תִּכּ֥וֹן תֵּ֝בֵ֗ל בַּל־תִּמּֽוֹט׃
נָכ֣וֹן כִּסְאֲךָ֣ מֵאָ֑ז מֵעוֹלָ֣ם אָֽתָּה׃
נָשְׂא֤וּ נְהָר֨וֹת ׀ יְיָ֗ נָשְׂא֣וּ נְהָר֣וֹת קוֹלָ֑ם יִשְׂא֖וּ נְהָר֣וֹת דׇּכְיָֽם׃
מִקֹּל֨וֹת ׀ מַ֤יִם רַבִּ֗ים אַדִּירִ֣ים מִשְׁבְּרֵי־יָ֑ם אַדִּ֖יר בַּמָּר֣וֹם יְיָ׃
עֵֽדֹתֶ֨יךָ ׀ נֶאֶמְנ֬וּ מְאֹ֗ד לְבֵיתְךָ֥ נַאֲוָה־קֹ֑דֶשׁ יְ֝יָ֗ לְאֹ֣רֶךְ יָמִֽים׃

\mournerskaddish

\ifboolexpr{togl {includeshabbat}}{
\ssubsection{\adforn{48} במה מדליקין \adforn{22}}

\englishinst{The following chapter of Mishna is not recited on Festivals, including Shabbat \d{H}ol HaMo'ed.}
%\instruction{אין אומרים במה מדליקין בשבת חול המועד:}\\
\firstword{(א) בַּמֶּה מַדְלִיקִין}\source{שבת פרק ב}
וּבַמָּה אֵין מַדְלִיקִין׃ אֵין מַדְלִיקִין לֹא בְלֶֽכֶשׁ וְלֹא בְחֹֽסֶן וְלֹא בְכַלָּךְ וְלֹא בִּפְתִילַת הָאִידָן וְלֹא בִּפְתִילַת הַמִּדְבָּר וְלֹא בִּירוֹקָה שֶׁעַל פְּנֵי הַמָּֽיִם \middot לֹא בְזֶֽפֶת וְלֹא בְשַׁעֲוָה וְלֹא בְּשֶֽׁמֶן קִיק וְלֹא בְּשֶֽׁמֶן שְׂרֵפָה וְלֹא בְאַלְיָה וְלֹא בְחֵֽלֶב׃ נַחוּם הַמָּדִי אוֹמֵר׃ מַדְלִיקִין בְּחֵֽלֶב מְבֻשָּׁל \middot וַחֲכָמִים אוֹמְרִים׃ אֶחָד מְבֻשָּׁל וְאֶחָד שֶׁאֵינוֹ מְבֻשָּׁל אֵין מַדְלִיקִין בּוֹ׃

(ב) אֵין מַדְלִיקִין בְּשֶֽׁמֶן שְׂרֵפָה בְּיוֹם טוֹב׃ רַבִּי יִשְׁמָעֵאל אוֹמֵר׃ אֵין מַדְלִיקִין בְּעִטְרָן מִפְּנֵי כְּבוֹד הַשַּׁבָּת׃ וַחֲכָמִים מַתִּירִין בְּכׇל־הַשְּׁמָנִים׃ בְּשֶֽׁמֶן שֻׁמְשְׁמִין בְּשֶֽׁמֶן אֱגוֹזִים בְּשֶֽׁמֶן צְנוֹנוֹת בְּשֶֽׁמֶן דָּגִים בְּשֶֽׁמֶן פַּקֻּעוֹת בְּעִטְרָן וּבְנֵפְטְ׃ רַבִּי טַרְפוֹן אוֹמֵר׃ אֵין מַדְלִיקִין אֶלָּא בְּשֶֽׁמֶן זַֽיִת בִּלְבָד׃

(ג) כׇּל־הַיּוֹצֵא מִן הָעֵץ אֵין מַדְלִיקִין בּוֹ אֶלָּא פִשְׁתָּן \middot וְכׇל־הַיּוֹצֵא מִן הָעֵץ אֵינוֹ מִטַּמֵּא טֻמְאַת אֹהָלִים אֶלָּא פִשְׁתָּן׃ פְּתִילַת הַבֶּֽגֶד שֶׁקִּפְּלָהּ וְלֹא הִבְהֲבָהּ \middot רַבִּי אֱלִיעֶֽזֶר אוֹמֵר׃ טְמֵאָה הִיא וְאֵין מַדְלִיקִין בָּהּ׃ רַבִּי עֲקִיבָא אוֹמֵר׃ טְהוֹרָה הִיא וּמַדְלִיקִין בָּהּ׃

(ד) לֹא יִקּוֹב אָדָם שְׁפוֹפֶֽרֶת שֶׁל בֵּיצָה וִימַלְּאֶֽנָּה שֶֽׁמֶן וְיִתְּנֶֽנָּה עַל פִּי הַנֵּר בִּשְׁבִיל שֶׁתְּהֵא מְנַטֶּֽפֶת וַאֲפִילוּ הִיא שֶׁל חֶֽרֶס \middot וְרַבִּי יְהוּדָה מַתִּיר׃ אֲבָל אִם חִבְּרָהּ הַיּוֹצֵר מִתְּחִלָּה \middot מֻתָּר מִפְּנֵי שֶׁהוּא כֶּֽלִי אֶחָד׃ לֹא יְמַלֵּא אָדָם קְעָרָה שֶֽׁמֶן וְיִתְּנֶֽנָּה בְּצַד הַנֵּר וְיִתֵּן רֹאשׁ הַפְּתִילָה בְּתוֹכָהּ בִּשְׁבִיל שֶׁתְּהֵא שׁוֹאֶֽבֶת \middot וְרַבִּי יְהוּדָה מַתִּיר׃

(ה) הַמְכַבֶּה אֶת־הַנֵּר מִפְּנֵי שֶׁהוּא מִתְיָרֵא מִפְּנֵי גוֹיִם מִפְּנֵי לִסְטִים מִפְּנֵי רֽוּחַ רָעָה אוֹ בִּשְׁבִיל הַחוֹלֶה שֶׁיִּישָׁן פָּטוּר׃ כְּחָס עַל הַנֵּר כְּחָס עַל הַשֶּֽׁמֶן כְּחָס עַל הַפְּתִילָה חַיָּב׃ רַבִּי יוֹסֵי פּוֹטֵר בְּכֻלָּן חוּץ מִן הַפְּתִילָה מִפְּנֵי שֶׁהוּא עוֹשָׂהּ פֶּחָם׃

(ו) עַל שָׁלֹשׁ עֲבֵרוֹת נָשִׁים מֵתוֹת בִּשְׁעַת לֵדָתָן׃ עַל שֶׁאֵינָן זְהִירוֹת בְּנִדָּה בְּחַלָּה וּבְהַדְלָקַת הַנֵּר׃

(ז) שְׁלֹשָׁה דְבָרִים צָרִיךְ אָדָם לוֹמַר בְּתוֹךְ בֵּיתוֹ עֶֽרֶב שַׁבָּת עִם חֲשֵׁכָה׃ עִשַׂרְתֶּם עֵרַבְתֶּם הַדְלִֽיקוּ אֶת־הַנֵּר׃ סָפֵק חֲשֵׁכָה סָפֵק אֵינָהּ חֲשֵׁכָה \middot אֵין מְעַשְּׂרִין אֶת־הַוַּדָּי וְאֵין מַטְבִּילִין אֶת־הַכֵּלִים וְאֵין מַדְלִיקִין אֶת־הַנֵּרוֹת \middot אֲבָל מְעַשְּׂרִין אֶת־הַדְּמָי וּמְעָרְבִין וְטוֹמְנִין אֶת־הַחַמִּין׃

\sofberakhot


\rabbiskaddish}{}

\vspace{\baselineskip}

\ifboolexpr{togl {includeshabbat} and togl {includefestival}}{\chapter[ערבית לשבת ויו״ט]{\adforn{47} ערבית לשבת ויו״ט \adforn{19}}}{
	\ifboolexpr{togl {includeshabbat}}{\chapter[ערבית לשבת]{\adforn{47} ערבית לשבת \adforn{19}}}{}
	\ifboolexpr{togl {includefestival}}{\chapter[ערבית ליו״ט]{\adforn{47} ערבית ליו״ט \adforn{19}}}{}}

\barachu

\hamaarivaravim

\ahavasolam

\shema

\veahavta

\vehaya

\vayomer{}

\emesveemuna

\hashkiveinu{וּפְרוֹס עָלֵֽינוּ סֻכַּת שְׁלוֹמֶֽךָ׃ בָּרוּךְ אַתָּה יְיָ הַפּוֹרֵס סֻכַּת שָׁלוֹם עָלֵֽינוּ וְעַל כׇּל־עַמּוֹ יִשְׂרָאֵל וְעַל יְרוּשָׁלַ‍ִם׃}

\ifboolexpr{togl {includeshabbat}}{\instruction{הקהל ביחד}\\}{\instruction{הקהל ביחד בשבת}}
\veshameru

\ifboolexpr{togl {includefestival} and not togl {includeshabbat}}{\instruction{ברגלים:}}{}

\ifboolexpr{togl {includefestival}}{\textbf{
		וַיְדַבֵּ֣ר מֹשֶׁ֔ה אֶת־מֹעֲדֵ֖י יְיָ֑ אֶל־בְּנֵ֖י יִשְׂרָאֵֽל׃
	}\source{ןיקרא כג}}{}



\halfkaddish

\englishinst{On Festivals, recite the Amidah on page \pageref{YTamidah}.}

\section[תפילת העמידה]{\adforn{53} תפילת העמידה \adforn{25}}

\amidaopening{\shabbosshuva}{}

%\shabboskiddushhashem
%\ifboolexpr{togl {includeshabbat} and togl {includefestival}}{
%	\englishinst{On festivals, continue on page \pageref{maarivyt}.}
%}{}
%
%\ifboolexpr{togl {includeshabbat}}{
\firstword{אַתָּה קִדַּֽשְׁתָּ}
אֶת־יוֹם הַשְּׁבִיעִי לִשְׁמֶֽךָ תַּכְלִית מַעֲשֵׂה שָׁמַֽיִם וָאָֽרֶץ וּבֵרַכְתּוֹ מִכׇּל־הַיָּמִים וְקִדַּשְׁתּוֹ מִכׇּל־הַזְּמַנִּים וְכֵן כָּתוּב בְּתוֹרָתֶֽךָ׃

\firstword{וַיְכֻלּ֛וּ}\source{בראשית ב}
הַשָּׁמַ֥יִם וְהָאָ֖רֶץ וְכׇל־צְבָאָֽם׃ וַיְכַ֤ל אֱלֹהִים֙ בַּיּ֣וֹם הַשְּׁבִיעִ֔י מְלַאכְתּ֖וֹ אֲשֶׁ֣ר עָשָׂ֑ה וַיִּשְׁבֹּת֙ בַּיּ֣וֹם הַשְּׁבִיעִ֔י מִכׇּל־מְלַאכְתּ֖וֹ אֲשֶׁ֥ר עָשָֽׂה׃ וַיְבָ֤רֶךְ אֱלֹהִים֙ אֶת־י֣וֹם הַשְּׁבִיעִ֔י וַיְקַדֵּ֖שׁ אֹת֑וֹ כִּ֣י ב֤וֹ שָׁבַת֙ מִכׇּל־מְלַאכְתּ֔וֹ אֲשֶׁר־בָּרָ֥א אֱלֹהִ֖ים לַֽעֲשֽׂוֹת׃

\shabboskiddushhayom{}
%}{}

%\ifboolexpr{togl {includeshabbat} and togl {includefestival}}{\instruction{רצה וכו׳}
%
%\sepline}{}
%
%\ifboolexpr{togl {includefestival}}{\label{maarivyt}
%\ytkiddushhayom{\YTShabboshavdalah}}{}
%
%
%\ifboolexpr{togl {includeshabbat} and togl {includefestival}}{\sepline}{}

\retzeh

\yaalehveyavo

\zion

\maarivmodim

\ifboolexpr{togl {includeshabbat}}{\shabboschanukah

\shabboshodos}{}

\shabbosshalomrav

\tachanunim

\ifboolexpr{togl {includefestival}}{\englishinst{On Shabbat, the congregation stands and says the following together:}}{\englishinst{The congregation stands and says the following together:}}
\label{vayachulu}
\firstword{וַיְכֻלּ֛וּ} \source{בראשית ב}
הַשָּׁמַ֥יִם וְהָאָ֖רֶץ וְכׇל־צְבָאָֽם׃ וַיְכַ֤ל אֱלֹהִים֙ בַּיּ֣וֹם הַשְּׁבִיעִ֔י מְלַאכְתּ֖וֹ אֲשֶׁ֣ר עָשָׂ֑ה וַיִּשְׁבֹּת֙ בַּיּ֣וֹם הַשְּׁבִיעִ֔י מִכׇּל־מְלַאכְתּ֖וֹ אֲשֶׁ֥ר עָשָֽׂה׃ וַיְבָ֤רֶךְ אֱלֹהִים֙ אֶת־י֣וֹם הַשְּׁבִיעִ֔י וַיְקַדֵּ֖שׁ אֹת֑וֹ כִּ֣י ב֤וֹ שָׁבַת֙ מִכׇּל־מְלַאכְתּ֔וֹ אֲשֶׁר־בָּרָ֥א אֱלֹהִ֖ים לַעֲשֽׂוֹת׃


\ssubsection{\adforn{48} מעין שבע \adforn{22}}\\
\englishinst{The following is not said on the first nights of Pesa\d{h}.}
בָּרוּךְ אַתָּה יְיָ אֱלֹהֵֽינוּ וֵאלֹהֵי אֲבוֹתֵֽינוּ \middot אֱלֹהֵי אַבְרָהָם אֱלֹהֵי יִצְחָק וֵאלֹהֵי יַעֲקֹב \middot הָאֵל הַגָּדוֹל הַגִּבּוֹר וְהַנּוֹרָא אֵל עֶלְיוֹן קֹנֵה שָׁמַֽיִם וָאָֽרֶץ׃

\englishinst{The congregation says the following paragraph, and the leader repeats it and continues with the paragraph after:}
מָגֵן אָבוֹת בִּדְבָרוֹ מְחַיֵּה מֵתִים בְּמַאֲמָרוֹ הָאֵל
(\instruction{בשבת שובה:} הַמֶּֽלֶךְ)
הַקָּדוֹשׁ שֶׁאֵין כָּמֽוֹהוּ הַמֵּנִֽיחַ לְעַמּוֹ בְּיוֹם שַׁבַּת קׇדְשׁוֹ כִּי בָם רָצָה לְהָנִֽיחַ לָהֶם׃ לְפָנָיו נַעֲבוֹד בְּיִרְאָה וָפַֽחַד וְנוֹדֶה לִשְׁמוֹ בְּכׇל־יוֹם תָּמִיד מֵעֵין הַבְּרָכוֹת׃ אֵל הַהוֹדָאוֹת אֲדוֹן הַשָּׁלוֹם מְקַדֵּשׁ הַשַּׁבָּת וּמְבָרֵךְ שְׁבִיעִי וּמֵנִֽיחַ בִּקְדֻשָּׁה לְעַם מְדֻשְּׁנֵי עֹֽנֶג זֵֽכֶר לְמַעֲשֵׂה בְרֵאשִׁית׃

\shabboskiddushhayom{}

\fullkaddish

\newcommand{\kiddushshabbateve}{\firstword{בָּרוּךְ}
	אַתָּה יְיָ אֱלֹהֵֽינוּ מֶֽלֶךְ הָעוֹלָם אֲשֶׁר קִדְּשָֽׁנוּ בְּמִצְוֹתָיו וְרָֽצָה בָֽנוּ וְשַׁבַּת קׇדְשׁוֹ בְּאַהֲבָה וּבְרָצוֹן הִנְחִילָֽנוּ זִכָּרוֹן לְמַעֲשֵׂה בְרֵאשִׁית׃ כִּי הוּא יוֹם תְּחִלָּה לְמִקְרָאֵי קֹֽדֶשׁ זֵֽכֶר לִיצִיאַת מִצְרָֽיִם׃ כִּי בָֽנוּ בָחַֽרְתָּ וְאוֹתָֽנוּ קִדַּֽשְׁתָּ מִכׇּל־הָעַמִּים׃ וְשַׁבַּת קׇדְשְׁךָ בְּאַהֲבָה וּבְרָצוֹן הִנְחַלְתָּֽנוּ׃ בָּרוּךְ אַתָּה יְיָ מְקַדֵּשׁ הַשַּׁבָּת׃}
\newcommand{\kiddushYTeve}{\firstword{בָּרוּךְ}
	אַתָּה יְיָ אֱלֹהֵֽינוּ מֶֽלֶךְ הָעוֹלָם אֲשֶׁר בָּֽחַר בָּֽנוּ מִכׇּל־עָם וְרוֹמְמָֽנוּ מִכׇּל־לָשׁוֹן וְקִדְּשָֽׁנוּ בְּמִצְוֹתָיו׃ וַתִּתֶּן לָֽנוּ יְיָ אֱלֹהֵֽינוּ בְּאַהֲבָה
	\shabaddition{שַׁבָּתוֹת לִמְנוּחָה וּ}
	מוֹעֲדִים לְשִׂמְחָה חַגִּים וּזְמַנִּים לְשָׂשׂוֹן׃ אֶת־יוֹם
	\shabaddition{הַשַּׁבָּת הַזֶּה וְאֶת יוֹם} \\
	\begin{tabular}{>{\centering\arraybackslash}m{.2\textwidth} | >{\centering\arraybackslash}m{.2\textwidth} | >{\centering\arraybackslash}m{.2\textwidth} | >{\centering\arraybackslash}m{.25\textwidth}}
		\instruction{לפסח} & \instruction{לשבעות} & \instruction{לסכות} &
		\instruction{לשמיני עצרת ולשמ״ת}
		\\
		חַג הַמַּצּוֹת הַזֶּה זְמַן חֵרוּתֵֽנוּ&
		חַג הַשָּׁבֻעוֹת הַזֶּה זְמַן מַתַּן תּוֹרָתֵֽנוּ&
		חַג הַסֻּכּוֹת הַזֶּה זְמַן שִׂמְחָתֵֽנוּ &
		שְׁמִינִי חַג הָעֲצֶֽרֶת הַזֶּה זְמַן שִׂמְחָתֵֽנוּ\\
		
	\end{tabular}
	
	\shabaddition{בְּאַהֲבָה}
	מִקְרָא קֹֽדֶשׁ זֵֽכֶר לִיצִיאַת מִצְרָֽיִם׃ כִּי בָֽנוּ בָחַֽרְתָּ וְאוֹתָֽנוּ קִדַּֽשְׁתָּ מִכׇּל־הָעַמִּים \shabaddition{וְשַׁבַּת} וּמוֹעֲדֵי קׇדְשֶֽׁךָ \shabaddition{בְּאַהֲבָה וּבְרָצוֹן} בְּשִׂמְחָה וּבְשָׂשׂוֹן הִנְחַלְתָּֽנוּ׃ בָּרוּךְ אַתָּה יְיָ מְקַדֵּשׁ \shabaddition{הַשַּׁבָּת וְ} יִשְׂרָאֵל וְהַזְּמַנִּים׃
	
	\begin{sometimes}
		
		\englishinst{On Saturday night, include Havdalah:}
		בָּרוּךְ אַתָּה יְיָ אֱלֹהֵֽינוּ מֶֽלֶךְ הָעוֹלָם בּוֹרֵא מְאוֹרֵי הָאֵשׁ׃
		
		בָּרוּךְ אַתָּה יְיָ אֱלֹהֵֽינוּ מֶֽלֶךְ הָעוֹלָם הַמַּבְדִיל בֵּין קֹֽדֶשׁ לְחוֹל בֵּין אוֹר לְחֹֽשֶׁךְ בֵּין יִשְׂרָאֵל לָעַמִּים בֵּין יוֹם הַשְּׁבִיעִי לְשֵֽׁשֶׁת יְמֵי הַמַּעֲשֶׂה׃ בֵּין קְדֻשַּׁת שַׁבָּת לִקְדֻשַּׁת יוֹם טוֹב הִבְדַּֽלְתָּ וְאֶת־יוֹם הַשְּׁבִיעִי מִשֵּֽׁשֶׁת יְמֵי הַמַּעֲשֶׂה קִדַּֽשְׁתָּ הִבְדַּֽלְתָּ וְקִדַּֽשְׁתָּ אֶת־עַמְּךָ יִשְׂרָאֵל בִּקְדֻשָּׁתֶֽךָ׃ בָּרוּךְ אַתָּה יְיָ הַמַּבְדִּיל בֵּין קֹֽדֶשׁ לְקֹֽדֶשׁ׃
		
\end{sometimes}}

\englishinst{In many congregations, the reader recites kiddush here.}
\firstword{בָּרוּךְ}
אַתָּה יְיָ אֱלֹהֵֽינוּ מֶֽלֶךְ הָעוֹלָם בּוֹרֵא פְּרִי הַגָּֽפֶן׃

\ifboolexpr{togl {includefestival} and togl {includeshabbat}}{\instruction{בשבת׃}}{}
\ifboolexpr{togl {includeshabbat}}{\kiddushshabbateve}{}

%\ifboolexpr{togl {includefestival} and togl {includeshabbat}}{\instruction{ביו״ט׃}}{}
%\ifboolexpr{togl {includefestival}}{\kiddushYTeve}{}

\englishinst{Between Pesa\d{h} and Shavu'ot, count the Omer on page \pageref{sefiras haomer}.}
%\instruction{בימי ספירה, סופרים כאן את העומר בעמ׳ \pageref{sefiras haomer}}

\aleinu

\ifboolexpr{togl {includeshabbat}}{\ledavid}{}

\mournerskaddish

\yigdal 

\vfill

\adforn{43}\quad\adforn{4}\quad\adforn{42}\\

\section[קידוש ליל שבת]{\adforn{47} סדר קידוש ליל שבת בבית \adforn{19}}



\medskip

\newcommand{\birkashabonim}{
\ssubsection{\adforn{18} ברכת הבנים \adforn{17}}

\begin{tabular}{>{\centering\arraybackslash}m{.4\textwidth} | >{\centering\arraybackslash}m{.4\textwidth}}
\instruction{לבנים} & \instruction{לבנות}\\
יְשִֽׂמְךָ֣ אֱלֹהִ֔ים כְּאֶפְרַ֖יִם וְכִמְנַשֶּׁ֑ה׃ \mdsource{בראשית מח}&
יְשִׂמֵךְ אֱלׂהִים כְּשָׂרָה, רִבְקָה, רָחֵל, וְלֵאָה [אֲשֶׁר בָּנוּ אֶת־בֵּית יִשְׂרָאֵל׃] \mdsource{רות ד}׃
\end{tabular}

יְבָרֶכְךָ֥ יְיָ֖ וְיִשְׁמְרֶֽךָ׃\\ \source{במידבר ו}
יָאֵ֨ר יְיָ֧ ׀ פָּנָ֛יו אֵלֶ֖יךָ וִֽיחֻנֶּֽךָּ׃\\
יִשָּׂ֨א יְיָ֤ ׀ פָּנָיו֙ אֵלֶ֔יךָ וְיָשֵׂ֥ם לְךָ֖ שָׁלֽוֹם׃ \\

}



\birkashabonim

\medskip

\ssubsection{\adforn{18} שלום עליכם \adforn{17}}


\firstword{שָׁלוֹם}
עֲלֵיכֶם מַלְאֲכֵי הַשָּׁרֵת מַלְאֲכֵי עֶלְיוֹן\\ מִמֶֽלֶךְ מַלְכֵי הַמְּלָכִים הַקָּדוֹשׁ בָּרוּךְ הוּא׃ \\
בּוֹאֲכֶם לְשָׁלוֹם מַלְאֲכֵי הַשָּׁלוֹם מַלְאֲכֵי עֶלְיוֹן\\ מִמֶֽלֶךְ מַלְכֵי הַמְּלָכִים הַקָּדוֹשׁ בָּרוּךְ הוּא׃\\
בָּרְכֽוּנִי לְשָׁלוֹם מַלְאֲכֵי הַשָּׁלוֹם מַלְאֲכֵי עֶלְיוֹן \\ מִמֶֽלֶךְ מַלְכֵי הַמְּלָכִים הַקָּדוֹשׁ בָּרוּךְ הוּא׃\\
צֵאתְכֶם לְשָׁלוֹם מַלְאֲכֵי הַשָּׁלוֹם מַלְאֲכֵי עֶלְיוֹן\\ מִמֶֽלֶךְ מַלְכֵי הַמְּלָכִים הַקָּדוֹשׁ בָּרוּךְ הוּא׃

\vfill
%\clearpage

\ssubsection{\adforn{18} אשת חיל \adforn{17}}

\acrostic{אֵֽ}שֶׁת־חַ֭יִל\source{משלי לא}
מִ֣י יִמְצָ֑א וְרָחֹ֖ק מִפְּנִינִ֣ים מִכְרָֽהּ׃ \hfill\break
\acrostic{בָּ֣}טַח בָּ֭הּ לֵ֣ב בַּעְלָ֑הּ וְ֝שָׁלָ֗ל לֹ֣א יֶחְסָֽר׃ \hfill\break
\acrostic{גְּ}מָלַ֣תְהוּ ט֣וֹב וְלֹא־רָ֑ע כֹּ֗֝ל יְמֵ֣י חַיֶּֽיהָ׃ \hfill\break
\acrostic{דָּ֭}רְשָׁה צֶ֣מֶר וּפִשְׁתִּ֑ים וַ֝תַּ֗עַשׂ בְּחֵ֣פֶץ כַּפֶּֽיהָ׃ \hfill\break
\acrostic{הָ֭}יְתָה כׇּאֳנִיּ֣וֹת סוֹחֵ֑ר מִ֝מֶּרְחָ֗ק תָּבִ֥יא לַחְמָֽהּ׃ \hfill\break
\acrostic{וַ}תָּ֤קׇם ׀ בְּע֬וֹד לַ֗יְלָה וַתִּתֵּ֣ן טֶ֣רֶף לְבֵיתָ֑הּ וְ֝חֹ֗ק לְנַעֲרֹתֶֽיהָ׃ \hfill\break
\acrostic{זָֽ}מְמָ֣ה שָׂ֭דֶה וַתִּקָּחֵ֑הוּ מִפְּרִ֥י כַ֝פֶּ֗יהָ נָ֣טְעָה כָּֽרֶם׃ \hfill\break
\acrostic{חָֽ}גְרָ֣ה בְע֣וֹז מׇתְנֶ֑יהָ וַ֝תְּאַמֵּ֗ץ זְרוֹעֹתֶֽיהָ׃ \hfill\break
\acrostic{טָ֭}עֲמָה כִּי־ט֣וֹב סַחְרָ֑הּ לֹא־יִכְבֶּ֖ה בַלַּ֣יְלָה נֵרָֽהּ׃ \hfill\break
\acrostic{יָ֭}דֶיהָ שִׁלְּחָ֣ה בַכִּישׁ֑וֹר וְ֝כַפֶּ֗יהָ תָּ֣מְכוּ פָֽלֶךְ׃ \hfill\break
\acrostic{כַּ֭}פָּהּ פָּֽרְשָׂ֣ה לֶעָנִ֑י וְ֝יָדֶ֗יהָ שִׁלְּחָ֥ה לָאֶבְיֽוֹן׃ \hfill\break
\acrostic{לֹ}א־תִירָ֣א לְבֵיתָ֣הּ מִשָּׁ֑לֶג כִּ֥י כׇל־בֵּ֝יתָ֗הּ לָבֻ֥שׁ שָׁנִֽים׃ \hfill\break
\acrostic{מַ}רְבַדִּ֥ים עָֽשְׂתָה־לָּ֑הּ שֵׁ֖שׁ וְאַרְגָּמָ֣ן לְבוּשָֽׁהּ׃ \hfill\break
\acrostic{נ}וֹדָ֣ע בַּשְּׁעָרִ֣ים בַּעְלָ֑הּ בְּ֝שִׁבְתּ֗וֹ עִם־זִקְנֵי־אָֽרֶץ׃ \hfill\break
\acrostic{סָ}דִ֣ין עָ֭שְׂתָה וַתִּמְכֹּ֑ר וַ֝חֲג֗וֹר נָתְנָ֥ה לַֽכְּנַעֲנִֽי׃ \hfill\break
\acrostic{עֹ}ז־וְהָדָ֥ר לְבוּשָׁ֑הּ וַ֝תִּשְׂחַ֗ק לְי֣וֹם אַחֲרֽוֹן׃ \hfill\break
\acrostic{פִּ֭}יהָ פָּתְחָ֣ה בְחׇכְמָ֑ה וְת֥וֹרַת חֶ֝֗סֶד עַל־לְשׁוֹנָֽהּ׃ \hfill\break
\acrostic{צ֭}וֹפִיָּה הֲלִיכ֣וֹת בֵּיתָ֑הּ וְלֶ֥חֶם עַ֝צְל֗וּת לֹ֣א תֹאכֵֽל׃ \hfill\break
\acrostic{קָ֣}מוּ בָ֭נֶיהָ וַֽיְאַשְּׁר֑וּהָ בַּ֝עְלָ֗הּ וַֽיְהַלְלָֽהּ׃ \hfill\break
\acrostic{רַ}בּ֣וֹת בָּ֭נוֹת עָ֣שׂוּ חָ֑יִל וְ֝אַ֗תְּ עָלִ֥ית עַל־כֻּלָּֽנָה׃ \hfill\break
\acrostic{שֶׁ֣}קֶר הַ֭חֵן וְהֶ֣בֶל הַיֹּ֑פִי אִשָּׁ֥ה יִרְאַת־יְ֝יָ֗ הִ֣יא תִתְהַלָּֽל׃ \hfill\break
\acrostic{תְּ}נוּ־לָ֭הּ מִפְּרִ֣י יָדֶ֑יהָ וִיהַלְל֖וּהָ בַשְּׁעָרִ֣ים מַֽעֲשֶֽׂיהָ׃ \hfill\break


\section[קידוש ליל שבת]{\adforn{18} קידוש ליל שבת \adforn{17}}

\label{shabbatkiddush}
\begin{footnotesize}וַֽיְהִי־עֶ֥רֶב וַֽיְהִי־בֹ֖קֶר\end{footnotesize}
י֥וֹם הַשִּׁשִּֽׁי׃ וַיְכֻלּ֛וּ \source{בראשית ב}הַשָּׁמַ֥יִם וְהָאָ֖רֶץ וְכׇל־צְבָאָֽם׃ וַיְכַ֤ל אֱלֹהִים֙ בַּיּ֣וֹם הַשְּׁבִיעִ֔י מְלַאכְתּ֖וֹ אֲשֶׁ֣ר עָשָׂ֑ה וַיִּשְׁבֹּת֙ בַּיּ֣וֹם הַשְּׁבִיעִ֔י מִכׇּל־מְלַאכְתּ֖וֹ אֲשֶׁ֥ר עָשָֽׂה׃ וַיְבָ֤רֶךְ אֱלֹהִים֙ אֶת־י֣וֹם הַשְּׁבִיעִ֔י וַיְקַדֵּ֖שׁ אֹת֑וֹ כִּ֣י ב֤וֹ שָׁבַת֙ מִכׇּל־מְלַאכְתּ֔וֹ אֲשֶׁר־בָּרָ֥א אֱלֹהִ֖ים לַעֲשֽׂוֹת׃

\savri
\firstword{בָּרוּךְ}
אַתָּה יְיָ אֱלֹהֵֽינוּ מֶֽלֶךְ הָעוֹלָם בּוֹרֵא פְּרִי הַגָּֽפֶן׃

\kiddushshabbateve

\ifboolexpr{togl {includefestival} or togl {includeChM}}{
\begin{sometimes}

\instruction{בחול המועד סוכות:}\\
בָּרוּךְ אַתָּה יְיָ אֱלֹהֵינוּ מֶלֶךְ הָעוֹלָם, אֲשֶׁר קִדְּשָׁנוּ בְּמִצְוֹתָיו, וְצִוָּנוּ לֵישֵׁב בַּסֻּכָּה׃

\end{sometimes}}{}

\adforn{43}\quad\adforn{4}\quad\adforn{42}\\

\ifboolexpr{togl {includefestival}}{\section[קידוש ליל יום טוב]{\adforn{53} קידוש ליל יום טוב \adforn{25}}
\label{kiddush leil yom tov}
\instruction{בשבת׃}
\begin{footnotesize}וַֽיְהִי־עֶ֥רֶב וַֽיְהִי־בֹ֖קֶר\end{footnotesize}
י֥וֹם הַשִּׁשִּֽׁי׃ וַיְכֻלּ֛וּ \source{בראשית ב}הַשָּׁמַ֥יִם וְהָאָ֖רֶץ וְכׇל־צְבָאָֽם׃ וַיְכַ֤ל אֱלֹהִים֙ בַּיּ֣וֹם הַשְּׁבִיעִ֔י מְלַאכְתּ֖וֹ אֲשֶׁ֣ר עָשָׂ֑ה וַיִּשְׁבֹּת֙ בַּיּ֣וֹם הַשְּׁבִיעִ֔י מִכׇּל־מְלַאכְתּ֖וֹ אֲשֶׁ֥ר עָשָֽׂה׃ וַיְבָ֤רֶךְ אֱלֹהִים֙ אֶת־י֣וֹם הַשְּׁבִיעִ֔י וַיְקַדֵּ֖שׁ אֹת֑וֹ כִּ֣י ב֤וֹ שָׁבַת֙ מִכׇּל־מְלַאכְתּ֔וֹ אֲשֶׁר־בָּרָ֥א אֱלֹהִ֖ים לַעֲשֽׂוֹת׃

\sepline


\savri
\firstword{בָּרוּךְ}
אַתָּה יְיָ אֱלֹהֵֽינוּ מֶֽלֶךְ הָעוֹלָם בּוֹרֵא פְּרִי הַגָּֽפֶן׃


\kiddushYTeve

\englishinst{On Sukkot (on the second night of Sukkot many say shehe\d{h}eyanu before this blessing):}
בָּרוּךְ אַתָּה יְיָ אֱלֹהֵינוּ מֶלֶךְ הָעוֹלָם, אֲשֶׁר קִדְּשָׁנוּ בְּמִצְוֹתָיו וְצִוָּנוּ לֵישֵׁב בַּסֻּכָּה׃

\englishinst{The following is omitted on the last nights of Pesa\d{h}:}
\firstword{בָּרוּךְ}
אַתָּה יְיָ אֱלֹהֵינוּ מֶלֶךְ הָעוֹלָם, שֶׁהֶחֱיָנוּ וְקִיְּמָנוּ וְהִגִּיעָנוּ לַזְּמַן הַזֶּה׃
\vfill
\adforn{43}\quad\adforn{4}\quad\adforn{42}\\
}


%\vspace{\baselineskip}
%{\let\clearpage\relax
%\chapter[קידוש ליל ראש השנה]{\adforn{53} קידוש ליל ראש השנה \adforn{25}}}
%\begin{footnotesize}
%סַבְרִי מָרָנָן וְרְבָּנָן וְרַבּוֹתַי\\
%\end{footnotesize}
%בָּרוּךְ אַתָּה יְיָ אֱלֹהֵֽינוּ מֶֽלֶךְ הָעוֹלָם בּוֹרֵא פְּרִי הַגָּֽפֶן׃
%
%בָּרוּךְ אַתָּה יְיָ אֱלֹהֵֽינוּ מֶֽלֶךְ הָעוֹלָם אֲשֶׁר בָּֽחַר בָּֽנוּ מִכׇּל־עָם וְרוֹמְמָֽנוּ מִכׇּל־לָשׁוֹן וְקִדְּשָֽׁנוּ בְּמִצְוֹתָיו׃ וַתִּתֶּן לָֽנוּ יְיָ אֱלֹהֵֽינוּ בְּאַהֲבָה יוֹם [הַשַּׁבָּת הַזֶּה וְאֶת יוֹם] הַזִכָּרוֹן הַזֶּה יוֹם [זִכְרוֹן] תְּרוּעָה [בְּאַהֲבָה] מִקְרָא קֹֽדֶשׁ זֵֽכֶר לִיצִיאַת מִצְרָֽיִם׃ כִּי בָֽנוּ בָחַֽרְתָּ וְאוֹתָֽנוּ קִדַּֽשְׁתָּ מִכׇּל־הָעַמִּים וּדְבָרְךָ מַלְכֵּֽנוּ אֱמֶת וְקַיָּם לָעַד׃ בָּרוּךְ אַתָּה יְיָ מֶֽלֶךְ עַל כׇּל־הָאָֽרֶץ מְקַדֵּשׁ [הַשַּׁבָּת וְ] יִשְׂרָאֵל וְיוֹם הַזִּכָּרוֹן׃
%
%
%
%\begin{sometimes}
%
%\instruction{במוצאי שבת אומרים הבדלה:}\\
%בָּרוּךְ אַתָּה יְיָ אֱלֹהֵֽינוּ מֶֽלֶךְ הָעוֹלָם בּוֹרֵא מְאוֹרֵי הָאֵשׁ׃
%
%בָּרוּךְ אַתָּה יְיָ אֱלֹהֵֽינוּ מֶֽלֶךְ הָעוֹלָם הַמַּבְדִיל בֵּין קֹֽדֶשׁ לְחוֹל בֵּין אוֹר לְחֹֽשֶׁךְ בֵּין יִשְׂרָאֵל לָעַמִּים בֵּין יוֹם הַשְּׁבִיעִי לְשֵֽׁשֶׁת יְמֵי הַמַּעֲשֶׂה׃ בֵּין קְדֻשַּׁת שַׁבָּת לִקְדֻשַּׁת יוֹם טוֹב הִבְדַּֽלְתָּ וְאֶת־יוֹם הַשְּׁבִיעִי מִשֵּֽׁשֶׁת יְמֵי הַמַּעֲשֶׂה קִדַּֽשְׁתָּ הִבְדַּֽלְתָּ וְקִדַּֽשְׁתָּ אֶת־עַמְּךָ יִשְׂרָאֵל בִּקְדֻשָּׁתֶֽךָ׃ בָּרוּךְ אַתָּה יְיָ הַמַּבְדִּיל בֵּין קֹֽדֶשׁ לְקֹֽדֶשׁ׃
%
%\end{sometimes}
%
%\firstword{בָּרוּךְ}
% אַתָּה יְיָ אֱלֹהֵינוּ מֶלֶךְ הָעוֹלָם, שֶׁהֶחֱיָנוּ וְקִיְּמָנוּ וְהִגִּיעָנוּ לַזְמַן הַזֶּה׃




\endgroup

\clearpage

\begingroup
\let\clearpage\relax
\chapter[ברכות השחר]{\adforn{47} ברכות השחר \adforn{19}}
%\chapter[ברכות השחר Blessings Morning]{\adforn{47} Blessings Morning \adforn{19}\\ ברכות השחר }

\englishinst{Upon waking:}
%\instruction{כשמתעורר בבוקר׃}
\firstword{מוֹדֶה/מוֹדָה}
אֲנִי לְפָנֶיךָ מֶלֶךְ חַי וְקַיָּם \middot שֶׁהֶחֱזַרְתָּ בִּי נִשְׁמָתִי בְּחֶמְלָה \middot רַבָּה אֱמוּנָתֶךָ׃\\
\englishinst{On washing hands:}
\firstword{בָּרוּךְ}
אַתָּה יְיָ אֱלֹהֵֽינוּ מֶֽלֶךְ הָעוֹלָם \middot אֲשֶׁר קִדְּשָֽׁנוּ בְּמִצְוֹתָיו וְצִוָּֽנוּ עַל נְטִילַת יָדָֽיִם׃

\firstword{בָּרוּךְ}
אַתָּה יְיָ אֱלֹהֵֽינוּ מֶֽלֶךְ הָעוֹלָם אֲשֶׁר יָצַר אֶת־הָאָדָם בְּחׇכְמָה וּבָרָא בוֹ נְקָבִים נְקָבִים חֲלוּלִים חֲלוּלִים \middot גָּלוּי וְיָדֽוּעַ לִפְנֵי כִסֵּא כְבוֹדֶֽךָ שֶׁאִם יִפָּתֵֽחַ אֶחָד מֵהֶם אוֹ יִסָּתֵם אֶחָד מֵהֶם אִי אֶפְשַׁר לְהִתְקַיֵּם וְלַעֲמוֹד לְפָנֶֽיךָ׃ בָּרוּךְ אַתָּה יְיָ רוֹפֵא כׇל־בָּשָׂר וּמַפְלִיא לַעֲשׂוֹת׃


\firstword{אֱלֹהַי}
נְשָׁמָה שֶׁנָּתַֽתָּ בִּי טְהוֹרָה הִיא \middot אַתָּה בְרָאתָהּ אַתָּה יְצַרְתָּהּ אַתָּה נְפַחְתָּהּ בִּי וְאַתָּה מְשַׁמְּרָהּ בְּקִרְבִּי וְאַתָּה עָתִיד לִטְּלָהּ מִמֶּֽנִּי וּלְהַחֲזִירָהּ בִּי לֶעָתִיד לָבוֹא \middot כׇּל־זְמַן שֶׁהַנְּשָׁמָה בְּקִרְבִּי מוֹדֶה/מוֹדָה אֲנִי לְפָנֶֽיךָ יְיָ אֱלֹהַי וֵאלֹהֵי אֲבוֹתַי רִבּוֹן כׇּל־הַמַּעֲשִׂים אֲדוֹן כׇּל־הַנְּשָׁמוֹת׃ בָּרוּךְ אַתָּה יְיָ הַמַּחֲזִיר נְשָׁמוֹת לִפְגָרִים מֵתִים׃

\englishinst{One who does not wear a talli\thav\space gadol and wears a talli\thav\space katan blesses as follows before putting it on:}
%\instruction{מי שלא לובש טלית גדול מתלבש בטלית קטן ומברך׃}\\
\firstword{בָּרוּךְ}
אַתָּה יְיָ אֱלֹהֵֽינוּ מֶלֶךְ הָעוֹלָם אֲשֶׁר קִדְּשָׁנוּ בְּמִצְוֹתָיו וְצִוָּנוּ עַל מִצְוַת צִיצִית׃

\newcommand{\birkothatorah}{
	בָּרוּךְ אַתָּה יְיָ אֱלֹהֵֽינוּ מֶֽלֶךְ הָעוֹלָם אֲשֶׁר קִדְּשָֽׁנוּ בְּמִצְוֹתָיו וְצִוָּֽנוּ לַעֲסוֹק בְּדִבְרֵי תוֹרָה׃ וְהַעֲרֶב־נָא יְיָ אֱלֹהֵֽינוּ אֶת־דִּבְרֵי תוֹרָתְךָ בְּפִֽינוּ וּבְפִיפִיּוֹת עַמְּךָ בֵּית יִשְׂרָאֵל \middot וְנִהְיֶה אֲנַֽחְנוּ וְצֶאֱצָאֵֽינוּ וְצֶאֱצָאֵי עַמְּךָ בֵּית יִשְׂרָאֵל כֻּלָּֽנוּ יוֹדְעֵי שְׁמֶֽךָ וְלוֹמְדֵי תוֹרָתֶֽךָ לִשְׁמָהּ׃ בָּרוּךְ אַתָּה יְיָ הַמְלַמֵּד תּוֹרָה לְעַמּוֹ יִשְׂרָאֵל׃
	
	בָּרוּךְ אַתָּה יְיָ אֱלֹהֵֽינוּ מֶֽלֶךְ הָעוֹלָם אֲשֶׁר בָּֽחַר־בָּֽנוּ מִכׇּל־הָעַמִּים וְנָֽתַן־לָֽנוּ אֶת־תּוֹרָתוֹ׃ בָּרוּךְ אַתָּה יְיָ נוֹתֵן הַתּוֹרָה׃\\}

\englishinst{Some postpone recitation of the blessings on the Torah below and recite them before the sacrificial readings on page \pageref{korbanos}.}
%\instruction{יש אומרים ברכות התורה לפני אמירת הקרבנות}
\birkothatorah
יְבָֽרֶכְךָ֥ יְיָ֖ וְיִשְׁמְרֶֽךָ׃ \source{במדבר ו}יָאֵ֨ר יְיָ֧ פָּנָ֛יו אֵלֶ֖יךָ וִֽיחֻנֶּֽךָּ׃ יִשָּׂ֨א יְיָ֤ פָּנָיו֙ אֵלֶ֔יךָ וְיָשֵׂ֥ם לְךָ֖ שָׁלֽוֹם׃\\
אֵֽלּוּ דְבָרִים שֶׁאֵין לָהֶם שִׁעוּר׃ \source{משנה פאה א}הַפֵּאָה וְהַבִּכּוּרִים וְהָרֵאָיוֹן וּגְמִילוּת חֲסָדִים וְתַלְמוּד תּוֹרָה׃\\
אֵֽלּוּ \source{שבת קכז}דְבָרִים שֶׁאָדָם אוֹכֵל פֵּירוֹתֵיהֶם בָּעוֹלָם הַזֶּה וְהַקֶּֽרֶן קַיֶּֽמֶת לוֹ לָעוֹלָם הַבָּא׃ וְאֵֽלּוּ הֵן - כִּבּוּד אָב וָאֵם וּגְמִילוּת חֲסָדִים וְהַשְׁכָּמַת בֵּית הַמִּדְרָשׁ שַׁחֲרִית וְעַרְבִית וְהַכְנָסַת אוֹרְחִים וּבִקּוּר חוֹלִים וְהַכְנָסַת כַּלָּה וְהַלְוָיַת הַמֵּת וְעִיּוּן תְּפִלָּה וַהֲבָאַת שָׁלוֹם בֵּין אָדָם לַחֲבֵרוֹ - וְתַלְמוּד תּוֹרָה כְּנֶֽגֶד כֻּלָּם׃

\englishinst{Stand for recitation of the following blessings:}
\firstword{בָּרוּךְ}
אַתָּה יְיָ אֱלֹהֵֽינוּ מֶֽלֶךְ הָעוֹלָם אֲשֶׁר נָתַן לַשֶּֽׂכְוִי בִינָה לְהַבְחִין בֵּין יוֹם וּבֵין לָֽיְלָה׃\hfill \break
%\firstword{בָּרוּךְ}
%אַתָּה יְיָ אֱלֹהֵֽינוּ מֶֽלֶךְ הָעוֹלָם שֶׁעָשַֽׂנִי יִשְׂרָאֵל׃\hfill\break
\firstword{בָּרוּךְ}
אַתָּה יְיָ אֱלֹהֵֽינוּ מֶֽלֶךְ הָעוֹלָם...\hfill\break
%\begin{small}
	\begin{tabular}{>{\centering\arraybackslash}m{.45\textwidth} | >{\centering\arraybackslash}m{.45\textwidth}}
		
		\instruction{גברים׃} & \instruction{נשים׃}\\% & \instruction{נוסח שויוני׃}\\
		שֶׁלֹּא עָשַֽׂנִי גּוֹי׃
		&
		שֶׁלֹּא עָשַֽׂנִי גּוֹיָה׃
		%& 
		%שֶׁעָשַֽׂנִי בְּצַלְמוֹ׃
		\\
		
		שֶׁלֹּא עָשַׂנִי עְָבֶד׃
		&
		שֶׁלֹּא עָשַׂנִי שִׁפְחָה׃
	%	&
		%\begin{large}שֶׁעָשַֽׂנִי יִשְׂרָאֵל׃\end{large}
		\\
		
		שֶׁלֹּא עָשַֽׂנִי אִשָּׁה׃
		&
		שֶׁעָשַֽׂנִי כִּרְצוֹנוֹ׃
	%	&
		%שֶׁעָשַֽׂנִי בֶּן־/בַּת־חוֹרִין׃
	\end{tabular}
%\end{small}
%(\instruction{יש אומרים:}
%\firstword{בָּרוּךְ}
%אַתָּה יְיָ אֱלֹהֵֽינוּ מֶֽלֶךְ הָעוֹלָם מַגְבִּֽיהַּ שְׁפָלִים׃)\hfill \break
\firstword{בָּרוּךְ}
אַתָּה יְיָ אֱלֹהֵֽינוּ מֶֽלֶךְ הָעוֹלָם פּוֹקֵֽחַ עִוְרִים׃\hfill \break
\firstword{בָּרוּךְ}
אַתָּה יְיָ אֱלֹהֵֽינוּ מֶֽלֶךְ הָעוֹלָם מַלְבִּישׁ עַרֻמִּים׃\hfill \break
\firstword{בָּרוּךְ}
אַתָּה יְיָ אֱלֹהֵֽינוּ מֶֽלֶךְ הָעוֹלָם מַתִּיר אֲסוּרִים׃\hfill \break
\firstword{בָּרוּךְ}
אַתָּה יְיָ אֱלֹהֵֽינוּ מֶֽלֶךְ הָעוֹלָם זוֹקֵף כְּפוּפִים׃\hfill \break
\firstword{בָּרוּךְ}
אַתָּה יְיָ אֱלֹהֵֽינוּ מֶֽלֶךְ הָעוֹלָם רוֹקַע הָאָֽרֶץ עַל־הַמָּֽיִם׃\hfill \break
\firstword{בָּרוּךְ}
אַתָּה יְיָ אֱלֹהֵֽינוּ מֶֽלֶךְ הָעוֹלָם הַמֵּכִין מִצְעֲדֵי־גָֽבֶר׃\hfill \break
\firstword{בָּרוּךְ}
אַתָּה יְיָ אֱלֹהֵֽינוּ מֶֽלֶךְ הָעוֹלָם שֶׁעָֽשָׂה לִי כׇּל־צׇרְכִּי׃\hfill \break
\firstword{בָּרוּךְ}
אַתָּה יְיָ אֱלֹהֵֽינוּ מֶֽלֶךְ הָעוֹלָם אוֹזֵר יִשְׂרָאֵל בִּגְבוּרָה׃\hfill \break
\firstword{בָּרוּךְ}
אַתָּה יְיָ אֱלֹהֵֽינוּ מֶֽלֶךְ הָעוֹלָם עוֹטֵר יִשְׂרָאֵל בְּתִפְאָרָה׃\hfill \break
\firstword{בָּרוּךְ}
אַתָּה יְיָ אֱלֹהֵֽינוּ מֶֽלֶךְ הָעוֹלָם הַנּוֹתֵן לַיָּעֵף כֹּֽחַ׃\hfill

\firstword{בָּרוּךְ}
אַתָּה יְיָ אֱלֹהֵֽינוּ מֶֽלֶךְ הָעוֹלָם הַמַּעֲבִיר שֵׁנָה מֵעֵינָי וּתְנוּמָה מֵעַפְעַפָּי \middot וִיהִי רָצוֹן מִלְּפָנֶֽיךָ יְיָ אֱלֹהֵֽינוּ וֵאלֹהֵי אֲבוֹתֵֽינוּ שֶׁתַּרְגִּילֵֽנוּ בְּתוֹרָתֶֽךָ וְתַדְבִּיקֵֽנוּ בְּמִצְוׂתֶֽיךָ וְאַל תְּבִיאֵֽנוּ לֹא לִידֵי חֵטְא וְלֹא לִידֵי עֲבֵרָה וְעָוׂן וְלֹא לִידֵי נִסָּיוֹן וְלֹא לִידֵי בִזָּיוֹן וְאַל תַּשְׁלֶט־בָּנוּ יֵֽצֶר הָרָע וְהַרְחִיקֵֽנוּ מֵאָדָם רָע וּמֵחָבֵר רָע וְדַבְּקֵֽנוּ בְּיֵֽצֶר טוֹב וּבְמַעֲשִׂים טוֹבִים וְכֹף אֶת־יִצְרֵֽנוּ לְהִשְׁתַּעְבֶּד־לָךְ \middot וּתְנֵנוּ הַיּוֹם וּבְכׇל־יוֹם לְחֵן וּלְחֶסֶד וּלְרַחֲמִים בְּעֵינֶיךָ וּבְעֵינֵי כׇל־רוֹאֵינוּ וְתִגְמְלֵנוּ חֲסָדִים טוֹבִים׃ בָּרוּךְ אַתָּה יְיָ גּוֹמֵל חֲסָדִים טוֹבִים לְעַמּוֹ יִשְׂרָאֵל׃\\
יְהִי רָצוֹן מִלְּפָנֶֽיךָ יְיָ אֱלֹהַי וֵאלֹהֵי אֲבוֹתַי שֶׁתַּצִּילֵֽנִי הַיּוֹם וּבְכׇל־יוֹם מֵעַזֵּי פָנִים וּמֵעַזּוּת פָּנִים מֵאָדָם רַע וּמֵחָבֵר רַע וּמִשָּׁכֵן רַע וּמִפֶּֽגַע רַע וּמִשָּׂטָן הַמַּשְׁחִית מִדִּין קָשֶׁה וּמִבַּֽעַל דִּין קָשֶׁה בֵּין שְׁהוּא בֶן־בְּרִית וּבֵין שֶׁאֵינוֹ בֶן־בְּרִית׃

%\ssubsection{\adforn{18} קבלת עול מלכות שמים \adforn{17}}\\
\instruction{לְעוֹלָם יְהֵא אָדָם יְרֵא שָׁמַיִם בַּסֵּתֶר וּבַגָּלוּי וּמוֹדֶה עַל הָאֱמֶת וְדוֹבֵר אֱמֶת בִּלְבָבוֹ וְיַשְׁכֵּם וְיֹאמַר׃}

\firstword{רִבּוֹן}
כׇּל־הָעוֹלָמִים לֹא עַל־צִדְקוֹתֵֽינוּ אֲנַֽחְנוּ מַפִּילִים תַּחֲנוּנֵֽינוּ לְפָנֶֽיךָ כִּי עַל רַחֲמֶֽיךָ הָרַבִּים \middot מָה אֲנַחְנוּ מֶה חַיֵּֽינוּ מֶה חַסְדֵּֽנוּ מַה־צִּדְקֵֽנוּ מַה־יְּשׁוּעָתֵֽנוּ מַה־כֹּחֵֽנוּ מַה־גְּבוּרָתֵֽנוּ \middot מַה־נֹּאמַר לְפָנֶֽיךָ יְיָ אֱלֹהֵֽינוּ וֵאלֹהֵי אֲבוֹתֵֽינוּ הֲלֹא כׇל־הַגִּבּוֹרִים כְּאַֽיִן לְפָנֶֽיךָ וְאַנְשֵׁי הַשֵּׁם כְּלֹא הָיוּ וַחֲכָמִים כִּבְלִי מַדָּע וּנְבוֹנִים כִּבְלִי הַשְׂכֵּל \middot כִּי רֹב מַעֲשֵׂיהֶם תֹּֽהוּ וִימֵי חַיֵּיהֶם הֶֽבֶל לְפָנֶֽיךָ
וּמוֹתַ֨ר \source{קהלת ג}הָֽאָדָ֤ם מִן־הַבְּהֵמָה֙ אָֽ֔יִן כִּ֥י הַכֹּ֖ל הָֽבֶל׃ \\
\firstword{אֲבָל}
אֲנַֽחְנוּ עַמְּךָ בְּנֵי בְרִיתֶֽךָ בְּנֵי אַבְרָהָם אֹהַבְךָ שֶׁנִּשְׁבַּֽעְתָּ לּוֹ בְּהַר הַמֹּרִיָּה זֶֽרַע יִצְחָק יְחִידוֹ שֶׁנֶּעֱקַד עַל גַּבֵּי הַמִּזְבֵּֽחַ עֲדַת יַעֲקֹב בִּנְךָ בְּכוֹרֶֽךָ שֶׁמֵּאַהֲבָתְךָ שֶׁאָהַֽבְתָּ אֹתוֹ וּמִשִּׂמְחָתְךָ שֶׁשָּׂמַֽחְתָּ בּוֹ קָרָֽאתָ אֶת־שְׁמוֹ יִשְׂרָאֵל וִישֻׁרוּן׃ \\
\firstword{לְפִיכָךְ}
אֲנַֽחְנוּ חַיָּבִים לְהוֹדוֹת לְךָ וּלְשַׁבֵּחֲךָ וּלְפָאֶרְךָ וּלְבָרֵךְ וּלְקַדֵּשׁ וְלִתֵּן־שֶֽׁבַח וְהוֹדָיָה לִשְׁמֶֽךָ \middot אַשְׁרֵֽינוּ מַה־טּוֹב חֶלְקֵֽנוּ וּמַה־נָּעִים גּוֹרָלֵֽינוּ וּמַה־יָּפָה יְרֻשָּׁתֵֽינוּ \middot אַשְׁרֵֽינוּ שֶׁאֲנַֽחְנוּ מַשְׁכִּימִים וּמַעֲרִיבִים עֶֽרֶב וָבֹֽקֶר וְאוֹמְרִים פַּעֲמַֽיִם בְּכׇל־יוֹם׃

\englishinst{If the communal recitation of morning Shema\ayin\space will be after its time, recite the full Shema\ayin\space on page \pageref{morningshema}.}
%\instruction{אם השעה מאוחר קוראים כאן ק״ש בעמ׳ \pageref{morningshema}}
\begin{Large}
	\textbf{שְׁמַ֖ע יִשְׂרָאֵ֑ל יְיָ֥ אֱלֹהֵ֖ינוּ יְיָ֥ ׀ אֶחָֽד׃}
\end{Large}\source{דברים ו}

\instruction{בלחש׃}
\textbf{%
	בָּרוּךְ שֵׁם כְּבוֹד מַלְכוּתוֹ לְעוֹלָם וָעֶד׃
}


\firstword{אַתָּה}
הוּא עַד שֶׁלֹּא נִבְרָא הָעוֹלָם אַתָּה הוּא מִשֶּׁנִּבְרָא הָעוֹלָם \middot אַתָּה הוּא בָּעוֹלָם הַזֶּה וְאַתָּה הוּא לָעוֹלָם הַבָּא \middot קַדֵּשׁ אֶת־שִׁמְךָ עַל מַקְדִּישֵׁי שְׁמֶֽךָ וְקַדֵּשׁ אֶת־שִׁמְךָ בְּעוֹלָמֶֽךָ וּבִישׁוּעָתְךָ תָּרוּם וְתַגְבִּֽיהַּ קַרְנֵֽנוּ׃ בָּרוּךְ אַתָּה יְיָ מְקַדֵּשׁ אֶת־שִׁמְךָ בָּרַבִּים׃

\firstword{אַתָּה}
הוּא יְיָ אֱלֹהֵֽינוּ בַּשָּׁמַֽיִם וּבָאָֽרֶץ וּבִשְׁמֵי הַשָּׁמַֽיִם הָעֶלְיוֹנִים׃ אֱמֶת אַתָּה הוּא רִאשׁוֹן וְאַתָּה הוּא אַחֲרוֹן וּמִבַּלְעָדֶֽיךָ אֵין אֱלֹהִים׃ קַבֵּץ קֹוֶֽיךָ מֵאַרְבַּע כַּנְפוֹת הָאָֽרֶץ׃ יַכִּֽירוּ וְיֵדְעוּ כׇּל־בָּאֵי עוֹלָם
כִּי \source{מל״ב יט}אַתָּה־ה֤וּא הָאֱלֹהִים֙ לְבַדְּךָ֔ לְכֹ֖ל מַמְלְכ֣וֹת הָאָ֑רֶץ אַתָּ֣ה עָשִׂ֔יתָ אֶת־הַשָּׁמַ֖יִם וְאֶת־הָאָֽרֶץ׃ \source{שמות כ}אֶת־הַיָּם וְאֶת־כׇּל־אֲשֶׁר־בָּם׃
וּמִי בְּכׇל־מַעֲשֵׂה יָדֶֽיךָ בָּעֶלְיוֹנִים אוֹ בַתַּחְתּוֹנִים שֶׁיֹּאמַר לְךָ מַה תַּעֲשֶׂה׃\\
אָבִֽינוּ שֶׁבַּשָּׁמַֽיִם עֲשֵׂה עִמָּֽנוּ חֶֽסֶד בַּעֲבוּר שִׁמְךָ הַגָּדוֹל שֶׁנִּקְרָא עָלֵֽינוּ וְקַיֶּם לָֽנוּ יְיָ אֱלֹהֵֽינוּ מַה־שֶׁכָּתוּב׃ \source{צפניה ד}%
בָּעֵ֤ת הַהִיא֙ אָבִ֣יא אֶתְכֶ֔ם וּבָעֵ֖ת קַבְּצִ֣י אֶתְכֶ֑ם כִּֽי־אֶתֵּ֨ן אֶתְכֶ֜ם לְשֵׁ֣ם וְלִתְהִלָּ֗ה בְּכֹל֙ עַמֵּ֣י הָאָֽ֔רֶץ בְּשׁוּבִ֧י אֶת־שְׁבֽוּתֵיכֶ֛ם לְעֵֽינֵיכֶ֖ם אָמַ֥ר יְיָ׃

%\section[קרבנות Sacrifices]{\adforn{18} קרבנות Sacrifices \adforn{17}}
\section[קרבנות]{\adforn{18} קרבנות \adforn{17}}

%\source{שמות ל}
%וַיְדַבֵּ֥ר יְיָ֖ אֶל־מֹשֶׁ֥ה לֵּאמֹֽר׃ וְעָשִׂ֜יתָ כִּיּ֥וֹר נְחֹ֛שֶׁת וְכַנּ֥וֹ נְחֹ֖שֶׁת לְרׇחְצָ֑ה וְנָתַתָּ֣ אֹת֗וֹ בֵּֽין־אֹ֤הֶל מוֹעֵד֙ וּבֵ֣ין הַמִּזְבֵּ֔חַ וְנָתַתָּ֥ שָׁ֖מָּה מָֽיִם׃ וְרָחֲצ֛וּ אַהֲרֹ֥ן וּבָנָ֖יו מִמֶּ֑נּוּ אֶת־יְדֵיהֶ֖ם וְאֶת־רַגְלֵיהֶֽם׃ בְּבֹאָ֞ם אֶל־אֹ֧הֶל מוֹעֵ֛ד יִרְחֲצוּ־מַ֖יִם וְלֹ֣א יָמֻ֑תוּ א֣וֹ בְגִשְׁתָּ֤ם אֶל־הַמִּזְבֵּ֙חַ֙ לְשָׁרֵ֔ת לְהַקְטִ֥יר אִשֶּׁ֖ה לַֽייָ׃ וְרָחֲצ֛וּ יְדֵיהֶ֥ם וְרַגְלֵיהֶ֖ם וְלֹ֣א יָמֻ֑תוּ וְהָיְתָ֨ה לָהֶ֧ם חׇק־עוֹלָ֛ם ל֥וֹ וּלְזַרְע֖וֹ לְדֹרֹתָֽם׃
%\source{ויקרא ו}
%וַיְדַבֵּ֥ר יְיָ֖ אֶל־מֹשֶׁ֥ה לֵּאמֹֽר׃ צַ֤ו אֶֽת־אַהֲרֹן֙ וְאֶת־בָּנָ֣יו לֵאמֹ֔ר זֹ֥את תּוֹרַ֖ת הָעֹלָ֑ה הִ֣וא הָעֹלָ֡ה עַל֩ מוֹקְדָ֨ה עַל־הַמִּזְבֵּ֤חַ כׇּל־הַלַּ֙יְלָה֙ עַד־הַבֹּ֔קֶר וְאֵ֥שׁ הַמִּזְבֵּ֖חַ תּ֥וּקַד בּֽוֹ׃ וְלָבַ֨שׁ הַכֹּהֵ֜ן מִדּ֣וֹ בַ֗ד וּמִֽכְנְסֵי־בַד֮ יִלְבַּ֣שׁ עַל־בְּשָׂרוֹ֒ וְהֵרִ֣ים אֶת־הַדֶּ֗שֶׁן אֲשֶׁ֨ר תֹּאכַ֥ל הָאֵ֛שׁ אֶת־הָעֹלָ֖ה עַל־הַמִּזְבֵּ֑חַ וְשָׂמ֕וֹ אֵ֖צֶל הַמִּזְבֵּֽחַ׃ וּפָשַׁט֙ אֶת־בְּגָדָ֔יו וְלָבַ֖שׁ בְּגָדִ֣ים אֲחֵרִ֑ים וְהוֹצִ֤יא אֶת־הַדֶּ֙שֶׁן֙ אֶל־מִח֣וּץ לַֽמַּחֲנֶ֔ה אֶל־מָק֖וֹם טָהֽוֹר׃ וְהָאֵ֨שׁ עַל־הַמִּזְבֵּ֤חַ תּֽוּקַד־בּוֹ֙ לֹ֣א תִכְבֶּ֔ה וּבִעֵ֨ר עָלֶ֧יהָ הַכֹּהֵ֛ן עֵצִ֖ים בַּבֹּ֣קֶר בַּבֹּ֑קֶר וְעָרַ֤ךְ עָלֶ֙יהָ֙ הָֽעֹלָ֔ה וְהִקְטִ֥יר עָלֶ֖יהָ חֶלְבֵ֥י הַשְּׁלָמִֽים׃ אֵ֗שׁ תָּמִ֛יד תּוּקַ֥ד עַל־הַמִּזְבֵּ֖חַ לֹ֥א תִכְבֶּֽה׃

%\footnote{
	\englishinst{If the blessings on the Torah were not said previously, they are recited now.}%\instruction{אם לא בירך ברכות התורה, חייב לברך קודם אמירת הקרבנות }
{\footnotesize \birkothatorah}
%}
\label{korbanos}
\tamid

%\ketoret
%
%\pitumhaketoret
%תַּנְיָא, רַבִּי נָתָן אוֹמֵר: כְּשֶׁהוּא שׁוֹחֵק, אוֹמֵר הָדֵק הֵיטֵב, הֵיטֵב הָדֵק, מִפְּנֵי שֶׁהַקּוֹל יָפֶה לַבְּשָׂמִים. פִּטְּמָהּ לַחֲצָאִין, כְּשֵׁרָה; לִשְׁלִישׁ וְלִרְבִֽיעַ, לֹא שָׁמַֽעְנוּ. אָמַר רַבִּי יְהוּדָה: זֶה הַכְּלָל: אִם כְּמִדָּתָהּ, כְּשֵׁרָה לַחֲצָאִין; וְאִם חִסַּר אַחַת מִכׇּל־סַמָּנֶֽיהָ, חַיָּב מִיתָה׃

%תַּנְיָא, בַּר קַפָּרָא אוֹמֵר: אַחַת לְשִׁשִּׁים אוֹ לְשִׁבְעִים שָׁנָה הָיְתָה בָאָה שֶׁל שִׁירַֽיִם לַחֲצָאִין. וְעוֹד תָּנֵי בַּר קַפָּרָא: אִלּוּ הָיָה נוֹתֵן בָּהּ קֹרְטוֹב שֶׁל דְּבַשׁ, אֵין אָדָם יָכוֹל לַעֲמֹד מִפְּנֵי רֵיחָהּ; וְלָמָה אֵין מְעָרְבִין בָּהּ דְּבַשׁ, מִפְּנֵי שֶׁהַתּוֹרָה אָמְרָה: כִּ֤י כׇל־שְׂאֹר֙ וְכׇל־דְּבַ֔שׁ לֹֽא־תַקְטִ֧ירוּ מִמֶּ֛נּוּ אִשֶּׁ֖ה לַֽייָֽ׃

%\firstword{אַבַּיֵּי הֲוָה מְסַדֵּר סֵדֶר הַמַּעֲרָכָה} \source{יומא לג} מִשְּׁמָא דִגְמָרָא, וְאַלִּבָּא דְאַבָּא שָׁאוּל, מַעֲרָכָה גְדוֹלָה קוֹדֶמֶת לְמַעֲרָכָה שְׁנִיָּה שֶׁל קְטֹרֶת, וּמַעֲרָכָה שְׁנִיָּה שֶׁל קְטֹרֶת קוֹדֶמֶת לְסִדּוּר שְׁנֵי גִזְרֵי עֵצִים, וְסִדּוּר שְׁנֵי גִזְרֵי עֵצִים קוֹדֶם לְדִשּׁוּן מִזְבֵּחַ הַפְּנִימִי, וְדִשּׁוּן מִזְבֵּחַ הַפְּנִימִי קוֹדֶם לְהַטָבַת חָמֵשׁ נֵרוֹת, וְהַטָבַת חָמֵשׁ נֵרוֹת קוֹדֶמֶת לְדַם הַתָּמִיד, וְדַם הַתָּמִיד קוֹדֶם לְהַטָבַת שְׁתֵּי נֵרוֹת, וְהַטָבַת שְׁתֵּי נֵרוֹת קוֹדֶמֶת לִקְטֹרֶת, וּקְטֹרֶת קוֹדֶמֶת לְאֵבָרִים, וְאֵבָרִים לְמִנְחָה, וּמִנְחָה לַחֲבִתִּין, וַחֲבִתִּין לִנְסָכִין, וּנְסָכִין לְמוּסָפִין, וּמוּסָפִין לְבָזִיכִין, וּבָזִיכִין קוֹדְמִין לְתָמִיד שֶׁל בֵּין הָעַרְבָּיִם. שֶׁנֶּאֱמַר, וְעָרַ֤ךְ \source{ויקרא ו} עָלֶ֙יהָ֙ הָֽעֹלָ֔ה וְהִקְטִ֥יר עָלֶ֖יהָ חֶלְבֵ֥י הַשְּׁלָמִֽים׃ עָלֶיהָ הַשְׁלֵם כׇּל־הַקׇּרְבָּנוֹת כֻּלָּם׃

%\firstword{רִבּוֹן הָעוֹלָמִים,} אַתָּה צִוִּיתָֽנוּ לְהַקְרִיב קׇרְבַּן הַתָּמִיד בְּמוֹעֲדוֹ, וְלִהְיוֹת כֹּהֲנִים בַּעֲבוֹדָתָם, וּלְוִיִּם בְּדוּכָנָם, וְיִשְׂרָאֵל בְּמַעֲמָדָם; וְעַתָּה בַּעֲוֺנוֹתֵֽינוּ חָרֵב בֵּית הַמִּקְדָּשׁ וּבָטֵל הַתָּמִיד, וְאֵין לָֽנוּ לֹא כֹהֵן בַּעֲבוֹדָתוֹ, וְלֹא לֵוִי בְּדוּכָנוֹ, וְלֹא יִשְׂרָאֵל בְּמַעֲמָדוֹ. וְאַתָּה אָמַרְתָּ׃ וּֽנְשַׁלְּמָ֥ה \source{הושע יד} פָרִ֖ים שְׂפָתֵֽינוּ׃ לָכֵן יְהִי רָצוֹן מִלְּפָנֶֽיךָ, יְיָ אֱלֹהֵֽינוּ וֵאלֹהֵי אֲבוֹתֵֽינוּ, שֶׁיְּהֵא שִֽׂיחַ שִׂפְתוֹתֵֽינוּ חָשׁוּב וּמְקֻבָּל וּמְרֻצֶּה לְפָנֶֽיךָ כְּאִלּוּ הִקְרַֽבְנוּ קׇרְבַּן הַתָּמִיד בְּמוֹעֲדוֹ וְעָמַֽדְנוּ עַל מַעֲמָדוֹ.\\
\ifboolexpr{togl {includeshabbat} and (togl {includeweekday} or togl {includefestival})}{\shabbos
}{}
\ifboolexpr{togl {includeshabbat} or togl {includefestival}}{\shabmusafpesukim}{}

\ifboolexpr{togl {includeRCh}}{
\instruction{בראש חודש׃}
\firstword{וּבְרָאשֵׁי֙ חׇדְשֵׁיכֶ֔ם}\source{במדבר כח}
תַּקְרִ֥יבוּ עֹלָ֖ה לַייָ֑ פָּרִ֨ים בְּנֵֽי־בָקָ֤ר שְׁנַ֙יִם֙ וְאַ֣יִל אֶחָ֔ד כְּבָשִׂ֧ים בְּנֵי־שָׁנָ֛ה שִׁבְעָ֖ה תְּמִימִֽם׃
וּשְׁלֹשָׁ֣ה עֶשְׂרֹנִ֗ים סֹ֤לֶת מִנְחָה֙ בְּלוּלָ֣ה בַשֶּׁ֔מֶן לַפָּ֖ר הָאֶחָ֑ד וּשְׁנֵ֣י עֶשְׂרֹנִ֗ים סֹ֤לֶת מִנְחָה֙ בְּלוּלָ֣ה בַשֶּׁ֔מֶן לָאַ֖יִל הָֽאֶחָֽד׃
וְעִשָּׂרֹ֣ן עִשָּׂר֗וֹן סֹ֤לֶת מִנְחָה֙ בְּלוּלָ֣ה בַשֶּׁ֔מֶן לַכֶּ֖בֶשׂ הָאֶחָ֑ד עֹלָה֙ רֵ֣יחַ נִיחֹ֔חַ אִשֶּׁ֖ה לַייָ׃
וְנִסְכֵּיהֶ֗ם חֲצִ֣י הַהִין֩ יִהְיֶ֨ה לַפָּ֜ר וּשְׁלִישִׁ֧ת הַהִ֣ין לָאַ֗יִל וּרְבִיעִ֥ת הַהִ֛ין לַכֶּ֖בֶשׂ יָ֑יִן זֹ֣את עֹלַ֥ת חֹ֙דֶשׁ֙ בְּחׇדְשׁ֔וֹ לְחׇדְשֵׁ֖י הַשָּׁנָֽה׃
וּשְׂעִ֨יר עִזִּ֥ים אֶחָ֛ד לְחַטָּ֖את לַייָ֑ עַל־עֹלַ֧ת הַתָּמִ֛יד יֵעָשֶׂ֖ה וְנִסְכּֽוֹ׃}{}

\firstword{(א) אֵיזֶהוּ מְקוֹמָן}
\source{זבחים פ״ה}%
שֶׁל זְבָחִים? קׇדְשֵׁי קׇדָשִׁים - שְׁחִיטָתָן בַּצָּפוֹן׃ פָּר וְשָׂעִיר שֶׁל יוֹם הַכִּפּוּרִים - שְׁחִיטָתָן בַּצָּפוֹן \middot וְקִבּוּל דָּמָן בִּכְלֵי שָׁרֵת בַּצָּפוֹן \middot וְדָמָן טָעוּן הַזָּיָה עַל־בֵּין הַבַּדִּים וְעַל הַפָּרֹֽכֶת וְעַל־מִזְבַּח הַזָּהָב׃ מַתָּנָה אַֽחַת מֵהֶן מְעַכֶּֽבֶת׃ שְׁיָרֵי הַדָּם הָיָה שׁוֹפֵךְ עַל יְסוֹד מַעֲרָבִי שֶׁל מִזְבֵּחַ הַחִיצוֹן׃ אִם לֹא נָתַן - לֹא עִכֵּב׃

(ב) פָּרִים הַנִּשְׂרָפִים וּשְׂעִירִים הַנִּשְׂרָפִים - שְׁחִיטָתָן בַּצָּפוֹן \middot וְקִבּוּל דָּמָן בִּכְלִי שָׁרֵת בַּצָּפוֹן \middot וְדָמָן טָעוּן הַזָּיָה עַל־הַפָּרֹֽכֶת וְעַל־מִזְבַּח הַזָּהָב׃ מַתָּנָה אַֽחַת מֵהֶן מְעַכֶּֽבֶת׃ שִׁיְרֵי הַדָּם הָיָה שׁוֹפֵךְ עַל יְסוֹד מַעֲרָבִי שֶׁל מִזְבֵּחַ הַחִיצוֹן אִם לֹא נָתַן לֹא עִכֵּב׃ אֵֽלּוּ וָאֵֽלּוּ נִשְׂרָפִין בְּבֵית הַדֶּֽשֶׁן׃

(ג) חַטֹּאת הַצִּבּוּר וְהַיָּחִיד - אֵֽלּוּ הֵן חַטֹּאת הַצִּבּוּר׃ שְׂעִירֵי רָאשֵׁי חֳדָשִׁים וְשֶׁל מוׁעֲדוׂת - שְׁחִיטָתָן בַּצָּפוֹן \middot וְקִבּוּל דָּמָן בִּכְלִי שָׁרֵת בַּצָּפוֹן \middot וְדָמָן טָעוּן אַרְבַּע מַתָּנוֹת עַל אַרְבַּע קְרָנוֹת׃ כֵּיצַד? עָלָה בַכֶּֽבֶשׁ \middot וּפָנָה לַסּוֹבֵב \middot וּבָא לוֹ לְקֶֽרֶן דְּרוֹמִית מִזְרָחִית, מִזְרָחִית צְפוֹנִית, צְפוֹנִית מַעֲרָבִית, מַעֲרָבִית דְּרוֹמִית׃ שִׁיְּרֵי הַדָּם הָיָה שֹׁפֵךְ עַל יְסוֹד דְּרוֹמִי׃ וְנֶאֱכָלִין לִפְנִים מִן־הַקְּלָעִים לְזִכְרֵי כְהֻנָּה בְּכׇל־מַאֲכָל לְיוֹם וָלַֽיְלָה עַד חֲצוֹת׃

(ד) הָעוֹלָה - קֹֽדֶשׁ קׇדָשִׁים - שְׁחִיטָתָהּ בַּצָּפוֹן \middot וְקִבּוּל דָּמָהּ בִּכְלִי שָׁרֵת בַּצָּפוֹן \middot וְדָמָהּ טָעוּן שְׁתֵּי מַתָּנוֹת שֶׁהֵן אַרְבַּע \middot וּטְעוּנָה הֶפְשֵׁט וְנִתּֽוּחַ וְכָלִיל לָאִשִּׁים׃

(ה) זִבְחֵי שַׁלְמֵי צִבּוּר וַאֲשָׁמוֹת - אֵֽלוּ הֵן אֲשָׁמוֹת׃ אֲשַׁם גְּזֵלוֹת, אֲשַׁם מְעִילוֹת, אֲשַׁם שִׁפְחָה חֲרוּפָה, אֲשַׁם נָזִיר, אֲשַׁם מְצוֹרָע, אָשָׁם תָּלוּי - שְׁחִיטָתָן בַּצָּפוֹן \middot וְקִבּוּל דָּמָן בִּכְלִי שָׁרֵת בַּצָּפוֹן \middot וְדָמָן טָעוּן שְׁתֵּי מַתָּנוֹת שֶׁהֵן אַרְבַּע \middot וְנֶאֱכָלִין לִפְנִים מִן הַקְּלָעִים לְזִכְרֵי כְהֻנָּה בְּכׇל־מַאֲכָל לְיוֹם וָלַֽיְלָה עַד חֲצוֹת׃

(ו) הַתּוֹדָה וְאֵיל נָזִיר - קׇדָשִׁים קַלִּים - שְׁחִיטָתָן בְּכׇל־מָקוֹם בָּעֲזָרָה \middot וְדָמָן טָעוּן שְׁתֵּי מַתָּנוֹת שֶׁהֵן אַרְבַּע \middot וְנֶאֱכָלִין בְּכׇל־הָעִיר לְכׇל־אָדָם בְּכׇל־מַאֲכָל לְיוֹם וָלַֽיְלָה עַד חֲצוֹת׃ הַמּוּרָם מֵהֶם - כַּיּוֹצֵא בָהֶם \middot אֶלָּא שֶׁהַמּוּרָם נֶאֱכָל לַכֹּהֲנִים לִנְשֵׁיהֶם וְלִבְנֵיהֶם וּלְעַבְדֵיהֶם׃

(ז) שְׁלָמִים - קׇדָשִׁים קַלִּים - שְׁחִיטָתָן בְּכׇל־מָקוֹם בָּעֲזָרָה \middot וְדָמָן טָעוּן שְׁתֵּי מַתָּנוֹת שֶׁהֵן אַרְבַּע \middot וְנֶאֱכָלִין בְּכׇל־הָעִיר לְכׇל־אָדָם בְּכׇל־מַאֲכָל לִשְׁנֵי יָמִים וְלַֽיְלָה אֶחָד׃ הַמּוּרָם מֵהֶם - כַּיּוֹצֵא בָהֶן \middot אֶלָּא שֶׁהַמּוּרָם נֶאֱכָל לַכֹּהֲנִים לִנְשֵׁיהֶם וְלִבְנֵיהֶם וּלְעַבְדֵיהֶם׃

(ח) הַבְּכוֹר וְהַמַּעֲשֵׂר וְהַפֶּֽסַח - קׇדָשִׁים קַלִּים - שְׁחִיטָתָן בְּכׇל־מָקוֹם בָּעֲזָרָה \middot וְדָמָן טָעוּן מַתָּנָה אֶחָת - וּבִלְבָד שֶׁיִּתֵּן כְּנֶֽגֶד הַיְסוֹד׃ שִׁנָּה בַאֲכִילָתָן \middot הַבְּכוֹר נֶאֱכָל לַכֹּהֲנִים \middot וְהַמַּעֲשֵׂר לְכׇל־אָדָם \middot וְנֶּאֱכָלִין בְּכׇל־הָעִיר בְּכׇל־מַאֲכָל לִשְׁנֵי יָמִים וְלַֽיְלָה אֶחָד׃ הַפֶּֽסַח אֵינוֹ נֶאֱכָל אֶלָּא בַלַּֽיְלָה וְאֵינוֹ נֶאֱכָל אֶלָּא עַד־חֲצוֹת וְאֵינוֹ נֶאֱכָל אֶלָּא לִמְנוּיָיו וְאֵינוֹ נֶאֱכָל אֶלָּא צָלִי׃

\firstword{רַבִּי יִשְׁמָעֵאל אוֹמֵר}\source{ספרא ויקרא}
בִּשְׁלֹשׁ עֶשְׂרֵה מִדּוֹת הַתּוֹרָה נִדְרֶֽשֶׁת׃\hfill \break
(א) מִקַּל וָחֹמֶר (ב) וּמִגְּזֵרָה שָׁוָה (ג) מִבִּנְיַן אָב מִכָּתוּב אֶחָד \middot וּמִבִּנְיַן אָב מִשְּׁנֵי כְתוּבִים (ד) מִכְּלָל וּפְרָט (ה) וּמִפְּרָט וּכְלָל (ו) כְּלָל וּפְרָט וּכְלָל \middot אֵי אַתָּה דָן אֶלָּא כְּעֵין הַפְּרָט (ז) מִכְּלָל שֶׁהוּא צָרִיךְ לִפְרָט \middot וּמִפְּרָט שֶׁהוּא צָרִיךְ לִכְלָל (ח) כׇּל־דָּבָר שֶׁהָיָה בִּכְלָל וְיָצָא מִן הַכְּלָל לְלַמֵּד \middot לֹא לְלַמֵּד עַל עַצְמוֹ יָצָא אֶלָּא לְלַמֵּד עַל הַכְּלָל כֻּלּוֹ יָצָא (ט) כׇּל־דָּבָר שֶׁהָיָה בִּכְלָל וְיָצָא לִטְעוֹן טַעַן אֶחָד שֶׁהוּא כְעִנְיָנוֹ \middot יָצָא לְהָקֵל וְלֹא לְהַחֲמִיר (י) כׇּל־דָּבָר שֶׁהָיָה בִּכְלָל וְיָצָא לִטְעוֹן טַעַן אַחֵר שֶׁלֹּא כְעִנְיָנוֹ \middot יָצָא לְהָקֵל וּלְהַחֲמִיר (יא) כׇּל־דָּבָר שֶׁהָיָה בִּכְלָל וְיָצָא לִדּוֹן בְּדָבָר חָדָשׁ \middot אֵי אַתָּה יָכוֹל לְהַחֲזִירוֹ לִכְלָלוֹ עַד שֶׁיַּחֲזִירֶנּוּ הַכָּתוּב לִכְלָלוֹ בְּפֵרוּשׁ (יב) דָּבָר הַלָּמֵד מֵעִנְיָנוֹ \middot וְדָבָר הַלָּמֵד מִסּוֹפוֹ (יג) וְכֵן שְׁנֵי כְתוּבִים הַמַּכְחִישִׁים זֶה אֶת־זֶה \middot עַד שֶׁיָּבוֹא הַכָּתוּב הַשְּׁלִישִׁי וְיַכְרִיעַ בֵּינֵיהֶם׃

יְהִי רָצוֹן מִלְּפָנֶֽיךָ יְיָ אֱלֹהֵֽינוּ וֵאלֹהֵי אֲבוֹתֵֽינוּ שֶׁיִּבָּנֶה בֵּית הַמִּקְדָּשׁ בִּמְהֵרָה בְיָמֵֽינוּ וְתֵן חֶלְקֵֽנוּ בְּתוֹרָתֶֽךָ \middot וְשָׁם נַעֲבׇדְךָ בְּיִרְאָה כִּימֵי עוֹלָם וּכְשָׁנִים קַדְמֹנִיּוֹת׃

\rabbiskaddish

%\section[כניסה לבהכ״נ Synagogue at Arriving]{\adforn{53} Synagogue at Arriving \adforn{25}\\כניסה לבהכ״נ }
\section[כניסה לבהכ״נ]{\adforn{53} כניסה לבהכ״נ \adforn{25}}

\englishinst{On entering the synagogue:}
\firstword{מַה־טֹּ֥בוּ}\source{במדבר נה}
אֹהָלֶ֖יךָ יַעֲקֹ֑ב מִשְׁכְּנֹתֶ֖יךָ יִשְׂרָאֵֽל׃
וַאֲנִ֗י\source{תהלים ה}
בְּרֹ֣ב חַ֭סְדְּךָ אָב֣וֹא בֵיתֶ֑ךָ אֶשְׁתַּחֲוֶ֥ה אֶל־הֵיכַל־קׇ֝דְשְׁךָ֗ בְּיִרְאָתֶֽךָ׃\\
\source{תהלים כו}
יְיָ֗ אָ֭הַבְתִּי מְע֣וֹן בֵּיתֶ֑ךָ וּ֝מְק֗וֹם מִשְׁכַּ֥ן כְּבוֹדֶֽךָ׃
וַאֲנִי אֶשְׁתַּחֲוֶה וְאֶכְרָֽעָה אֶבְרְכָה לִפְנֵי־יְיָ עֹשִׂי׃
וַאֲנִ֤י
\source{תהלים סט}%
תְפִלָּֽתִי־לְךָ֨ ׀ יְיָ֡ עֵ֤ת רָצ֗וֹן אֱלֹהִ֥ים בְּרׇב־חַסְדֶּ֑ךָ עֲ֝נֵ֗נִי בֶּאֱמֶ֥ת יִשְׁעֶֽךָ׃

{\footnotesize 
שִׁ֥יר\source{תהילים קכב} הַֽמַּעֲל֗וֹת לְדָ֫וִ֥ד שָׂ֭מַחְתִּי בְּאֹמְרִ֣ים לִ֑י בֵּ֖ית יְיָ֣ נֵלֵֽךְ׃ עֹ֭מְדוֹת הָי֣וּ רַגְלֵ֑ינוּ בִּ֝שְׁעָרַ֗יִךְ יְרוּשָׁלָֽ͏ִם׃ יְרוּשָׁלַ֥͏ִם הַבְּנוּיָ֑ה כְּ֝עִ֗יר שֶׁחֻבְּרָה־לָּ֥הּ יַחְדָּֽו׃ שֶׁשָּׁ֨ם עָל֪וּ שְׁבָטִ֡ים שִׁבְטֵי־יָ֭הּ עֵד֣וּת לְיִשְׂרָאֵ֑ל לְ֝הֹד֗וֹת לְשֵׁ֣ם יְיָ׃ כִּ֤י שָׁ֨מָּה ׀ יָשְׁב֣וּ כִסְא֣וֹת לְמִשְׁפָּ֑ט כִּ֝סְא֗וֹת לְבֵ֣ית דָּוִֽד׃ שַׁ֭אֲלוּ שְׁל֣וֹם יְרוּשָׁלָ֑͏ִם יִ֝שְׁלָ֗יוּ אֹהֲבָֽיִךְ׃ יְהִי־שָׁל֥וֹם בְּחֵילֵ֑ךְ שַׁ֝לְוָ֗ה בְּאַרְמְנוֹתָֽיִךְ׃ לְ֭מַעַן אַחַ֣י וְרֵעָ֑י אֲדַבְּרָה־נָּ֖א שָׁל֣וֹם בָּֽךְ׃ לְ֭מַעַן בֵּית־יְיָ֣ אֱלֹהֵ֑ינוּ אֲבַקְשָׁ֖ה ט֣וֹב לָֽךְ׃}


\ssubsection{\adforn{18} עטיפת טלית \adforn{17}}
\\
\englishinst{On putting on the talli\thav\space gadol:}
\firstword{בָּרוּךְ}
אַתָּה יְיָ אֱלֹהֵֽינוּ מֶֽלֶךְ הָעוֹלָם \middot אֲשֶׁר קִדְּשָֽׁנוּ בְּמִצְוֹתָיו וְצִוָּֽנוּ לְהִתְעַטֵּף בַּצִּיצִת׃\\
\begin{footnotesize}
	מַה־יָּקָ֥ר\source{תהילים לו} חַסְדְּךָ֗ אֱלֹ֫הִ֥ים וּבְנֵ֥י אָדָ֑ם בְּצֵ֥ל כְּ֝נָפֶ֗יךָ יֶחֱסָיֽוּן׃ יִ֭רְוְיֻן מִדֶּ֣שֶׁן בֵּיתֶ֑ךָ וְנַ֖חַל עֲדָנֶ֣יךָ תַשְׁקֵֽם׃ כִּֽי־עִ֭מְּךָ מְק֣וֹר חַיִּ֑ים בְּ֝אוֹרְךָ֗ נִרְאֶה־אֽוֹר׃ מְשֹׁ֣ךְ חַ֭סְדְּךָ לְיֹדְעֶ֑יךָ וְ֝צִדְקָֽתְךָ֗ לְיִשְׁרֵי־לֵֽב׃\\
	יְהִי רָצוֹן מִלְּפָנֶֽיךָ יְיָ אֱלֹהֵֽינוּ וֵאלֹהֵי אֲבוֹתֵֽינוּ. שֶׁתְּהֵא חֲשׁוּבָה מִצְוַת צִיצִת זוֹ כְּאִלּוּ קִיַּמְתִּֽיהָ בְּכׇל־פְּרָטֶֽיהָ וְדִקְדּוּקֶֽיהָ וְכַוָּנוֹתֶֽיהָ וְתַרְיָ"ג מִצְוֺת הַתְּלוּיִם בָּהּ. אָמֵן סֶֽלָה. 
\end{footnotesize}

\ifboolexpr{togl {includeweekday}}{
\ssubsection{\adforn{18} הנחת תפילין \adforn{17}}\\
\englishinst{Before laying on tefillin on the arm:}
%\instruction{לפני הנחת תפילין של יד׃ }
\firstword{בָּרוּךְ}
אַתָּה יְיָ אֱלֹהֵֽינוּ מֶֽלֶךְ הָעוֺלָם \middot אֲשֶׁר קִדְּשָֽׁנוּ בְּמִצְוֹתָיו וְצִוָֽנוּ לְהַנִּֽיחַ תְּפִלִּין׃

\englishinst{Before laying on the tefillin on the head:}
%\instruction{לפני הנחת תפילין של ראש׃ }
\firstword{בָּרוּךְ}
אַתָּה יְיָ, אֱלֹהֵֽינוּ מֶֽלֶךְ הָעוֺלָם \middot אֲשֶׁר קִדְּשָֽׁנוּ בְּמִצְוֹתָיו וְצִוָֽנוּ עַל־מִצְוַת תְּפִלִּין׃

\englishinst{After the tefillin is secured on the head:}
%\instruction{אחרי הנחת תפילין של ראש׃ }
בָּרוּךְ שֵׁם כְּבוֺד מַלְכוּתוֺ לְעוֺלָם וָעֶד׃

\englishinst{As the tefillin is wrapped around the fingers:}
%\instruction{כשקושר הרצועה על היד׃ }\\
וְאֵרַשְׂתִּ֥יךְ \source{הושע ב}לִ֖י לְעוֹלָ֑ם וְאֵרַשְׂתִּ֥יךְ לִי֙ בְּצֶ֣דֶק וּבְמִשְׁפָּ֔ט וּבְחֶ֖סֶד וּֽבְרַחֲמִֽים׃ וְאֵרַשְׂתִּ֥יךְ לִ֖י בֶּאֱמוּנָ֑ה וְיָדַ֖עַתְּ אֶת־יְיָ׃

\englishinst{Some say the following after laying on the tefillin:}
%\instruction{יש אומרים אחר הנחת תפילין:}\\
יְהִי רָצוֹן מִלְּפָנֶֽיךָ יְיָ אֱלֹהֵֽינוּ וֵאלֹהֵי אֲבוֹתֵֽינוּ שֶׁתְּהֵא חֲשׁוּבָה מִצְוַת הֲנָחַת תְּפִלִּין זוֹ כְּאִלּוּ קִיַּמְתִּֽיהָ בְּכׇל־פְּרָטֶֽיהָ וְדִקְדּוּקֶֽיהָ וְכַוׇּנוֹתֶֽיהָ וְתַרְיַ"ג מִצְוֺת הַתְּלוּיִם בָּהּ׃ אָמֵן סֶֽלָה׃

\begin{footnotesize}	
וַיְדַבֵּ֥ר יְיָ֖ אֶל־מֹשֶׁ֥ה לֵּאמֹֽר׃ \source{שמות יג}קַדֶּשׁ־לִ֨י כׇל־בְּכ֜וֹר פֶּ֤טֶר כׇּל־רֶ֙חֶם֙ בִּבְנֵ֣י יִשְׂרָאֵ֔ל בָּאָדָ֖ם וּבַבְּהֵמָ֑ה לִ֖י הֽוּא׃ וַיֹּ֨אמֶר מֹשֶׁ֜ה אֶל־הָעָ֗ם זָכ֞וֹר אֶת־הַיּ֤וֹם הַזֶּה֙ אֲשֶׁ֨ר יְצָאתֶ֤ם מִמִּצְרַ֙יִם֙ מִבֵּ֣ית עֲבָדִ֔ים כִּ֚י בְּחֹ֣זֶק יָ֔ד הוֹצִ֧יא יְיָ֛ אֶתְכֶ֖ם מִזֶּ֑ה וְלֹ֥א יֵאָכֵ֖ל חָמֵֽץ׃ הַיּ֖וֹם אַתֶּ֣ם יֹצְאִ֑ים בְּחֹ֖דֶשׁ הָאָבִֽיב׃ וְהָיָ֣ה כִֽי־יְבִיאֲךָ֣ יְיָ֡ אֶל־אֶ֣רֶץ הַֽ֠כְּנַעֲנִ֠י וְהַחִתִּ֨י וְהָאֱמֹרִ֜י וְהַחִוִּ֣י וְהַיְבוּסִ֗י אֲשֶׁ֨ר נִשְׁבַּ֤ע לַאֲבֹתֶ֙יךָ֙ לָ֣תֶת לָ֔ךְ אֶ֛רֶץ זָבַ֥ת חָלָ֖ב וּדְבָ֑שׁ וְעָבַדְתָּ֛ אֶת־הָעֲבֹדָ֥ה הַזֹּ֖את בַּחֹ֥דֶשׁ הַזֶּֽה׃ שִׁבְעַ֥ת יָמִ֖ים תֹּאכַ֣ל מַצֹּ֑ת וּבַיּוֹם֙ הַשְּׁבִיעִ֔י חַ֖ג לַייָ׃ מַצּוֹת֙ יֵֽאָכֵ֔ל אֵ֖ת שִׁבְעַ֣ת הַיָּמִ֑ים וְלֹֽא־יֵרָאֶ֨ה לְךָ֜ חָמֵ֗ץ וְלֹֽא־יֵרָאֶ֥ה לְךָ֛ שְׂאֹ֖ר בְּכׇל־גְּבֻלֶֽךָ׃ וְהִגַּדְתָּ֣ לְבִנְךָ֔ בַּיּ֥וֹם הַה֖וּא לֵאמֹ֑ר בַּעֲב֣וּר זֶ֗ה עָשָׂ֤ה יְיָ֙ לִ֔י בְּצֵאתִ֖י מִמִּצְרָֽיִם׃ וְהָיָה֩ לְךָ֨ לְא֜וֹת עַל־יָדְךָ֗ וּלְזִכָּרוֹן֙ בֵּ֣ין עֵינֶ֔יךָ לְמַ֗עַן תִּהְיֶ֛ה תּוֹרַ֥ת יְיָ֖ בְּפִ֑יךָ כִּ֚י בְּיָ֣ד חֲזָקָ֔ה הוֹצִֽאֲךָ֥ יְיָ֖ מִמִּצְרָֽיִם׃ וְשָׁמַרְתָּ֛ אֶת־הַחֻקָּ֥ה הַזֹּ֖את לְמוֹעֲדָ֑הּ מִיָּמִ֖ים יָמִֽימָה׃\hfill\break
וְהָיָ֞ה כִּֽי־יְבִאֲךָ֤ יְיָ֙ אֶל־אֶ֣רֶץ הַֽכְּנַעֲנִ֔י כַּאֲשֶׁ֛ר נִשְׁבַּ֥ע לְךָ֖ וְלַֽאֲבֹתֶ֑יךָ וּנְתָנָ֖הּ לָֽךְ׃ וְהַעֲבַרְתָּ֥ כׇל־פֶּֽטֶר־רֶ֖חֶם לַֽייָ֑ וְכׇל־פֶּ֣טֶר ׀ שֶׁ֣גֶר בְּהֵמָ֗ה אֲשֶׁ֨ר יִהְיֶ֥ה לְךָ֛ הַזְּכָרִ֖ים לַייָ׃ וְכׇל־פֶּ֤טֶר חֲמֹר֙ תִּפְדֶּ֣ה בְשֶׂ֔ה וְאִם־לֹ֥א תִפְדֶּ֖ה וַעֲרַפְתּ֑וֹ וְכֹ֨ל בְּכ֥וֹר אָדָ֛ם בְּבָנֶ֖יךָ תִּפְדֶּֽה׃ וְהָיָ֞ה כִּֽי־יִשְׁאָלְךָ֥ בִנְךָ֛ מָחָ֖ר לֵאמֹ֣ר מַה־זֹּ֑את וְאָמַרְתָּ֣ אֵלָ֔יו בְּחֹ֣זֶק יָ֗ד הוֹצִיאָ֧נוּ יְיָ֛ מִמִּצְרַ֖יִם מִבֵּ֥ית עֲבָדִֽים׃ וַיְהִ֗י כִּֽי־הִקְשָׁ֣ה פַרְעֹה֮ לְשַׁלְּחֵ֒נוּ֒ וַיַּהֲרֹ֨ג יְיָ֤ כׇּל־בְּכוֹר֙ בְּאֶ֣רֶץ מִצְרַ֔יִם מִבְּכֹ֥ר אָדָ֖ם וְעַד־בְּכ֣וֹר בְּהֵמָ֑ה עַל־כֵּן֩ אֲנִ֨י זֹבֵ֜חַ לַֽייָ֗ כׇּל־פֶּ֤טֶר רֶ֙חֶם֙ הַזְּכָרִ֔ים וְכׇל־בְּכ֥וֹר בָּנַ֖י אֶפְדֶּֽה׃ וְהָיָ֤ה לְאוֹת֙ עַל־יָ֣דְכָ֔ה וּלְטוֹטָפֹ֖ת בֵּ֣ין עֵינֶ֑יךָ כִּ֚י בְּחֹ֣זֶק יָ֔ד הוֹצִיאָ֥נוּ יְיָ֖ מִמִּצְרָֽיִם׃
\end{footnotesize}

}{}
\vspace{0.4in}
\chapter[פסוקי דזמרא]{\adforn{47} פסוקי דזמרא \adforn{19}}
\vspace{0.15in}
\newcommand{\longPDZ}{
	לַמְנַצֵּ֗חַ\source{תהילים יט} מִזְמ֥וֹר לְדָוִֽד׃ הַשָּׁמַ֗יִם מְֽסַפְּרִ֥ים כְּבֽוֹד־אֵ֑ל וּֽמַעֲשֵׂ֥ה יָ֝דָ֗יו מַגִּ֥יד הָרָקִֽיעַ׃ י֣וֹם לְ֭יוֹם יַבִּ֣יעַֽ אֹ֑מֶר וְלַ֥יְלָה לְּ֝לַ֗יְלָה יְחַוֶּה־דָּֽעַת׃ אֵֽין־אֹ֭מֶר וְאֵ֣ין דְּבָרִ֑ים בְּ֝לִ֗י נִשְׁמָ֥ע קוֹלָֽם׃ בְּכׇל־הָאָ֨רֶץ ׀ יָ֘צָ֤א קַוָּ֗ם וּבִקְצֵ֣ה תֵ֭בֵל מִלֵּיהֶ֑ם לַ֝שֶּׁ֗מֶשׁ שָֽׂם־אֹ֥הֶל בָּהֶֽם׃ וְה֗וּא כְּ֭חָתָן יֹצֵ֣א מֵחֻפָּת֑וֹ יָשִׂ֥ישׂ כְּ֝גִבּ֗וֹר לָר֥וּץ אֹֽרַח׃ מִקְצֵ֤ה הַשָּׁמַ֨יִם ׀ מֽוֹצָא֗וֹ וּתְקוּפָת֥וֹ עַל־קְצוֹתָ֑ם וְאֵ֥ין נִ֝סְתָּ֗ר מֵחַמָּתֽוֹ׃ תּ֘וֹרַ֤ת יְיָ֣ תְּ֭מִימָה מְשִׁ֣יבַת נָ֑פֶשׁ עֵד֥וּת יְיָ֥ נֶ֝אֱמָנָ֗ה מַחְכִּ֥ימַת פֶּֽתִי׃ פִּקּ֘וּדֵ֤י יְיָ֣ יְ֭שָׁרִים מְשַׂמְּחֵי־לֵ֑ב מִצְוַ֥ת יְיָ֥ בָּ֝רָ֗ה מְאִירַ֥ת עֵינָֽיִם׃ יִרְאַ֤ת יְיָ֨ ׀ טְהוֹרָה֮ עוֹמֶ֢דֶת לָ֫עַ֥ד מִֽשְׁפְּטֵי־יְיָ֥ אֱמֶ֑ת צָֽדְק֥וּ יַחְדָּֽו׃ הַֽנֶּחֱמָדִ֗ים מִ֭זָּהָב וּמִפַּ֣ז רָ֑ב וּמְתוּקִ֥ים מִ֝דְּבַ֗שׁ וְנֹ֣פֶת צוּפִֽים׃ גַּֽם־עַ֭בְדְּךָ נִזְהָ֣ר בָּהֶ֑ם בְּ֝שׇׁמְרָ֗ם עֵ֣קֶב רָֽב׃ שְׁגִיא֥וֹת מִֽי־יָבִ֑ין מִֽנִּסְתָּר֥וֹת נַקֵּֽנִי׃ גַּ֤ם מִזֵּדִ֨ים ׀ חֲשֹׂ֬ךְ עַבְדֶּ֗ךָ אַֽל־יִמְשְׁלוּ־בִ֣י אָ֣ז אֵיתָ֑ם וְ֝נִקֵּ֗יתִי מִפֶּ֥שַֽׁע רָֽב׃ יִ֥הְיֽוּ לְרָצ֨וֹן ׀ אִמְרֵי־פִ֡י וְהֶגְי֣וֹן לִבִּ֣י לְפָנֶ֑יךָ יְ֝יָ֗ צוּרִ֥י וְגֹאֲלִֽי׃
	
	
	\enlargethispage{\baselineskip}
	
	לְדָוִ֗ד\source{תהילים לד} בְּשַׁנּוֹת֣וֹ אֶת־טַ֭עְמוֹ לִפְנֵ֣י אֲבִימֶ֑לֶךְ וַ֝יְגָרְשֵׁ֗הוּ וַיֵּלַֽךְ׃ אֲבָרְכָ֣ה אֶת־יְיָ֣ בְּכׇל־עֵ֑ת תָּ֝מִ֗יד תְּֽהִלָּת֥וֹ בְּפִֽי׃ בַּייָ֭ תִּתְהַלֵּ֣ל נַפְשִׁ֑י יִשְׁמְע֖וּ עֲנָוִ֣ים וְיִשְׂמָֽחוּ׃ גַּדְּל֣וּ לַייָ֣ אִתִּ֑י וּנְרוֹמְמָ֖ה שְׁמ֣וֹ יַחְדָּֽו׃ דָּרַ֣שְׁתִּי אֶת־יְיָ֣ וְעָנָ֑נִי וּמִכׇּל־מְ֝גוּרוֹתַ֗י הִצִּילָֽנִי׃ הִבִּ֣יטוּ אֵלָ֣יו וְנָהָ֑רוּ וּ֝פְנֵיהֶ֗ם אַל־יֶחְפָּֽרוּ׃ זֶ֤ה עָנִ֣י קָ֭רָא וַייָ֣ שָׁמֵ֑עַ וּמִכׇּל־צָ֝רוֹתָ֗יו הוֹשִׁיעֽוֹ׃ חֹנֶ֤ה מַלְאַךְ־יְיָ֓ סָ֘בִ֤יב לִירֵאָ֗יו וַֽיְחַלְּצֵֽם׃ טַעֲמ֣וּ וּ֭רְאוּ כִּֽי־ט֣וֹב יְיָ֑ אַֽשְׁרֵ֥י הַ֝גֶּ֗בֶר יֶחֱסֶה־בּֽוֹ׃ יְר֣אוּ אֶת־יְיָ֣ קְדֹשָׁ֑יו כִּי־אֵ֥ין מַ֝חְס֗וֹר לִירֵאָֽיו׃ כְּ֭פִירִים רָשׁ֣וּ וְרָעֵ֑בוּ וְדֹרְשֵׁ֥י יְ֝יָ֗ לֹא־יַחְסְר֥וּ כׇל־טֽוֹב׃ לְֽכוּ־בָ֭נִים שִׁמְעוּ־לִ֑י יִֽרְאַ֥ת יְ֝יָ֗ אֲלַמֶּדְכֶֽם׃ מִֽי־הָ֭אִישׁ הֶחָפֵ֣ץ חַיִּ֑ים אֹהֵ֥ב יָ֝מִ֗ים לִרְא֥וֹת טֽוֹב׃ נְצֹ֣ר לְשׁוֹנְךָ֣ מֵרָ֑ע וּ֝שְׂפָתֶ֗יךָ מִדַּבֵּ֥ר מִרְמָֽה׃ ס֣וּר מֵ֭רָע וַעֲשֵׂה־ט֑וֹב בַּקֵּ֖שׁ שָׁל֣וֹם וְרׇדְפֵֽהוּ׃ עֵינֵ֣י יְיָ֭ אֶל־צַדִּיקִ֑ים וְ֝אׇזְנָ֗יו אֶל־שַׁוְעָתָֽם׃ פְּנֵ֣י יְיָ֭ בְּעֹ֣שֵׂי רָ֑ע לְהַכְרִ֖ית מֵאֶ֣רֶץ זִכְרָֽם׃ צָ֭עֲקוּ וַייָ֣ שָׁמֵ֑עַ וּמִכׇּל־צָ֝רוֹתָ֗ם הִצִּילָֽם׃ קָר֣וֹב יְיָ֭ לְנִשְׁבְּרֵי־לֵ֑ב וְֽאֶת־דַּכְּאֵי־ר֥וּחַ יוֹשִֽׁיעַ׃ רַ֭בּוֹת רָע֣וֹת צַדִּ֑יק וּ֝מִכֻּלָּ֗ם יַצִּילֶ֥נּוּ יְיָ׃ שֹׁמֵ֥ר כׇּל־עַצְמוֹתָ֑יו אַחַ֥ת מֵ֝הֵ֗נָּה לֹ֣א נִשְׁבָּֽרָה׃ תְּמוֹתֵ֣ת רָשָׁ֣ע רָעָ֑ה וְשֹׂנְאֵ֖י צַדִּ֣יק יֶאְשָֽׁמוּ׃ פֹּדֶ֣ה יְיָ֭ נֶ֣פֶשׁ עֲבָדָ֑יו וְלֹ֥א יֶ֝אְשְׁמ֗וּ כׇּֽל־הַחֹסִ֥ים בּֽוֹ׃
	
	
	תְּפִלָּה֮\source{תהילים צ} לְמֹשֶׁ֢ה אִֽישׁ־הָאֱלֹ֫הִ֥ים אֲֽדֹנָ֗י מָע֣וֹן אַ֭תָּה הָיִ֥יתָ לָּ֗נוּ בְּדֹ֣ר וָדֹֽר׃ בְּטֶ֤רֶם ׀ הָ֘רִ֤ים יֻלָּ֗דוּ וַתְּח֣וֹלֵֽל אֶ֣רֶץ וְתֵבֵ֑ל וּֽמֵעוֹלָ֥ם עַד־ע֝וֹלָ֗ם אַתָּ֥ה אֵֽל׃ תָּשֵׁ֣ב אֱ֭נוֹשׁ עַד־דַּכָּ֑א וַ֝תֹּ֗אמֶר שׁ֣וּבוּ בְנֵֽי־אָדָֽם׃ כִּ֤י אֶ֪לֶף שָׁנִ֡ים בְּֽעֵינֶ֗יךָ כְּי֣וֹם אֶ֭תְמוֹל כִּ֣י יַֽעֲבֹ֑ר וְאַשְׁמוּרָ֥ה בַלָּֽיְלָה׃ זְ֭רַמְתָּם שֵׁנָ֣ה יִהְי֑וּ בַּ֝בֹּ֗קֶר כֶּחָצִ֥יר יַחֲלֹֽף׃ בַּ֭בֹּקֶר יָצִ֣יץ וְחָלָ֑ף לָ֝עֶ֗רֶב יְמוֹלֵ֥ל וְיָבֵֽשׁ׃ כִּֽי־כָלִ֥ינוּ בְאַפֶּ֑ךָ וּֽבַחֲמָתְךָ֥ נִבְהָֽלְנוּ׃ שַׁתָּ֣ עֲוֺנֹתֵ֣ינוּ לְנֶגְדֶּ֑ךָ עֲ֝לֻמֵ֗נוּ לִמְא֥וֹר פָּנֶֽיךָ׃ כִּ֣י כׇל־יָ֭מֵינוּ פָּנ֣וּ בְעֶבְרָתֶ֑ךָ כִּלִּ֖ינוּ שָׁנֵ֣ינוּ כְמוֹ־הֶֽגֶה׃ יְמֵֽי־שְׁנוֹתֵ֨ינוּ בָהֶ֥ם שִׁבְעִ֪ים שָׁנָ֡ה וְאִ֤ם בִּגְבוּרֹ֨ת ׀ שְׁמ֘וֹנִ֤ים שָׁנָ֗ה וְ֭רׇהְבָּם עָמָ֣ל וָאָ֑וֶן כִּי־גָ֥ז חִ֝֗ישׁ וַנָּעֻֽפָה׃ מִֽי־י֭וֹדֵעַ עֹ֣ז אַפֶּ֑ךָ וּ֝כְיִרְאָתְךָ֗ עֶבְרָתֶֽךָ׃ לִמְנ֣וֹת יָ֭מֵינוּ כֵּ֣ן הוֹדַ֑ע וְ֝נָבִ֗א לְבַ֣ב חׇכְמָֽה׃ שׁוּבָ֣ה יְיָ֭ עַד־מָתָ֑י וְ֝הִנָּחֵ֗ם עַל־עֲבָדֶֽיךָ׃ שַׂבְּעֵ֣נוּ בַבֹּ֣קֶר חַסְדֶּ֑ךָ וּֽנְרַנְּנָ֥ה וְ֝נִשְׂמְחָ֗ה בְּכׇל־יָמֵֽינוּ׃ שַׂ֭מְּחֵנוּ כִּימ֣וֹת עִנִּיתָ֑נוּ שְׁ֝נ֗וֹת רָאִ֥ינוּ רָעָֽה׃ יֵרָאֶ֣ה אֶל־עֲבָדֶ֣יךָ פׇעֳלֶ֑ךָ וַ֝הֲדָרְךָ֗ עַל־בְּנֵיהֶֽם׃ וִיהִ֤י ׀ נֹ֤עַם אֲדֹנָ֥י אֱלֹהֵ֗ינוּ עָ֫לֵ֥ינוּ וּמַעֲשֵׂ֣ה יָ֭דֵינוּ כּוֹנְנָ֥ה עָלֵ֑ינוּ וּֽמַעֲשֵׂ֥ה יָ֝דֵ֗ינוּ כּוֹנְנֵֽהוּ׃
	
	\tzadialeph
	
	הַ֥לְלוּ־יָ֨הּ\source{תהילים קלה} ׀ הַֽ֭לְלוּ אֶת־שֵׁ֣ם יְיָ֑ הַֽ֝לְל֗וּ עַבְדֵ֥י יְיָ׃ שֶׁ֣֭עֹמְדִים בְּבֵ֣ית יְיָ֑ בְּ֝חַצְר֗וֹת בֵּ֣ית אֱלֹהֵֽינוּ׃ הַֽלְלוּ־יָ֭הּ כִּֽי־ט֣וֹב יְיָ֑ זַמְּר֥וּ לִ֝שְׁמ֗וֹ כִּ֣י נָעִֽים׃ כִּֽי־יַעֲקֹ֗ב בָּחַ֣ר ל֣וֹ יָ֑הּ יִ֝שְׂרָאֵ֗ל לִסְגֻלָּתֽוֹ׃ כִּ֤י אֲנִ֣י יָ֭דַעְתִּי כִּֽי־גָד֣וֹל יְיָ֑ וַ֝אֲדֹנֵ֗ינוּ מִכׇּל־אֱלֹהִֽים׃ כֹּ֤ל אֲשֶׁר־חָפֵ֥ץ יְיָ֗ עָ֫שָׂ֥ה בַּשָּׁמַ֥יִם וּבָאָ֑רֶץ בַּ֝יַּמִּ֗ים וְכׇל־תְּהֹמֽוֹת׃ מַעֲלֶ֣ה נְשִׂאִים֮ מִקְצֵ֢ה הָ֫אָ֥רֶץ בְּרָקִ֣ים לַמָּטָ֣ר עָשָׂ֑ה מֽוֹצֵא־ר֝֗וּחַ מֵאֽוֹצְרוֹתָֽיו׃ שֶׁ֭הִכָּה בְּכוֹרֵ֣י מִצְרָ֑יִם מֵ֝אָדָ֗ם עַד־בְּהֵמָֽה׃ שָׁלַ֤ח ׀ אוֹתֹ֣ת וּ֭מֹפְתִים בְּתוֹכֵ֣כִי מִצְרָ֑יִם בְּ֝פַרְעֹ֗ה וּבְכׇל־עֲבָדָֽיו׃ שֶׁ֭הִכָּה גּוֹיִ֣ם רַבִּ֑ים וְ֝הָרַ֗ג מְלָכִ֥ים עֲצוּמִֽים׃ לְסִיח֤וֹן ׀ מֶ֤לֶךְ הָאֱמֹרִ֗י וּ֭לְעוֹג מֶ֣לֶךְ הַבָּשָׁ֑ן וּ֝לְכֹ֗ל מַמְלְכ֥וֹת כְּנָֽעַן׃ וְנָתַ֣ן אַרְצָ֣ם נַחֲלָ֑ה נַ֝חֲלָ֗ה לְיִשְׂרָאֵ֥ל עַמּֽוֹ׃ יְיָ֭ שִׁמְךָ֣ לְעוֹלָ֑ם יְ֝יָ֗ זִכְרְךָ֥ לְדֹר־וָדֹֽר׃ כִּֽי־יָדִ֣ין יְיָ֣ עַמּ֑וֹ וְעַל־עֲ֝בָדָ֗יו יִתְנֶחָֽם׃ עֲצַבֵּ֣י הַ֭גּוֹיִם כֶּ֣סֶף וְזָהָ֑ב מַ֝עֲשֵׂ֗ה יְדֵ֣י אָדָֽם׃ פֶּֽה־לָ֭הֶם וְלֹ֣א יְדַבֵּ֑רוּ עֵינַ֥יִם לָ֝הֶ֗ם וְלֹ֣א יִרְאֽוּ׃ אׇזְנַ֣יִם לָ֭הֶם וְלֹ֣א יַאֲזִ֑ינוּ אַ֝֗ף אֵין־יֶשׁ־ר֥וּחַ בְּפִיהֶֽם׃ כְּ֭מוֹהֶם יִהְי֣וּ עֹשֵׂיהֶ֑ם כֹּ֖ל אֲשֶׁר־בֹּטֵ֣חַ בָּהֶֽם׃ בֵּ֣ית יִ֭שְׂרָאֵל בָּרְכ֣וּ אֶת־יְיָ֑ בֵּ֥ית אַ֝הֲרֹ֗ן בָּרְכ֥וּ אֶת־יְיָ׃ בֵּ֣ית הַ֭לֵּוִי בָּרְכ֣וּ אֶת־יְיָ֑ יִֽרְאֵ֥י יְ֝יָ֗ בָּרְכ֥וּ אֶת־יְיָ׃ בָּ֘ר֤וּךְ יְיָ֨ ׀ מִצִּיּ֗וֹן שֹׁ֘כֵ֤ן יְֽרוּשָׁלָ֗͏ִם הַֽלְלוּ־יָֽהּ׃
	
	
	הוֹד֣וּ לַייָ֣ כִּי־ט֑וֹב\source{תהלים קלו} \hfill
	כִּ֖י לְעוֹלָ֣ם חַסְדּֽוֹ׃\\
	ה֭וֹדוּ לֵאלֹהֵ֣י הָאֱלֹהִ֑ים \hfill כִּ֖י לְעוֹלָ֣ם חַסְדּֽוֹ׃\\
	ה֭וֹדוּ לַאֲדֹנֵ֣י הָאֲדֹנִ֑ים \hfill כִּ֖י לְעוֹלָ֣ם חַסְדּֽוֹ׃\\
	לְעֹ֘שֵׂ֤ה נִפְלָא֣וֹת גְּדֹל֣וֹת לְבַדּ֑וֹ \hfill כִּ֖י לְעוֹלָ֣ם חַסְדּֽוֹ׃\\
	לְעֹשֵׂ֣ה הַ֭שָּׁמַיִם בִּתְבוּנָ֑ה \hfill כִּ֖י לְעוֹלָ֣ם חַסְדּֽוֹ׃\\
	לְרֹקַ֣ע הָ֭אָרֶץ עַל־הַמָּ֑יִם \hfill כִּ֖י לְעוֹלָ֣ם חַסְדּֽוֹ׃\\
	לְ֭עֹשֵׂה אוֹרִ֣ים גְּדֹלִ֑ים \hfill כִּ֖י לְעוֹלָ֣ם חַסְדּֽוֹ׃\\
	אֶת־הַ֭שֶּׁמֶשׁ לְמֶמְשֶׁ֣לֶת בַּיּ֑וֹם \hfill כִּ֖י לְעוֹלָ֣ם חַסְדּֽוֹ׃\\
	אֶת־הַיָּרֵ֣חַ וְ֭כוֹכָבִים \hfill\break לְמֶמְשְׁל֣וֹת בַּלָּ֑יְלָה \hfill כִּ֖י לְעוֹלָ֣ם חַסְדּֽוֹ׃\\
	לְמַכֵּ֣ה מִ֭צְרַיִם בִּבְכוֹרֵיהֶ֑ם \hfill כִּ֖י לְעוֹלָ֣ם חַסְדּֽוֹ׃\\
	וַיּוֹצֵ֣א יִ֭שְׂרָאֵל מִתּוֹכָ֑ם \hfill כִּ֖י לְעוֹלָ֣ם חַסְדּֽוֹ׃\\
	בְּיָ֣ד חֲ֭זָקָה וּבִזְר֣וֹעַ נְטוּיָ֑ה \hfill כִּ֖י לְעוֹלָ֣ם חַסְדּֽוֹ׃\\
	לְגֹזֵ֣ר יַם־ס֭וּף לִגְזָרִ֑ים \hfill כִּ֖י לְעוֹלָ֣ם חַסְדּֽוֹ׃\\
	וְהֶעֱבִ֣יר יִשְׂרָאֵ֣ל בְּתוֹכ֑וֹ \hfill כִּ֖י לְעוֹלָ֣ם חַסְדּֽוֹ׃\\
	וְנִ֘עֵ֤ר פַּרְעֹ֣ה וְחֵיל֣וֹ בְיַם־ס֑וּף \hfill כִּ֖י לְעוֹלָ֣ם חַסְדּֽוֹ׃\\
	לְמוֹלִ֣יךְ עַ֭מּוֹ בַּמִּדְבָּ֑ר \hfill כִּ֖י לְעוֹלָ֣ם חַסְדּֽוֹ׃\\
	לְ֭מַכֵּה מְלָכִ֣ים גְּדֹלִ֑ים \hfill כִּ֖י לְעוֹלָ֣ם חַסְדּֽוֹ׃\\
	וַֽ֭יַּהֲרֹג מְלָכִ֣ים אַדִּירִ֑ים \hfill כִּ֖י לְעוֹלָ֣ם חַסְדּֽוֹ׃\\
	לְ֭סִיחוֹן מֶ֣לֶךְ הָאֱמֹרִ֑י \hfill כִּ֖י לְעוֹלָ֣ם חַסְדּֽוֹ׃\\
	וּ֭לְעוֹג מֶ֣לֶךְ הַבָּשָׁ֑ן \hfill כִּ֖י לְעוֹלָ֣ם חַסְדּֽוֹ׃\\
	וְנָתַ֣ן אַרְצָ֣ם לְנַחֲלָ֑ה \hfill כִּ֖י לְעוֹלָ֣ם חַסְדּֽוֹ׃\\
	נַ֭חֲלָה לְיִשְׂרָאֵ֣ל עַבְדּ֑וֹ \hfill כִּ֖י לְעוֹלָ֣ם חַסְדּֽוֹ׃\\
	שֶׁ֭בְּשִׁפְלֵנוּ זָ֣כַר לָ֑נוּ \hfill כִּ֖י לְעוֹלָ֣ם חַסְדּֽוֹ׃\\
	וַיִּפְרְקֵ֥נוּ מִצָּרֵ֑ינוּ \hfill כִּ֖י לְעוֹלָ֣ם חַסְדּֽוֹ׃\\
	נֹתֵ֣ן לֶ֭חֶם לְכׇל־בָּשָׂ֑ר \hfill כִּ֖י לְעוֹלָ֣ם חַסְדּֽוֹ׃\\
	ה֭וֹדוּ לְאֵ֣ל הַשָּׁמָ֑יִם \hfill כִּ֖י לְעוֹלָ֣ם חַסְדּֽוֹ׃\\
	
	
	רַנְּנ֣וּ\source{תהילים לג} צַ֭דִּיקִים בַּייָ֑ לַ֝יְשָׁרִ֗ים נָאוָ֥ה תְהִלָּֽה׃ הוֹד֣וּ לַייָ֣ בְּכִנּ֑וֹר בְּנֵ֥בֶל עָ֝שׂ֗וֹר זַמְּרוּ־לֽוֹ׃ שִֽׁירוּ־ל֭וֹ שִׁ֣יר חָדָ֑שׁ הֵיטִ֥יבוּ נַ֝גֵּ֗ן בִּתְרוּעָֽה׃ כִּֽי־יָשָׁ֥ר דְּבַר־יְיָ֑ וְכׇל־מַ֝עֲשֵׂ֗הוּ בֶּאֱמוּנָֽה׃ אֹ֭הֵב צְדָקָ֣ה וּמִשְׁפָּ֑ט חֶ֥סֶד יְ֝יָ֗ מָלְאָ֥ה הָאָֽרֶץ׃ בִּדְבַ֣ר יְיָ֭ שָׁמַ֣יִם נַעֲשׂ֑וּ וּבְר֥וּחַ פִּ֝֗יו כׇּל־צְבָאָֽם׃ כֹּנֵ֣ס כַּ֭נֵּד מֵ֣י הַיָּ֑ם נֹתֵ֖ן בְּאוֹצָר֣וֹת תְּהוֹמֽוֹת׃ יִֽירְא֣וּ מֵ֭ייָ כׇּל־הָאָ֑רֶץ מִמֶּ֥נּוּ יָ֝ג֗וּרוּ כׇּל־יֹשְׁבֵ֥י תֵבֵֽל׃ כִּ֤י ה֣וּא אָמַ֣ר וַיֶּ֑הִי הֽוּא־צִ֝וָּ֗ה וַֽיַּעֲמֹֽד׃ יְיָ֗ הֵפִ֥יר עֲצַת־גּוֹיִ֑ם הֵ֝נִ֗יא מַחְשְׁב֥וֹת עַמִּֽים׃ עֲצַ֣ת יְיָ֭ לְעוֹלָ֣ם תַּעֲמֹ֑ד מַחְשְׁב֥וֹת לִ֝בּ֗וֹ לְדֹ֣ר וָדֹֽר׃ אַשְׁרֵ֣י הַ֭גּוֹי אֲשֶׁר־יְיָ֣ אֱלֹהָ֑יו הָעָ֓ם ׀ בָּחַ֖ר לְנַחֲלָ֣ה לֽוֹ׃ מִ֭שָּׁמַיִם הִבִּ֣יט יְיָ֑ רָ֝אָ֗ה אֶֽת־כׇּל־בְּנֵ֥י הָאָדָֽם׃ מִֽמְּכוֹן־שִׁבְתּ֥וֹ הִשְׁגִּ֑יחַ אֶ֖ל כׇּל־יֹשְׁבֵ֣י הָאָֽרֶץ׃ הַיֹּצֵ֣ר יַ֣חַד לִבָּ֑ם הַ֝מֵּבִ֗ין אֶל־כׇּל־מַעֲשֵׂיהֶֽם׃ אֵֽין־הַ֭מֶּלֶךְ נוֹשָׁ֣ע בְּרׇב־חָ֑יִל גִּ֝בּ֗וֹר לֹא־יִנָּצֵ֥ל בְּרׇב־כֹּֽחַ׃ שֶׁ֣קֶר הַ֭סּוּס לִתְשׁוּעָ֑ה וּבְרֹ֥ב חֵ֝יל֗וֹ לֹ֣א יְמַלֵּֽט׃ הִנֵּ֤ה עֵ֣ין יְיָ֭ אֶל־יְרֵאָ֑יו לַֽמְיַחֲלִ֥ים לְחַסְדּֽוֹ׃ לְהַצִּ֣יל מִמָּ֣וֶת נַפְשָׁ֑ם וּ֝לְחַיּוֹתָ֗ם בָּרָעָֽב׃ נַ֭פְשֵׁנוּ חִכְּתָ֣ה לַֽייָ֑ עֶזְרֵ֖נוּ וּמָגִנֵּ֣נוּ הֽוּא׃ כִּי־ב֭וֹ יִשְׂמַ֣ח לִבֵּ֑נוּ כִּ֤י בְשֵׁ֖ם קׇדְשׁ֣וֹ בָטָֽחְנוּ׃ יְהִי־חַסְדְּךָ֣ יְיָ֣ עָלֵ֑ינוּ כַּ֝אֲשֶׁ֗ר יִחַ֥לְנוּ לָֽךְ׃
	
	\mizmorshabbat
	
	יְיָ֣ מָלָךְ֮ גֵּא֢וּת לָ֫בֵ֥שׁ\source{תהלים צג}
	לָבֵ֣שׁ יְיָ֭ עֹ֣ז הִתְאַזָּ֑ר אַף־תִּכּ֥וֹן תֵּ֝בֵ֗ל בַּל־תִּמּֽוֹט׃
	נָכ֣וֹן כִּסְאֲךָ֣ מֵאָ֑ז מֵעוֹלָ֣ם אָֽתָּה׃
	נָשְׂא֤וּ נְהָר֨וֹת ׀ יְיָ֗ נָשְׂא֣וּ נְהָר֣וֹת קוֹלָ֑ם יִשְׂא֖וּ נְהָר֣וֹת דׇּכְיָֽם׃
	מִקֹּל֨וֹת ׀ מַ֤יִם רַבִּ֗ים אַדִּירִ֣ים מִשְׁבְּרֵי־יָ֑ם אַדִּ֖יר בַּמָּר֣וֹם יְיָ׃
	עֵֽדֹתֶ֨יךָ ׀ נֶאֶמְנ֬וּ מְאֹ֗ד לְבֵיתְךָ֥ נַאֲוָה־קֹ֑דֶשׁ יְ֝יָ֗ לְאֹ֣רֶךְ יָמִֽים׃
}

\newcommand{\todah}{
מִזְמ֥וֹר\source{תהילים ק} לְתוֹדָ֑ה הָרִ֥יעוּ לַ֝ייָ֗ כׇּל־הָאָֽרֶץ׃ עִבְד֣וּ אֶת־יְיָ֣ בְּשִׂמְחָ֑ה בֹּ֥אוּ לְ֝פָנָ֗יו בִּרְנָנָֽה׃ דְּע֗וּ כִּֽי־יְיָ ה֤וּא אֱלֹ֫הִ֥ים הֽוּא־עָ֭שָׂנוּ (ולא) [וְל֣וֹ] אֲנַ֑חְנוּ עַ֝מּ֗וֹ וְצֹ֣אן מַרְעִיתֽוֹ׃ בֹּ֤אוּ שְׁעָרָ֨יו ׀ בְּתוֹדָ֗ה חֲצֵרֹתָ֥יו בִּתְהִלָּ֑ה הוֹדוּ־ל֝֗וֹ בָּרְכ֥וּ שְׁמֽוֹ׃ כִּי־ט֣וֹב יְיָ֭ לְעוֹלָ֣ם חַסְדּ֑וֹ וְעַד־דֹּ֥ר וָ֝דֹ֗ר אֱמוּנָתֽוֹ׃}

\instruction{יש אומרים מזמור זו ואח״כ קדיש יתום}\\\chanukat

\firstword{בָּרוּךְ שֶׁאָמַר}
וְהָיָה הָעוֹלָם בָּרוּךְ הוּא׃
בָּרוּךְ עוֹשֶׂה בְרֵאשִׁית בָּרוּךְ אוֹמֵר וְעוֹשֶׂה׃
בָּרוּךְ גּוֹזֵר וּמְקַיֵּם בָּרוּךְ מְרַחֵם עַל הָאָֽרֶץ׃
בָּרוּךְ מְרַחֵם עַל הַבְּרִיּוֹת בָּרוּךְ מְשַׁלֵּם שָׂכָר טוֹב לִירֵאָיו׃
בָּרוּךְ חַי לָעַד וְקַיָּם לָנֶֽצַח בָּרוּךְ פּוֹדֶה וּמַצִּיל בָּרוּךְ שְׁמוֹ׃
בָּרוּךְ אַתָּה יְיָ אֱלֹהֵֽינוּ מֶֽלֶךְ הָעוֹלָם הָאֵל אָב הָרַחֲמָן הַמְהֻלָּל בְּפִי עַמּוֹ מְשֻׁבָּח וּמְפֹאָר בִּלְשׁוֹן חֲסִידָיו וַעֲבָדָיו וּבְשִׁירֵי דָוִד עַבְדֶּֽךָ נְהַלֶּלְךָ יְיָ אֱלֹהֵֽינוּ בִּשְׁבָחוֹת וּבִזְמִירוֹת׃ נְגַדֶּלְךָ וּנְשַׁבֵּחֲךָ וּנְפָאֶרְךָ וְנַמְלִיכְךָ וְנַזְכִּיר שִׁמְךָ מַלְכֵּֽנוּ אֱלֹהֵֽינוּ׃
יָחִיד חֵי הָעוֹלָמִים מֶֽלֶךְ מְשֻׁבָּח וּמְפֹאָר עֲדֵי עַד שְׁמוֹ הַגָּדוֹל׃ בָּרוּךְ אַתָּה יְיָ מֶֽלֶךְ מְהֻלָּל בַּתֻּשְׁבָּחוֹת׃

\firstword{הוֹד֤וּ}
לַֽייָ֙ קִרְא֣וּ בִשְׁמ֔וֹ\source{דה״א טז}
הוֹדִ֥יעוּ בָעַמִּ֖ים עֲלִילֹתָֽיו׃
שִׁ֤ירוּ לוֹ֙ זַמְּרוּ־ל֔וֹ שִׂ֖יחוּ בְּכׇל־נִפְלְאֹתָֽיו׃
הִֽתְהַלְלוּ֙ בְּשֵׁ֣ם קׇדְשׁ֔וֹ יִשְׂמַ֕ח לֵ֖ב מְבַקְשֵׁ֥י יְיָ׃
דִּרְשׁ֤וּ יְיָ֙ וְעֻזּ֔וֹ בַּקְּשׁ֥וּ פָנָ֖יו תָּמִֽיד׃
זִכְר֗וּ נִפְלְאֹתָיו֙ אֲשֶׁ֣ר עָשָׂ֔ה מֹפְתָ֖יו וּמִשְׁפְּטֵי־פִֽיהוּ׃
זֶ֚רַע יִשְׂרָאֵ֣ל עַבְדּ֔וֹ בְּנֵ֥י יַעֲקֹ֖ב בְּחִירָֽיו׃
ה֚וּא יְיָ֣ אֱלֹהֵ֔ינוּ בְּכׇל־הָאָ֖רֶץ מִשְׁפָּטָֽיו׃
זִכְר֤וּ לְעוֹלָם֙ בְּרִית֔וֹ דָּבָ֥ר צִוָּ֖ה לְאֶ֥לֶף דּֽוֹר׃
אֲשֶׁ֤ר כָּרַת֙ אֶת־אַבְרָהָ֔ם וּשְׁבוּעָת֖וֹ לְיִצְחָֽק׃
וַיַּעֲמִידֶ֤הָ לְיַֽעֲקֹב֙ לְחֹ֔ק לְיִשְׂרָאֵ֖ל בְּרִ֥ית עוֹלָֽם׃
לֵאמֹ֗ר לְךָ֙ אֶתֵּ֣ן אֶֽרֶץ־כְּנָ֔עַן חֶ֖בֶל נַחֲלַתְכֶֽם׃
בִּהְיֽוֹתְכֶם֙ מְתֵ֣י מִסְפָּ֔ר כִּמְעַ֖ט וְגָרִ֥ים בָּֽהּ׃
וַיִּֽתְהַלְּכוּ֙ מִגּ֣וֹי אֶל־גּ֔וֹי וּמִמַּמְלָכָ֖ה אֶל־עַ֥ם אַחֵֽר׃
לֹֽא־הִנִּ֤יחַ לְאִישׁ֙ לְעׇשְׁקָ֔ם וַיּ֥וֹכַח עֲלֵיהֶ֖ם מְלָכִֽים׃
אַֽל־תִּגְּעוּ֙ בִּמְשִׁיחָ֔י וּבִנְבִיאַ֖י אַל־תָּרֵֽעוּ׃
שִׁ֤ירוּ לַֽייָ֙ כׇּל־הָאָ֔רֶץ בַּשְּׂר֥וּ מִיּֽוֹם־אֶל־י֖וֹם יְשׁוּעָתֽוֹ׃
סַפְּר֤וּ בַגּוֹיִם֙ אֶת־כְּבוֹד֔וֹ בְּכׇל־הָעַמִּ֖ים נִפְלְאֹתָֽיו׃
כִּי֩ גָד֨וֹל יְיָ֤ וּמְהֻלָּל֙ מְאֹ֔ד וְנוֹרָ֥א ה֖וּא עַל־כׇּל־אֱלֹהִֽים׃
כִּ֠י כׇּל־אֱלֹהֵ֤י הָֽעַמִּים֙ אֱלִילִ֔ים...וַייָ֖ שָׁמַ֥יִם עָשָֽׂה׃

ה֤וֹד וְהָדָר֙ לְפָנָ֔יו עֹ֥ז וְחֶדְוָ֖ה בִּמְקֹמֽוֹ׃
הָב֤וּ לַֽייָ֙ מִשְׁפְּח֣וֹת עַמִּ֔ים הָב֥וּ לַייָ֖ כָּב֥וֹד וָעֹֽז׃
הָב֥וּ לַֽייָ֖ כְּב֣וֹד שְׁמ֑וֹ שְׂא֤וּ מִנְחָה֙ וּבֹ֣אוּ לְפָנָ֔יו
הִשְׁתַּחֲו֥וּ לַֽייָ֖ בְּהַדְרַת־קֹֽדֶשׁ׃ חִ֤ילוּ מִלְּפָנָיו֙ כׇּל־הָאָ֔רֶץ
אַף־תִּכּ֥וֹן תֵּבֵ֖ל בַּל־תִּמּֽוֹט׃ יִשְׂמְח֤וּ הַשָּׁמַ֙יִם֙ וְתָגֵ֣ל הָאָ֔רֶץ
וְיֹאמְר֥וּ בַגּוֹיִ֖ם יְיָ֥ מָלָֽךְ׃ יִרְעַ֤ם הַיָּם֙ וּמְלוֹא֔וֹ
יַעֲלֹ֥ץ הַשָּׂדֶ֖ה וְכׇל־אֲשֶׁר־בּֽוֹ׃ אָ֥ז יְרַנְּנ֖וּ עֲצֵ֣י הַיָּ֑עַר
מִלִּפְנֵ֣י יְיָ֔ כִּי־בָ֖א לִשְׁפּ֥וֹט אֶת־הָאָֽרֶץ׃ הוֹד֤וּ לַֽייָ֙ כִּ֣י ט֔וֹב
כִּ֥י לְעוֹלָ֖ם חַסְדּֽוֹ׃ וְאִמְר֕וּ הוֹשִׁיעֵ֙נוּ֙ אֱלֹהֵ֣י יִשְׁעֵ֔נוּ
וְקַבְּצֵ֥נוּ וְהַצִּילֵ֖נוּ מִן־הַגּוֹיִ֑ם לְהֹדוֹת֙ לְשֵׁ֣ם קׇדְשֶׁ֔ךָ
לְהִשְׁתַּבֵּ֖חַ בִּתְהִלָּתֶֽךָ׃ בָּר֤וּךְ יְיָ֙ אֱלֹהֵ֣י יִשְׂרָאֵ֔ל
מִן־הָעוֹלָ֖ם וְעַ֣ד הָעֹלָ֑ם וַיֹּאמְר֤וּ כׇל־הָעָם֙ אָמֵ֔ן וְהַלֵּ֖ל לַייָ׃\\

רוֹמְמ֡וּ\source{תהילים צט} יְ֘יָ֤ אֱלֹהֵ֗ינוּ וְֽ֭הִשְׁתַּחֲווּ לַהֲדֹ֥ם רַגְלָ֗יו קָד֥וֹשׁ הֽוּא׃
רוֹמְמ֡וּ\source{תהילים צט} יְ֘יָ֤ אֱלֹהֵ֗ינוּ וְֽ֭הִשְׁתַּחֲווּ לְהַ֣ר קׇדְשׁ֑וֹ כִּי־קָ֝ד֗וֹשׁ יְיָ֥ אֱלֹהֵֽינוּ׃
\\
וְה֤וּא\source{תהילים עח} רַח֨וּם ׀ יְכַפֵּ֥ר עָוֺן֮ וְֽלֹא־יַֽ֫שְׁחִ֥ית וְ֭הִרְבָּה לְהָשִׁ֣יב אַפּ֑וֹ וְלֹא־יָ֝עִ֗יר כׇּל־חֲמָתֽוֹ׃
אַתָּ֤ה\source{תהילים מ} יְיָ֗ לֹֽא־תִכְלָ֣א רַחֲמֶ֣יךָ מִמֶּ֑נִּי חַסְדְּךָ֥ וַ֝אֲמִתְּךָ֗ תָּמִ֥יד יִצְּרֽוּנִי׃
זְכֹר־רַחֲמֶ֣יךָ\source{תהילים כה} יְיָ֭ וַחֲסָדֶ֑יךָ כִּ֖י מֵעוֹלָ֣ם הֵֽמָּה׃
תְּנ֥וּ\source{תהילים סח} עֹ֗ז לֵאלֹ֫הִ֥ים עַֽל־יִשְׂרָאֵ֥ל גַּאֲוָת֑וֹ וְ֝עֻזּ֗וֹ בַּשְּׁחָקִֽים׃ נ֤וֹרָ֥א אֱלֹהִ֗ים מִֽמִּקְדָּ֫שֶׁ֥יךָ אֵ֤ל יִשְׂרָאֵ֗ל ה֤וּא נֹתֵ֨ן ׀ עֹ֖ז וְתַעֲצֻמ֥וֹת לָעָ֗ם בָּר֥וּךְ אֱלֹהִֽים׃
אֵל־נְקָמ֥וֹת\source{תהילים צד} יְיָ֑ אֵ֖ל נְקָמ֣וֹת הוֹפִֽיעַ׃ הִ֭נָּשֵׂא שֹׁפֵ֣ט הָאָ֑רֶץ הָשֵׁ֥ב גְּ֝מ֗וּל עַל־גֵּאִֽים׃
לַֽייָ֥\source{תהילים ג} הַיְשׁוּעָ֑ה עַֽל־עַמְּךָ֖ בִרְכָתֶ֣ךָ סֶּֽלָה׃
יְיָ֣\source{תהילים מו} צְבָא֣וֹת עִמָּ֑נוּ מִשְׂגָּֽב־לָ֨נוּ אֱלֹהֵ֖י יַֽעֲקֹ֣ב סֶֽלָה׃
יְיָ֥\source{תהילים פד} צְבָא֑וֹת אַֽשְׁרֵ֥י אָ֝דָ֗ם בֹּטֵ֥חַ בָּֽךְ׃
יְיָ֥\source{תהילים כ} הוֹשִׁ֑יעָה הַ֝מֶּ֗לֶךְ יַעֲנֵ֥נוּ בְיוֹם־קׇרְאֵֽנוּ׃
\\
הוֹשִׁ֤יעָה\source{תהילים כח} ׀ אֶת־עַמֶּ֗ךָ וּבָרֵ֥ךְ אֶת־נַחֲלָתֶ֑ךָ וּֽרְעֵ֥ם וְ֝נַשְּׂאֵ֗ם עַד־הָעוֹלָֽם׃
נַ֭פְשֵׁנוּ\source{תהילים לג} חִכְּתָ֣ה לַֽייָ֑ עֶזְרֵ֖נוּ וּמָגִנֵּ֣נוּ הֽוּא׃ כִּי־ב֭וֹ יִשְׂמַ֣ח לִבֵּ֑נוּ כִּ֤י בְשֵׁ֖ם קׇדְשׁ֣וֹ בָטָֽחְנוּ׃ יְהִי־חַסְדְּךָ֣ יְיָ֣ עָלֵ֑ינוּ כַּ֝אֲשֶׁ֗ר יִחַ֥לְנוּ לָֽךְ׃
הַרְאֵ֣נוּ\source{תהילים פה} יְיָ֣ חַסְדֶּ֑ךָ וְ֝יֶשְׁעֲךָ֗ תִּתֶּן־לָֽנוּ׃
ק֭וּמָֽה\source{תהילים מד} עֶזְרָ֣תָה לָּ֑נוּ וּ֝פְדֵ֗נוּ לְמַ֣עַן חַסְדֶּֽךָ׃
אָֽנֹכִ֨י\source{תהילים פא} ׀ יְ֘יָ֤ אֱלֹהֶ֗יךָ הַֽ֭מַּעַלְךָ מֵאֶ֣רֶץ מִצְרָ֑יִם הַרְחֶב־פִּ֝֗יךָ וַאֲמַלְאֵֽהוּ׃
אַשְׁרֵ֣י\source{תהילים קמד} הָ֭עָם שֶׁכָּ֣כָה לּ֑וֹ אַֽשְׁרֵ֥י הָ֝עָ֗ם שֱׁייָ֥ אֱלֹהָֽיו׃
וַאֲנִ֤י\source{תהילים יג} ׀ בְּחַסְדְּךָ֣ בָטַחְתִּי֮ יָ֤גֵ֥ל לִבִּ֗י בִּֽישׁוּעָ֫תֶ֥ךָ אָשִׁ֥ירָה לַֽייָ֑ כִּ֖י גָמַ֣ל עָלָֽי׃

\negline

\ifboolexpr{togl {includeweekday} and (togl {includeshabbat} or togl {includefestival})}{ 
\begin{narrow}
\instruction{א״א מזמור לתודה בשבת, ביו״ט, בערב יום כפור, בערב פסח, בחול המועד פסח}
\todah
\end{narrow}
}{\ifboolexpr{togl {includeweekday} and not togl{includeshabbat} and not togl {includefestival}}{\todah}{}}

\ifboolexpr{(togl {includeshabbat} or togl {includefestival}) and (togl {includeweekday} or togl {includeChM})} {\instruction{בשבת וביו״ט ובהשענא רבא אומרים פסד״ז ארוך, בשאר ימים ממשיכים בעמ׳ \pageref{yehikvod}}
	\begin{narrow}	\longPDZ\end{narrow}} {
\ifboolexpr{(togl {includeshabbat} or togl {includefestival}) and not togl {includeweekday}} {\longPDZ} {}}


\label{yehikvod}
\firstword{יְהִ֤י כְב֣וֹד}\source{תהלים קד}
יְיָ֣ לְעוֹלָ֑ם יִשְׂמַ֖ח יְיָ֣ בְּמַעֲשָֽׂיו׃
\source{תהלים קיג}יְהִ֤י שֵׁ֣ם יְיָ֣ מְבֹרָ֑ךְ מֵ֝עַתָּ֗ה וְעַד־עוֹלָֽם׃
מִמִּזְרַח־שֶׁ֥מֶשׁ עַד־מְבוֹא֑וֹ מְ֝הֻלָּ֗ל שֵׁ֣ם יְיָ׃
רָ֖ם עַל־כׇּל־גּוֹיִ֥ם ׀ יְיָ֑ עַ֖ל הַשָּׁמַ֣יִם כְּבוֹדֽוֹ׃
\source{תהלים קלה} יְיָ֭ שִׁמְךָ֣ לְעוֹלָ֑ם יְ֝יָ֗ זִכְרְךָ֥ לְדֹר־וָדֹֽר׃
\source{תהלים קג} יְיָ֗ בַּ֭שָּׁמַיִם הֵכִ֣ין כִּסְא֑וֹ וּ֝מַלְכוּת֗וֹ בַּכֹּ֥ל מָשָֽׁלָה׃
\source{ד״ה א טז} יִשְׂמְח֤וּ הַשָּׁמַ֙יִם֙ וְתָגֵ֣ל הָאָ֔רֶץ וְיֹאמְר֥וּ בַגּוֹיִ֖ם יְיָ֥ מָלָֽךְ׃
\melekhmalakhyimlokh
\source{תהלים י} יְיָ֣ מֶ֭לֶךְ עוֹלָ֣ם וָעֶ֑ד אָבְד֥וּ ג֝וֹיִ֗ם מֵאַרְצֽוֹ׃
\source{תהלים לג} יְיָ֗ הֵפִ֥יר עֲצַת־גּוֹיִ֑ם הֵ֝נִ֗יא מַחְשְׁב֥וֹת עַמִּֽים׃
\source{משלי יט}רַבּ֣וֹת מַחֲשָׁב֣וֹת בְּלֶב־אִ֑ישׁ וַעֲצַ֥ת יְ֝יָ֗ הִ֣יא תָקֽוּם׃
\source{תהלים לג}עֲצַ֣ת יְיָ֭ לְעוֹלָ֣ם תַּעֲמֹ֑ד מַחְשְׁב֥וֹת לִ֝בּ֗וֹ לְדֹ֣ר וָדֹֽר׃
כִּ֤י ה֣וּא אָמַ֣ר וַיֶּ֑הִי הֽוּא־צִ֝וָּ֗ה וַֽיַּעֲמֹֽד׃
\source{תהלים קלב}כִּי־בָחַ֣ר יְיָ֣ בְּצִיּ֑וֹן אִ֝וָּ֗הּ לְמוֹשָׁ֥ב לֽוֹ׃
\source{תהלים קלה}כִּֽי־יַעֲקֹ֗ב בָּחַ֣ר ל֣וֹ יָ֑הּ יִ֝שְׂרָאֵ֗ל לִסְגֻלָּתֽוֹ׃
\source{תהלים צד}כִּ֤י ׀ לֹא־יִטֹּ֣שׁ יְיָ֣ עַמּ֑וֹ וְ֝נַחֲלָת֗וֹ לֹ֣א יַעֲזֹֽב׃
\source{תהלים עח}וְה֤וּא רַח֨וּם ׀ יְכַפֵּ֥ר עָוֺן֮ וְֽלֹא־יַֽ֫שְׁחִ֥ית וְ֭הִרְבָּה לְהָשִׁ֣יב אַפּ֑וֹ
וְלֹא־יָ֝עִ֗יר כׇּל־חֲמָתֽוֹ׃
\source{תהלים כ} יְיָ֥ הוֹשִׁ֑יעָה הַ֝מֶּ֗לֶךְ יַעֲנֵ֥נוּ בְיוֹם־קׇרְאֵֽנוּ׃

\ashrei
\enlargethispage{\baselineskip}

\firstword{הַֽלְלוּ־יָ֡הּ}\source{תהלים קמו}
הַֽלְלִ֥י נַ֝פְשִׁ֗י אֶת־יְיָ׃
אֲהַלְלָ֣ה יְיָ֣ בְּחַיָּ֑י אֲזַמְּרָ֖ה לֵאלֹהַ֣י בְּעוֹדִֽי׃
אַל־תִּבְטְח֥וּ בִנְדִיבִ֑ים בְּבֶן־אָדָ֓ם ׀ שֶׁ֤אֵ֖ין ל֥וֹ תְשׁוּעָֽה׃
תֵּצֵ֣א ר֭וּחוֹ יָשֻׁ֣ב לְאַדְמָת֑וֹ בַּיּ֥וֹם הַ֝ה֗וּא אָבְד֥וּ עֶשְׁתֹּֽנֹתָֽיו׃
אַשְׁרֵ֗י שֶׁ֤אֵ֣ל יַעֲקֹ֣ב בְּעֶזְר֑וֹ שִׂ֝בְר֗וֹ עַל־יְיָ֥ אֱלֹהָֽיו׃
עֹשֶׂ֤ה ׀ שָׁ֘מַ֤יִם וָאָ֗רֶץ אֶת־הַיָּ֥ם וְאֶת־כׇּל־אֲשֶׁר־בָּ֑ם הַשֹּׁמֵ֖ר אֱמֶ֣ת לְעוֹלָֽם׃
עֹשֶׂ֤ה מִשְׁפָּ֨ט ׀ לָעֲשׁוּקִ֗ים נֹתֵ֣ן לֶ֭חֶם לָרְעֵבִ֑ים יְ֝יָ֗ מַתִּ֥יר אֲסוּרִֽים׃
יְיָ֤ ׀ פֹּ֘קֵ֤חַ עִוְרִ֗ים יְיָ֭ זֹקֵ֣ף כְּפוּפִ֑ים יְ֝יָ֗ אֹהֵ֥ב צַדִּיקִֽים׃
יְיָ֤ ׀ שֹׁ֘מֵ֤ר אֶת־גֵּרִ֗ים יָת֣וֹם וְאַלְמָנָ֣ה יְעוֹדֵ֑ד וְדֶ֖רֶךְ רְשָׁעִ֣ים יְעַוֵּֽת׃
יִמְלֹ֤ךְ יְיָ֨ ׀ לְעוֹלָ֗ם\\ אֱלֹהַ֣יִךְ צִ֭יּוֹן לְדֹ֥ר וָדֹ֗ר הַֽלְלוּ־יָֽהּ׃



\firstword{הַ֥לְלוּ יָ֨הּ} ׀\source{תהלים קמז}
כִּי־ט֭וֹב זַמְּרָ֣ה אֱלֹהֵ֑ינוּ כִּי־נָ֝עִ֗ים נָאוָ֥ה תְהִלָּֽה׃
בּוֹנֵ֣ה יְרֽוּשָׁלַ֣‍ִם יְיָ֑ נִדְחֵ֖י יִשְׂרָאֵ֣ל יְכַנֵּֽס׃
הָ֭רֹפֵא לִשְׁב֣וּרֵי לֵ֑ב וּ֝מְחַבֵּ֗שׁ לְעַצְּבוֹתָֽם׃
מוֹנֶ֣ה מִ֭סְפָּר לַכּוֹכָבִ֑ים לְ֝כֻלָּ֗ם שֵׁמ֥וֹת יִקְרָֽא׃
גָּד֣וֹל אֲדוֹנֵ֣ינוּ וְרַב־כֹּ֑חַ לִ֝תְבוּנָת֗וֹ אֵ֣ין מִסְפָּֽר׃
מְעוֹדֵ֣ד עֲנָוִ֣ים יְיָ֑ מַשְׁפִּ֖יל רְשָׁעִ֣ים עֲדֵי־אָֽרֶץ׃
עֱנ֣וּ לַֽייָ֣ בְּתוֹדָ֑ה זַמְּר֖וּ לֵאלֹהֵ֣ינוּ בְכִנּֽוֹר׃
הַֽמְכַסֶּ֬ה שָׁמַ֨יִם ׀ בְּעָבִ֗ים הַמֵּכִ֣ין לָאָ֣רֶץ מָטָ֑ר הַמַּצְמִ֖יחַ הָרִ֣ים חָצִֽיר׃
נוֹתֵ֣ן לִבְהֵמָ֣ה לַחְמָ֑הּ לִבְנֵ֥י עֹ֝רֵ֗ב אֲשֶׁ֣ר יִקְרָֽאוּ׃
לֹ֤א בִגְבוּרַ֣ת הַסּ֣וּס יֶחְפָּ֑ץ לֹא־בְשׁוֹקֵ֖י הָאִ֣ישׁ יִרְצֶֽה׃
רוֹצֶ֣ה יְיָ֭ אֶת־יְרֵאָ֑יו אֶת־הַֽמְיַחֲלִ֥ים לְחַסְדּֽוֹ׃
שַׁבְּחִ֣י יְ֭רוּשָׁלַ‍ִם אֶת־יְיָ֑ הַֽלְלִ֖י אֱלֹהַ֣יִךְ צִיּֽוֹן׃
כִּֽי־חִ֭זַּק בְּרִיחֵ֣י שְׁעָרָ֑יִךְ בֵּרַ֖ךְ בָּנַ֣יִךְ בְּקִרְבֵּֽךְ׃
הַשָּׂם־גְּבוּלֵ֥ךְ שָׁל֑וֹם חֵ֥לֶב חִ֝טִּ֗ים יַשְׂבִּיעֵֽךְ׃
הַשֹּׁלֵ֣חַ אִמְרָת֣וֹ אָ֑רֶץ עַד־מְ֝הֵרָ֗ה יָר֥וּץ דְּבָרֽוֹ׃
הַנֹּתֵ֣ן שֶׁ֣לֶג כַּצָּ֑מֶר כְּ֝פ֗וֹר כָּאֵ֥פֶר יְפַזֵּֽר׃
מַשְׁלִ֣יךְ קַֽרְח֣וֹ כְפִתִּ֑ים לִפְנֵ֥י קָ֝רָת֗וֹ מִ֣י יַעֲמֹֽד׃
יִשְׁלַ֣ח דְּבָר֣וֹ וְיַמְסֵ֑ם יַשֵּׁ֥ב ר֝וּח֗וֹ יִזְּלוּ־מָֽיִם׃
מַגִּ֣יד דְּבָרָ֣ו לְיַעֲקֹ֑ב חֻקָּ֥יו וּ֝מִשְׁפָּטָ֗יו לְיִשְׂרָאֵֽל׃
לֹ֘א עָ֤שָׂה כֵ֨ן ׀ לְכׇל־גּ֗וֹי וּמִשְׁפָּטִ֥ים בַּל־יְדָע֗וּם הַֽלְלוּ־יָֽהּ׃


\firstword{הַ֥לְלוּ יָ֨הּ} ׀\source{תהלים קמח}
הַֽלְל֣וּ אֶת־יְיָ֭ מִן־הַשָּׁמַ֑יִם הַֽ֝לְל֗וּהוּ בַּמְּרוֹמִֽים׃
הַֽלְל֥וּהוּ כׇל־מַלְאָכָ֑יו הַ֝לְל֗וּהוּ כׇּל־צְבָאָֽו׃
הַֽ֭לְלוּהוּ שֶׁ֣מֶשׁ וְיָרֵ֑חַ הַֽ֝לְל֗וּהוּ כׇּל־כּ֥וֹכְבֵי אֽוֹר׃
הַֽ֭לְלוּהוּ שְׁמֵ֣י הַשָּׁמָ֑יִם וְ֝הַמַּ֗יִם אֲשֶׁ֤ר ׀ מֵעַ֬ל הַשָּׁמָֽיִם׃
יְֽ֭הַלְלוּ אֶת־שֵׁ֣ם יְיָ֑ כִּ֤י ה֖וּא צִוָּ֣ה וְנִבְרָֽאוּ׃
וַיַּעֲמִידֵ֣ם לָעַ֣ד לְעוֹלָ֑ם חׇק־נָ֝תַ֗ן וְלֹ֣א יַעֲבֽוֹר׃
הַֽלְל֣וּ אֶת־יְיָ֭ מִן־הָאָ֑רֶץ תַּ֝נִּינִ֗ים וְכׇל־תְּהֹמֽוֹת׃
אֵ֣שׁ וּ֭בָרָד שֶׁ֣לֶג וְקִיט֑וֹר ר֥וּחַ סְ֝עָרָ֗ה עֹשָׂ֥ה דְבָרֽוֹ׃
הֶהָרִ֥ים וְכׇל־גְּבָע֑וֹת עֵ֥ץ פְּ֝רִ֗י וְכׇל־אֲרָזִֽים׃
הַחַיָּ֥ה וְכׇל־בְּהֵמָ֑ה רֶ֗֝מֶשׂ וְצִפּ֥וֹר כָּנָֽף׃
מַלְכֵי־אֶ֭רֶץ וְכׇל־לְאֻמִּ֑ים שָׂ֝רִ֗ים וְכׇל־שֹׁ֥פְטֵי אָֽרֶץ׃
בַּחוּרִ֥ים וְגַם־בְּתוּל֑וֹת זְ֝קֵנִ֗ים עִם־נְעָרִֽים׃
יְהַלְל֤וּ ׀ אֶת־שֵׁ֬ם יְיָ֗ כִּֽי־נִשְׂגָּ֣ב שְׁמ֣וֹ לְבַדּ֑וֹ
ה֝וֹד֗וֹ עַל־אֶ֥רֶץ וְשָׁמָֽיִם׃ וַיָּ֤רֶם קֶ֨רֶן ׀ לְעַמּ֡וֹ תְּהִלָּ֤ה לְֽכׇל־חֲסִידָ֗יו
לִבְנֵ֣י יִ֭שְׂרָאֵל עַ֥ם קְרֹב֗וֹ הַֽלְלוּ־יָֽהּ׃\\
\firstword{הַ֥לְלוּ יָ֨הּ} ׀\source{תהלים קמט}
שִׁ֣ירוּ לַֽייָ֭ שִׁ֣יר חָדָ֑שׁ תְּ֝הִלָּת֗וֹ בִּקְהַ֥ל חֲסִידִֽים׃
יִשְׂמַ֣ח יִשְׂרָאֵ֣ל בְּעֹשָׂ֑יו בְּנֵֽי־צִ֝יּ֗וֹן יָגִ֥ילוּ בְמַלְכָּֽם׃
יְהַלְל֣וּ שְׁמ֣וֹ בְמָח֑וֹל בְּתֹ֥ף וְ֝כִנּ֗וֹר יְזַמְּרוּ־לֽוֹ׃
כִּֽי־רוֹצֶ֣ה יְיָ֣ בְּעַמּ֑וֹ יְפָאֵ֥ר עֲ֝נָוִ֗ים בִּישׁוּעָֽה׃
יַעְלְז֣וּ חֲסִידִ֣ים בְּכָב֑וֹד יְ֝רַנְּנ֗וּ עַל־מִשְׁכְּבוֹתָֽם׃
רוֹמְמ֣וֹת אֵ֭ל בִּגְרוֹנָ֑ם וְחֶ֖רֶב פִּיפִיּ֣וֹת בְּיָדָֽם׃
לַעֲשׂ֣וֹת נְ֭קָמָה בַּגּוֹיִ֑ם תּ֝וֹכֵח֗וֹת בַּלְאֻמִּֽים׃
לֶאְסֹ֣ר מַלְכֵיהֶ֣ם בְּזִקִּ֑ים וְ֝נִכְבְּדֵיהֶ֗ם בְּכַבְלֵ֥י בַרְזֶֽל׃
לַעֲשׂ֤וֹת בָּהֶ֨ם ׀ מִשְׁפָּ֬ט כָּת֗וּב הָדָ֣ר ה֭וּא לְכׇל־חֲסִידָ֗יו הַֽלְלוּ־יָֽהּ׃

\firstword{הַ֥לְלוּ יָ֨הּ} ׀\source{תהלים קנ}
הַֽלְלוּ־אֵ֥ל בְּקׇדְשׁ֑וֹ הַֽ֝לְל֗וּהוּ בִּרְקִ֥יעַ עֻזּֽוֹ׃
הַלְל֥וּהוּ בִגְבוּרֹתָ֑יו הַ֝לְל֗וּהוּ כְּרֹ֣ב גֻּדְלֽוֹ׃
הַ֭לְלוּהוּ בְּתֵ֣קַע שׁוֹפָ֑ר הַ֝לְל֗וּהוּ בְּנֵ֣בֶל וְכִנּֽוֹר׃
הַ֭לְלוּהוּ בְּתֹ֣ף וּמָח֑וֹל הַֽ֝לְל֗וּהוּ בְּמִנִּ֥ים וְעֻגָֽב׃
הַלְל֥וּהוּ בְצִלְצְלֵי־שָׁ֑מַע הַֽ֝לְל֗וּהוּ בְּֽצִלְצְלֵ֥י תְרוּעָֽה׃
כֹּ֣ל הַ֭נְּשָׁמָה תְּהַלֵּ֥ל יָ֗הּ הַֽלְלוּ־יָֽהּ׃
\scriptsize{כֹּ֣ל הַ֭נְּשָׁמָה תְּהַלֵּ֥ל יָ֗הּ הַֽלְלוּ־יָֽהּ׃ \\}
\normalsize{}

\negline

\firstword{בָּר֖וּךְ}\source{תהלים פט}
יְיָ֥ לְ֝עוֹלָ֗ם אָ֘מֵ֥ן ׀ וְאָמֵֽן׃ \hfill \break
\source{תהלים קלה}בָּ֘ר֤וּךְ יְיָ֨ ׀ מִצִּיּ֗וֹן שֹׁ֘כֵ֤ן יְֽרוּשָׁלָ֗‍ִם הַֽלְלוּ־יָֽהּ׃ \hfill \break
\source{תהלים עב}בָּר֤וּךְ ׀ יְיָ֣ אֱ֭לֹהִים אֱלֹהֵ֣י יִשְׂרָאֵ֑ל עֹשֵׂ֖ה נִפְלָא֣וֹת לְבַדּֽוֹ׃ וּבָר֤וּךְ ׀ שֵׁ֥ם כְּבוֹד֗וֹ לְע֫וֹלָ֥ם וְיִמָּלֵ֣א כְ֭בוֹדוֹ אֶת־כֹּ֥ל הָאָ֗רֶץ אָ֘מֵ֥ן ׀ וְאָמֵֽן׃





\firstword{וַיְבָ֤רֶךְ}\source{דה״א כט}
דָּוִיד֙ אֶת־יְיָ֔ לְעֵינֵ֖י כׇּל־הַקָּהָ֑ל וַיֹּ֣אמֶר דָּוִ֗יד בָּר֨וּךְ אַתָּ֤ה יְיָ֙ אֱלֹהֵי֙ יִשְׂרָאֵ֣ל אָבִ֔ינוּ מֵעוֹלָ֖ם וְעַד־עוֹלָֽם׃
לְךָ֣ יְ֠יָ֠ הַגְּדֻלָּ֨ה וְהַגְּבוּרָ֤ה וְהַתִּפְאֶ֙רֶת֙ וְהַנֵּ֣צַח וְהַה֔וֹד כִּי־כֹ֖ל בַּשָּׁמַ֣יִם וּבָאָ֑רֶץ לְךָ֤ יְיָ֙ הַמַּמְלָכָ֔ה וְהַמִּתְנַשֵּׂ֖א לְכֹ֥ל ׀ לְרֹֽאשׁ׃
וְהָעֹ֤שֶׁר וְהַכָּבוֹד֙ מִלְּפָנֶ֔יךָ וְאַתָּה֙ מוֹשֵׁ֣ל בַּכֹּ֔ל וּבְיָדְךָ֖ כֹּ֣חַ וּגְבוּרָ֑ה וּבְיָ֣דְךָ֔ לְגַדֵּ֥ל וּלְחַזֵּ֖ק לַכֹּֽל׃
וְעַתָּ֣ה אֱלֹהֵ֔ינוּ מוֹדִ֥ים אֲנַ֖חְנוּ לָ֑ךְ וּֽמְהַלְלִ֖ים לְשֵׁ֥ם תִּפְאַרְתֶּֽךָ׃


אַתָּה־ה֣וּא\source{נחמיה ט}
יְיָ לְבַדֶּ֒ךָ֒ אַתָּ֣ עָשִׂ֡יתָ אֶֽת־הַשָּׁמַ֩יִם֩ שְׁמֵ֨י הַשָּׁמַ֜יִם וְכׇל־צְבָאָ֗ם הָאָ֜רֶץ וְכׇל־אֲשֶׁ֤ר עָלֶ֙יהָ֙ הַיַּמִּים֙ וְכׇל־אֲשֶׁ֣ר בָּהֶ֔ם וְאַתָּ֖ה מְחַיֶּ֣ה אֶת־כֻּלָּ֑ם וּצְבָ֥א הַשָּׁמַ֖יִם לְךָ֥ מִשְׁתַּחֲוִֽים׃
אַתָּה־הוּא֙ יְיָ֣ הָאֱלֹהִ֔ים אֲשֶׁ֤ר בָּחַ֙רְתָּ֙ בְּאַבְרָ֔ם וְהוֹצֵאת֖וֹ מֵא֣וּר כַּשְׂדִּ֑ים וְשַׂ֥מְתָּ שְּׁמ֖וֹ אַבְרָהָֽם׃ וּמָצָ֣אתָ אֶת־לְבָבוֹ֮ נֶאֱמָ֣ן לְפָנֶ֒יךָ֒

וְכָר֨וֹת עִמּ֜וֹ הַבְּרִ֗ית לָתֵ֡ת אֶת־אֶ֩רֶץ֩ הַכְּנַעֲנִ֨י הַחִתִּ֜י הָאֱמֹרִ֧י וְהַפְּרִזִּ֛י וְהַיְבוּסִ֥י וְהַגִּרְגָּשִׁ֖י לָתֵ֣ת לְזַרְע֑וֹ וַתָּ֙קֶם֙ אֶת־דְּבָרֶ֔יךָ כִּ֥י צַדִּ֖יק אָֽתָּה׃ וַתֵּ֛רֶא אֶת־עֳנִ֥י אֲבֹתֵ֖ינוּ בְּמִצְרָ֑יִם וְאֶת־זַעֲקָתָ֥ם שָׁמַ֖עְתָּ עַל־יַם־סֽוּף׃ וַ֠תִּתֵּ֠ן אֹתֹ֨ת וּמֹֽפְתִ֜ים בְּפַרְעֹ֤ה וּבְכׇל־עֲבָדָיו֙ וּבְכׇל־עַ֣ם אַרְצ֔וֹ כִּ֣י יָדַ֔עְתָּ כִּ֥י הֵזִ֖ידוּ עֲלֵיהֶ֑ם וַתַּֽעַשׂ־לְךָ֥ שֵׁ֖ם כְּהַיּ֥וֹם הַזֶּֽה׃
וְהַיָּם֙ בָּקַ֣עְתָּ לִפְנֵיהֶ֔ם וַיַּֽעַבְר֥וּ בְתוֹךְ־הַיָּ֖ם בַּיַּבָּשָׁ֑ה וְֽאֶת־רֹ֨דְפֵיהֶ֜ם הִשְׁלַ֧כְתָּ בִמְצוֹלֹ֛ת כְּמוֹ־אֶ֖בֶן בְּמַ֥יִם עַזִּֽים׃

וַיּ֨וֹשַׁע\source{שמות יד}
יְיָ֜ בַּיּ֥וֹם הַה֛וּא אֶת־יִשְׂרָאֵ֖ל מִיַּ֣ד מִצְרָ֑יִם וַיַּ֤רְא יִשְׂרָאֵל֙ אֶת־מִצְרַ֔יִם מֵ֖ת עַל־שְׂפַ֥ת הַיָּֽם׃
וַיַּ֨רְא יִשְׂרָאֵ֜ל אֶת־הַיָּ֣ד הַגְּדֹלָ֗ה אֲשֶׁ֨ר עָשָׂ֤ה יְיָ֙ בְּמִצְרַ֔יִם וַיִּֽירְא֥וּ הָעָ֖ם אֶת־יְיָ֑ וַיַּֽאֲמִ֙ינוּ֙ בַּייָ֔ וּבְמֹשֶׁ֖ה עַבְדּֽוֹ׃

אָ֣ז\source{שמות טו} \hfill
יָשִֽׁיר־מֹשֶׁה֩ \hfill וּבְנֵ֨י \hfill יִשְׂרָאֵ֜ל \hfill אֶת־הַשִּׁירָ֤ה \hfill הַזֹּאת֙ \hfill לַֽייָ֔ \hfill וַיֹּאמְר֖וּ \\
לֵאמֹ֑ר \hfill אָשִׁ֤ירָה לַֽייָ֙ כִּֽי־גָאֹ֣ה גָּאָ֔ה \hfill ס֥וּס \\
וְרֹכְב֖וֹ רָמָ֥ה בַיָּֽם׃ \hfill עׇזִּ֤י וְזִמְרָת֙ יָ֔הּ וַֽיְהִי־לִ֖י \\
לִֽישׁוּעָ֑ה \hfill זֶ֤ה אֵלִי֙ וְאַנְוֵ֔הוּ \hfill אֱלֹהֵ֥י \\
אָבִ֖י וַאֲרֹמְמֶֽנְהוּ׃ \hfill יְיָ֖ אִ֣ישׁ מִלְחָמָ֑ה יְיָ֖ \\
שְׁמֽוֹ׃ \hfill מַרְכְּבֹ֥ת פַּרְעֹ֛ה וְחֵיל֖וֹ יָרָ֣ה בַיָּ֑ם \hfill וּמִבְחַ֥ר\\
שָֽׁלִשָׁ֖יו טֻבְּע֥וּ בְיַם־סֽוּף׃ \hfill תְּהֹמֹ֖ת יְכַסְיֻ֑מוּ יָרְד֥וּ בִמְצוֹלֹ֖ת כְּמוֹ־\\
אָֽבֶן׃ \hfill יְמִֽינְךָ֣ יְיָ֔ נֶאְדָּרִ֖י בַּכֹּ֑חַ \hfill יְמִֽינְךָ֥ \\
יְיָ֖ תִּרְעַ֥ץ אוֹיֵֽב׃ \hfill וּבְרֹ֥ב גְּאוֹנְךָ֖ תַּהֲרֹ֣ס \\
קָמֶ֑יךָ \hfill תְּשַׁלַּח֙ חֲרֹ֣נְךָ֔ יֹאכְלֵ֖מוֹ כַּקַּֽשׁ׃ \hfill וּבְר֤וּחַ \\
אַפֶּ֙יךָ֙ נֶ֣עֶרְמוּ מַ֔יִם \hfill נִצְּב֥וּ כְמוֹ־נֵ֖ד \\
נֹזְלִ֑ים \hfill קָֽפְא֥וּ תְהֹמֹ֖ת בְּלֶב־יָֽם׃ \hfill אָמַ֥ר \\
אוֹיֵ֛ב אֶרְדֹּ֥ף אַשִּׂ֖יג \hfill אֲחַלֵּ֣ק שָׁלָ֑ל תִּמְלָאֵ֣מוֹ \\
נַפְשִׁ֔י \hfill אָרִ֣יק חַרְבִּ֔י תּוֹרִישֵׁ֖מוֹ יָדִֽי׃ \hfill נָשַׁ֥פְתָּ \\
בְרוּחֲךָ֖ כִּסָּ֣מוֹ יָ֑ם \hfill צָֽלְלוּ֙ כַּֽעוֹפֶ֔רֶת בְּמַ֖יִם \\
אַדִּירִֽים׃ \hfill מִֽי־כָמֹ֤כָה בָּֽאֵלִם֙ יְיָ֔ \hfill מִ֥י \\
כָּמֹ֖כָה נֶאְדָּ֣ר בַּקֹּ֑דֶשׁ \hfill נוֹרָ֥א תְהִלֹּ֖ת עֹ֥שֵׂה \\
פֶֽלֶא׃ \hfill נָטִ֙יתָ֙ יְמִ֣ינְךָ֔ תִּבְלָעֵ֖מוֹ אָֽרֶץ׃ \hfill נָחִ֥יתָ \\
בְחַסְדְּךָ֖ עַם־ז֣וּ גָּאָ֑לְתָּ \hfill נֵהַ֥לְתָּ בְעׇזְּךָ֖ אֶל־נְוֵ֥ה \\
קׇדְשֶֽׁךָ׃ \hfill שָֽׁמְע֥וּ עַמִּ֖ים יִרְגָּז֑וּן \hfill חִ֣יל \\
אָחַ֔ז יֹשְׁבֵ֖י פְּלָֽשֶׁת׃ \hfill אָ֤ז נִבְהֲלוּ֙ אַלּוּפֵ֣י \\
אֱד֔וֹם \hfill אֵילֵ֣י מוֹאָ֔ב יֹֽאחֲזֵ֖מוֹ רָ֑עַד \hfill נָמֹ֕גוּ \\
כֹּ֖ל יֹשְׁבֵ֥י כְנָֽעַן׃ \hfill תִּפֹּ֨ל עֲלֵיהֶ֤ם אֵימָ֙תָה֙ \\
וָפַ֔חַד \hfill בִּגְדֹ֥ל זְרוֹעֲךָ֖ יִדְּמ֣וּ כָּאָ֑בֶן \hfill עַד־\\
יַעֲבֹ֤ר עַמְּךָ֙ יְיָ֔ \hfill עַֽד־יַעֲבֹ֖ר עַם־ז֥וּ \\
קָנִֽיתָ׃ \hfill תְּבִאֵ֗מוֹ וְתִטָּעֵ֙מוֹ֙ בְּהַ֣ר נַחֲלָֽתְךָ֔ \hfill מָכ֧וֹן \\
לְשִׁבְתְּךָ֛ פָּעַ֖לְתָּ יְיָ֑ \hfill מִקְּדָ֕שׁ אֲדֹנָ֖י כּוֹנְנ֥וּ \\
יָדֶֽיךָ׃ \hfill יְיָ֥ ׀ יִמְלֹ֖ךְ לְעֹלָ֥ם וָעֶֽד׃ \begin{footnotesize}יְיָ֥ ׀ יִמְלֹ֖ךְ לְעֹלָ֥ם וָעֶֽד׃\end{footnotesize}
\hfill \begin{small} כִּ֣י \\
	בָא֩ ס֨וּס פַּרְעֹ֜ה בְּרִכְבּ֤וֹ וּבְפָרָשָׁיו֙ בַּיָּ֔ם \hfill וַיָּ֧שֶׁב יְיָ֛ עֲלֵהֶ֖ם \\
	אֶת־מֵ֣י הַיָּ֑ם \hfill וּבְנֵ֧י יִשְׂרָאֵ֛ל הָלְכ֥וּ בַיַּבָּשָׁ֖ה בְּת֥וֹךְ\hfill הַיָּֽם׃\\
	וַתִּקַּח֩ מִרְיָ֨ם הַנְּבִיאָ֜ה אֲח֧וֹת אַהֲרֹ֛ן אֶת־הַתֹּ֖ף בְּיָדָ֑הּ וַתֵּצֶ֤אןָ כׇֽל־הַנָּשִׁים֙ אַחֲרֶ֔יהָ בְּתֻפִּ֖ים וּבִמְחֹלֹֽת׃ וַתַּ֥עַן לָהֶ֖ם מִרְיָ֑ם שִׁ֤ירוּ לַֽייָ֙ כִּֽי־גָאֹ֣ה גָּאָ֔ה ס֥וּס וְרֹכְב֖וֹ רָמָ֥ה בַיָּֽם׃\hfill\break 
\end{small}


כִּ֣י \source{תהלים כב}לַ֭ייָ֭ הַמְּלוּכָ֑ה וּ֝מֹשֵׁ֗ל בַּגּוֹיִֽם׃
וְעָל֤וּ \source{עובדיה א}מֽוֹשִׁעִים֙ בְּהַ֣ר צִיּ֔וֹן לִשְׁפֹּ֖ט אֶת־הַ֣ר עֵשָׂ֑ו וְהָיְתָ֥ה לַֽייָ֖ הַמְּלוּכָֽה׃
וְהָיָ֧ה \source{זכריה יד}יְיָ֛ לְמֶ֖לֶךְ עַל־כׇּל־הָאָ֑רֶץ בַּיּ֣וֹם הַה֗וּא יִהְיֶ֧ה יְיָ֛ אֶחָ֖ד וּשְׁמ֥וֹ אֶחָֽד׃

\ifboolexpr{togl {includeweekday} and not togl {includeshabbat} and not togl {includefestival}} {\yishtabach
		
		\mimaamakim
		
		\halfkaddish
		
		\enlargethispage{\baselineskip}
		
		\vspace{1.25\baselineskip}}{\ifboolexpr{togl {includeweekday} and (togl {includeshabbat} or togl {includefestival}}{
		
\instruction{בשבת וביו״ט ממשיכים בעמ׳ \pageref{nishmas}}\\

\yishtabach

\mimaamakim

\halfkaddish

\enlargethispage{\baselineskip}

\vspace{1.25\baselineskip}}}


\ifboolexpr{togl {includeshabbat} and togl {includefestival}}{\chapter[שחרית לשבת ויו״ט]{\adforn{47} שחרית לשבת ויו״ט \adforn{19}}}{
\ifboolexpr{togl {includeshabbat}}{\chapter[שחרית לשבת]{\adforn{47} שחרית לשבת \adforn{19}}}{}
\ifboolexpr{togl {includefestival}}{\chapter[שחרית ליו״ט]{\adforn{47} שחרית ליו״ט \adforn{19}}}{}}

\label{nishmas}
\nishmat

%\ifboolexpr{togl {includefestival} and togl {includeshabbat}}{\englishinst{On festivals, the leader for Sha\d{h}arit begins here.}}{}
%\ifboolexpr{togl {includefestival} and not togl {includeshabbat}}{}{}

\hael

\ifboolexpr{togl {includefestival} and togl {includeshabbat}}{\instruction{בשבת החזן מתחיל כאן}}{}

\shochenad

\yishtabach
\ifboolexpr{togl {includeshabbat}}{\mimaamakim}{}
\halfkaddish

\section[קריאת שמע וברכותיה]{\adforn{53} קריאת שמע וברכותיה \adforn{25}}

\barachu

\firstword{
	בָּרוּךְ אַתָּה יְיָ אֱלֹהֵֽינוּ מֶֽלֶךְ הָעוֹלָם \middot יוֹצֵר אוֹר וּבוֹרֵא חֹֽשֶׁךְ עֹשֶׂה שָׁלוֹם וּבוֹרֵא אֶת־הַכֹּל׃}

%\instruction{כשאומרים יוצרות:}
%אוֹר עוֹלָם בְּאוֹצַר חַיִּים אוֹרוֹת מֵאוֹפֶל אָמַר וַיֶהִי׃
\ifboolexpr{togl {includefestival} and togl {includeshabbat}}{
\footnote{\instruction{ביום טוב שחל בחול:}\\
\hameir 
\instruction{תִּתְבָּרַךְ וכו׳ עמ׳ \pageref{tisbarach}
}}}{}
%
\ifboolexpr{togl {includefestival} and not togl {includeshabbat}}{
\insturction{ביו״ט שחל בחול}\\
\hameir
\instruction{תִּתְבָּרַךְ וכו׳ עמ׳ \pageref{tisbarach}}

\instruction{ביו״ט שחל בשבת}}{}
\firstword{הַכֹּל יוֹדֽוּךָ}
וְהַכֹּל יְשַׁבְּחֽוּךָ \middot וְהַכֹּל יֹאמְרוּ אֵין קָדוֹשׁ כַּיָי׃ הַכֹּל יְרוֹמְמֽוּךָ סֶּֽלָה יוֹצֵר הַכֹּל \middot הָאֵל הַפּוֹתֵֽחַ בְּכׇל־יוֹם דַּלְתוֹת שַׁעֲרֵי מִזְרָח \middot וּבוֹקֵֽעַ חַלּוֹנֵי רָקִֽיעַ מוֹצִיא חַמָּה מִמְּקוֹמָהּ וּלְבָנָה מִמְּכוֹן שִׁבְתָּהּ׃ וּמֵאִיר לָעוֹלָם כֻּלּוֹ וּלְיוֹשְׁבָיו שֶׁבָּרָא בְּמִדַּת רַחֲמִים׃ 
הַמֵּאִיר לָאָֽרֶץ וְלַדָּרִים עָלֶֽיהָ בְּרַחֲמִים \middot וּבְטוּבוֹ מְחַדֵּשׁ בְּכׇל־יוֹם תָּמִיד מַעֲשֵׂה בְרֵאשִׁית׃
הַמֶּֽלֶךְ הַמְּרוֹמָם לְבַדּוֹ מֵאָז \middot הַמְשֻׁבָּח וְהַמְפֹאָר וְהַמִּתְנַשֵּׂא מִימוֹת עוֹלָם׃
אֱלֹהֵי עוֹלָם בְּרַחֲמֶֽיךָ הָרַבִּים רַחֵם עָלֵֽינוּ \middot אֲדוֹן עֻזֵּֽנוּ צוּר מִשְׂגַּבֵּֽנוּ מָגֵן יִשְׁעֵֽנוּ מִשְׂגָּב בַּעֲדֵֽנוּ׃
אֵין כְּעֶרְכֶּֽךָ וְאֵין זוּלָתֶֽךָ \middot אֶפֶס בִּלְתֶּֽךָ וּמִי דּֽוֹמֶה לָּךְ׃
אֵין כְּעֶרְכְּךָ יְיָ אֱלֹהֵֽינוּ בָּעוֹלָם הַזֶּה \middot וְאֵין זוּלָתְךָ מַלְכֵּֽנוּ לְחַיֵּי הָעוֹלָם הַבָּא׃
אֶֽפֶס בִּלְתְּךָ גּוֹאֲלֵֽנוּ לִימוֹת הַמָּשִֽׁיחַ \middot וְאֵין דּֽוֹמֶה לְּךָ מוֹשִׁיעֵֽנוּ לִתְחִיַּת הַמֵּתִים׃

\firstword{אֵׄל אָדוֹן}
עַל כׇּל־הַמַּעֲשִׂים \hfill בָּׄרוּךְ וּמְבֹרָךְ בְּפִי כׇּל־נְשָׁמָה׃ \\
גׇּׄדְלוֹ וְטוּבוֹ מָלֵא עוֹלָם \hfill דַּֽׄעַת וּתְבוּנָה סוֹבְבִים אוֹתוֹ׃

הַׄמִּתְגָּאֶה עַל חַיּוֹת הַקֹּֽדֶשׁ \hfill וְׄנֶהְדָּר בְּכָבוֹד עַל הַמֶּרְכָּבָה׃\\
זְׄכוּת וּמִישׁוֹר לִפְנֵי כִסְאוֹ \hfill חֶֽׄסֶד וְרַחֲמִים לִפְנֵי כְבוֹדוֹ׃

טׄוֹבִים מְאוֹרוֹת שֶׁבָּרָא אֱלֹהֵֽינוּ \hfill יְׄצָרָם בְּדַֽעַת בְּבִינָה וּבְהַשְׂכֵּל׃\\
כֹּֽׄחַ וּגְבוּרָה נָתַן בָּהֶם \hfill לִׄהְיוֹת מוֹשְׁלִים בְּקֶֽרֶב תֵּבֵל׃

מְׄלֵאִים זִיו וּמְפִיקִים נֹֽגַהּ \hfill נָׄאֶה זִיוָם בְּכׇל־הָעוֹלָם׃ \\
שְׂׄמֵחִים בְּצֵאתָם וְשָׂשִׂים בְּבוֹאָם \hfill עׄוֹשִׂים בְּאֵימָה רְצוֹן קוׂנָם׃

פְּׄאֵר וְכָבוֹד נוֹתְנִים לִשְׁמוֹ \hfill צׇׄהֳלָה וְרִנָּה לְזֵֽכֶר מַלְכוּתוֹ׃ \\
קָׄרָא לַשֶּֽׁמֶשׁ וַיִּזְרַח אוֹר \hfill רָׄאָה וְהִתְקִין צוּרַת הַלְּבָנָה׃

שֶֽׁׄבַח נוֹתְנִים לוֹ\hfill כׇּל־צְבָא מָרוֹם \\ תִּׄפְאֶֽרֶת וּגְדֻלָּה\hfill שְׂרָפִים וְאוֹפַנִּים וְחַיּוֹת הַקֹּֽדֶשׁ׃

\firstword{לָאֵל}
אֲשֶׁר שָׁבַת מִכׇּל־הַמַּעֲשִׂים בַּיּוֹם הַשְּׁבִיעִי נִתְעַלָּה וְיָשַׁב עַל כִּסֵּא כְבוֹדוֹ \middot תִּפְאֶֽרֶת עָטָה לְיוֹם הַמְּנוּחָה עֹֽנֶג קָרָא לְיוֹם הַשַּׁבָּת׃
זֶה שֶֽׁבַח שֶׁלַּיּוֹם הַשְּׁבִיעִי שֶׁבּוֹ שָֽׁבַת אֵל מִכׇּל־מְלַאכְתּוֹ׃ וְיוֹם הַשְּׁבִיעִי מְשַׁבֵּֽחַ וְאוֹמֵר׃
\source{תהלים צב}%
מִזְמ֥וֹר שִׁ֗יר לְי֣וֹם הַשַּׁבָּֽת׃ ט֗וֹב לְהֹד֥וֹת לַייָ֑
לְפִיכָךְ יְפָאֲרוּ וִיבָרְכוּ לָאֵל כׇּל־יְצוּרָיו \middot שֶֽׁבַח יְקָר וּגְדֻלָּה יִתְּנוּ לְאֵל מֶֽלֶךְ יוֹצֵר כֹּל \middot הַמַּנְחִיל מְנוּחָה לְעַמּוֹ יִשְׂרָאֵל בִּקְדֻשָּׁתוֹ בְּיוֹם שַׁבַּת קֹֽדֶשׁ׃
שִׁמְךָ יְיָ אֱלֹהֵֽינוּ יִתְקַדַּשׁ \middot וְזִכְרְךָ מַלְכֵּֽנוּ יִתְפָּאַר בַּשָּׁמַֽיִם מִמַּֽעַל וְעַל הָאָֽרֶץ מִתָּֽחַת׃ תִּתְבָּרַךְ מוֹשִׁיעֵֽנוּ עַל שֶֽׁבַח מַעֲשֵׂה יָדֶֽיךָ וְעַל מְאוֹרֵי אוֹר שֶׁעָשִֽׂיתָ יְפָאֲרֽוּךָ סֶּֽלָה׃


\label{tisbarach}
\yotzerhameoros

\ahavaraba

\ifboolexpr{not togl {includeweekday}}{\label{morningshema}}{}
\shema

\veahavta

\vehaya

\vayomer{}

\emesveyatziv

\ezrasavoseinu

\gaalyisroel\\

%\nextpage

\section[תפילת העמידה]{\adforn{53} תפילת העמידה \adforn{25}}

\amidaopening{\shabbosshuva}{\englishinst{During the repetition of the Amidah, Kedusha is said here}}
\nextpage
\begin{Center}\ssubsection{\adforn{48} קדושה \adforn{22}}\end{Center}

\begin{footnotesize}
\begin{longtable}{ l p{.8\textwidth} }

\shatz &
נְקַדֵּשׁ אֶת־שִׁמְךָ בָּעוֹלָם כְּשֵׁם שֶׁמַּקְדִּישִׁים אוֹתוֹ בִּשְׁמֵי מָרוֹם כַּכָּתוּב עַל יַד נְבִיאֶךָ קָרָ֨א זֶ֤ה אֶל־זֶה֙ וְאָמַ֔ר \\

\vshatzkahal &
\kadoshkadoshkadosh\\

\shatz &
אָז בְּקוֹל־רַֽעַשׁ גָּדוֹל אַדִּיר וְחָזָק מַשְׁמִיעִים קוֹל מִתְנַשְּׂאִים לְעֻמַּת שְׂרָפִים לְעֻמָּתָם בָּרוּךְ יֹאמֵֽרוּ׃ \\

\vshatzkahal &
\textbf{בָּר֥וּךְ כְּבוֹד־יְיָ֖ מִמְּקוֹמֽוֹ׃} \\

\shatz &
מִמְּקוֹמְךָ מַלְכֵּֽנוּ תוֹפִֽיעַ וְתִמְלֹךְ עָלֵֽינוּ כִּי מְחַכִּים אֲנַֽחְנוּ לָךְ׃ מָתַי תִּמְלֹךְ בְּצִיּוֹן בְּקָרוֹב בְּיָמֵֽינוּ לְעֹלָם וָעֶד תִּשְׁכּוֹן׃ תִּתְגַּדַּל וְתִתְקַדַּשׁ בְּתוֹךְ יְרוּשָׁלַֽיִם עִירְךָ לְדוֹר וָדוֹר וּלְנֵֽצַח נְצָחִים׃ וְעֵינֵֽינוּ תִרְאֶֽינָה מַלְכוּתְךָ כַּדָּבָר הָאָמוּר בְּשִׁירֵי עֻזֶּךָ עַל יְדֵי דָּוִד מְשִֽׁיחַ צִדְקֶֽךָ׃ \\

\vshatzkahal &
\textbf{יִמְלֹ֤ךְ יְיָ֨ לְֽעוֹלָ֗ם אֱלֹהַ֣יִךְ צִ֭יּוֹן לְדֹ֥ר וָ֝דֹ֗ר הַֽלְלוּיָֽהּ׃} \\

\shatz &
לְדוֹר וָדוֹר נַגִּיד גׇּדְלֶךָ וּלְנֵצַח נְצָחִים קְדֻשָּׁתְךָ נַקְדִּישׁ וְשִׁבְחֲךָ אֱלֹהֵֽינוּ מִפִּינוּ לֹא יָמוּשׁ לְעוֹלָם וָעֶד כִּי אֵל מֶלֶךְ גָּדוֹל וְקָדוֹשׁ אַֽתָּה׃ בָּרוּךְ אַתָּה יְיָ הָאֵל
(\instruction{בשבת שובה:} הַמֶּֽלֶךְ)
הַקָּדוֹשׁ׃
\end{longtable}
\end{footnotesize}
\vspace{-24pt}
\sepline

\ifboolexpr{togl {includefestival} and togl {includeshabbat}}{%\instruction{ביו״ט ממשיכים בעמ׳ \pageref{ytshacharit}}
	\englishinst{On Festivals, including when they fall on Shabbat, continue on page \pageref{ytshacharit}.}
}{}
\ifboolexpr{togl {includeshabbat}}{
\firstword{יִשְׂמַח}
מֹשֶׁה בְּמַתְּנַת חֶלְקוֹ כִּי עֶבֶד נֶאֶמָן קָרָאתָ לּוֹ׃ כְּלִיל תִּפְאֶרֶת בְּרֹאשׁוֹ נָתַתָּ בְּעָמְדוֹ לְפָנֶיךָ עַל הַר סִינַי׃ וּשְׁנֵי לֻחֹת אֲבָנִים הוֹרִיד בְּיָדוֹ וְכָתוּב בָּהֶם שְׁמִירַת שַׁבָּת וְכֵן כָּתוּב בְּתוֹרָתֶךָ׃

\veshameru

\firstword{וְלֹא נְתַתּוֹ}
יְיָ אֱלֹהֵינוּ לְגוֹיֵי הָאֲרָצוֹת וְלֹא הִנְחַלְתּוֹ מַלְכֵּנוּ לְעוֹבְדֵי פְסִילִים וְגַם בִּמְנוּחָתוֹ לֹא יִשְׁכְּנוּ עֲרֵלִים׃ כִּי לְיִשְׂרָאֵל עַמְּךָ נְתַתּוֹ בְּאַהֲבָה לְזֶרַע יַעֲקֹב אֲשֶׁר בָּם בָּחָֽרְתָּ׃ עַם מְקַדְּשֵׁי שְׁבִיעִי כֻּלָּם יִשְׂבְּעוּ וְיִתְעַנְּגוּ מִטּוּבֶךָ׃ וּבַשְּׁבִיעִי רָצִיתָ בּוֹ וְקִדַּשְׁתּוֹ חֶמְדַּת יָמִים אוֹתוֹ קָרָאתָ זֵֽכֶר לְמַעֲשֵׂה בְרֵאשִׁית׃

%\shabboskiddushhayom{\footnote{\instruction{נ״א:} וְיָנוּחוּ בָוֹ}} \instruction{רצה וכו׳}
\shabboskiddushhayom{}}{}

\ifboolexpr{togl {includefestival} and togl {includeshabbat}}{\instruction{רצה וכו׳}

\sepline

\label{ytshacharit}

\ytkiddushhayom
\vspace{16pt}
\sepline

}{}

\ifboolexpr{togl {includefestival} and not togl {includeshabbat}}{\ytkiddushhayom{}}{}

\retzeh

\yaalehveyavo

\zion

\modim

\ifboolexpr{togl {includeshabbat}}{

\shabboschanukah

\shabboshodos

}{}



%\bircaskohanim{ בחזרת הש״ץ בארץ ישראל אם יש כוהנים, הם מברכים את הקהל׃}{בחזרת הש״ץ בחו״ל או בא״י עם אין כהנים׃ \vspace{.3\baselineskip}}

\shabbossimshalom

\tachanunim


\ifboolexpr{togl {includefestival}}{\instruction{בסוכת (אבל לא בשבת) נוטלים הלולב:}\\}{}
\ifboolexpr{togl {includeshabbat}}{\instruction{בראש חודש, ביום טוב, בחול המועד, ובחנוכה אומרים הלל}\\
\instruction{בשאר שבתות השנה ממשיכים עם קדיש תתקבל עמ׳ \pageref{shacharitShabbatYTtitkabel}}}{}

\ifboolexpr{togl {includefestival}}{
\section[נטילת הלולב]{\adforn{53} נטילת הלולב \adforn{25}}
\label{lulav}
\instruction{בסוכת לפני הלל:}\\
בָּרוּךְ אַתָּה יְיָ אֱלֹהֵינוּ מֶלֶךְ הָעוֹלָם אֲשֶׁר קִדְּשָׁנוּ בְּמִצְוֹתָיו וְצִוָּנוּ עַל נְטִילַת לוּלָב׃



\instruction{ביום הראשון של סוכת מברך גם שהחיינו:}\\
בָּרוּךְ אַתָּה יְיָ אֱלֹהֵינוּ מֶלֶךְ הָעוֹלָם שֶׁהֶחֱיָנוּ וְקִיְּמָנוּ וְהִגִּיעָנוּ לַזְמַן הַזֶּה׃
\\}{}
%\let\clearpage\relax{
\ifboolexpr{togl {includefestival} or togl {includeChM}}{
\section[נטילת הלולב]{\adforn{53} נטילת הלולב \adforn{25}}
\label{lulav}

בָּרוּךְ אַתָּה יְיָ אֱלֹהֵינוּ מֶלֶךְ הָעוֹלָם אֲשֶׁר קִדְּשָׁנוּ בְּמִצְוֹתָיו וְצִוָּנוּ עַל נְטִילַת לוּלָב׃

\englishinst{When first taking Lulav, add the following blessing:}
בָּרוּךְ אַתָּה יְיָ אֱלֹהֵינוּ מֶלֶךְ הָעוֹלָם שֶׁהֶחֱיָנוּ וְקִיְּמָנוּ וְהִגִּיעָנוּ לַזְמַן הַזֶּה׃
}{}

\newcommand{\diluginst}{\englishinst{}}
\ifboolexpr{(togl {includeweekday} or togl {includeshabbat}) and not togl {includefestival} and not togl {includeChM}}{
	\renewcommand{\diluginst}{\englishinst{This section is skipped on Rosh \d{H}odesh (except on \d{H}anukka).}}
	\chapter[הלל‎]{\adforn{53} הלל‎ \adforn{25}}
}{}
\ifboolexpr{(togl {includefestival} or togl {includeChM}) and (togl {includeweekday} or togl {includeshabbat})}{
	\renewcommand{\diluginst}{\englishinst{This section is skipped on Rosh \d{H}odesh (except on \d{H}anukka) and the intermediate and final days of Passover.}}
	
\chapter[הלל‎]{\adforn{53} הלל‎ \adforn{25}}
}{}
\ifboolexpr{(togl {includefestival} or togl {includeChM}) and not togl {includeweekday} and not togl {includeshabbat}}{
	\renewcommand{\diluginst}{\englishinst{This section is skipped on the intermediate and final days of Passover.}}
	
	\section[הלל‎]{\adforn{53} הלל‎ \adforn{25}}
}{}

\label{hallel}

%\instruction{החזן אומר הברכה בקול רם, הקהל אומר אמן ואחר כך חוזרים ומברכים:}\\
\firstword{בָּרוּךְ}
אַתָּה יְיָ אֱלֹהֵֽינוּ מֶֽלֶךְ הָעוֹלָם אֲשֶׁר קִדְּשָֽׁנוּ בְּמִצְוֹתָיו וְצִוָּֽנוּ לִקְרֹא אֶת־הַהַלֵּל׃

\firstword{הַ֥לְלוּ יָ֨הּ}\source{תהלים קיג}
׀ הַ֭לְלוּ עַבְדֵ֣י יְיָ֑ הַֽ֝לְל֗וּ אֶת־שֵׁ֥ם יְיָ׃
יְהִ֤י שֵׁ֣ם יְיָ֣ מְבֹרָ֑ךְ מֵ֝עַתָּ֗ה וְעַד־עוֹלָֽם׃
מִמִּזְרַח־שֶׁ֥מֶשׁ עַד־מְבוֹא֑וֹ מְ֝הֻלָּ֗ל שֵׁ֣ם יְיָ׃
רָ֖ם עַל־כׇּל־גּוֹיִ֥ם ׀ יְיָ֑ עַ֖ל הַשָּׁמַ֣יִם כְּבוֹדֽוֹ׃
מִ֭י כַּייָ֣ אֱלֹהֵ֑ינוּ הַֽמַּגְבִּיהִ֥י לָשָֽׁבֶת׃
הַֽמַּשְׁפִּילִ֥י לִרְא֑וֹת בַּשָּׁמַ֥יִם וּבָאָֽרֶץ׃
מְקִ֥ימִ֣י מֵעָפָ֣ר דָּ֑ל מֵ֝אַשְׁפֹּ֗ת יָרִ֥ים אֶבְיֽוֹן׃
לְהוֹשִׁיבִ֥י עִם־נְדִיבִ֑ים עִ֗֝ם נְדִיבֵ֥י עַמּֽוֹ׃
מֽוֹשִׁיבִ֨י ׀ עֲקֶ֬רֶת הַבַּ֗יִת אֵֽם־הַבָּנִ֥ים שְׂמֵחָ֗ה הַֽלְלוּ־יָֽהּ׃

\firstword{בְּצֵ֣את יִ֭שְׂרָאֵל}\source{תהלים קיד}
מִמִּצְרָ֑יִם בֵּ֥ית יַ֝עֲקֹ֗ב מֵעַ֥ם לֹעֵֽז׃
הָיְתָ֣ה יְהוּדָ֣ה לְקׇדְשׁ֑וֹ יִ֝שְׂרָאֵ֗ל מַמְשְׁלוֹתָֽיו׃
הַיָּ֣ם רָ֭אָה וַיָּנֹ֑ס הַ֝יַּרְדֵּ֗ן יִסֹּ֥ב לְאָחֽוֹר׃
הֶ֭הָרִים רָקְד֣וּ כְאֵילִ֑ים גְּ֝בָע֗וֹת כִּבְנֵי־צֹֽאן׃
מַה־לְּךָ֣ הַ֭יָּם כִּ֣י תָנ֑וּס הַ֝יַּרְדֵּ֗ן תִּסֹּ֥ב לְאָחֽוֹר׃
הֶ֭הָרִים תִּרְקְד֣וּ כְאֵילִ֑ים גְּ֝בָע֗וֹת כִּבְנֵי־צֹֽאן׃
מִלִּפְנֵ֣י אָ֭דוֹן ח֣וּלִי אָ֑רֶץ מִ֝לִּפְנֵ֗י אֱל֣וֹהַּ יַעֲקֹֽב׃
הַהֹפְכִ֣י הַצּ֣וּר אֲגַם־מָ֑יִם חַ֝לָּמִ֗ישׁ לְמַעְיְנוֹ־מָֽיִם׃

\diluginst
\begin{narrow}\vspace{-12pt}
	\firstword{לֹ֤א לָ֥נוּ}\source{תהלים קטו}
יְיָ֗ לֹ֫א לָ֥נוּ כִּֽי־לְ֭שִׁמְךָ תֵּ֣ן כָּב֑וֹד עַל־חַ֝סְדְּךָ֗ עַל־אֲמִתֶּֽךָ׃
לָ֭מָּה יֹאמְר֣וּ הַגּוֹיִ֑ם אַיֵּה־נָ֗֝א אֱלֹהֵיהֶֽם׃
וֵאלֹהֵ֥ינוּ בַשָּׁמָ֑יִם כֹּ֖ל אֲשֶׁר־חָפֵ֣ץ עָשָֽׂה׃
עֲֽ֭צַבֵּיהֶם כֶּ֣סֶף וְזָהָ֑ב מַ֝עֲשֵׂ֗ה יְדֵ֣י אָדָֽם׃
פֶּֽה־לָ֭הֶם וְלֹ֣א יְדַבֵּ֑רוּ עֵינַ֥יִם לָ֝הֶ֗ם וְלֹ֣א יִרְאֽוּ׃
אׇזְנַ֣יִם לָ֭הֶם וְלֹ֣א יִשְׁמָ֑עוּ אַ֥ף לָ֝הֶ֗ם וְלֹ֣א יְרִיחֽוּן׃
יְדֵיהֶ֤ם ׀ וְלֹ֬א יְמִישׁ֗וּן רַ֭גְלֵיהֶם וְלֹ֣א יְהַלֵּ֑כוּ לֹא־יֶ֝הְגּ֗וּ בִּגְרוֹנָֽם׃
כְּ֭מוֹהֶם יִהְי֣וּ עֹשֵׂיהֶ֑ם כֹּ֖ל אֲשֶׁר־בֹּטֵ֣חַ בָּהֶֽם׃
יִ֭שְׂרָאֵל בְּטַ֣ח בַּייָ֑ עֶזְרָ֖ם וּמָגִנָּ֣ם הֽוּא׃
בֵּ֣ית אַ֭הֲרֹן בִּטְח֣וּ בַייָ֑ עֶזְרָ֖ם וּמָגִנָּ֣ם הֽוּא׃
יִרְאֵ֣י יְיָ֭ בִּטְח֣וּ בַייָ֑ עֶזְרָ֖ם וּמָגִנָּ֣ם הֽוּא׃
\end{narrow}
\firstword{יְיָ זְכָרָ֢נוּ}
יְ֭בָרֵךְ אֶת־בֵּ֣ית יִשְׂרָאֵ֑ל יְ֝בָרֵ֗ךְ אֶת־בֵּ֥ית אַהֲרֹֽן׃
יְ֭בָרֵךְ יִרְאֵ֣י יְיָ֑ הַ֝קְּטַנִּ֗ים עִם־הַגְּדֹלִֽים׃
יֹסֵ֣ף יְיָ֣ עֲלֵיכֶ֑ם עֲ֝לֵיכֶ֗ם וְעַל־בְּנֵיכֶֽם׃
בְּרוּכִ֣ים אַ֭תֶּם לַייָ֑ עֹ֝שֵׂ֗ה שָׁמַ֥יִם וָאָֽרֶץ׃
הַשָּׁמַ֣יִם שָׁ֭מַיִם לַייָ֑ וְ֝הָאָ֗רֶץ נָתַ֥ן לִבְנֵי־אָדָֽם׃
לֹ֣א הַ֭מֵּתִים יְהַֽלְלוּ־יָ֑הּ וְ֝לֹ֗א כׇּל־יֹרְדֵ֥י דוּמָֽה׃
וַאֲנַ֤חְנוּ ׀ נְבָ֘רֵ֤ךְ יָ֗הּ מֵעַתָּ֥ה וְעַד־עוֹלָ֗ם הַֽלְלוּ־יָֽהּ׃

\diluginst
\begin{narrow}\vspace{-12pt}
	\firstword{אָ֭הַבְתִּי}\source{תהלים קטז}
כִּי־יִשְׁמַ֥ע ׀ יְיָ֑ אֶת־ק֝וֹלִ֗י תַּחֲנוּנָֽי׃
כִּי־הִטָּ֣ה אׇזְנ֣וֹ לִ֑י וּבְיָמַ֥י אֶקְרָֽא׃
אֲפָפ֤וּנִי ׀ חֶבְלֵי־מָ֗וֶת וּמְצָרֵ֣י שְׁא֣וֹל מְצָא֑וּנִי צָרָ֖ה וְיָג֣וֹן אֶמְצָֽא׃
וּבְשֵֽׁם־יְיָ֥ אֶקְרָ֑א אָנָּ֥ה יְ֝יָ֗ מַלְּטָ֥ה נַפְשִֽׁי׃
חַנּ֣וּן יְיָ֣ וְצַדִּ֑יק וֵ֖אלֹהֵ֣ינוּ מְרַחֵֽם׃
שֹׁמֵ֣ר פְּתָאיִ֣ם יְיָ֑ דַּ֝לֹּתִ֗י וְלִ֣י יְהוֹשִֽׁיעַ׃
שׁוּבִ֣י נַ֭פְשִׁי לִמְנוּחָ֑יְכִי כִּֽי־יְ֝יָ֗ גָּמַ֥ל עָלָֽיְכִי׃
כִּ֤י חִלַּ֥צְתָּ נַפְשִׁ֗י מִ֫מָּ֥וֶת אֶת־עֵינִ֥י מִן־דִּמְעָ֑ה אֶת־רַגְלִ֥י מִדֶּֽחִי׃
אֶ֭תְהַלֵּךְ לִפְנֵ֣י יְיָ֑ בְּ֝אַרְצ֗וֹת הַחַיִּֽים׃
הֶ֭אֱמַנְתִּי כִּ֣י אֲדַבֵּ֑ר אֲ֝נִ֗י עָנִ֥יתִי מְאֹֽד׃
אֲ֭נִי אָמַ֣רְתִּי בְחׇפְזִ֑י כׇּֽל־הָאָדָ֥ם כֹּזֵֽב׃
\end{narrow}

\firstword{מָה־אָשִׁ֥יב לַייָ֑}
כׇּֽל־תַּגְמוּל֥וֹהִי עָלָֽי׃
כּוֹס־יְשׁוּע֥וֹת אֶשָּׂ֑א וּבְשֵׁ֖ם יְיָ֣ אֶקְרָֽא׃
נְ֭דָרַי לַייָ֣ אֲשַׁלֵּ֑ם נֶגְדָה־נָּ֗֝א לְכׇל־עַמּֽוֹ׃
יָ֭קָר בְּעֵינֵ֣י יְיָ֑ הַ֝מָּ֗וְתָה לַחֲסִידָֽיו׃
אָנָּ֣ה יְיָ כִּֽי־אֲנִ֢י עַ֫בְדֶּ֥ךָ אֲנִי־עַ֭בְדְּךָ בֶּן־אֲמָתֶ֑ךָ פִּ֝תַּ֗חְתָּ לְמֽוֹסֵרָֽי׃
לְֽךָ־אֶ֭זְבַּח זֶ֣בַח תּוֹדָ֑ה וּבְשֵׁ֖ם יְיָ֣ אֶקְרָֽא׃
נְ֭דָרַי לַייָ֣ אֲשַׁלֵּ֑ם נֶגְדָה־נָּ֗֝א לְכׇל־עַמּֽוֹ׃
בְּחַצְר֤וֹת ׀ בֵּ֤ית יְיָ֗ בְּֽת֘וֹכֵ֤כִי יְֽרוּשָׁלָ֗‍ִם הַֽלְלוּ־יָֽהּ׃

\firstword{הַֽלְל֣וּ}\source{תהלים קיז}
אֶת־יְיָ֭ כׇּל־גּוֹיִ֑ם שַׁ֝בְּח֗וּהוּ כׇּל־הָאֻמִּֽים׃
כִּ֥י גָ֘בַ֤ר עָלֵ֨ינוּ ׀ חַסְדּ֗וֹ וֶאֱמֶת־יְיָ֥ לְעוֹלָ֗ם הַֽלְלוּ־יָֽהּ׃

\shatz \source{תהלים קיח}הוֹד֣וּ לַייָ֣ כִּי־ט֑וֹב כִּ֖י לְעוֹלָ֣ם חַסְדּֽוֹ׃ \hfill \break
\kahal \begin{small}הוֹד֣וּ לַייָ֣ כִּי־ט֑וֹב כִּ֖י לְעוֹלָ֣ם חַסְדּֽוֹ׃ \end{small}\\
\shatz יֹאמַר־נָ֥א יִשְׂרָאֵ֑ל כִּ֖י לְעוֹלָ֣ם חַסְדּֽוֹ׃\hfill \break
\kahal \begin{small}הוֹד֣וּ לַייָ֣ כִּי־ט֑וֹב כִּ֖י לְעוֹלָ֣ם חַסְדּֽוֹ׃ \end{small}\\
\shatz יֹאמְרוּ־נָ֥א בֵֽית־אַהֲרֹ֑ן כִּ֖י לְעוֹלָ֣ם חַסְדּֽוֹ׃ \hfill \break
\kahal \begin{small}הוֹד֣וּ לַייָ֣ כִּי־ט֑וֹב כִּ֖י לְעוֹלָ֣ם חַסְדּֽוֹ׃ \end{small}\\
\shatz יֹאמְרוּ־נָ֭א יִרְאֵ֣י יְיָ֑ כִּ֖י לְעוֹלָ֣ם חַסְדּֽוֹ׃\hfill \break
\kahal \begin{small}הוֹד֣וּ לַייָ֣ כִּי־ט֑וֹב כִּ֖י לְעוֹלָ֣ם חַסְדּֽוֹ׃ \end{small}\\
\firstword{מִֽן־הַ֭מֵּצַר}\source{תהלים קיח}
קָרָ֣אתִי יָּ֑הּ עָנָ֖נִי בַמֶּרְחָ֣ב יָֽהּ׃
יְיָ֣ לִ֭י לֹ֣א אִירָ֑א מַה־יַּעֲשֶׂ֖ה לִ֣י אָדָֽם׃
יְיָ֣ לִ֭י בְּעֹזְרָ֑י וַ֝אֲנִ֗י אֶרְאֶ֥ה בְשֹׂנְאָֽי׃
ט֗וֹב לַחֲס֥וֹת בַּייָ֑ מִ֝בְּטֹ֗חַ בָּאָדָֽם׃
ט֗וֹב לַחֲס֥וֹת בַּייָ֑ מִ֝בְּטֹ֗חַ בִּנְדִיבִֽים׃
כׇּל־גּוֹיִ֥ם סְבָב֑וּנִי בְּשֵׁ֥ם יְ֝יָ֗ כִּ֣י אֲמִילַֽם׃
סַבּ֥וּנִי גַם־סְבָב֑וּנִי בְּשֵׁ֥ם יְ֝יָ֗ כִּ֣י אֲמִילַֽם׃
סַבּ֤וּנִי כִדְבוֹרִ֗ים דֹּ֭עֲכוּ כְּאֵ֣שׁ קוֹצִ֑ים בְּשֵׁ֥ם יְ֝יָ֗ כִּ֣י אֲמִילַֽם׃
דַּחֹ֣ה דְחִיתַ֣נִי לִנְפֹּ֑ל וַ֖ייָ֣ עֲזָרָֽנִי׃
עָזִּ֣י וְזִמְרָ֣ת יָ֑הּ וַֽיְהִי־לִ֗֝י לִישׁוּעָֽה׃
ק֤וֹל ׀ רִנָּ֬ה וִישׁוּעָ֗ה בְּאׇהֳלֵ֥י צַדִּיקִ֑ים יְמִ֥ין יְ֝יָ֗ עֹ֣שָׂה חָֽיִל׃
יְמִ֣ין יְיָ֭ רוֹמֵמָ֑ה יְמִ֥ין יְ֝יָ֗ עֹ֣שָׂה חָֽיִל׃
לֹא־אָמ֥וּת כִּֽי־אֶחְיֶ֑ה וַ֝אֲסַפֵּ֗ר מַעֲשֵׂ֥י יָֽהּ׃
יַסֹּ֣ר יִסְּרַ֣נִּי יָּ֑הּ וְ֝לַמָּ֗וֶת לֹ֣א נְתָנָֽנִי׃
פִּתְחוּ־לִ֥י שַׁעֲרֵי־צֶ֑דֶק אָבֹא־בָ֗֝ם אוֹדֶ֥ה יָֽהּ׃
זֶה־הַשַּׁ֥עַר לַייָ֑ צַ֝דִּיקִ֗ים יָבֹ֥אוּ בֽוֹ׃\\
א֭וֹדְךָ כִּ֣י עֲנִיתָ֑נִי וַתְּהִי־לִ֗֝י לִישׁוּעָֽה׃ \\
\scriptsize{ א֭וֹדְךָ כִּ֣י עֲנִיתָ֑נִי וַתְּהִי־לִ֗֝י לִישׁוּעָֽה׃ \\}\normalsize{}
אֶ֭בֶן מָאֲס֣וּ הַבּוֹנִ֑ים הָ֝יְתָ֗ה לְרֹ֣אשׁ פִּנָּֽה׃ \\
\scriptsize{ אֶ֭בֶן מָאֲס֣וּ הַבּוֹנִ֑ים הָ֝יְתָ֗ה לְרֹ֣אשׁ פִּנָּֽה׃ \\}\normalsize{}
מֵאֵ֣ת יְיָ֭ הָ֣יְתָה זֹּ֑את הִ֖יא נִפְלָ֣את בְּעֵינֵֽינוּ׃ \\
\scriptsize{ מֵאֵ֣ת יְיָ֭ הָ֣יְתָה זֹּ֑את הִ֖יא נִפְלָ֣את בְּעֵינֵֽינוּ׃ \\}\normalsize{}
זֶה־הַ֭יּוֹם עָשָׂ֣ה יְיָ֑ נָגִ֖ילָה וְנִשְׂמְחָ֣ה בֽוֹ׃ \\
\scriptsize{ זֶה־הַ֭יּוֹם עָשָׂ֣ה יְיָ֑ נָגִ֖ילָה וְנִשְׂמְחָ֣ה בֽוֹ׃ } \normalsize{}


\instruction{ש״ץ ואח״כ הקהל׃}\\
אָנָּ֣א יְיָ֭ הוֹשִׁ֘יעָ֥ה נָּ֑א \hfill אָנָּ֣א יְיָ֭ הוֹשִׁ֘יעָ֥ה נָּ֑א\\
אָנָּ֥א יְ֝יָ֗ הַצְלִ֘יחָ֥ה נָּֽא׃ \hfill אָנָּ֥א יְ֝יָ֗ הַצְלִ֘יחָ֥ה נָּֽא׃\\
בָּר֣וּךְ הַ֭בָּא בְּשֵׁ֣ם יְיָ֑ בֵּ֝רַ֥כְנוּכֶ֗ם מִבֵּ֥ית יְיָ׃\\
\scriptsize{בָּר֣וּךְ הַ֭בָּא בְּשֵׁ֣ם יְיָ֑ בֵּ֝רַ֥כְנוּכֶ֗ם מִבֵּ֥ית יְיָ׃}\\
\normalsize{אֵ֤ל ׀ יְיָ וַיָּ֢אֶ֫ר לָ֥נוּ אִסְרוּ־חַ֥ג בַּעֲבֹתִ֑ים עַד־קַ֝רְנ֗וֹת הַמִּזְבֵּֽחַ׃}\\
\scriptsize{אֵ֤ל ׀ יְיָ וַיָּ֢אֶ֫ר לָ֥נוּ אִסְרוּ־חַ֥ג בַּעֲבֹתִ֑ים עַד־קַ֝רְנ֗וֹת הַמִּזְבֵּֽחַ׃}\\
\normalsize{אֵלִ֣י אַתָּ֣ה וְאוֹדֶ֑ךָּ אֱ֝לֹהַ֗י אֲרוֹמְמֶֽךָּ׃}\\
\scriptsize{אֵלִ֣י אַתָּ֣ה וְאוֹדֶ֑ךָּ אֱ֝לֹהַ֗י אֲרוֹמְמֶֽךָּ׃}\\
\normalsize{הוֹד֣וּ לַייָ֣ כִּי־ט֑וֹב כִּ֖י לְעוֹלָ֣ם חַסְדּֽוֹ׃}\\
\scriptsize{הוֹד֣וּ לַייָ֣ כִּי־ט֑וֹב כִּ֖י לְעוֹלָ֣ם חַסְדּֽוֹ׃} \\
\normalsize{}



\negline

\firstword{יְהַלְלֽוּךָ}
יְיָ אֱלֹהֵֽינוּ כׇּל־מַעֲשֶֽׂיךָ וַחֲסִידֶֽיךָ צַדִּיקִים עוֹשֵׂי רְצֹנֶֽךָ וְכׇל־עַמְּךָ בֵּית יִשְׂרָאֵל בְּרִנָּה יוֹדוּ וִיבָרְכוּ וִישַׁבְּחוּ וִיפָאֲרוּ וִירוֹמֲמוּ וְיַעֲרִֽיצוּ וְיַקְדִּֽישׁוּ וְיַמְלִֽיכוּ אֶת־שִׁמְךָ מַלְכֵּֽנוּ כִּי לְךָ טוֹב לְהוֹדוֹת וּלְשִׁמְךָ נָאֶה לְזַמֵּר כִּי מֵעוֹלָם וְעַד עוֹלָם אַתָּה אֵל׃ בָּרוּךְ אַתָּה יְיָ מֶֽלֶךְ מְהֻלָּל בַּתִּשְׁבָּחוֹת׃\\

%\ifboolexpr{togl {includeshabbat} and togl {includeweekday} and not togl {includeChM}}{\englishinst{On Shabbat, continue with Full Kaddish on page \pageref{shacharitShabbatYTtitkabel}. On weekday Rosh \d{H}odesh, continue with Full Kaddish on page \pageref{end of shacharis}, followed by the Psalm of the Day as relevant on page \pageref{shir_shel_yom}. On \d{H}anukka that is not Rosh \d{H}odesh, continue with Half Kaddish on page \pageref{hatzi_kaddish}.}}{}

\englishinst{On Shabbat, Festivals, and Rosh \d{H}odesh, recite Full Kaddish followed by the Psalm of the Day. Then Mourner's Kaddish is read, followed by the Torah service on page \pageref{shabYTtorah} for Shabbat and Festivals and page \pageref{weekday torah} for Intermediate Festival Days and Rosh \d{H}odesh.}

%}

%\vfill
\label{shacharitShabbatYTtitkabel}
\fullkaddish
\section[שיר של יום]{\adforn{53} שיר של יום‎ \adforn{25}}

\ifboolexpr{togl {includefestival}}{\weekdayshir}{}
\begin{small}הַיּוֹם שַּׁבָּת קֹֽדֶשׁ שֶׁבּוֹ הָיוּ הַלְוִיִּם אוֹמְרִים בְּבֵית־הַמִּקְדָּשׁ׃\end{small}\\
\mizmorshabbat\\

\ifboolexpr{togl {includeshabbat}}{
\RChBarekhi

\instruction{בחנכה׃}
\chanukat

\ledavid\\}{}
\mournerskaddish
\vspace{0.25in}
\chapter[סדר קריאת התורה]{\adforn{53} סדר קריאת התורה \adforn{25}}

\longpesicha

\brikhshmei

\textbf{שְׁמַ֖ע יִשְׂרָאֵ֑ל יְיָ֥ אֱלֹהֵ֖ינוּ יְיָ֥ ׀ אֶחָֽד׃}

\textbf{אֶחָד אֱלֹהֵֽינוּ גָּדוֹל אֲדוֹנֵֽנוּ קָדוֹשׁ שְׁמוֹ׃}

\gadlu

\label{al hakol}
%\firstword{עַל הַכֹּל}
%יִתְגַּדַּל וְיִתְקַדַּשׁ וְיִשְׁתַּבַּח וְיִתְפָּאַר וְיִתְרוֹמַם וְיִתְנַשֵּׂא שְׁמוֹ שֶׁל מֶֽלֶךְ מַלְכֵי הַמְּלָכִים הַקְָּדוֹשׁ בָּרוּךְ הוּא בָּעוֹלָמוֹת שֶׁבָּרָא הָעוֹלָם הַזֶּה וְהָעוֹלָם הַבָּא כִּרְצוֹנוֹ וְכִרְצוֹן יְרֵאָיו וְכִרְצוֹן כׇּל־בֵּית יִשְׂרָאֵל׃ צוּר הָעוֹלָמִים אֲדוֹן כׇּל־הַבְּרִיּוֹת אֱלֽוֹהַּ כׇּל־הַנְּפָשׁוֹת הַיּוֹשֵׁב בְּמֶרְחֲבֵי מָרוֹם הַשּׁוֹכֵן בִּשְׁמֵי שְׁמֵי קֶֽדֶם׃ קְדֻשָּׁתוֹ עַל הַחַיּוֹת וּקְדֻשָּׁתוֹ עַל כִּסֵּא הַכָּבוֹד׃ וּבְכֵן יִתְקַדַּשׁ שִׁמְךָ בָּֽנוּ יְיָ אֱלֹהֵֽינוּ לְעֵינֵי כׇּל־חָי׃ וְנֹאמַר לְפָנָיו שִׁיר חָדָשׁ כַּכָּתוּב׃
%שִׁ֥ירוּ \source{תהלים סח}לֵֽאלֹהִֽ֘ים זַמְּר֢וּ שְׁ֫מ֥וֹ סֹ֡לּוּ לָֽרֹכֵ֣ב בָּֽ֭עֲרָבוֹת בְּיָ֥הּ שְׁ֝מ֗וֹ וְעִלְז֥וּ לְפָנָֽיו׃ וְנִרְאֵֽהוּ עַֽיִן בְּעַֽיִן בְּשׁוּבוֹ אֶל נָוֵֽהוּ כַּכָּתוּב׃
%\source{ישעיה נב}%
%כִּ֣י עַ֤יִן בְּעַ֨יִן֙ יִרְא֔וּ בְּשׁ֥וּב יְיָ֖ צִיּֽוֹן׃ וְנֶאֱמַר׃
%וְנִגְלָ֖ה \source{ישעיה מ}כְּב֣וֹד יְיָ֑ וְרָא֤וּ כׇל־בָּשָׂר֙ יַחְדָּ֔ו כִּ֛י פִּ֥י יְיָ֖ דִּבֵּֽר׃

\avharachamim

\vayaazor

\torahbarachu

\hagomel

\misheberakhbaby

%\misheberakhbarmitzva

%\instruction{מי שבירך לחתן:}\\
%מִי שֶׁבֵּרַךְ אֲבוֹתֵֽינוּ אַבְרָהָם יִצְחָק וְיַעֲקֹב הוּא יְבָרֵךְ אֶת־הֶחָתָן \instruction{(פלוני בן פלוני)} וְאֶת־כַּלָתוֹ \instruction{פלונית בת פלוני} בַּעֲבוּר שֶׁנָדְרוּ נִדְבַת לִבָּם... בִּשְׂכַר זֶה הַקָדוֹשׁ בָּרוּךְ הוּא יְבָרֵךְ אוֹתָם וְיִתֵּן לָהֶם בְּרָכָה וְהַצְלָחָה בְּכׇל־מַעֲשֵׂה יְדֵיהֶם וְיִזְכוּ לִבְנוֹת בַּֽיִת בְּיִשְׂרָאֵל לְשֵׁם וְלִתְהִלָה וְנֹאמַר אָמֵן:

\misheberakhcholim{‏שַׁבָּת הִיא מִלִּזְעוֹק}

%\misheberakholim{וְלִכְבוֹד הַשַּׁבָּת}
\englishinst{Half kaddish is recited before the Maftir is read.}
\halfkaddish

\hagbaha

\ssubsection{\adforn{18} סדר קריאת ההפטרה \adforn{17}}

\englishinst{Before reading the Haftara:}
\firstword{בָּר֙וּךְ}
אַתָּ֤ה יְ֙יָ אֱלֹ֙הֵֽינוּ֙ מֶ֣לֶךְ הָעוֹלָ֔ם אֲשֶׁ֤ר בָּחַר֙ בִּנְבִיאִ֣ים טוֹבִ֔ים וְרָצָ֥ה בְדִבְרֵיהֶ֖ם הַנֶּֽאֱמָרִ֣ים בֶּאֱמֶ֑ת בָּר֨וּךְ אַתָּ֜ה יְיָ֗ הַבּוֹחֵר֚ בַּתּוֹרָה֙ וּבְמֹשֶׁ֣ה עַבְדּ֔וֹ וּבְיִשְׂרָאֵ֣ל עַמּ֔וֹ וּבִנְבִיאֵ֥י הָֽאֱמֶ֖ת וְהַצֶֽדֶק׃

\englishinst{After reading the Haftara:}
\firstword{בָּרוּךְ}
אַתָּה יְיָ אֱלֹהֵֽינוּ מֶֽלֶךְ הָעוֹלָם צוּר כׇּל־הָעוֹלָמִים צַדִּיק בְּכׇל־הַדּוֹרוֹת הָאֵל הַנֶּאֱמָן הָאוֹמֵר וְעוֹשֶׂה הַמְדַבֵּר וּמְקַיֵּם שֶׁכׇּל־דְּבָרָיו אֱמֶת וָצֶֽדֶק׃ נֶאֱמָן אַתָּה הוּא יְיָ אֱלֹהֵֽינוּ וְנֶאֱמָנִים דְּבָרֶֽיךָ וְדָבָר אֶחָד מִדְּבָרֶֽיךָ אָחוֹר לֹא יָשׁוּב רֵיקָם כִּי אֵל מֶֽלֶךְ נֶאֱמָן וְרַחֲמָן אַֽתָּה׃ בָּרוּךְ אַתָּה יְיָ הָאֵל הַנֶּאֱמָן בְּכׇל־דְּבָרָיו׃

\firstword{רַחֵם}
עַל צִיּוֹן כִּי הִיא בֵּית חַיֵּֽינוּ וְלַעֲלֽוּבַת נֶֽפֶשׁ %תִּנְקֺם נָקָם
תּוֹשִֽׁיעַ בִּמְהֵרָה בְיָמֵֽינוּ׃ בָּרוּךְ אַתָּה יְיָ מְשַׂמֵּֽחַ צִיּוֹן בְּבָנֶֽיהָ׃

\firstword{שַׂמְּחֵֽנוּ}
יְיָ אֱלֹהֵֽינוּ בְּאֵלִיָּֽהוּ הַנָּבִיא עַבְדֶּֽךָ וּבְמַלְכוּת בֵּית דָּוִד מְשִׁיחֶֽךָ בִּמְהֵרָה יָבֹא וְיָגֵל לִבֵּֽנוּ עַל כִּסְאוֹ לֹא יֵשֵׁב זָר וְלֹא יִנְחֲלוּ עוֹד אֲחֵרִים אֶת־כְּבוֹדוֹ כִּי בְשֵׁם קׇדְשְׁךָ נִשְׁבַּעְתָּ לוֹ שֶׁלֹּא יִכְבֶּה נֵרוֹ לְעוֹלָם וָעֶד׃ בָּרוּךְ אַתָּה יְיָ מָגֵן דָּוִד׃

\firstword{עַל הַתּוֹרָה}
וְעַל הָעֲבוֹדָה וְעַל הַנְּבִיאִים וְעַל יוֹם הַשַּׁבָּת הַזֶּה שֶׁנָּתַֽתָּ לָֽנוּ יְיָ אֱלֹהֵֽינוּ לִקְדֻשָּׁה וְלִמְנוּחָה לְכָבוֹד וּלְתִפְאָֽרֶת׃ עַל הַכֹּל יְיָ אֱלֹהֵֽינוּ אָֽנוּ מוֹדִים לָךְ וּמְבָרְכִים אוֹתָךְ יִתְבָּרַךְ שִׁמְךָ בְּפִי כׇל־חַי תָּמִיד לְעוֹלָם וָעֶד׃ בָּרוּךְ אַתָּה יְיָ מְקַדֵּשׁ הַשַּׁבָּת׃


\ssubsection{\adforn{18} יקום פרקן \adforn{17}}

\yekumpurkans

\prayergovt

\ssubsection{\adforn{18} ברכת החודש \adforn{17}}

יְהִי רָצוֹן מִלְּפָנֶֽיךָ יְיָ אֱלֹהֵֽינוּ וֵאלֹהֵי אֲבוֹתֵֽינוּ
שֶׁתְּחַדֵּשׁ עָלֵֽינוּ אֶת־הַחֹדֶשׁ הַזֶּה לְטוֹבָה וְלִבְרָכָה \middot
וְתִתֶּן־לָנוּ חַיִּים אֲרוּכִים,
חַיִּים שֶׁל שָׁלוֹם,
חַיִּים שֶׁל טוֹבָה,
חַיִּים שֶׁל בְּרָכָה,
חַיִּים שֶׁל פַּרְנָסָה,
חַיִּים שֶׁל חִלּוּץ עֲצָמוֹת,
חַיִּים שֶׁיֵשׁ בָּהֶם יִרְאַת שָׁמַֽיִם וְיִרְאַת חֵטְא,
חַיִּים שֶׁאֵין בָּהֶם בּוּשָׁה וּכְלִמָּה,
חַיִּים שֶׁל עֽשֶׁר וְכָבוֹד,
חַיִּים שֶׁתְּהֵא בָֽנוּ אַהֲבַת תּוֹרָה וְיִרְאַת שָׁמַֽיִם,
חַיִּים שֶׁיְּמַּלֵא יְיָ מִשְׁאֲלוֹת לִבֵּנוּ לְטוֹבָה \middot אָמֵן סֶלָה׃

\englishinst{Some announce the Molad at this point.}
\firstword{מִי שֶׁעָשָׂה}
נִסִּים לַאֲבוֹתֵֽינוּ וְגָאַל אוֹתָם מֵעַבְדוּת לְחֵרוּת \middot הוּא יִגְאַל אוֹתָנוּ בְּקָרוֹב וִיקַבֵּץ נִדָּחֵינוּ מֵאַרְבַּע כַּנְפוֹת הָאָֽרֶץ חֲבֵרִים כׇּל־יִשְׂרָאֵל \middot וְנֹאמַר אָמֵן׃

רֹאשׁ חֹדֶש ... יִהְיֶה בְּיוֹם ... הַבָּא עָלֵֽינוּ וְעַל כׇּל־יִשְׂרָאֵל לְטוֹבָה׃

יְחַדְּשֵׁהוּ הַקָּדוֹשׁ בָּרוּךְ הוּא, עָלֵֽינוּ וְעַל כׇּל־עַמּוֹ בֵּית יִשְׂרָאֵל, לְחַיִּים וּלְשָׁלוֹם, לְשָׂשׂוֹן וּלְשִׂמְחָה, לִישׁוּעָה וּלְנֶחָמָה, וְנֹאמַר אָמֵן׃

\label{avharachamim}
%\instruction{א״א אב הרחמים ביו״ט ובשבת מברכים (אבל אומרים בימי ספירה ובר״ח אב ובימים שאומרים יזכור)}\\
\englishinst{There are several customs as to when to say Av HaRa\d{h}amim. Most omit it on Shabbat Mevarekhim (except during the Omer and the Three Weeks, when it is said) or on other festive Shabbatot (such as the Four Parshiyot or Rosh \d{H}odesh). Some recite it only on the Shabbat before Shavu'ot and Shabbat \d{H}azon.}
\firstword{אַב הָרַחֲמִים}
שׁוֹכֵן מְרוֹמִים בְּרַחֲמָיו הָעֲצוּמִים הוּא יִפְקוֹד בְּרַחֲמִים הַחֲסִידִים וְהַיְשָׁרִים וְהַתְּמִימִים קְהִילּוֹת הַקֹּֽדֶשׁ שֶׁמָּסְרוּ נַפְשָׁם עַל קְדֻשַּׁת הַשֵּׁם \source{שמ״ב א}הַנֶּאֱהָבִ֤ים וְהַנְּעִימִם֙ בְּחַיֵּיהֶ֔ם וּבְמוֹתָ֖ם לֹ֣א נִפְרָ֑דוּ׃ מִנְּשָׁרִים קַֽלּוּ וּמֵאֲרָיוֹת גָּבֵֽרוּ לַעֲשׂוֹת רְצוֹן קוֹנָם וְחֵֽפֶץ צוּרָם׃ יִזְכְּרֵם אֱלֹהֵֽינוּ לְטוֹבָה עִם שְׁאָר צַדִּיקֵי עוֹלָם וְיִנְקוֹם בְּיָמֵֽינוּ לְעֵינֵֽינוּ נִקְמַת דַּם עֲבָדָיו הַשָּׁפוּךְ כַּכָּתוּב בְּתוֹרַת מֹשֶׁה אִישׁ הָאֱלֹהִים׃ \source{דברים לב}הַרְנִ֤ינוּ גוֹיִם֙ עַמּ֔וֹ כִּ֥י דַם־עֲבָדָ֖יו יִקּ֑וֹם וְנָקָם֙ יָשִׁ֣יב לְצָרָ֔יו וְכִפֶּ֥ר אַדְמָת֖וֹ עַמּֽוֹ׃ וְעַל יְדֵי עֲבָדֶֽיךָ הַנְּבִיאִים כָּתוּב לֵאמֹר׃ \source{יואל ד}וְנִקֵּ֖יתִי דָּמָ֣ם לֹֽא־נִקֵּ֑יתִי וַֽיְיָ֖ שֹׁכֵ֥ן בְּצִיּֽוֹן׃ וּבְכִתְבֵי הַקֹּֽדֶשׁ נֶאֱמַר׃ \source{תהלים עט}לָ֤מָּה יֹֽאמְר֣וּ הַגּוֹיִם֘ אַיֵּ֢ה אֱלֹֽהֵ֫יהֶ֥ם יִוָּדַ֣ע בַּגֹּייִ֣ם לְעֵינֵ֑ינוּ נִ֝קְמַ֗ת דַּם־עֲבָדֶ֥יךָ הַשָּׁפֽוּךְ׃ וְאוֹמֵר׃ \source{תהלים ט}כִּ֤י דֹרֵ֣שׁ דָּ֭מִים אֹתָ֣ם זָכָ֑ר לֹ֥א שָׁ֝כַ֗ח צַֽעֲקַ֥ת עֲנָוִֽים׃ וְאוֹמֵר׃ \source{תהלים קי}יָדִ֣ין בַּ֭גּוֹיִם מָלֵ֣א גְוִיּ֑וֹת מָ֥חַץ רֹ֝֗אשׁ עַל־אֶ֥רֶץ רַבָּֽה׃ מִ֭נַּחַל בַּדֶּ֣רֶךְ יִשְׁתֶּ֑ה עַל־כֵּ֝֗ן יָרִ֥ים רֹֽאשׁ׃

\ashrei

\yehalelu

\firstword{מִזְמ֗וֹר לְדָ֫וִ֥ד}\source{תהלים כט} %
הָב֣וּ לַ֭ייָ בְּנֵ֣י אֵלִ֑ים הָב֥וּ לַ֝ייָ֗ כָּב֥וֹד וָעֹֽז׃
הָב֣וּ לַ֭ייָ כְּב֣וֹד שְׁמ֑וֹ הִשְׁתַּחֲו֥וּ לַ֝ייָ֗ בְּהַדְרַת־קֹֽדֶשׁ׃
ק֥וֹל יְיָ֗ עַל־הַ֫מָּ֥יִם אֵֽל־הַכָּב֥וֹד הִרְעִ֑ים יְ֝יָ֗ עַל־מַ֥יִם רַבִּֽים׃
קוֹל־יְיָ֥ בַּכֹּ֑חַ ק֥וֹל יְ֝יָ֗ בֶּהָדָֽר׃
ק֣וֹל יְיָ֭ שֹׁבֵ֣ר אֲרָזִ֑ים וַיְשַׁבֵּ֥ר יְ֝יָ֗ אֶת־אַרְזֵ֥י הַלְּבָנֽוֹן׃
וַיַּרְקִידֵ֥ם כְּמוֹ־עֵ֑גֶל לְבָנ֥וֹן וְ֝שִׂרְיֹ֗ן כְּמ֣וֹ בֶן־רְאֵמִֽים׃
קוֹל־יְיָ֥ חֹצֵ֗ב לַהֲב֥וֹת אֵֽשׁ׃
ק֣וֹל יְיָ֭ יָחִ֣יל מִדְבָּ֑ר יָחִ֥יל יְ֝יָ֗ מִדְבַּ֥ר קָדֵֽשׁ׃
ק֤וֹל יְיָ֨ ׀ יְחוֹלֵ֣ל אַיָּלוֹת֮ וַֽיֶּחֱשֹׂ֢ף יְעָ֫ר֥וֹת וּבְהֵיכָל֑וֹ כֻּ֝לּ֗וֹ אֹמֵ֥ר כָּבֽוֹד׃
יְיָ֭ לַמַּבּ֣וּל יָשָׁ֑ב וַיֵּ֥שֶׁב יְ֝יָ֗ מֶ֣לֶךְ לְעוֹלָֽם׃
יְיָ֗ עֹ֭ז לְעַמּ֣וֹ יִתֵּ֑ן יְיָ֓ ׀ יְבָרֵ֖ךְ אֶת־עַמּ֣וֹ בַשָּׁלֽוֹם׃

\etzchaim

\halfkaddish


\vspace{0.4in}
\chapter[מוסף לשבת]{\adforn{47} מוסף לשבת \adforn{19}}
\nopagebreak
\vspace{0.25in}
\nopagebreak
\amidaopening{\shabbosshuva}{\englishinst{During the repetition of the Amidah, Kedusha is said here}}

\begin{Center}\ssubsection{\adforn{48} קדושה \adforn{22}}\end{Center}
\nopagebreak
\begin{footnotesize}
\begin{longtable}{l p{3.5in}}
\shatz &
נַעֲרִיצְךָ וְנַקְדִּישְׁךָ כְּסוֹד שִֽׂיחַ שַׂרְפֵי קֹֽדֶשׁ \middot הַמַּקְדִּישִׁים שִׁמְךָ בַּקֹּֽדֶשׁ כַּכָּתוּב עַל יַד נְבִיאֶךָ וְקָרָ֨א זֶ֤ה אֶל־זֶה֙ וְאָמַ֔ר \\

\vkahalchazzan &
\kadoshkadoshkadosh\\

\shatz &
כְּבוֹדוֹ מָלֵא עוֹלָם מְשָׁרְתָיו שׁוֹאֲלִים זֶה לָזֶה \middot אַיֵּה מְקוֹם כְּבוֹדוֹ לְעֻמָּתָם בָּרוּךְ יֹאמֵֽרוּ׃\\

\vkahalchazzan &
\barukhhashem \\

\shatz &
מִמְּקוֹמוֹ הוּא יִֽפֶן בְּרַחֲמִים וְיָחוֹן עַם הַמְיַחֲדִים שְׁמוֹ \middot עֶֽרֶב וָבֹֽקֶר בְּכׇל־יוֹם תָּמִיד פַּעֲמַֽיִם בְּאַהֲבָה שְׁמַע אוֹמְרִים׃ \\

\vkahalchazzan &
\textbf{שְׁמַ֖ע יִשְׂרָאֵ֑ל יְיָ֥ אֱלֹהֵ֖ינוּ יְיָ֥ אֶחָֽד׃} \\

\shatz &
[אֶחָד] הוּא אֱלֹהֵֽינוּ הוּא אָבִֽינוּ הוּא מַלְכֵּֽנוּ הוּא מוֹשִׁיעֵֽנוּ \middot וְהוּא יַשְׁמִיעֵֽנוּ בְּרַחֲמָיו שֵׁנִית לְעֵינֵי כׇּל־חַי לִהְיוֹת לָכֶם לֵאלֹהִים \\

\vkahalchazzan &
\textbf{אֲנִ֖י יְיָ֥ אֱלֹהֵיכֶֽם׃}\\
%\source{במדבר טו}

\shatz &
וּבְדִבְרֵי קׇדְשְׁךָ כָּתוּב לֵאמֹר׃ \\

\vkahalchazzan &
\yimloch\\

\shatz &
לְדוֹר וָדוֹר נַגִּיד גׇּדְלֶֽךָ וּלְנֵֽצַח נְצָחִים קְדֻשָּׁתְךָ נַקְדִּישׁ וְשִׁבְחֲךָ אֱלֹהֵֽינוּ מִפִּֽינוּ לֹא יָמוּשׁ לְעוֹלָם וָעֶד כִּי אֵל מֶֽלֶךְ גָּדוֹל וְקָדוֹשׁ אַֽתָּה׃ בָּרוּךְ אַתָּה יְיָ *הָאֵל
(*\instruction{בשבת שובה:}
הַמֶּֽלֶךְ)
הַקָּדוֹשׁ׃\\

\end{longtable}
\end{footnotesize}
\vspace{-24pt}
%\instruction{לשבת שובה:}\\
%אֱלֹֽהֵיכֶֽם שׁוֹפֵט צֶֽדֶק וּבְמִישׁוֹר לְאֻמִּים מִדַּת הַדִּין יַהֲפוֹךְ לְרַחֲמִים וִיכַפֵּר זְדוֹנוֹת וּשְׁגָגוֹת עֲוֹנוֹת וַאֲשָׁמִים אָדוֹן חֲטָאִים הַצְהֵר וְהַלְבֵּן כְּתָמִים לְכַפָּרָה מְקַוִּים לֵילוֹת וְיָמִים חֲזֵה עַמְּךָ מִתְעַנִּים וְצָמִים זְכוֹר בְּרִית רִאשׁוֹנִים הַיְשָׁרִים וְהַתְּמִימִים קַיֵּם לִבְנֵיהֶם שְׁבוּעַת קְדוּמִים וְיִמְלֹךְ מַלְכָּם לְעוֹלְמֵי עוֹלָמִים׃

%\instruction{לשבת בראשית:}\\
%אֱלֹֽהֵיכֶֽם יַשְׂכִּיל עַבְדּוֹ יִכּוֹן כִּסְאוֹ כְּמֵרֵאשִׁית הָעִיר עַל תִּלָּהּ יִבְנֶה וְאַרְמוֹן יָשִׁית וּמִזְבְּחוֹ יִרְפָּא וּמְקוֹם נִסּוּךְ הָשִּׁית דְּרוֹר יִקְרָא הֱיוֹתָהּ חׇפְשִׁית הָעוֹבֵד בָּהּ יִמָּחֵץ כְּעַל יַד לָקַח כּוּשִׁית חֹזֶק זַעְמּוֹ קָמָיו יָפִיץ חֲרִישִׁית יִמְלוֹךְ עַל עַם שׁוֹמֵר שַׁבָּת בְּרֵאשִׁית׃
%
%\instruction{לשבת חנוכה:}\\
%אֱלֹֽהֵיכֶֽם יָבִיא מְשִׁיחוֹ אֵזוֹר צֶֽדֶק וּמִשְׁפָּט רָבִיד הָרִשְׁעָה יָמִית בְּשֵֽׁבֶט פִּיו אוֹיֵב יַאֲבִיד וְעַל הַנִּסִּים הַלֵּל לִגְמוֹר צָעִיר מֵהַעֲבִיד דָּת חֲדָשָׁה יְחַדֵּשׁ הַטּוֹת אֹֽזֶן מֵהַכְבִּיד חֲלוֹף אֱלִילִים זְבוּל קֹֽדֶשׁ יְקוֹמֵם יַזְבִּיד וְנָשִׁיר \source{תהלים ל}מִזְמ֡וֹר שִׁ֤יר חֲנֻכַּ֖ת הַבַּ֣יִת לְדָוִֽד׃
%
%\instruction{לשבת נחמו:}\\
%אֱלֹֽהֵיכֶֽם יוֹסִיף יָדוֹ לְקַבֵּץ נְפוּצוֹתֵֽיכֶם הָעֵת יָחִישׁ לֵאמֹר צְאוּ מֵאֲסוּרֵיכֶם וּמַלְאַךְ בְּרִיתוֹ יִשְׁלַח לְהָשִׁיב לְבַבְכֶם דַּרְכּוֹ פַּנּוּ וְהָיָה הֶעָקֹב לְמִישׁוֹר וּלְעִירוֹ יְקַבֶּצְכֶם הַשּׁוֹמְרִים שַׁבָּת בְּרִית הִיא בֵּינוֹ וּבֵינֵיכֶם בָּרֲכוּ שְׁמוֹ וְאַל תִּתְּנוּ דֳּמִי לָכֶם חַי זֵֽכֶר קָדְשׁוֹ נַֽחֲמ֖וּ עַמִּ֑י יֹאמַ֖ר אֱלֹהֵיכֶֽם׃ כִּי אֲנִי הוּא מַלְכְּכֶם וְאֶמְלֹךְ עֲלֵיכֶם׃
%
%
%\instruction{לשבת נשואין:}\\
%אֱלֹֽהֵיכֶֽם שִׁכְנוֹ שָׂם כֵּס עֵילָמוֹ מַלְאָכָיו רוּחוֹת רָז לֹא לָֽמוֹ וְאַיֵּה מְקוֹם כְּבוֹדוֹ הוֹלָמוֹ אוֹמְרִים בָּר֥וּךְ כְּבוֹד־יְיָ֖ מִמְּקוֹמֽוֹ׃ לָעֵת יִמְלוֹךְ לְיוֹם קוּמוֹ חֲזֵה צִיּוֹן גָּנוֹן בִּמְקוֹמוֹ זֶה יִרְאוּ וְיֵדְעוּ כׇּל־יְקוּמוֹ קְרוֹא לִירֽוּשָׁלַֽםִ כִּסֵּא מְקוֹמוֹ׃
%
%\instruction{לשבת ברית מילה:}\\
%אֱלֹֽהֵיכֶֽם אֲנִי זוֹכֵר הַבְּרִית הִנֵּה שׁוֹלֵֽחַ לִשְׁאֵרִית אֶת־נִסְתָּר בְּנַֽחַל כְּרִית לְבַשֵּׂר טוֹב וְשָׁלוֹם בְּאַחֲרִית יֵדְעוּ זֹאת כׇּל־כְּרֽוּתֵי בְרִית אָמַר לְצִיּוֹן מָלַךְ אֱלֹהַֽיִךְ זְהוֹרִית׃
%
%\end{footnotesize}



\sepline

\englishinst{On Shabbat Rosh \d{H}odesh, recite the text beginning \hebineng{יצרת אתה} on page \pageref{shabbosroshchodesh}.}
\firstword{תִּכַּנְתָּ שַׁבָּת}
רָצִֽיתָ קׇרְבְּנוֹתֶֽיהָ \middot צִוִּֽיתָ פֵּרוּשֶֽׁיהָ עִם סִדּוּרֵי נְסָכֶֽיהָ מְעַנְּגֶֽיהָ לְעוֹלָם כָּבוֹד יִנְחָֽלוּ טוֹעֲמֶֽיהָ חַיִּים זָכוּ וְגַם הָאוֹהֲבִים דְּבָרֶֽיהָ גְּדֻלָּה בָּחָֽרוּ \middot אָז מִסִּינַי נִצְטַוּוּ עָלֶֽיהָ וַתְּצַוֵּֽנוּ יְיָ אֱלֹהֵֽינוּ לְהַקְרִיב בָּהּ קׇרְבַּן מוּסַף שַׁבָּת כָּרָאוּי׃ 
יְהִי רָצוֹן מִלְּפָנֶֽיךָ יְיָ אֱלֹהֵֽינוּ וֵאלֹהֵי אֲבוֹתֵֽינוּ שֶׁתַּעֲלֵֽנוּ בְשִׂמְחָה לְאַרְצֵֽנוּ וְתִטָּעֵֽנוּ בִּגְבוּלֵֽנוּ \middot וְשָׁם נַעֲשֶׂה לְפָנֶֽיךָ אֶת־קׇרְבְּנוֹת חוֹבוֹתֵֽינוּ תְּמִידִים כְּסִדְרָם וּמוּסָפִים כְּהִלְכָתָם׃ וְאֶת־מוּסַף יוֹם הַשַּׁבָּת הַזֶּה נַעֲשֶׂה וְנַקְרִיב לְפָנֶֽיךָ בְּאַהֲבָה כְּמִצְוַת רְצוֹנֶֽךָ \middot כְּמוֹ שֶׁכָּתַֽבְתָּ עָלֵֽינוּ בְּתוֹרָתֶֽךָ עַל יְדֵי מֹשֶׁה עַבְדְּךָ מִפִּי כְבוֹדֶֽךָ כָּאָמוּר׃

\shabmusafpesukim

\firstword{יִשְׂמְחוּ בְמַלְכוּתְךָ}
שׁוֹמְרֵי שַׁבָּת וְקֽוֹרְאֵי עֹֽנֶג \middot עַם מְקַדְּשֵׁי שְׁבִיעִי כֻּלָּם יִשְׂבְּעוּ וְיִתְעַנְּגוּ מִטּוּבֶֽךָ \middot וּבַשְּׁבִיעִי רָצִֽיתָ בּוֹ וְקִדַּשְׁתּוֹ \middot חֶמְדַּת יָמִים אוֹתוֹ קָרָֽאתָ זֵֽכֶר לְמַעֲשֵׂה בְרֵאשִׁית׃

\shabboskiddushhayom{} \instruction{רצה...}

\begin{sometimes}

\label{shabbosroshchodesh}
\instruction{בראש חדש:}\\
\textbf{אַתָּה יָצַֽרְתָּ}
עוֹלָמְךָ מִקֶּֽדֶם כִּלִּֽיתָ מְלַאכְתְּךָ בַּיוֹם הַשְּׁבִיעִי \middot אָהַֽבְתָּ אוֹתָֽנוּ וְרָצִֽיתָ בָּֽנוּ וְרוֹמַמְתָּֽנוּ מִכׇּל־הַלְּשׁוֹנוֹת וְקִדַּשְׁתָּֽנוּ בְּמִצְוֹתֶֽיךָ וְקֵרַבְתָּֽנוּ מַלְכֵּנוּ לַעֲבוֹדָתֶֽךָ וְשִׁמְךָ הַגָּדוֹל וְהַקָּדוֹשׁ עָלֵֽינוּ קָרָֽאתָ׃ וַתִּתֶּן לָֽנוּ יְיָ אֱלֹהֵֽינוּ בְּאַהֲבָה שַׁבָּתוֹת לִמְנוּחָה וְרָאשֵׁי חֳדָשִׁים לְכַפָּרָה׃ וּלְפִי שֶׁחָטָֽאנוּ לְפָנֶֽיךָ אֲנַֽחְנוּ וַאֲבוֹתֵֽינוּ חָרְבָה עִירֵֽנוּ וְשָׁמֵם בֵּית מִקְדָּשֵֽׁנוּ וְגָלָה יְקָרֵֽנוּ וְנֻּטַּל כָּבוֹד מִבֵּית חַיֵּֽינוּ \middot וְאֵין אֲנַֽחְנוּ יְכוֹלִים לַעֲשׂוֹת חוֹבוֹתֵֽינוּ בְּבֵית בְּחִירָתֶֽךָ בַּבַּֽיִת הַגָּדוֹל וְהַקָּדוֹשׁ שֶׁנִּקְרָא שִׁמְךָ עָלָיו מִפְּנֵי הַיָּד שְׁנִּשְׁתַּלְּחָה בְּמִקְדָּשֶֽׁךָ׃ 
יְהִי רָצוֹן מִלְּפָנֶֽיךָ יְיָ אֱלֹהֵֽינוּ וֵאלֹהֵי אֲבוֹתֵֽינוּ שֶׁתַּעֲלֵֽנוּ בְשִׂמְחָה לְאַרְצֵֽנוּ וְתִטָּעֵֽנוּ בִּגְבוּלֵֽנוּ וְשָׁם נַעֲשֶׂה לְפָנֶֽיךָ אֶת־קׇרְבְּנוֹת חוֹבוֹתֵֽינוּ תְּמִידִים כְּסִדְרָם וּמוּסָפִים כְּהִלְכָתָם׃

\firstword{וְאֶת מוּסְפֵי}
יוֹם הַשַּׁבָּת הַזֶּה וְיוֹם רֹאשׁ הַחֹֽדֶשׁ הַזֶּה נַעֲשֶׂה וְנַקְרִיב לְפָנֶֽיךָ בְּאַהֲבָה כְּמִצְוַת רְצוֹנֶֽךָ כְּמוֹ שֶׁכָּתַֽבְתָּ עָלֵֽינוּ בְּתוֹרָתֶֽךָ עַל יְדֵי מֹשֶׁה עַבְדְּךָ מִפִּי כְבוֹדֶֽךָ כָּאָמוּר׃ \\
\firstword{וּבְיוֹם֙ הַשַּׁבָּ֔ת}\source{במדבר כח}
שְׁנֵֽי־כְבָשִׂ֥ים בְּנֵֽי־שָׁנָ֖ה תְּמִימִ֑ם וּשְׁנֵ֣י עֶשְׂרֹנִ֗ים סֹ֧לֶת מִנְחָ֛ה בְּלוּלָ֥ה בַשֶּׁ֖מֶן וְנִסְכּֽוֹ׃ עֹלַ֥ת שַׁבַּ֖ת בְּשַׁבַּתּ֑וֹ עַל־עֹלַ֥ת הַתָּמִ֖יד וְנִסְכָּֽהּ׃ \\
%זֶה קָרְבַּן שַׁבָּת וְקָרְבַּן רֹאשׁ חֹֽדֶשׁ כָּאָמוּר׃ \\
וּבְרָאשֵׁי֙ \source{במדבר כח} חׇדְשֵׁיכֶ֔ם תַּקְרִ֥יבוּ עֹלָ֖ה לַייָ֑ פָּרִ֨ים בְּנֵֽי־בָקָ֤ר שְׁנַ֙יִם֙ וְאַ֣יִל אֶחָ֔ד כְּבָשִׂ֧ים בְּנֵי־שָׁנָ֛ה שִׁבְעָ֖ה תְּמִימִֽם׃ וּמִנְחָתָם וְנִסְכֵּיהֶם כִּמְדֻבָּר שְׁלֹשָׁה עֶשְׂרֹנִים לַפָּר וּשְׁנֵי עֶשְׂרֹנִים לָאָֽיִל וְעִשָּׂרוֹן לַכֶּֽבֶשׂ וְיַֽיִן כְּנִסְכּוֹ וְשָׂעִיר לְכַפֵּר וּשְׁנֵי תְמִידִים כְּהִלְכָתָם׃

\textbf{יִשְׂמְחוּ בְמַלְכוּתְךָ}
שׁוֹמְרֵי שַׁבָּת וְקֽוֹרְאֵי עֹֽנֶג \middot עַם מְקַדְּשֵׁי שְׁבִיעִי כֻּלָּם יִשְׂבְּעוּ וְיִתְעַנְּגוּ מִטּוּבֶֽךָ \middot וּבַשְּׁבִיעִי רָצִֽיתָ בּוֹ וְקִדַּשְׁתּוֹ \middot חֶמְדַּת יָמִים אוֹתוֹ קָרָֽאתָ זֵֽכֶר לְמַעֲשֵׂה בְרֵאשִׁית׃

\firstword{אֱלֹהֵֽינוּ} וֵאלֹהֵי אֲבוֹתֵֽינוּ רְצֵה בִמְנוּחָתֵֽנוּ וְחַדֵּשׁ עָלֵֽינוּ בְּיוֹם הַשַּׁבָּת הַזֶּה אֶת־הַחֹֽדֶשׁ הַזֶּה לְטוֹבָה וְלִבְרָכָה לְשָׂשׂוֹן וּלְשִׂמְחָה לִישׁוּעָה וּלְנֶחָמָה לְפַרְנָסָה וּלְכַלְכָּלָה לְחַיִּים וּלְשָׁלוֹם לִמְחִֽילַת חֵטְא וְלִסְלִיחַת עָוֹן 
(\instruction{בשנה מעוברת עד בכלל ר״ח אדר ב׳׃}
וּלְכַפָּרַת פָּֽשַׁע): כִּי בְעַמְּךָ יִשְׂרָאֵל בָּחַֽרְתָּ מִכׇּל־הָאֻמּוֹת וְשַׁבַּת קׇדְשְׁךָ לָהֶם הוֹדָֽעְתָּ וְחֻקֵּי רָאשֵׁי חֳדָשִׁים לָהֶם קָבָֽעְתָּ: בָּרוּךְ אַתָּה יְיָ מְקַדֵּשׁ הַשַּׁבָּת וְיִשְׂרָאֵל וְרָאשֵׁי חֳדָשִׁים: 
%ֺ\instruction{רצה...}

\end{sometimes}

\firstword{רְצֵה}
יְיָ אֱלֹהֵֽינוּ בְּעַמְּךָ יִשְׂרָאֵל וּבִתְפִלָּתָם וְהָשֵׁב הָעֲבוֹדָה לִדְבִיר בֵּיתֶֽךָ׃ וְאִשֵּׁי יִשְׂרָאֵל וּתְפִלָּתָם בְּאַהֲבָה תְקַבֵּל בְּרָצוֹן וּתְהִי לְרָצוֹן תָּמִיד עֲבוֹדַת יִשְׂרָאֵל עַמֶּֽךָ׃ וְתֶחֱזֶֽינָה עֵינֵֽינוּ בְּשׁוּבְךָ לְצִיּוֹן בְּרַחֲמִים׃
בָּרוּךְ אַתָּה יְיָ הַמַּחֲזִיר שְׁכִינָתוֹ לְצִיּוֹן׃

\modim

\shabboschanukah

\shabboshodos

\shatzbirkaskohanim{בחזרת הש״ץ׃}

\shabbossimshalom

\tachanunim

\fullkaddish

\label{einkelokeinu}

\conclusionshabYT

\section[שיר הכבוד]{\adforn{48} שיר הכבוד \adforn{22}}
%\instruction{בשבת אומרים שיר הכבוד}
\nopagebreak
\begin{footnotesize}
\begin{longtable}{l p{.9\textwidth}}

\shatz &
אַנְעִים זְמִירוֹת וְשִׁירִים אֶאֱרֹג \middot כִּי אֵלֶיךָ נַפְשִׁי תַעֲרֹג׃ \\

\kahal &
נַפְשִׁי חָמְדָה בְּצֵל יָדֶֽךָ \middot לָדַֽעַת כׇּל־רָז סוֹדֶֽךָ׃ \\

\shatz &
מִדֵּי דַבְּרִי בִּכְבוֹדֶֽךָ \middot הוֹמֶה לִבִּי אֶל דּוֹדֶֽיךָ׃ \\

\kahal &
עַל כֵּן אֲדַבֵּר בְּךָ נִכְבָּדוֹת \middot וְשִׁמְךָ אֲכַבֵּד בְּשִׁירֵי יְדִידוֹת׃ \\

\shatz &
\acrostic{אֲ}סַפְּרָה כְבוֹדְךָ וְלֹא רְאִיתִֽיךָ \middot אֲדַמְּךָ אֲכַנְּךָ וְלֹא יְדַעְתִּֽיךָ׃ \\

\kahal &
\acrostic{בְּ}יַד נְבִיאֶֽיךָ בְּסוֹד עֲבָדֶֽיךָ \middot דִּמִּֽיתָ הֲדַר כְּבוֹד הוֹדֶֽךָ׃ \\

\shatz &
\acrostic{גְּ}דֻלָּתְךָ וּגְבוּרָתֶֽךָ \middot כִּנּוּ לְתֹקֶף פְּעֻלָּתֶֽךָ׃ \\

\kahal &
\acrostic{דִּ}מּוּ אוֹתְךָ וְלֹא כְּפִי יֶשְׁךָ \middot וַיְשַׁוּוּךָ לְפִי מַעֲשֶֽׂיךָ׃ \\

\shatz &
\acrostic{הִ}מְשִׁילֽוּךָ בְּרֹב חֶזְיוֹנוֹת \middot הִנְּךָ אֶחָד בְּכׇל־דִּמְיוֹנוֹת׃ \\

\kahal &
\acrostic{וַ}יֶּחֱזוּ בְךָ זִקְנָה וּבַחֲרוּת \middot וּשְׂעַר רֹאשְׁךָ בְּשֵׂיבָה וְשַׁחֲרוּת׃ \\

\shatz &
\acrostic{זִ}קְנָה בְּיוֹם דִּין וּבַחֲרוּת בְּיוֹם קְרָב \middot כְּאִישׁ מִלְחָמוֹת יָדָיו לוֹ רָב׃ \\

\kahal &
\acrostic{חָ}בַשׁ כּֽוֹבַע יְשׁוּעָה בְּרֹאשׁוֹ \middot הוֹשִֽׁיעָה לּוֹ יְמִינוֹ וּזְרוֹעַ קׇדְשׁוֹ׃ \\

\shatz &
\acrostic{טַ}לְלֵי אוֹרוֹת רֹאשׁוֹ נִמְלָא \middot קְוֻצּוֹתָיו רְסִֽיסֵי לָיְלָה׃ \\

\kahal &
\acrostic{יִ}תְפָּאֵר בִּי כִּי חָֽפֵץ בִּי \middot וְהוּא יִהְיֶה לִּי לַעֲטֶֽרֶת צְבִי׃ \\

\shatz &
\acrostic{כֶּֽ}תֶם טָהוֹר פָּז דְּמוּת רֹאשׁוֹ \middot וְחַק עַל מֵֽצַח כְּבוֹד שֵׁם קׇדְשׁוֹ׃ \\

\kahal &
\acrostic{לְ}חֵן וּלְכָבוֹד צְבִי תִפְאָרָה \middot אֻמָּתוֹ לוֹ עִטְּרָה עֲטָרָה׃ \\

\shatz &
\acrostic{מַ}חְלְפוֹת רֹאשׁוֹ כְּבִימֵי בְחוּרוֹת \middot קְוֻצּוֹתָיו תַּלְתַּלִּים שְׁחוֹרוֹת׃ \\

\kahal &
\acrostic{נְ}וֵה הַצֶּֽדֶק צְבִי תִפְאַרְתּוֹ \middot יַעֲלֶה נָּא עַל רֹאשׁ שִׂמְחָתוֹ׃ \\

\shatz &
\acrostic{סְ}גֻלָּתוֹ תְּהִי בְיָדוֹ עֲטֶֽרֶת \middot וּצְנִיף מְלוּכָה צְבִי תִפְאֶֽרֶת׃ \\

\kahal &
\acrostic{עֲ}מוּסִים נְשָׂאָם עֲטֶֽרֶת עִנְּדָם \middot מֵאֲשֶׁר יָקְרוּ בְעֵינָיו כִּבְּדָם׃ \\

\shatz &
\acrostic{פְּ}אֵרוֹ עָלַי וּפְאֵרִי עָלָיו \middot וְקָרוֹב אֵלַי בְּקׇרְאִי אֵלָיו׃ \\

\kahal &
\acrostic{צַ}ח וְאָדֹם לִלְבוּשׁוֹ אָדֹם \middot פּוּרָה בְּדָרְכוֹ בְּבוֹאוֹ מֵאֱדוֹם׃ \\

\shatz &
\acrostic{קֶֽ}שֶׁר תְּפִלִּין הֶרְאָה לֶעָנָו \middot תְּמוּנַת יְיָ לְנֶֽגֶד עֵינָיו׃ \\

\kahal &
\acrostic{ר}וֹצֶה בְעַמּוֹ עֲנָוִים יְפָאֵר \middot יוֹשֵׁב תְּהִלּוֹת בָּם לְהִתְפָּאֵר׃ \\

\shatz &
\acrostic{רֹ}אשׁ דְּבָרְךָ אֱמֶת קוֹרֵא מֵרֹאשׁ \middot דּוֹר וָדוֹר עַם דּוֹרֶשְׁךָ דְּרֹשׁ׃ \\

\kahal &
\acrostic{שִׁ}ית הֲמוֹן שִׁירַי־נָא עָלֶֽיךָ \middot וְרִנָּתִי תִקְרַב אֵלֶֽיךָ׃ \\

\shatz &
\acrostic{תְּ}הִלָּתִי תְּהִי לְרֹאשְׁךָ עֲטֶֽרֶת \middot וּתְפִלָּתִי תִּכּוֹן קְטֹֽרֶת׃ \\

\kahal &
\acrostic{תִּ}יקַר שִׁירַת רָשׁ בְּעֵינֶֽיךָ \middot כַּשִּׁיר יוּשַׁר עַל קׇרְבָּנֶֽיךָ׃ \\

\shatz &
בִּרְכָתִי תַּעֲלֶה לְרֹאשׁ מַשְׁבִּיר \middot מְחוֹלֵל וּמוֹלִיד צַדִּיק כַּבִּיר׃ \\

\kahal &
וּבְבִרְכָתִי תְּנַעֲנַע לִי רֹאשׁ \middot וְאוֹתָהּ קַח לְךָ כִּבְשָׂמִים רֹאשׁ׃ \\

\shatz &
יֶעֱרַב־נָא שִׂיחִי עָלֶֽיךָ \middot כִּי נַפְשִׁי תַעֲרֹג אֵלֶֽיךָ׃ \\

\end{longtable}

לְךָ֣\source{דה״א כט} יְ֠יָ֠ הַגְּדֻלָּ֨ה וְהַגְּבוּרָ֤ה וְהַתִּפְאֶ֙רֶת֙ וְהַנֵּ֣צַח וְהַה֔וֹד כִּי־כֹ֖ל בַּשָּׁמַ֣יִם וּבָאָ֑רֶץ לְךָ֤ יְיָ֙ הַמַּמְלָכָ֔ה וְהַמִּתְנַשֵּׂ֖א לְכֹ֥ל ׀ לְרֹֽאשׁ׃
מִ֗י\source{תהלים קו} יְ֭מַלֵּל גְּבוּר֣וֹת יְיָ֑ יַ֝שְׁמִ֗יעַ כׇּל־תְּהִלָּתֽוֹ׃
\end{footnotesize}




%

\mournerskaddish
\adonolam

\chapter[קידושא רבא לשבת]{\adforn{47} קידושא רבא \adforn{19}}

\begin{footnotesize}
	אִם־תָּשִׁ֤יב\source{ישעיה נח} מִשַּׁבָּת֙ רַגְלֶ֔ךָ עֲשׂ֥וֹת חֲפָצֶ֖ךָ בְּי֣וֹם קׇדְשִׁ֑י וְקָרָ֨אתָ לַשַּׁבָּ֜ת עֹ֗נֶג לִקְד֤וֹשׁ יְיָ֙ מְכֻבָּ֔ד וְכִבַּדְתּוֹ֙ מֵעֲשׂ֣וֹת דְּרָכֶ֔יךָ מִמְּצ֥וֹא חֶפְצְךָ֖ וְדַבֵּ֥ר דָּבָֽר׃ אָ֗ז תִּתְעַנַּג֙ עַל־יְיָ֔ וְהִרְכַּבְתִּ֖יךָ עַל־בָּ֣מֳתֵי במותי אָ֑רֶץ וְהַאֲכַלְתִּ֗יךָ נַחֲלַת֙ יַֽעֲקֹ֣ב אָבִ֔יךָ כִּ֛י פִּ֥י יְיָ֖ דִּבֵּֽר׃
	
	וְשָֽׁמְר֥וּ\source{שמות לא} בְנֵֽי־יִשְׂרָאֵ֖ל אֶת־הַשַּׁבָּ֑ת לַעֲשׂ֧וֹת אֶת־הַשַּׁבָּ֛ת לְדֹרֹתָ֖ם בְּרִ֥ית עוֹלָֽם׃ בֵּינִ֗י וּבֵין֙ בְּנֵ֣י יִשְׂרָאֵ֔ל א֥וֹת הִ֖וא לְעֹלָ֑ם כִּי־שֵׁ֣שֶׁת יָמִ֗ים עָשָׂ֤ה יְיָ֙ אֶת־הַשָּׁמַ֣יִם וְאֶת־הָאָ֔רֶץ וּבַיּוֹם֙ הַשְּׁבִיעִ֔י שָׁבַ֖ת וַיִּנָּפַֽשׁ׃
	
	זָכ֛וֹר\source{שמות כ} אֶת־י֥וֹם הַשַּׁבָּ֖ת לְקַדְּשֽׁוֹ׃ שֵׁ֤שֶׁת יָמִים֙ תַּֽעֲבֹ֔ד וְעָשִׂ֖יתָ כׇּל־מְלַאכְתֶּֽךָ׃ וְיוֹם֙ הַשְּׁבִיעִ֔י שַׁבָּ֖ת לַייָ֣ אֱלֹהֶ֑יךָ לֹֽא־תַעֲשֶׂ֨ה כׇל־מְלָאכָ֜ה אַתָּ֣ה ׀ וּבִנְךָ֣ וּבִתֶּ֗ךָ עַבְדְּךָ֤ וַאֲמָֽתְךָ֙ וּבְהֶמְתֶּ֔ךָ וְגֵרְךָ֖ אֲשֶׁ֥ר בִּשְׁעָרֶֽיךָ׃ כִּ֣י שֵֽׁשֶׁת־יָמִים֩ עָשָׂ֨ה יְיָ֜ אֶת־הַשָּׁמַ֣יִם וְאֶת־הָאָ֗רֶץ אֶת־הַיָּם֙ וְאֶת־כׇּל־אֲשֶׁר־בָּ֔ם וַיָּ֖נַח בַּיּ֣וֹם הַשְּׁבִיעִ֑י\end{footnotesize} 

עַל־כֵּ֗ן בֵּרַ֧ךְ יְיָ֛ אֶת־י֥וֹם הַשַּׁבָּ֖ת וַֽיְקַדְּשֵֽׁהוּ׃

\savri
\firstword{בָּרוּךְ}
אַתָּה יְיָ אֱלֹהֵֽינוּ מֶֽלֶךְ הָעוֹלָם בּוֹרֵא פְּרִי הַגָֽפֶן׃\\

\englishinst{When visiting the sick on Shabbat:}
שַׁבָּת הִיא מִלִּזְעוֹק וּרְפוּאָה קְרוֹבָה לָבוֹא וְשִבְתוֹ בַּשָׁלוֹם׃

\englishinst{When visiting a mourner on Shabbat:}
שַׁבָּת הִיא מִלְּנַחֵם וְנֶחָמָה קְרוֹבָה לָבוֹא וְשִבְתוֹ בַּשָׁלוֹם׃
\endgroup

\clearpage

\begingroup
\let\clearpage\relax
\vspace{0.2in}
\ifboolexpr{togl {includeshabbat} and togl {includefestival}}{\chapter[מנחה לשבת ויו״ט]{\adforn{47} מנחה לשבת ויו״ט \adforn{19}}}{
	\ifboolexpr{togl {includeshabbat}}{\chapter[מנחה לשבת]{\adforn{47} מנחה לשבת \adforn{19}}}{}
	\ifboolexpr{togl {includefestival}}{\chapter[מנחה ליו״ט]{\adforn{47} מנחה ליו״ט \adforn{19}}}{}}
	
\vspace{0.5in}
\ashrei

\uvaletzion

\halfkaddish

\ifboolexpr{togl {includeshabbat}}{
\ifboolexpr{togl {includefestival}}{\instruction{ביום טוב שאין בשבת אומרים תפילת יום טוב עמ׳ \pageref{shabYTamidah}}}{}

\textbf{
וַאֲנִ֤י תְפִלָּֽתִי־לְךָ֨ ׀ יְיָ֡ עֵ֤ת רָצ֗וֹן אֱלֹהִ֥ים בְּרׇב־חַסְדֶּ֑ךָ עֲ֝נֵ֗נִי בֶּאֱמֶ֥ת יִשְׁעֶֽךָ׃ } \source{תהלים סט}

\section[סדר קריאת התורה]{\adforn{53} סדר קריאת התורה \adforn{25}}


\pesicha

\brikhshmei

\gadlu

\avharachamim

\vesigale

ֺ%\instruction{קריאת התורה בעמוד \pageref{torah}}

\torahbarachu

%\hagomel

\hagbaha

\englishinst{Some chant the following Psalm while the Torah is being wrapped:}

\instruction{יש אומרים בזמן גלילת התורה׃}

\begin{footnotesize}
	\mizmorshabbat
\end{footnotesize}

\yehalelu

\instruction{כשחוזרים הספר תורה לארון:}\\
\kafdalet

\nextpage
\etzchaim

\instruction{סוגרים הארון}}{}
\label{shabYTamidah}
\halfkaddish

\section[תפילת העמידה]{\adforn{53} תפילת העמידה \adforn{25}}


\amidaopening{\shabbosshuva}{\englishinst{During the repetition of the Amidah, Kedusha is said here}}

\weekdaysakedusha

\sepline

\ifboolexpr{togl {includeshabbat}}{
	\ifboolexpr{togl {includefestival}}{\instruction{ביו״ט ממשיכים בעמ׳ \pageref{ytmincha}}}{}

\firstword{אַתָּה}
אֶחָד וְשִׁמְךָ אֶחָד \source{דה״א יז}וּמִי֙ כְּעַמְּךָ֣ יִשְׂרָאֵ֔ל גּ֥וֹי אֶחָ֖ד בָּאָ֑רֶץ תִּפְאֶֽרֶת גְּדֻלָּה וַעֲטֶֽרֶת יְשׁוּעָה, יוֹם מְנוּחָה וּקְדֻשָּׁה לְעַמְּךָ נָתַֽתָּ, אַבְרָהָם יָגֵל יִצְחָק יְרַנֵּן יַעֲקֹב וּבָנָיו יָנֽוּחוּ בוֹ מְנוּחַת אַהֲבָה וּנְדָבָה מְנוּחַת אֱמֶת וֶאֱמוּנָה מְנוּחַת שָׁלוֹם וְשַׁלְוָה וְהַשְׁקֵט וָבֶֽטַח מְנוּחָה שְׁלֵמָה שָׁאַתָּה רוֹצֶה בָּהּ יַכִּֽירוּ בָנֶֽיךָ וְיֵדְעוּ כִּי מֵאִתְּךָ הִיא מְנוּחָתָם וְעַל מְנוּחָתָם יַקְדִּֽישׁוּ אֶת־שְׁמֶֽךָ׃

%\shabboskiddushhayom{\footnote{\instruction{נ״א:} (שַׁבּתוֹת קׇדְשֶׁךָ) וְיָנוּחוּ בָם}} \instruction{רצה וכו׳}
\shabboskiddushhayom{} \ifboolexpr{togl {includefestival}}{\instruction{רצה וכו׳}}{}


\sepline}{}

\ifboolexpr{togl {includefestival}}{\label{ytmincha}
\ytkiddushhayom{}

\ifboolexpr{togl {includeshabbat}}{\sepline}{}
}{}

\retzeh

\yaalehveyavo

\zion

\modim

\ifboolexpr{togl {includeshabbat}}{\shabboschanukah

\shabboshodos

\shabbossimshalom

\tachanunim

\instruction{בימי שאין אומרים תחנון בחול אין אומרים צו״צ בשבת.}\\
צִדְקָתְךָ֣ \source{תהלים קיט}צֶ֣דֶק לְעוֹלָ֑ם וְֽתוֹרָתְךָ֥ אֱמֶֽת׃ \source{תהלים עא}וְצִדְקָתְךָ֥ אֱלֹהִ֗ים עַד־מָ֫ר֥וֹם אֲשֶׁר־עָשִׂ֥יתָ גְדֹל֑וֹת אֱ֝לֹהִ֗ים מִ֣י כָמֽוֹךָ׃ \source{תהלים לו}צִדְקָתְךָ֨ ׀ כְּֽהַרְרֵי־אֵ֗ל מִ֭שְׁפָּטֶיךָ תְּה֣וֹם רַבָּ֑ה אָ֤דָֽם וּבְהֵמָ֖ה תוֹשִׁ֣יעַ יְיָ׃
}{
\simshalomplain

\tachanunim
}

\fullkaddish

\aleinu
\vspace{0.25in}
\chapter[ערבית לחול]{\adforn{47} ערבית לחול \adforn{19}}

\textbf{וְה֤וּא רַח֨וּם}\source{תהלים עט}
׀ יְכַפֵּ֥ר עָוֺן֮ וְֽלֹא־יַֽ֫שְׁחִ֥ית וְ֭הִרְבָּה לְהָשִׁ֣יב אַפּ֑וֹ וְלֹא־יָ֝עִ֗יר כׇּל־חֲמָתֽוֹ׃
יְיָ֥\source{תהלים כ} הוֹשִׁ֑יעָה הַ֝מֶּ֗לֶךְ יַעֲנֵ֥נוּ בְיוֹם־קׇרְאֵֽנוּ׃


\barachu


\hamaarivaravim

\ahavasolam

\shema

\veahavta

\vehaya

\vayomer{}

\emesveemuna

\hashkiveinu{בָּרוּךְ אַתָּה יְיָ שׁוֹמֵר עַמּוֹ יִשְׂרָאֵל לָעַד׃}

\boruchhashemleolam

\halfkaddish

\section[תפילת העמידה]{\adforn{53} תפילת העמידה \adforn{25}}


\amidaopening{\ayt}{}

\firstword{אַתָּה חוֹנֵן}
לְאָדָם דַּֽעַת וּמְלַמֵּד לֶאֱנוֹשׁ בִּינָה.

\begin{sometimes}

\instruction{במוצאי שבת:}\\
אַתָּה חוֹנַנְתָּֽנוּ לְמַדַּע תּוֹרָתֶֽךָ וַתְּלַמְּדֵֽנוּ לַעֲשׂוֹת חֻקֵּי רְצוֹנֶֽךָ וַתַּבְדִּילֵֽנוּ יְיָ אֱלֹהֵֽינוּ בֵּין קֹֽדֶשׁ לְחוֹל בֵּין אוֹר לְחֹֽשֶׁךְ בֵּין יִשְׂרָאֵל לָעַמִּים בֵּין יוֹם הַשְּׁבִיעִי לְשֵֽׁשֶׁת יְמֵי הַמַּעֲשֶׂה׃ אָבִֽינוּ מַלְכֵּֽנוּ הָחֵל עָלֵֽינוּ הַיָּמִים הַבָּאִים לִקְרָאתֵֽנוּ לְשָׁלוֹם חֲשׂוּכִים מִכׇּל־חֵטְא וּמְנֻקִּים מִכׇּל־עָוֹן וּמְדֻבָּקִים בְּיִרְאָתֶֽךָ׃ וְ...

\end{sometimes}

חׇנֵּֽנוּ מֵאִתְּךָ בִּינָה דֵּעָה וְהַשְׂכֵּל׃ בָּרוּךְ אַתָּה יְיָ חוֹנֵן הַדָּֽעַת׃

\weekdaysateshuva

\weekdaysaselichah

\weekdaysageulah

\weekdaysarefuah

\weekdaysaberacha

\weekdaysashofar

\weekdaysamishpat

\weekdaysaminim

\weekdaysatzadikim

\weekdaysayerushelayim

\weekdaysamalchus

\weekdaysashemakoleinu

\retzeh

\yaalehveyavo

\zion

\maarivmodim

\alhanisim

\weekdaysahodos

\firstword{שָׁלוֹם}
רָב עַל יִשְׂרָאֵל עַמְּךָ תָּשִׂים לְעוֹלָם כִּי אַתָּה הוּא מֶֽלֶךְ אָדוֹן לְכׇל־הַשָּׁלוֹם׃ וְטוֹב בְּעֵינֶֽיךָ לְבָרֵךְ אֶת־עַמְּךָ יִשְׂרָאֵל בְּכׇל־עֵת וּבְכׇל־שָׁעָה בִּשְׁלוֹמֶךָ׃
\vspace{-0.4\baselineskip}
\columnratio{0.7}
\begin{paracol}{2}

\instruction{בעשי״ת:}
\begin{small}
בְּסֵֽפֶר חַיִּים בְּרָכָה וְשָׁלוֹם וּפַרְנָסָה טוֹבָה נִזָּכֵר וְנִכָּתֵב לְפָנֶֽיךָ אָֽנוּ וְכׇל־עַמְּךָ בֵּית יִשְׂרָאֵל לְחַיִּים וּלְשָׁלוֹם׃ בָּרוּךְ אַתָּה יְיָ עוֹשֵׂה הַשָּׁלוֹם׃

\end{small}
\switchcolumn
בָּרוּךְ אַתָּה יְיָ הַמְבָרֵךְ אֶת־עַמּוֹ יִשְׂרָאֵל בַּשָּׁלוֹם׃

\end{paracol}



\tachanunim

\vspace{\baselineskip}

\begin{sometimes}

\instruction{במוצ״ש אם אין בשבוע הבא יו״ט:}

\halfkaddish


\label{vihi noam}

\firstword{וִיהִ֤י ׀ נֹ֤עַם}\source{תהלים צ}
אֲדֹנָ֥י אֱלֹהֵ֗ינוּ עָ֫לֵ֥ינוּ וּמַעֲשֵׂ֣ה יָ֭דֵינוּ כּוֹנְנָ֥ה עָלֵ֑ינוּ וּֽמַעֲשֵׂ֥ה יָ֝דֵ֗ינוּ כּוֹנְנֵֽהוּ׃\\
%\tzadialeph

\label{v ata kadosh}
\kedushadesidra

\end{sometimes}

\fullkaddish

\vfill

\instruction{בחנוכה מדליקים את המנורה עמ׳ \pageref{chanukah}}\\
\instruction{בפורים קוראים את מגילת אסתר עמ׳ \pageref{purim}}\\

\aleinu

\ledavid

\mournerskaddish

\vfill


\instruction{סופרים כאן העומר עמ׳ \pageref{sefiras haomer}} \\
\instruction{מדליקים את המנורה לחנוכה כאן עמ׳ \pageref{chanukah}}

\endgroup

\section[ספירת העמר]{\adforn{53} ספירת העמר \adforn{25}}
\newcommand{\omerend}{בָּעֹֽמֶר}

\label{sefiras haomer}

%הִנְנִי מְקַיֵּם מִצְוַת עֲשֵה שֶׁל סְפִירַת הָעֽׂמֶר כְּמוׂ שֶׁכָּתוּב בַּתּוׂרָה׃ וּסְפַרְתֶּ֤ם \source{ויקרא כב}לָכֶם֙ מִמׇּחֳרַ֣ת הַשַּׁבָּ֔ת מִיּוֹם֙ הֲבִ֣יאֲכֶ֔ם אֶת־עֹ֖מֶר הַתְּנוּפָ֑ה שֶׁ֥בַע שַׁבָּת֖וֹת תְּמִימֹ֥ת תִּהְיֶֽינָה׃ עַ֣ד מִֽמׇּחֳרַ֤ת הַשַּׁבָּת֙ הַשְּׁבִיעִ֔ת תִּסְפְּר֖וּ חֲמִשִּׁ֣ים י֑וֹם׃

בָּרוּךְ אַתָּה יְיָ אֱלֹהֵֽינוּ מֶֽלֶךְ הָעוֹלָם אֲשֶׁר קִדְּשָֽׁנוּ בְּמִצְוֹתָיו וְצִוָּֽנוּ עַל סְפִירַת הָעֹֽמֶר׃\\

\begin{scriptsize}
	\begin{longtable}{ l | r | p{.75\textwidth} }
		1 & ט״ז ניסן & הַיּוֹם יוֹם אֶחָד \omerend׃ \\
		2 & י״ז ניסן & הַיּוֹם שְׁנֵי יָמִים \omerend׃ \\
		3 & י״ח ניסן & הַיּוֹם שְׁלֹשָׁה יָמִים \omerend׃ \\
		4 & י״ט ניסן & הַיּוֹם אַרְבָּעָה יָמִים \omerend׃ \\
		5 & כ׳ ניסן & הַיּוֹם חֲמִשָּׁה יָמִים \omerend׃ \\
		6 & כ״א ניסן & הַיוֹם שִׁשָּׁה יָמִים \omerend׃ \\
		7 & כ״ב ניסן & הַיּוֹם שִׁבְעָה יָמִים שֶׁהֵם שָׁבֽוּעַ אֶחָד \omerend׃ \\
		8 & כ״ג ניסן & הַיּוֹם שְׁמוֹנָה יָמִים שֶׁהֵם שָׁבֽוּעַ אֶחָד וְיוֹם אֶחָד \omerend׃ \\
		9 & כ״ד ניסן & הַיּוֹם תִּשְׁעָה יָמִים שֶׁהֵם שָׁבֽוּעַ אֶחָד וּשְׁנֵי יָמִים \omerend׃ \\
		10 & כ״ה ניסן & הַיּוֹם עֲשָׂרָה יָמִים שֶׁהֵם שָׁבֽוּעַ אֶחָד וּשְׁלֹשָׁה יָמִים \omerend׃ \\
		11 & כ״ו ניסן & הַיּוֹם אַחַד עָשָׂר יוֹם שֶׁהֵם שָׁבֽוּעַ אֶחָד וְאַרְבָּעָה יָמִים \omerend׃ \\
		12 & כ״ז ניסן & הַיּוֹם שְׁנֵים עָשָׂר יוֹם שֶׁהֵם שָׁבֽוּעַ אֶחָד וַחֲמִשָּׁה יָמִים \omerend׃ \\
		13 & כ״ח ניסן & הַיּוֹם שְׁלֹשָׁה עָשָׂר יוֹם שֶׁהֵם שָׁבֽוּעַ אֶחָד וְשִׁשָּׁה יָמִים \omerend׃ \\
		14 & כ״ט ניסן & הַיּוֹם אַרְבָּעָה עָשָׂר יוֹם שֶׁהֵם שְׁנֵי שָׁבוּעוֹת \omerend׃ \\
		15 & ל׳ ניסן & הַיּוֹם חֲמִשָּׁה עָשָׂר יוֹם שֶׁהֵם שְׁנֵי שָׁבוּעוֹת וְיוֹם אֶחָד \omerend׃ \\
		16 & א׳ אייר & הַיּוֹם שִׁשָּׁה עָשָׂר יוֹם שֶׁהֵם שְׁנֵי שָׁבוּעוֹת וּשְׁנֵי יָמִים \omerend׃ \\
		17 & ב׳ אייר & הַיּוֹם שִׁבְעָה עָשָׂר יוֹם שֶׁהֵם שְׁנֵי שָׁבוּעוֹת וּשְׁלֹשָׁה יָמִים \omerend׃ \\
		18 & ג׳ אייר & הַיּוֹם שְׁמוֹנָה עָשָׂר יוֹם שֶׁהֵם שְׁנֵי שָׁבוּעוֹת וְאַרְבָּעָה יָמִים \omerend׃ \\
		19 & ד׳ אייר & הַיּוֹם תִּשְׁעָה עָשָׂר יוֹם שֶׁהֵם שְׁנֵי שָׁבוּעוֹת וַחֲמִשָּׁה יָמִים \omerend׃ \\
		20 & ה׳ אייר & הַיּוֹם עֶשְׂרִים יוֹם שֶׁהֵם שְׁנֵי שָׁבוּעוֹת וְשִׁשָּׁה יָמִים \omerend׃ \\
		21 & ו׳ אייר & הַיּוֹם אֶחָד וְעֶשְׂרִים יוֹם שֶׁהֵם שְׁלֹשָׁה שָׁבוּעוֹת \omerend׃ \\
		22 & ז׳ אייר & הַיּוֹם שְׁנֵים וְעֶשְׂרִים יוֹם שֶׁהֵם שְׁלֹשָׁה שָׁבוּעוֹת וְיוֹם אֶחָד \omerend׃ \\
		23 & ח׳ אייר & הַיּוֹם שְׁלֹשָׁה וְעֶשְׂרִים יוֹם שֶׁהֵם שְׁלֹשָׁה שָׁבוּעוֹת וּשְׁנֵי יָמִים \omerend׃ \\
		24 & ט׳ אייר & הַיּוֹם אַרְבָּעָה וְעֶשְׂרִים יוֹם שֶׁהֵם שְׁלֹשָׁה שָׁבוּעוֹת וּשְׁלֹשָׁה יָמִים \omerend׃ \\
		25 & י׳ אייר & הַיּוֹם חֲמִשָּׁה וְעֶשְׂרִים יוֹם שֶׁהֵם שְׁלֹשָׁה שָׁבוּעוֹת וְאַרְבָּעָה יָמִים \omerend׃ \\
		26 & י״א אייר & הַיּוֹם שִׁשָּׁה וְעֶשְׂרִים יוֹם שֶׁהֵם שְׁלֹשָׁה שָׁבוּעוֹת וַחֲמִשָּׁה יָמִים \omerend׃ \\
		27 & י״ב אייר & הַיּוֹם שִׁבְעָה וְעֶשְׂרִים יוֹם שֶׁהֵם שְׁלֹשָׁה שָׁבוּעוֹת וְשִׁשָּׁה יָמִים \omerend׃ \\
		28 & י״ג אייר & הַיּוֹם שְׁמוֹנָה וְעֶשְׂרִים יוֹם שֶׁהֵם אַרְבָּעָה שָׁבוּעוֹת \omerend׃ \\
		29 & י״ד אייר & הַיּוֹם תִּשְׁעָה וְעֶשְׂרִים יוֹם שֶׁהֵם אַרְבָּעָה שָׁבוּעוֹת וְיוֹם אֶחָד \omerend׃ \\
		30 & ט״ו אייר & הַיּוֹם שְׁלֹשִׁים יוֹם שֶׁהֵם אַרְבָּעָה שָׁבוּעוֹת וּשְׁנֵי יָמִים \omerend׃ \\
		31 & ט״ז אייר & הַיּוֹם אֶחָד וּשְׁלֹשִׁים יוֹם שֶׁהֵם אַרְבָּעָה שָׁבוּעוֹת וּשְׁלֹשָׁה יָמִים \omerend׃ \\
		32 & י״ז אייר & הַיּוֹם שְׁנֵים וּשְׁלֹשִׁים יוֹם שֶׁהֵם אַרְבָּעָה שָׁבוּעוֹת וְאַרְבָּעָה יָמִים \omerend׃ \\
		33 & י״ח אייר & הַיּוֹם שְׁלֹשָׁה וּשְלֹשִׁים יוֹם שֶׁהֵם אַרְבָּעָה שָׁבוּעוֹת וַחֲמִשָּׁה יָמִים \omerend׃ \\
		34 & י״ט אייר & הַיּוֹם אַרְבָּעָה וּשְׁלֹשִׁים יוֹם שֶׁהֵם אַרְבָּעָה שָׁבוּעוֹת וְשִׁשָּׁה יָמִים \omerend׃ \\
		35 & כ׳ אייר & הַיּוֹם חֲמִשָּׁה וּשְׁלֹשִׁים יוֹם שֶׁהֵם חֲמִשָּׁה שָׁבוּעוֹת \omerend׃ \\
		36 & כ״א אייר & הַיּוֹם שִׁשָּׁה וּשְׁלֹשִׁים יוֹם שֶׁהֵם חֲמִשָּׁה שָׁבוּעוֹת וְיוֹם אֶחָד \omerend׃ \\
		37 & כ״ב אייר & הַיּוֹם שִׁבְעָה וּשְׁלֹשִׁים יוֹם שֶׁהֵם חֲמִשָּׁה שָׁבוּעוֹת וּשְׁנֵי יָמִים \omerend׃ \\
		38 & כ״ג אייר & הַיּוֹם שְׁמוֹנָה וּשְׁלֹשִׁים יוֹם שֶׁהֵם חֲמִשָּׁה שָׁבוּעוֹת וּשְׁלֹשָׁה יָמִים \omerend׃ \\
		39 & כ״ד אייר & הַיּוֹם תִּשְׁעָה וּשְׁלֹשִׁים יוֹם שֶׁהֵם חֲמִשָּׁה שָׁבוּעוֹת וְאַרְבָּעָה יָמִים \omerend׃ \\
		40 & כ״ה אייר & הַיּוֹם אַרְבָּעִים יוֹם שֶׁהֵם חֲמִשָּׁה שָׁבוּעוֹת וַחֲמִשָּׁה יָמִים \omerend׃ \\
		41 & כ״ו אייר & הַיּוֹם אֶחָד וְאַרְבָּעִים יוֹם שֶׁהֵם חֲמִשָּׁה שָׁבוּעוֹת וְשִׁשָּׁה יָמִים \omerend׃ \\
		42 & כ״ז אייר & הַיּוֹם שְׁנֵים וְאַרְבָּעִים יוֹם שֶׁהֵם שִׁשָּׁה שָׁבוּעוֹת \omerend׃ \\
		43 & כ״ח אייר & הַיּוֹם שְׁלֹשָׁה וְאַרְבָּעִים יוֹם שֶׁהֵם שִׁשָּׁה שָׁבוּעוֹת וְיוֹם אֶחָד \omerend׃ \\
		44 & כ״ט אייר & הַיּוֹם אַרְבָּעָה וְאַרְבָּעִים יוֹם שֶׁהֵם שִׁשָּׁה שָׁבוּעוֹת וּשְׁנֵי יָמִים \omerend׃ \\
		45 & א׳ סיון & הַיּוֹם חֲמִשָּׁה וְאַרְבָּעִים יוֹם שֶׁהֵם שִׁשָּׁה שָׁבוּעוֹת וּשְׁלֹשָׁה יָמִים \omerend׃ \\
		46 & ב׳ סיון & הַיּוֹם שִׁשָּׁה וְאַרְבָּעִים יוֹם שֶׁהֵם שִׁשָּׁה שָׁבוּעוֹת וְאַרְבָּעָה יָמִים \omerend׃ \\
		47 & ג׳ סיון & הַיּוֹם שִׁבְעָה וְאַרְבָּעִים יוֹם שֶׁהֵם שִׁשָּׁה שָבוּעוֹת וַחֲמִשָּׁה יָמִים \omerend׃ \\
		48 & ד׳ סיון & הַיּוֹם שְׁמוֹנָה וְאַרְבָּעִים יוֹם שֶׁהֵם שִׁשָּׁה שָׁבוּעוֹת וְשִׁשָּׁה יָמִים \omerend׃ \\
		49 & ה׳ סיון & הַיּוֹם תִּשְׁעָה וְאַרְבָּעִים יוֹם שֶׁהֵם שִׁבְעָה שָׁבוּעוֹת \omerend׃
	\end{longtable}
\end{scriptsize}



הָרַחֲמָן הוּא יַחֲזִיר עֲבוֹדַת בֵּית הַמִקְדָּשׁ לִמְקוֹמָהּ׃
יְהִי רָצוֹן מִלְּפָנֶֽיךָ יְיָ אֱלֹהֵֽינוּ וֵאלֹהֵי אֲבוֹתֵֽינוּ שֶׁיִבָּנֶה בֵּית הַמִּקְדָּשׁ בִּמְהֵרָה בְיָמֵֽינוּ וְתֵן חֶלְקֵֽנוּ בְּתוֹרָתֶֽךָ׃ וְשָׁם נַעֲבׇדְךָ בְּיִרְאָה כִּימֵי עוֹלָם וּכְשָׁנִים קַדְמֹנִיּוֹת׃



\chapter[ק״ש שעל המיטה]{\adforn{53} קריאת שמע שעל המיטה \adforn{25}}

רִבּוֹנוֹ שֶׁל עוֹלָם הֲרֵינִי מוֹחֵל לְכׇל־מִי שֶׁהִכְעִיס וְהִקְנִיט אוֹתִי אוֹ שֶׁחָטָא כְּנֶגְדִּי בֵּין בְּגוּפִי בֵּין בְּמָמוֹנִי בֵּין בִּכְבוֹדִי בֵּין בְּכׇל־אֲשֶׁר לִי בֵּין בְּאוֹנֶס בֵּין בְּרָצוֹן בֵּין בְּשׁוֹגֵג בֵּין בְּמֵזִיד בֵּין בְּמַחֲשָׁבָה בֵּין בְּדִבּוּר בֵּין בְּמַעֲשֶׂה. וְלֹא יֵעָנֵשׁ שׁוּם אָדָם בְּסִבָּתִי.

\englishinst{A person who read \ayin arvit after nightfall reads only the first paragraph of the Shema\ayin.  Otherwise, read all three paragraphs, and the line preceding the Shema\ayin.}

%\instruction{מי שמתפלל ערבית לפני זמן צאת הכוכבים קורא ק״ש. מי שמתפלל ערבית בזמנה קורא רק פרשה רשאונה}

\begin{footnotesize}\instruction{מי שעדיין לא קרא ק״ש בזמנו׃ }
	אֵל מֶֽלֶךְ נֶאֱמָן
\end{footnotesize}

\begin{Large}
	\textbf{
		שְׁמַ֖ע יִשְׂרָאֵ֑ל יְיָ֥ אֱלֹהֵ֖ינוּ יְיָ֥ ׀ אֶחָֽד׃} \source{דברים ו}\\
\end{Large}
\begin{large}
	\instruction{בלחש:} \textbf{בָּרוּךְ שֵׁם כְּבוֹד מַלְכוּתוֹ לְעוֹלָם וָעֶד:}
\end{large}

\veahavta

\begin{footnotesize}
	\vehaya
	
	\vayomer{}
\end{footnotesize}

\begin{small}
	
	\firstword{וִיהִ֤י ׀ נֹ֤עַם}\source{תהלים צ}
	אֲדֹנָ֥י אֱלֹהֵ֗ינוּ עָ֫לֵ֥ינוּ וּמַעֲשֵׂ֣ה יָ֭דֵינוּ כּוֹנְנָ֥ה עָלֵ֑ינוּ וּֽמַעֲשֵׂ֥ה יָ֝דֵ֗ינוּ כּוֹנְנֵֽהוּ׃\\
	\tzadialeph
	
	יְ֖יָ \source{תהלים ג}מָה־רַבּ֣וּ צָרָ֑י רַ֝בִּ֗ים קָמִ֥ים עָלָֽי׃ רַבִּים֮ אֹמְרִ֢ים לְנַ֫פְשִׁ֥י אֵ֤ין יְֽשׁוּעָ֓תָה לּ֬וֹ בֵאלֹהִ֬ים סֶֽלָה׃ וְאַתָּ֣ה יְיָ֭ מָגֵ֣ן בַּעֲדִ֑י כְּ֝בוֹדִ֗י וּמֵרִ֥ים רֹאשִֽׁי׃ ק֭וֹלִי אֶל־יְיָ֣ אֶקְרָ֑א וַיַּעֲנֵ֨נִי מֵהַ֖ר קׇדְשׁ֣וֹ סֶֽלָה׃ אֲנִ֥י שָׁכַ֗בְתִּי וָאִ֫ישָׁ֥נָה הֱקִיצ֑וֹתִי כִּ֖י יְיָ֣ יִסְמְכֵֽנִי׃ לֹֽא־אִ֭ירָא מֵרִבְב֥וֹת עָ֑ם אֲשֶׁ֥ר סָ֝בִ֗יב שָׁ֣תוּ עָלָֽי׃ ק֘וּמָ֤ה יְיָ֨ ׀ הוֹשִׁ֘יעֵ֤נִי אֱלֹהַ֗י כִּֽי־הִכִּ֣יתָ אֶת־כׇּל־אֹיְבַ֣י לֶ֑חִי שִׁנֵּ֖י רְשָׁעִ֣ים שִׁבַּֽרְתָּ׃ לַֽייָ֥ הַיְשׁוּעָ֑ה עַֽל־עַמְּךָ֖ בִרְכָתֶ֣ךָ סֶּֽלָה
	
	\hashkiveinu{}
	
	\barukhbayom
	
	\yerueinnu
	
	\source{בראשית מח}%
	הַמַּלְאָךְ֩ הַגֹּאֵ֨ל אֹתִ֜י מִכׇּל־רָ֗ע יְבָרֵךְ֮ אֶת־הַנְּעָרִים֒ וְיִקָּרֵ֤א בָהֶם֙ שְׁמִ֔י וְשֵׁ֥ם אֲבֹתַ֖י אַבְרָהָ֣ם וְיִצְחָ֑ק וְיִדְגּ֥וּ לָרֹ֖ב בְּקֶ֥רֶב הָאָֽרֶץ׃
	\source{שמות טו}%
	וַיֹּ֩אמֶר֩ אִם־שָׁמ֨וֹעַ תִּשְׁמַ֜ע לְק֣וֹל ׀ יְיָ֣ אֱלֹהֶ֗יךָ וְהַיָּשָׁ֤ר בְּעֵינָיו֙ תַּעֲשֶׂ֔ה וְהַֽאֲזַנְתָּ֙ לְמִצְוֺתָ֔יו וְשָׁמַרְתָּ֖ כׇּל־חֻקָּ֑יו כׇּֽל־הַמַּחֲלָ֞ה אֲשֶׁר־שַׂ֤מְתִּי בְמִצְרַ֙יִם֙ לֹא־אָשִׂ֣ים עָלֶ֔יךָ כִּ֛י אֲנִ֥י יְיָ֖ רֹפְאֶֽךָ׃
	\source{זכריה ג}%
	וַיֹּ֨אמֶר יְיָ֜ אֶל־הַשָּׂטָ֗ן יִגְעַ֨ר יְיָ֤ בְּךָ֙ הַשָּׂטָ֔ן וְיִגְעַ֤ר יְיָ֙ בְּךָ֔ הַבֹּחֵ֖ר בִּירֽוּשָׁלָ֑‍ִם הֲל֧וֹא זֶ֦ה א֖וּד מֻצָּ֥ל מֵאֵֽשׁ׃
	\source{שה״ש ג}%
	הִנֵּ֗ה מִטָּתוֹ֙ שֶׁלִּשְׁלֹמֹ֔ה שִׁשִּׁ֥ים גִּבֹּרִ֖ים סָבִ֣יב לָ֑הּ מִגִּבֹּרֵ֖י יִשְׂרָאֵֽל׃ כֻּלָּם֙ אֲחֻ֣זֵי חֶ֔רֶב מְלֻמְּדֵ֖י מִלְחָמָ֑ה אִ֤ישׁ חַרְבּוֹ֙ עַל־יְרֵכ֔וֹ מִפַּ֖חַד בַּלֵּילֽוֹת׃
	
	\source{במדבר ו}%
	יְבָרֶכְךָ֥ יְיָ֖ וְיִשְׁמְרֶֽךָ׃ יָאֵ֨ר יְיָ֧ ׀ פָּנָ֛יו אֵלֶ֖יךָ וִֽיחֻנֶּֽךָּ׃ יִשָּׂ֨א יְיָ֤ ׀ פָּנָיו֙ אֵלֶ֔יךָ וְיָשֵׂ֥ם לְךָ֖ שָׁלֽוֹם׃
	\source{תהלים קכא}%
	הִנֵּ֣ה לֹֽא־יָ֭נוּם וְלֹ֣א יִישָׁ֑ן שׁ֝וֹמֵ֗ר יִשְׂרָאֵֽל׃
	
	לִֽישׁוּעָֽתְךָ֖ \source{בראשית מט}קִוִּ֥יתִי יְיָ׃ קִוִּֽיתִי יְיָ לִישׁוּעָתְךָ׃ לִישׁוּעָתְךָ יְיָ קִוִּ֥יתִי׃
	
	%\source{תהלים קכח}\source{תהלים ד}%
	\adonolam \end{small}

\firstword{בָּרוּךְ}
אַתָּה יְיָ אֱלֹהֵֽינוּ מֶֽלֶךְ הָעוֹלָם הַמַּפִּיל חֶבְלֵי שֵׁנָה עַל עֵינַי וּתְנוּמָה עַל עַפְעַפָּי׃ וִיהִי רָצוֹן מִלְּפָנֶֽיךָ יְיָ אֱלֹהַי וֵאלֹהֵי אֲבוֹתַי שֶׁתַּשְׁכִּיבֵֽנִי לְשָׁלוֹם וְתַעֲמִידֵֽנִי לְשָׁלוֹם׃ וְאַל יְבַהֲלֽוּנִי רַעְיוֹנַי וַחֲלוֹמוֹת רָעִים וְהִרְהוּרִים רָעִים׃ וּתְהֵא מִטָּתִי שְׁלֵמָה לְפָנֶֽיךָ׃ וְהָאֵר עֵינַי פֶּן אִישַׁן הַמָּֽוֶת׃ כִּי אַתָּה הַמֵּאִיר לְאִישׁוֹן בַּת עָֽיִן׃ בָּרוּךְ אַתָּה יְיָ הַמֵּאִיר לְעוֹלָם כֻּלּוֹ בִּכְבוֹדוֹ׃\\

\begingroup
\let\clearpage\relax
\chapter[בדיקת וביטול חמץ]{\adforn{47} בדיקת וביטול חמץ \adforn{25}}

\instruction{לפני הבדיקה בליל י״ד בניסן:}\\
\firstword{בָּרוּךְ}
אַתָּה יְיָ אֱלֹהֵֽינוּ מֶֽלֶךְ הָעוֹלָם אֲשֶׁר קִדְּשָֽׁנוּ בְּמִצְוֹתָיו וְצִוָּֽנוּ עַל־בִּעוּר חָמֵץ׃


\instruction{אחר הבדיקת הלילה מבטלים את החמץ:}\\
כׇּל־חֲמִירָא וַחֲמִיעָא דְּאִיכָּא בִרְשׁוּתִי, דְּלָא חֲמִתֵּיהּ, וּדְלָא בַּעֲרִתֵּיהּ, לִבְטִיל וְלֶהֱוֵי כְּעַפְרָא דִאַרְעָא׃

\instruction{ישמור את החמץ שמצא עד הבוקר.}

\instruction{בערב פסח, לפני שעה השישית, שורפים את החמץ שמצה אתמול בלילה, ואומרים:}\\
כׇּל־חֲמִירָא וַחֲמִיעָא דְּאִיכָּא בִרְשׁוּתִי, דַּחֲמִתֵּיהּ דְּלָא חֲמִתֵּיהּ, דּבַעַרִתֵּיהּ וּדְלָא בַּעֲרִתֵּיהּ, לִבְטִיל וְלֶהֱוֵי כְּעַפְרָא דִאַרְעָא׃\\



\chapter[התרת נדרים]{\adforn{47} התרת נדרים \adforn{25}}

\instruction{בערב ראש השנה אומרים התרת נדרים לפני בית דין של שלושה.}

שִׁמְעוּ־נָא רַבּוׂתַי דַּיָּנִים מֻמְחִים, כׇּל־נֶדֶר אוׂ שְׁבוּעָה אוׂ אִסָּר אוׂ קוׂנָם אוׂ חֵרֶם, שֶׁנָּדַרְתִּי אוׂ נִשְׁבַּעְתִּי בְּהָקִיץ אוׂ בַחֲלוׂם, אוׂ נִשְׁבַּעְתִּי בַּשֵּׁמוׂת הַקְּדוׂשִׁים שֶׁאֵינָם נִמְחָקִים וּבְשֵׁם הוי״ה בָּרוּךְ הוּא, וְכׇל־מִינֵי נְזִירוּת שֶׁקִּבַּלְתִּי עָלַי וַאֲפִלּוּ נְזִירוּת שִׁמְשׁוׂן, וְכׇל־שׁוּם אִסּוּר, וַאֲפִלּוּ אִסּוּר הֲנָאָה שֶׁאָסַרְתִּי עָלַי אוׂ עַל אֲחֵרִים, בְּכׇל־לָשׁוׂן שֶׁל אִסּוּר בֵּין בִּלְשׁוׂן אִסּוּר אוׂ חֵרֶם אוׂ קוׂנָם וְכׇל־שׁוּם קַבָּלָה אֲפִלּוּ שֶׁל מִצְוָה שֶׁקִּבַּלְתִּי עָלַי, בֵּין בִּלְשׁון נֶדֶר בֵּין בִּלְשׁון נְדָבָה בֵּין בִּלְשׁוׂן שְׁבוּעָה בֵּין בִּלְשׁוׂן נְזִירוּת בֵּין בְּכׇל־לָשׁוׂן, וְגַם הַנַּעֲשֶׂה בִּתְקִיעַת כָּף. בֵּין כׇּל־נֶדֶר וּבֵין כׇּל־נְדָבָה, וּבֵין שׁוּם מִנְהָג שֶׁל מִצְוָה שֶׁנָּהַגְתִּי אֶת־עַצְמִי וְכׇל־מוׂצָא שְׂפָתַי שֶׁיָּצָא מִפִּי, אוׂ שֶׁנָּדַרְתִּי וְגָמַרְתִּי בְלִבִּי לַעֲשוׂת שׁוּם מִצְוָה מֵהַמִּצְוׂת, אוׂ אֵיזוׂ הַנְהָגָה טוׂבָה אוׂ אֵיזֶה דָבָר שֶׁנָּהַגְתִּי שָׁלשׁ פְּעָמִים וְלא הִתְנֵיתִי שֶׁיְּהֵא בְּלִי נֶדֶר, הֵן דָּבָר שֶׁעָשִׂיתִי עַל עַצְמִי הֵן עַל אֲחֵרִים, הֵן אוׂתָן הַיְדוּעִים לִי הֵן אוׂתָן שֶׁכְּבָר נִשְׁכְּחוּ מִמֶּנּי, בְּכֻלְּהוׂן מִתְחָרַטְנָא בְהוׂן מֵעִקָּרָא, וְשׁוׂאֵל/וְשׁוׂאֶלֶת וּמְבַקֵּשׁ/וּמְבַקֶּשֶׁת אֲנִי מִמַּעֲלַתְכֶם הַתָּרָה עֲלֵיהֶם, כִּי יָרֵאתִי פֶּן אֶכָּשֵׁל וְנִלְכַּדְתִּי חַס וְשָׁלוׂם בַּעֲוׂן נְדָרִים, וּשְׁבוּעות וּנְזִירוּת, וַחֲרָמוׂת וְאִסּוּרִין, וְקוׂנָמוׂת וְהַסְכָּמוׂת.

וְאֵין אֲנִי תוׂהֵא/תוׂהָא חַס וְשָׁלוׂם עַל קִיּוּם הַמַּעֲשִׂים הַטוׂבִים הָהֵם שֶׁעָשִׂיתִי, רַק אֲנִי מִתְחָרֵט/מִתְחָרֶטֶת עַל קַבָּלַת הָעִנְיָנִים בִּלְשׁוׂן נֶדֶר אוׂ שְׁבוּעָה, אוׂ נְזִירוּת אוׂ אִסּוּר, אוׂ חֵרֶם, אוׂ קוׂנָם, אוׂ הַסְכָּמָה, אוׂ קַבָּלָה בְּלֵב. וּמִתְחָרֵט/מִתְחָרֶטֶת אֲנִי עַל זֶה שֶׁלּא אָמַרְתִּי הִנְנִי עוׂשֶׂה/עוׂשָׂה דָבָר זֶה בְּלִי נֶדֶר וּשְׁבוּעָה, וּנְזִירוּת וְחֵרֶם וְאִסּוּר, וְקוׂנָם וְקַבָּלָה בְּלֵב. לָכֵן אֲנִי שׁוׂאֵל/שׁוׂאֶלֶת הַתָּרָה בְּכֻלְּהוׂן, וַאֲנִי מִתְחָרֵט/מִתְחָרֶטֶת עַל כׇּל־הַנִּזְכָּר, בֵּין אִם הָיוּ הַמַּעֲשִׂים מִדְּבָרִים הַנּוׂגְעִים בְּמָמוׂן, בֵּין מֵהַדְּבָרִים הַנּוׂגְעִים בְּגוּף, בֵּין מֵהַדְּבָרִים הַנּוׂגְעִים אֶל הַנְּשָׁמָה, בְּכֻלְּהוׂן אֲנִי מִתְחָרֵט/מִתְחָרֶטֶת, עַל לְשׁוׂן נֶדֶר וּשְׁבוּעָה וּנְזִירוּת וְאִסּוּר וְחֵרֶם וְקוׂנָם וְקַבָּלָה בְלֵב. וְהִנֵּה מִצַּד הַדִּין, הַמִּתְחָרֵט/מִתְחָרֶטֶת וְהַמְבַקֵּשׁ/וְהַמְבַקֶּשֶׁת הַתָּרָה, צָרִיךְ לִפְרוׂט הַנֶּדֶר, אַךְ דְּעוּ־נָא רַבּוׂתַי, כִּי אִי אֶפְשָׁר לְפָרְטָם, כִּי רַבִּים הֵם. וְאֵין אֲנִי מְבַקֵּשׁ/מְבַקֶּשֶׁת הַתָּרָה עַל אוׂתָם הַנְּדָרִים שֶׁאֵין לְהַתִּיר אוׂתָם, עַל כֵּן יִהְיוּ־נָא בְעֵינֵיכֶם כְּאִלּוּ הָיִיתִי פוׂרְטָם׃

\instruction{הדיינים ג׳ פעמים:}\\
הַכּל יִהְיוּ מֻתָּרִים לָךְ, הַכּל מְחוּלִים לָךְ, הַכּל שְׁרוּיִם לָךְ, אֵין כַּאן לֺא נֶדֶר, וְלֺא שְׁבוּעָה, וְלֺא נְזִירוּת, וְלֺא חֵרֶם, וְלֺא אִסָּר וְלֺא קוׂנָם. יֵשׁ כַּאן מְחִילָה סְלִיחָה וְכַפָּרָה. וּכְשֵׁם שֶׁמַּתִּירִים הַבֵּית דִּין שֶׁל מַטָּה כַּךְ יִהְיוּ מֻתָּרִים מִבֵּית דִּין שֶׁל מַעְלָה׃

\instruction{המבקש ממשיך:}\\
הֲרֵי אֲנִי מוֺסֵר/מוֺסֶרֶת מוׂדָעָה לִפְנֵיכֶם וַאֲנִי מְבַטֵּל/מְבַטֶּלֶת מִכַּאן וּלְהַבָּא מַה שֶּׁאֲקַבֵּל עָלַי כׇּל־הַנְּדָרִים וְכׇל־שְׁבוּעוׂת וּנְזִירוּת וְאִסָּרִין וְקוׂנָמוׂת וַחֲרָמוׂת וְהַסְכָּמוׂת וְקַבָּלָה בְלֵב, הֵן בְּהָקִיץ הֵן בַּחֲלוׂם חוּץ מִנִּדְרֵי תַעֲנִית בִּשְׁעַת מִנְחָה. וּבְאִם שֶׁאֶשְׁכַּח לִתְנַאי מוׂדָעָה הַזּאת וְאֶדּׂר מֵהַיּוׂם עוׂד, מֵעַתָּה אֲנִי מִתְחָרֵט/מִתְחָרֶטֶת עֲלֵיהֶם וּמַתְנֶה/וּמַתְנָה עֲלֵיהֶם שֶׁיִּהְיוּ כֻּלָּם בְּטֵלִים וּמְבֻטָּלִים לָא שְׁרִירִין וְלָא קַיָּמִין וְלָא יְהוׂן חָלִין כְּלָל וּכְלָל. בְּכֻלָּן מִתְחָרַטְנָא בְהוׂן מֵעַתָּה וְעַד עוׂלָם׃

\vfill


\chapter[וידוי]{\adforn{47} וידוי \adforn{19}}

אֱלֹהֵֽינוּ וֵאלֹהֵי אֲבוֹתֵֽינוּ תָּבֹא לְפָנֶֽיךָ תְּפִילָּתֵֽנוּ וְאַל תִּתְעַלַּם מִתְּחִנָּתֵֽינוּ. שֶׁאֵין אָֽנוּ עַזֵּי פָנִים וּקְשֵׁי עֹֽרֶף לוֹמַר לְפָנֶֽיךָ יְיָ אֱלֹהֵֽינוּ וֵאלֹהֵי אֲבוֹתֵֽינוּ צַדִּיקִים אֲנַֽחֲנוּ וְלֹא חָטָֽאנוּ אֲבָל אֲנַֽחְנוּ וַאֲבוֹתֵֽינוּ חָטָֽאנוּ׃

אָשַֽׁמְנוּ׃ בָּגַֽדְנוּ׃ גָּזַֽלְנוּ׃ דִּבַּֽרְנוּ דֹֽפִי׃ \\
הֶעֱוִֽינוּ׃ וְהִרְשַֽׁעְנוּ׃ זַֽדְנוּ׃ חָמַֽסְנוּ׃ טָפַֽלְנוּ שֶֽׁקֶר׃\\
יָעַֽצְנוּ רַע׃ כִּזַּֽבְנוּ׃ לַֽצְנוּ׃ מָרַֽדְנוּ׃ נִאַֽצְנוּ׃ \\
סָרַֽרְנוּ׃ עָוִֽינוּ׃ פָּשַֽׁעְנוּ׃ צָרַֽרְנוּ׃ קִשִּֽׁינוּ עֹֽרֶף׃\\
רָשַֽׁעְנוּ׃ שִׁחַֽתְנוּ׃ תִּעַֽבְנוּ׃ תָּעִֽינוּ׃ תִּעְתָּֽעְנוּ׃

סַֽרְנוּ מִמִּצְוֹתֶֽיךָ וּמִמִּשְׁפָּטֶֽיךָ הַטּוֹבִים וְלֹא שָֽׁוָה לָֽנוּ׃ \source{נחמיה ט}וְאַתָּ֣ה צַדִּ֔יק עַ֖ל כׇּל־הַבָּ֣א עָלֵ֑ינוּ כִּֽי־אֱמֶ֥ת עָשִׂ֖יתָ וַאֲנַ֥חְנוּ הִרְשָֽׁעְנוּ׃ מַה נֹּאמַר לְפָנֶֽיךָ יוֹשֵׁב מָרוֹם וּמַה נְסַפֵּר לְפָנֶֽיךָ שׁוֹכֵן שְׁחָקִים הֲלֹא כׇּל־הַנִּסְתָּרוֹת וְהַנִּגְלוֹת אַתָּה יוֹדֵֽעַ׃

אַתָּה יוֹדֵֽעַ רָזֵי עוֹלָם וְתַעֲלוּמוֹת סִתְרֵי כׇּל־חָי׃ אַתָּה חֹפֵשׂ כׇּל־חַדְרֵי־בָֽטֶן וּבֹּחֵן כְּלָיוֹת וָלֵב. אֵין דָּבָר נֶעֱלָּם מִמֶּֽךָּ וְאֵין נִסְתָּר מִנֶּֽגֶד עֵינֶֽיךָ׃ וּבְכֵן יְהִי רָצוֹן מִלְּפָנֶיךָ יְיָ אֱלֹהֵֽינוּ וֵאלֹהֵי אֲבוֹתֵֽינוּ שֶׁתִּסְלַח לָֽנוּ עַל כׇּל־חַטֹּאתֵֽינוּ וְתִמְחַל־לָֽנוּ עַל כׇּל־עֲוֹנוֹתֵֽינוּ וּתְכַפֵר־לָנוּ עַל פְּשָׁעֵֽינוּ׃



עַל חֵטְא שֶׁחָטָֽאנוּ לְפָנֶֽיךָ בְּאֹֽנֵס וּבְרָצוֹן׃\\ וְעַל חֵטְא שֶׁחָטָֽאנוּ לְפָנֶֽיךָ בְּאִמוּץ הַלֵּב׃ \\
עַל חֵטְא שֶׁחָטָֽאנוּ לְפָנֶֽיךָ בִּבְלִי דָֽעַת׃ \\ וְעַל חֵטְא שֶׁחָטָֽאנוּ לְפָנֶֽיךָ בְּבִטּוּי שְׂפָתָֽיִם׃\\
%עַל חֵטְא שֶׁחָטָֽאנוּ לְפָנֶֽיךָ בַּגָלוּי וּבַסָּֽתֶר׃ \\ וְעַל חֵטְא שֶׁחָטָֽאנוּ לְפָנֶֽיךָ בְּגִלּוּי עֲרָיוֹת׃ \\
עַל חֵטְא שֶׁחָטָֽאנוּ לְפָנֶֽיךָ בְּגִלּוּי עֲרָיוֹת׃ \\ וְעַל חֵטְא שֶׁחָטָֽאנוּ לְפָנֶֽיךָ בַּגָלוּי וּבַסָּֽתֶר׃ \\
%עַל חֵטְא שֶׁחָטָֽאנוּ לְפָנֶֽיךָ בְּדִבּוּר פֶּה׃ \\ וְעַל חֵטְא שֶׁחָטָֽאנוּ לְפָנֶֽיךָ בְּדַֽעַת וּבְמִרְמָה׃ \\
עַל חֵטְא שֶׁחָטָֽאנוּ לְפָנֶֽיךָ בְּדַֽעַת וּבְמִרְמָה׃ \\ וְעַל חֵטְא שֶׁחָטָֽאנוּ לְפָנֶֽיךָ בְּדִבּוּר פֶּה׃ \\
%עַל חֵטְא שֶׁחָטָֽאנוּ לְפָנֶֽיךָ בְּהִרְהוּר הַלֵּב׃ \\ וְעַל חֵטְא שֶׁחָטָֽאנוּ לְפָנֶֽיךָ בְּהוֹנָֽאַת רֵֽעַ׃ \\
עַל חֵטְא שֶׁחָטָֽאנוּ לְפָנֶֽיךָ בְּהוֹנָֽאַת רֵֽעַ׃ \\ וְעַל חֵטְא שֶׁחָטָֽאנוּ לְפָנֶֽיךָ בְּהִרְהוּר הַלֵּב׃ \\
%עַל חֵטְא שֶׁחָטָֽאנוּ לְפָנֶֽיךָ בְּוִדּוּי פֶּה׃\\ וְעַל חֵטְא שֶׁחָטָֽאנוּ לְפָנֶֽיךָ בִּוְעִידַת זְנוּת׃ \\
עַל חֵטְא שֶׁחָטָֽאנוּ לְפָנֶֽיךָ בִּוְעִידַת זְנוּת׃\\ וְעַל חֵטְא שֶׁחָטָֽאנוּ לְפָנֶֽיךָ בְּוִדּוּי פֶּה׃ \\
%עַל חֵטְא שֶׁחָטָֽאנוּ לְפָנֶֽיךָ בְּזָדוֹן וּבִשְׁגָגָה׃ \\ וְעַל חֵטְא שֶׁחָטָֽאנוּ לְפָנֶֽיךָ בְּזִלזוּל הוֹרִים וּמוֹרִים׃\\
עַל חֵטְא שֶׁחָטָֽאנוּ לְפָנֶֽיךָ בְּזִלזוּל הוֹרִים וּמוֹרִים׃ \\ וְעַל חֵטְא שֶׁחָטָֽאנוּ לְפָנֶֽיךָ  בְּזָדוֹן וּבִשְׁגָגָה׃\\
עַל חֵטְא שֶׁחָטָֽאנוּ לְפָנֶֽיךָ בְּחֹֽזֶק יָד׃ \\ וְעַל חֵטְא שֶׁחָטָֽאנוּ לְפָנֶֽיךָ בְּחִלּוּל הַשֵּׁם׃ \\
%עַל חֵטְא שֶׁחָטָֽאנוּ לְפָנֶֽיךָ בְּטִפְשׁוּת פֶּה׃ \\ וְעַל חֵטְא שֶׁחָטָֽאנוּ לְפָנֶֽיךָ בְּטֻמְאַת שְׂפָתָֽיִם׃ \\
עַל חֵטְא שֶׁחָטָֽאנוּ לְפָנֶֽיךָ בְּטֻמְאַת שְׂפָתָֽיִם׃ \\ וְעַל חֵטְא שֶׁחָטָֽאנוּ לְפָנֶֽיךָ בְּטִפְשׁוּת פֶּה׃ \\
עַל חֵטְא שֶׁחָטָֽאנוּ לְפָנֶֽיךָ בְּיֵֽצֶר הָרָע׃ \\ וְעַל חֵטְא שֶׁחָטָֽאנוּ לְפָנֶֽיךָ בְּיוֹדְעִים וּבְלֹא יוֹדְעִים׃

\textbf{וְעַל כֻּלָם סְלַח לָֽנוּ מְחַל לָֽנוּ כַּפֵּר לָֽנוּ׃}

עַל חֵטְא שֶׁחָטָֽאנוּ לְפָנֶֽיךָ בְּכַפַּת־שֹֽׁחַד׃ \\ וְעַל חֵטְא שֶׁחָטָֽאנוּ לְפָנֶֽיךָ בְּכַּֽחַשׁ וּבְכָזָב׃ \\
עַל חֵטְא שֶׁחָטָֽאנוּ לְפָנֶֽיךָ בְּלָשׁוֹן הָרָע׃\\ וְעַל חֵטְא שֶׁחָטָֽאנוּ לְפָנֶֽיךָ בְּלָצוֹן׃\\
עַל חֵטְא שֶׁחָטָֽאנוּ לְפָנֶֽיךָ בְּמַשָּׂא וּבְמַתָּן׃ \\ וְעַל חֵטְא שֶׁחָטָֽאנוּ לְפָנֶֽיךָ בְּמַאֲכׇל וּבְמִשְׁתֶּה׃\\
עַל חֵטְא שֶׁחָטָֽאנוּ לְפָנֶֽיךָ בְּנֶֽשֶׁךְ וּבְמַרְבִּית׃\\ וְעַל חֵטְא שֶׁחָטָֽאנוּ לְפָנֶֽיךָ בִּנְטִיַּת גָּרוֹן׃ \\
עַל חֵטְא שֶׁחָטָֽאנוּ לְפָנֶֽיךָ בְּשִׂקּוּר עָֽיִן׃\\ וְעַל חֵטְא שֶׁחָטָֽאנוּ לְפָנֶֽיךָ בְּשִֽׂיחַ שִׂפְתוֹתֵֽינוּ׃ \\
עַל חֵטְא שֶׁחָטָֽאנוּ לְפָנֶֽיךָ בְּעֵינַֽיִם רָמוֹת׃\\ וְעַל חֵטְא שֶׁחָטָֽאנוּ לְפָנֶֽיךָ בְּעַזּוּת מֶֽצַח׃

\textbf{וְעַל כֻּלָם סְלַח לָֽנוּ מְחַל לָֽנוּ כַּפֵּר לָֽנוּ׃}

עַל חֵטְא שֶׁחָטָֽאנוּ לְפָנֶֽיךָ בִּפְרִֽיקַת עֹל׃\\ וְעַל חֵטְא שֶׁחָטָֽאנוּ לְפָנֶֽיךָ בִּפְלִילוֹת׃ \\
עַל חֵטְא שֶׁחָטָֽאנוּ לְפָנֶֽיךָ בִּצְדִיַּת רֵֽעַ׃ \\ וְעַל חֵטְא שֶׁחָטָֽאנוּ לְפָנֶֽיךָ בְּצָרוּת עָֽיִן׃ \\
עַל חֵטְא שֶׁחָטָֽאנוּ לְפָנֶֽיךָ בְּקַלּוּת רֹאשׁ׃\\ וְעַל חֵטְא שֶׁחָטָֽאנוּ לְפָנֶֽיךָ בְּקַשְׁיוּת עֹֽרֶף׃ \\
עַ״חֵ שֶׁחָטָֽאנוּ לְפָנֶֽיךָ בְּרִיצַת רַגְלַֽיִם לְהָרַע׃\\ וְעַל חֵטְא שֶׁחָטָֽאנוּ לְפָנֶֽיךָ בִּרְכִילוּת׃ \\
עַל חֵטְא שֶׁחָטָֽאנוּ לְפָנֶֽיךָ בִּשְׁבֽוּעַת שָׁוְא׃ \\ וְעַל חֵטְא שֶׁחָטָֽאנוּ לְפָנֶֽיךָ בְּשִׂנְאַת חִנָם׃ \\
עַל חֵטְא שֶׁחָטָֽאנוּ לְפָנֶֽיךָ בִּתְשֽׂוּמֶת יָד׃\\ וְעַל חֵטְא שֶׁחָטָֽאנוּ לְפָנֶֽיךָ בְּתִמְהוֹן לֵבָב׃

\textbf{וְעַל כֻּלָם סְלַח לָֽנוּ מְחַל לָֽנוּ כַּפֵּר לָֽנוּ׃}

וְעַל חֲטָאִים שֶׁאָֽנוּ חַיָּבִים עֲלֵיהֶם עוֹלָה׃\\
וְעַל חֲטָאִים שֶׁאָֽנוּ חַיָּבִים עֲלֵיהֶם חַטָּאת׃\\
וְעַל חֲטָאִים שֶׁאָֽנוּ חַיָּבִים עֲלֵיהֶם קׇרְבָּן עוֹלֶה וְיוֹרֵד׃\\
וְעַל חֲטָאִים שֶׁאָֽנוּ חַיָּבִים עֲלֵיהֶם אָשָׁם וַדַאי וְתָּלוּי׃\\
וְעַל חֲטָאִים שֶׁאָֽנוּ חַיָּבִים עֲלֵיהֶם מַכַּת מַרְדוּת׃
וְעַל חֲטָאִים שֶׁאָֽנוּ חַיָּבִים עֲלֵיהֶם מַּלְקּוּת אַרְבָּעִים׃\\
וְעַל חֲטָאִים שֶׁאָֽנוּ חַיָּבִים עֲלֵיהֶם מִיתָה בִּידֵי שָׁמַיִם׃\\
וְעַל חֲטָאִים שֶׁאָֽנוּ חַיָּבִים עֲלֵיהֶם כָּרֵת וְעֲרִירִי׃\\
וְעַל חֲטָאִים שֶׁאָֽנוּ חַיָּיבִים עֲלֵיהֶם אַרְבַּע מִיתוֹת בֵּית דִּין - סְקִילָה שְׂרֵיפָה הֶֽרֶג וְחֶֽנֶק׃

עַל מִצְוַת עֲשֵׂה וְעַל מִצְוַת לֹא תַעֲשֶׂה בֵּין שֶׁיֵשׁ־בָּהּ קוּם עֲשֵׂה וּבֵין שֶׁאֵין בָּהּ קוּם עֲשֵׂה׃ אֶת־הַגְּלוּיִם לָֽנוּ וְאֶת־שֶׁאֵינָם גְּלוּיִם לָֽנוּ. אֶת־שֶׁגְּלוּיִם לָנוּ כְּבָר אֲמַרְנוּם לְפָנֶיךָ וְהוֹדִינוּ לְךָ עֲלֵיהֶם. וְאֶת־שֶׁאֵינָם גְּלוּיִם לָנוּ לְפָנֶיךָ הֵם גְּלוּיִם וִידוּעִים כַּדָּבָר שֶׁנֶּאֱמַר- הַנִּ֨סְתָּרֹ֔ת
\source{דברים כט}
לַֽיְיָ֖ אֱלֹהֵ֑ינוּ וְהַנִּגְלֹ֞ת לָֹ֤נֹוֹּ וֹּלְֹבָֹנֵֹ֨יֹנֹוֹּ֙ עַד־עוֹלָ֔ם לַֽעֲשׂ֕וֹת אֶת־כׇּל־דִּבְרֵ֖י הַתּוֹרָ֥ה הַזֹּֽאת׃

כִּי אַתָּה סׇלְחָן לְיִשְׂרָאֵל וּמׇחֳלָן לְשִׁבְטֵי יְשֻׁרוּן בְּכׇל־דּוֹר וָדוֹר וּמִבַּלְעָדֶיךָ אֵין לׇׇֽנּוּ מֶֽלֶךְ מוֹחֵל וְסוֹלֵֽחַ אֶלָּא אַֽתָּה׃

אֱלֹהַי עַד שֶׁלֹא נוֹצַרתִּי אֵינִי כְּדַאי וְעַכְשָׁיו שֶׁנוֹצַרתִּי כְּאִלוּ לֹא נוֹצַרתִּי עָפָר אֲנִי בְּחַיָי קַּל וָחוֹמֶר בְּמִיתָתִי הֲרֵינִי לְפָּנֶיךָ כִּכְלֵי מָלֵא בוּשָׁה וּכְלִמָה׃ יְהִי רָצוֹן מִלְּפָנֶֽיךָ יְיָ אֱלֹהַי וֵאלֹהֵי אֲבוֹתַי שֶׁלֹא אֶחֱטָא עוֹד וּמַה שֶׁחָטָאתִי לְפָנֶיךָ מְחוֹק בְּרַחֲמֶיךָ הָרַבִּים אֲבָל לֹא עַל יְדֵי יְסוּרִים וַחֲלָאִים רָעִים׃

\vfill

\chapter[חנכה]{\adforn{47} חנכה \adforn{19}}
\label{chanukah}

\instruction{לפני הדלקת הנרות:}\\
\firstword{בָּרוּךְ}
אַתָּה יְיָ אֱלֹהֵֽינוּ מֶֽלֶךְ הָעוֹלָם
אֲשֶׁר קִדְּשָֽׁנוּ בְּמִצְוֹתָיו וְצִוָּֽנוּ לְהַדְלִיק נֵר שֶׁל חֲנֻכָּה׃

\firstword{בָּרוּךְ}
אַתָּה יְיָ אֱלֹהֵֽינוּ מֶֽלֶךְ הָעוֹלָם שֶׁעָשָׂה נִסִּים לַאֲבוֹתֵֽינוּ בַּיָּמִים הָהֵם בַּזְּמַן הַזֶּה׃

\instruction{רק בליל ראשון:}\\
\firstword{בָּרוּךְ}
אַתָּה יְיָ אֱלֹהֵֽינוּ מֶֽלֶךְ הָעוֹלָם שֶׁהֶחֱיָֽנוּ וְקִיְּמָֽנוּ וְהִגִּיעָֽנוּ לַזְּמַן הַזֶּה׃\\

\instruction{אחר הדלקת הנרות:}\\
\firstword{הַנֵּרוֹת הַלָּלוּ}
אָנוּ מַדְלִיקִין
עַל הַנִּסִּים וְעַל הַתְּשׁוּעוֹת
וְעַל הַנִּפְלָאוֹת
וְעַל הַמִּלְחָמוֹת
שֶׁעָשִׂיתָ לַאֲבוֹתֵינוּ
בַּיָּמִים הָהֵם בַּזְּמַן הַזֶּה
עַל יְדֵי כּהֲנֶיךָ הַקְּדוֹשִׁים.
וְכׇל־שְׁמוֹנַת יְמֵי חֲנֻכָּה
הַנֵּרוֹת הַלָּלוּ קֹדֶשׁ הֵם
וְאֵין לָנוּ רְשׁוּת לְהִשְׁתַּמֵּשׁ בָּהֶם
אֶלָּא לִרְאוֹתָם בִּלְבָד
כְּדֵי לְהוֹדוֹת וּלְהַלֵּל לְשִׁמְךָ הַגָּדוֹל
עַל נִסֶּיךָ וְעַל יְשׁוּעָתֶךָ
וְעַל נִפְלְאוֹתֶיךָ׃

\begin{quote}
\leftskip=0pt plus-.5fil
\rightskip=0pt plus.5fil
\parfillskip=0pt plus1fil
\firstword{מָעוֹז צוּר}
יְשׁוּעָתִי \hfill לְךָ נָאֶה לְשַׁבֵּֽחַ \\ תִּכּוֹן בֵּית תְּפִלָּתִי \hfill וְשָׁם תּוֹדָה נְזַבֵּֽחַ \\
לְעֵת תָּכִין מַטְבֵּֽחַ \hfill מִצָּר הַמְנַבֵּֽחַ \\ אָז אֶגְמוֹר\hfill בְּשִׁיר מִזְמוֹר \hfill חֲנֻכַּת הַמִּזְבֵּֽחַ׃

\firstword{רָעוֹת}
שָׂבְעָה נַפְשִׁי \hfill בְּיָגוֹן כֹּחִי כָּלָה \\ חַיַּי מֵרֲרוּ בָּקֳשִׁי \hfill בְּשִׁעְבּוּד מַלְכוּת עֶגְלָה \\
וּבְיָדוֹ הַגְּדֻלָּה \hfill הוֹצִיא אֶת־הַסְּגֻלָּה \\ חֵיל פַּרְעֹה\hfill וְכׇל־זַרְעוֹ \hfill יָרְדוּ כְּאֶֽבֶן מְצוּלָה׃

\firstword{דְּבִיר}
קָדְשׁוֹ הֱבִיאַֽנִי \hfill וְגַם שָׁם לֹא שָׁקַֽטְתִּי \\ וּבָא נוֹגֵשׂ וְהִגְלַֽנִי \hfill כִּי זָרִים עָבַֽדְתִּי \\
וְיֵין רַֽעַל מָֽסַכְתִּי \hfill כִּמְעַט שֶׁעָבַֽרְתִּי \\ קֵץ בָּבֶל\hfill זְרֻבָּבֶל \hfill לְקֵץ שִׁבְעִים נוֹשָֽׁעְתִּי׃

\firstword{כְּרוֹת}
קוֹמַת בְּרוֹשׁ בִּקֵּשׁ \hfill אֲגָגִי בֶּן־הַמְּדָֽתָא \\ וְנִהְיָֽתָה לּוֹ לְמוֹקֵשׁ \hfill וְגַאֲוָתוֹ נִשְׁבָּֽתָה \\
רֹאשׁ יְמִינִי נִשֵּֽׂאתָ \hfill וְאוֹיֵב שְׁמוֹ מָחִֽיתָ \\ רֹב בָּנָיו\hfill וְקִנְיָנָיו \hfill עַל הָעֵץ תָּלִֽיתָ׃

\firstword{יְוָנִים}
נִקְבְּצוּ עָלַי \hfill אֲזַי בִּימֵי חַשְׁמַנִּים \\ וּפָרְצוּ חוֹמוֹת מִגְדָּלַי \hfill וְטִמְּאוּ כׇּל־הַשְּׁמָנִים \\
וּמִנּוֹתַר קַנְקַנִּים \hfill נַעֲשָׂה נֵס לְשׁוֹשַׁנִּים \\ בְּנֵי בִינָה\hfill יְמֵי שְׁמֹנָה \hfill קָבְעוּ שִׁיר וּרְנָנִים׃

\firstword{חֲשׂוֹף}
זְרוֹעַ קׇדְשֶׁךָ \hfill וְקָרֵב קֵץ הַיְשׁוּעָה\\נְקֹם נִקְמַת נַפשִׁי \hfill מִיַּד מַלכוּת הָרְשָׁעָה\\
כִּי אָרְכָה לִי הַשָּׁעָה \hfill וְאֵין קֵץ לְזׂאת הָרָעָה\\בְּצֵל צַלְמוֹן\hfill תִּרְדּׂף אַדְמוֹן \hfill וְתבִיא רוֹעִים שִׁבְעָה׃

\end{quote}

\vfill

\chapter[פורים]{\adforn{47} פורים \adforn{19}}

\label{purim}

\instruction{מברכים לפני קריאת המגילה:}\\
\firstword{בָּרוּךְ}
אַתָּה יְיָ אֱלֹהֵֽינוּ מֶֽלֶךְ הָעוֹלָם
אֲשֶׁר קִדְּשָֽׁנוּ בְּמִצְוֹתָיו וְצִוָּֽנוּ עַל מִקְרָא מְגִלָּה׃


\firstword{בָּרוּךְ}
אַתָּה יְיָ אֱלֹהֵֽינוּ מֶֽלֶךְ הָעוֹלָם שֶׁעָשָׂה נִסִּים לַאֲבוֹתֵֽינוּ בַּיָּמִים הָהֵם בַּזְּמַן הַזֶּה׃

\instruction{רק בערבית:}\\
\firstword{בָּרוּךְ}
אַתָּה יְיָ אֱלֹהֵֽינוּ מֶֽלֶךְ הָעוֹלָם שֶׁהֶחֱיָֽנוּ וְקִיְּמָֽנוּ וְהִגִּיעָֽנוּ לַזְּמַן הַזֶּה׃


\instruction{לאחר הקריאה:} \\
\firstword{בָּרוּךְ}
אַתָּה יְיָ אֱלֹהֵֽינוּ מֶֽלֶךְ הָעוֹלָם הָרָב אֶת־רִיבֵֽנוּ וְהַדָּן אֶת־דִּינֵֽנוּ וְהַנּוֹקֵם אֶת־נִקְמָתֵֽנוּ וְהַמְשַׁלֵּם גְּמוּל לְכׇל־אוֹיְבֵי נַפְשֵֽׁנוּ וְהַנִּפְרָע לָֽנוּ מִצָּרֵֽינוּ׃ בָּרוּךְ אַתָּה יְיָ הַנִּפְרָע לְעַמּוֹ יִשְׂרָאֵל מִכׇּל־צָרֵיהֶם הָאֵל הַמּוֹשִֽׁיעַ׃

\instruction{לאחר הקריאה. בשחרית ממשיכים בשושנת יעקב:}


\firstword{אֲ֗שֶׁר}
הֵנִיא עֲצַת גּוֹיִם וַיָּֽפֶר מַחְשְׁבוֹת עֲרוּמִים׃ \hfill \break
בְּ֗קוּם עָלֵֽינוּ אָדָם רָשָׁע נֵֽצֶר זָדוֹן מִזֶּֽרַע עֲמָלֵק׃ \hfill \break
גָּ֗אָה בְעָשְׁרוֹ וְכָֽרָה לוֹ בּוֹר וּגְדֻלָּתוֹ יָֽקְשָׁה לּוֹ לָֽכֶד׃ \hfill \break
דִּ֗מָּה בְנַפְשׁוֹ לִלְכּוֹד וְנִלְכָּד בִּקֵּשׁ לְהַשְׁמִיד וְנִשְׁמַד מְהֵרָה׃ \hfill \break
הָ֗מָן הוֹדִֽיעַ אֵיבַת אֲבוֹתָיו וְעוֹרֵר שִׂנְאַת אַחִים לַבָּנִים׃ \hfill \break
וְ֗לֹא זָכַר רַחֲמֵי שָׁאוּל כִּי בְחֶמְלָתוֹ עַל אֲגָג נוֹלַד אוֹיֵב׃ \hfill \break
זָ֗מַם רָשָׁע לְהַכְרִית צַדִּיק וְנִלְכַּד טָמֵא בִּידֵי טָהוֹר׃ \hfill \break
חֶֽ֗סֶד גָּבַר עַל שִׁגְגַת אָב וְרָשָׁע הוֹסִיף חֵטְא עַל חֲטָאָיו׃ \hfill \break
טָ֗מַן בְּלִבּוֹ מַחְשְׁבוֹת עֲרוּמָיו וַיִּתְמַכֵּר לַעֲשׂוֹת רָעָה׃ \hfill \break
יָ֗דוֹ שָׁלַח בִּקְדֽוֹשֵי אֵל כַּסְפּוֹ נָתַן לְהַכְרִית זִכְרָם׃ \hfill \break
כִּ֗רְאוֹת מׇרְדֳּכַי כִּי יָֽצָא קֶֽצֶף וְדָתֵי הָמָן נִתְּנוּ בְּשׁוּשָׁן׃ \hfill \break
לָ֗בַשׁ שַׂק וְקָשַׁר מִסְפֵּד וְגָזַר צוֹם וַיֵּֽשֶׁב עַל הָאֵֽפֶר׃ \hfill \break
מִ֗י זֶה יַעֲמוֹד לְכַפֵּר שְׁגָגָה וְלִמְחוֹל חַטַּאת עֲוֹן אֲבוֹתֵֽינוּ׃ \hfill \break
נֵ֗ץ פָּרַח מִלּוּלָב הֵן הֲדַסָּה עָמְדָה לְעוֹרֵר יְשֵׁנִים׃ \hfill \break
סָ֗רִיסֶֽיהָ הִבְהִֽילוּ לְהָמָן לְהַשְׁקוֹתוֹ יֵין חֲמַת תַּנִּינִים׃ \hfill \break
עָ֗מַד בְעָשְׁרוֹ וְנָפַל בְּרִשְׁעוֹ עָֽשָׂה לוֹ עֵץ וְנִתְלָה עָלָיו׃ \hfill \break
פִּ֗יהֶם פָּתְחוּ כׇּל־יוֹשְׁבֵי תֵבֵל כִּי פוּר הָמָן נֶהְפַּךְ לְפוּרֵֽנוּ׃ \hfill \break
צַ֗דִּיק נֶחֱלַץ מִיַּד רָשָׁע אוֹיֵב נִתַּן תַּֽחַת נַפְשׁוֹ׃ \hfill \break
קִ֗יְּמוּ עֲלֵיהֶם לַעֲשׂוֹת פּוּרִים וְלִשְׂמוֹחַ בְּכׇל־שָׁנָה וְשָׁנָה׃ \hfill \break
רָ֗אִֽיתָ אֶת־תְּפִלַּת מׇרְדְּכַי וְאֶסְתֵּר הָמָן וּבָנָיו עַל הָעֵץ תָּלִֽיתָ׃ \hfill \break
\firstword{שׁ֗וֹשַׁנַּת יַעֲקֹב}
צָהֲלָה וְשָׂמֵֽחָה בִּרְאוֹתָם יַֽחַד תְּכֵֽלֶת מׇרְדְּכָי׃ \hfill \break
תְּ֗שׁוּעָתָם הָיִֽיתָ לָנֶֽצַח וְתִקְוָתָם בְּכׇל־דּוֹר וָדוֹר׃ \hfill \break
לְהוֹדִֽיעַ שֶׁכׇּל־קֹוֶֽיךָ לֹא יֵבֹֽשׁוּ וְלֹא יִכָּלְמוּ לָנֶֽצַח כׇּל־הַחוֹסִים בָּךְ׃ \hfill \break
אָרוּר הָמָן אֲשֶׁר בִּקֵּשׁ לְאַבְּדִי בָּרוּךְ מׇרְדְּכַי הַיְּהוּדִי׃ \hfill \break
אֲרוּרָה זֶֽרֶשׁ אֵֽשֶׁת מַפְחִידִי בְּרוּכָה אֶסְתֵּר מְגִינָּה בַּעֲדִי וְגַם חַרְבוֹנָה זָכוּר לְטּוֹב׃
\vfill
\chapter[ברכת המזון]{\adforn{47} ברכת המזון \adforn{19}}

%\source{תהלים קלז}
%
%\columnratio{0.63}
%\begin{paracol}{2}
%\instruction{בימים שיש בהם תחנון:}\\
%\firstword{עַ֥ל נַהֲר֨וֹת}
% בָּבֶ֗ל שָׁ֣ם יָ֭שַׁבְנוּ גַּם־בָּכִ֑ינוּ בְּ֝זׇכְרֵ֗נוּ אֶת־צִיּֽוֹן׃ עַֽל־עֲרָבִ֥ים בְּתוֹכָ֑הּ תָּ֝לִ֗ינוּ כִּנֹּרוֹתֵֽינוּ׃ כִּ֤י שָׁ֨ם שְֽׁאֵל֪וּנוּ שׁוֹבֵ֡ינוּ דִּבְרֵי־שִׁ֭יר וְתוֹלָלֵ֣ינוּ שִׂמְחָ֑ה שִׁ֥ירוּ לָ֝֗נוּ מִשִּׁ֥יר צִיּֽוֹן׃ אֵ֗יךְ נָשִׁ֥יר אֶת־שִׁיר־יְיָ֑ עַ֝֗ל אַדְמַ֥ת נֵכָֽר׃ אִֽם־אֶשְׁכָּחֵ֥ךְ יְֽרוּשָׁלִָ֗ם תִּשְׁכַּ֥ח יְמִינִֽי׃ תִּדְבַּ֥ק־לְשׁוֹנִ֨י לְחִכִּי֮ אִם־לֹ֪א אֶ֫זְכְּרֵ֥כִי אִם־לֹ֣א אַ֭עֲלֶה אֶת־יְרוּשָׁלִַ֑ם עַ֝֗ל רֹ֣אשׁ שִׂמְחָתִֽי׃ זְכֹ֤ר יְיָ֨ לִבְנֵ֬י אֱד֗וֹם אֵת֮ י֤וֹם יְֽרוּשָׁ֫לִָ֥ם הָ֭אֹ֣מְרִים עָ֤רוּ | עָ֑רוּ עַ֝֗ד הַיְס֥וֹד בָּֽהּ׃ בַּת־בָּבֶ֗ל הַשְּׁד֫וּדָ֥ה אַשְׁרֵ֥י שֶׁיְשַׁלֶּם־לָ֑ךְ אֶת־גְּ֝מוּלֵ֗ךְ שֶׁגָּמַ֥לְתְּ לָֽנוּ׃ אַשְׁרֵ֤י שֶׁיֹּאחֵ֓ז וְנִפֵּ֬ץ אֶֽת־עֹ֝לָלַ֗יִךְ אֶל־הַסָּֽלַע׃
%
%\switchcolumn

\ifboolexpr{togl {includeweekday}}{\englishinst{The following is added on festive occasions:}}{}
\firstword{שִׁ֗יר הַֽמַּֽ֫עֲל֥וֹת}\source{תהלים קכו}
בְּשׁ֣וּב יְ֖יָ אֶת־שִׁיבַ֣ת צִיּ֑וֹן הָ֝יִ֗ינוּ כְּחֹלְמִֽים׃ אָ֤ז יִמָּלֵ֢א שְׂחֹ֡ק פִּינוּ֘ וּלְשׁוֹנֵ֢נוּ רִ֫נָּ֥ה אָ֭ז יֹֽאמְר֣וּ בַגּוֹיִ֑ם הִגְדִּ֥יל יְ֜יָ֗ לַֽעֲשׂ֥וֹת עִם־אֵֽלֶּה׃ הִגְדִּ֥יל יְ֖יָ לַֽעֲשׂ֣וֹת עִמָּ֑נוּ הָ֜יִ֗ינוּ שְׂמֵחִֽים׃ שׁוּבָ֣ה יְ֖יָ אֶת־שְׁבִיתֵ֑נוּ כַּֽאֲפִיקִ֥ים בַּנֶּֽגֶב׃ הַזֹּֽרְעִ֥ים בְּדִמְעָ֗ה בְּרִנָּ֥ה יִקְצֹֽרוּ׃ הָ֘ל֤וֹךְ יֵלֵ֨ךְ וּבָכֹה֘ נֹשֵׂ֢א מֶֽשֶׁךְ־הַ֫זָּ֥רַע בֹּֽא־יָבֹ֥א בְרִנָּ֗ה נֹשֵׂ֥א אֲלֻמֹּתָֽיו׃
%\end{paracol}

\englishinst{Three or more who ate together have one person invite the others to bless with a Zimmun.}
\begin{small}
\begin{tabular}{l p{.8\textwidth}}

\instruction{המזמן:} &
רַבּוֹתַי נְבָרֵךְ! \instruction{או} רַבּוֹתַי מיר וועלן בענטשן! \instruction{או} הַב לָן וְנִבְרִךְ!\\
\instruction{כולם:} &
יְהִ֤י שֵׁ֣ם יְיָ֣ מְבֹרָ֑ךְ מֵֽ֝עַתָּ֗ה וְעַד־עוֹלָֽם׃\\
\instruction{המזמן:} &
בִּרְשׁוּת ... נְבָרֵךְ (\instruction{בעשרה} אֱלֹהֵֽינוּ) שֶׁאָכַלְנוּ מִשֶּׁלּוֹ:\\
\instruction{כולם:} &
בָּרוּךְ (\instruction{בעשרה:} אֱלֹהֵֽינוּ) שֶׁאָכַֽלְנוּ מִשֶּׁלּוֹ וּבְטוּבוֹ חָיִֽינוּ:\\
(\instruction{מי שלא אכל:} &
בָּרוּךְ וּמְבֹרָךְ שְׁמוֹ תָּמִיד לְעוֹלָם וָעֶד׃)\\
\instruction{המזמן:} &
בָּרוּךְ (\instruction{בעשרה:} אֱלֹהֵֽינוּ) שֶׁאָכַֽלְנוּ מִשֶּׁלּוֹ וּבְטוּבוֹ חָיִֽינוּ:
\end{tabular}

בָּרוּךְ הוּא וּבָרוּךְ שְׁמוֹ׃\\
\end{small}

\firstword{בָּרוּךְ}
אַתָּה יְיָ אֱלֹהֵֽינוּ מֶֽלֶךְ הָעוֹלָם הַזָּן אֶת־הָעוֹלָם כֻּלּוֹ בְּטוּבוֹ בְּחֵן בְּחֶֽסֶד וּבְרַחֲמִים הוּא נֹתֵ֣ן \source{תהלים קלו}לֶ֭חֶם לְכׇל־בָּשָׂ֑ר כִּ֖י לְעוֹלָ֣ם חַסְדּֽוֹ׃ וּבְטוּבוֹ הַגָּדוֹל תָּמִיד לֹא חָסַר לָֽנוּ וְאַל יֶחְסַר לָֽנוּ מָזוֹן לְעוֹלָם וָעֶד׃ בַּעֲבוּר שְׁמוֹ הַגָּדוֹל כִּי הוּא זָן וּמְפַרְנֵס לַכֹּל וּמֵטִיב לַכֹּל וּמֵכִין מָזוֹן לְכׇל־בְּרִיּוֹתָיו אֲשֶׁר בָּרָא׃ בָּרוּךְ אַתָּה יְיָ הַזָּן אֶת־הַכֹּל׃



\firstword{נוֹדֶה}
לְךָ יְיָ אֱלֹהֵֽינוּ עַל שֶׁהִנְחַֽלְתָּ לַאֲבוֹתֵֽינוּ אֶֽרֶץ חֶמְדָה טוֹבָה וּרְחָבָה׃ וְעַל שֶׁהוֹצֵאתָֽנוּ יְיָ אֱלֹהֵֽינוּ מֵאֶֽרֶץ מִצְרַֽיִם וּפְדִיתָֽנוּ מִבֵּית עֲבָדִים וְעַל בְּרִיתְךָ שֶׁחָתַֽמְתָּ בִּבְשָׂרֵֽנוּ וְעַל תּוֹרָתְךָ שֶׁלִּמַּדְתָּֽנוּ וְעַל חֻקֶּֽיךָ שֶׁהוֹדַעְתָּֽנוּ וְעַל חַיִּים חֵן וָחֶֽסֶד שֶׁחוֹנַנְתָּֽנוּ וְעַל אֲכִילַת מָזוֹן שָׁאַתָּה זָן וּמְפַרְנֵס אוֹתָֽנוּ תָּמִיד בְּכׇל־יוֹם וּבְכׇל־עֵת וּבְכׇל־שָׁעָה׃


\alhanisim

\firstword{וְעַל הַכֹּל}
יְיָ אֱלֹהֵֽינוּ אֲנַֽחְנוּ מוֹדִים לָךְ וּמְבָרְכִים אוֹתָךְ יִתְבָּרַךְ שִׁמְךָ בְּפִי כׇל־חַי תָּמִיד לְעוֹלָם וָעֶד׃ כַּכָּתוּב׃ \source{דברים ח}%
וְאָכַלְתָּ֖ וְשָׂבָ֑עְתָּ וּבֵֽרַכְתָּ֙ אֶת־יְיָ֣ אֱלֹהֶ֔יךָ עַל־הָאָ֥רֶץ הַטֹּבָ֖ה אֲשֶׁ֥ר נָֽתַן־לָֽךְ׃
בָּרוּךְ אַתָּה יְיָ עַל הָאָֽרֶץ וְעַל הַמָּזוֹן׃



\firstword{רַחֵם}
יְיָ אֱלֹהֵֽינוּ עָלֵֽינוּ וְעַל יִשְׂרָאֵל עַמֶּךָ וְעַל יְרוּשָׁלַ‍ִם עִירֶֽךָ וְעַל צִיּוֹן מִשְׁכַּן כְּבוֹדֶֽךָ וְעַל מַלְכוּת בֵּית דָּוִד מְשִׁיחֶֽךָ וְעַל הַבַּֽיִת הַגָּדוֹל וְהַקָּדוֹשׁ שֶׁנִּקְרָא שִׁמְךָ עָלָיו׃ אֱלֹהֵֽינוּ אָבִֽינוּ רְעֵֽנוּ זוּנֵֽנוּ פַרְנְסֵֽנוּ וְכַלְכְּלֵֽנוּ וְהַרְוִיחֵֽנוּ וְהַרְוַח לָֽנוּ יְיָ אֱלֹהֵֽינוּ מְהֵרָה מִכׇּל־צָרוֹתֵֽינוּ׃ וְנָא אַל תַּצְרִיכֵֽנוּ יְיָ אֱלֹהֵֽינוּ לֹא לִידֵי מַתְּנַת בָּשָׂר וָדָם וְלֹא לִידֵי הַלְוָאָתָם. כִּי אִם לְיָדְךָ הַמְּלֵאָה הַפְּתוּחָה הַקְּדוֹשָׁה וְהָרְחָבָה שֶׁלֹּא נֵבוֹשׁ וְלֹא נִכָּלֵם לְעוֹלָם וָעֶד׃

%\enlargethispage{\baselineskip}
%
%\vspace{-.25\baselineskip}
\ifboolexpr{togl {includefestival} or (togl {includeshabbat} and togl {includeweekday})}{
\begin{sometimes}

\shabbos
רְצֵה וְהַחֲלִיצֵֽנוּ יְיָ אֱלֹהֵֽינוּ בְּמִצְוֹתֶֽיךָ וּבְמִצְוַת יוֹם הַשְּׁבִיעִי הַשַּׁבָּת הַגָּדוֹל וְהַקָּדוֹשׁ הַזֶּה כִּי יוֹם זֶה גָּדוֹל וְקָדוֹשׁ הוּא לְפָנֶֽיךָ לִשְׁבָּת בּוֹ וְלָנֽוּחַ בּוֹ בְּאַהֲבָה כְּמִצְוַת רְצוֹנֶךָ׃ בִּרְצוֹנְךָ הָנִֽיחַ לָֽנוּ יְיָ אֱלֹהֵֽינוּ שֶׁלֹא תְהֵי צָרָה וְיָגוֹן וַאֲנָחָה בְּיוֹם מְנוּחָתֵֽנוּ וְהַרְאֵֽנוּ יְיָ אֱלֹהֵֽינוּ בְּנֶחָמוֹת צִיּוֹן עִירֶֽךָ וּבְבִנְיַן יְרוּשָׁלַ‍ִם עִיר קׇדְשֶֽׁךָ כִּי אַתָּה הוּא בַּֽעַל הַיְשׁוּעוֹת וּבַֽעַל הַנֶּחָמוֹת׃


\sepline %These are really two "sometimes's". Sepline to separate them

\vspace{-.25\baselineskip}
\instruction{בראש חודש ומועדים:}\\
אֱלֹהֵֽינוּ וֵאלֹהֵי אֲבוֹתֵֽינוּ יַעֲלֶה וְיָבֹא וְיַגִּיעַ וְיֵרָאֶה וְיֵרָצֶה וְיִשָּׁמַע וְיִפָּקֵד וְיִזָּכֵר זִכְרוֹנֵֽנוּ וּפִקְדּוֹנֵֽנוּ וְזִכְרוֹן אֲבוֹתֵֽינוּ וְזִכְרוֹן מָשִׁיחַ בֶּן דָּוִד עַבְדֶּֽךָ וְזִכְרוֹן יְרוּשָׁלַ‍ִם עִיר קׇדְשֶֽׁךָ וְזִכְרוֹן כׇּל־עַמְּךָ בֵּית יִשְׂרָאֵל לְפָנֶיךָ לִפְלֵיטָה וּלְטוֹבָה וּלְחֵן וּלְחֶֽסֶד וּלְרַחֲמִים וּלְחַיִּים וּלְשָׁלוֹם בְּיוֹם\\
\begin{tabular}{c|c|c}
רֹאשׁ הַחֹֽדֶשׁ & חַג הַמַּצוֹת & חַג הַשָּׁבֻעוֹת\\ \hline
\end{tabular}\\
\begin{tabular}{c|c|c}
הַזִּכָּרוֹן & חַג הַסֻּכּוֹת & שְׁמִינִי חַג הָעֲצֶֽרֶת
\end{tabular}\\
הַזֶּה זׇכְרֵֽנּוּ יְיָ אֱלֹהֵֽינוּ בּוֹ לְטוֹבָה וּפׇקְדֵֽנוּ בוֹ לִבְרָכָה וְהוֹשִׁיעֵֽנוּ בוֹ לְחַיִּים וּבִדְבַר יְשׁוּעָה וְרַחֲמִים חוּס וְחׇׇׇׇנֵּנוּ וְרַחֵם עָלֵֽינוּ וְהוֹשִׁיעֵֽנוּ כִּי אֵלֶֽיךָ עֵינֵֽינוּ כִּי אֵל מֶֽלֶךְ חַנּוּן וְרַחוּם אַֽתָּה׃

\end{sometimes}}{
\ifboolexpr{togl {includeshabbat}}{
	רְצֵה וְהַחֲלִיצֵֽנוּ יְיָ אֱלֹהֵֽינוּ בְּמִצְוֹתֶֽיךָ וּבְמִצְוַת יוֹם הַשְּׁבִיעִי הַשַּׁבָּת הַגָּדוֹל וְהַקָּדוֹשׁ הַזֶּה כִּי יוֹם זֶה גָּדוֹל וְקָדוֹשׁ הוּא לְפָנֶֽיךָ לִשְׁבָּת בּוֹ וְלָנֽוּחַ בּוֹ בְּאַהֲבָה כְּמִצְוַת רְצוֹנֶךָ׃ בִּרְצוֹנְךָ הָנִֽיחַ לָֽנוּ יְיָ אֱלֹהֵֽינוּ שֶׁלֹא תְהֵי צָרָה וְיָגוֹן וַאֲנָחָה בְּיוֹם מְנוּחָתֵֽנוּ וְהַרְאֵֽנוּ יְיָ אֱלֹהֵֽינוּ בְּנֶחָמוֹת צִיּוֹן עִירֶֽךָ וּבְבִנְיַן יְרוּשָׁלַ‍ִם עִיר קׇדְשֶֽׁךָ כִּי אַתָּה הוּא בַּֽעַל הַיְשׁוּעוֹת וּבַֽעַל הַנֶּחָמוֹת׃
	
	\instruction{בר״ח׃}
	\yaalehveyavotemplate{רֹאשׁ הַחֹֽדֶשׁ}
}{}
}

\firstword{וּבְנֵה}
יְרוּשָׁלַ‍ִם עִיר הַקֹּֽדֶשׁ בִּמְהֵרָה בְּיָמֵֽינוּ׃ בָּרוּךְ אַתָּה יְיָ בֹּֽנֶה בְרַחֲמָיו יְרוּשָׁלַ‍ִם אָמֵן׃

%\begin{sometimes}
%
%\instruction{אם שכח רצה או יעלה ויבא:}\\
%בָּרוּךְ אַתָּה יְיָ אֱלֹהֵֽינוּ מֶֽלֶךְ הָעוֹלָם אֲשֶׁר נָתַן (שַׁבָּתוֹת לִמְנוּחָה לְעַמּוֹ יִשְׂרָאֵל בְּאַהֲבָה לְאוֹת וְלִבְרִית)
%(וְיָמִים טוֹבִים לְשָׂשׂוֹן וּלְשִׂמְחָה אֶת־יוֹם חַג ... הַזֶּה)(וְרָאשֵׁי חֳדָשִׁים לְזִכָּרוֹן \instruction{מסיים כאן בחול}):
%בָּרוּךְ אַתָּה יְיָ מְקַדֵּשׁ (הַשַּׁבָּת) ([וְ]יִשְׂרָאֵל וְהַזְּמַנִּים)(וְיִשְׂרָאֵל וְרָאשֵׁי חֳדָשִׁים׃)׃
%
%\end{sometimes}


\firstword{בָּרוּךְ}
אַתָּה יְיָ אֱלֹהֵֽינוּ מֶֽלֶךְ הָעוֹלָם הָאֵל אָבִֽינוּ מַלְכֵּֽנוּ אַדִּירֵֽנוּ בּוֹרְאֵֽנוּ גֹאֲלֵֽנוּ יוֹצְרֵֽנוּ קְדוֹשֵֽׁנוּ קְדוֹשׁ יַעֲקֹב רוֹעֵֽנוּ רוֹעֵה יִשְׂרָאֵל הַמֶּֽלֶךְ הַטּוֹב וְהַמֵּטִיב לַכֹּל שֶׁבְּכׇל־יוֹם וָיוֹם הוּא הֵטִיב הוּא מֵטִיב הוּא יֵיטִיב לָֽנוּ׃ הוּא גְמָלָֽנוּ הוּא גוֹמְלֵנוּ הוּא יִגְמְלֵנוּ לָעַד לְחֵן לְחֶֽסֶד וּלְרַחֲמִים וּלְרֶֽוַח הַצָּלָה וְהַצְלָחָה בְּרָכָה וִישׁוּעָה נֶחָמָה פַּרְנָסָה וְכַלְכָּלָה וְרַחֲמִים וְחַיִּים וְשָׁלוֹם וְכׇל־טוֹב וּמִכׇּל־טוֹב אַל יְחַסְּרֵֽנוּ׃

\firstword{הָרַחֲמָן}
הוּא יִמְלֹךְ עָלֵֽינוּ לְעוֹלָם וָעֶד׃
\firstword{הָרַחֲמָן}
הוּא יִתְבָּרַךְ בַּשָּׁמַֽיִם וּבָאָֽרֶץ׃
\firstword{הָרַחֲמָן}
הוּא יִשְׁתַּבַּח לְדוֹר דּוֹרִים וְיִתְפָּֽאַר בָּֽנוּ לָנֵֽצַח נְצָחִים
וְיִתְהַדַּר בָּֽנוּ לָעַד וּלְעוֹלְמֵי עוֹלָמִים׃
\firstword{הָרַחֲמָן}
הוּא יְפַרְנְסֵֽנוּ בְּכָבוֹד׃
\firstword{הָרַחֲמָן}
הוּא יִשְׁבּוֹר עֻלֵּֽנוּ מֵעַל צַוָּארֵֽנוּ וְהוּא יוֹלִיכֵֽנוּ קוֹמְמִיּוּת לְאַרְצֵֽנוּ׃
\firstword{הָרַחֲמָן}
הוּא יִשְׁלַח בְּרָכָה מְרֻבָּה בְּבַֽיִת זֶה וְעַל שֻׁלְחָן זֶה שֶׁאָכַֽלְנוּ עָלָיו׃
\firstword{הָרַחֲמָן}
הוּא יִשְׁלַח לָֽנוּ אֶת־אֵלִיָּֽהוּ הַנָּבִיא זָכוּר לַטּוֹב וִיבַשֵּׂר לָנוּ בְּשׂוֹרוֹת טוֹבוֹת יְשׁוּעוֹת וְנֶחָמוֹת׃


\begin{footnotesize}
\instruction{אורחים אומרים:}\\
יְהִי רָצוֹן שֶׁלֹא יֵבוֹשׁ בַּעַל הַבַּיִת בָּעוֹלָם הַזֶּה וְלֹא יִכָּלֵם לָעוֹלָם הַבָּא וְיִצְלַח מְאֹד בְּכׇל־נְכָסָיו וְיִהְיוּ נְכָסָיו מֻצְלָחִים וּקְרוֹבִים לָעִיר וְאַל יִשְׁלוֹט שָׂטָן לֹא בְּמַעֲשֵׂי יָדָיו וְלֹא בְּמַעֲשֵׂי יָדֵינוּ וְאַל יִזְדַקֵּר לֹא לְפָנָיו וְלֹא לְפָנֵינוּ שׁוּם דְבַר הִרְהוּר חֵטְא וַעֲבֵרָה וְעָוֹן מֵעַתָּה וְעַד עוֹלָם׃

\end{footnotesize}

\firstword{הָרַחֲמָן}
הוּא יְבָרֵךְ אֶת־[אָבִי מוֹרִי] בַּעַל הַבַּֽיִת הַזֶּה וְאֶת־[אִמִּי מוֹרָתִי] בַּעֲלַת הַבַּֽיִת הַזֶּה׃ אוֹתָם וְאֶת־בֵּיתָם וְאֶת־זַרְעָם וְאֶת־כׇּל־אַשֶׁר לָהֶם, אוֹתָנוּ וְאֶת־כׇּל־אַשֶׁר לָֽנוּ כְּמוֹ שֶׁנִּתְבָּרְכוּ אֲבוֹתֵֽינוּ אַבְרָהָם יִצְחָק וְיַעֲקֹב בַּכֹּל מִכֹּל כֹּל כֵּן יְבָרֵךְ אוֹתָֽנוּ כֻּלָּנוּ יַֽחַד בִּבְרָכָה שְׁלֵמָה וְנֹאמַר אָמֵן׃

\begin{sometimes}

\englishinst{At the meal following a Berit Mila:}\nopagebreak
\begin{center}
\textbf{הָרַחֲמָן}
הוּא אֲשֶׁר חָנַן אֶת־הַיֶּלֶד הַזֶּה לְאָבִיו וּלְאִמּוֹ הוּא יָגֵן עָלָיו מִמְּרוֹמוֹ וּבְשָׁלוֹם יָבֹא עַל־מְקוֹמוֹ וִיהִי אֱלֹהָיו עִמּוֹ וּכְאֶפְרַיִם וְכִמְנַשֶּׁה לְשׂוּמוֹ ְויִקָּרֵא בְיִשְׂרָאֵל שְׁמוֹ׃

\textbf{הָרַחֲמָן}
הוּא פָּקוֹד יִפְקְדֵהוּ בְּרַחֲמִים לַהֲבִינוֹ בְּדָת חִכּוּמִים וִיבַלֶּה בַטּוֹב יָמִים וּשְׁנוֹתָיו בַּנְּעִימִים יַעַבְדוּהוּ עַמִּים וְיִשְׁתַּחֲווּ לוֹ לְאֻמִּים׃

\textbf{הָרַחֲמָן}
הוּא רַבּוֹת שָׁנִים יְחַיֵּהוּ צֶדֶק לְרַגְלָיו יִקְרָאֵהוּ וְנֶחָמַת צִיּוֹן יַרְאֵהוּ וְאֶת־עַמּוֹ לְשָׁלוֹם יְבָרְכֵהוּ וִיעוֹרֵר חֲסָדָיו לְרַחֲמֵהוּ כִּי חָפֵץ חֶסֶד הוּא׃

\textbf{הָרַחֲמָן}
הוּא יְבָרֵךְ אֶת־הֶחָתָן הַזֶּה וּבַעַל בְּרִיתוֹ יִשְׂמַח אָבִיו וְתָגֵל יוֹלַדְתּוֹ וְיִתְבָּרַכוּ הַמְסֻבִּים בִּסְעוּדָתוֹ וְכֵן יִזְכּוּ שֶׁיִּשְׂמְחוּ בַּחֲתֻנָּתוֹ בְּנֵי בָנִים לְהַרְאוֹתוֹ וְיֵשׁ תִּקְוָה לְאַחֲרִיתוֹ׃

\textbf{הָרַחֲמָן}
הוּא מְהֵרָה יִזְכֹּר זֹאת מִצְוָתוֹ בָּהּ יִפְדֶה אֲיֻמָּתוֹ רַחֲמִים יְעוֹרֵר לַעֲדָתוֹ קְהָלָיו יְקַבֵּץ בְּחֶמְלָתוֹ בְּהַרְאֹתוֹ אֶת־עֹשֶׁר כְּבוֹד מַלְכוּתוֹ וְאֶת־יְקָר תִּפְאֶרֶת גְּדֻלָּתוֹ׃

\textbf{הָרַחֲמָן}
הוּא יְבָרֵךְ אֶת־הֶחָתָן הַזֶּה וּבַּעַל בְּרִיתוֹ וְאֶת אָבִיו וְאֶת אִמּוֹ וְאֶת רַבּוֹתֵינוּ וְאֶת־אַחֵינוּ הַיּוֹשְׁבִים פֹּה כְּמוֹ שֶׁנִתְבָּרְכוּ אֲבוֹתֵֽינוּ אַבְרָהָם יִצְחָק וְיַעֲקֹב בַּכֹּל מִכֹּל כֹּל כֵּן יְבָרֵךְ אוֹתָֽנוּ כֻּלָּנוּ יַֽחַד בִּבְרָכָה שְׁלֵמָה וְנֹאמַר אָמֵן׃
\end{center}
\end{sometimes}

\begin{center}
\firstword{בַּמָּרוֹם}
יְלַמְּדוּ עֲלֵיהֶם וְעָלֵֽינוּ זְכוּת שֶׁתְּהֵא לְמִשְׁמֶֽרֶת שָׁלוֹם׃ וְנִשָּׂא בְרָכָה מֵאֵת יְיָ וּצְדָקָה מֵאֱלֹהֵי יִשְׁעֵנוּ וְנִמְצָא חֵן וְשֵֽׂכֶל טוֹב בְּעֵינֵי אֱלֹהִים וְאָדָם׃
\end{center}

\begin{longtable}{l p{.8\textwidth}}

\shabbos &
הָרַחֲמָן הוּא יַנְחִילֵֽנוּ לְיּוֹם שֶׁכֻּלּוֹ שַׁבָּת וּמְנוּחָה לְחַיֵּי הָעוֹלָמִים׃ \\

\instruction{בראש חודש:} &
הָרַחֲמָן הוּא יְחַדֵּשׁ עָלֵֽינוּ אֶת־הַחֹֽדֶשׁ הַזֶּה לְטוֹבָה וְלִבְרָכָה׃ \\

\instruction{בשלש רגלים:} &
הָרַחֲמָן הוּא יַנְחִילֵֽנוּ לְיּוֹם שֶׁכֻּלּוֹ טוֹב׃ \\

\instruction{בראש השנה:} &
הָרַחֲמָן הוּא יְחַדֵּשׁ עָלֵֽינוּ אֶת־הַשָּׁנָה הַזֹּאת לְטוֹבָה וְלִבְרָכָה׃ \\

\instruction{בסכות:} &
הָרַחֲמָן הוּא יָקִים לָֽנוּ אֶת־סֻכַּ֥ת דָּוִ֖יד הַנֹּפֶ֑לֶת׃ \mdsource{עמוס ט}

\end{longtable}

\firstword{הָרַחֲמָן}
הוּא יְזַכֵּֽנוּ לִימוֹת הַמָּשִֽׁיחַ וּלְחַיֵּי עוֹלָם הַבָּא׃

\firstword{מַגְדִּיל֘}\source{תהלים יח}
(\instruction{בשבת, יו״ט, ור״ח׃ }
מִגְדּ֖וֹל)
יְשׁוּע֢וֹת מַ֫לְכּ֥וֹ וְעֹ֤שֶׂה חֶ֨סֶד ׀ לִמְשִׁיח֗וֹ לְדָוִ֥ד וּלְזַרְע֗וֹ עַד־עוֹלָֽם׃
עֹשֶׂה שָׁלוֹם בִּמְרוֹמָיו הוּא יַעֲשֶׂה שָׁלוֹם עָלֵֽינוּ וְעַל כׇּל־יִשְׂרָאֵל וְאִמְרוּ אָמֵן׃\\
יְר֣אוּ \source{תהלים לד}אֶת־יְיָ֣ קְדֹשָׁ֑יו כִּי־אֵ֥ין מַ֝חְס֗וֹר לִירֵאָֽיו׃ כְּ֭פִירִים רָשׁ֣וּ וְרָעֵ֑בוּ
וְדֹרְשֵׁ֥י יְ֝יָ֗ לֹא־יַחְסְר֥וּ כׇל־טֽוֹב׃
הוֹד֣וּ לַֽיְיָ֑ \source{תהלים קיח}כִּי־ט֑וֹב כִּ֖י לְעוֹלָ֣ם חַסְדּֽוֹ׃
פּוֹתֵ֥חַ \source{תהלים קמה}אֶת־יָדֶ֑ךָ וּמַשְׂבִּ֖יעַ לְכׇל־חַ֣י רָצֽוֹן׃
בָּר֣וּךְ \source{ירמיהו יז}הַגֶּ֔בֶר אֲשֶׁ֥ר יִבְטַ֖ח בַּייָ֑ וְהָיָ֥ה יְיָ֖ מִבְטַחֽוֹ׃
נַ֤עַר \source{תהלים לז}׀ הָיִ֗יתִי גַּם־זָ֫קַ֥נְתִּי וְֽלֹא־רָ֭אִיתִי צַדִּ֣יק נֶעֱזָ֑ב וְ֝זַרְע֗וֹ מְבַקֶּשׁ־לָֽחֶם׃
יְיָ֗ \source{תהלים כט}עֹ֭ז לְעַמּ֣וֹ יִתֵּ֑ן יְיָ֓ ׀ יְבָרֵ֖ךְ אֶת־עַמּ֣וֹ בַשָּׁלֽוֹם׃

\bigskip

\sepline

\bigskip

\instruction{המזמן:}
בָּרוּךְ אַתָּה יְיָ אֱלֹהֵֽינוּ מֶֽלֶךְ הָעוֹלָם בּוֹרֵא פְּרִי הַגָּֽפֶן׃

\vfill
\sepline

\ifboolexpr{togl {includeweekday}}{
\section[ברכת המזון בבית אבל]{\adforn{53} ברכת המזון בבית אבל \adforn{25}}

\instruction{זימון בבית־אבל}

\begin{small}
	\begin{tabular}{l p{.8\textwidth}}
		
		\instruction{המזמן:} &
		רַבּוֹתַי נְבָרֵךְ! \instruction{או} רַבּוֹתַי מיר וועלן בענטשן! \instruction{או} הַב לָן וְנִבְרִךְ!\\
		\instruction{כולם:} &
		יְהִ֤י שֵׁ֣ם יְיָ֣ מְבֹרָ֑ךְ מֵֽ֝עַתָּ֗ה וְעַד־עוֹלָֽם׃\\
		\instruction{המזמן:} &
		בִּרְשׁוּת ... נְבָרֵךְ (\instruction{בעשרה} אֱלֹהֵֽינוּ) מְנַחֵם אֲבֵלִים שֶׁאָכַלְנוּ מִשֶּׁלּוֹ:\\
		\instruction{כולם:} &
		בָּרוּךְ (\instruction{בעשרה:} אֱלֹהֵֽינוּ) מְנַחֵם אֲבֵלִים שֶׁאָכַֽלְנוּ מִשֶּׁלּוֹ וּבְטוּבוֹ חָיִֽינוּ:\\
		(\instruction{מי שלא אכל:} &
		בָּרוּךְ מְנַחֵם אֲבֵלִים וּמְבֹרָךְ שְׁמוֹ תָּמִיד לְעוֹלָם וָעֶד׃)\\
		\instruction{המזמן:} &
		בָּרוּךְ (\instruction{בעשרה:} אֱלֹהֵֽינוּ) מְנַחֵם אֲבֵלִים שֶׁאָכַֽלְנוּ מִשֶּׁלּוֹ וּבְטוּבוֹ חָיִֽינוּ:
	\end{tabular}

בָּרוּךְ הוּא וּבָרוּךְ שְׁמוֹ׃

\end{small}

נַחֵם יְיָ אֱלֹהֵינוּ אֶת אֲבֵלֵי יְרוּשָׁלַיִם. וְאֶת הָאֲבֵלִים הַמִּתְאַבְּלִים בָּאֵֽבֶל הַזֶּה. נְחַמֵּם מֵאֶבְלָם וְשֶׁמֵּחָם מִיגוֹנָם כָּאָמוּר׃\source{ישעיה סו} כְּאִ֕ישׁ אֲשֶׁ֥ר אִמּ֖וֹ תְּנַחֲמֶ֑נּוּ כֵּ֤ן אָֽנֹכִי֙ אֲנַ֣חֶמְכֶ֔ם וּבִירֽוּשָׁלַ֖͏ִם תְּנֻחָֽמוּ׃ בָּרוּךְ אַתָּה יְיָ מְנַחֵם צִיּוֹן בְּבִנְיַן יְרוּשָׁלַיִם׃

\firstword{בָּרוּךְ}
אַתָּה יְיָ אֱלֹהֵֽינוּ מֶֽלֶךְ הָעוֹלָם הָאֵל אָבִֽינוּ מַלְכֵּֽנוּ אַדִּירֵֽנוּ בּוֹרְאֵֽנוּ גֹאֲלֵֽנוּ יוֹצְרֵֽנוּ קְדוֹשֵֽׁנוּ קְדוֹשׁ יַעֲקֹב. הַמֶּלֶךְ הַחַי הַטּוֹב וְהַמֵּטִיב. אֵל אֱמֶת, דַּיַּן אֱמֶת, שׁוֹפֵט בְּצֶדֶק, לוֹקֵחַ נְפָשׁוֹת בַּמִּשְׁפָּט. שַׁלִּיט בְּעוֹלָמוֹ לַעֲשׂוֹת בּוֹ כִּרְצוֹנוֹ כִּי כׇל־דְּרָכָיו בַּמִּשְׁפָּט, וַאֲנַחְנוּ עַמּוֹ וַעֲבָדָיו, וּבַכֹּל אֲנַחְנוּ חַיָּבִים לְהוֹדוֹת לוֹ וּלְבָרְכוֹ. גּוֹדֵר פְּרָצוֹת יִשְׂרָאֵל הוּא יִגְדֹּר הַפִּרְצָה הַזֹּאת מֵעָלֵינוּ וּמֵעַל אֲבָל זֶה לְחַיִּים וּלְשָׁלוֹם. הוּא יִגְמְלֵנוּ לָעַד לְחֵן לְחֶֽסֶד וּלְרַחֲמִים וּלְרֶֽוַח הַצָּלָה וְהַצְלָחָה בְּרָכָה וִישׁוּעָה נֶחָמָה פַּרְנָסָה וְכַלְכָּלָה וְרַחֲמִים וְחַיִּים וְשָׁלוֹם וְכׇל־טוֹב וּמִכׇּל־טוֹב אַל יְחַסְּרֵֽנוּ׃}{}


\section[ברכה מעין שלש]{\adforn{53} ברכה מעין שלש \adforn{25}}

\englishinst{After eating foods made from the five grains (besides bread), grapes, figs, pomegranate, olives, dates, or drinking wine, recite the following blessing.}
\firstword{בָּרוּךְ}
אַתָּה יְיָ אֱלֹהֵֽינוּ מֶֽלֶךְ הָעוֹלָם עַל

\begin{tabular}{>{\centering\arraybackslash}m{.3\textwidth} | >{\centering\arraybackslash}m{.3\textwidth} | >{\centering\arraybackslash}m{.3\textwidth}}

הָעֵץ וְעַל פְּרִי הָעֵץ
&
הַמִּחְיָה וְעַל הַכַּלְכָּלָה
&
הַגֶּֽפֶן וְעַל פְּרִי הַגֶּֽפֶן \\

\end{tabular}

וְעַל תְּנוּבַת הַשָּׂדֶה וְעַל אֶֽרֶץ חֶמְדָּה טוֹבָה וּרְחָבָה
שֶׁרָצִֽיתָ וְהִנְחַֽלְתָּ לַאֲבוֹתֵֽינוּ לֶאֱכוֹל מִפִּרְיָהּ וְלִשְׂבּֽוֹעַ מִטּוּבָהּ׃
רַחֶם יְיָ אֱלֹהֵֽינוּ עַל יִשְׂרָאֵל עַמֶּֽךָ וְעַל יְרוּשָׁלַֽיִם עִירֶֽךָ וְעַל צִיּוֹן מִשְׁכַּן כְּבוֹדֶֽךָ וְעַל מִזְבַּחֲךָ וְעַל הֵיכָלֶֽךָ׃ וּבְנֵה יְרוּשָׁלַֽיִם עִיר הַקֹּדֶשׁ בִּמְהֵרָה בְּיָמֵֽינוּ וְהַעֲלֵֽנוּ לְתוֹכָהּ וְשַׂמְּחֵֽנוּ בְּבִנְיָנָהּ וְנֹאכַל מִפִּרְיָהּ וְנִשְׂבַּע מִטּוּבָהּ וּנְבָרֶכְךָ עָלֶיהָ בִּקְדֻשָּׁה וּבְטׇהֳרָה׃

\begin{small}

\begin{tabular}{l p{.7\textwidth}}
\instruction{שבת:}&
וּרְצֵה וְהַחֲלִיצֵֽנוּ בְּיוֹם הַשַּׁבָּת הַזֶּה׃ \\


\instruction{ראש חודש:}&
וְזׇכְרֵֽנוּ לְטוֹבָה
בְּיוֹם רֹאשׁ הַחֹֽדֶשׁ הַזֶּה׃ \\

\instruction{שלוש רגלים:}&
וְשַׂמְּחֵֽנוּ בְּיוֹם
חַג הַמַּצּוֹת \textbackslash \space הַשָּׁבֻעוֹת \textbackslash \space הַסֻּכּוֹת \textbackslash \space שְׁמִינִי חַג הָעֲצֶֽרֶת הַזֶּה׃\\


\instruction{ראש השנה:}&
וְזׇכְרֵֽנוּ לְטוֹבָה בְּיוֹם חַזִּכָּרוֹן הַזֶּה׃\\

\end{tabular}

\end{small}

כִּי אַתָּה טוֹב וּמֵטִיב לַכֹּל וְנוֹדֶה לְךָ עַל הָאָֽרֶץ,

\begin{tabular}{c|c|c}
וְעַל הַפֵּרוֹת & וְעַל הַמִּחְיָה & וְעַל פְּרִי הַגָּֽפֶן
\end{tabular}

בָּרוּךְ אַתָּה יְיָ עַל הָאָֽרֶץ

\begin{tabular}{c|c|c}
וְעַל הַפֵּרוֹת׃ & וְעַל הַמִּחְיָה׃ & וְעַל פְּרִי הַגָּֽפֶן׃
\end{tabular}

\englishinst{After all other foods:}
\firstword{בָּרוּךְ}
אַתָּה יְיָ אֱלֹהֵֽינוּ מֶֽלֶךְ הָעוֹלָם בּוֹרֵא נְפָשׁוֹת רַבּוֹת וְחֶסְרוֹנָן
עַל כׇּל־מַה שֶּׁבָּרָא לְהַחֲיוֹת בָּהֶם נֶֽפֶשׁ כׇּל־חָי׃ בָּרוּךְ חַי הָעוֹלָמִים׃\\

\chapter[ברכות]{\adforn{47} ברכות \adforn{19}}

\newcommand{\berakha}[2]{\englishinst{#1}
בָּרוּךְ אַתָּה יְיָ אֱלֺהֵֽינוּ מֶֽלֶךְ הָעוֹלָם #2׃}
\newcommand{\berakhamitzva}[2]{\berakha{#1}{אֲשֶׁר קִדְּשָֽׁנוּ בְּמִצְוֺתָיו וְצִוָּֽנוּ #2}}


\ssubsection{\adforn{18} ברכות על אכילה \adforn{17}}

\ifboolexpr{not togl {includeweekday}}{
\berakhamitzva{On washing hands before eating bread:}{עַל נְטִילַת יָדָיִם}
\berakha{Before eating bread:}{הַמּֽוֹצִיא לֶֽחֶם מִן הָאָֽרֶץ}
}{
\berakha{Before eating bread, wash hands and recite its blessing in the following section, then recite the following blessing before eating:}{הַמּֽוֹצִיא לֶֽחֶם מִן הָאָֽרֶץ}
}

\berakha{On food made from grains besides bread (e.g. crackers or cake):}{בּוֹרֵא מִינֵי מְזוֹנוֹת}

\berakha{Before drinking wine:}{בּוֹרֵא פְּרִי הַגָּֽפֶן}

\berakha{Before eating fruit that grows on trees:}{בּוֹרֵא פְּרִי הָעֵץ}

\berakha{Before eating fruits or vegetables:}{בּוֹרֵא פְּרִי הָאֲדָמָה}

\berakha{On all other foods:}{שֶׁהַכֹּל נִהְיֶה בִּדְבָרוֹ}

\ssubsection{\adforn{18} ברכות הריח \adforn{17}}

\berakha{On pleasant-smelling products from trees:}{בּוֹרֵא עֲצֵי בְשָׂמִים}

\berakha{On fragrant shrubs and grasses:}{בּוֹרֵא עִשְׂבֵי בְשָׂמִים}

\berakha{On fragrant fruits:}{הַנּוֹתֵן רֵֽיחַ טוֹב בַּפֵּרוֹת}

\berakha{On balsam oil:}{בּוֹרֵא שֶֽׁמֶן עָרֵב}

\berakha{On other pleasant scents:}{בּוֹרֵא מִינֵי בְשָׂמִים}

\ssubsection{\adforn{18} ברכות הראייה והשמיעה \adforn{17}}

\berakha{On seeing natural phenomena, such as lightning, mountains, or canyons:}{עֹשֶׂה מַעֲשֵׂה בְרֵאשִׁית}

\berakha{On hearing thunder or experiencing very strong winds:}{שֶׁכֹּחוֹ וּגְבוּרָתוֹ מָלֵא עוֹלָם}

\berakha{On seeing a rainbow:}{זוֹכֵר הַבְּרִית וְנֶאֱמָן בִּבְרִיתוֹ וְקַיָּם בְּמַאֲמָרוֹ}

\berakha{On seeing the ocean, for the first time in thirty days:}{שֶׁעָשָׂה אֶת הַיָּם הַגָּדוֹל}

\berakha{On seeing beautiful animals:}{שֶׁכָּֽכָה לּוֹ בְּעוֹלָמוֹ}

\berakha{On seeing trees blooming for the first time in spring:}{שֶׁלֹּא חִסַּר בְּעוֹלָמוֹ דָּבָר, וּבָרָא בוֹ בְּרִיּוֹת טוֹבוֹת וְאִילָנוֹת טוֹבִים לְהַנּוֹת בָּהֶם בְּנֵי אָדָם}

\berakha{On seeing unusual animals:}{מְשַׁנֶּה הַבְּרִיּוֹת}

\berakha{On seeing a monarch:}{שֶׁנָּתַן מִכְּבוֹדוֹ לַבָּשָׂר וָדָם}

\berakha{On seeing a Torah scholar:}{שֶׁחָלַק מֵחׇכְמָתוֹ לִירֵאָיו}

\berakha{On seeing scholars of other subjects:}{שֶׁנָּתַן מֵחׇכְמָתוֹ לְבָשָׂר וָדָם}

\berakha{On news that is good for multiple people:}{הַטוֹב וְהַמֵּטִיב}

\berakha{On good news for an individual:}{שֶׁהֶחֱיָֽנוּ וְקִיְּמָֽנוּ וְהִגִּיעָֽנוּ לַזְּמַן הַזֶּה}

\berakha{On hearing bad news:}{דַּיַּן הָאֱמֶת}

\berakha{On wearing a significant item of new clothing:}{מַלְבִּישׁ עַרֻמִּים}

\ifboolexpr{togl {includeweekday}}{
\ssubsection{\adforn{18} ברכות המצוה \adforn{17}}

\berakhamitzva{On washing hands before eating bread:}{עַל נְטִילַת יָדָיִם}

\berakhamitzva{On ritual immersion:}{עַל הַטְּבִילָה}

\berakhamitzva{On immersion of vessels:}{עַל טְבִילַת כֵּלִים}

\berakhamitzva{On fixing a mezuza to a doorpost:}{לִקְבּֽוֹעַ מְזוּזָה}

\berakhamitzva{On performing she\d{h}ita:}{עַל הַשְּׁחִיטָה}

\berakhamitzva{On covering the blood of she\d{h}ted fowl or deer:}{עַל כִּיסוּי הַדָּם}

\section[תפלת הדרך]{\adforn{47} תפלת הדרך \adforn{19}}

\englishinst{On embarking on a journey:}
יְהִי רָצוֹן מִלְּפָנֶֽיךָ יְיָ אֱלֹהֵֽינוּ וֵאלֹהֵי אֲבוֹתֵֽינוּ שֶׁתּוֹלִיכֵֽנוּ לְשָׁלוֹם וְתַצְעִידֵֽנוּ לְשָׁלוֹם וְתַדְרִיכֵֽנוּ לְשָׁלוֹם׃ וְתַגִּיעֵֽנוּ לִמְחוֹז חֶפְצֵֽנוּ לְחַיִּים וּלְשִׂמְחָה וּלְשָׁלוֹם וְתַצִּילֵֽנוּ מִכַּף כׇּל־אוֹיֵב וְאוֹרֵב בַּדֶּֽרֶךְ וְתִתְּנֵֽנוּ לְחֵן וּלְחֶֽסֶד וּלְרַחֲמִים בְּעֵינֶֽיךָ וּבְעֵינֵי כׇל־רוֹאֵֽנוּ וְתִשְׁמַע קוֹל תַּחֲנוּנֵֽינוּ כִּי אֵל שׁוֹמֵֽעַ תְּפִלָה וְתַחֲנוּן אַֽתָּה׃ בָּרוּךְ אַתָּה יְיָ שׁוֹמֵֽעַ תְּפִלָּה׃}{}

\endgroup

\end{document}